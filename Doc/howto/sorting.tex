\documentclass{howto}

\title{Sorting Mini-HOWTO}

% Increment the release number whenever significant changes are made.
% The author and/or editor can define 'significant' however they like.
\release{0.01}

\author{Andrew Dalke}
\authoraddress{\email{dalke@bioreason.com}}

\begin{document}
\maketitle

\begin{abstract}
\noindent
This document is a little tutorial
showing a half dozen ways to sort a list with the built-in
\method{sort()} method.  

This document is available from the Python HOWTO page at
\url{http://www.python.org/doc/howto}.
\end{abstract}

\tableofcontents

Python lists have a built-in \method{sort()} method.  There are many
ways to use it to sort a list and there doesn't appear to be a single,
central place in the various manuals describing them, so I'll do so
here.

\section{Sorting basic data types}

A simple ascending sort is easy; just call the \method{sort()} method of a list.

\begin{verbatim}
>>> a = [5, 2, 3, 1, 4]
>>> a.sort()
>>> print a
[1, 2, 3, 4, 5]
\end{verbatim}

Sort takes an optional function which can be called for doing the
comparisons.  The default sort routine is equivalent to

\begin{verbatim}
>>> a = [5, 2, 3, 1, 4]
>>> a.sort(cmp)
>>> print a
[1, 2, 3, 4, 5]
\end{verbatim}

where \function{cmp} is the built-in function which compares two objects, \code{x} and
\code{y}, and returns -1, 0 or 1 depending on whether $x<y$, $x==y$, or $x>y$.  During
the course of the sort the relationships must stay the same for the
final list to make sense.

If you want, you can define your own function for the comparison.  For 
integers (and numbers in general) we can do:

\begin{verbatim}
>>> def numeric_compare(x, y):
>>>    return x-y
>>> 
>>> a = [5, 2, 3, 1, 4]
>>> a.sort(numeric_compare)
>>> print a
[1, 2, 3, 4, 5]
\end{verbatim}

By the way, this function won't work if result of the subtraction
is out of range, as in \code{sys.maxint - (-1)}.

Or, if you don't want to define a new named function you can create an
anonymous one using \keyword{lambda}, as in:

\begin{verbatim}
>>> a = [5, 2, 3, 1, 4]
>>> a.sort(lambda x, y: x-y)
>>> print a
[1, 2, 3, 4, 5]
\end{verbatim}

If you want the numbers sorted in reverse you can do

\begin{verbatim}
>>> a = [5, 2, 3, 1, 4]
>>> def reverse_numeric(x, y):
>>>     return y-x
>>> 
>>> a.sort(reverse_numeric)
>>> print a
[5, 4, 3, 2, 1]
\end{verbatim}

(a more general implementation could return \code{cmp(y,x)} or \code{-cmp(x,y)}).

However, it's faster if Python doesn't have to call a function for
every comparison, so if you want a reverse-sorted list of basic data
types, do the forward sort first, then use the \method{reverse()} method.

\begin{verbatim}
>>> a = [5, 2, 3, 1, 4]
>>> a.sort()
>>> a.reverse()
>>> print a
[5, 4, 3, 2, 1]
\end{verbatim}

Here's a case-insensitive string comparison using a \keyword{lambda} function:

\begin{verbatim}
>>> a = "This is a test string from Andrew".split()
>>> a.sort(lambda x, y: cmp(x.lower(), y.lower()))
>>> print a
['a', 'Andrew', 'from', 'is', 'string', 'test', 'This']
\end{verbatim}

This goes through the overhead of converting a word to lower case
every time it must be compared.  At times it may be faster to compute
these once and use those values, and the following example shows how.

\begin{verbatim}
>>> words = string.split("This is a test string from Andrew.")
>>> offsets = []
>>> for i in range(len(words)):
>>>     offsets.append( (string.lower(words[i]), i) )
>>> 
>>> offsets.sort()
>>> new_words = []
>>> for dontcare, i in offsets:
>>>      new_words.append(words[i])
>>> 
>>> print new_words
\end{verbatim}

The \code{offsets} list is initialized to a tuple of the lower-case string
and its position in the \code{words} list.  It is then sorted.  Python's
sort method sorts tuples by comparing terms; given \code{x} and \code{y}, compare
\code{x[0]} to \code{y[0]}, then \code{x[1]} to \code{y[1]}, etc. until there is a difference.

The result is that the \code{offsets} list is ordered by its first
term, and the second term can be used to figure out where the original
data was stored.  (The \code{for} loop assigns \code{dontcare} and
\code{i} to the two fields of each term in the list, but we only need the
index value.)

Another way to implement this is to store the original data as the
second term in the \code{offsets} list, as in:

\begin{verbatim}
>>> words = string.split("This is a test string from Andrew.")
>>> offsets = []
>>> for word in words:
>>>     offsets.append( (string.lower(word), word) )
>>> 
>>> offsets.sort()
>>> new_words = []
>>> for word in offsets:
>>>     new_words.append(word[1])
>>> 
>>> print new_words
\end{verbatim}

This isn't always appropriate because the second terms in the list
(the word, in this example) will be compared when the first terms are
the same.  If this happens many times, then there will be the unneeded
performance hit of comparing the two objects.  This can be a large
cost if most terms are the same and the objects define their own
\method{__cmp__} method, but there will still be some overhead to determine if
\method{__cmp__} is defined.

Still, for large lists, or for lists where the comparison information
is expensive to calculate, the last two examples are likely to be the
fastest way to sort a list.  It will not work on weakly sorted data,
like complex numbers, but if you don't know what that means, you
probably don't need to worry about it.

\section{Comparing classes}

The comparison for two basic data types, like ints to ints or string to
string, is built into Python and makes sense.  There is a default way
to compare class instances, but the default manner isn't usually very
useful.  You can define your own comparison with the \method{__cmp__} method,
as in:

\begin{verbatim}
>>> class Spam:
>>>     def __init__(self, spam, eggs):
>>>         self.spam = spam
>>>         self.eggs = eggs
>>>     def __cmp__(self, other):
>>>         return cmp(self.spam+self.eggs, other.spam+other.eggs)
>>>     def __str__(self):
>>>         return str(self.spam + self.eggs)
>>> 
>>> a = [Spam(1, 4), Spam(9, 3), Spam(4,6)]
>>> a.sort()
>>> for spam in a:
>>>   print str(spam)
5
10
12
\end{verbatim}

Sometimes you may want to sort by a specific attribute of a class.  If
appropriate you should just define the \method{__cmp__} method to compare
those values, but you cannot do this if you want to compare between
different attributes at different times.  Instead, you'll need to go
back to passing a comparison function to sort, as in:

\begin{verbatim}
>>> a = [Spam(1, 4), Spam(9, 3), Spam(4,6)]
>>> a.sort(lambda x, y: cmp(x.eggs, y.eggs))
>>> for spam in a:
>>>   print spam.eggs, str(spam)
3 12
4 5
6 10
\end{verbatim}

If you want to compare two arbitrary attributes (and aren't overly
concerned about performance) you can even define your own comparison
function object.  This uses the ability of a class instance to emulate
an function by defining the \method{__call__} method, as in:

\begin{verbatim}
>>> class CmpAttr:
>>>     def __init__(self, attr):
>>>         self.attr = attr
>>>     def __call__(self, x, y):
>>>         return cmp(getattr(x, self.attr), getattr(y, self.attr))
>>> 
>>> a = [Spam(1, 4), Spam(9, 3), Spam(4,6)]
>>> a.sort(CmpAttr("spam"))  # sort by the "spam" attribute
>>> for spam in a:
>>>    print spam.spam, spam.eggs, str(spam)
1 4 5
4 6 10
9 3 12

>>> a.sort(CmpAttr("eggs"))   # re-sort by the "eggs" attribute
>>> for spam in a:
>>>    print spam.spam, spam.eggs, str(spam)
9 3 12
1 4 5
4 6 10
\end{verbatim}

Of course, if you want a faster sort you can extract the attributes
into an intermediate list and sort that list.


So, there you have it; about a half-dozen different ways to define how
to sort a list:
\begin{itemize}
 \item sort using the default method
 \item sort using a comparison function
 \item reverse sort not using a comparison function
 \item sort on an intermediate list (two forms)
 \item sort using class defined __cmp__ method
 \item sort using a sort function object
\end{itemize}

\end{document}
% LocalWords:  maxint

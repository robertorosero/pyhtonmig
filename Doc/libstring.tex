\section{Standard Module \sectcode{string}}

\stmodindex{string}

This module defines some constants useful for checking character
classes, some exceptions, and some useful string functions.
The constants are:

\renewcommand{\indexsubitem}{(data in module string)}
\begin{datadesc}{digits}
  The string \code{'0123456789'}.
\end{datadesc}

\begin{datadesc}{hexdigits}
  The string \code{'0123456789abcdefABCDEF'}.
\end{datadesc}

\begin{datadesc}{letters}
  The concatenation of the strings \code{lowercase} and
  \code{uppercase} described below.
\end{datadesc}

\begin{datadesc}{lowercase}
  A string containing all the characters that are considered lowercase
  letters.  On most systems this is the string
  \code{'abcdefghijklmnopqrstuvwxyz'}.  Do not change its definition --
  the effect on the routines \code{upper} and \code{swapcase} is
  undefined.
\end{datadesc}

\begin{datadesc}{octdigits}
  The string \code{'01234567'}.
\end{datadesc}

\begin{datadesc}{uppercase}
  A string containing all the characters that are considered uppercase
  letters.  On most systems this is the string
  \code{'ABCDEFGHIJKLMNOPQRSTUVWXYZ'}.  Do not change its definition --
  the effect on the routines \code{lower} and \code{swapcase} is
  undefined.
\end{datadesc}

\begin{datadesc}{whitespace}
  A string containing all characters that are considered whitespace.
  On most systems this includes the characters space, tab, linefeed,
  return, formfeed, and vertical tab.  Do not change its definition --
  the effect on the routines \code{strip} and \code{split} is
  undefined.
\end{datadesc}

The exceptions are:

\renewcommand{\indexsubitem}{(exception in module string)}

\begin{excdesc}{atof_error}
Exception raised by
\code{atof}
when a non-float string argument is detected.
The exception argument is the offending string.
\end{excdesc}

\begin{excdesc}{atoi_error}
Exception raised by
\code{atoi}
when a non-integer string argument is detected.
The exception argument is the offending string.
\end{excdesc}

\begin{excdesc}{atol_error}
Exception raised by
\code{atol}
when a non-integer string argument is detected.
The exception argument is the offending string.
\end{excdesc}

\begin{excdesc}{index_error}
Exception raised by \code{index} when \var{sub} is not found.
The exception argument is undefined (it may be a tuple containing the
offending arguments to \code{index} or it may be the constant string
\code{'substring not found'}).
\end{excdesc}

The functions are:

\renewcommand{\indexsubitem}{(in module string)}

\begin{funcdesc}{atof}{s}
Convert a string to a floating point number.  The string must have
the standard syntax for a floating point literal in Python, optionally
preceded by a sign (\samp{+} or \samp{-}).
\end{funcdesc}

\begin{funcdesc}{atoi}{s}
Convert a string to an integer.  The string must consist of one or more
digits, optionally preceded by a sign (\samp{+} or \samp{-}).
\end{funcdesc}

\begin{funcdesc}{atol}{s}
Convert a string to a long integer.  The string must consist of one
or more digits, optionally preceded by a sign (\samp{+} or \samp{-}).
\end{funcdesc}

\begin{funcdesc}{expandtabs}{s\, tabsize}
Expand tabs in a string, i.e. replace them by one or more spaces,
depending on the current column and the given tab size.  The column
number is reset to zero after each newline occurring in the string.
This doesn't understand other non-printing characters or escape
sequences.
\end{funcdesc}

\begin{funcdesc}{find}{s\, sub\, i}
Return the lowest index in \var{s} not smaller than \var{i} where the
substring \var{sub} is found.  Return \code{-1} when \var{sub}
does not occur as a substring of \var{s} with index at least \var{i}.
If \var{i} is omitted, it defaults to \code{0}.  If \var{i} is
negative, \code{len(\var{s})} is added.
\end{funcdesc}

\begin{funcdesc}{rfind}{s\, sub\, i}
Like \code{find} but finds the highest index.
\end{funcdesc}

\begin{funcdesc}{index}{s\, sub\, i}
Like \code{find} but raise \code{index_error} when the substring is
not found.
\end{funcdesc}

\begin{funcdesc}{rindex}{s\, sub\, i}
Like \code{rfind} but raise \code{index_error} when the substring is
not found.
\end{funcdesc}

\begin{funcdesc}{lower}{s}
Convert letters to lower case.
\end{funcdesc}

\begin{funcdesc}{split}{s}
Returns a list of the whitespace-delimited words of the string
\var{s}.
\end{funcdesc}

\begin{funcdesc}{splitfields}{s\, sep}
  Returns a list containing the fields of the string \var{s}, using
  the string \var{sep} as a separator.  The list will have one more
  items than the number of non-overlapping occurrences of the
  separator in the string.  Thus, \code{string.splitfields(\var{s}, '
  ')} is not the same as \code{string.split(\var{s})}, as the latter
  only returns non-empty words.  As a special case,
  \code{splitfields(\var{s}, '')} returns \code{[\var{s}]}, for any string
  \var{s}.  (See also \code{regsub.split()}.)
\end{funcdesc}

\begin{funcdesc}{join}{words}
Concatenate a list or tuple of words with intervening spaces.
\end{funcdesc}

\begin{funcdesc}{joinfields}{words\, sep}
Concatenate a list or tuple of words with intervening separators.
It is always true that
\code{string.joinfields(string.splitfields(\var{t}, \var{sep}), \var{sep})}
equals \var{t}.
\end{funcdesc}

\begin{funcdesc}{strip}{s}
Removes leading and trailing whitespace from the string
\var{s}.
\end{funcdesc}

\begin{funcdesc}{swapcase}{s}
Converts lower case letters to upper case and vice versa.
\end{funcdesc}

\begin{funcdesc}{upper}{s}
Convert letters to upper case.
\end{funcdesc}

\begin{funcdesc}{ljust}{s\, width}
\funcline{rjust}{s\, width}
\funcline{center}{s\, width}
These functions respectively left-justify, right-justify and center a
string in a field of given width.
They return a string that is at least
\var{width}
characters wide, created by padding the string
\var{s}
with spaces until the given width on the right, left or both sides.
The string is never truncated.
\end{funcdesc}

\begin{funcdesc}{zfill}{s\, width}
Pad a numeric string on the left with zero digits until the given
width is reached.  Strings starting with a sign are handled correctly.
\end{funcdesc}

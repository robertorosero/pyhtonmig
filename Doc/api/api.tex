\documentstyle[twoside,11pt,myformat]{report}

\title{Python/C API Reference Manual}

\author{
	Guido van Rossum \\
	Dept. AA, CWI, P.O. Box 94079 \\
	1090 GB Amsterdam, The Netherlands \\
	E-mail: {\tt guido@cwi.nl}
}

\date{17 March 1995 \\ Release 1.2-proof-2} % XXX update before release!


\makeindex			% tell \index to actually write the .idx file


\begin{document}

\pagenumbering{roman}

\maketitle

\strong{BEOPEN.COM TERMS AND CONDITIONS FOR PYTHON 2.0}

\centerline{\strong{BEOPEN PYTHON OPEN SOURCE LICENSE AGREEMENT VERSION 1}}

\begin{enumerate}

\item
This LICENSE AGREEMENT is between BeOpen.com (``BeOpen''), having an
office at 160 Saratoga Avenue, Santa Clara, CA 95051, and the
Individual or Organization (``Licensee'') accessing and otherwise
using this software in source or binary form and its associated
documentation (``the Software'').

\item
Subject to the terms and conditions of this BeOpen Python License
Agreement, BeOpen hereby grants Licensee a non-exclusive,
royalty-free, world-wide license to reproduce, analyze, test, perform
and/or display publicly, prepare derivative works, distribute, and
otherwise use the Software alone or in any derivative version,
provided, however, that the BeOpen Python License is retained in the
Software, alone or in any derivative version prepared by Licensee.

\item
BeOpen is making the Software available to Licensee on an ``AS IS''
basis.  BEOPEN MAKES NO REPRESENTATIONS OR WARRANTIES, EXPRESS OR
IMPLIED.  BY WAY OF EXAMPLE, BUT NOT LIMITATION, BEOPEN MAKES NO AND
DISCLAIMS ANY REPRESENTATION OR WARRANTY OF MERCHANTABILITY OR FITNESS
FOR ANY PARTICULAR PURPOSE OR THAT THE USE OF THE SOFTWARE WILL NOT
INFRINGE ANY THIRD PARTY RIGHTS.

\item
BEOPEN SHALL NOT BE LIABLE TO LICENSEE OR ANY OTHER USERS OF THE
SOFTWARE FOR ANY INCIDENTAL, SPECIAL, OR CONSEQUENTIAL DAMAGES OR LOSS
AS A RESULT OF USING, MODIFYING OR DISTRIBUTING THE SOFTWARE, OR ANY
DERIVATIVE THEREOF, EVEN IF ADVISED OF THE POSSIBILITY THEREOF.

\item
This License Agreement will automatically terminate upon a material
breach of its terms and conditions.

\item
This License Agreement shall be governed by and interpreted in all
respects by the law of the State of California, excluding conflict of
law provisions.  Nothing in this License Agreement shall be deemed to
create any relationship of agency, partnership, or joint venture
between BeOpen and Licensee.  This License Agreement does not grant
permission to use BeOpen trademarks or trade names in a trademark
sense to endorse or promote products or services of Licensee, or any
third party.  As an exception, the ``BeOpen Python'' logos available
at http://www.pythonlabs.com/logos.html may be used according to the
permissions granted on that web page.

\item
By copying, installing or otherwise using the software, Licensee
agrees to be bound by the terms and conditions of this License
Agreement.
\end{enumerate}


\centerline{\strong{CNRI OPEN SOURCE LICENSE AGREEMENT}}

Python 1.6 is made available subject to the terms and conditions in
CNRI's License Agreement.  This Agreement together with Python 1.6 may
be located on the Internet using the following unique, persistent
identifier (known as a handle): 1895.22/1012.  This Agreement may also
be obtained from a proxy server on the Internet using the following
URL: \url{http://hdl.handle.net/1895.22/1012}.


\centerline{\strong{CWI PERMISSIONS STATEMENT AND DISCLAIMER}}

Copyright \copyright{} 1991 - 1995, Stichting Mathematisch Centrum
Amsterdam, The Netherlands.  All rights reserved.

Permission to use, copy, modify, and distribute this software and its
documentation for any purpose and without fee is hereby granted,
provided that the above copyright notice appear in all copies and that
both that copyright notice and this permission notice appear in
supporting documentation, and that the name of Stichting Mathematisch
Centrum or CWI not be used in advertising or publicity pertaining to
distribution of the software without specific, written prior
permission.

STICHTING MATHEMATISCH CENTRUM DISCLAIMS ALL WARRANTIES WITH REGARD TO
THIS SOFTWARE, INCLUDING ALL IMPLIED WARRANTIES OF MERCHANTABILITY AND
FITNESS, IN NO EVENT SHALL STICHTING MATHEMATISCH CENTRUM BE LIABLE
FOR ANY SPECIAL, INDIRECT OR CONSEQUENTIAL DAMAGES OR ANY DAMAGES
WHATSOEVER RESULTING FROM LOSS OF USE, DATA OR PROFITS, WHETHER IN AN
ACTION OF CONTRACT, NEGLIGENCE OR OTHER TORTIOUS ACTION, ARISING OUT
OF OR IN CONNECTION WITH THE USE OR PERFORMANCE OF THIS SOFTWARE.


\begin{abstract}

\noindent
This manual documents the API used by C (or C++) programmers who want
to write extension modules or embed Python.  It is a companion to
``Extending and Embedding the Python Interpreter'', which describes
the general principles of extension writing but does not document the
API functions in detail.

\strong{Warning:} The current version of this document is incomplete.
I hope that it is nevertheless useful.  I will continue to work on it,
and release new versions from time to time, independent from Python
source code releases.

\end{abstract}

\pagebreak

{
\parskip = 0mm
\tableofcontents
}

\pagebreak

\pagenumbering{arabic}

% XXX Consider moving all this back to ext.tex and giving api.tex
% XXX a *really* short intro only.

\chapter{Introduction}

The Application Programmer's Interface to Python gives C and C++
programmers access to the Python interpreter at a variety of levels.
The API is equally usable from C++, but for brevity it is generally
referred to as the Python/C API.  There are two fundamentally
different reasons for using the Python/C API.  The first reason is to
write ``extension modules'' for specific purposes; these are C modules
that extend the Python interpreter.  This is probably the most common
use.  The second reason is to use Python as a component in a larger
application; this technique is generally referred to as ``embedding''
Python in an application.

Writing an extension module is a relatively well-understood process, 
where a ``cookbook'' approach works well.  There are several tools 
that automate the process to some extent.  While people have embedded 
Python in other applications since its early existence, the process of 
embedding Python is less straightforward that writing an extension.  
Python 1.5 introduces a number of new API functions as well as some 
changes to the build process that make embedding much simpler.  
This manual describes the 1.5 state of affair.
% XXX Eventually, take the historical notes out

Many API functions are useful independent of whether you're embedding 
or extending Python; moreover, most applications that embed Python 
will need to provide a custom extension as well, so it's probably a 
good idea to become familiar with writing an extension before 
attempting to embed Python in a real application.

\section{Include Files}

All function, type and macro definitions needed to use the Python/C
API are included in your code by the following line:

\code{\#include "Python.h"}

This implies inclusion of the following standard header files:
stdio.h, string.h, errno.h, and stdlib.h (if available).

All user visible names defined by Python.h (except those defined by
the included standard headers) have one of the prefixes \code{Py} or
\code{_Py}.  Names beginning with \code{_Py} are for internal use
only.  Structure member names do not have a reserved prefix.

Important: user code should never define names that begin with
\code{Py} or \code{_Py}.  This confuses the reader, and jeopardizes
the portability of the user code to future Python versions, which may
define additional names beginning with one of these prefixes.

\section{Objects, Types and Reference Counts}

Most Python/C API functions have one or more arguments as well as a
return value of type \code{PyObject *}.  This type is a pointer
(obviously!)  to an opaque data type representing an arbitrary Python
object.  Since all Python object types are treated the same way by the
Python language in most situations (e.g., assignments, scope rules,
and argument passing), it is only fitting that they should be
represented by a single C type.  All Python objects live on the heap:
you never declare an automatic or static variable of type
\code{PyObject}, only pointer variables of type \code{PyObject *} can 
be declared.

All Python objects (even Python integers) have a ``type'' and a 
``reference count''.  An object's type determines what kind of object 
it is (e.g., an integer, a list, or a user-defined function; there are 
many more as explained in the Python Language Reference Manual).  For 
each of the well-known types there is a macro to check whether an 
object is of that type; for instance, \code{PyList_Check(a)} is true 
iff the object pointed to by \code{a} is a Python list.

\subsection{Reference Counts}

The reference count is important because today's computers have a 
finite (and often severly limited) memory size; it counts how many 
different places there are that have a reference to an object.  Such a 
place could be another object, or a global (or static) C variable, or 
a local variable in some C function.  When an object's reference count 
becomes zero, the object is deallocated.  If it contains references to 
other objects, their reference count is decremented.  Those other 
objects may be deallocated in turn, if this decrement makes their 
reference count become zero, and so on.  (There's an obvious problem 
with objects that reference each other here; for now, the solution is 
``don't do that''.)

Reference counts are always manipulated explicitly.  The normal way is 
to use the macro \code{Py_INCREF(a)} to increment an object's 
reference count by one, and \code{Py_DECREF(a)} to decrement it by 
one.  The decref macro is considerably more complex than the incref one, 
since it must check whether the reference count becomes zero and then 
cause the object's deallocator, which is a function pointer contained 
in the object's type structure.  The type-specific deallocator takes 
care of decrementing the reference counts for other objects contained 
in the object, and so on, if this is a compound object type such as a 
list.  There's no chance that the reference count can overflow; at 
least as many bits are used to hold the reference count as there are 
distinct memory locations in virtual memory (assuming 
\code{sizeof(long) >= sizeof(char *)}).  Thus, the reference count 
increment is a simple operation.

It is not necessary to increment an object's reference count for every 
local variable that contains a pointer to an object.  In theory, the 
oject's reference count goes up by one when the variable is made to 
point to it and it goes down by one when the variable goes out of 
scope.  However, these two cancel each other out, so at the end the 
reference count hasn't changed.  The only real reason to use the 
reference count is to prevent the object from being deallocated as 
long as our variable is pointing to it.  If we know that there is at 
least one other reference to the object that lives at least as long as 
our variable, there is no need to increment the reference count 
temporarily.  An important situation where this arises is in objects 
that are passed as arguments to C functions in an extension module 
that are called from Python; the call mechanism guarantees to hold a 
reference to every argument for the duration of the call.

However, a common pitfall is to extract an object from a list and 
holding on to it for a while without incrementing its reference count.  
Some other operation might conceivably remove the object from the 
list, decrementing its reference count and possible deallocating it.  
The real danger is that innocent-looking operations may invoke 
arbitrary Python code which could do this; there is a code path which 
allows control to flow back to the user from a \code{Py_DECREF()}, so 
almost any operation is potentially dangerous.

A safe approach is to always use the generic operations (functions 
whose name begins with \code{PyObject_}, \code{PyNumber_}, 
\code{PySequence_} or \code{PyMapping_}).  These operations always 
increment the reference count of the object they return.  This leaves 
the caller with the responsibility to call \code{Py_DECREF()} when 
they are done with the result; this soon becomes second nature.

\subsubsection{Reference Count Details}

The reference count behavior of functions in the Python/C API is best 
expelained in terms of \emph{ownership of references}.  Note that we 
talk of owning references, never of owning objects; objects are always 
shared!  When a function owns a reference, it has to dispose of it 
properly -- either by passing ownership on (usually to its caller) or 
by calling \code{Py_DECREF()} or \code{Py_XDECREF()}.  When a function 
passes ownership of a reference on to its caller, the caller is said 
to receive a \emph{new} reference.  When no ownership is transferred, 
the caller is said to \emph{borrow} the reference.  Nothing needs to 
be done for a borrowed reference.

Conversely, when calling a function passes it a reference to an 
object, there are two possibilities: the function \emph{steals} a 
reference to the object, or it does not.  Few functions steal 
references; the two notable exceptions are \code{PyList_SetItem()} and 
\code{PyTuple_SetItem()}, which steal a reference to the item (but not to 
the tuple or list into which the item it put!).  These functions were
designed to steal a reference because of a common idiom for populating
a tuple or list with newly created objects; for example, the code to
create the tuple \code{(1, 2, "three")} could look like this
(forgetting about error handling for the moment; a better way to code
this is shown below anyway):

\begin{verbatim}
PyObject *t;
t = PyTuple_New(3);
PyTuple_SetItem(t, 0, PyInt_FromLong(1L));
PyTuple_SetItem(t, 1, PyInt_FromLong(2L));
PyTuple_SetItem(t, 2, PyString_FromString("three"));
\end{verbatim}

Incidentally, \code{PyTuple_SetItem()} is the \emph{only} way to set
tuple items; \code{PySequence_SetItem()} and \code{PyObject_SetItem()}
refuse to do this since tuples are an immutable data type.  You should
only use \code{PyTuple_SetItem()} for tuples that you are creating
yourself.

Equivalent code for populating a list can be written using 
\code{PyList_New()} and \code{PyList_SetItem()}.  Such code can also 
use \code{PySequence_SetItem()}; this illustrates the difference 
between the two (the extra \code{Py_DECREF()} calls):

\begin{verbatim}
PyObject *l, *x;
l = PyList_New(3);
x = PyInt_FromLong(1L);
PySequence_SetItem(l, 0, x); Py_DECREF(x);
x = PyInt_FromLong(2L);
PySequence_SetItem(l, 1, x); Py_DECREF(x);
x = PyString_FromString("three");
PySequence_SetItem(l, 2, x); Py_DECREF(x);
\end{verbatim}

You might find it strange that the ``recommended'' approach takes more
code.  However, in practice, you will rarely use these ways of
creating and populating a tuple or list.  There's a generic function,
\code{Py_BuildValue()}, that can create most common objects from C 
values, directed by a ``format string''.  For example, the above two 
blocks of code could be replaced by the following (which also takes 
care of the error checking!):

\begin{verbatim}
PyObject *t, *l;
t = Py_BuildValue("(iis)", 1, 2, "three");
l = Py_BuildValue("[iis]", 1, 2, "three");
\end{verbatim}

It is much more common to use \code{PyObject_SetItem()} and friends 
with items whose references you are only borrowing, like arguments 
that were passed in to the function you are writing.  In that case, 
their behaviour regarding reference counts is much saner, since you 
don't have to increment a reference count so you can give a reference 
away (``have it be stolen'').  For example, this function sets all 
items of a list (actually, any mutable sequence) to a given item:

\begin{verbatim}
int set_all(PyObject *target, PyObject *item)
{
    int i, n;
    n = PyObject_Length(target);
    if (n < 0)
        return -1;
    for (i = 0; i < n; i++) {
        if (PyObject_SetItem(target, i, item) < 0)
            return -1;
    }
    return 0;
}
\end{verbatim}

The situation is slightly different for function return values.  
While passing a reference to most functions does not change your 
ownership responsibilities for that reference, many functions that 
return a referece to an object give you ownership of the reference.
The reason is simple: in many cases, the returned object is created 
on the fly, and the reference you get is the only reference to the 
object!  Therefore, the generic functions that return object 
references, like \code{PyObject_GetItem()} and 
\code{PySequence_GetItem()}, always return a new reference (i.e., the 
caller becomes the owner of the reference).

It is important to realize that whether you own a reference returned 
by a function depends on which function you call only -- \emph{the 
plumage} (i.e., the type of the type of the object passed as an 
argument to the function) \emph{don't enter into it!}  Thus, if you 
extract an item from a list using \code{PyList_GetItem()}, you don't 
own the reference -- but if you obtain the same item from the same 
list using \code{PySequence_GetItem()} (which happens to take exactly 
the same arguments), you do own a reference to the returned object.

Here is an example of how you could write a function that computes the 
sum of the items in a list of integers; once using 
\code{PyList_GetItem()}, once using \code{PySequence_GetItem()}.

\begin{verbatim}
long sum_list(PyObject *list)
{
    int i, n;
    long total = 0;
    PyObject *item;
    n = PyList_Size(list);
    if (n < 0)
        return -1; /* Not a list */
    for (i = 0; i < n; i++) {
        item = PyList_GetItem(list, i); /* Can't fail */
        if (!PyInt_Check(item)) continue; /* Skip non-integers */
        total += PyInt_AsLong(item);
    }
    return total;
}
\end{verbatim}

\begin{verbatim}
long sum_sequence(PyObject *sequence)
{
    int i, n;
    long total = 0;
    PyObject *item;
    n = PyObject_Size(list);
    if (n < 0)
        return -1; /* Has no length */
    for (i = 0; i < n; i++) {
        item = PySequence_GetItem(list, i);
        if (item == NULL)
            return -1; /* Not a sequence, or other failure */
        if (PyInt_Check(item))
            total += PyInt_AsLong(item);
        Py_DECREF(item); /* Discard reference ownership */
    }
    return total;
}
\end{verbatim}

\subsection{Types}

There are few other data types that play a significant role in 
the Python/C API; most are simple C types such as \code{int}, 
\code{long}, \code{double} and \code{char *}.  A few structure types 
are used to describe static tables used to list the functions exported 
by a module or the data attributes of a new object type.  These will 
be discussed together with the functions that use them.

\section{Exceptions}

The Python programmer only needs to deal with exceptions if specific 
error handling is required; unhandled exceptions are automatically 
propagated to the caller, then to the caller's caller, and so on, till 
they reach the top-level interpreter, where they are reported to the 
user accompanied by a stack traceback.

For C programmers, however, error checking always has to be explicit.  
All functions in the Python/C API can raise exceptions, unless an 
explicit claim is made otherwise in a function's documentation.  In 
general, when a function encounters an error, it sets an exception, 
discards any object references that it owns, and returns an 
error indicator -- usually \NULL{} or \code{-1}.  A few functions 
return a Boolean true/false result, with false indicating an error.
Very few functions return no explicit error indicator or have an 
ambiguous return value, and require explicit testing for errors with 
\code{PyErr_Occurred()}.

Exception state is maintained in per-thread storage (this is 
equivalent to using global storage in an unthreaded application).  A 
thread can be on one of two states: an exception has occurred, or not.  
The function \code{PyErr_Occurred()} can be used to check for this: it 
returns a borrowed reference to the exception type object when an 
exception has occurred, and \NULL{} otherwise.  There are a number 
of functions to set the exception state: \code{PyErr_SetString()} is 
the most common (though not the most general) function to set the 
exception state, and \code{PyErr_Clear()} clears the exception state.

The full exception state consists of three objects (all of which can 
be \NULL{} ): the exception type, the corresponding exception 
value, and the traceback.  These have the same meanings as the Python 
object \code{sys.exc_type}, \code{sys.exc_value}, 
\code{sys.exc_traceback}; however, they are not the same: the Python 
objects represent the last exception being handled by a Python 
\code{try...except} statement, while the C level exception state only 
exists while an exception is being passed on between C functions until 
it reaches the Python interpreter, which takes care of transferring it 
to \code{sys.exc_type} and friends.

(Note that starting with Python 1.5, the preferred, thread-safe way to 
access the exception state from Python code is to call the function 
\code{sys.exc_info()}, which returns the per-thread exception state 
for Python code.  Also, the semantics of both ways to access the 
exception state have changed so that a function which catches an 
exception will save and restore its thread's exception state so as to 
preserve the exception state of its caller.  This prevents common bugs 
in exception handling code caused by an innocent-looking function 
overwriting the exception being handled; it also reduces the often 
unwanted lifetime extension for objects that are referenced by the 
stack frames in the traceback.)

As a general principle, a function that calls another function to 
perform some task should check whether the called function raised an 
exception, and if so, pass the exception state on to its caller.  It 
should discards any object references that it owns, and returns an 
error indicator, but it should \emph{not} set another exception -- 
that would overwrite the exception that was just raised, and lose 
important reason about the exact cause of the error.

A simple example of detecting exceptions and passing them on is shown 
in the \code{sum_sequence()} example above.  It so happens that that 
example doesn't need to clean up any owned references when it detects 
an error.  The following example function shows some error cleanup.  
First, to remind you why you like Python, we show the equivalent
Python code:

\begin{verbatim}
def incr_item(dict, key):
    try:
        item = dict[key]
    except KeyError:
        item = 0
    return item + 1
\end{verbatim}

Here is the corresponding C code, in all its glory:

\begin{verbatim}
int incr_item(PyObject *dict, PyObject *key)
{
    /* Objects all initialized to NULL for Py_XDECREF */
    PyObject *item = NULL, *const_one = NULL, *incremented_item = NULL;
    int rv = -1; /* Return value initialized to -1 (failure) */

    item = PyObject_GetItem(dict, key);
    if (item == NULL) {
        /* Handle keyError only: */
        if (!PyErr_ExceptionMatches(PyExc_keyError)) goto error;

        /* Clear the error and use zero: */
        PyErr_Clear();
        item = PyInt_FromLong(0L);
        if (item == NULL) goto error;
    }

    const_one = PyInt_FromLong(1L);
    if (const_one == NULL) goto error;

    incremented_item = PyNumber_Add(item, const_one);
    if (incremented_item == NULL) goto error;

    if (PyObject_SetItem(dict, key, incremented_item) < 0) goto error;
    rv = 0; /* Success */
    /* Continue with cleanup code */

 error:
    /* Cleanup code, shared by success and failure path */

    /* Use Py_XDECREF() to ignore NULL references */
    Py_XDECREF(item);
    Py_XDECREF(const_one);
    Py_XDECREF(incremented_item);

    return rv; /* -1 for error, 0 for success */
}
\end{verbatim}

This example represents an endorsed use of the \code{goto} statement 
in C!  It illustrates the use of \code{PyErr_ExceptionMatches()} and 
\code{PyErr_Clear()} to handle specific exceptions, and the use of 
\code{Py_XDECREF()} to dispose of owned references that may be 
\NULL{} (note the `X' in the name; \code{Py_DECREF()} would crash 
when confronted with a \NULL{} reference).  It is important that 
the variables used to hold owned references are initialized to 
\NULL{} for this to work; likewise, the proposed return value is 
initialized to \code{-1} (failure) and only set to success after
the final call made is successful.


\section{Embedding Python}

The one important task that only embedders (as opposed to extension
writers) of the Python interpreter have to worry about is the
initialization, and possibly the finalization, of the Python
interpreter.  Most functionality of the interpreter can only be used
after the interpreter has been initialized.

The basic initialization function is \code{Py_Initialize()}.  This 
initializes the table of loaded modules, and creates the fundamental 
modules \code{__builtin__}, \code{__main__} and \code{sys}.  It also 
initializes the module search path (\code{sys.path}).

\code{Py_Initialize()} does not set the ``script argument list'' 
(\code{sys.argv}).  If this variable is needed by Python code that 
will be executed later, it must be set explicitly with a call to 
\code{PySys_SetArgv(\var{argc}, \var{argv})} subsequent to the call 
to \code{Py_Initialize()}.

On most systems (in particular, on Unix and Windows, although the
details are slightly different), \code{Py_Initialize()} calculates the
module search path based upon its best guess for the location of the
standard Python interpreter executable, assuming that the Python
library is found in a fixed location relative to the Python
interpreter executable.  In particular, it looks for a directory named
\code{lib/python1.5} (replacing \code{1.5} with the current
interpreter version) relative to the parent directory where the
executable named \code{python} is found on the shell command search
path (the environment variable \code{\$PATH}).

For instance, if the Python executable is found in
\code{/usr/local/bin/python}, it will assume that the libraries are in
\code{/usr/local/lib/python1.5}.  (In fact, this particular path is
also the ``fallback'' location, used when no executable file named
\code{python} is found along \code{\$PATH}.)  The user can override
this behavior by setting the environment variable \code{\$PYTHONHOME},
or insert additional directories in front of the standard path by
setting \code{\$PYTHONPATH}.

The embedding application can steer the search by calling 
\code{Py_SetProgramName(\var{file})} \emph{before} calling 
\code{Py_Initialize()}.  Note that \code{\$PYTHONHOME} still overrides 
this and \code{\$PYTHONPATH} is still inserted in front of the 
standard path.  An application that requires total control has to
provide its own implementation of \code{Py_GetPath()},
\code{Py_GetPrefix()}, \code{Py_GetExecPrefix()},
\code{Py_GetProgramFullPath()} (all defined in
\file{Modules/getpath.c}).

Sometimes, it is desirable to ``uninitialize'' Python.  For instance, 
the application may want to start over (make another call to 
\code{Py_Initialize()}) or the application is simply done with its 
use of Python and wants to free all memory allocated by Python.  This
can be accomplished by calling \code{Py_Finalize()}.  The function
\code{Py_IsInitialized()} returns true iff Python is currently in the
initialized state.  More information about these functions is given in
a later chapter.


\chapter{Basic Utilities}

XXX These utilities should be moved to some other section...

\begin{cfuncdesc}{void}{Py_FatalError}{char *message}
Print a fatal error message and kill the process.  No cleanup is
performed.  This function should only be invoked when a condition is
detected that would make it dangerous to continue using the Python
interpreter; e.g., when the object administration appears to be
corrupted.  On Unix, the standard C library function \code{abort()} is 
called which will attempt to produce a \file{core} file.
\end{cfuncdesc}

\begin{cfuncdesc}{void}{Py_Exit}{int status}
Exit the current process.  This calls \code{Py_Finalize()} and then
calls the standard C library function \code{exit(0)}.
\end{cfuncdesc}

\begin{cfuncdesc}{int}{Py_AtExit}{void (*func) ()}
Register a cleanup function to be called by \code{Py_Finalize()}.  The
cleanup function will be called with no arguments and should return no
value.  At most 32 cleanup functions can be registered.  When the
registration is successful, \code{Py_AtExit} returns 0; on failure, it
returns -1.  The cleanup function registered last is called first.
Each cleanup function will be called at most once.
\end{cfuncdesc}


\chapter{Reference Counting}

The macros in this section are used for managing reference counts
of Python objects.

\begin{cfuncdesc}{void}{Py_INCREF}{PyObject *o}
Increment the reference count for object \code{o}.  The object must
not be \NULL{}; if you aren't sure that it isn't \NULL{}, use
\code{Py_XINCREF()}.
\end{cfuncdesc}

\begin{cfuncdesc}{void}{Py_XINCREF}{PyObject *o}
Increment the reference count for object \code{o}.  The object may be
\NULL{}, in which case the macro has no effect.
\end{cfuncdesc}

\begin{cfuncdesc}{void}{Py_DECREF}{PyObject *o}
Decrement the reference count for object \code{o}.  The object must
not be \NULL{}; if you aren't sure that it isn't \NULL{}, use
\code{Py_XDECREF()}.  If the reference count reaches zero, the object's
type's deallocation function (which must not be \NULL{}) is invoked.

\strong{Warning:} The deallocation function can cause arbitrary Python
code to be invoked (e.g. when a class instance with a \code{__del__()}
method is deallocated).  While exceptions in such code are not
propagated, the executed code has free access to all Python global
variables.  This means that any object that is reachable from a global
variable should be in a consistent state before \code{Py_DECREF()} is
invoked.  For example, code to delete an object from a list should
copy a reference to the deleted object in a temporary variable, update
the list data structure, and then call \code{Py_DECREF()} for the
temporary variable.
\end{cfuncdesc}

\begin{cfuncdesc}{void}{Py_XDECREF}{PyObject *o}
Decrement the reference count for object \code{o}.The object may be
\NULL{}, in which case the macro has no effect; otherwise the
effect is the same as for \code{Py_DECREF()}, and the same warning
applies.
\end{cfuncdesc}

The following functions or macros are only for internal use:
\code{_Py_Dealloc}, \code{_Py_ForgetReference}, \code{_Py_NewReference},
as well as the global variable \code{_Py_RefTotal}.

XXX Should mention Py_Malloc(), Py_Realloc(), Py_Free(),
PyMem_Malloc(), PyMem_Realloc(), PyMem_Free(), PyMem_NEW(),
PyMem_RESIZE(), PyMem_DEL(), PyMem_XDEL().


\chapter{Exception Handling}

The functions in this chapter will let you handle and raise Python
exceptions.  It is important to understand some of the basics of
Python exception handling.  It works somewhat like the Unix
\code{errno} variable: there is a global indicator (per thread) of the
last error that occurred.  Most functions don't clear this on success,
but will set it to indicate the cause of the error on failure.  Most
functions also return an error indicator, usually \NULL{} if they are
supposed to return a pointer, or -1 if they return an integer
(exception: the \code{PyArg_Parse*()} functions return 1 for success and
0 for failure).  When a function must fail because some function it
called failed, it generally doesn't set the error indicator; the
function it called already set it.

The error indicator consists of three Python objects corresponding to
the Python variables \code{sys.exc_type}, \code{sys.exc_value} and
\code{sys.exc_traceback}.  API functions exist to interact with the
error indicator in various ways.  There is a separate error indicator
for each thread.

% XXX Order of these should be more thoughtful.
% Either alphabetical or some kind of structure.

\begin{cfuncdesc}{void}{PyErr_Print}{}
Print a standard traceback to \code{sys.stderr} and clear the error
indicator.  Call this function only when the error indicator is set.
(Otherwise it will cause a fatal error!)
\end{cfuncdesc}

\begin{cfuncdesc}{PyObject *}{PyErr_Occurred}{}
Test whether the error indicator is set.  If set, return the exception
\code{type} (the first argument to the last call to one of the
\code{PyErr_Set*()} functions or to \code{PyErr_Restore()}).  If not
set, return \NULL{}.  You do not own a reference to the return value,
so you do not need to \code{Py_DECREF()} it.  Note: do not compare the
return value to a specific exception; use
\code{PyErr_ExceptionMatches} instead, shown below.
\end{cfuncdesc}

\begin{cfuncdesc}{int}{PyErr_ExceptionMatches}{PyObject *exc}
\strong{(NEW in 1.5a4!)}
Equivalent to
\code{PyErr_GivenExceptionMatches(PyErr_Occurred(), \var{exc})}.
This should only be called when an exception is actually set.
\end{cfuncdesc}

\begin{cfuncdesc}{int}{PyErr_GivenExceptionMatches}{PyObject *given, PyObject *exc}
\strong{(NEW in 1.5a4!)}
Return true if the \var{given} exception matches the exception in
\var{exc}.  If \var{exc} is a class object, this also returns true
when \var{given} is a subclass.  If \var{exc} is a tuple, all
exceptions in the tuple (and recursively in subtuples) are searched
for a match.  This should only be called when an exception is actually
set.
\end{cfuncdesc}

\begin{cfuncdesc}{void}{PyErr_NormalizeException}{PyObject**exc, PyObject**val, PyObject**tb}
\strong{(NEW in 1.5a4!)}
Under certain circumstances, the values returned by
\code{PyErr_Fetch()} below can be ``unnormalized'', meaning that
\var{*exc} is a class object but \var{*val} is not an instance of the
same class.  This function can be used to instantiate the class in
that case.  If the values are already normalized, nothing happens.
\end{cfuncdesc}

\begin{cfuncdesc}{void}{PyErr_Clear}{}
Clear the error indicator.  If the error indicator is not set, there
is no effect.
\end{cfuncdesc}

\begin{cfuncdesc}{void}{PyErr_Fetch}{PyObject **ptype, PyObject **pvalue, PyObject **ptraceback}
Retrieve the error indicator into three variables whose addresses are
passed.  If the error indicator is not set, set all three variables to
\NULL{}.  If it is set, it will be cleared and you own a reference to
each object retrieved.  The value and traceback object may be \NULL{}
even when the type object is not.  \strong{Note:} this function is
normally only used by code that needs to handle exceptions or by code
that needs to save and restore the error indicator temporarily.
\end{cfuncdesc}

\begin{cfuncdesc}{void}{PyErr_Restore}{PyObject *type, PyObject *value, PyObject *traceback}
Set  the error indicator from the three objects.  If the error
indicator is already set, it is cleared first.  If the objects are
\NULL{}, the error indicator is cleared.  Do not pass a \NULL{} type
and non-\NULL{} value or traceback.  The exception type should be a
string or class; if it is a class, the value should be an instance of
that class.  Do not pass an invalid exception type or value.
(Violating these rules will cause subtle problems later.)  This call
takes away a reference to each object, i.e. you must own a reference
to each object before the call and after the call you no longer own
these references.  (If you don't understand this, don't use this
function.  I warned you.)  \strong{Note:} this function is normally
only used by code that needs to save and restore the error indicator
temporarily.
\end{cfuncdesc}

\begin{cfuncdesc}{void}{PyErr_SetString}{PyObject *type, char *message}
This is the most common way to set the error indicator.  The first
argument specifies the exception type; it is normally one of the
standard exceptions, e.g. \code{PyExc_RuntimeError}.  You need not
increment its reference count.  The second argument is an error
message; it is converted to a string object.
\end{cfuncdesc}

\begin{cfuncdesc}{void}{PyErr_SetObject}{PyObject *type, PyObject *value}
This function is similar to \code{PyErr_SetString()} but lets you
specify an arbitrary Python object for the ``value'' of the exception.
You need not increment its reference count.
\end{cfuncdesc}

\begin{cfuncdesc}{void}{PyErr_SetNone}{PyObject *type}
This is a shorthand for \code{PyErr_SetString(\var{type}, Py_None}.
\end{cfuncdesc}

\begin{cfuncdesc}{int}{PyErr_BadArgument}{}
This is a shorthand for \code{PyErr_SetString(PyExc_TypeError,
\var{message})}, where \var{message} indicates that a built-in operation
was invoked with an illegal argument.  It is mostly for internal use.
\end{cfuncdesc}

\begin{cfuncdesc}{PyObject *}{PyErr_NoMemory}{}
This is a shorthand for \code{PyErr_SetNone(PyExc_MemoryError)}; it
returns \NULL{} so an object allocation function can write
\code{return PyErr_NoMemory();} when  it runs out of memory.
\end{cfuncdesc}

\begin{cfuncdesc}{PyObject *}{PyErr_SetFromErrno}{PyObject *type}
This is a convenience function to raise an exception when a C library
function has returned an error and set the C variable \code{errno}.
It constructs a tuple object whose first item is the integer
\code{errno} value and whose second item is the corresponding error
message (gotten from \code{strerror()}), and then calls
\code{PyErr_SetObject(\var{type}, \var{object})}.  On \UNIX{}, when
the \code{errno} value is \code{EINTR}, indicating an interrupted
system call, this calls \code{PyErr_CheckSignals()}, and if that set
the error indicator, leaves it set to that.  The function always
returns \NULL{}, so a wrapper function around a system call can write 
\code{return PyErr_NoMemory();} when  the system call returns an error.
\end{cfuncdesc}

\begin{cfuncdesc}{void}{PyErr_BadInternalCall}{}
This is a shorthand for \code{PyErr_SetString(PyExc_TypeError,
\var{message})}, where \var{message} indicates that an internal
operation (e.g. a Python/C API function) was invoked with an illegal
argument.  It is mostly for internal use.
\end{cfuncdesc}

\begin{cfuncdesc}{int}{PyErr_CheckSignals}{}
This function interacts with Python's signal handling.  It checks
whether a signal has been sent to the processes and if so, invokes the
corresponding signal handler.  If the \code{signal} module is
supported, this can invoke a signal handler written in Python.  In all
cases, the default effect for \code{SIGINT} is to raise the
\code{KeyboadInterrupt} exception.  If an exception is raised the
error indicator is set and the function returns 1; otherwise the
function returns 0.  The error indicator may or may not be cleared if
it was previously set.
\end{cfuncdesc}

\begin{cfuncdesc}{void}{PyErr_SetInterrupt}{}
This function is obsolete (XXX or platform dependent?).  It simulates
the effect of a \code{SIGINT} signal arriving -- the next time
\code{PyErr_CheckSignals()} is called, \code{KeyboadInterrupt} will be
raised.
\end{cfuncdesc}

\begin{cfuncdesc}{PyObject *}{PyErr_NewException}{char *name,
PyObject *base, PyObject *dict}
\strong{(NEW in 1.5a4!)}
This utility function creates and returns a new exception object.  The
\var{name} argument must be the name of the new exception, a C string
of the form \code{module.class}.  The \var{base} and \var{dict}
arguments are normally \NULL{}.  Normally, this creates a class
object derived from the root for all exceptions, the built-in name
\code{Exception} (accessible in C as \code{PyExc_Exception}).  In this
case the \code{__module__} attribute of the new class is set to the
first part (up to the last dot) of the \var{name} argument, and the
class name is set to the last part (after the last dot).  When the
user has specified the \code{-X} command line option to use string
exceptions, for backward compatibility, or when the \var{base}
argument is not a class object (and not \NULL{}), a string object
created from the entire \var{name} argument is returned.  The
\var{base} argument can be used to specify an alternate base class.
The \var{dict} argument can be used to specify a dictionary of class
variables and methods.
\end{cfuncdesc}


\section{Standard Exceptions}

All standard Python exceptions are available as global variables whose
names are \code{PyExc_} followed by the Python exception name.
These have the type \code{PyObject *}; they are all string objects.
For completeness, here are all the variables (the first four are new
in Python 1.5a4):
\code{PyExc_Exception},
\code{PyExc_StandardError},
\code{PyExc_ArithmeticError},
\code{PyExc_LookupError},
\code{PyExc_AssertionError},
\code{PyExc_AttributeError},
\code{PyExc_EOFError},
\code{PyExc_FloatingPointError},
\code{PyExc_IOError},
\code{PyExc_ImportError},
\code{PyExc_IndexError},
\code{PyExc_KeyError},
\code{PyExc_KeyboardInterrupt},
\code{PyExc_MemoryError},
\code{PyExc_NameError},
\code{PyExc_OverflowError},
\code{PyExc_RuntimeError},
\code{PyExc_SyntaxError},
\code{PyExc_SystemError},
\code{PyExc_SystemExit},
\code{PyExc_TypeError},
\code{PyExc_ValueError},
\code{PyExc_ZeroDivisionError}.


\chapter{Utilities}

The functions in this chapter perform various utility tasks, such as
parsing function arguments and constructing Python values from C
values.

\section{OS Utilities}

\begin{cfuncdesc}{int}{Py_FdIsInteractive}{FILE *fp, char *filename}
Return true (nonzero) if the standard I/O file \code{fp} with name
\code{filename} is deemed interactive.  This is the case for files for
which \code{isatty(fileno(fp))} is true.  If the global flag
\code{Py_InteractiveFlag} is true, this function also returns true if
the \code{name} pointer is \NULL{} or if the name is equal to one of
the strings \code{"<stdin>"} or \code{"???"}.
\end{cfuncdesc}

\begin{cfuncdesc}{long}{PyOS_GetLastModificationTime}{char *filename}
Return the time of last modification of the file \code{filename}.
The result is encoded in the same way as the timestamp returned by
the standard C library function \code{time()}.
\end{cfuncdesc}


\section{Importing modules}

\begin{cfuncdesc}{PyObject *}{PyImport_ImportModule}{char *name}
This is a simplified interface to \code{PyImport_ImportModuleEx}
below, leaving the \var{globals} and \var{locals} arguments set to
\NULL{}.  When the \var{name} argument contains a dot (i.e., when
it specifies a submodule of a package), the \var{fromlist} argument is
set to the list \code{['*']} so that the return value is the named
module rather than the top-level package containing it as would
otherwise be the case.  (Unfortunately, this has an additional side
effect when \var{name} in fact specifies a subpackage instead of a
submodule: the submodules specified in the package's \code{__all__}
variable are loaded.)  Return a new reference to the imported module,
or \NULL{} with an exception set on failure (the module may still
be created in this case).
\end{cfuncdesc}

\begin{cfuncdesc}{PyObject *}{PyImport_ImportModuleEx}{char *name, PyObject *globals, PyObject *locals, PyObject *fromlist}
\strong{(NEW in 1.5a4!)}
Import a module.  This is best described by referring to the built-in
Python function \code{__import()__}, as the standard
\code{__import__()} function calls this function directly.

The return value is a new reference to the imported module or
top-level package, or \NULL{} with an exception set on failure
(the module may still be created in this case).  Like for
\code{__import__()}, the return value when a submodule of a package
was requested is normally the top-level package, unless a non-empty
\var{fromlist} was given.
\end{cfuncdesc}

\begin{cfuncdesc}{PyObject *}{PyImport_Import}{PyObject *name}
This is a higher-level interface that calls the current ``import hook
function''.  It invokes the \code{__import__()} function from the
\code{__builtins__} of the current globals.  This means that the
import is done using whatever import hooks are installed in the
current environment, e.g. by \code{rexec} or \code{ihooks}.
\end{cfuncdesc}

\begin{cfuncdesc}{PyObject *}{PyImport_ReloadModule}{PyObject *m}
Reload a module.  This is best described by referring to the built-in
Python function \code{reload()}, as the standard \code{reload()}
function calls this function directly.  Return a new reference to the
reloaded module, or \NULL{} with an exception set on failure (the
module still exists in this case).
\end{cfuncdesc}

\begin{cfuncdesc}{PyObject *}{PyImport_AddModule}{char *name}
Return the module object corresponding to a module name.  The
\var{name} argument may be of the form \code{package.module}).  First
check the modules dictionary if there's one there, and if not, create
a new one and insert in in the modules dictionary.  Because the former
action is most common, this does not return a new reference, and you
do not own the returned reference.  Return \NULL{} with an
exception set on failure.
\end{cfuncdesc}

\begin{cfuncdesc}{PyObject *}{PyImport_ExecCodeModule}{char *name, PyObject *co}
Given a module name (possibly of the form \code{package.module}) and a
code object read from a Python bytecode file or obtained from the
built-in function \code{compile()}, load the module.  Return a new
reference to the module object, or \NULL{} with an exception set
if an error occurred (the module may still be created in this case).
(This function would reload the module if it was already imported.)
\end{cfuncdesc}

\begin{cfuncdesc}{long}{PyImport_GetMagicNumber}{}
Return the magic number for Python bytecode files (a.k.a. \code{.pyc}
and \code{.pyo} files).  The magic number should be present in the
first four bytes of the bytecode file, in little-endian byte order.
\end{cfuncdesc}

\begin{cfuncdesc}{PyObject *}{PyImport_GetModuleDict}{}
Return the dictionary used for the module administration
(a.k.a. \code{sys.modules}).  Note that this is a per-interpreter
variable.
\end{cfuncdesc}

\begin{cfuncdesc}{void}{_PyImport_Init}{}
Initialize the import mechanism.  For internal use only.
\end{cfuncdesc}

\begin{cfuncdesc}{void}{PyImport_Cleanup}{}
Empty the module table.  For internal use only.
\end{cfuncdesc}

\begin{cfuncdesc}{void}{_PyImport_Fini}{}
Finalize the import mechanism.  For internal use only.
\end{cfuncdesc}

\begin{cfuncdesc}{extern PyObject *}{_PyImport_FindExtension}{char *, char *}
For internal use only.
\end{cfuncdesc}

\begin{cfuncdesc}{extern PyObject *}{_PyImport_FixupExtension}{char *, char *}
For internal use only.
\end{cfuncdesc}

\begin{cfuncdesc}{int}{PyImport_ImportFrozenModule}{char *}
Load a frozen module.  Return \code{1} for success, \code{0} if the
module is not found, and \code{-1} with an exception set if the
initialization failed.  To access the imported module on a successful
load, use \code{PyImport_ImportModule())}.
(Note the misnomer -- this function would reload the module if it was
already imported.)
\end{cfuncdesc}

\begin{ctypedesc}{struct _frozen}
This is the structure type definition for frozen module descriptors,
as generated by the \code{freeze} utility (see \file{Tools/freeze/} in
the Python source distribution).  Its definition is:
\begin{verbatim}
struct _frozen {
    char *name;
    unsigned char *code;
    int size;
};
\end{verbatim}
\end{ctypedesc}

\begin{cvardesc}{struct _frozen *}{PyImport_FrozenModules}
This pointer is initialized to point to an array of \code{struct
_frozen} records, terminated by one whose members are all \NULL{}
or zero.  When a frozen module is imported, it is searched in this
table.  Third party code could play tricks with this to provide a
dynamically created collection of frozen modules.
\end{cvardesc}


\chapter{Debugging}

XXX Explain Py_DEBUG, Py_TRACE_REFS, Py_REF_DEBUG.


\chapter{The Very High Level Layer}

The functions in this chapter will let you execute Python source code
given in a file or a buffer, but they will not let you interact in a
more detailed way with the interpreter.

\begin{cfuncdesc}{int}{PyRun_AnyFile}{FILE *, char *}
\end{cfuncdesc}

\begin{cfuncdesc}{int}{PyRun_SimpleString}{char *}
\end{cfuncdesc}

\begin{cfuncdesc}{int}{PyRun_SimpleFile}{FILE *, char *}
\end{cfuncdesc}

\begin{cfuncdesc}{int}{PyRun_InteractiveOne}{FILE *, char *}
\end{cfuncdesc}

\begin{cfuncdesc}{int}{PyRun_InteractiveLoop}{FILE *, char *}
\end{cfuncdesc}

\begin{cfuncdesc}{struct _node *}{PyParser_SimpleParseString}{char *, int}
\end{cfuncdesc}

\begin{cfuncdesc}{struct _node *}{PyParser_SimpleParseFile}{FILE *, char *, int}
\end{cfuncdesc}

\begin{cfuncdesc}{PyObject *}{PyRun_String}{char *, int, PyObject *, PyObject *}
\end{cfuncdesc}

\begin{cfuncdesc}{PyObject *}{PyRun_File}{FILE *, char *, int, PyObject *, PyObject *}
\end{cfuncdesc}

\begin{cfuncdesc}{PyObject *}{Py_CompileString}{char *, char *, int}
\end{cfuncdesc}


\chapter{Abstract Objects Layer}

The functions in this chapter interact with Python objects regardless
of their type, or with wide classes of object types (e.g. all
numerical types, or all sequence types).  When used on object types
for which they do not apply, they will flag a Python exception.

\section{Object Protocol}

\begin{cfuncdesc}{int}{PyObject_Print}{PyObject *o, FILE *fp, int flags}
Print an object \code{o}, on file \code{fp}.  Returns -1 on error
The flags argument is used to enable certain printing
options. The only option currently supported is \code{Py_Print_RAW}. 
\end{cfuncdesc}

\begin{cfuncdesc}{int}{PyObject_HasAttrString}{PyObject *o, char *attr_name}
Returns 1 if o has the attribute attr_name, and 0 otherwise.
This is equivalent to the Python expression:
\code{hasattr(o,attr_name)}.
This function always succeeds.
\end{cfuncdesc}

\begin{cfuncdesc}{PyObject*}{PyObject_GetAttrString}{PyObject *o, char *attr_name}
Retrieve an attribute named attr_name from object o.
Returns the attribute value on success, or \NULL{} on failure.
This is the equivalent of the Python expression: \code{o.attr_name}.
\end{cfuncdesc}


\begin{cfuncdesc}{int}{PyObject_HasAttr}{PyObject *o, PyObject *attr_name}
Returns 1 if o has the attribute attr_name, and 0 otherwise.
This is equivalent to the Python expression:
\code{hasattr(o,attr_name)}. 
This function always succeeds.
\end{cfuncdesc}


\begin{cfuncdesc}{PyObject*}{PyObject_GetAttr}{PyObject *o, PyObject *attr_name}
Retrieve an attribute named attr_name from object o.
Returns the attribute value on success, or \NULL{} on failure.
This is the equivalent of the Python expression: o.attr_name.
\end{cfuncdesc}


\begin{cfuncdesc}{int}{PyObject_SetAttrString}{PyObject *o, char *attr_name, PyObject *v}
Set the value of the attribute named \code{attr_name}, for object \code{o},
to the value \code{v}. Returns -1 on failure.  This is
the equivalent of the Python statement: \code{o.attr_name=v}.
\end{cfuncdesc}


\begin{cfuncdesc}{int}{PyObject_SetAttr}{PyObject *o, PyObject *attr_name, PyObject *v}
Set the value of the attribute named \code{attr_name}, for
object \code{o},
to the value \code{v}. Returns -1 on failure.  This is
the equivalent of the Python statement: \code{o.attr_name=v}.
\end{cfuncdesc}


\begin{cfuncdesc}{int}{PyObject_DelAttrString}{PyObject *o, char *attr_name}
Delete attribute named \code{attr_name}, for object \code{o}. Returns -1 on
failure.  This is the equivalent of the Python
statement: \code{del o.attr_name}.
\end{cfuncdesc}


\begin{cfuncdesc}{int}{PyObject_DelAttr}{PyObject *o, PyObject *attr_name}
Delete attribute named \code{attr_name}, for object \code{o}. Returns -1 on
failure.  This is the equivalent of the Python
statement: \code{del o.attr_name}.
\end{cfuncdesc}


\begin{cfuncdesc}{int}{PyObject_Cmp}{PyObject *o1, PyObject *o2, int *result}
Compare the values of \code{o1} and \code{o2} using a routine provided by
\code{o1}, if one exists, otherwise with a routine provided by \code{o2}.
The result of the comparison is returned in \code{result}.  Returns
-1 on failure.  This is the equivalent of the Python
statement: \code{result=cmp(o1,o2)}.
\end{cfuncdesc}


\begin{cfuncdesc}{int}{PyObject_Compare}{PyObject *o1, PyObject *o2}
Compare the values of \code{o1} and \code{o2} using a routine provided by
\code{o1}, if one exists, otherwise with a routine provided by \code{o2}.
Returns the result of the comparison on success.  On error,
the value returned is undefined. This is equivalent to the
Python expression: \code{cmp(o1,o2)}.
\end{cfuncdesc}


\begin{cfuncdesc}{PyObject*}{PyObject_Repr}{PyObject *o}
Compute the string representation of object, \code{o}.  Returns the
string representation on success, \NULL{} on failure.  This is
the equivalent of the Python expression: \code{repr(o)}.
Called by the \code{repr()} built-in function and by reverse quotes.
\end{cfuncdesc}


\begin{cfuncdesc}{PyObject*}{PyObject_Str}{PyObject *o}
Compute the string representation of object, \code{o}.  Returns the
string representation on success, \NULL{} on failure.  This is
the equivalent of the Python expression: \code{str(o)}.
Called by the \code{str()} built-in function and by the \code{print}
statement.
\end{cfuncdesc}


\begin{cfuncdesc}{int}{PyCallable_Check}{PyObject *o}
Determine if the object \code{o}, is callable.  Return 1 if the
object is callable and 0 otherwise.
This function always succeeds.
\end{cfuncdesc}


\begin{cfuncdesc}{PyObject*}{PyObject_CallObject}{PyObject *callable_object, PyObject *args}
Call a callable Python object \code{callable_object}, with
arguments given by the tuple \code{args}.  If no arguments are
needed, then args may be \NULL{}.  Returns the result of the
call on success, or \NULL{} on failure.  This is the equivalent
of the Python expression: \code{apply(o, args)}.
\end{cfuncdesc}

\begin{cfuncdesc}{PyObject*}{PyObject_CallFunction}{PyObject *callable_object, char *format, ...}
Call a callable Python object \code{callable_object}, with a
variable number of C arguments. The C arguments are described
using a mkvalue-style format string. The format may be \NULL{},
indicating that no arguments are provided.  Returns the
result of the call on success, or \NULL{} on failure.  This is
the equivalent of the Python expression: \code{apply(o,args)}.
\end{cfuncdesc}


\begin{cfuncdesc}{PyObject*}{PyObject_CallMethod}{PyObject *o, char *m, char *format, ...}
Call the method named \code{m} of object \code{o} with a variable number of
C arguments.  The C arguments are described by a mkvalue
format string.  The format may be \NULL{}, indicating that no
arguments are provided. Returns the result of the call on
success, or \NULL{} on failure.  This is the equivalent of the
Python expression: \code{o.method(args)}.
Note that Special method names, such as "\code{__add__}",
"\code{__getitem__}", and so on are not supported. The specific
abstract-object routines for these must be used.
\end{cfuncdesc}


\begin{cfuncdesc}{int}{PyObject_Hash}{PyObject *o}
Compute and return the hash value of an object \code{o}.  On
failure, return -1.  This is the equivalent of the Python
expression: \code{hash(o)}.
\end{cfuncdesc}


\begin{cfuncdesc}{int}{PyObject_IsTrue}{PyObject *o}
Returns 1 if the object \code{o} is considered to be true, and
0 otherwise. This is equivalent to the Python expression:
\code{not not o}.
This function always succeeds.
\end{cfuncdesc}


\begin{cfuncdesc}{PyObject*}{PyObject_Type}{PyObject *o}
On success, returns a type object corresponding to the object
type of object \code{o}. On failure, returns \NULL{}.  This is
equivalent to the Python expression: \code{type(o)}.
\end{cfuncdesc}

\begin{cfuncdesc}{int}{PyObject_Length}{PyObject *o}
Return the length of object \code{o}.  If the object \code{o} provides
both sequence and mapping protocols, the sequence length is
returned. On error, -1 is returned.  This is the equivalent
to the Python expression: \code{len(o)}.
\end{cfuncdesc}


\begin{cfuncdesc}{PyObject*}{PyObject_GetItem}{PyObject *o, PyObject *key}
Return element of \code{o} corresponding to the object \code{key} or \NULL{}
on failure. This is the equivalent of the Python expression:
\code{o[key]}.
\end{cfuncdesc}


\begin{cfuncdesc}{int}{PyObject_SetItem}{PyObject *o, PyObject *key, PyObject *v}
Map the object \code{key} to the value \code{v}.
Returns -1 on failure.  This is the equivalent
of the Python statement: \code{o[key]=v}.
\end{cfuncdesc}


\begin{cfuncdesc}{int}{PyObject_DelItem}{PyObject *o, PyObject *key, PyObject *v}
Delete the mapping for \code{key} from \code{*o}.  Returns -1
on failure.
This is the equivalent of the Python statement: \code{del o[key]}.
\end{cfuncdesc}


\section{Number Protocol}

\begin{cfuncdesc}{int}{PyNumber_Check}{PyObject *o}
Returns 1 if the object \code{o} provides numeric protocols, and
false otherwise. 
This function always succeeds.
\end{cfuncdesc}


\begin{cfuncdesc}{PyObject*}{PyNumber_Add}{PyObject *o1, PyObject *o2}
Returns the result of adding \code{o1} and \code{o2}, or null on failure.
This is the equivalent of the Python expression: \code{o1+o2}.
\end{cfuncdesc}


\begin{cfuncdesc}{PyObject*}{PyNumber_Subtract}{PyObject *o1, PyObject *o2}
Returns the result of subtracting \code{o2} from \code{o1}, or null on
failure.  This is the equivalent of the Python expression:
\code{o1-o2}.
\end{cfuncdesc}


\begin{cfuncdesc}{PyObject*}{PyNumber_Multiply}{PyObject *o1, PyObject *o2}
Returns the result of multiplying \code{o1} and \code{o2}, or null on
failure.  This is the equivalent of the Python expression:
\code{o1*o2}.
\end{cfuncdesc}


\begin{cfuncdesc}{PyObject*}{PyNumber_Divide}{PyObject *o1, PyObject *o2}
Returns the result of dividing \code{o1} by \code{o2}, or null on failure.
This is the equivalent of the Python expression: \code{o1/o2}.
\end{cfuncdesc}


\begin{cfuncdesc}{PyObject*}{PyNumber_Remainder}{PyObject *o1, PyObject *o2}
Returns the remainder of dividing \code{o1} by \code{o2}, or null on
failure.  This is the equivalent of the Python expression:
\code{o1\%o2}.
\end{cfuncdesc}


\begin{cfuncdesc}{PyObject*}{PyNumber_Divmod}{PyObject *o1, PyObject *o2}
See the built-in function divmod.  Returns \NULL{} on failure.
This is the equivalent of the Python expression:
\code{divmod(o1,o2)}.
\end{cfuncdesc}


\begin{cfuncdesc}{PyObject*}{PyNumber_Power}{PyObject *o1, PyObject *o2, PyObject *o3}
See the built-in function pow.  Returns \NULL{} on failure.
This is the equivalent of the Python expression:
\code{pow(o1,o2,o3)}, where \code{o3} is optional.
\end{cfuncdesc}


\begin{cfuncdesc}{PyObject*}{PyNumber_Negative}{PyObject *o}
Returns the negation of \code{o} on success, or null on failure.
This is the equivalent of the Python expression: \code{-o}.
\end{cfuncdesc}


\begin{cfuncdesc}{PyObject*}{PyNumber_Positive}{PyObject *o}
Returns \code{o} on success, or \NULL{} on failure.
This is the equivalent of the Python expression: \code{+o}.
\end{cfuncdesc}


\begin{cfuncdesc}{PyObject*}{PyNumber_Absolute}{PyObject *o}
Returns the absolute value of \code{o}, or null on failure.  This is
the equivalent of the Python expression: \code{abs(o)}.
\end{cfuncdesc}


\begin{cfuncdesc}{PyObject*}{PyNumber_Invert}{PyObject *o}
Returns the bitwise negation of \code{o} on success, or \NULL{} on
failure.  This is the equivalent of the Python expression:
\code{\~o}.
\end{cfuncdesc}


\begin{cfuncdesc}{PyObject*}{PyNumber_Lshift}{PyObject *o1, PyObject *o2}
Returns the result of left shifting \code{o1} by \code{o2} on success, or
\NULL{} on failure.  This is the equivalent of the Python
expression: \code{o1 << o2}.
\end{cfuncdesc}


\begin{cfuncdesc}{PyObject*}{PyNumber_Rshift}{PyObject *o1, PyObject *o2}
Returns the result of right shifting \code{o1} by \code{o2} on success, or
\NULL{} on failure.  This is the equivalent of the Python
expression: \code{o1 >> o2}.
\end{cfuncdesc}


\begin{cfuncdesc}{PyObject*}{PyNumber_And}{PyObject *o1, PyObject *o2}
Returns the result of "anding" \code{o2} and \code{o2} on success and \NULL{}
on failure. This is the equivalent of the Python
expression: \code{o1 and o2}.
\end{cfuncdesc}


\begin{cfuncdesc}{PyObject*}{PyNumber_Xor}{PyObject *o1, PyObject *o2}
Returns the bitwise exclusive or of \code{o1} by \code{o2} on success, or
\NULL{} on failure.  This is the equivalent of the Python
expression: \code{o1\^{ }o2}.
\end{cfuncdesc}

\begin{cfuncdesc}{PyObject*}{PyNumber_Or}{PyObject *o1, PyObject *o2}
Returns the result of \code{o1} and \code{o2} on success, or \NULL{} on
failure.  This is the equivalent of the Python expression: 
\code{o1 or o2}.
\end{cfuncdesc}


\begin{cfuncdesc}{PyObject*}{PyNumber_Coerce}{PyObject *o1, PyObject *o2}
This function takes the addresses of two variables of type
\code{PyObject*}.

If the objects pointed to by \code{*p1} and \code{*p2} have the same type,
increment their reference count and return 0 (success).
If the objects can be converted to a common numeric type,
replace \code{*p1} and \code{*p2} by their converted value (with 'new'
reference counts), and return 0.
If no conversion is possible, or if some other error occurs,
return -1 (failure) and don't increment the reference counts.
The call \code{PyNumber_Coerce(\&o1, \&o2)} is equivalent to the Python
statement \code{o1, o2 = coerce(o1, o2)}.
\end{cfuncdesc}


\begin{cfuncdesc}{PyObject*}{PyNumber_Int}{PyObject *o}
Returns the \code{o} converted to an integer object on success, or
\NULL{} on failure.  This is the equivalent of the Python
expression: \code{int(o)}.
\end{cfuncdesc}


\begin{cfuncdesc}{PyObject*}{PyNumber_Long}{PyObject *o}
Returns the \code{o} converted to a long integer object on success,
or \NULL{} on failure.  This is the equivalent of the Python
expression: \code{long(o)}.
\end{cfuncdesc}


\begin{cfuncdesc}{PyObject*}{PyNumber_Float}{PyObject *o}
Returns the \code{o} converted to a float object on success, or \NULL{}
on failure.  This is the equivalent of the Python expression:
\code{float(o)}.
\end{cfuncdesc}


\section{Sequence protocol}

\begin{cfuncdesc}{int}{PySequence_Check}{PyObject *o}
Return 1 if the object provides sequence protocol, and 0
otherwise.  
This function always succeeds.
\end{cfuncdesc}


\begin{cfuncdesc}{PyObject*}{PySequence_Concat}{PyObject *o1, PyObject *o2}
Return the concatenation of \code{o1} and \code{o2} on success, and \NULL{} on
failure.   This is the equivalent of the Python
expression: \code{o1+o2}.
\end{cfuncdesc}


\begin{cfuncdesc}{PyObject*}{PySequence_Repeat}{PyObject *o, int count}
Return the result of repeating sequence object \code{o} \code{count} times,
or \NULL{} on failure.  This is the equivalent of the Python
expression: \code{o*count}.
\end{cfuncdesc}


\begin{cfuncdesc}{PyObject*}{PySequence_GetItem}{PyObject *o, int i}
Return the ith element of \code{o}, or \NULL{} on failure. This is the
equivalent of the Python expression: \code{o[i]}.
\end{cfuncdesc}


\begin{cfuncdesc}{PyObject*}{PySequence_GetSlice}{PyObject *o, int i1, int i2}
Return the slice of sequence object \code{o} between \code{i1} and \code{i2}, or
\NULL{} on failure. This is the equivalent of the Python
expression, \code{o[i1:i2]}.
\end{cfuncdesc}


\begin{cfuncdesc}{int}{PySequence_SetItem}{PyObject *o, int i, PyObject *v}
Assign object \code{v} to the \code{i}th element of \code{o}.
Returns -1 on failure.  This is the equivalent of the Python
statement, \code{o[i]=v}.
\end{cfuncdesc}

\begin{cfuncdesc}{int}{PySequence_DelItem}{PyObject *o, int i}
Delete the \code{i}th element of object \code{v}.  Returns
-1 on failure.  This is the equivalent of the Python
statement: \code{del o[i]}.
\end{cfuncdesc}

\begin{cfuncdesc}{int}{PySequence_SetSlice}{PyObject *o, int i1, int i2, PyObject *v}
Assign the sequence object \code{v} to the slice in sequence
object \code{o} from \code{i1} to \code{i2}.  This is the equivalent of the Python
statement, \code{o[i1:i2]=v}.
\end{cfuncdesc}

\begin{cfuncdesc}{int}{PySequence_DelSlice}{PyObject *o, int i1, int i2}
Delete the slice in sequence object, \code{o}, from \code{i1} to \code{i2}.
Returns -1 on failure. This is the equivalent of the Python
statement: \code{del o[i1:i2]}.
\end{cfuncdesc}

\begin{cfuncdesc}{PyObject*}{PySequence_Tuple}{PyObject *o}
Returns the \code{o} as a tuple on success, and \NULL{} on failure.
This is equivalent to the Python expression: \code{tuple(o)}.
\end{cfuncdesc}

\begin{cfuncdesc}{int}{PySequence_Count}{PyObject *o, PyObject *value}
Return the number of occurrences of \code{value} on \code{o}, that is,
return the number of keys for which \code{o[key]==value}.  On
failure, return -1.  This is equivalent to the Python
expression: \code{o.count(value)}.
\end{cfuncdesc}

\begin{cfuncdesc}{int}{PySequence_In}{PyObject *o, PyObject *value}
Determine if \code{o} contains \code{value}.  If an item in \code{o} is equal to
\code{value}, return 1, otherwise return 0.  On error, return -1.  This
is equivalent to the Python expression: \code{value in o}.
\end{cfuncdesc}

\begin{cfuncdesc}{int}{PySequence_Index}{PyObject *o, PyObject *value}
Return the first index for which \code{o[i]==value}.  On error,
return -1.    This is equivalent to the Python
expression: \code{o.index(value)}.
\end{cfuncdesc}

\section{Mapping protocol}

\begin{cfuncdesc}{int}{PyMapping_Check}{PyObject *o}
Return 1 if the object provides mapping protocol, and 0
otherwise.  
This function always succeeds.
\end{cfuncdesc}


\begin{cfuncdesc}{int}{PyMapping_Length}{PyObject *o}
Returns the number of keys in object \code{o} on success, and -1 on
failure.  For objects that do not provide sequence protocol,
this is equivalent to the Python expression: \code{len(o)}.
\end{cfuncdesc}


\begin{cfuncdesc}{int}{PyMapping_DelItemString}{PyObject *o, char *key}
Remove the mapping for object \code{key} from the object \code{o}.
Return -1 on failure.  This is equivalent to
the Python statement: \code{del o[key]}.
\end{cfuncdesc}


\begin{cfuncdesc}{int}{PyMapping_DelItem}{PyObject *o, PyObject *key}
Remove the mapping for object \code{key} from the object \code{o}.
Return -1 on failure.  This is equivalent to
the Python statement: \code{del o[key]}.
\end{cfuncdesc}


\begin{cfuncdesc}{int}{PyMapping_HasKeyString}{PyObject *o, char *key}
On success, return 1 if the mapping object has the key \code{key}
and 0 otherwise.  This is equivalent to the Python expression:
\code{o.has_key(key)}. 
This function always succeeds.
\end{cfuncdesc}


\begin{cfuncdesc}{int}{PyMapping_HasKey}{PyObject *o, PyObject *key}
Return 1 if the mapping object has the key \code{key}
and 0 otherwise.  This is equivalent to the Python expression:
\code{o.has_key(key)}. 
This function always succeeds.
\end{cfuncdesc}


\begin{cfuncdesc}{PyObject*}{PyMapping_Keys}{PyObject *o}
On success, return a list of the keys in object \code{o}.  On
failure, return \NULL{}. This is equivalent to the Python
expression: \code{o.keys()}.
\end{cfuncdesc}


\begin{cfuncdesc}{PyObject*}{PyMapping_Values}{PyObject *o}
On success, return a list of the values in object \code{o}.  On
failure, return \NULL{}. This is equivalent to the Python
expression: \code{o.values()}.
\end{cfuncdesc}


\begin{cfuncdesc}{PyObject*}{PyMapping_Items}{PyObject *o}
On success, return a list of the items in object \code{o}, where
each item is a tuple containing a key-value pair.  On
failure, return \NULL{}. This is equivalent to the Python
expression: \code{o.items()}.
\end{cfuncdesc}

\begin{cfuncdesc}{int}{PyMapping_Clear}{PyObject *o}
Make object \code{o} empty.  Returns 1 on success and 0 on failure.
This is equivalent to the Python statement:
\code{for key in o.keys(): del o[key]}
\end{cfuncdesc}


\begin{cfuncdesc}{PyObject*}{PyMapping_GetItemString}{PyObject *o, char *key}
Return element of \code{o} corresponding to the object \code{key} or \NULL{}
on failure. This is the equivalent of the Python expression:
\code{o[key]}.
\end{cfuncdesc}

\begin{cfuncdesc}{PyObject*}{PyMapping_SetItemString}{PyObject *o, char *key, PyObject *v}
Map the object \code{key} to the value \code{v} in object \code{o}.  Returns 
-1 on failure.  This is the equivalent of the Python
statement: \code{o[key]=v}.
\end{cfuncdesc}


\section{Constructors}

\begin{cfuncdesc}{PyObject*}{PyFile_FromString}{char *file_name, char *mode}
On success, returns a new file object that is opened on the
file given by \code{file_name}, with a file mode given by \code{mode},
where \code{mode} has the same semantics as the standard C routine,
fopen.  On failure, return -1.
\end{cfuncdesc}

\begin{cfuncdesc}{PyObject*}{PyFile_FromFile}{FILE *fp, char *file_name, char *mode, int close_on_del}
Return a new file object for an already opened standard C
file pointer, \code{fp}.  A file name, \code{file_name}, and open mode,
\code{mode}, must be provided as well as a flag, \code{close_on_del}, that
indicates whether the file is to be closed when the file
object is destroyed.  On failure, return -1.
\end{cfuncdesc}

\begin{cfuncdesc}{PyObject*}{PyFloat_FromDouble}{double v}
Returns a new float object with the value \code{v} on success, and
\NULL{} on failure.
\end{cfuncdesc}

\begin{cfuncdesc}{PyObject*}{PyInt_FromLong}{long v}
Returns a new int object with the value \code{v} on success, and
\NULL{} on failure.
\end{cfuncdesc}

\begin{cfuncdesc}{PyObject*}{PyList_New}{int l}
Returns a new list of length \code{l} on success, and \NULL{} on
failure.
\end{cfuncdesc}

\begin{cfuncdesc}{PyObject*}{PyLong_FromLong}{long v}
Returns a new long object with the value \code{v} on success, and
\NULL{} on failure.
\end{cfuncdesc}

\begin{cfuncdesc}{PyObject*}{PyLong_FromDouble}{double v}
Returns a new long object with the value \code{v} on success, and
\NULL{} on failure.
\end{cfuncdesc}

\begin{cfuncdesc}{PyObject*}{PyDict_New}{}
Returns a new empty dictionary on success, and \NULL{} on
failure.
\end{cfuncdesc}

\begin{cfuncdesc}{PyObject*}{PyString_FromString}{char *v}
Returns a new string object with the value \code{v} on success, and
\NULL{} on failure.
\end{cfuncdesc}

\begin{cfuncdesc}{PyObject*}{PyString_FromStringAndSize}{char *v, int l}
Returns a new string object with the value \code{v} and length \code{l}
on success, and \NULL{} on failure.
\end{cfuncdesc}

\begin{cfuncdesc}{PyObject*}{PyTuple_New}{int l}
Returns a new tuple of length \code{l} on success, and \NULL{} on
failure.
\end{cfuncdesc}


\chapter{Concrete Objects Layer}

The functions in this chapter are specific to certain Python object
types.  Passing them an object of the wrong type is not a good idea;
if you receive an object from a Python program and you are not sure
that it has the right type, you must perform a type check first;
e.g. to check that an object is a dictionary, use
\code{PyDict_Check()}.


\chapter{Defining New Object Types}

\begin{cfuncdesc}{PyObject *}{_PyObject_New}{PyTypeObject *type}
\end{cfuncdesc}

\begin{cfuncdesc}{PyObject *}{_PyObject_NewVar}{PyTypeObject *type, int size}
\end{cfuncdesc}

\begin{cfuncdesc}{TYPE}{_PyObject_NEW}{TYPE, PyTypeObject *}
\end{cfuncdesc}

\begin{cfuncdesc}{TYPE}{_PyObject_NEW_VAR}{TYPE, PyTypeObject *, int size}
\end{cfuncdesc}

\chapter{Initialization, Finalization, and Threads}

\begin{cfuncdesc}{void}{Py_Initialize}{}
Initialize the Python interpreter.  In an application embedding 
Python, this should be called before using any other Python/C API 
functions; with the exception of \code{Py_SetProgramName()}, 
\code{PyEval_InitThreads()}, \code{PyEval_ReleaseLock()}, and 
\code{PyEval_AcquireLock()}.  This initializes the table of loaded 
modules (\code{sys.modules}), and creates the fundamental modules 
\code{__builtin__}, \code{__main__} and \code{sys}.  It also 
initializes the module search path (\code{sys.path}).  It does not set 
\code{sys.argv}; use \code{PySys_SetArgv()} for that.  This is a no-op
when called for a second time (without calling \code{Py_Finalize()}
first).  There is no return value; it is a fatal error if the
initialization fails.
\end{cfuncdesc}

\begin{cfuncdesc}{int}{Py_IsInitialized}{}
\strong{(NEW in 1.5a4!)}
Return true (nonzero) when the Python interpreter has been
initialized, false (zero) if not.  After \code{Py_Finalize()} is
called, this returns false until \code{Py_Initialize()} is called
again.
\end{cfuncdesc}

\begin{cfuncdesc}{void}{Py_Finalize}{}
\strong{(NEW in 1.5a3!)}
Undo all initializations made by \code{Py_Initialize()} and subsequent 
use of Python/C API functions, and destroy all sub-interpreters (see 
\code{Py_NewInterpreter()} below) that were created and not yet 
destroyed since the last call to \code{Py_Initialize()}.  Ideally,
this frees all memory allocated by the Python interpreter.  This is a
no-op when called for a second time (without calling
\code{Py_Initialize()} again first).  There is no return value; errors
during finalization are ignored.

This function is provided for a number of reasons.  An embedding 
application might want to restart Python without having to restart the 
application itself.  An application that has loaded the Python 
interpreter from a dynamically loadable library (or DLL) might want to 
free all memory allocated by Python before unloading the DLL. During a 
hunt for memory leaks in an application a developer might want to free 
all memory allocated by Python before exiting from the application.

\emph{Bugs and caveats:} The destruction of modules and objects in 
modules is done in random order; this may cause destructors 
(\code{__del__} methods) to fail when they depend on other objects 
(even functions) or modules.  Dynamically loaded extension modules 
loaded by Python are not unloaded.  Small amounts of memory allocated 
by the Python interpreter may not be freed (if you find a leak, please 
report it).  Memory tied up in circular references between objects is 
not freed.  Some memory allocated by extension modules may not be 
freed.  Some extension may not work properly if their initialization 
routine is called more than once; this can happen if an applcation 
calls \code{Py_Initialize()} and \code{Py_Finalize()} more than once.
\end{cfuncdesc}

\begin{cfuncdesc}{PyThreadState *}{Py_NewInterpreter}{}
\strong{(NEW in 1.5a3!)}
Create a new sub-interpreter.  This is an (almost) totally separate 
environment for the execution of Python code.  In particular, the new 
interpreter has separate, independent versions of all imported 
modules, including the fundamental modules \code{__builtin__}, 
\code{__main__} and \code{sys}.  The table of loaded modules 
(\code{sys.modules}) and the module search path (\code{sys.path}) are 
also separate.  The new environment has no \code{sys.argv} variable.  
It has new standard I/O stream file objects \code{sys.stdin}, 
\code{sys.stdout} and \code{sys.stderr} (however these refer to the 
same underlying \code{FILE} structures in the C library).

The return value points to the first thread state created in the new 
sub-interpreter.  This thread state is made the current thread state.  
Note that no actual thread is created; see the discussion of thread 
states below.  If creation of the new interpreter is unsuccessful, 
\NULL{} is returned; no exception is set since the exception state 
is stored in the current thread state and there may not be a current 
thread state.  (Like all other Python/C API functions, the global 
interpreter lock must be held before calling this function and is 
still held when it returns; however, unlike most other Python/C API 
functions, there needn't be a current thread state on entry.)

Extension modules are shared between (sub-)interpreters as follows: 
the first time a particular extension is imported, it is initialized 
normally, and a (shallow) copy of its module's dictionary is 
squirreled away.  When the same extension is imported by another 
(sub-)interpreter, a new module is initialized and filled with the 
contents of this copy; the extension's \code{init} function is not 
called.  Note that this is different from what happens when as 
extension is imported after the interpreter has been completely 
re-initialized by calling \code{Py_Finalize()} and 
\code{Py_Initialize()}; in that case, the extension's \code{init} 
function \emph{is} called again.

\emph{Bugs and caveats:} Because sub-interpreters (and the main 
interpreter) are part of the same process, the insulation between them 
isn't perfect -- for example, using low-level file operations like 
\code{os.close()} they can (accidentally or maliciously) affect each 
other's open files.  Because of the way extensions are shared between 
(sub-)interpreters, some extensions may not work properly; this is 
especially likely when the extension makes use of (static) global 
variables, or when the extension manipulates its module's dictionary 
after its initialization.  It is possible to insert objects created in 
one sub-interpreter into a namespace of another sub-interpreter; this 
should be done with great care to avoid sharing user-defined 
functions, methods, instances or classes between sub-interpreters, 
since import operations executed by such objects may affect the 
wrong (sub-)interpreter's dictionary of loaded modules.  (XXX This is 
a hard-to-fix bug that will be addressed in a future release.)
\end{cfuncdesc}

\begin{cfuncdesc}{void}{Py_EndInterpreter}{PyThreadState *tstate}
\strong{(NEW in 1.5a3!)}
Destroy the (sub-)interpreter represented by the given thread state.  
The given thread state must be the current thread state.  See the 
discussion of thread states below.  When the call returns, the current 
thread state is \NULL{}.  All thread states associated with this 
interpreted are destroyed.  (The global interpreter lock must be held 
before calling this function and is still held when it returns.)  
\code{Py_Finalize()} will destroy all sub-interpreters that haven't 
been explicitly destroyed at that point.
\end{cfuncdesc}

\begin{cfuncdesc}{void}{Py_SetProgramName}{char *name}
\strong{(NEW in 1.5a3!)}
This function should be called before \code{Py_Initialize()} is called 
for the first time, if it is called at all.  It tells the interpreter 
the value of the \code{argv[0]} argument to the \code{main()} function 
of the program.  This is used by \code{Py_GetPath()} and some other 
functions below to find the Python run-time libraries relative to the 
interpreter executable.  The default value is \code{"python"}.  The 
argument should point to a zero-terminated character string in static 
storage whose contents will not change for the duration of the 
program's execution.  No code in the Python interpreter will change 
the contents of this storage.
\end{cfuncdesc}

\begin{cfuncdesc}{char *}{Py_GetProgramName}{}
Return the program name set with \code{Py_SetProgramName()}, or the 
default.  The returned string points into static storage; the caller 
should not modify its value.
\end{cfuncdesc}

\begin{cfuncdesc}{char *}{Py_GetPrefix}{}
Return the ``prefix'' for installed platform-independent files.  This 
is derived through a number of complicated rules from the program name 
set with \code{Py_SetProgramName()} and some environment variables; 
for example, if the program name is \code{"/usr/local/bin/python"}, 
the prefix is \code{"/usr/local"}.  The returned string points into 
static storage; the caller should not modify its value.  This 
corresponds to the \code{prefix} variable in the top-level 
\code{Makefile} and the \code{--prefix} argument to the 
\code{configure} script at build time.  The value is available to 
Python code as \code{sys.prefix}.  It is only useful on Unix.  See 
also the next function.
\end{cfuncdesc}

\begin{cfuncdesc}{char *}{Py_GetExecPrefix}{}
Return the ``exec-prefix'' for installed platform-\emph{de}pendent 
files.  This is derived through a number of complicated rules from the 
program name set with \code{Py_SetProgramName()} and some environment 
variables; for example, if the program name is 
\code{"/usr/local/bin/python"}, the exec-prefix is 
\code{"/usr/local"}.  The returned string points into static storage; 
the caller should not modify its value.  This corresponds to the 
\code{exec_prefix} variable in the top-level \code{Makefile} and the 
\code{--exec_prefix} argument to the \code{configure} script at build 
time.  The value is available to Python code as 
\code{sys.exec_prefix}.  It is only useful on Unix.

Background: The exec-prefix differs from the prefix when platform 
dependent files (such as executables and shared libraries) are 
installed in a different directory tree.  In a typical installation, 
platform dependent files may be installed in the 
\code{"/usr/local/plat"} subtree while platform independent may be 
installed in \code{"/usr/local"}.

Generally speaking, a platform is a combination of hardware and 
software families, e.g.  Sparc machines running the Solaris 2.x 
operating system are considered the same platform, but Intel machines 
running Solaris 2.x are another platform, and Intel machines running 
Linux are yet another platform.  Different major revisions of the same 
operating system generally also form different platforms.  Non-Unix 
operating systems are a different story; the installation strategies 
on those systems are so different that the prefix and exec-prefix are 
meaningless, and set to the empty string.  Note that compiled Python 
bytecode files are platform independent (but not independent from the 
Python version by which they were compiled!).

System administrators will know how to configure the \code{mount} or 
\code{automount} programs to share \code{"/usr/local"} between platforms 
while having \code{"/usr/local/plat"} be a different filesystem for each 
platform.
\end{cfuncdesc}

\begin{cfuncdesc}{char *}{Py_GetProgramFullPath}{}
\strong{(NEW in 1.5a3!)}
Return the full program name of the Python executable; this is 
computed as a side-effect of deriving the default module search path 
from the program name (set by \code{Py_SetProgramName()} above).  The 
returned string points into static storage; the caller should not 
modify its value.  The value is available to Python code as 
\code{sys.executable}.
\end{cfuncdesc}

\begin{cfuncdesc}{char *}{Py_GetPath}{}
Return the default module search path; this is computed from the 
program name (set by \code{Py_SetProgramName()} above) and some 
environment variables.  The returned string consists of a series of 
directory names separated by a platform dependent delimiter character.  
The delimiter character is \code{':'} on Unix, \code{';'} on 
DOS/Windows, and \code{'\\n'} (the ASCII newline character) on 
Macintosh.  The returned string points into static storage; the caller 
should not modify its value.  The value is available to Python code 
as the list \code{sys.path}, which may be modified to change the 
future search path for loaded modules.

% XXX should give the exact rules
\end{cfuncdesc}

\begin{cfuncdesc}{const char *}{Py_GetVersion}{}
Return the version of this Python interpreter.  This is a string that 
looks something like

\begin{verbatim}
"1.5a3 (#67, Aug 1 1997, 22:34:28) [GCC 2.7.2.2]"
\end{verbatim}

The first word (up to the first space character) is the current Python 
version; the first three characters are the major and minor version 
separated by a period.  The returned string points into static storage; 
the caller should not modify its value.  The value is available to 
Python code as the list \code{sys.version}.
\end{cfuncdesc}

\begin{cfuncdesc}{const char *}{Py_GetPlatform}{}
Return the platform identifier for the current platform.  On Unix, 
this is formed from the ``official'' name of the operating system, 
converted to lower case, followed by the major revision number; e.g., 
for Solaris 2.x, which is also known as SunOS 5.x, the value is 
\code{"sunos5"}.  On Macintosh, it is \code{"mac"}.  On Windows, it 
is \code{"win"}.  The returned string points into static storage; 
the caller should not modify its value.  The value is available to 
Python code as \code{sys.platform}.
\end{cfuncdesc}

\begin{cfuncdesc}{const char *}{Py_GetCopyright}{}
Return the official copyright string for the current Python version, 
for example

\code{"Copyright 1991-1995 Stichting Mathematisch Centrum, Amsterdam"}

The returned string points into static storage; the caller should not 
modify its value.  The value is available to Python code as the list 
\code{sys.copyright}.
\end{cfuncdesc}

\begin{cfuncdesc}{const char *}{Py_GetCompiler}{}
Return an indication of the compiler used to build the current Python 
version, in square brackets, for example

\code{"[GCC 2.7.2.2]"}

The returned string points into static storage; the caller should not 
modify its value.  The value is available to Python code as part of 
the variable \code{sys.version}.
\end{cfuncdesc}

\begin{cfuncdesc}{const char *}{Py_GetBuildInfo}{}
Return information about the sequence number and build date and time 
of the current Python interpreter instance, for example

\begin{verbatim}
"#67, Aug  1 1997, 22:34:28"
\end{verbatim}

The returned string points into static storage; the caller should not 
modify its value.  The value is available to Python code as part of 
the variable \code{sys.version}.
\end{cfuncdesc}

\begin{cfuncdesc}{int}{PySys_SetArgv}{int argc, char **argv}
% XXX
\end{cfuncdesc}

% XXX Other PySys thingies (doesn't really belong in this chapter)

\section{Thread State and the Global Interpreter Lock}

The Python interpreter is not fully thread safe.  In order to support
multi-threaded Python programs, there's a global lock that must be
held by the current thread before it can safely access Python objects.
Without the lock, even the simplest operations could cause problems in
a multi-threaded proram: for example, when two threads simultaneously
increment the reference count of the same object, the reference count
could end up being incremented only once instead of twice.

Therefore, the rule exists that only the thread that has acquired the
global interpreter lock may operate on Python objects or call Python/C
API functions.  In order to support multi-threaded Python programs,
the interpreter regularly release and reacquires the lock -- by
default, every ten bytecode instructions (this can be changed with
\code{sys.setcheckinterval()}).  The lock is also released and
reacquired around potentially blocking I/O operations like reading or
writing a file, so that other threads can run while the thread that
requests the I/O is waiting for the I/O operation to complete.

The Python interpreter needs to keep some bookkeeping information
separate per thread -- for this it uses a data structure called
PyThreadState.  This is new in Python 1.5; in earlier versions, such
state was stored in global variables, and switching threads could
cause problems.  In particular, exception handling is now thread safe,
when the application uses \code{sys.exc_info()} to access the exception
last raised in the current thread.

There's one global variable left, however: the pointer to the current
PyThreadState structure.  While most thread packages have a way to
store ``per-thread global data'', Python's internal platform
independent thread abstraction doesn't support this (yet).  Therefore,
the current thread state must be manipulated explicitly.

This is easy enough in most cases.  Most code manipulating the global
interpreter lock has the following simple structure:

\begin{verbatim}
Save the thread state in a local variable.
Release the interpreter lock.
...Do some blocking I/O operation...
Reacquire the interpreter lock.
Restore the thread state from the local variable.
\end{verbatim}

This is so common that a pair of macros exists to simplify it:

\begin{verbatim}
Py_BEGIN_ALLOW_THREADS
...Do some blocking I/O operation...
Py_END_ALLOW_THREADS
\end{verbatim}

The BEGIN macro opens a new block and declares a hidden local
variable; the END macro closes the block.  Another advantage of using
these two macros is that when Python is compiled without thread
support, they are defined empty, thus saving the thread state and lock
manipulations.

When thread support is enabled, the block above expands to the
following code:

\begin{verbatim}
{
    PyThreadState *_save;
    _save = PyEval_SaveThread();
    ...Do some blocking I/O operation...
    PyEval_RestoreThread(_save);
}
\end{verbatim}

Using even lower level primitives, we can get roughly the same effect
as follows:

\begin{verbatim}
{
    PyThreadState *_save;
    _save = PyThreadState_Swap(NULL);
    PyEval_ReleaseLock();
    ...Do some blocking I/O operation...
    PyEval_AcquireLock();
    PyThreadState_Swap(_save);
}
\end{verbatim}

There are some subtle differences; in particular,
\code{PyEval_RestoreThread()} saves and restores the value of the
global variable \code{errno}, since the lock manipulation does not
guarantee that \code{errno} is left alone.  Also, when thread support
is disabled, \code{PyEval_SaveThread()} and
\code{PyEval_RestoreThread()} don't manipulate the lock; in this case,
\code{PyEval_ReleaseLock()} and \code{PyEval_AcquireLock()} are not
available.  (This is done so that dynamically loaded extensions
compiled with thread support enabled can be loaded by an interpreter
that was compiled with disabled thread support.)

The global interpreter lock is used to protect the pointer to the
current thread state.  When releasing the lock and saving the thread
state, the current thread state pointer must be retrieved before the
lock is released (since another thread could immediately acquire the
lock and store its own thread state in the global variable).
Reversely, when acquiring the lock and restoring the thread state, the
lock must be acquired before storing the thread state pointer.

Why am I going on with so much detail about this?  Because when
threads are created from C, they don't have the global interpreter
lock, nor is there a thread state data structure for them.  Such
threads must bootstrap themselves into existence, by first creating a
thread state data structure, then acquiring the lock, and finally
storing their thread state pointer, before they can start using the
Python/C API.  When they are done, they should reset the thread state
pointer, release the lock, and finally free their thread state data
structure.

When creating a thread data structure, you need to provide an
interpreter state data structure.  The interpreter state data
structure hold global data that is shared by all threads in an
interpreter, for example the module administration
(\code{sys.modules}).  Depending on your needs, you can either create
a new interpreter state data structure, or share the interpreter state
data structure used by the Python main thread (to access the latter,
you must obtain the thread state and access its \code{interp} member;
this must be done by a thread that is created by Python or by the main
thread after Python is initialized).

XXX More?

\begin{ctypedesc}{PyInterpreterState}
\strong{(NEW in 1.5a3!)}
This data structure represents the state shared by a number of
cooperating threads.  Threads belonging to the same interpreter
share their module administration and a few other internal items.
There are no public members in this structure.

Threads belonging to different interpreters initially share nothing,
except process state like available memory, open file descriptors and
such.  The global interpreter lock is also shared by all threads,
regardless of to which interpreter they belong.
\end{ctypedesc}

\begin{ctypedesc}{PyThreadState}
\strong{(NEW in 1.5a3!)}
This data structure represents the state of a single thread.  The only
public data member is \code{PyInterpreterState *interp}, which points
to this thread's interpreter state.
\end{ctypedesc}

\begin{cfuncdesc}{void}{PyEval_InitThreads}{}
Initialize and acquire the global interpreter lock.  It should be
called in the main thread before creating a second thread or engaging
in any other thread operations such as \code{PyEval_ReleaseLock()} or
\code{PyEval_ReleaseThread(tstate)}.  It is not needed before
calling \code{PyEval_SaveThread()} or \code{PyEval_RestoreThread()}.

This is a no-op when called for a second time.  It is safe to call
this function before calling \code{Py_Initialize()}.

When only the main thread exists, no lock operations are needed.  This
is a common situation (most Python programs do not use threads), and
the lock operations slow the interpreter down a bit.  Therefore, the
lock is not created initially.  This situation is equivalent to having
acquired the lock: when there is only a single thread, all object
accesses are safe.  Therefore, when this function initializes the
lock, it also acquires it.  Before the Python \code{thread} module
creates a new thread, knowing that either it has the lock or the lock
hasn't been created yet, it calls \code{PyEval_InitThreads()}.  When
this call returns, it is guaranteed that the lock has been created and
that it has acquired it.

It is \strong{not} safe to call this function when it is unknown which
thread (if any) currently has the global interpreter lock.

This function is not available when thread support is disabled at
compile time.
\end{cfuncdesc}

\begin{cfuncdesc}{void}{PyEval_AcquireLock}{}
\strong{(NEW in 1.5a3!)}
Acquire the global interpreter lock.  The lock must have been created
earlier.  If this thread already has the lock, a deadlock ensues.
This function is not available when thread support is disabled at
compile time.
\end{cfuncdesc}

\begin{cfuncdesc}{void}{PyEval_ReleaseLock}{}
\strong{(NEW in 1.5a3!)}
Release the global interpreter lock.  The lock must have been created
earlier.  This function is not available when thread support is
disabled at
compile time.
\end{cfuncdesc}

\begin{cfuncdesc}{void}{PyEval_AcquireThread}{PyThreadState *tstate}
\strong{(NEW in 1.5a3!)}
Acquire the global interpreter lock and then set the current thread
state to \var{tstate}, which should not be \NULL{}.  The lock must
have been created earlier.  If this thread already has the lock,
deadlock ensues.  This function is not available when thread support
is disabled at
compile time.
\end{cfuncdesc}

\begin{cfuncdesc}{void}{PyEval_ReleaseThread}{PyThreadState *tstate}
\strong{(NEW in 1.5a3!)}
Reset the current thread state to \NULL{} and release the global
interpreter lock.  The lock must have been created earlier and must be
held by the current thread.  The \var{tstate} argument, which must not
be \NULL{}, is only used to check that it represents the current
thread state -- if it isn't, a fatal error is reported.  This function
is not available when thread support is disabled at
compile time.
\end{cfuncdesc}

\begin{cfuncdesc}{PyThreadState *}{PyEval_SaveThread}{}
\strong{(Different return type in 1.5a3!)}
Release the interpreter lock (if it has been created and thread
support is enabled) and reset the thread state to \NULL{},
returning the previous thread state (which is not \NULL{}).  If
the lock has been created, the current thread must have acquired it.
(This function is available even when thread support is disabled at
compile time.)
\end{cfuncdesc}

\begin{cfuncdesc}{void}{PyEval_RestoreThread}{PyThreadState *tstate}
\strong{(Different argument type in 1.5a3!)}
Acquire the interpreter lock (if it has been created and thread
support is enabled) and set the thread state to \var{tstate}, which
must not be \NULL{}.  If the lock has been created, the current
thread must not have acquired it, otherwise deadlock ensues.  (This
function is available even when thread support is disabled at compile
time.)
\end{cfuncdesc}

% XXX These aren't really C types, but the ctypedesc macro is the simplest!
\begin{ctypedesc}{Py_BEGIN_ALLOW_THREADS}
This macro expands to
\code{\{ PyThreadState *_save; _save = PyEval_SaveThread();}.
Note that it contains an opening brace; it must be matched with a
following \code{Py_END_ALLOW_THREADS} macro.  See above for further
discussion of this macro.  It is a no-op when thread support is
disabled at compile time.
\end{ctypedesc}

\begin{ctypedesc}{Py_END_ALLOW_THREADS}
This macro expands to
\code{PyEval_RestoreThread(_save); \} }.
Note that it contains a closing brace; it must be matched with an
earlier \code{Py_BEGIN_ALLOW_THREADS} macro.  See above for further
discussion of this macro.  It is a no-op when thread support is
disabled at compile time.
\end{ctypedesc}

\begin{ctypedesc}{Py_BEGIN_BLOCK_THREADS}
This macro expands to \code{PyEval_RestoreThread(_save);} i.e. it
is equivalent to \code{Py_END_ALLOW_THREADS} without the closing
brace.  It is a no-op when thread support is disabled at compile
time.
\end{ctypedesc}

\begin{ctypedesc}{Py_BEGIN_UNBLOCK_THREADS}
This macro expands to \code{_save = PyEval_SaveThread();} i.e. it is
equivalent to \code{Py_BEGIN_ALLOW_THREADS} without the opening brace
and variable declaration.  It is a no-op when thread support is
disabled at compile time.
\end{ctypedesc}

All of the following functions are only available when thread support
is enabled at compile time, and must be called only when the
interpreter lock has been created.  They are all new in 1.5a3.

\begin{cfuncdesc}{PyInterpreterState *}{PyInterpreterState_New}{}
Create a new interpreter state object.  The interpreter lock must be
held.
\end{cfuncdesc}

\begin{cfuncdesc}{void}{PyInterpreterState_Clear}{PyInterpreterState *interp}
Reset all information in an interpreter state object.  The interpreter
lock must be held.
\end{cfuncdesc}

\begin{cfuncdesc}{void}{PyInterpreterState_Delete}{PyInterpreterState *interp}
Destroy an interpreter state object.  The interpreter lock need not be
held.  The interpreter state must have been reset with a previous
call to \code{PyInterpreterState_Clear()}.
\end{cfuncdesc}

\begin{cfuncdesc}{PyThreadState *}{PyThreadState_New}{PyInterpreterState *interp}
Create a new thread state object belonging to the given interpreter
object.  The interpreter lock must be held.
\end{cfuncdesc}

\begin{cfuncdesc}{void}{PyThreadState_Clear}{PyThreadState *tstate}
Reset all information in a thread state object.  The interpreter lock
must be held.
\end{cfuncdesc}

\begin{cfuncdesc}{void}{PyThreadState_Delete}{PyThreadState *tstate}
Destroy a thread state object.  The interpreter lock need not be
held.  The thread state must have been reset with a previous
call to \code{PyThreadState_Clear()}.
\end{cfuncdesc}

\begin{cfuncdesc}{PyThreadState *}{PyThreadState_Get}{}
Return the current thread state.  The interpreter lock must be held.
When the current thread state is \NULL{}, this issues a fatal
error (so that the caller needn't check for \NULL{}).
\end{cfuncdesc}

\begin{cfuncdesc}{PyThreadState *}{PyThreadState_Swap}{PyThreadState *tstate}
Swap the current thread state with the thread state given by the
argument \var{tstate}, which may be \NULL{}.  The interpreter lock
must be held.
\end{cfuncdesc}


\section{Defining New Object Types}

XXX To be done:

PyObject, PyVarObject

PyObject_HEAD, PyObject_HEAD_INIT, PyObject_VAR_HEAD

Typedefs:
unaryfunc, binaryfunc, ternaryfunc, inquiry, coercion, intargfunc,
intintargfunc, intobjargproc, intintobjargproc, objobjargproc,
getreadbufferproc, getwritebufferproc, getsegcountproc,
destructor, printfunc, getattrfunc, getattrofunc, setattrfunc,
setattrofunc, cmpfunc, reprfunc, hashfunc

PyNumberMethods

PySequenceMethods

PyMappingMethods

PyBufferProcs

PyTypeObject

DL_IMPORT

PyType_Type

Py*_Check

Py_None, _Py_NoneStruct

_PyObject_New, _PyObject_NewVar

PyObject_NEW, PyObject_NEW_VAR


\chapter{Specific Data Types}

This chapter describes the functions that deal with specific types of 
Python objects.  It is structured like the ``family tree'' of Python 
object types.


\section{Fundamental Objects}

This section describes Python type objects and the singleton object 
\code{None}.


\subsection{Type Objects}

\begin{ctypedesc}{PyTypeObject}

\end{ctypedesc}

\begin{cvardesc}{PyObject *}{PyType_Type}

\end{cvardesc}


\subsection{The None Object}

\begin{cvardesc}{PyObject *}{Py_None}
XXX macro
\end{cvardesc}


\section{Sequence Objects}

Generic operations on sequence objects were discussed in the previous 
chapter; this section deals with the specific kinds of sequence 
objects that are intrinsic to the Python language.


\subsection{String Objects}

\begin{ctypedesc}{PyStringObject}
This subtype of \code{PyObject} represents a Python string object.
\end{ctypedesc}

\begin{cvardesc}{PyTypeObject}{PyString_Type}
This instance of \code{PyTypeObject} represents the Python string type.
\end{cvardesc}

\begin{cfuncdesc}{int}{PyString_Check}{PyObject *o}

\end{cfuncdesc}

\begin{cfuncdesc}{PyObject *}{PyString_FromStringAndSize}{const char *, int}

\end{cfuncdesc}

\begin{cfuncdesc}{PyObject *}{PyString_FromString}{const char *}

\end{cfuncdesc}

\begin{cfuncdesc}{int}{PyString_Size}{PyObject *}

\end{cfuncdesc}

\begin{cfuncdesc}{char *}{PyString_AsString}{PyObject *}

\end{cfuncdesc}

\begin{cfuncdesc}{void}{PyString_Concat}{PyObject **, PyObject *}

\end{cfuncdesc}

\begin{cfuncdesc}{void}{PyString_ConcatAndDel}{PyObject **, PyObject *}

\end{cfuncdesc}

\begin{cfuncdesc}{int}{_PyString_Resize}{PyObject **, int}

\end{cfuncdesc}

\begin{cfuncdesc}{PyObject *}{PyString_Format}{PyObject *, PyObject *}

\end{cfuncdesc}

\begin{cfuncdesc}{void}{PyString_InternInPlace}{PyObject **}

\end{cfuncdesc}

\begin{cfuncdesc}{PyObject *}{PyString_InternFromString}{const char *}

\end{cfuncdesc}

\begin{cfuncdesc}{char *}{PyString_AS_STRING}{PyStringObject *}

\end{cfuncdesc}

\begin{cfuncdesc}{int}{PyString_GET_SIZE}{PyStringObject *}

\end{cfuncdesc}


\subsection{Tuple Objects}

\begin{ctypedesc}{PyTupleObject}
This subtype of \code{PyObject} represents a Python tuple object.
\end{ctypedesc}

\begin{cvardesc}{PyTypeObject}{PyTuple_Type}
This instance of \code{PyTypeObject} represents the Python tuple type.
\end{cvardesc}

\begin{cfuncdesc}{int}{PyTuple_Check}{PyObject *p}
Return true if the argument is a tuple object.
\end{cfuncdesc}

\begin{cfuncdesc}{PyTupleObject *}{PyTuple_New}{int s}
Return a new tuple object of size \code{s}
\end{cfuncdesc}

\begin{cfuncdesc}{int}{PyTuple_Size}{PyTupleObject *p}
akes a pointer to a tuple object, and returns the size
of that tuple.
\end{cfuncdesc}

\begin{cfuncdesc}{PyObject *}{PyTuple_GetItem}{PyTupleObject *p, int pos}
returns the object at position \code{pos} in the tuple pointed
to by \code{p}.
\end{cfuncdesc}

\begin{cfuncdesc}{PyObject *}{PyTuple_GET_ITEM}{PyTupleObject *p, int pos}
does the same, but does no checking of it's
arguments.
\end{cfuncdesc}

\begin{cfuncdesc}{PyTupleObject *}{PyTuple_GetSlice}{PyTupleObject *p,
            int low,
            int high}
takes a slice of the tuple pointed to by \code{p} from
\code{low} to \code{high} and returns it as a new tuple.
\end{cfuncdesc}

\begin{cfuncdesc}{int}{PyTuple_SetItem}{PyTupleObject *p,
            int pos,
            PyObject *o}
inserts a reference to object \code{o} at position \code{pos} of
the tuple pointed to by \code{p}. It returns 0 on success.
\end{cfuncdesc}

\begin{cfuncdesc}{void}{PyTuple_SET_ITEM}{PyTupleObject *p,
            int pos,
            PyObject *o}

does the same, but does no error checking, and
should \emph{only} be used to fill in brand new tuples.
\end{cfuncdesc}

\begin{cfuncdesc}{PyTupleObject *}{_PyTuple_Resize}{PyTupleObject *p,
            int new,
            int last_is_sticky}
can be used to resize a tuple. Because tuples are
\emph{supposed} to be immutable, this should only be used if there is only
one module referencing the object. Do \emph{not} use this if the tuple may
already be known to some other part of the code. \code{last_is_sticky} is
a flag - if set, the tuple will grow or shrink at the front, otherwise
it will grow or shrink at the end. Think of this as destroying the old
tuple and creating a new one, only more efficiently.
\end{cfuncdesc}


\subsection{List Objects}

\begin{ctypedesc}{PyListObject}
This subtype of \code{PyObject} represents a Python list object.
\end{ctypedesc}

\begin{cvardesc}{PyTypeObject}{PyList_Type}
This instance of \code{PyTypeObject} represents the Python list type.
\end{cvardesc}

\begin{cfuncdesc}{int}{PyList_Check}{PyObject *p}
returns true if it's argument is a \code{PyListObject}
\end{cfuncdesc}

\begin{cfuncdesc}{PyObject *}{PyList_New}{int size}

\end{cfuncdesc}

\begin{cfuncdesc}{int}{PyList_Size}{PyObject *}

\end{cfuncdesc}

\begin{cfuncdesc}{PyObject *}{PyList_GetItem}{PyObject *, int}

\end{cfuncdesc}

\begin{cfuncdesc}{int}{PyList_SetItem}{PyObject *, int, PyObject *}

\end{cfuncdesc}

\begin{cfuncdesc}{int}{PyList_Insert}{PyObject *, int, PyObject *}

\end{cfuncdesc}

\begin{cfuncdesc}{int}{PyList_Append}{PyObject *, PyObject *}

\end{cfuncdesc}

\begin{cfuncdesc}{PyObject *}{PyList_GetSlice}{PyObject *, int, int}

\end{cfuncdesc}

\begin{cfuncdesc}{int}{PyList_SetSlice}{PyObject *, int, int, PyObject *}

\end{cfuncdesc}

\begin{cfuncdesc}{int}{PyList_Sort}{PyObject *}

\end{cfuncdesc}

\begin{cfuncdesc}{int}{PyList_Reverse}{PyObject *}

\end{cfuncdesc}

\begin{cfuncdesc}{PyObject *}{PyList_AsTuple}{PyObject *}

\end{cfuncdesc}

\begin{cfuncdesc}{PyObject *}{PyList_GET_ITEM}{PyObject *list, int i}

\end{cfuncdesc}

\begin{cfuncdesc}{int}{PyList_GET_SIZE}{PyObject *list}

\end{cfuncdesc}


\section{Mapping Objects}

\subsection{Dictionary Objects}

\begin{ctypedesc}{PyDictObject}
This subtype of \code{PyObject} represents a Python dictionary object.
\end{ctypedesc}

\begin{cvardesc}{PyTypeObject}{PyDict_Type}
This instance of \code{PyTypeObject} represents the Python dictionary type.
\end{cvardesc}

\begin{cfuncdesc}{int}{PyDict_Check}{PyObject *p}
returns true if it's argument is a PyDictObject
\end{cfuncdesc}

\begin{cfuncdesc}{PyDictObject *}{PyDict_New}{}
returns a new empty dictionary.
\end{cfuncdesc}

\begin{cfuncdesc}{void}{PyDict_Clear}{PyDictObject *p}
empties an existing dictionary and deletes it.
\end{cfuncdesc}

\begin{cfuncdesc}{int}{PyDict_SetItem}{PyDictObject *p,
            PyObject *key,
            PyObject *val}
inserts \code{value} into the dictionary with a key of
\code{key}. Both \code{key} and \code{value} should be PyObjects, and \code{key} should
be hashable.
\end{cfuncdesc}

\begin{cfuncdesc}{int}{PyDict_SetItemString}{PyDictObject *p,
            char *key,
            PyObject *val}
inserts \code{value} into the dictionary using \code{key}
as a key. \code{key} should be a char *
\end{cfuncdesc}

\begin{cfuncdesc}{int}{PyDict_DelItem}{PyDictObject *p, PyObject *key}
removes the entry in dictionary \code{p} with key \code{key}.
\code{key} is a PyObject.
\end{cfuncdesc}

\begin{cfuncdesc}{int}{PyDict_DelItemString}{PyDictObject *p, char *key}
removes the entry in dictionary \code{p} which has a key
specified by the \code{char *}\code{key}.
\end{cfuncdesc}

\begin{cfuncdesc}{PyObject *}{PyDict_GetItem}{PyDictObject *p, PyObject *key}
returns the object from dictionary \code{p} which has a key
\code{key}.
\end{cfuncdesc}

\begin{cfuncdesc}{PyObject *}{PyDict_GetItemString}{PyDictObject *p, char *key}
does the same, but \code{key} is specified as a
\code{char *}, rather than a \code{PyObject *}.
\end{cfuncdesc}

\begin{cfuncdesc}{PyListObject *}{PyDict_Items}{PyDictObject *p}
returns a PyListObject containing all the items 
from the dictionary, as in the mapping method \code{items()} (see the Reference
Guide)
\end{cfuncdesc}

\begin{cfuncdesc}{PyListObject *}{PyDict_Keys}{PyDictObject *p}
returns a PyListObject containing all the keys 
from the dictionary, as in the mapping method \code{keys()} (see the Reference Guide)
\end{cfuncdesc}

\begin{cfuncdesc}{PyListObject *}{PyDict_Values}{PyDictObject *p}
returns a PyListObject containing all the values 
from the dictionary, as in the mapping method \code{values()} (see the Reference Guide)
\end{cfuncdesc}

\begin{cfuncdesc}{int}{PyDict_Size}{PyDictObject *p}
returns the number of items in the dictionary.
\end{cfuncdesc}

\begin{cfuncdesc}{int}{PyDict_Next}{PyDictObject *p,
            int ppos,
            PyObject **pkey,
            PyObject **pvalue}

\end{cfuncdesc}


\section{Numeric Objects}

\subsection{Plain Integer Objects}

\begin{ctypedesc}{PyIntObject}
This subtype of \code{PyObject} represents a Python integer object.
\end{ctypedesc}

\begin{cvardesc}{PyTypeObject}{PyInt_Type}
This instance of \code{PyTypeObject} represents the Python plain 
integer type.
\end{cvardesc}

\begin{cfuncdesc}{int}{PyInt_Check}{PyObject *}

\end{cfuncdesc}

\begin{cfuncdesc}{PyIntObject *}{PyInt_FromLong}{long ival}
creates a new integer object with a value of \code{ival}.

The current implementation keeps an array of integer objects for all
integers between -1 and 100, when you create an int in that range you
actually just get back a reference to the existing object. So it should
be possible to change the value of 1. I suspect the behaviour of python
in this case is undefined. :-)
\end{cfuncdesc}

\begin{cfuncdesc}{long}{PyInt_AS_LONG}{PyIntObject *io}
returns the value of the object \code{io}.
\end{cfuncdesc}

\begin{cfuncdesc}{long}{PyInt_AsLong}{PyObject *io}
will first attempt to cast the object to a PyIntObject, if
it is not already one, and the return it's value.
\end{cfuncdesc}

\begin{cfuncdesc}{long}{PyInt_GetMax}{}
returns the systems idea of the largest int it can handle
(LONG_MAX, as defined in the system header files)
\end{cfuncdesc}


\subsection{Long Integer Objects}

\begin{ctypedesc}{PyLongObject}
This subtype of \code{PyObject} represents a Python long integer object.
\end{ctypedesc}

\begin{cvardesc}{PyTypeObject}{PyLong_Type}
This instance of \code{PyTypeObject} represents the Python long integer type.
\end{cvardesc}

\begin{cfuncdesc}{int}{PyLong_Check}{PyObject *p}
returns true if it's argument is a \code{PyLongObject}
\end{cfuncdesc}

\begin{cfuncdesc}{PyObject *}{PyLong_FromLong}{long}

\end{cfuncdesc}

\begin{cfuncdesc}{PyObject *}{PyLong_FromUnsignedLong}{unsigned long}

\end{cfuncdesc}

\begin{cfuncdesc}{PyObject *}{PyLong_FromDouble}{double}

\end{cfuncdesc}

\begin{cfuncdesc}{long}{PyLong_AsLong}{PyObject *}

\end{cfuncdesc}

\begin{cfuncdesc}{unsigned long}{PyLong_AsUnsignedLong}{PyObject }

\end{cfuncdesc}

\begin{cfuncdesc}{double}{PyLong_AsDouble}{PyObject *}

\end{cfuncdesc}

\begin{cfuncdesc}{PyObject *}{PyLong_FromString}{char *, char **, int}

\end{cfuncdesc}


\subsection{Floating Point Objects}

\begin{ctypedesc}{PyFloatObject}
This subtype of \code{PyObject} represents a Python floating point object.
\end{ctypedesc}

\begin{cvardesc}{PyTypeObject}{PyFloat_Type}
This instance of \code{PyTypeObject} represents the Python floating 
point type.
\end{cvardesc}

\begin{cfuncdesc}{int}{PyFloat_Check}{PyObject *p}
returns true if it's argument is a \code{PyFloatObject}
\end{cfuncdesc}

\begin{cfuncdesc}{PyObject *}{PyFloat_FromDouble}{double}

\end{cfuncdesc}

\begin{cfuncdesc}{double}{PyFloat_AsDouble}{PyObject *}

\end{cfuncdesc}

\begin{cfuncdesc}{double}{PyFloat_AS_DOUBLE}{PyFloatObject *}

\end{cfuncdesc}


\subsection{Complex Number Objects}

\begin{ctypedesc}{Py_complex}
typedef struct {
   double real;
   double imag;
} 
\end{ctypedesc}

\begin{ctypedesc}{PyComplexObject}
This subtype of \code{PyObject} represents a Python complex number object.
\end{ctypedesc}

\begin{cvardesc}{PyTypeObject}{PyComplex_Type}
This instance of \code{PyTypeObject} represents the Python complex 
number type.
\end{cvardesc}

\begin{cfuncdesc}{int}{PyComplex_Check}{PyObject *p}
returns true if it's argument is a \code{PyComplexObject}
\end{cfuncdesc}

\begin{cfuncdesc}{Py_complex}{_Py_c_sum}{Py_complex, Py_complex}

\end{cfuncdesc}

\begin{cfuncdesc}{Py_complex}{_Py_c_diff}{Py_complex, Py_complex}

\end{cfuncdesc}

\begin{cfuncdesc}{Py_complex}{_Py_c_neg}{Py_complex}

\end{cfuncdesc}

\begin{cfuncdesc}{Py_complex}{_Py_c_prod}{Py_complex, Py_complex}

\end{cfuncdesc}

\begin{cfuncdesc}{Py_complex}{_Py_c_quot}{Py_complex, Py_complex}

\end{cfuncdesc}

\begin{cfuncdesc}{Py_complex}{_Py_c_pow}{Py_complex, Py_complex}

\end{cfuncdesc}

\begin{cfuncdesc}{PyObject *}{PyComplex_FromCComplex}{Py_complex}

\end{cfuncdesc}

\begin{cfuncdesc}{PyObject *}{PyComplex_FromDoubles}{double real, double imag}

\end{cfuncdesc}

\begin{cfuncdesc}{double}{PyComplex_RealAsDouble}{PyObject *op}

\end{cfuncdesc}

\begin{cfuncdesc}{double}{PyComplex_ImagAsDouble}{PyObject *op}

\end{cfuncdesc}

\begin{cfuncdesc}{Py_complex}{PyComplex_AsCComplex}{PyObject *op}

\end{cfuncdesc}



\section{Other Objects}

\subsection{File Objects}

\begin{ctypedesc}{PyFileObject}
This subtype of \code{PyObject} represents a Python file object.
\end{ctypedesc}

\begin{cvardesc}{PyTypeObject}{PyFile_Type}
This instance of \code{PyTypeObject} represents the Python file type.
\end{cvardesc}

\begin{cfuncdesc}{int}{PyFile_Check}{PyObject *p}
returns true if it's argument is a \code{PyFileObject}
\end{cfuncdesc}

\begin{cfuncdesc}{PyObject *}{PyFile_FromString}{char *name, char *mode}
creates a new PyFileObject pointing to the file
specified in \code{name} with the mode specified in \code{mode}
\end{cfuncdesc}

\begin{cfuncdesc}{PyObject *}{PyFile_FromFile}{FILE *fp,
              char *name, char *mode, int (*close})
creates a new PyFileObject from the already-open \code{fp}.
The function \code{close} will be called when the file should be closed.
\end{cfuncdesc}

\begin{cfuncdesc}{FILE *}{PyFile_AsFile}{PyFileObject *p}
returns the file object associated with \code{p} as a \code{FILE *}
\end{cfuncdesc}

\begin{cfuncdesc}{PyStringObject *}{PyFile_GetLine}{PyObject *p, int n}
undocumented as yet
\end{cfuncdesc}

\begin{cfuncdesc}{PyStringObject *}{PyFile_Name}{PyObject *p}
returns the name of the file specified by \code{p} as a 
PyStringObject
\end{cfuncdesc}

\begin{cfuncdesc}{void}{PyFile_SetBufSize}{PyFileObject *p, int n}
on systems with \code{setvbuf} only
\end{cfuncdesc}

\begin{cfuncdesc}{int}{PyFile_SoftSpace}{PyFileObject *p, int newflag}
same as the file object method \code{softspace}
\end{cfuncdesc}

\begin{cfuncdesc}{int}{PyFile_WriteObject}{PyObject *obj, PyFileObject *p}
writes object \code{obj} to file object \code{p}
\end{cfuncdesc}

\begin{cfuncdesc}{int}{PyFile_WriteString}{char *s, PyFileObject *p}
writes string \code{s} to file object \code{p}
\end{cfuncdesc}


\subsection{CObjects}

XXX


\documentclass{manual}

\title{Python/C API Reference Manual}

\author{
	Guido van Rossum \\
	Dept. AA, CWI, P.O. Box 94079 \\
	1090 GB Amsterdam, The Netherlands \\
	E-mail: {\tt guido@cwi.nl}
}

\date{17 March 1995 \\ Release 1.2-proof-2} % XXX update before release!


\makeindex			% tell \index to actually write the .idx file


\begin{document}

\maketitle

\strong{BEOPEN.COM TERMS AND CONDITIONS FOR PYTHON 2.0}

\centerline{\strong{BEOPEN PYTHON OPEN SOURCE LICENSE AGREEMENT VERSION 1}}

\begin{enumerate}

\item
This LICENSE AGREEMENT is between BeOpen.com (``BeOpen''), having an
office at 160 Saratoga Avenue, Santa Clara, CA 95051, and the
Individual or Organization (``Licensee'') accessing and otherwise
using this software in source or binary form and its associated
documentation (``the Software'').

\item
Subject to the terms and conditions of this BeOpen Python License
Agreement, BeOpen hereby grants Licensee a non-exclusive,
royalty-free, world-wide license to reproduce, analyze, test, perform
and/or display publicly, prepare derivative works, distribute, and
otherwise use the Software alone or in any derivative version,
provided, however, that the BeOpen Python License is retained in the
Software, alone or in any derivative version prepared by Licensee.

\item
BeOpen is making the Software available to Licensee on an ``AS IS''
basis.  BEOPEN MAKES NO REPRESENTATIONS OR WARRANTIES, EXPRESS OR
IMPLIED.  BY WAY OF EXAMPLE, BUT NOT LIMITATION, BEOPEN MAKES NO AND
DISCLAIMS ANY REPRESENTATION OR WARRANTY OF MERCHANTABILITY OR FITNESS
FOR ANY PARTICULAR PURPOSE OR THAT THE USE OF THE SOFTWARE WILL NOT
INFRINGE ANY THIRD PARTY RIGHTS.

\item
BEOPEN SHALL NOT BE LIABLE TO LICENSEE OR ANY OTHER USERS OF THE
SOFTWARE FOR ANY INCIDENTAL, SPECIAL, OR CONSEQUENTIAL DAMAGES OR LOSS
AS A RESULT OF USING, MODIFYING OR DISTRIBUTING THE SOFTWARE, OR ANY
DERIVATIVE THEREOF, EVEN IF ADVISED OF THE POSSIBILITY THEREOF.

\item
This License Agreement will automatically terminate upon a material
breach of its terms and conditions.

\item
This License Agreement shall be governed by and interpreted in all
respects by the law of the State of California, excluding conflict of
law provisions.  Nothing in this License Agreement shall be deemed to
create any relationship of agency, partnership, or joint venture
between BeOpen and Licensee.  This License Agreement does not grant
permission to use BeOpen trademarks or trade names in a trademark
sense to endorse or promote products or services of Licensee, or any
third party.  As an exception, the ``BeOpen Python'' logos available
at http://www.pythonlabs.com/logos.html may be used according to the
permissions granted on that web page.

\item
By copying, installing or otherwise using the software, Licensee
agrees to be bound by the terms and conditions of this License
Agreement.
\end{enumerate}


\centerline{\strong{CNRI OPEN SOURCE LICENSE AGREEMENT}}

Python 1.6 is made available subject to the terms and conditions in
CNRI's License Agreement.  This Agreement together with Python 1.6 may
be located on the Internet using the following unique, persistent
identifier (known as a handle): 1895.22/1012.  This Agreement may also
be obtained from a proxy server on the Internet using the following
URL: \url{http://hdl.handle.net/1895.22/1012}.


\centerline{\strong{CWI PERMISSIONS STATEMENT AND DISCLAIMER}}

Copyright \copyright{} 1991 - 1995, Stichting Mathematisch Centrum
Amsterdam, The Netherlands.  All rights reserved.

Permission to use, copy, modify, and distribute this software and its
documentation for any purpose and without fee is hereby granted,
provided that the above copyright notice appear in all copies and that
both that copyright notice and this permission notice appear in
supporting documentation, and that the name of Stichting Mathematisch
Centrum or CWI not be used in advertising or publicity pertaining to
distribution of the software without specific, written prior
permission.

STICHTING MATHEMATISCH CENTRUM DISCLAIMS ALL WARRANTIES WITH REGARD TO
THIS SOFTWARE, INCLUDING ALL IMPLIED WARRANTIES OF MERCHANTABILITY AND
FITNESS, IN NO EVENT SHALL STICHTING MATHEMATISCH CENTRUM BE LIABLE
FOR ANY SPECIAL, INDIRECT OR CONSEQUENTIAL DAMAGES OR ANY DAMAGES
WHATSOEVER RESULTING FROM LOSS OF USE, DATA OR PROFITS, WHETHER IN AN
ACTION OF CONTRACT, NEGLIGENCE OR OTHER TORTIOUS ACTION, ARISING OUT
OF OR IN CONNECTION WITH THE USE OR PERFORMANCE OF THIS SOFTWARE.


\begin{abstract}

\noindent
This manual documents the API used by \C{} (or \Cpp{}) programmers who
want to write extension modules or embed Python.  It is a companion to
\emph{Extending and Embedding the Python Interpreter}, which describes
the general principles of extension writing but does not document the
API functions in detail.

\strong{Warning:} The current version of this document is incomplete.
I hope that it is nevertheless useful.  I will continue to work on it,
and release new versions from time to time, independent from Python
source code releases.

\end{abstract}

\tableofcontents

% XXX Consider moving all this back to ext.tex and giving api.tex
% XXX a *really* short intro only.

\chapter{Introduction}
\label{intro}

The Application Programmer's Interface to Python gives \C{} and \Cpp{}
programmers access to the Python interpreter at a variety of levels.
The API is equally usable from \Cpp{}, but for brevity it is generally
referred to as the Python/\C{} API.  There are two fundamentally
different reasons for using the Python/\C{} API.  The first reason is
to write \emph{extension modules} for specific purposes; these are
\C{} modules that extend the Python interpreter.  This is probably the
most common use.  The second reason is to use Python as a component in
a larger application; this technique is generally referred to as
\dfn{embedding} Python in an application.

Writing an extension module is a relatively well-understood process, 
where a ``cookbook'' approach works well.  There are several tools 
that automate the process to some extent.  While people have embedded 
Python in other applications since its early existence, the process of 
embedding Python is less straightforward that writing an extension.  
Python 1.5 introduces a number of new API functions as well as some 
changes to the build process that make embedding much simpler.  
This manual describes the \version\ state of affairs.
% XXX Eventually, take the historical notes out

Many API functions are useful independent of whether you're embedding 
or extending Python; moreover, most applications that embed Python 
will need to provide a custom extension as well, so it's probably a 
good idea to become familiar with writing an extension before 
attempting to embed Python in a real application.

\section{Include Files}
\label{includes}

All function, type and macro definitions needed to use the Python/C
API are included in your code by the following line:

\begin{verbatim}
#include "Python.h"
\end{verbatim}

This implies inclusion of the following standard headers:
\code{<stdio.h>}, \code{<string.h>}, \code{<errno.h>}, and
\code{<stdlib.h>} (if available).

All user visible names defined by Python.h (except those defined by
the included standard headers) have one of the prefixes \samp{Py} or
\samp{_Py}.  Names beginning with \samp{_Py} are for internal use
only.  Structure member names do not have a reserved prefix.

\strong{Important:} user code should never define names that begin
with \samp{Py} or \samp{_Py}.  This confuses the reader, and
jeopardizes the portability of the user code to future Python
versions, which may define additional names beginning with one of
these prefixes.

\section{Objects, Types and Reference Counts}
\label{objects}

Most Python/C API functions have one or more arguments as well as a
return value of type \ctype{PyObject *}.  This type is a pointer
to an opaque data type representing an arbitrary Python
object.  Since all Python object types are treated the same way by the
Python language in most situations (e.g., assignments, scope rules,
and argument passing), it is only fitting that they should be
represented by a single \C{} type.  All Python objects live on the heap:
you never declare an automatic or static variable of type
\ctype{PyObject}, only pointer variables of type \ctype{PyObject *} can 
be declared.

All Python objects (even Python integers) have a \dfn{type} and a
\dfn{reference count}.  An object's type determines what kind of object 
it is (e.g., an integer, a list, or a user-defined function; there are 
many more as explained in the \emph{Python Reference Manual}).  For 
each of the well-known types there is a macro to check whether an 
object is of that type; for instance, \samp{PyList_Check(\var{a})} is
true iff the object pointed to by \var{a} is a Python list.

\subsection{Reference Counts}
\label{refcounts}

The reference count is important because today's computers have a 
finite (and often severely limited) memory size; it counts how many 
different places there are that have a reference to an object.  Such a 
place could be another object, or a global (or static) \C{} variable, or 
a local variable in some \C{} function.  When an object's reference count 
becomes zero, the object is deallocated.  If it contains references to 
other objects, their reference count is decremented.  Those other 
objects may be deallocated in turn, if this decrement makes their 
reference count become zero, and so on.  (There's an obvious problem 
with objects that reference each other here; for now, the solution is 
``don't do that''.)

Reference counts are always manipulated explicitly.  The normal way is 
to use the macro \cfunction{Py_INCREF()} to increment an object's 
reference count by one, and \cfunction{Py_DECREF()} to decrement it by 
one.  The decref macro is considerably more complex than the incref one, 
since it must check whether the reference count becomes zero and then 
cause the object's deallocator, which is a function pointer contained 
in the object's type structure.  The type-specific deallocator takes 
care of decrementing the reference counts for other objects contained 
in the object, and so on, if this is a compound object type such as a 
list.  There's no chance that the reference count can overflow; at 
least as many bits are used to hold the reference count as there are 
distinct memory locations in virtual memory (assuming 
\code{sizeof(long) >= sizeof(char *)}).  Thus, the reference count 
increment is a simple operation.

It is not necessary to increment an object's reference count for every 
local variable that contains a pointer to an object.  In theory, the 
object's reference count goes up by one when the variable is made to 
point to it and it goes down by one when the variable goes out of 
scope.  However, these two cancel each other out, so at the end the 
reference count hasn't changed.  The only real reason to use the 
reference count is to prevent the object from being deallocated as 
long as our variable is pointing to it.  If we know that there is at 
least one other reference to the object that lives at least as long as 
our variable, there is no need to increment the reference count 
temporarily.  An important situation where this arises is in objects 
that are passed as arguments to \C{} functions in an extension module 
that are called from Python; the call mechanism guarantees to hold a 
reference to every argument for the duration of the call.

However, a common pitfall is to extract an object from a list and
hold on to it for a while without incrementing its reference count.
Some other operation might conceivably remove the object from the
list, decrementing its reference count and possible deallocating it.
The real danger is that innocent-looking operations may invoke
arbitrary Python code which could do this; there is a code path which
allows control to flow back to the user from a \cfunction{Py_DECREF()},
so almost any operation is potentially dangerous.

A safe approach is to always use the generic operations (functions 
whose name begins with \samp{PyObject_}, \samp{PyNumber_}, 
\samp{PySequence_} or \samp{PyMapping_}).  These operations always 
increment the reference count of the object they return.  This leaves 
the caller with the responsibility to call \cfunction{Py_DECREF()}
when they are done with the result; this soon becomes second nature.

\subsubsection{Reference Count Details}
\label{refcountDetails}

The reference count behavior of functions in the Python/C API is best 
expelained in terms of \emph{ownership of references}.  Note that we 
talk of owning references, never of owning objects; objects are always 
shared!  When a function owns a reference, it has to dispose of it 
properly --- either by passing ownership on (usually to its caller) or 
by calling \cfunction{Py_DECREF()} or \cfunction{Py_XDECREF()}.  When
a function passes ownership of a reference on to its caller, the
caller is said to receive a \emph{new} reference.  When no ownership
is transferred, the caller is said to \emph{borrow} the reference.
Nothing needs to be done for a borrowed reference.

Conversely, when calling a function passes it a reference to an 
object, there are two possibilities: the function \emph{steals} a 
reference to the object, or it does not.  Few functions steal 
references; the two notable exceptions are
\cfunction{PyList_SetItem()} and \cfunction{PyTuple_SetItem()}, which
steal a reference to the item (but not to the tuple or list into which
the item is put!).  These functions were designed to steal a reference
because of a common idiom for populating a tuple or list with newly
created objects; for example, the code to create the tuple \code{(1,
2, "three")} could look like this (forgetting about error handling for
the moment; a better way to code this is shown below anyway):

\begin{verbatim}
PyObject *t;

t = PyTuple_New(3);
PyTuple_SetItem(t, 0, PyInt_FromLong(1L));
PyTuple_SetItem(t, 1, PyInt_FromLong(2L));
PyTuple_SetItem(t, 2, PyString_FromString("three"));
\end{verbatim}

Incidentally, \cfunction{PyTuple_SetItem()} is the \emph{only} way to
set tuple items; \cfunction{PySequence_SetItem()} and
\cfunction{PyObject_SetItem()} refuse to do this since tuples are an
immutable data type.  You should only use
\cfunction{PyTuple_SetItem()} for tuples that you are creating
yourself.

Equivalent code for populating a list can be written using 
\cfunction{PyList_New()} and \cfunction{PyList_SetItem()}.  Such code
can also use \cfunction{PySequence_SetItem()}; this illustrates the
difference between the two (the extra \cfunction{Py_DECREF()} calls):

\begin{verbatim}
PyObject *l, *x;

l = PyList_New(3);
x = PyInt_FromLong(1L);
PySequence_SetItem(l, 0, x); Py_DECREF(x);
x = PyInt_FromLong(2L);
PySequence_SetItem(l, 1, x); Py_DECREF(x);
x = PyString_FromString("three");
PySequence_SetItem(l, 2, x); Py_DECREF(x);
\end{verbatim}

You might find it strange that the ``recommended'' approach takes more
code.  However, in practice, you will rarely use these ways of
creating and populating a tuple or list.  There's a generic function,
\cfunction{Py_BuildValue()}, that can create most common objects from
\C{} values, directed by a \dfn{format string}.  For example, the
above two blocks of code could be replaced by the following (which
also takes care of the error checking):

\begin{verbatim}
PyObject *t, *l;

t = Py_BuildValue("(iis)", 1, 2, "three");
l = Py_BuildValue("[iis]", 1, 2, "three");
\end{verbatim}

It is much more common to use \cfunction{PyObject_SetItem()} and
friends with items whose references you are only borrowing, like
arguments that were passed in to the function you are writing.  In
that case, their behaviour regarding reference counts is much saner,
since you don't have to increment a reference count so you can give a
reference away (``have it be stolen'').  For example, this function
sets all items of a list (actually, any mutable sequence) to a given
item:

\begin{verbatim}
int set_all(PyObject *target, PyObject *item)
{
    int i, n;

    n = PyObject_Length(target);
    if (n < 0)
        return -1;
    for (i = 0; i < n; i++) {
        if (PyObject_SetItem(target, i, item) < 0)
            return -1;
    }
    return 0;
}
\end{verbatim}

The situation is slightly different for function return values.  
While passing a reference to most functions does not change your 
ownership responsibilities for that reference, many functions that 
return a referece to an object give you ownership of the reference.
The reason is simple: in many cases, the returned object is created 
on the fly, and the reference you get is the only reference to the 
object.  Therefore, the generic functions that return object 
references, like \cfunction{PyObject_GetItem()} and 
\cfunction{PySequence_GetItem()}, always return a new reference (i.e.,
the  caller becomes the owner of the reference).

It is important to realize that whether you own a reference returned 
by a function depends on which function you call only --- \emph{the
plumage} (i.e., the type of the type of the object passed as an
argument to the function) \emph{doesn't enter into it!}  Thus, if you 
extract an item from a list using \cfunction{PyList_GetItem()}, you
don't own the reference --- but if you obtain the same item from the
same list using \cfunction{PySequence_GetItem()} (which happens to
take exactly the same arguments), you do own a reference to the
returned object.

Here is an example of how you could write a function that computes the
sum of the items in a list of integers; once using 
\cfunction{PyList_GetItem()}, once using
\cfunction{PySequence_GetItem()}.

\begin{verbatim}
long sum_list(PyObject *list)
{
    int i, n;
    long total = 0;
    PyObject *item;

    n = PyList_Size(list);
    if (n < 0)
        return -1; /* Not a list */
    for (i = 0; i < n; i++) {
        item = PyList_GetItem(list, i); /* Can't fail */
        if (!PyInt_Check(item)) continue; /* Skip non-integers */
        total += PyInt_AsLong(item);
    }
    return total;
}
\end{verbatim}

\begin{verbatim}
long sum_sequence(PyObject *sequence)
{
    int i, n;
    long total = 0;
    PyObject *item;
    n = PyObject_Size(list);
    if (n < 0)
        return -1; /* Has no length */
    for (i = 0; i < n; i++) {
        item = PySequence_GetItem(list, i);
        if (item == NULL)
            return -1; /* Not a sequence, or other failure */
        if (PyInt_Check(item))
            total += PyInt_AsLong(item);
        Py_DECREF(item); /* Discard reference ownership */
    }
    return total;
}
\end{verbatim}

\subsection{Types}
\label{types}

There are few other data types that play a significant role in 
the Python/C API; most are simple \C{} types such as \ctype{int}, 
\ctype{long}, \ctype{double} and \ctype{char *}.  A few structure types 
are used to describe static tables used to list the functions exported 
by a module or the data attributes of a new object type.  These will 
be discussed together with the functions that use them.

\section{Exceptions}
\label{exceptions}

The Python programmer only needs to deal with exceptions if specific 
error handling is required; unhandled exceptions are automatically 
propagated to the caller, then to the caller's caller, and so on, till 
they reach the top-level interpreter, where they are reported to the 
user accompanied by a stack traceback.

For \C{} programmers, however, error checking always has to be explicit.  
All functions in the Python/C API can raise exceptions, unless an 
explicit claim is made otherwise in a function's documentation.  In 
general, when a function encounters an error, it sets an exception, 
discards any object references that it owns, and returns an 
error indicator --- usually \NULL{} or \code{-1}.  A few functions 
return a Boolean true/false result, with false indicating an error.
Very few functions return no explicit error indicator or have an 
ambiguous return value, and require explicit testing for errors with 
\cfunction{PyErr_Occurred()}.

Exception state is maintained in per-thread storage (this is 
equivalent to using global storage in an unthreaded application).  A 
thread can be in one of two states: an exception has occurred, or not.
The function \cfunction{PyErr_Occurred()} can be used to check for
this: it returns a borrowed reference to the exception type object
when an exception has occurred, and \NULL{} otherwise.  There are a
number of functions to set the exception state:
\cfunction{PyErr_SetString()} is the most common (though not the most
general) function to set the exception state, and
\cfunction{PyErr_Clear()} clears the exception state.

The full exception state consists of three objects (all of which can 
be \NULL{}): the exception type, the corresponding exception 
value, and the traceback.  These have the same meanings as the Python 
object \code{sys.exc_type}, \code{sys.exc_value}, 
\code{sys.exc_traceback}; however, they are not the same: the Python 
objects represent the last exception being handled by a Python 
\keyword{try} \ldots\ \keyword{except} statement, while the \C{} level
exception state only exists while an exception is being passed on
between \C{} functions until it reaches the Python interpreter, which
takes care of transferring it to \code{sys.exc_type} and friends.

Note that starting with Python 1.5, the preferred, thread-safe way to 
access the exception state from Python code is to call the function 
\function{sys.exc_info()}, which returns the per-thread exception state 
for Python code.  Also, the semantics of both ways to access the 
exception state have changed so that a function which catches an 
exception will save and restore its thread's exception state so as to 
preserve the exception state of its caller.  This prevents common bugs 
in exception handling code caused by an innocent-looking function 
overwriting the exception being handled; it also reduces the often 
unwanted lifetime extension for objects that are referenced by the 
stack frames in the traceback.

As a general principle, a function that calls another function to 
perform some task should check whether the called function raised an 
exception, and if so, pass the exception state on to its caller.  It 
should discard any object references that it owns, and returns an 
error indicator, but it should \emph{not} set another exception ---
that would overwrite the exception that was just raised, and lose
important information about the exact cause of the error.

A simple example of detecting exceptions and passing them on is shown 
in the \cfunction{sum_sequence()} example above.  It so happens that
that example doesn't need to clean up any owned references when it
detects an error.  The following example function shows some error
cleanup.  First, to remind you why you like Python, we show the
equivalent Python code:

\begin{verbatim}
def incr_item(dict, key):
    try:
        item = dict[key]
    except KeyError:
        item = 0
    return item + 1
\end{verbatim}

Here is the corresponding \C{} code, in all its glory:

\begin{verbatim}
int incr_item(PyObject *dict, PyObject *key)
{
    /* Objects all initialized to NULL for Py_XDECREF */
    PyObject *item = NULL, *const_one = NULL, *incremented_item = NULL;
    int rv = -1; /* Return value initialized to -1 (failure) */

    item = PyObject_GetItem(dict, key);
    if (item == NULL) {
        /* Handle KeyError only: */
        if (!PyErr_ExceptionMatches(PyExc_KeyError)) goto error;

        /* Clear the error and use zero: */
        PyErr_Clear();
        item = PyInt_FromLong(0L);
        if (item == NULL) goto error;
    }

    const_one = PyInt_FromLong(1L);
    if (const_one == NULL) goto error;

    incremented_item = PyNumber_Add(item, const_one);
    if (incremented_item == NULL) goto error;

    if (PyObject_SetItem(dict, key, incremented_item) < 0) goto error;
    rv = 0; /* Success */
    /* Continue with cleanup code */

 error:
    /* Cleanup code, shared by success and failure path */

    /* Use Py_XDECREF() to ignore NULL references */
    Py_XDECREF(item);
    Py_XDECREF(const_one);
    Py_XDECREF(incremented_item);

    return rv; /* -1 for error, 0 for success */
}
\end{verbatim}

This example represents an endorsed use of the \keyword{goto} statement 
in \C{}!  It illustrates the use of
\cfunction{PyErr_ExceptionMatches()} and \cfunction{PyErr_Clear()} to
handle specific exceptions, and the use of \cfunction{Py_XDECREF()} to
dispose of owned references that may be \NULL{} (note the \samp{X} in
the name; \cfunction{Py_DECREF()} would crash when confronted with a
\NULL{} reference).  It is important that the variables used to hold
owned references are initialized to \NULL{} for this to work;
likewise, the proposed return value is initialized to \code{-1}
(failure) and only set to success after the final call made is
successful.


\section{Embedding Python}
\label{embedding}

The one important task that only embedders (as opposed to extension
writers) of the Python interpreter have to worry about is the
initialization, and possibly the finalization, of the Python
interpreter.  Most functionality of the interpreter can only be used
after the interpreter has been initialized.

The basic initialization function is \cfunction{Py_Initialize()}.
This initializes the table of loaded modules, and creates the
fundamental modules \module{__builtin__}\refbimodindex{__builtin__},
\module{__main__}\refbimodindex{__main__} and 
\module{sys}\refbimodindex{sys}.  It also initializes the module
search path (\code{sys.path}).%
\indexiii{module}{search}{path}

\cfunction{Py_Initialize()} does not set the ``script argument list'' 
(\code{sys.argv}).  If this variable is needed by Python code that 
will be executed later, it must be set explicitly with a call to 
\code{PySys_SetArgv(\var{argc}, \var{argv})} subsequent to the call 
to \cfunction{Py_Initialize()}.

On most systems (in particular, on \UNIX{} and Windows, although the
details are slightly different), \cfunction{Py_Initialize()}
calculates the module search path based upon its best guess for the
location of the standard Python interpreter executable, assuming that
the Python library is found in a fixed location relative to the Python
interpreter executable.  In particular, it looks for a directory named
\file{lib/python1.5} (replacing \file{1.5} with the current
interpreter version) relative to the parent directory where the
executable named \file{python} is found on the shell command search
path (the environment variable \envvar{PATH}).

For instance, if the Python executable is found in
\file{/usr/local/bin/python}, it will assume that the libraries are in
\file{/usr/local/lib/python1.5}.  (In fact, this particular path
is also the ``fallback'' location, used when no executable file named
\file{python} is found along \envvar{PATH}.)  The user can override
this behavior by setting the environment variable \envvar{PYTHONHOME},
or insert additional directories in front of the standard path by
setting \envvar{PYTHONPATH}.

The embedding application can steer the search by calling 
\code{Py_SetProgramName(\var{file})} \emph{before} calling 
\cfunction{Py_Initialize()}.  Note that \envvar{PYTHONHOME} still
overrides this and \envvar{PYTHONPATH} is still inserted in front of
the standard path.  An application that requires total control has to
provide its own implementation of \cfunction{Py_GetPath()},
\cfunction{Py_GetPrefix()}, \cfunction{Py_GetExecPrefix()},
\cfunction{Py_GetProgramFullPath()} (all defined in
\file{Modules/getpath.c}).

Sometimes, it is desirable to ``uninitialize'' Python.  For instance, 
the application may want to start over (make another call to 
\cfunction{Py_Initialize()}) or the application is simply done with its 
use of Python and wants to free all memory allocated by Python.  This
can be accomplished by calling \cfunction{Py_Finalize()}.  The function
\cfunction{Py_IsInitialized()} returns true iff Python is currently in the
initialized state.  More information about these functions is given in
a later chapter.


\chapter{The Very High Level Layer}
\label{veryhigh}

The functions in this chapter will let you execute Python source code
given in a file or a buffer, but they will not let you interact in a
more detailed way with the interpreter.

\begin{cfuncdesc}{int}{PyRun_AnyFile}{FILE *fp, char *filename}
\end{cfuncdesc}

\begin{cfuncdesc}{int}{PyRun_SimpleString}{char *command}
\end{cfuncdesc}

\begin{cfuncdesc}{int}{PyRun_SimpleFile}{FILE *fp, char *filename}
\end{cfuncdesc}

\begin{cfuncdesc}{int}{PyRun_InteractiveOne}{FILE *fp, char *filename}
\end{cfuncdesc}

\begin{cfuncdesc}{int}{PyRun_InteractiveLoop}{FILE *fp, char *filename}
\end{cfuncdesc}

\begin{cfuncdesc}{struct _node*}{PyParser_SimpleParseString}{char *str,
                                                             int start}
\end{cfuncdesc}

\begin{cfuncdesc}{struct _node*}{PyParser_SimpleParseFile}{FILE *fp,
                                 char *filename, int start}
\end{cfuncdesc}

\begin{cfuncdesc}{PyObject*}{PyRun_String}{char *str, int start,
                                           PyObject *globals,
                                           PyObject *locals}
\end{cfuncdesc}

\begin{cfuncdesc}{PyObject*}{PyRun_File}{FILE *fp, char *filename,
                                         int start, PyObject *globals,
                                         PyObject *locals}
\end{cfuncdesc}

\begin{cfuncdesc}{PyObject*}{Py_CompileString}{char *str, char *filename,
                                               int start}
\end{cfuncdesc}


\chapter{Reference Counting}
\label{countingRefs}

The macros in this section are used for managing reference counts
of Python objects.

\begin{cfuncdesc}{void}{Py_INCREF}{PyObject *o}
Increment the reference count for object \var{o}.  The object must
not be \NULL{}; if you aren't sure that it isn't \NULL{}, use
\cfunction{Py_XINCREF()}.
\end{cfuncdesc}

\begin{cfuncdesc}{void}{Py_XINCREF}{PyObject *o}
Increment the reference count for object \var{o}.  The object may be
\NULL{}, in which case the macro has no effect.
\end{cfuncdesc}

\begin{cfuncdesc}{void}{Py_DECREF}{PyObject *o}
Decrement the reference count for object \var{o}.  The object must
not be \NULL{}; if you aren't sure that it isn't \NULL{}, use
\cfunction{Py_XDECREF()}.  If the reference count reaches zero, the
object's type's deallocation function (which must not be \NULL{}) is
invoked.

\strong{Warning:} The deallocation function can cause arbitrary Python
code to be invoked (e.g. when a class instance with a \method{__del__()}
method is deallocated).  While exceptions in such code are not
propagated, the executed code has free access to all Python global
variables.  This means that any object that is reachable from a global
variable should be in a consistent state before \cfunction{Py_DECREF()} is
invoked.  For example, code to delete an object from a list should
copy a reference to the deleted object in a temporary variable, update
the list data structure, and then call \cfunction{Py_DECREF()} for the
temporary variable.
\end{cfuncdesc}

\begin{cfuncdesc}{void}{Py_XDECREF}{PyObject *o}
Decrement the reference count for object \var{o}.  The object may be
\NULL{}, in which case the macro has no effect; otherwise the effect
is the same as for \cfunction{Py_DECREF()}, and the same warning
applies.
\end{cfuncdesc}

The following functions or macros are only for internal use:
\cfunction{_Py_Dealloc()}, \cfunction{_Py_ForgetReference()},
\cfunction{_Py_NewReference()}, as well as the global variable
\cdata{_Py_RefTotal}.

XXX Should mention Py_Malloc(), Py_Realloc(), Py_Free(),
PyMem_Malloc(), PyMem_Realloc(), PyMem_Free(), PyMem_NEW(),
PyMem_RESIZE(), PyMem_DEL(), PyMem_XDEL().


\chapter{Exception Handling}
\label{exceptionHandling}

The functions in this chapter will let you handle and raise Python
exceptions.  It is important to understand some of the basics of
Python exception handling.  It works somewhat like the \UNIX{}
\cdata{errno} variable: there is a global indicator (per thread) of the
last error that occurred.  Most functions don't clear this on success,
but will set it to indicate the cause of the error on failure.  Most
functions also return an error indicator, usually \NULL{} if they are
supposed to return a pointer, or \code{-1} if they return an integer
(exception: the \cfunction{PyArg_Parse*()} functions return \code{1} for
success and \code{0} for failure).  When a function must fail because
some function it called failed, it generally doesn't set the error
indicator; the function it called already set it.

The error indicator consists of three Python objects corresponding to
the Python variables \code{sys.exc_type}, \code{sys.exc_value} and
\code{sys.exc_traceback}.  API functions exist to interact with the
error indicator in various ways.  There is a separate error indicator
for each thread.

% XXX Order of these should be more thoughtful.
% Either alphabetical or some kind of structure.

\begin{cfuncdesc}{void}{PyErr_Print}{}
Print a standard traceback to \code{sys.stderr} and clear the error
indicator.  Call this function only when the error indicator is set.
(Otherwise it will cause a fatal error!)
\end{cfuncdesc}

\begin{cfuncdesc}{PyObject*}{PyErr_Occurred}{}
Test whether the error indicator is set.  If set, return the exception
\emph{type} (the first argument to the last call to one of the
\cfunction{PyErr_Set*()} functions or to \cfunction{PyErr_Restore()}).  If
not set, return \NULL{}.  You do not own a reference to the return
value, so you do not need to \cfunction{Py_DECREF()} it.
\strong{Note:} do not compare the return value to a specific
exception; use \cfunction{PyErr_ExceptionMatches()} instead, shown
below.
\end{cfuncdesc}

\begin{cfuncdesc}{int}{PyErr_ExceptionMatches}{PyObject *exc}
Equivalent to
\samp{PyErr_GivenExceptionMatches(PyErr_Occurred(), \var{exc})}.
This should only be called when an exception is actually set.
\end{cfuncdesc}

\begin{cfuncdesc}{int}{PyErr_GivenExceptionMatches}{PyObject *given, PyObject *exc}
Return true if the \var{given} exception matches the exception in
\var{exc}.  If \var{exc} is a class object, this also returns true
when \var{given} is a subclass.  If \var{exc} is a tuple, all
exceptions in the tuple (and recursively in subtuples) are searched
for a match.  This should only be called when an exception is actually
set.
\end{cfuncdesc}

\begin{cfuncdesc}{void}{PyErr_NormalizeException}{PyObject**exc, PyObject**val, PyObject**tb}
Under certain circumstances, the values returned by
\cfunction{PyErr_Fetch()} below can be ``unnormalized'', meaning that
\code{*\var{exc}} is a class object but \code{*\var{val}} is not an
instance of the  same class.  This function can be used to instantiate
the class in that case.  If the values are already normalized, nothing
happens.
\end{cfuncdesc}

\begin{cfuncdesc}{void}{PyErr_Clear}{}
Clear the error indicator.  If the error indicator is not set, there
is no effect.
\end{cfuncdesc}

\begin{cfuncdesc}{void}{PyErr_Fetch}{PyObject **ptype, PyObject **pvalue, PyObject **ptraceback}
Retrieve the error indicator into three variables whose addresses are
passed.  If the error indicator is not set, set all three variables to
\NULL{}.  If it is set, it will be cleared and you own a reference to
each object retrieved.  The value and traceback object may be \NULL{}
even when the type object is not.  \strong{Note:} this function is
normally only used by code that needs to handle exceptions or by code
that needs to save and restore the error indicator temporarily.
\end{cfuncdesc}

\begin{cfuncdesc}{void}{PyErr_Restore}{PyObject *type, PyObject *value, PyObject *traceback}
Set  the error indicator from the three objects.  If the error
indicator is already set, it is cleared first.  If the objects are
\NULL{}, the error indicator is cleared.  Do not pass a \NULL{} type
and non-\NULL{} value or traceback.  The exception type should be a
string or class; if it is a class, the value should be an instance of
that class.  Do not pass an invalid exception type or value.
(Violating these rules will cause subtle problems later.)  This call
takes away a reference to each object, i.e. you must own a reference
to each object before the call and after the call you no longer own
these references.  (If you don't understand this, don't use this
function.  I warned you.)  \strong{Note:} this function is normally
only used by code that needs to save and restore the error indicator
temporarily.
\end{cfuncdesc}

\begin{cfuncdesc}{void}{PyErr_SetString}{PyObject *type, char *message}
This is the most common way to set the error indicator.  The first
argument specifies the exception type; it is normally one of the
standard exceptions, e.g. \cdata{PyExc_RuntimeError}.  You need not
increment its reference count.  The second argument is an error
message; it is converted to a string object.
\end{cfuncdesc}

\begin{cfuncdesc}{void}{PyErr_SetObject}{PyObject *type, PyObject *value}
This function is similar to \cfunction{PyErr_SetString()} but lets you
specify an arbitrary Python object for the ``value'' of the exception.
You need not increment its reference count.
\end{cfuncdesc}

\begin{cfuncdesc}{void}{PyErr_SetNone}{PyObject *type}
This is a shorthand for \samp{PyErr_SetObject(\var{type}, Py_None)}.
\end{cfuncdesc}

\begin{cfuncdesc}{int}{PyErr_BadArgument}{}
This is a shorthand for \samp{PyErr_SetString(PyExc_TypeError,
\var{message})}, where \var{message} indicates that a built-in operation
was invoked with an illegal argument.  It is mostly for internal use.
\end{cfuncdesc}

\begin{cfuncdesc}{PyObject*}{PyErr_NoMemory}{}
This is a shorthand for \samp{PyErr_SetNone(PyExc_MemoryError)}; it
returns \NULL{} so an object allocation function can write
\samp{return PyErr_NoMemory();} when it runs out of memory.
\end{cfuncdesc}

\begin{cfuncdesc}{PyObject*}{PyErr_SetFromErrno}{PyObject *type}
This is a convenience function to raise an exception when a \C{} library
function has returned an error and set the \C{} variable \cdata{errno}.
It constructs a tuple object whose first item is the integer
\cdata{errno} value and whose second item is the corresponding error
message (gotten from \cfunction{strerror()}), and then calls
\samp{PyErr_SetObject(\var{type}, \var{object})}.  On \UNIX{}, when
the \cdata{errno} value is \constant{EINTR}, indicating an interrupted
system call, this calls \cfunction{PyErr_CheckSignals()}, and if that set
the error indicator, leaves it set to that.  The function always
returns \NULL{}, so a wrapper function around a system call can write 
\samp{return PyErr_SetFromErrno();} when  the system call returns an
error.
\end{cfuncdesc}

\begin{cfuncdesc}{void}{PyErr_BadInternalCall}{}
This is a shorthand for \samp{PyErr_SetString(PyExc_TypeError,
\var{message})}, where \var{message} indicates that an internal
operation (e.g. a Python/C API function) was invoked with an illegal
argument.  It is mostly for internal use.
\end{cfuncdesc}

\begin{cfuncdesc}{int}{PyErr_CheckSignals}{}
This function interacts with Python's signal handling.  It checks
whether a signal has been sent to the processes and if so, invokes the
corresponding signal handler.  If the
\module{signal}\refbimodindex{signal} module is supported, this can
invoke a signal handler written in Python.  In all cases, the default
effect for \constant{SIGINT} is to raise the
\exception{KeyboadInterrupt} exception.  If an exception is raised the 
error indicator is set and the function returns \code{1}; otherwise
the function returns \code{0}.  The error indicator may or may not be
cleared if it was previously set.
\end{cfuncdesc}

\begin{cfuncdesc}{void}{PyErr_SetInterrupt}{}
This function is obsolete (XXX or platform dependent?).  It simulates
the effect of a \constant{SIGINT} signal arriving --- the next time
\cfunction{PyErr_CheckSignals()} is called,
\exception{KeyboadInterrupt} will be raised.
\end{cfuncdesc}

\begin{cfuncdesc}{PyObject*}{PyErr_NewException}{char *name,
                                                 PyObject *base,
                                                 PyObject *dict}
This utility function creates and returns a new exception object.  The
\var{name} argument must be the name of the new exception, a \C{} string
of the form \code{module.class}.  The \var{base} and \var{dict}
arguments are normally \NULL{}.  Normally, this creates a class
object derived from the root for all exceptions, the built-in name
\exception{Exception} (accessible in \C{} as \cdata{PyExc_Exception}).
In this case the \member{__module__} attribute of the new class is set to the
first part (up to the last dot) of the \var{name} argument, and the
class name is set to the last part (after the last dot).  When the
user has specified the \code{-X} command line option to use string
exceptions, for backward compatibility, or when the \var{base}
argument is not a class object (and not \NULL{}), a string object
created from the entire \var{name} argument is returned.  The
\var{base} argument can be used to specify an alternate base class.
The \var{dict} argument can be used to specify a dictionary of class
variables and methods.
\end{cfuncdesc}


\section{Standard Exceptions}
\label{standardExceptions}

All standard Python exceptions are available as global variables whose
names are \samp{PyExc_} followed by the Python exception name.
These have the type \ctype{PyObject *}; they are all either class
objects or string objects, depending on the use of the \code{-X}
option to the interpreter.  For completeness, here are all the
variables:
\cdata{PyExc_Exception},
\cdata{PyExc_StandardError},
\cdata{PyExc_ArithmeticError},
\cdata{PyExc_LookupError},
\cdata{PyExc_AssertionError},
\cdata{PyExc_AttributeError},
\cdata{PyExc_EOFError},
\cdata{PyExc_FloatingPointError},
\cdata{PyExc_IOError},
\cdata{PyExc_ImportError},
\cdata{PyExc_IndexError},
\cdata{PyExc_KeyError},
\cdata{PyExc_KeyboardInterrupt},
\cdata{PyExc_MemoryError},
\cdata{PyExc_NameError},
\cdata{PyExc_OverflowError},
\cdata{PyExc_RuntimeError},
\cdata{PyExc_SyntaxError},
\cdata{PyExc_SystemError},
\cdata{PyExc_SystemExit},
\cdata{PyExc_TypeError},
\cdata{PyExc_ValueError},
\cdata{PyExc_ZeroDivisionError}.


\chapter{Utilities}
\label{utilities}

The functions in this chapter perform various utility tasks, such as
parsing function arguments and constructing Python values from \C{}
values.

\section{OS Utilities}
\label{os}

\begin{cfuncdesc}{int}{Py_FdIsInteractive}{FILE *fp, char *filename}
Return true (nonzero) if the standard I/O file \var{fp} with name
\var{filename} is deemed interactive.  This is the case for files for
which \samp{isatty(fileno(\var{fp}))} is true.  If the global flag
\cdata{Py_InteractiveFlag} is true, this function also returns true if
the \var{name} pointer is \NULL{} or if the name is equal to one of
the strings \code{"<stdin>"} or \code{"???"}.
\end{cfuncdesc}

\begin{cfuncdesc}{long}{PyOS_GetLastModificationTime}{char *filename}
Return the time of last modification of the file \var{filename}.
The result is encoded in the same way as the timestamp returned by
the standard \C{} library function \cfunction{time()}.
\end{cfuncdesc}


\section{Process Control}
\label{processControl}

\begin{cfuncdesc}{void}{Py_FatalError}{char *message}
Print a fatal error message and kill the process.  No cleanup is
performed.  This function should only be invoked when a condition is
detected that would make it dangerous to continue using the Python
interpreter; e.g., when the object administration appears to be
corrupted.  On \UNIX{}, the standard \C{} library function
\cfunction{abort()} is called which will attempt to produce a
\file{core} file.
\end{cfuncdesc}

\begin{cfuncdesc}{void}{Py_Exit}{int status}
Exit the current process.  This calls \cfunction{Py_Finalize()} and
then calls the standard \C{} library function
\code{exit(\var{status})}.
\end{cfuncdesc}

\begin{cfuncdesc}{int}{Py_AtExit}{void (*func) ()}
Register a cleanup function to be called by \cfunction{Py_Finalize()}.
The cleanup function will be called with no arguments and should
return no value.  At most 32 cleanup functions can be registered.
When the registration is successful, \cfunction{Py_AtExit()} returns
\code{0}; on failure, it returns \code{-1}.  The cleanup function
registered last is called first.  Each cleanup function will be called
at most once.  Since Python's internal finallization will have
completed before the cleanup function, no Python APIs should be called
by \var{func}.
\end{cfuncdesc}


\section{Importing Modules}
\label{importing}

\begin{cfuncdesc}{PyObject*}{PyImport_ImportModule}{char *name}
This is a simplified interface to \cfunction{PyImport_ImportModuleEx()}
below, leaving the \var{globals} and \var{locals} arguments set to
\NULL{}.  When the \var{name} argument contains a dot (i.e., when
it specifies a submodule of a package), the \var{fromlist} argument is
set to the list \code{['*']} so that the return value is the named
module rather than the top-level package containing it as would
otherwise be the case.  (Unfortunately, this has an additional side
effect when \var{name} in fact specifies a subpackage instead of a
submodule: the submodules specified in the package's \code{__all__}
variable are loaded.)  Return a new reference to the imported module,
or \NULL{} with an exception set on failure (the module may still
be created in this case --- examine \code{sys.modules} to find out).
\end{cfuncdesc}

\begin{cfuncdesc}{PyObject*}{PyImport_ImportModuleEx}{char *name, PyObject *globals, PyObject *locals, PyObject *fromlist}
Import a module.  This is best described by referring to the built-in
Python function \function{__import__()}\bifuncindex{__import__}, as
the standard \function{__import__()} function calls this function
directly.

The return value is a new reference to the imported module or
top-level package, or \NULL{} with an exception set on failure
(the module may still be created in this case).  Like for
\function{__import__()}, the return value when a submodule of a
package was requested is normally the top-level package, unless a
non-empty \var{fromlist} was given.
\end{cfuncdesc}

\begin{cfuncdesc}{PyObject*}{PyImport_Import}{PyObject *name}
This is a higher-level interface that calls the current ``import hook
function''.  It invokes the \function{__import__()} function from the
\code{__builtins__} of the current globals.  This means that the
import is done using whatever import hooks are installed in the
current environment, e.g. by \module{rexec}\refstmodindex{rexec} or
\module{ihooks}\refstmodindex{ihooks}.
\end{cfuncdesc}

\begin{cfuncdesc}{PyObject*}{PyImport_ReloadModule}{PyObject *m}
Reload a module.  This is best described by referring to the built-in
Python function \function{reload()}\bifuncindex{reload}, as the standard
\function{reload()} function calls this function directly.  Return a
new reference to the reloaded module, or \NULL{} with an exception set
on failure (the module still exists in this case).
\end{cfuncdesc}

\begin{cfuncdesc}{PyObject*}{PyImport_AddModule}{char *name}
Return the module object corresponding to a module name.  The
\var{name} argument may be of the form \code{package.module}).  First
check the modules dictionary if there's one there, and if not, create
a new one and insert in in the modules dictionary.  Because the former
action is most common, this does not return a new reference, and you
do not own the returned reference.  Return \NULL{} with an
exception set on failure.  \strong{Note:} this function returns
a ``borrowed'' reference.  
\end{cfuncdesc}

\begin{cfuncdesc}{PyObject*}{PyImport_ExecCodeModule}{char *name, PyObject *co}
Given a module name (possibly of the form \code{package.module}) and a
code object read from a Python bytecode file or obtained from the
built-in function \function{compile()}\bifuncindex{compile}, load the
module.  Return a new reference to the module object, or \NULL{} with
an exception set if an error occurred (the module may still be created
in this case).  (This function would reload the module if it was
already imported.)
\end{cfuncdesc}

\begin{cfuncdesc}{long}{PyImport_GetMagicNumber}{}
Return the magic number for Python bytecode files (a.k.a. \file{.pyc}
and \file{.pyo} files).  The magic number should be present in the
first four bytes of the bytecode file, in little-endian byte order.
\end{cfuncdesc}

\begin{cfuncdesc}{PyObject*}{PyImport_GetModuleDict}{}
Return the dictionary used for the module administration
(a.k.a. \code{sys.modules}).  Note that this is a per-interpreter
variable.
\end{cfuncdesc}

\begin{cfuncdesc}{void}{_PyImport_Init}{}
Initialize the import mechanism.  For internal use only.
\end{cfuncdesc}

\begin{cfuncdesc}{void}{PyImport_Cleanup}{}
Empty the module table.  For internal use only.
\end{cfuncdesc}

\begin{cfuncdesc}{void}{_PyImport_Fini}{}
Finalize the import mechanism.  For internal use only.
\end{cfuncdesc}

\begin{cfuncdesc}{PyObject*}{_PyImport_FindExtension}{char *, char *}
For internal use only.
\end{cfuncdesc}

\begin{cfuncdesc}{PyObject*}{_PyImport_FixupExtension}{char *, char *}
For internal use only.
\end{cfuncdesc}

\begin{cfuncdesc}{int}{PyImport_ImportFrozenModule}{char *}
Load a frozen module.  Return \code{1} for success, \code{0} if the
module is not found, and \code{-1} with an exception set if the
initialization failed.  To access the imported module on a successful
load, use \cfunction{PyImport_ImportModule()}.
(Note the misnomer --- this function would reload the module if it was
already imported.)
\end{cfuncdesc}

\begin{ctypedesc}{struct _frozen}
This is the structure type definition for frozen module descriptors,
as generated by the \program{freeze}\index{freeze utility} utility
(see \file{Tools/freeze/} in the Python source distribution).  Its
definition is:

\begin{verbatim}
struct _frozen {
    char *name;
    unsigned char *code;
    int size;
};
\end{verbatim}
\end{ctypedesc}

\begin{cvardesc}{struct _frozen*}{PyImport_FrozenModules}
This pointer is initialized to point to an array of \ctype{struct
_frozen} records, terminated by one whose members are all \NULL{}
or zero.  When a frozen module is imported, it is searched in this
table.  Third-party code could play tricks with this to provide a
dynamically created collection of frozen modules.
\end{cvardesc}


\chapter{Abstract Objects Layer}
\label{abstract}

The functions in this chapter interact with Python objects regardless
of their type, or with wide classes of object types (e.g. all
numerical types, or all sequence types).  When used on object types
for which they do not apply, they will flag a Python exception.

\section{Object Protocol}
\label{object}

\begin{cfuncdesc}{int}{PyObject_Print}{PyObject *o, FILE *fp, int flags}
Print an object \var{o}, on file \var{fp}.  Returns \code{-1} on error
The flags argument is used to enable certain printing
options. The only option currently supported is
\constant{Py_PRINT_RAW}.
\end{cfuncdesc}

\begin{cfuncdesc}{int}{PyObject_HasAttrString}{PyObject *o, char *attr_name}
Returns \code{1} if \var{o} has the attribute \var{attr_name}, and
\code{0} otherwise.  This is equivalent to the Python expression
\samp{hasattr(\var{o}, \var{attr_name})}.
This function always succeeds.
\end{cfuncdesc}

\begin{cfuncdesc}{PyObject*}{PyObject_GetAttrString}{PyObject *o, char *attr_name}
Retrieve an attribute named \var{attr_name} from object \var{o}.
Returns the attribute value on success, or \NULL{} on failure.
This is the equivalent of the Python expression
\samp{\var{o}.\var{attr_name}}.
\end{cfuncdesc}


\begin{cfuncdesc}{int}{PyObject_HasAttr}{PyObject *o, PyObject *attr_name}
Returns \code{1} if \var{o} has the attribute \var{attr_name}, and
\code{0} otherwise.  This is equivalent to the Python expression
\samp{hasattr(\var{o}, \var{attr_name})}. 
This function always succeeds.
\end{cfuncdesc}


\begin{cfuncdesc}{PyObject*}{PyObject_GetAttr}{PyObject *o, PyObject *attr_name}
Retrieve an attribute named \var{attr_name} from object \var{o}.
Returns the attribute value on success, or \NULL{} on failure.
This is the equivalent of the Python expression
\samp{\var{o}.\var{attr_name}}.
\end{cfuncdesc}


\begin{cfuncdesc}{int}{PyObject_SetAttrString}{PyObject *o, char *attr_name, PyObject *v}
Set the value of the attribute named \var{attr_name}, for object
\var{o}, to the value \var{v}. Returns \code{-1} on failure.  This is
the equivalent of the Python statement \samp{\var{o}.\var{attr_name} =
\var{v}}.
\end{cfuncdesc}


\begin{cfuncdesc}{int}{PyObject_SetAttr}{PyObject *o, PyObject *attr_name, PyObject *v}
Set the value of the attribute named \var{attr_name}, for
object \var{o},
to the value \var{v}. Returns \code{-1} on failure.  This is
the equivalent of the Python statement \samp{\var{o}.\var{attr_name} =
\var{v}}.
\end{cfuncdesc}


\begin{cfuncdesc}{int}{PyObject_DelAttrString}{PyObject *o, char *attr_name}
Delete attribute named \var{attr_name}, for object \var{o}. Returns
\code{-1} on failure.  This is the equivalent of the Python
statement: \samp{del \var{o}.\var{attr_name}}.
\end{cfuncdesc}


\begin{cfuncdesc}{int}{PyObject_DelAttr}{PyObject *o, PyObject *attr_name}
Delete attribute named \var{attr_name}, for object \var{o}. Returns
\code{-1} on failure.  This is the equivalent of the Python
statement \samp{del \var{o}.\var{attr_name}}.
\end{cfuncdesc}


\begin{cfuncdesc}{int}{PyObject_Cmp}{PyObject *o1, PyObject *o2, int *result}
Compare the values of \var{o1} and \var{o2} using a routine provided
by \var{o1}, if one exists, otherwise with a routine provided by
\var{o2}.  The result of the comparison is returned in \var{result}.
Returns \code{-1} on failure.  This is the equivalent of the Python
statement \samp{\var{result} = cmp(\var{o1}, \var{o2})}.
\end{cfuncdesc}


\begin{cfuncdesc}{int}{PyObject_Compare}{PyObject *o1, PyObject *o2}
Compare the values of \var{o1} and \var{o2} using a routine provided
by \var{o1}, if one exists, otherwise with a routine provided by
\var{o2}.  Returns the result of the comparison on success.  On error,
the value returned is undefined; use \cfunction{PyErr_Occurred()} to
detect an error.  This is equivalent to the
Python expression \samp{cmp(\var{o1}, \var{o2})}.
\end{cfuncdesc}


\begin{cfuncdesc}{PyObject*}{PyObject_Repr}{PyObject *o}
Compute the string representation of object, \var{o}.  Returns the
string representation on success, \NULL{} on failure.  This is
the equivalent of the Python expression \samp{repr(\var{o})}.
Called by the \function{repr()}\bifuncindex{repr} built-in function
and by reverse quotes.
\end{cfuncdesc}


\begin{cfuncdesc}{PyObject*}{PyObject_Str}{PyObject *o}
Compute the string representation of object \var{o}.  Returns the
string representation on success, \NULL{} on failure.  This is
the equivalent of the Python expression \samp{str(\var{o})}.
Called by the \function{str()}\bifuncindex{str} built-in function and
by the \keyword{print} statement.
\end{cfuncdesc}


\begin{cfuncdesc}{int}{PyCallable_Check}{PyObject *o}
Determine if the object \var{o}, is callable.  Return \code{1} if the
object is callable and \code{0} otherwise.
This function always succeeds.
\end{cfuncdesc}


\begin{cfuncdesc}{PyObject*}{PyObject_CallObject}{PyObject *callable_object, PyObject *args}
Call a callable Python object \var{callable_object}, with
arguments given by the tuple \var{args}.  If no arguments are
needed, then args may be \NULL{}.  Returns the result of the
call on success, or \NULL{} on failure.  This is the equivalent
of the Python expression \samp{apply(\var{o}, \var{args})}.
\end{cfuncdesc}

\begin{cfuncdesc}{PyObject*}{PyObject_CallFunction}{PyObject *callable_object, char *format, ...}
Call a callable Python object \var{callable_object}, with a
variable number of \C{} arguments. The \C{} arguments are described
using a \cfunction{Py_BuildValue()} style format string. The format may
be \NULL{}, indicating that no arguments are provided.  Returns the
result of the call on success, or \NULL{} on failure.  This is
the equivalent of the Python expression \samp{apply(\var{o},
\var{args})}.
\end{cfuncdesc}


\begin{cfuncdesc}{PyObject*}{PyObject_CallMethod}{PyObject *o, char *m, char *format, ...}
Call the method named \var{m} of object \var{o} with a variable number
of C arguments.  The \C{} arguments are described by a
\cfunction{Py_BuildValue()} format string.  The format may be \NULL{},
indicating that no arguments are provided. Returns the result of the
call on success, or \NULL{} on failure.  This is the equivalent of the
Python expression \samp{\var{o}.\var{method}(\var{args})}.
Note that Special method names, such as \method{__add__()},
\method{__getitem__()}, and so on are not supported. The specific
abstract-object routines for these must be used.
\end{cfuncdesc}


\begin{cfuncdesc}{int}{PyObject_Hash}{PyObject *o}
Compute and return the hash value of an object \var{o}.  On
failure, return \code{-1}.  This is the equivalent of the Python
expression \samp{hash(\var{o})}.
\end{cfuncdesc}


\begin{cfuncdesc}{int}{PyObject_IsTrue}{PyObject *o}
Returns \code{1} if the object \var{o} is considered to be true, and
\code{0} otherwise. This is equivalent to the Python expression
\samp{not not \var{o}}.
This function always succeeds.
\end{cfuncdesc}


\begin{cfuncdesc}{PyObject*}{PyObject_Type}{PyObject *o}
On success, returns a type object corresponding to the object
type of object \var{o}. On failure, returns \NULL{}.  This is
equivalent to the Python expression \samp{type(\var{o})}.
\bifuncindex{type}
\end{cfuncdesc}

\begin{cfuncdesc}{int}{PyObject_Length}{PyObject *o}
Return the length of object \var{o}.  If the object \var{o} provides
both sequence and mapping protocols, the sequence length is
returned. On error, \code{-1} is returned.  This is the equivalent
to the Python expression \samp{len(\var{o})}.
\end{cfuncdesc}


\begin{cfuncdesc}{PyObject*}{PyObject_GetItem}{PyObject *o, PyObject *key}
Return element of \var{o} corresponding to the object \var{key} or
\NULL{} on failure. This is the equivalent of the Python expression
\samp{\var{o}[\var{key}]}.
\end{cfuncdesc}


\begin{cfuncdesc}{int}{PyObject_SetItem}{PyObject *o, PyObject *key, PyObject *v}
Map the object \var{key} to the value \var{v}.
Returns \code{-1} on failure.  This is the equivalent
of the Python statement \samp{\var{o}[\var{key}] = \var{v}}.
\end{cfuncdesc}


\begin{cfuncdesc}{int}{PyObject_DelItem}{PyObject *o, PyObject *key, PyObject *v}
Delete the mapping for \var{key} from \var{o}.  Returns \code{-1} on
failure. This is the equivalent of the Python statement \samp{del
\var{o}[\var{key}]}.
\end{cfuncdesc}


\section{Number Protocol}
\label{number}

\begin{cfuncdesc}{int}{PyNumber_Check}{PyObject *o}
Returns \code{1} if the object \var{o} provides numeric protocols, and
false otherwise. 
This function always succeeds.
\end{cfuncdesc}


\begin{cfuncdesc}{PyObject*}{PyNumber_Add}{PyObject *o1, PyObject *o2}
Returns the result of adding \var{o1} and \var{o2}, or \NULL{} on
failure.  This is the equivalent of the Python expression
\samp{\var{o1} + \var{o2}}.
\end{cfuncdesc}


\begin{cfuncdesc}{PyObject*}{PyNumber_Subtract}{PyObject *o1, PyObject *o2}
Returns the result of subtracting \var{o2} from \var{o1}, or \NULL{}
on failure.  This is the equivalent of the Python expression
\samp{\var{o1} - \var{o2}}.
\end{cfuncdesc}


\begin{cfuncdesc}{PyObject*}{PyNumber_Multiply}{PyObject *o1, PyObject *o2}
Returns the result of multiplying \var{o1} and \var{o2}, or \NULL{} on
failure.  This is the equivalent of the Python expression
\samp{\var{o1} * \var{o2}}.
\end{cfuncdesc}


\begin{cfuncdesc}{PyObject*}{PyNumber_Divide}{PyObject *o1, PyObject *o2}
Returns the result of dividing \var{o1} by \var{o2}, or \NULL{} on
failure. 
This is the equivalent of the Python expression \samp{\var{o1} /
\var{o2}}.
\end{cfuncdesc}


\begin{cfuncdesc}{PyObject*}{PyNumber_Remainder}{PyObject *o1, PyObject *o2}
Returns the remainder of dividing \var{o1} by \var{o2}, or \NULL{} on
failure.  This is the equivalent of the Python expression
\samp{\var{o1} \% \var{o2}}.
\end{cfuncdesc}


\begin{cfuncdesc}{PyObject*}{PyNumber_Divmod}{PyObject *o1, PyObject *o2}
See the built-in function \function{divmod()}\bifuncindex{divmod}.
Returns \NULL{} on failure.  This is the equivalent of the Python
expression \samp{divmod(\var{o1}, \var{o2})}.
\end{cfuncdesc}


\begin{cfuncdesc}{PyObject*}{PyNumber_Power}{PyObject *o1, PyObject *o2, PyObject *o3}
See the built-in function \function{pow()}\bifuncindex{pow}.  Returns
\NULL{} on failure. This is the equivalent of the Python expression
\samp{pow(\var{o1}, \var{o2}, \var{o3})}, where \var{o3} is optional.
If \var{o3} is to be ignored, pass \cdata{Py_None} in its place.
\end{cfuncdesc}


\begin{cfuncdesc}{PyObject*}{PyNumber_Negative}{PyObject *o}
Returns the negation of \var{o} on success, or \NULL{} on failure.
This is the equivalent of the Python expression \samp{-\var{o}}.
\end{cfuncdesc}


\begin{cfuncdesc}{PyObject*}{PyNumber_Positive}{PyObject *o}
Returns \var{o} on success, or \NULL{} on failure.
This is the equivalent of the Python expression \samp{+\var{o}}.
\end{cfuncdesc}


\begin{cfuncdesc}{PyObject*}{PyNumber_Absolute}{PyObject *o}
Returns the absolute value of \var{o}, or \NULL{} on failure.  This is
the equivalent of the Python expression \samp{abs(\var{o})}.
\end{cfuncdesc}


\begin{cfuncdesc}{PyObject*}{PyNumber_Invert}{PyObject *o}
Returns the bitwise negation of \var{o} on success, or \NULL{} on
failure.  This is the equivalent of the Python expression
\samp{\~\var{o}}.
\end{cfuncdesc}


\begin{cfuncdesc}{PyObject*}{PyNumber_Lshift}{PyObject *o1, PyObject *o2}
Returns the result of left shifting \var{o1} by \var{o2} on success,
or \NULL{} on failure.  This is the equivalent of the Python
expression \samp{\var{o1} << \var{o2}}.
\end{cfuncdesc}


\begin{cfuncdesc}{PyObject*}{PyNumber_Rshift}{PyObject *o1, PyObject *o2}
Returns the result of right shifting \var{o1} by \var{o2} on success,
or \NULL{} on failure.  This is the equivalent of the Python
expression \samp{\var{o1} >> \var{o2}}.
\end{cfuncdesc}


\begin{cfuncdesc}{PyObject*}{PyNumber_And}{PyObject *o1, PyObject *o2}
Returns the result of ``anding'' \var{o2} and \var{o2} on success and
\NULL{} on failure. This is the equivalent of the Python
expression \samp{\var{o1} and \var{o2}}.
\end{cfuncdesc}


\begin{cfuncdesc}{PyObject*}{PyNumber_Xor}{PyObject *o1, PyObject *o2}
Returns the bitwise exclusive or of \var{o1} by \var{o2} on success,
or \NULL{} on failure.  This is the equivalent of the Python
expression \samp{\var{o1} \^{ }\var{o2}}.
\end{cfuncdesc}

\begin{cfuncdesc}{PyObject*}{PyNumber_Or}{PyObject *o1, PyObject *o2}
Returns the result of \var{o1} and \var{o2} on success, or \NULL{} on
failure.  This is the equivalent of the Python expression
\samp{\var{o1} or \var{o2}}.
\end{cfuncdesc}


\begin{cfuncdesc}{PyObject*}{PyNumber_Coerce}{PyObject **p1, PyObject **p2}
This function takes the addresses of two variables of type
\ctype{PyObject*}.

If the objects pointed to by \code{*\var{p1}} and \code{*\var{p2}}
have the same type, increment their reference count and return
\code{0} (success). If the objects can be converted to a common
numeric type, replace \code{*p1} and \code{*p2} by their converted
value (with 'new' reference counts), and return \code{0}.
If no conversion is possible, or if some other error occurs,
return \code{-1} (failure) and don't increment the reference counts.
The call \code{PyNumber_Coerce(\&o1, \&o2)} is equivalent to the
Python statement \samp{\var{o1}, \var{o2} = coerce(\var{o1},
\var{o2})}.
\end{cfuncdesc}


\begin{cfuncdesc}{PyObject*}{PyNumber_Int}{PyObject *o}
Returns the \var{o} converted to an integer object on success, or
\NULL{} on failure.  This is the equivalent of the Python
expression \samp{int(\var{o})}.
\end{cfuncdesc}


\begin{cfuncdesc}{PyObject*}{PyNumber_Long}{PyObject *o}
Returns the \var{o} converted to a long integer object on success,
or \NULL{} on failure.  This is the equivalent of the Python
expression \samp{long(\var{o})}.
\end{cfuncdesc}


\begin{cfuncdesc}{PyObject*}{PyNumber_Float}{PyObject *o}
Returns the \var{o} converted to a float object on success, or \NULL{}
on failure.  This is the equivalent of the Python expression
\samp{float(\var{o})}.
\end{cfuncdesc}


\section{Sequence Protocol}
\label{sequence}

\begin{cfuncdesc}{int}{PySequence_Check}{PyObject *o}
Return \code{1} if the object provides sequence protocol, and \code{0}
otherwise.  
This function always succeeds.
\end{cfuncdesc}


\begin{cfuncdesc}{PyObject*}{PySequence_Concat}{PyObject *o1, PyObject *o2}
Return the concatenation of \var{o1} and \var{o2} on success, and \NULL{} on
failure.   This is the equivalent of the Python
expression \samp{\var{o1} + \var{o2}}.
\end{cfuncdesc}


\begin{cfuncdesc}{PyObject*}{PySequence_Repeat}{PyObject *o, int count}
Return the result of repeating sequence object \var{o} \var{count}
times, or \NULL{} on failure.  This is the equivalent of the Python
expression \samp{\var{o} * \var{count}}.
\end{cfuncdesc}


\begin{cfuncdesc}{PyObject*}{PySequence_GetItem}{PyObject *o, int i}
Return the \var{i}th element of \var{o}, or \NULL{} on failure. This
is the equivalent of the Python expression \samp{\var{o}[\var{i}]}.
\end{cfuncdesc}


\begin{cfuncdesc}{PyObject*}{PySequence_GetSlice}{PyObject *o, int i1, int i2}
Return the slice of sequence object \var{o} between \var{i1} and
\var{i2}, or \NULL{} on failure. This is the equivalent of the Python
expression \samp{\var{o}[\var{i1}:\var{i2}]}.
\end{cfuncdesc}


\begin{cfuncdesc}{int}{PySequence_SetItem}{PyObject *o, int i, PyObject *v}
Assign object \var{v} to the \var{i}th element of \var{o}.
Returns \code{-1} on failure.  This is the equivalent of the Python
statement \samp{\var{o}[\var{i}] = \var{v}}.
\end{cfuncdesc}

\begin{cfuncdesc}{int}{PySequence_DelItem}{PyObject *o, int i}
Delete the \var{i}th element of object \var{v}.  Returns
\code{-1} on failure.  This is the equivalent of the Python
statement \samp{del \var{o}[\var{i}]}.
\end{cfuncdesc}

\begin{cfuncdesc}{int}{PySequence_SetSlice}{PyObject *o, int i1, int i2, PyObject *v}
Assign the sequence object \var{v} to the slice in sequence
object \var{o} from \var{i1} to \var{i2}.  This is the equivalent of
the Python statement \samp{\var{o}[\var{i1}:\var{i2}] = \var{v}}.
\end{cfuncdesc}

\begin{cfuncdesc}{int}{PySequence_DelSlice}{PyObject *o, int i1, int i2}
Delete the slice in sequence object \var{o} from \var{i1} to \var{i2}.
Returns \code{-1} on failure. This is the equivalent of the Python
statement \samp{del \var{o}[\var{i1}:\var{i2}]}.
\end{cfuncdesc}

\begin{cfuncdesc}{PyObject*}{PySequence_Tuple}{PyObject *o}
Returns the \var{o} as a tuple on success, and \NULL{} on failure.
This is equivalent to the Python expression \code{tuple(\var{o})}.
\end{cfuncdesc}

\begin{cfuncdesc}{int}{PySequence_Count}{PyObject *o, PyObject *value}
Return the number of occurrences of \var{value} in \var{o}, that is,
return the number of keys for which \code{\var{o}[\var{key}] ==
\var{value}}.  On failure, return \code{-1}.  This is equivalent to
the Python expression \samp{\var{o}.count(\var{value})}.
\end{cfuncdesc}

\begin{cfuncdesc}{int}{PySequence_In}{PyObject *o, PyObject *value}
Determine if \var{o} contains \var{value}.  If an item in \var{o} is
equal to \var{value}, return \code{1}, otherwise return \code{0}.  On
error, return \code{-1}.  This is equivalent to the Python expression
\samp{\var{value} in \var{o}}.
\end{cfuncdesc}

\begin{cfuncdesc}{int}{PySequence_Index}{PyObject *o, PyObject *value}
Return the first index \var{i} for which \code{\var{o}[\var{i}] ==
\var{value}}.  On error, return \code{-1}.    This is equivalent to
the Python expression \samp{\var{o}.index(\var{value})}.
\end{cfuncdesc}


\section{Mapping Protocol}
\label{mapping}

\begin{cfuncdesc}{int}{PyMapping_Check}{PyObject *o}
Return \code{1} if the object provides mapping protocol, and \code{0}
otherwise.  
This function always succeeds.
\end{cfuncdesc}


\begin{cfuncdesc}{int}{PyMapping_Length}{PyObject *o}
Returns the number of keys in object \var{o} on success, and \code{-1}
on failure.  For objects that do not provide sequence protocol,
this is equivalent to the Python expression \samp{len(\var{o})}.
\end{cfuncdesc}


\begin{cfuncdesc}{int}{PyMapping_DelItemString}{PyObject *o, char *key}
Remove the mapping for object \var{key} from the object \var{o}.
Return \code{-1} on failure.  This is equivalent to
the Python statement \samp{del \var{o}[\var{key}]}.
\end{cfuncdesc}


\begin{cfuncdesc}{int}{PyMapping_DelItem}{PyObject *o, PyObject *key}
Remove the mapping for object \var{key} from the object \var{o}.
Return \code{-1} on failure.  This is equivalent to
the Python statement \samp{del \var{o}[\var{key}]}.
\end{cfuncdesc}


\begin{cfuncdesc}{int}{PyMapping_HasKeyString}{PyObject *o, char *key}
On success, return \code{1} if the mapping object has the key \var{key}
and \code{0} otherwise.  This is equivalent to the Python expression
\samp{\var{o}.has_key(\var{key})}. 
This function always succeeds.
\end{cfuncdesc}


\begin{cfuncdesc}{int}{PyMapping_HasKey}{PyObject *o, PyObject *key}
Return \code{1} if the mapping object has the key \var{key} and
\code{0} otherwise.  This is equivalent to the Python expression
\samp{\var{o}.has_key(\var{key})}. 
This function always succeeds.
\end{cfuncdesc}


\begin{cfuncdesc}{PyObject*}{PyMapping_Keys}{PyObject *o}
On success, return a list of the keys in object \var{o}.  On
failure, return \NULL{}. This is equivalent to the Python
expression \samp{\var{o}.keys()}.
\end{cfuncdesc}


\begin{cfuncdesc}{PyObject*}{PyMapping_Values}{PyObject *o}
On success, return a list of the values in object \var{o}.  On
failure, return \NULL{}. This is equivalent to the Python
expression \samp{\var{o}.values()}.
\end{cfuncdesc}


\begin{cfuncdesc}{PyObject*}{PyMapping_Items}{PyObject *o}
On success, return a list of the items in object \var{o}, where
each item is a tuple containing a key-value pair.  On
failure, return \NULL{}. This is equivalent to the Python
expression \samp{\var{o}.items()}.
\end{cfuncdesc}

\begin{cfuncdesc}{int}{PyMapping_Clear}{PyObject *o}
Make object \var{o} empty.  Returns \code{1} on success and \code{0}
on failure.  This is equivalent to the Python statement
\samp{for key in \var{o}.keys(): del \var{o}[key]}.
\end{cfuncdesc}


\begin{cfuncdesc}{PyObject*}{PyMapping_GetItemString}{PyObject *o, char *key}
Return element of \var{o} corresponding to the object \var{key} or
\NULL{} on failure. This is the equivalent of the Python expression
\samp{\var{o}[\var{key}]}.
\end{cfuncdesc}

\begin{cfuncdesc}{PyObject*}{PyMapping_SetItemString}{PyObject *o, char *key, PyObject *v}
Map the object \var{key} to the value \var{v} in object \var{o}.
Returns \code{-1} on failure.  This is the equivalent of the Python
statement \samp{\var{o}[\var{key}] = \var{v}}.
\end{cfuncdesc}


\section{Constructors}

\begin{cfuncdesc}{PyObject*}{PyFile_FromString}{char *file_name, char *mode}
On success, returns a new file object that is opened on the
file given by \var{file_name}, with a file mode given by \var{mode},
where \var{mode} has the same semantics as the standard \C{} routine
\cfunction{fopen()}.  On failure, return \code{-1}.
\end{cfuncdesc}

\begin{cfuncdesc}{PyObject*}{PyFile_FromFile}{FILE *fp, char *file_name, char *mode, int close_on_del}
Return a new file object for an already opened standard \C{} file
pointer, \var{fp}.  A file name, \var{file_name}, and open mode,
\var{mode}, must be provided as well as a flag, \var{close_on_del},
that indicates whether the file is to be closed when the file object
is destroyed.  On failure, return \code{-1}.
\end{cfuncdesc}

\begin{cfuncdesc}{PyObject*}{PyFloat_FromDouble}{double v}
Returns a new float object with the value \var{v} on success, and
\NULL{} on failure.
\end{cfuncdesc}

\begin{cfuncdesc}{PyObject*}{PyInt_FromLong}{long v}
Returns a new int object with the value \var{v} on success, and
\NULL{} on failure.
\end{cfuncdesc}

\begin{cfuncdesc}{PyObject*}{PyList_New}{int len}
Returns a new list of length \var{len} on success, and \NULL{} on
failure.
\end{cfuncdesc}

\begin{cfuncdesc}{PyObject*}{PyLong_FromLong}{long v}
Returns a new long object with the value \var{v} on success, and
\NULL{} on failure.
\end{cfuncdesc}

\begin{cfuncdesc}{PyObject*}{PyLong_FromDouble}{double v}
Returns a new long object with the value \var{v} on success, and
\NULL{} on failure.
\end{cfuncdesc}

\begin{cfuncdesc}{PyObject*}{PyDict_New}{}
Returns a new empty dictionary on success, and \NULL{} on
failure.
\end{cfuncdesc}

\begin{cfuncdesc}{PyObject*}{PyString_FromString}{char *v}
Returns a new string object with the value \var{v} on success, and
\NULL{} on failure.
\end{cfuncdesc}

\begin{cfuncdesc}{PyObject*}{PyString_FromStringAndSize}{char *v, int len}
Returns a new string object with the value \var{v} and length
\var{len} on success, and \NULL{} on failure.  If \var{v} is \NULL{},
the contents of the string are uninitialized.
\end{cfuncdesc}

\begin{cfuncdesc}{PyObject*}{PyTuple_New}{int len}
Returns a new tuple of length \var{len} on success, and \NULL{} on
failure.
\end{cfuncdesc}


\chapter{Concrete Objects Layer}
\label{concrete}

The functions in this chapter are specific to certain Python object
types.  Passing them an object of the wrong type is not a good idea;
if you receive an object from a Python program and you are not sure
that it has the right type, you must perform a type check first;
e.g. to check that an object is a dictionary, use
\cfunction{PyDict_Check()}.  The chapter is structured like the
``family tree'' of Python object types.


\section{Fundamental Objects}
\label{fundamental}

This section describes Python type objects and the singleton object 
\code{None}.


\subsection{Type Objects}
\label{typeObjects}

\begin{ctypedesc}{PyTypeObject}

\end{ctypedesc}

\begin{cvardesc}{PyObject *}{PyType_Type}

\end{cvardesc}


\subsection{The None Object}
\label{noneObject}

\begin{cvardesc}{PyObject *}{Py_None}
The Python \code{None} object, denoting lack of value.  This object has
no methods.
\end{cvardesc}


\section{Sequence Objects}
\label{sequenceObjects}

Generic operations on sequence objects were discussed in the previous 
chapter; this section deals with the specific kinds of sequence 
objects that are intrinsic to the Python language.


\subsection{String Objects}
\label{stringObjects}

\begin{ctypedesc}{PyStringObject}
This subtype of \ctype{PyObject} represents a Python string object.
\end{ctypedesc}

\begin{cvardesc}{PyTypeObject}{PyString_Type}
This instance of \ctype{PyTypeObject} represents the Python string type.
\end{cvardesc}

\begin{cfuncdesc}{int}{PyString_Check}{PyObject *o}
Returns true if the object \var{o} is a string object.
\end{cfuncdesc}

\begin{cfuncdesc}{PyObject*}{PyString_FromStringAndSize}{const char *v,
                                                          int len}
Returns a new string object with the value \var{v} and length
\var{len} on success, and \NULL{} on failure.  If \var{v} is \NULL{},
the contents of the string are uninitialized.
\end{cfuncdesc}

\begin{cfuncdesc}{PyObject*}{PyString_FromString}{const char *v}
Returns a new string object with the value \var{v} on success, and
\NULL{} on failure.
\end{cfuncdesc}

\begin{cfuncdesc}{int}{PyString_Size}{PyObject *string}
Returns the length of the string in string object \var{string}.
\end{cfuncdesc}

\begin{cfuncdesc}{char*}{PyString_AsString}{PyObject *string}
Resturns a \NULL{} terminated representation of the contents of \var{string}.
\end{cfuncdesc}

\begin{cfuncdesc}{void}{PyString_Concat}{PyObject **string,
                                         PyObject *newpart}
Creates a new string object in \var{*string} containing the contents
of \var{newpart} appended to \var{string}.
\end{cfuncdesc}

\begin{cfuncdesc}{void}{PyString_ConcatAndDel}{PyObject **string,
                                               PyObject *newpart}
Creates a new string object in \var{*string} containing the contents
of \var{newpart} appended to \var{string}.  This version decrements
the reference count of \var{newpart}.
\end{cfuncdesc}

\begin{cfuncdesc}{int}{_PyString_Resize}{PyObject **string, int newsize}
A way to resize a string object even though it is ``immutable''.  
Only use this to build up a brand new string object; don't use this if
the string may already be known in other parts of the code.
\end{cfuncdesc}

\begin{cfuncdesc}{PyObject*}{PyString_Format}{PyObject *format,
                                              PyObject *args}
Returns a new string object from \var{format} and \var{args}.  Analogous
to \code{\var{format} \% \var{args}}.  The \var{args} argument must be
a tuple.
\end{cfuncdesc}

\begin{cfuncdesc}{void}{PyString_InternInPlace}{PyObject **string}
Intern the argument \var{*string} in place.  The argument must be the
address of a pointer variable pointing to a Python string object.
If there is an existing interned string that is the same as
\var{*string}, it sets \var{*string} to it (decrementing the reference 
count of the old string object and incrementing the reference count of
the interned string object), otherwise it leaves \var{*string} alone
and interns it (incrementing its reference count).  (Clarification:
even though there is a lot of talk about reference counts, think of
this function as reference-count-neutral; you own the object after
the call if and only if you owned it before the call.)
\end{cfuncdesc}

\begin{cfuncdesc}{PyObject*}{PyString_InternFromString}{const char *v}
A combination of \cfunction{PyString_FromString()} and
\cfunction{PyString_InternInPlace()}, returning either a new string object
that has been interned, or a new (``owned'') reference to an earlier
interned string object with the same value.
\end{cfuncdesc}

\begin{cfuncdesc}{char*}{PyString_AS_STRING}{PyObject *string}
Macro form of \cfunction{PyString_AsString()} but without error checking.
\end{cfuncdesc}

\begin{cfuncdesc}{int}{PyString_GET_SIZE}{PyObject *string}
Macro form of \cfunction{PyString_GetSize()} but without error checking.
\end{cfuncdesc}



\subsection{Tuple Objects}
\label{tupleObjects}

\begin{ctypedesc}{PyTupleObject}
This subtype of \ctype{PyObject} represents a Python tuple object.
\end{ctypedesc}

\begin{cvardesc}{PyTypeObject}{PyTuple_Type}
This instance of \ctype{PyTypeObject} represents the Python tuple type.
\end{cvardesc}

\begin{cfuncdesc}{int}{PyTuple_Check}{PyObject *p}
Return true if the argument is a tuple object.
\end{cfuncdesc}

\begin{cfuncdesc}{PyObject*}{PyTuple_New}{int s}
Return a new tuple object of size \var{s}.
\end{cfuncdesc}

\begin{cfuncdesc}{int}{PyTuple_Size}{PyTupleObject *p}
Takes a pointer to a tuple object, and returns the size
of that tuple.
\end{cfuncdesc}

\begin{cfuncdesc}{PyObject*}{PyTuple_GetItem}{PyTupleObject *p, int pos}
Returns the object at position \var{pos} in the tuple pointed
to by \var{p}.  If \var{pos} is out of bounds, returns \NULL{} and
sets an \exception{IndexError} exception.  \strong{Note:} this
function returns a ``borrowed'' reference.
\end{cfuncdesc}

\begin{cfuncdesc}{PyObject*}{PyTuple_GET_ITEM}{PyTupleObject *p, int pos}
Does the same, but does no checking of its arguments.
\end{cfuncdesc}

\begin{cfuncdesc}{PyObject*}{PyTuple_GetSlice}{PyTupleObject *p,
            int low,
            int high}
Takes a slice of the tuple pointed to by \var{p} from
\var{low} to \var{high} and returns it as a new tuple.
\end{cfuncdesc}

\begin{cfuncdesc}{int}{PyTuple_SetItem}{PyTupleObject *p,
            int pos,
            PyObject *o}
Inserts a reference to object \var{o} at position \var{pos} of
the tuple pointed to by \var{p}. It returns \code{0} on success.
\end{cfuncdesc}

\begin{cfuncdesc}{void}{PyTuple_SET_ITEM}{PyTupleObject *p,
            int pos,
            PyObject *o}

Does the same, but does no error checking, and
should \emph{only} be used to fill in brand new tuples.
\end{cfuncdesc}

\begin{cfuncdesc}{int}{_PyTuple_Resize}{PyTupleObject *p,
            int new,
            int last_is_sticky}
Can be used to resize a tuple. Because tuples are
\emph{supposed} to be immutable, this should only be used if there is only
one module referencing the object. Do \emph{not} use this if the tuple may
already be known to some other part of the code. \var{last_is_sticky} is
a flag --- if set, the tuple will grow or shrink at the front, otherwise
it will grow or shrink at the end. Think of this as destroying the old
tuple and creating a new one, only more efficiently.
\end{cfuncdesc}


\subsection{List Objects}
\label{listObjects}

\begin{ctypedesc}{PyListObject}
This subtype of \ctype{PyObject} represents a Python list object.
\end{ctypedesc}

\begin{cvardesc}{PyTypeObject}{PyList_Type}
This instance of \ctype{PyTypeObject} represents the Python list type.
\end{cvardesc}

\begin{cfuncdesc}{int}{PyList_Check}{PyObject *p}
Returns true if its argument is a \ctype{PyListObject}.
\end{cfuncdesc}

\begin{cfuncdesc}{PyObject*}{PyList_New}{int size}
Returns a new list of length \var{len} on success, and \NULL{} on
failure.
\end{cfuncdesc}

\begin{cfuncdesc}{int}{PyList_Size}{PyObject *list}
Returns the length of the list object in \var{list}.
\end{cfuncdesc}

\begin{cfuncdesc}{PyObject*}{PyList_GetItem}{PyObject *list, int index}
Returns the object at position \var{pos} in the list pointed
to by \var{p}.  If \var{pos} is out of bounds, returns \NULL{} and
sets an \exception{IndexError} exception.  \strong{Note:} this
function returns a ``borrowed'' reference.
\end{cfuncdesc}

\begin{cfuncdesc}{int}{PyList_SetItem}{PyObject *list, int index,
                                       PyObject *item}
Sets the item at index \var{index} in list to \var{item}.
\end{cfuncdesc}

\begin{cfuncdesc}{int}{PyList_Insert}{PyObject *list, int index,
                                      PyObject *item}
Inserts the item \var{item} into list \var{list} in front of index
\var{index}.  Returns 0 if successful; returns -1 and sets an
exception if unsuccessful.  Analogous to \code{list.insert(index, item)}.
\end{cfuncdesc}

\begin{cfuncdesc}{int}{PyList_Append}{PyObject *list, PyObject *item}
Appends the object \var{item} at the end of list \var{list}.  Returns
0 if successful; returns -1 and sets an exception if unsuccessful.
Analogous to \code{list.append(item)}.
\end{cfuncdesc}

\begin{cfuncdesc}{PyObject*}{PyList_GetSlice}{PyObject *list,
                                              int low, int high}
Returns a list of the objects in \var{list} containing the objects 
\emph{between} \var{low} and \var{high}.  Returns NULL and sets an
exception if unsuccessful.
Analogous to \code{list[low:high]}.
\end{cfuncdesc}

\begin{cfuncdesc}{int}{PyList_SetSlice}{PyObject *list,
                                        int low, int high,
                                        PyObject *itemlist}
Sets the slice of \var{list} between \var{low} and \var{high} to the contents
of \var{itemlist}.  Analogous to \code{list[low:high]=itemlist}.  Returns 0
on success, -1 on failure.
\end{cfuncdesc}

\begin{cfuncdesc}{int}{PyList_Sort}{PyObject *list}
Sorts the items of \var{list} in place.  Returns 0 on success, -1 on failure.
\end{cfuncdesc}

\begin{cfuncdesc}{int}{PyList_Reverse}{PyObject *list}
Reverses the items of \var{list} in place.  Returns 0 on success, -1 on failure.
\end{cfuncdesc}

\begin{cfuncdesc}{PyObject*}{PyList_AsTuple}{PyObject *list}
Returns a new tuple object containing the contents of \var{list}.
\end{cfuncdesc}

\begin{cfuncdesc}{PyObject*}{PyList_GET_ITEM}{PyObject *list, int i}
Macro form of \cfunction{PyList_GetItem()} without error checking.
\end{cfuncdesc}

\begin{cfuncdesc}{PyObject*}{PyList_SET_ITEM}{PyObject *list, int i,
                                              PyObject *o}
Macro form of \cfunction{PyList_SetItem()} without error checking.
\end{cfuncdesc}

\begin{cfuncdesc}{int}{PyList_GET_SIZE}{PyObject *list}
Macro form of \cfunction{PyList_GetSize()} without error checking.
\end{cfuncdesc}


\section{Mapping Objects}
\label{mapObjects}

\subsection{Dictionary Objects}
\label{dictObjects}

\begin{ctypedesc}{PyDictObject}
This subtype of \ctype{PyObject} represents a Python dictionary object.
\end{ctypedesc}

\begin{cvardesc}{PyTypeObject}{PyDict_Type}
This instance of \ctype{PyTypeObject} represents the Python dictionary type.
\end{cvardesc}

\begin{cfuncdesc}{int}{PyDict_Check}{PyObject *p}
Returns true if its argument is a \ctype{PyDictObject}.
\end{cfuncdesc}

\begin{cfuncdesc}{PyObject*}{PyDict_New}{}
Returns a new empty dictionary.
\end{cfuncdesc}

\begin{cfuncdesc}{void}{PyDict_Clear}{PyDictObject *p}
Empties an existing dictionary of all key/value pairs.
\end{cfuncdesc}

\begin{cfuncdesc}{int}{PyDict_SetItem}{PyDictObject *p,
            PyObject *key,
            PyObject *val}
Inserts \var{value} into the dictionary with a key of \var{key}.  Both
\var{key} and \var{value} should be PyObjects, and \var{key} should be
hashable.
\end{cfuncdesc}

\begin{cfuncdesc}{int}{PyDict_SetItemString}{PyDictObject *p,
            char *key,
            PyObject *val}
Inserts \var{value} into the dictionary using \var{key}
as a key. \var{key} should be a \ctype{char *}.  The key object is
created using \code{PyString_FromString(\var{key})}.
\end{cfuncdesc}

\begin{cfuncdesc}{int}{PyDict_DelItem}{PyDictObject *p, PyObject *key}
Removes the entry in dictionary \var{p} with key \var{key}.
\var{key} is a PyObject.
\end{cfuncdesc}

\begin{cfuncdesc}{int}{PyDict_DelItemString}{PyDictObject *p, char *key}
Removes the entry in dictionary \var{p} which has a key
specified by the \ctype{char *}\var{key}.
\end{cfuncdesc}

\begin{cfuncdesc}{PyObject*}{PyDict_GetItem}{PyDictObject *p, PyObject *key}
Returns the object from dictionary \var{p} which has a key
\var{key}.  Returns \NULL{} if the key \var{key} is not present, but
without (!) setting an exception.  \strong{Note:}  this function
returns a ``borrowed'' reference.
\end{cfuncdesc}

\begin{cfuncdesc}{PyObject*}{PyDict_GetItemString}{PyDictObject *p, char *key}
This is the same as \cfunction{PyDict_GetItem()}, but \var{key} is
specified as a \ctype{char *}, rather than a \ctype{PyObject *}.
\end{cfuncdesc}

\begin{cfuncdesc}{PyObject*}{PyDict_Items}{PyDictObject *p}
Returns a \ctype{PyListObject} containing all the items 
from the dictionary, as in the dictinoary method \method{items()} (see
the \emph{Python Library Reference}).
\end{cfuncdesc}

\begin{cfuncdesc}{PyObject*}{PyDict_Keys}{PyDictObject *p}
Returns a \ctype{PyListObject} containing all the keys 
from the dictionary, as in the dictionary method \method{keys()} (see the
\emph{Python Library Reference}).
\end{cfuncdesc}

\begin{cfuncdesc}{PyObject*}{PyDict_Values}{PyDictObject *p}
Returns a \ctype{PyListObject} containing all the values 
from the dictionary \var{p}, as in the dictionary method
\method{values()} (see the \emph{Python Library Reference}).
\end{cfuncdesc}

\begin{cfuncdesc}{int}{PyDict_Size}{PyDictObject *p}
Returns the number of items in the dictionary.
\end{cfuncdesc}

\begin{cfuncdesc}{int}{PyDict_Next}{PyDictObject *p,
            int ppos,
            PyObject **pkey,
            PyObject **pvalue}

\end{cfuncdesc}


\section{Numeric Objects}
\label{numericObjects}

\subsection{Plain Integer Objects}
\label{intObjects}

\begin{ctypedesc}{PyIntObject}
This subtype of \ctype{PyObject} represents a Python integer object.
\end{ctypedesc}

\begin{cvardesc}{PyTypeObject}{PyInt_Type}
This instance of \ctype{PyTypeObject} represents the Python plain 
integer type.
\end{cvardesc}

\begin{cfuncdesc}{int}{PyInt_Check}{PyObject *}

\end{cfuncdesc}

\begin{cfuncdesc}{PyObject*}{PyInt_FromLong}{long ival}
Creates a new integer object with a value of \var{ival}.

The current implementation keeps an array of integer objects for all
integers between \code{-1} and \code{100}, when you create an int in
that range you actually just get back a reference to the existing
object. So it should be possible to change the value of \code{1}. I
suspect the behaviour of Python in this case is undefined. :-)
\end{cfuncdesc}

\begin{cfuncdesc}{long}{PyInt_AS_LONG}{PyIntObject *io}
Returns the value of the object \var{io}.  No error checking is
performed.
\end{cfuncdesc}

\begin{cfuncdesc}{long}{PyInt_AsLong}{PyObject *io}
Will first attempt to cast the object to a \ctype{PyIntObject}, if
it is not already one, and then return its value.
\end{cfuncdesc}

\begin{cfuncdesc}{long}{PyInt_GetMax}{}
Returns the systems idea of the largest integer it can handle
(\constant{LONG_MAX}, as defined in the system header files).
\end{cfuncdesc}


\subsection{Long Integer Objects}
\label{longObjects}

\begin{ctypedesc}{PyLongObject}
This subtype of \ctype{PyObject} represents a Python long integer
object.
\end{ctypedesc}

\begin{cvardesc}{PyTypeObject}{PyLong_Type}
This instance of \ctype{PyTypeObject} represents the Python long
integer type.
\end{cvardesc}

\begin{cfuncdesc}{int}{PyLong_Check}{PyObject *p}
Returns true if its argument is a \ctype{PyLongObject}.
\end{cfuncdesc}

\begin{cfuncdesc}{PyObject*}{PyLong_FromLong}{long v}
Returns a new \ctype{PyLongObject} object from \var{v}.
\end{cfuncdesc}

\begin{cfuncdesc}{PyObject*}{PyLong_FromUnsignedLong}{unsigned long v}
Returns a new \ctype{PyLongObject} object from an unsigned \C{} long.
\end{cfuncdesc}

\begin{cfuncdesc}{PyObject*}{PyLong_FromDouble}{double v}
Returns a new \ctype{PyLongObject} object from the integer part of \var{v}.
\end{cfuncdesc}

\begin{cfuncdesc}{long}{PyLong_AsLong}{PyObject *pylong}
Returns a \C{} \ctype{long} representation of the contents of \var{pylong}.  
WHAT HAPPENS IF \var{pylong} is greater than \constant{LONG_MAX}?
\end{cfuncdesc}

\begin{cfuncdesc}{unsigned long}{PyLong_AsUnsignedLong}{PyObject *pylong}
Returns a \C{} \ctype{unsigned long} representation of the contents of 
\var{pylong}.  WHAT HAPPENS IF \var{pylong} is greater than
\constant{ULONG_MAX}?
\end{cfuncdesc}

\begin{cfuncdesc}{double}{PyLong_AsDouble}{PyObject *pylong}
Returns a \C{} \ctype{double} representation of the contents of \var{pylong}.
\end{cfuncdesc}

\begin{cfuncdesc}{PyObject*}{PyLong_FromString}{char *str, char **pend,
                                                int base}
\end{cfuncdesc}


\subsection{Floating Point Objects}
\label{floatObjects}

\begin{ctypedesc}{PyFloatObject}
This subtype of \ctype{PyObject} represents a Python floating point
object.
\end{ctypedesc}

\begin{cvardesc}{PyTypeObject}{PyFloat_Type}
This instance of \ctype{PyTypeObject} represents the Python floating
point type.
\end{cvardesc}

\begin{cfuncdesc}{int}{PyFloat_Check}{PyObject *p}
Returns true if its argument is a \ctype{PyFloatObject}.
\end{cfuncdesc}

\begin{cfuncdesc}{PyObject*}{PyFloat_FromDouble}{double v}
Creates a \ctype{PyFloatObject} object from \var{v}.
\end{cfuncdesc}

\begin{cfuncdesc}{double}{PyFloat_AsDouble}{PyObject *pyfloat}
Returns a \C{} \ctype{double} representation of the contents of \var{pyfloat}.
\end{cfuncdesc}

\begin{cfuncdesc}{double}{PyFloat_AS_DOUBLE}{PyObject *pyfloat}
Returns a \C{} \ctype{double} representation of the contents of
\var{pyfloat}, but without error checking.
\end{cfuncdesc}


\subsection{Complex Number Objects}
\label{complexObjects}

\begin{ctypedesc}{Py_complex}
The \C{} structure which corresponds to the value portion of a Python
complex number object.  Most of the functions for dealing with complex
number objects use structures of this type as input or output values,
as appropriate.  It is defined as:

\begin{verbatim}
typedef struct {
   double real;
   double imag;
} Py_complex;
\end{verbatim}
\end{ctypedesc}

\begin{ctypedesc}{PyComplexObject}
This subtype of \ctype{PyObject} represents a Python complex number object.
\end{ctypedesc}

\begin{cvardesc}{PyTypeObject}{PyComplex_Type}
This instance of \ctype{PyTypeObject} represents the Python complex 
number type.
\end{cvardesc}

\begin{cfuncdesc}{int}{PyComplex_Check}{PyObject *p}
Returns true if its argument is a \ctype{PyComplexObject}.
\end{cfuncdesc}

\begin{cfuncdesc}{Py_complex}{_Py_c_sum}{Py_complex left, Py_complex right}
\end{cfuncdesc}

\begin{cfuncdesc}{Py_complex}{_Py_c_diff}{Py_complex left, Py_complex right}
\end{cfuncdesc}

\begin{cfuncdesc}{Py_complex}{_Py_c_neg}{Py_complex complex}
\end{cfuncdesc}

\begin{cfuncdesc}{Py_complex}{_Py_c_prod}{Py_complex left, Py_complex right}
\end{cfuncdesc}

\begin{cfuncdesc}{Py_complex}{_Py_c_quot}{Py_complex dividend,
                                          Py_complex divisor}
\end{cfuncdesc}

\begin{cfuncdesc}{Py_complex}{_Py_c_pow}{Py_complex num, Py_complex exp}
\end{cfuncdesc}

\begin{cfuncdesc}{PyObject*}{PyComplex_FromCComplex}{Py_complex v}
\end{cfuncdesc}

\begin{cfuncdesc}{PyObject*}{PyComplex_FromDoubles}{double real, double imag}
Returns a new \ctype{PyComplexObject} object from \var{real} and \var{imag}.
\end{cfuncdesc}

\begin{cfuncdesc}{double}{PyComplex_RealAsDouble}{PyObject *op}
Returns the real part of \var{op} as a \C{} \ctype{double}.
\end{cfuncdesc}

\begin{cfuncdesc}{double}{PyComplex_ImagAsDouble}{PyObject *op}
Returns the imaginary part of \var{op} as a \C{} \ctype{double}.
\end{cfuncdesc}

\begin{cfuncdesc}{Py_complex}{PyComplex_AsCComplex}{PyObject *op}
\end{cfuncdesc}



\section{Other Objects}
\label{otherObjects}

\subsection{File Objects}
\label{fileObjects}

\begin{ctypedesc}{PyFileObject}
This subtype of \ctype{PyObject} represents a Python file object.
\end{ctypedesc}

\begin{cvardesc}{PyTypeObject}{PyFile_Type}
This instance of \ctype{PyTypeObject} represents the Python file type.
\end{cvardesc}

\begin{cfuncdesc}{int}{PyFile_Check}{PyObject *p}
Returns true if its argument is a \ctype{PyFileObject}.
\end{cfuncdesc}

\begin{cfuncdesc}{PyObject*}{PyFile_FromString}{char *name, char *mode}
Creates a new \ctype{PyFileObject} pointing to the file
specified in \var{name} with the mode specified in \var{mode}.
\end{cfuncdesc}

\begin{cfuncdesc}{PyObject*}{PyFile_FromFile}{FILE *fp,
              char *name, char *mode, int (*close)}
Creates a new \ctype{PyFileObject} from the already-open \var{fp}.
The function \var{close} will be called when the file should be
closed.
\end{cfuncdesc}

\begin{cfuncdesc}{FILE *}{PyFile_AsFile}{PyFileObject *p}
Returns the file object associated with \var{p} as a \ctype{FILE *}.
\end{cfuncdesc}

\begin{cfuncdesc}{PyObject*}{PyFile_GetLine}{PyObject *p, int n}
undocumented as yet
\end{cfuncdesc}

\begin{cfuncdesc}{PyObject*}{PyFile_Name}{PyObject *p}
Returns the name of the file specified by \var{p} as a 
\ctype{PyStringObject}.
\end{cfuncdesc}

\begin{cfuncdesc}{void}{PyFile_SetBufSize}{PyFileObject *p, int n}
Available on systems with \cfunction{setvbuf()} only.  This should
only be called immediately after file object creation.
\end{cfuncdesc}

\begin{cfuncdesc}{int}{PyFile_SoftSpace}{PyFileObject *p, int newflag}
Sets the \member{softspace} attribute of \var{p} to \var{newflag}.
Returns the previous value.  This function clears any errors, and will
return \code{0} as the previous value if the attribute either does not
exist or if there were errors in retrieving it.  There is no way to
detect errors from this function, but doing so should not be needed.
\end{cfuncdesc}

\begin{cfuncdesc}{int}{PyFile_WriteObject}{PyObject *obj, PyFileObject *p,
                                           int flags}
Writes object \var{obj} to file object \var{p}.
\end{cfuncdesc}

\begin{cfuncdesc}{int}{PyFile_WriteString}{char *s, PyFileObject *p,
                                           int flags}
Writes string \var{s} to file object \var{p}.
\end{cfuncdesc}


\subsection{CObjects}
\label{cObjects}

\begin{ctypedesc}{PyCObject}
This subtype of \ctype{PyObject} represents an opaque value, useful for
\C{} extension modules who need to pass an opaque value (as a
\ctype{void *} pointer) through Python code to other \C{} code.  It is
often used to make a C function pointer defined in one module
available to other modules, so the regular import mechanism can be
used to access C APIs defined in dynamically loaded modules.
\end{ctypedesc}

\begin{cfuncdesc}{PyObject *}{PyCObject_FromVoidPtr}{void* cobj, 
	void (*destr)(void *)}
Creates a \ctype{PyCObject} from the \code{void *} \var{cobj}.  The
\var{destr} function will be called when the object is reclaimed.
\end{cfuncdesc}

\begin{cfuncdesc}{PyObject *}{PyCObject_FromVoidPtrAndDesc}{void* cobj,
	void* desc, void (*destr)(void *, void *) }
Creates a \ctype{PyCObject} from the \ctype{void *}\var{cobj}.  The
\var{destr} function will be called when the object is reclaimed.  The
\var{desc} argument can be used to pass extra callback data for the
destructor function.
\end{cfuncdesc}

\begin{cfuncdesc}{void *}{PyCObject_AsVoidPtr}{PyObject* self}
Returns the object \ctype{void *} that the \ctype{PyCObject} \var{self}
was created with.
\end{cfuncdesc}

\begin{cfuncdesc}{void *}{PyCObject_GetDesc}{PyObject* self}
Returns the description \ctype{void *} that the \ctype{PyCObject}
\var{self} was created with.
\end{cfuncdesc}

\chapter{Initialization, Finalization, and Threads}
\label{initialization}

\begin{cfuncdesc}{void}{Py_Initialize}{}
Initialize the Python interpreter.  In an application embedding 
Python, this should be called before using any other Python/C API 
functions; with the exception of \cfunction{Py_SetProgramName()},
\cfunction{PyEval_InitThreads()}, \cfunction{PyEval_ReleaseLock()},
and \cfunction{PyEval_AcquireLock()}.  This initializes the table of
loaded modules (\code{sys.modules}), and creates the fundamental
modules \module{__builtin__}\refbimodindex{__builtin__},
\module{__main__}\refbimodindex{__main__} and
\module{sys}\refbimodindex{sys}.  It also initializes the module
search path (\code{sys.path}).%
\indexiii{module}{search}{path}
It does not set \code{sys.argv}; use \cfunction{PySys_SetArgv()} for
that.  This is a no-op when called for a second time (without calling
\cfunction{Py_Finalize()} first).  There is no return value; it is a
fatal error if the initialization fails.
\end{cfuncdesc}

\begin{cfuncdesc}{int}{Py_IsInitialized}{}
Return true (nonzero) when the Python interpreter has been
initialized, false (zero) if not.  After \cfunction{Py_Finalize()} is
called, this returns false until \cfunction{Py_Initialize()} is called
again.
\end{cfuncdesc}

\begin{cfuncdesc}{void}{Py_Finalize}{}
Undo all initializations made by \cfunction{Py_Initialize()} and
subsequent use of Python/C API functions, and destroy all
sub-interpreters (see \cfunction{Py_NewInterpreter()} below) that were
created and not yet destroyed since the last call to
\cfunction{Py_Initialize()}.  Ideally, this frees all memory allocated
by the Python interpreter.  This is a no-op when called for a second
time (without calling \cfunction{Py_Initialize()} again first).  There
is no return value; errors during finalization are ignored.

This function is provided for a number of reasons.  An embedding 
application might want to restart Python without having to restart the 
application itself.  An application that has loaded the Python 
interpreter from a dynamically loadable library (or DLL) might want to 
free all memory allocated by Python before unloading the DLL. During a 
hunt for memory leaks in an application a developer might want to free 
all memory allocated by Python before exiting from the application.

\strong{Bugs and caveats:} The destruction of modules and objects in 
modules is done in random order; this may cause destructors 
(\method{__del__()} methods) to fail when they depend on other objects 
(even functions) or modules.  Dynamically loaded extension modules 
loaded by Python are not unloaded.  Small amounts of memory allocated 
by the Python interpreter may not be freed (if you find a leak, please 
report it).  Memory tied up in circular references between objects is 
not freed.  Some memory allocated by extension modules may not be 
freed.  Some extension may not work properly if their initialization 
routine is called more than once; this can happen if an applcation 
calls \cfunction{Py_Initialize()} and \cfunction{Py_Finalize()} more
than once.
\end{cfuncdesc}

\begin{cfuncdesc}{PyThreadState*}{Py_NewInterpreter}{}
Create a new sub-interpreter.  This is an (almost) totally separate
environment for the execution of Python code.  In particular, the new
interpreter has separate, independent versions of all imported
modules, including the fundamental modules
\module{__builtin__}\refbimodindex{__builtin__},
\module{__main__}\refbimodindex{__main__} and
\module{sys}\refbimodindex{sys}.  The table of loaded modules
(\code{sys.modules}) and the module search path (\code{sys.path}) are
also separate.  The new environment has no \code{sys.argv} variable.
It has new standard I/O stream file objects \code{sys.stdin},
\code{sys.stdout} and \code{sys.stderr} (however these refer to the
same underlying \ctype{FILE} structures in the \C{} library).

The return value points to the first thread state created in the new 
sub-interpreter.  This thread state is made the current thread state.  
Note that no actual thread is created; see the discussion of thread 
states below.  If creation of the new interpreter is unsuccessful, 
\NULL{} is returned; no exception is set since the exception state 
is stored in the current thread state and there may not be a current 
thread state.  (Like all other Python/C API functions, the global 
interpreter lock must be held before calling this function and is 
still held when it returns; however, unlike most other Python/C API 
functions, there needn't be a current thread state on entry.)

Extension modules are shared between (sub-)interpreters as follows: 
the first time a particular extension is imported, it is initialized 
normally, and a (shallow) copy of its module's dictionary is 
squirreled away.  When the same extension is imported by another 
(sub-)interpreter, a new module is initialized and filled with the 
contents of this copy; the extension's \code{init} function is not
called.  Note that this is different from what happens when an
extension is imported after the interpreter has been completely
re-initialized by calling \cfunction{Py_Finalize()} and
\cfunction{Py_Initialize()}; in that case, the extension's \code{init}
function \emph{is} called again.

\strong{Bugs and caveats:} Because sub-interpreters (and the main 
interpreter) are part of the same process, the insulation between them 
isn't perfect --- for example, using low-level file operations like 
\function{os.close()} they can (accidentally or maliciously) affect each 
other's open files.  Because of the way extensions are shared between 
(sub-)interpreters, some extensions may not work properly; this is 
especially likely when the extension makes use of (static) global 
variables, or when the extension manipulates its module's dictionary 
after its initialization.  It is possible to insert objects created in 
one sub-interpreter into a namespace of another sub-interpreter; this 
should be done with great care to avoid sharing user-defined 
functions, methods, instances or classes between sub-interpreters, 
since import operations executed by such objects may affect the 
wrong (sub-)interpreter's dictionary of loaded modules.  (XXX This is 
a hard-to-fix bug that will be addressed in a future release.)
\end{cfuncdesc}

\begin{cfuncdesc}{void}{Py_EndInterpreter}{PyThreadState *tstate}
Destroy the (sub-)interpreter represented by the given thread state.  
The given thread state must be the current thread state.  See the 
discussion of thread states below.  When the call returns, the current 
thread state is \NULL{}.  All thread states associated with this 
interpreted are destroyed.  (The global interpreter lock must be held 
before calling this function and is still held when it returns.)  
\cfunction{Py_Finalize()} will destroy all sub-interpreters that haven't 
been explicitly destroyed at that point.
\end{cfuncdesc}

\begin{cfuncdesc}{void}{Py_SetProgramName}{char *name}
This function should be called before \cfunction{Py_Initialize()} is called 
for the first time, if it is called at all.  It tells the interpreter 
the value of the \code{argv[0]} argument to the \cfunction{main()} function 
of the program.  This is used by \cfunction{Py_GetPath()} and some other 
functions below to find the Python run-time libraries relative to the 
interpreter executable.  The default value is \code{"python"}.  The 
argument should point to a zero-terminated character string in static 
storage whose contents will not change for the duration of the 
program's execution.  No code in the Python interpreter will change 
the contents of this storage.
\end{cfuncdesc}

\begin{cfuncdesc}{char*}{Py_GetProgramName}{}
Return the program name set with \cfunction{Py_SetProgramName()}, or the 
default.  The returned string points into static storage; the caller 
should not modify its value.
\end{cfuncdesc}

\begin{cfuncdesc}{char*}{Py_GetPrefix}{}
Return the \emph{prefix} for installed platform-independent files.  This 
is derived through a number of complicated rules from the program name 
set with \cfunction{Py_SetProgramName()} and some environment variables; 
for example, if the program name is \code{"/usr/local/bin/python"}, 
the prefix is \code{"/usr/local"}.  The returned string points into 
static storage; the caller should not modify its value.  This 
corresponds to the \makevar{prefix} variable in the top-level 
\file{Makefile} and the \code{-}\code{-prefix} argument to the 
\program{configure} script at build time.  The value is available to 
Python code as \code{sys.prefix}.  It is only useful on \UNIX{}.  See 
also the next function.
\end{cfuncdesc}

\begin{cfuncdesc}{char*}{Py_GetExecPrefix}{}
Return the \emph{exec-prefix} for installed platform-\emph{de}pendent 
files.  This is derived through a number of complicated rules from the 
program name set with \cfunction{Py_SetProgramName()} and some environment 
variables; for example, if the program name is 
\code{"/usr/local/bin/python"}, the exec-prefix is 
\code{"/usr/local"}.  The returned string points into static storage; 
the caller should not modify its value.  This corresponds to the 
\makevar{exec_prefix} variable in the top-level \file{Makefile} and the 
\code{-}\code{-exec_prefix} argument to the \program{configure} script
at build  time.  The value is available to Python code as 
\code{sys.exec_prefix}.  It is only useful on \UNIX{}.

Background: The exec-prefix differs from the prefix when platform 
dependent files (such as executables and shared libraries) are 
installed in a different directory tree.  In a typical installation, 
platform dependent files may be installed in the 
\code{"/usr/local/plat"} subtree while platform independent may be 
installed in \code{"/usr/local"}.

Generally speaking, a platform is a combination of hardware and 
software families, e.g.  Sparc machines running the Solaris 2.x 
operating system are considered the same platform, but Intel machines 
running Solaris 2.x are another platform, and Intel machines running 
Linux are yet another platform.  Different major revisions of the same 
operating system generally also form different platforms.  Non-\UNIX{} 
operating systems are a different story; the installation strategies 
on those systems are so different that the prefix and exec-prefix are 
meaningless, and set to the empty string.  Note that compiled Python 
bytecode files are platform independent (but not independent from the 
Python version by which they were compiled!).

System administrators will know how to configure the \program{mount} or 
\program{automount} programs to share \code{"/usr/local"} between platforms 
while having \code{"/usr/local/plat"} be a different filesystem for each 
platform.
\end{cfuncdesc}

\begin{cfuncdesc}{char*}{Py_GetProgramFullPath}{}
Return the full program name of the Python executable; this is 
computed as a side-effect of deriving the default module search path 
from the program name (set by \cfunction{Py_SetProgramName()} above).  The 
returned string points into static storage; the caller should not 
modify its value.  The value is available to Python code as 
\code{sys.executable}.
\end{cfuncdesc}

\begin{cfuncdesc}{char*}{Py_GetPath}{}
\indexiii{module}{search}{path}
Return the default module search path; this is computed from the 
program name (set by \cfunction{Py_SetProgramName()} above) and some 
environment variables.  The returned string consists of a series of 
directory names separated by a platform dependent delimiter character.  
The delimiter character is \character{:} on \UNIX{}, \character{;} on
DOS/Windows, and \character{\\n} (the \ASCII{} newline character) on
Macintosh.  The returned string points into static storage; the caller
should not modify its value.  The value is available to Python code 
as the list \code{sys.path}, which may be modified to change the 
future search path for loaded modules.

% XXX should give the exact rules
\end{cfuncdesc}

\begin{cfuncdesc}{const char*}{Py_GetVersion}{}
Return the version of this Python interpreter.  This is a string that 
looks something like

\begin{verbatim}
"1.5 (#67, Dec 31 1997, 22:34:28) [GCC 2.7.2.2]"
\end{verbatim}

The first word (up to the first space character) is the current Python 
version; the first three characters are the major and minor version 
separated by a period.  The returned string points into static storage; 
the caller should not modify its value.  The value is available to 
Python code as the list \code{sys.version}.
\end{cfuncdesc}

\begin{cfuncdesc}{const char*}{Py_GetPlatform}{}
Return the platform identifier for the current platform.  On \UNIX{}, 
this is formed from the ``official'' name of the operating system, 
converted to lower case, followed by the major revision number; e.g., 
for Solaris 2.x, which is also known as SunOS 5.x, the value is 
\code{"sunos5"}.  On Macintosh, it is \code{"mac"}.  On Windows, it 
is \code{"win"}.  The returned string points into static storage; 
the caller should not modify its value.  The value is available to 
Python code as \code{sys.platform}.
\end{cfuncdesc}

\begin{cfuncdesc}{const char*}{Py_GetCopyright}{}
Return the official copyright string for the current Python version, 
for example

\code{"Copyright 1991-1995 Stichting Mathematisch Centrum, Amsterdam"}

The returned string points into static storage; the caller should not 
modify its value.  The value is available to Python code as the list 
\code{sys.copyright}.
\end{cfuncdesc}

\begin{cfuncdesc}{const char*}{Py_GetCompiler}{}
Return an indication of the compiler used to build the current Python 
version, in square brackets, for example:

\begin{verbatim}
"[GCC 2.7.2.2]"
\end{verbatim}

The returned string points into static storage; the caller should not 
modify its value.  The value is available to Python code as part of 
the variable \code{sys.version}.
\end{cfuncdesc}

\begin{cfuncdesc}{const char*}{Py_GetBuildInfo}{}
Return information about the sequence number and build date and time 
of the current Python interpreter instance, for example

\begin{verbatim}
"#67, Aug  1 1997, 22:34:28"
\end{verbatim}

The returned string points into static storage; the caller should not 
modify its value.  The value is available to Python code as part of 
the variable \code{sys.version}.
\end{cfuncdesc}

\begin{cfuncdesc}{int}{PySys_SetArgv}{int argc, char **argv}
% XXX
\end{cfuncdesc}

% XXX Other PySys thingies (doesn't really belong in this chapter)

\section{Thread State and the Global Interpreter Lock}
\label{threads}

The Python interpreter is not fully thread safe.  In order to support
multi-threaded Python programs, there's a global lock that must be
held by the current thread before it can safely access Python objects.
Without the lock, even the simplest operations could cause problems in
a multi-threaded program: for example, when two threads simultaneously
increment the reference count of the same object, the reference count
could end up being incremented only once instead of twice.

Therefore, the rule exists that only the thread that has acquired the
global interpreter lock may operate on Python objects or call Python/C
API functions.  In order to support multi-threaded Python programs,
the interpreter regularly release and reacquires the lock --- by
default, every ten bytecode instructions (this can be changed with
\function{sys.setcheckinterval()}).  The lock is also released and
reacquired around potentially blocking I/O operations like reading or
writing a file, so that other threads can run while the thread that
requests the I/O is waiting for the I/O operation to complete.

The Python interpreter needs to keep some bookkeeping information
separate per thread --- for this it uses a data structure called
\ctype{PyThreadState}.  This is new in Python 1.5; in earlier versions,
such state was stored in global variables, and switching threads could
cause problems.  In particular, exception handling is now thread safe,
when the application uses \function{sys.exc_info()} to access the
exception last raised in the current thread.

There's one global variable left, however: the pointer to the current
\ctype{PyThreadState} structure.  While most thread packages have a way
to store ``per-thread global data,'' Python's internal platform
independent thread abstraction doesn't support this yet.  Therefore,
the current thread state must be manipulated explicitly.

This is easy enough in most cases.  Most code manipulating the global
interpreter lock has the following simple structure:

\begin{verbatim}
Save the thread state in a local variable.
Release the interpreter lock.
...Do some blocking I/O operation...
Reacquire the interpreter lock.
Restore the thread state from the local variable.
\end{verbatim}

This is so common that a pair of macros exists to simplify it:

\begin{verbatim}
Py_BEGIN_ALLOW_THREADS
...Do some blocking I/O operation...
Py_END_ALLOW_THREADS
\end{verbatim}

The \code{Py_BEGIN_ALLOW_THREADS} macro opens a new block and declares
a hidden local variable; the \code{Py_END_ALLOW_THREADS} macro closes
the block.  Another advantage of using these two macros is that when
Python is compiled without thread support, they are defined empty,
thus saving the thread state and lock manipulations.

When thread support is enabled, the block above expands to the
following code:

\begin{verbatim}
{
    PyThreadState *_save;
    _save = PyEval_SaveThread();
    ...Do some blocking I/O operation...
    PyEval_RestoreThread(_save);
}
\end{verbatim}

Using even lower level primitives, we can get roughly the same effect
as follows:

\begin{verbatim}
{
    PyThreadState *_save;
    _save = PyThreadState_Swap(NULL);
    PyEval_ReleaseLock();
    ...Do some blocking I/O operation...
    PyEval_AcquireLock();
    PyThreadState_Swap(_save);
}
\end{verbatim}

There are some subtle differences; in particular,
\cfunction{PyEval_RestoreThread()} saves and restores the value of the
global variable \cdata{errno}, since the lock manipulation does not
guarantee that \cdata{errno} is left alone.  Also, when thread support
is disabled, \cfunction{PyEval_SaveThread()} and
\cfunction{PyEval_RestoreThread()} don't manipulate the lock; in this
case, \cfunction{PyEval_ReleaseLock()} and
\cfunction{PyEval_AcquireLock()} are not available.  This is done so
that dynamically loaded extensions compiled with thread support
enabled can be loaded by an interpreter that was compiled with
disabled thread support.

The global interpreter lock is used to protect the pointer to the
current thread state.  When releasing the lock and saving the thread
state, the current thread state pointer must be retrieved before the
lock is released (since another thread could immediately acquire the
lock and store its own thread state in the global variable).
Reversely, when acquiring the lock and restoring the thread state, the
lock must be acquired before storing the thread state pointer.

Why am I going on with so much detail about this?  Because when
threads are created from \C{}, they don't have the global interpreter
lock, nor is there a thread state data structure for them.  Such
threads must bootstrap themselves into existence, by first creating a
thread state data structure, then acquiring the lock, and finally
storing their thread state pointer, before they can start using the
Python/C API.  When they are done, they should reset the thread state
pointer, release the lock, and finally free their thread state data
structure.

When creating a thread data structure, you need to provide an
interpreter state data structure.  The interpreter state data
structure hold global data that is shared by all threads in an
interpreter, for example the module administration
(\code{sys.modules}).  Depending on your needs, you can either create
a new interpreter state data structure, or share the interpreter state
data structure used by the Python main thread (to access the latter,
you must obtain the thread state and access its \member{interp} member;
this must be done by a thread that is created by Python or by the main
thread after Python is initialized).

XXX More?

\begin{ctypedesc}{PyInterpreterState}
This data structure represents the state shared by a number of
cooperating threads.  Threads belonging to the same interpreter
share their module administration and a few other internal items.
There are no public members in this structure.

Threads belonging to different interpreters initially share nothing,
except process state like available memory, open file descriptors and
such.  The global interpreter lock is also shared by all threads,
regardless of to which interpreter they belong.
\end{ctypedesc}

\begin{ctypedesc}{PyThreadState}
This data structure represents the state of a single thread.  The only
public data member is \ctype{PyInterpreterState *}\member{interp},
which points to this thread's interpreter state.
\end{ctypedesc}

\begin{cfuncdesc}{void}{PyEval_InitThreads}{}
Initialize and acquire the global interpreter lock.  It should be
called in the main thread before creating a second thread or engaging
in any other thread operations such as
\cfunction{PyEval_ReleaseLock()} or
\code{PyEval_ReleaseThread(\var{tstate})}.  It is not needed before
calling \cfunction{PyEval_SaveThread()} or
\cfunction{PyEval_RestoreThread()}.

This is a no-op when called for a second time.  It is safe to call
this function before calling \cfunction{Py_Initialize()}.

When only the main thread exists, no lock operations are needed.  This
is a common situation (most Python programs do not use threads), and
the lock operations slow the interpreter down a bit.  Therefore, the
lock is not created initially.  This situation is equivalent to having
acquired the lock: when there is only a single thread, all object
accesses are safe.  Therefore, when this function initializes the
lock, it also acquires it.  Before the Python
\module{thread}\refbimodindex{thread} module creates a new thread,
knowing that either it has the lock or the lock hasn't been created
yet, it calls \cfunction{PyEval_InitThreads()}.  When this call
returns, it is guaranteed that the lock has been created and that it
has acquired it.

It is \strong{not} safe to call this function when it is unknown which
thread (if any) currently has the global interpreter lock.

This function is not available when thread support is disabled at
compile time.
\end{cfuncdesc}

\begin{cfuncdesc}{void}{PyEval_AcquireLock}{}
Acquire the global interpreter lock.  The lock must have been created
earlier.  If this thread already has the lock, a deadlock ensues.
This function is not available when thread support is disabled at
compile time.
\end{cfuncdesc}

\begin{cfuncdesc}{void}{PyEval_ReleaseLock}{}
Release the global interpreter lock.  The lock must have been created
earlier.  This function is not available when thread support is
disabled at compile time.
\end{cfuncdesc}

\begin{cfuncdesc}{void}{PyEval_AcquireThread}{PyThreadState *tstate}
Acquire the global interpreter lock and then set the current thread
state to \var{tstate}, which should not be \NULL{}.  The lock must
have been created earlier.  If this thread already has the lock,
deadlock ensues.  This function is not available when thread support
is disabled at compile time.
\end{cfuncdesc}

\begin{cfuncdesc}{void}{PyEval_ReleaseThread}{PyThreadState *tstate}
Reset the current thread state to \NULL{} and release the global
interpreter lock.  The lock must have been created earlier and must be
held by the current thread.  The \var{tstate} argument, which must not
be \NULL{}, is only used to check that it represents the current
thread state --- if it isn't, a fatal error is reported.  This
function is not available when thread support is disabled at compile
time.
\end{cfuncdesc}

\begin{cfuncdesc}{PyThreadState*}{PyEval_SaveThread}{}
Release the interpreter lock (if it has been created and thread
support is enabled) and reset the thread state to \NULL{},
returning the previous thread state (which is not \NULL{}).  If
the lock has been created, the current thread must have acquired it.
(This function is available even when thread support is disabled at
compile time.)
\end{cfuncdesc}

\begin{cfuncdesc}{void}{PyEval_RestoreThread}{PyThreadState *tstate}
Acquire the interpreter lock (if it has been created and thread
support is enabled) and set the thread state to \var{tstate}, which
must not be \NULL{}.  If the lock has been created, the current
thread must not have acquired it, otherwise deadlock ensues.  (This
function is available even when thread support is disabled at compile
time.)
\end{cfuncdesc}

% XXX These aren't really C types, but the ctypedesc macro is the simplest!
\begin{ctypedesc}{Py_BEGIN_ALLOW_THREADS}
This macro expands to
\samp{\{ PyThreadState *_save; _save = PyEval_SaveThread();}.
Note that it contains an opening brace; it must be matched with a
following \code{Py_END_ALLOW_THREADS} macro.  See above for further
discussion of this macro.  It is a no-op when thread support is
disabled at compile time.
\end{ctypedesc}

\begin{ctypedesc}{Py_END_ALLOW_THREADS}
This macro expands to
\samp{PyEval_RestoreThread(_save); \}}.
Note that it contains a closing brace; it must be matched with an
earlier \code{Py_BEGIN_ALLOW_THREADS} macro.  See above for further
discussion of this macro.  It is a no-op when thread support is
disabled at compile time.
\end{ctypedesc}

\begin{ctypedesc}{Py_BEGIN_BLOCK_THREADS}
This macro expands to \samp{PyEval_RestoreThread(_save);} i.e. it
is equivalent to \code{Py_END_ALLOW_THREADS} without the closing
brace.  It is a no-op when thread support is disabled at compile
time.
\end{ctypedesc}

\begin{ctypedesc}{Py_BEGIN_UNBLOCK_THREADS}
This macro expands to \samp{_save = PyEval_SaveThread();} i.e. it is
equivalent to \code{Py_BEGIN_ALLOW_THREADS} without the opening brace
and variable declaration.  It is a no-op when thread support is
disabled at compile time.
\end{ctypedesc}

All of the following functions are only available when thread support
is enabled at compile time, and must be called only when the
interpreter lock has been created.

\begin{cfuncdesc}{PyInterpreterState*}{PyInterpreterState_New}{}
Create a new interpreter state object.  The interpreter lock must be
held.
\end{cfuncdesc}

\begin{cfuncdesc}{void}{PyInterpreterState_Clear}{PyInterpreterState *interp}
Reset all information in an interpreter state object.  The interpreter
lock must be held.
\end{cfuncdesc}

\begin{cfuncdesc}{void}{PyInterpreterState_Delete}{PyInterpreterState *interp}
Destroy an interpreter state object.  The interpreter lock need not be
held.  The interpreter state must have been reset with a previous
call to \cfunction{PyInterpreterState_Clear()}.
\end{cfuncdesc}

\begin{cfuncdesc}{PyThreadState*}{PyThreadState_New}{PyInterpreterState *interp}
Create a new thread state object belonging to the given interpreter
object.  The interpreter lock must be held.
\end{cfuncdesc}

\begin{cfuncdesc}{void}{PyThreadState_Clear}{PyThreadState *tstate}
Reset all information in a thread state object.  The interpreter lock
must be held.
\end{cfuncdesc}

\begin{cfuncdesc}{void}{PyThreadState_Delete}{PyThreadState *tstate}
Destroy a thread state object.  The interpreter lock need not be
held.  The thread state must have been reset with a previous
call to \cfunction{PyThreadState_Clear()}.
\end{cfuncdesc}

\begin{cfuncdesc}{PyThreadState*}{PyThreadState_Get}{}
Return the current thread state.  The interpreter lock must be held.
When the current thread state is \NULL{}, this issues a fatal
error (so that the caller needn't check for \NULL{}).
\end{cfuncdesc}

\begin{cfuncdesc}{PyThreadState*}{PyThreadState_Swap}{PyThreadState *tstate}
Swap the current thread state with the thread state given by the
argument \var{tstate}, which may be \NULL{}.  The interpreter lock
must be held.
\end{cfuncdesc}


\chapter{Defining New Object Types}
\label{newTypes}

\begin{cfuncdesc}{PyObject*}{_PyObject_New}{PyTypeObject *type}
\end{cfuncdesc}

\begin{cfuncdesc}{PyObject*}{_PyObject_NewVar}{PyTypeObject *type, int size}
\end{cfuncdesc}

\begin{cfuncdesc}{TYPE}{_PyObject_NEW}{TYPE, PyTypeObject *}
\end{cfuncdesc}

\begin{cfuncdesc}{TYPE}{_PyObject_NEW_VAR}{TYPE, PyTypeObject *, int size}
\end{cfuncdesc}

Py_InitModule (!!!)

PyArg_ParseTupleAndKeywords, PyArg_ParseTuple, PyArg_Parse

Py_BuildValue

PyObject, PyVarObject

PyObject_HEAD, PyObject_HEAD_INIT, PyObject_VAR_HEAD

Typedefs:
unaryfunc, binaryfunc, ternaryfunc, inquiry, coercion, intargfunc,
intintargfunc, intobjargproc, intintobjargproc, objobjargproc,
getreadbufferproc, getwritebufferproc, getsegcountproc,
destructor, printfunc, getattrfunc, getattrofunc, setattrfunc,
setattrofunc, cmpfunc, reprfunc, hashfunc

PyNumberMethods

PySequenceMethods

PyMappingMethods

PyBufferProcs

PyTypeObject

DL_IMPORT

PyType_Type

Py*_Check

Py_None, _Py_NoneStruct


\chapter{Debugging}
\label{debugging}

XXX Explain Py_DEBUG, Py_TRACE_REFS, Py_REF_DEBUG.


\documentclass{manual}

\title{Python/C API Reference Manual}

\author{
	Guido van Rossum \\
	Dept. AA, CWI, P.O. Box 94079 \\
	1090 GB Amsterdam, The Netherlands \\
	E-mail: {\tt guido@cwi.nl}
}

\date{17 March 1995 \\ Release 1.2-proof-2} % XXX update before release!


\makeindex			% tell \index to actually write the .idx file


\begin{document}

\maketitle

\strong{BEOPEN.COM TERMS AND CONDITIONS FOR PYTHON 2.0}

\centerline{\strong{BEOPEN PYTHON OPEN SOURCE LICENSE AGREEMENT VERSION 1}}

\begin{enumerate}

\item
This LICENSE AGREEMENT is between BeOpen.com (``BeOpen''), having an
office at 160 Saratoga Avenue, Santa Clara, CA 95051, and the
Individual or Organization (``Licensee'') accessing and otherwise
using this software in source or binary form and its associated
documentation (``the Software'').

\item
Subject to the terms and conditions of this BeOpen Python License
Agreement, BeOpen hereby grants Licensee a non-exclusive,
royalty-free, world-wide license to reproduce, analyze, test, perform
and/or display publicly, prepare derivative works, distribute, and
otherwise use the Software alone or in any derivative version,
provided, however, that the BeOpen Python License is retained in the
Software, alone or in any derivative version prepared by Licensee.

\item
BeOpen is making the Software available to Licensee on an ``AS IS''
basis.  BEOPEN MAKES NO REPRESENTATIONS OR WARRANTIES, EXPRESS OR
IMPLIED.  BY WAY OF EXAMPLE, BUT NOT LIMITATION, BEOPEN MAKES NO AND
DISCLAIMS ANY REPRESENTATION OR WARRANTY OF MERCHANTABILITY OR FITNESS
FOR ANY PARTICULAR PURPOSE OR THAT THE USE OF THE SOFTWARE WILL NOT
INFRINGE ANY THIRD PARTY RIGHTS.

\item
BEOPEN SHALL NOT BE LIABLE TO LICENSEE OR ANY OTHER USERS OF THE
SOFTWARE FOR ANY INCIDENTAL, SPECIAL, OR CONSEQUENTIAL DAMAGES OR LOSS
AS A RESULT OF USING, MODIFYING OR DISTRIBUTING THE SOFTWARE, OR ANY
DERIVATIVE THEREOF, EVEN IF ADVISED OF THE POSSIBILITY THEREOF.

\item
This License Agreement will automatically terminate upon a material
breach of its terms and conditions.

\item
This License Agreement shall be governed by and interpreted in all
respects by the law of the State of California, excluding conflict of
law provisions.  Nothing in this License Agreement shall be deemed to
create any relationship of agency, partnership, or joint venture
between BeOpen and Licensee.  This License Agreement does not grant
permission to use BeOpen trademarks or trade names in a trademark
sense to endorse or promote products or services of Licensee, or any
third party.  As an exception, the ``BeOpen Python'' logos available
at http://www.pythonlabs.com/logos.html may be used according to the
permissions granted on that web page.

\item
By copying, installing or otherwise using the software, Licensee
agrees to be bound by the terms and conditions of this License
Agreement.
\end{enumerate}


\centerline{\strong{CNRI OPEN SOURCE LICENSE AGREEMENT}}

Python 1.6 is made available subject to the terms and conditions in
CNRI's License Agreement.  This Agreement together with Python 1.6 may
be located on the Internet using the following unique, persistent
identifier (known as a handle): 1895.22/1012.  This Agreement may also
be obtained from a proxy server on the Internet using the following
URL: \url{http://hdl.handle.net/1895.22/1012}.


\centerline{\strong{CWI PERMISSIONS STATEMENT AND DISCLAIMER}}

Copyright \copyright{} 1991 - 1995, Stichting Mathematisch Centrum
Amsterdam, The Netherlands.  All rights reserved.

Permission to use, copy, modify, and distribute this software and its
documentation for any purpose and without fee is hereby granted,
provided that the above copyright notice appear in all copies and that
both that copyright notice and this permission notice appear in
supporting documentation, and that the name of Stichting Mathematisch
Centrum or CWI not be used in advertising or publicity pertaining to
distribution of the software without specific, written prior
permission.

STICHTING MATHEMATISCH CENTRUM DISCLAIMS ALL WARRANTIES WITH REGARD TO
THIS SOFTWARE, INCLUDING ALL IMPLIED WARRANTIES OF MERCHANTABILITY AND
FITNESS, IN NO EVENT SHALL STICHTING MATHEMATISCH CENTRUM BE LIABLE
FOR ANY SPECIAL, INDIRECT OR CONSEQUENTIAL DAMAGES OR ANY DAMAGES
WHATSOEVER RESULTING FROM LOSS OF USE, DATA OR PROFITS, WHETHER IN AN
ACTION OF CONTRACT, NEGLIGENCE OR OTHER TORTIOUS ACTION, ARISING OUT
OF OR IN CONNECTION WITH THE USE OR PERFORMANCE OF THIS SOFTWARE.


\begin{abstract}

\noindent
This manual documents the API used by \C{} (or \Cpp{}) programmers who
want to write extension modules or embed Python.  It is a companion to
\emph{Extending and Embedding the Python Interpreter}, which describes
the general principles of extension writing but does not document the
API functions in detail.

\strong{Warning:} The current version of this document is incomplete.
I hope that it is nevertheless useful.  I will continue to work on it,
and release new versions from time to time, independent from Python
source code releases.

\end{abstract}

\tableofcontents

% XXX Consider moving all this back to ext.tex and giving api.tex
% XXX a *really* short intro only.

\chapter{Introduction}
\label{intro}

The Application Programmer's Interface to Python gives \C{} and \Cpp{}
programmers access to the Python interpreter at a variety of levels.
The API is equally usable from \Cpp{}, but for brevity it is generally
referred to as the Python/\C{} API.  There are two fundamentally
different reasons for using the Python/\C{} API.  The first reason is
to write \emph{extension modules} for specific purposes; these are
\C{} modules that extend the Python interpreter.  This is probably the
most common use.  The second reason is to use Python as a component in
a larger application; this technique is generally referred to as
\dfn{embedding} Python in an application.

Writing an extension module is a relatively well-understood process, 
where a ``cookbook'' approach works well.  There are several tools 
that automate the process to some extent.  While people have embedded 
Python in other applications since its early existence, the process of 
embedding Python is less straightforward that writing an extension.  
Python 1.5 introduces a number of new API functions as well as some 
changes to the build process that make embedding much simpler.  
This manual describes the \version\ state of affairs.
% XXX Eventually, take the historical notes out

Many API functions are useful independent of whether you're embedding 
or extending Python; moreover, most applications that embed Python 
will need to provide a custom extension as well, so it's probably a 
good idea to become familiar with writing an extension before 
attempting to embed Python in a real application.

\section{Include Files}
\label{includes}

All function, type and macro definitions needed to use the Python/C
API are included in your code by the following line:

\begin{verbatim}
#include "Python.h"
\end{verbatim}

This implies inclusion of the following standard headers:
\code{<stdio.h>}, \code{<string.h>}, \code{<errno.h>}, and
\code{<stdlib.h>} (if available).

All user visible names defined by Python.h (except those defined by
the included standard headers) have one of the prefixes \samp{Py} or
\samp{_Py}.  Names beginning with \samp{_Py} are for internal use
only.  Structure member names do not have a reserved prefix.

\strong{Important:} user code should never define names that begin
with \samp{Py} or \samp{_Py}.  This confuses the reader, and
jeopardizes the portability of the user code to future Python
versions, which may define additional names beginning with one of
these prefixes.

\section{Objects, Types and Reference Counts}
\label{objects}

Most Python/C API functions have one or more arguments as well as a
return value of type \ctype{PyObject *}.  This type is a pointer
to an opaque data type representing an arbitrary Python
object.  Since all Python object types are treated the same way by the
Python language in most situations (e.g., assignments, scope rules,
and argument passing), it is only fitting that they should be
represented by a single \C{} type.  All Python objects live on the heap:
you never declare an automatic or static variable of type
\ctype{PyObject}, only pointer variables of type \ctype{PyObject *} can 
be declared.

All Python objects (even Python integers) have a \dfn{type} and a
\dfn{reference count}.  An object's type determines what kind of object 
it is (e.g., an integer, a list, or a user-defined function; there are 
many more as explained in the \emph{Python Reference Manual}).  For 
each of the well-known types there is a macro to check whether an 
object is of that type; for instance, \samp{PyList_Check(\var{a})} is
true iff the object pointed to by \var{a} is a Python list.

\subsection{Reference Counts}
\label{refcounts}

The reference count is important because today's computers have a 
finite (and often severely limited) memory size; it counts how many 
different places there are that have a reference to an object.  Such a 
place could be another object, or a global (or static) \C{} variable, or 
a local variable in some \C{} function.  When an object's reference count 
becomes zero, the object is deallocated.  If it contains references to 
other objects, their reference count is decremented.  Those other 
objects may be deallocated in turn, if this decrement makes their 
reference count become zero, and so on.  (There's an obvious problem 
with objects that reference each other here; for now, the solution is 
``don't do that''.)

Reference counts are always manipulated explicitly.  The normal way is 
to use the macro \cfunction{Py_INCREF()} to increment an object's 
reference count by one, and \cfunction{Py_DECREF()} to decrement it by 
one.  The decref macro is considerably more complex than the incref one, 
since it must check whether the reference count becomes zero and then 
cause the object's deallocator, which is a function pointer contained 
in the object's type structure.  The type-specific deallocator takes 
care of decrementing the reference counts for other objects contained 
in the object, and so on, if this is a compound object type such as a 
list.  There's no chance that the reference count can overflow; at 
least as many bits are used to hold the reference count as there are 
distinct memory locations in virtual memory (assuming 
\code{sizeof(long) >= sizeof(char *)}).  Thus, the reference count 
increment is a simple operation.

It is not necessary to increment an object's reference count for every 
local variable that contains a pointer to an object.  In theory, the 
object's reference count goes up by one when the variable is made to 
point to it and it goes down by one when the variable goes out of 
scope.  However, these two cancel each other out, so at the end the 
reference count hasn't changed.  The only real reason to use the 
reference count is to prevent the object from being deallocated as 
long as our variable is pointing to it.  If we know that there is at 
least one other reference to the object that lives at least as long as 
our variable, there is no need to increment the reference count 
temporarily.  An important situation where this arises is in objects 
that are passed as arguments to \C{} functions in an extension module 
that are called from Python; the call mechanism guarantees to hold a 
reference to every argument for the duration of the call.

However, a common pitfall is to extract an object from a list and
hold on to it for a while without incrementing its reference count.
Some other operation might conceivably remove the object from the
list, decrementing its reference count and possible deallocating it.
The real danger is that innocent-looking operations may invoke
arbitrary Python code which could do this; there is a code path which
allows control to flow back to the user from a \cfunction{Py_DECREF()},
so almost any operation is potentially dangerous.

A safe approach is to always use the generic operations (functions 
whose name begins with \samp{PyObject_}, \samp{PyNumber_}, 
\samp{PySequence_} or \samp{PyMapping_}).  These operations always 
increment the reference count of the object they return.  This leaves 
the caller with the responsibility to call \cfunction{Py_DECREF()}
when they are done with the result; this soon becomes second nature.

\subsubsection{Reference Count Details}
\label{refcountDetails}

The reference count behavior of functions in the Python/C API is best 
expelained in terms of \emph{ownership of references}.  Note that we 
talk of owning references, never of owning objects; objects are always 
shared!  When a function owns a reference, it has to dispose of it 
properly --- either by passing ownership on (usually to its caller) or 
by calling \cfunction{Py_DECREF()} or \cfunction{Py_XDECREF()}.  When
a function passes ownership of a reference on to its caller, the
caller is said to receive a \emph{new} reference.  When no ownership
is transferred, the caller is said to \emph{borrow} the reference.
Nothing needs to be done for a borrowed reference.

Conversely, when calling a function passes it a reference to an 
object, there are two possibilities: the function \emph{steals} a 
reference to the object, or it does not.  Few functions steal 
references; the two notable exceptions are
\cfunction{PyList_SetItem()} and \cfunction{PyTuple_SetItem()}, which
steal a reference to the item (but not to the tuple or list into which
the item is put!).  These functions were designed to steal a reference
because of a common idiom for populating a tuple or list with newly
created objects; for example, the code to create the tuple \code{(1,
2, "three")} could look like this (forgetting about error handling for
the moment; a better way to code this is shown below anyway):

\begin{verbatim}
PyObject *t;

t = PyTuple_New(3);
PyTuple_SetItem(t, 0, PyInt_FromLong(1L));
PyTuple_SetItem(t, 1, PyInt_FromLong(2L));
PyTuple_SetItem(t, 2, PyString_FromString("three"));
\end{verbatim}

Incidentally, \cfunction{PyTuple_SetItem()} is the \emph{only} way to
set tuple items; \cfunction{PySequence_SetItem()} and
\cfunction{PyObject_SetItem()} refuse to do this since tuples are an
immutable data type.  You should only use
\cfunction{PyTuple_SetItem()} for tuples that you are creating
yourself.

Equivalent code for populating a list can be written using 
\cfunction{PyList_New()} and \cfunction{PyList_SetItem()}.  Such code
can also use \cfunction{PySequence_SetItem()}; this illustrates the
difference between the two (the extra \cfunction{Py_DECREF()} calls):

\begin{verbatim}
PyObject *l, *x;

l = PyList_New(3);
x = PyInt_FromLong(1L);
PySequence_SetItem(l, 0, x); Py_DECREF(x);
x = PyInt_FromLong(2L);
PySequence_SetItem(l, 1, x); Py_DECREF(x);
x = PyString_FromString("three");
PySequence_SetItem(l, 2, x); Py_DECREF(x);
\end{verbatim}

You might find it strange that the ``recommended'' approach takes more
code.  However, in practice, you will rarely use these ways of
creating and populating a tuple or list.  There's a generic function,
\cfunction{Py_BuildValue()}, that can create most common objects from
\C{} values, directed by a \dfn{format string}.  For example, the
above two blocks of code could be replaced by the following (which
also takes care of the error checking):

\begin{verbatim}
PyObject *t, *l;

t = Py_BuildValue("(iis)", 1, 2, "three");
l = Py_BuildValue("[iis]", 1, 2, "three");
\end{verbatim}

It is much more common to use \cfunction{PyObject_SetItem()} and
friends with items whose references you are only borrowing, like
arguments that were passed in to the function you are writing.  In
that case, their behaviour regarding reference counts is much saner,
since you don't have to increment a reference count so you can give a
reference away (``have it be stolen'').  For example, this function
sets all items of a list (actually, any mutable sequence) to a given
item:

\begin{verbatim}
int set_all(PyObject *target, PyObject *item)
{
    int i, n;

    n = PyObject_Length(target);
    if (n < 0)
        return -1;
    for (i = 0; i < n; i++) {
        if (PyObject_SetItem(target, i, item) < 0)
            return -1;
    }
    return 0;
}
\end{verbatim}

The situation is slightly different for function return values.  
While passing a reference to most functions does not change your 
ownership responsibilities for that reference, many functions that 
return a referece to an object give you ownership of the reference.
The reason is simple: in many cases, the returned object is created 
on the fly, and the reference you get is the only reference to the 
object.  Therefore, the generic functions that return object 
references, like \cfunction{PyObject_GetItem()} and 
\cfunction{PySequence_GetItem()}, always return a new reference (i.e.,
the  caller becomes the owner of the reference).

It is important to realize that whether you own a reference returned 
by a function depends on which function you call only --- \emph{the
plumage} (i.e., the type of the type of the object passed as an
argument to the function) \emph{doesn't enter into it!}  Thus, if you 
extract an item from a list using \cfunction{PyList_GetItem()}, you
don't own the reference --- but if you obtain the same item from the
same list using \cfunction{PySequence_GetItem()} (which happens to
take exactly the same arguments), you do own a reference to the
returned object.

Here is an example of how you could write a function that computes the
sum of the items in a list of integers; once using 
\cfunction{PyList_GetItem()}, once using
\cfunction{PySequence_GetItem()}.

\begin{verbatim}
long sum_list(PyObject *list)
{
    int i, n;
    long total = 0;
    PyObject *item;

    n = PyList_Size(list);
    if (n < 0)
        return -1; /* Not a list */
    for (i = 0; i < n; i++) {
        item = PyList_GetItem(list, i); /* Can't fail */
        if (!PyInt_Check(item)) continue; /* Skip non-integers */
        total += PyInt_AsLong(item);
    }
    return total;
}
\end{verbatim}

\begin{verbatim}
long sum_sequence(PyObject *sequence)
{
    int i, n;
    long total = 0;
    PyObject *item;
    n = PyObject_Size(list);
    if (n < 0)
        return -1; /* Has no length */
    for (i = 0; i < n; i++) {
        item = PySequence_GetItem(list, i);
        if (item == NULL)
            return -1; /* Not a sequence, or other failure */
        if (PyInt_Check(item))
            total += PyInt_AsLong(item);
        Py_DECREF(item); /* Discard reference ownership */
    }
    return total;
}
\end{verbatim}

\subsection{Types}
\label{types}

There are few other data types that play a significant role in 
the Python/C API; most are simple \C{} types such as \ctype{int}, 
\ctype{long}, \ctype{double} and \ctype{char *}.  A few structure types 
are used to describe static tables used to list the functions exported 
by a module or the data attributes of a new object type.  These will 
be discussed together with the functions that use them.

\section{Exceptions}
\label{exceptions}

The Python programmer only needs to deal with exceptions if specific 
error handling is required; unhandled exceptions are automatically 
propagated to the caller, then to the caller's caller, and so on, till 
they reach the top-level interpreter, where they are reported to the 
user accompanied by a stack traceback.

For \C{} programmers, however, error checking always has to be explicit.  
All functions in the Python/C API can raise exceptions, unless an 
explicit claim is made otherwise in a function's documentation.  In 
general, when a function encounters an error, it sets an exception, 
discards any object references that it owns, and returns an 
error indicator --- usually \NULL{} or \code{-1}.  A few functions 
return a Boolean true/false result, with false indicating an error.
Very few functions return no explicit error indicator or have an 
ambiguous return value, and require explicit testing for errors with 
\cfunction{PyErr_Occurred()}.

Exception state is maintained in per-thread storage (this is 
equivalent to using global storage in an unthreaded application).  A 
thread can be in one of two states: an exception has occurred, or not.
The function \cfunction{PyErr_Occurred()} can be used to check for
this: it returns a borrowed reference to the exception type object
when an exception has occurred, and \NULL{} otherwise.  There are a
number of functions to set the exception state:
\cfunction{PyErr_SetString()} is the most common (though not the most
general) function to set the exception state, and
\cfunction{PyErr_Clear()} clears the exception state.

The full exception state consists of three objects (all of which can 
be \NULL{}): the exception type, the corresponding exception 
value, and the traceback.  These have the same meanings as the Python 
object \code{sys.exc_type}, \code{sys.exc_value}, 
\code{sys.exc_traceback}; however, they are not the same: the Python 
objects represent the last exception being handled by a Python 
\keyword{try} \ldots\ \keyword{except} statement, while the \C{} level
exception state only exists while an exception is being passed on
between \C{} functions until it reaches the Python interpreter, which
takes care of transferring it to \code{sys.exc_type} and friends.

Note that starting with Python 1.5, the preferred, thread-safe way to 
access the exception state from Python code is to call the function 
\function{sys.exc_info()}, which returns the per-thread exception state 
for Python code.  Also, the semantics of both ways to access the 
exception state have changed so that a function which catches an 
exception will save and restore its thread's exception state so as to 
preserve the exception state of its caller.  This prevents common bugs 
in exception handling code caused by an innocent-looking function 
overwriting the exception being handled; it also reduces the often 
unwanted lifetime extension for objects that are referenced by the 
stack frames in the traceback.

As a general principle, a function that calls another function to 
perform some task should check whether the called function raised an 
exception, and if so, pass the exception state on to its caller.  It 
should discard any object references that it owns, and returns an 
error indicator, but it should \emph{not} set another exception ---
that would overwrite the exception that was just raised, and lose
important information about the exact cause of the error.

A simple example of detecting exceptions and passing them on is shown 
in the \cfunction{sum_sequence()} example above.  It so happens that
that example doesn't need to clean up any owned references when it
detects an error.  The following example function shows some error
cleanup.  First, to remind you why you like Python, we show the
equivalent Python code:

\begin{verbatim}
def incr_item(dict, key):
    try:
        item = dict[key]
    except KeyError:
        item = 0
    return item + 1
\end{verbatim}

Here is the corresponding \C{} code, in all its glory:

\begin{verbatim}
int incr_item(PyObject *dict, PyObject *key)
{
    /* Objects all initialized to NULL for Py_XDECREF */
    PyObject *item = NULL, *const_one = NULL, *incremented_item = NULL;
    int rv = -1; /* Return value initialized to -1 (failure) */

    item = PyObject_GetItem(dict, key);
    if (item == NULL) {
        /* Handle KeyError only: */
        if (!PyErr_ExceptionMatches(PyExc_KeyError)) goto error;

        /* Clear the error and use zero: */
        PyErr_Clear();
        item = PyInt_FromLong(0L);
        if (item == NULL) goto error;
    }

    const_one = PyInt_FromLong(1L);
    if (const_one == NULL) goto error;

    incremented_item = PyNumber_Add(item, const_one);
    if (incremented_item == NULL) goto error;

    if (PyObject_SetItem(dict, key, incremented_item) < 0) goto error;
    rv = 0; /* Success */
    /* Continue with cleanup code */

 error:
    /* Cleanup code, shared by success and failure path */

    /* Use Py_XDECREF() to ignore NULL references */
    Py_XDECREF(item);
    Py_XDECREF(const_one);
    Py_XDECREF(incremented_item);

    return rv; /* -1 for error, 0 for success */
}
\end{verbatim}

This example represents an endorsed use of the \keyword{goto} statement 
in \C{}!  It illustrates the use of
\cfunction{PyErr_ExceptionMatches()} and \cfunction{PyErr_Clear()} to
handle specific exceptions, and the use of \cfunction{Py_XDECREF()} to
dispose of owned references that may be \NULL{} (note the \samp{X} in
the name; \cfunction{Py_DECREF()} would crash when confronted with a
\NULL{} reference).  It is important that the variables used to hold
owned references are initialized to \NULL{} for this to work;
likewise, the proposed return value is initialized to \code{-1}
(failure) and only set to success after the final call made is
successful.


\section{Embedding Python}
\label{embedding}

The one important task that only embedders (as opposed to extension
writers) of the Python interpreter have to worry about is the
initialization, and possibly the finalization, of the Python
interpreter.  Most functionality of the interpreter can only be used
after the interpreter has been initialized.

The basic initialization function is \cfunction{Py_Initialize()}.
This initializes the table of loaded modules, and creates the
fundamental modules \module{__builtin__}\refbimodindex{__builtin__},
\module{__main__}\refbimodindex{__main__} and 
\module{sys}\refbimodindex{sys}.  It also initializes the module
search path (\code{sys.path}).%
\indexiii{module}{search}{path}

\cfunction{Py_Initialize()} does not set the ``script argument list'' 
(\code{sys.argv}).  If this variable is needed by Python code that 
will be executed later, it must be set explicitly with a call to 
\code{PySys_SetArgv(\var{argc}, \var{argv})} subsequent to the call 
to \cfunction{Py_Initialize()}.

On most systems (in particular, on \UNIX{} and Windows, although the
details are slightly different), \cfunction{Py_Initialize()}
calculates the module search path based upon its best guess for the
location of the standard Python interpreter executable, assuming that
the Python library is found in a fixed location relative to the Python
interpreter executable.  In particular, it looks for a directory named
\file{lib/python1.5} (replacing \file{1.5} with the current
interpreter version) relative to the parent directory where the
executable named \file{python} is found on the shell command search
path (the environment variable \envvar{PATH}).

For instance, if the Python executable is found in
\file{/usr/local/bin/python}, it will assume that the libraries are in
\file{/usr/local/lib/python1.5}.  (In fact, this particular path
is also the ``fallback'' location, used when no executable file named
\file{python} is found along \envvar{PATH}.)  The user can override
this behavior by setting the environment variable \envvar{PYTHONHOME},
or insert additional directories in front of the standard path by
setting \envvar{PYTHONPATH}.

The embedding application can steer the search by calling 
\code{Py_SetProgramName(\var{file})} \emph{before} calling 
\cfunction{Py_Initialize()}.  Note that \envvar{PYTHONHOME} still
overrides this and \envvar{PYTHONPATH} is still inserted in front of
the standard path.  An application that requires total control has to
provide its own implementation of \cfunction{Py_GetPath()},
\cfunction{Py_GetPrefix()}, \cfunction{Py_GetExecPrefix()},
\cfunction{Py_GetProgramFullPath()} (all defined in
\file{Modules/getpath.c}).

Sometimes, it is desirable to ``uninitialize'' Python.  For instance, 
the application may want to start over (make another call to 
\cfunction{Py_Initialize()}) or the application is simply done with its 
use of Python and wants to free all memory allocated by Python.  This
can be accomplished by calling \cfunction{Py_Finalize()}.  The function
\cfunction{Py_IsInitialized()} returns true iff Python is currently in the
initialized state.  More information about these functions is given in
a later chapter.


\chapter{The Very High Level Layer}
\label{veryhigh}

The functions in this chapter will let you execute Python source code
given in a file or a buffer, but they will not let you interact in a
more detailed way with the interpreter.

\begin{cfuncdesc}{int}{PyRun_AnyFile}{FILE *fp, char *filename}
\end{cfuncdesc}

\begin{cfuncdesc}{int}{PyRun_SimpleString}{char *command}
\end{cfuncdesc}

\begin{cfuncdesc}{int}{PyRun_SimpleFile}{FILE *fp, char *filename}
\end{cfuncdesc}

\begin{cfuncdesc}{int}{PyRun_InteractiveOne}{FILE *fp, char *filename}
\end{cfuncdesc}

\begin{cfuncdesc}{int}{PyRun_InteractiveLoop}{FILE *fp, char *filename}
\end{cfuncdesc}

\begin{cfuncdesc}{struct _node*}{PyParser_SimpleParseString}{char *str,
                                                             int start}
\end{cfuncdesc}

\begin{cfuncdesc}{struct _node*}{PyParser_SimpleParseFile}{FILE *fp,
                                 char *filename, int start}
\end{cfuncdesc}

\begin{cfuncdesc}{PyObject*}{PyRun_String}{char *str, int start,
                                           PyObject *globals,
                                           PyObject *locals}
\end{cfuncdesc}

\begin{cfuncdesc}{PyObject*}{PyRun_File}{FILE *fp, char *filename,
                                         int start, PyObject *globals,
                                         PyObject *locals}
\end{cfuncdesc}

\begin{cfuncdesc}{PyObject*}{Py_CompileString}{char *str, char *filename,
                                               int start}
\end{cfuncdesc}


\chapter{Reference Counting}
\label{countingRefs}

The macros in this section are used for managing reference counts
of Python objects.

\begin{cfuncdesc}{void}{Py_INCREF}{PyObject *o}
Increment the reference count for object \var{o}.  The object must
not be \NULL{}; if you aren't sure that it isn't \NULL{}, use
\cfunction{Py_XINCREF()}.
\end{cfuncdesc}

\begin{cfuncdesc}{void}{Py_XINCREF}{PyObject *o}
Increment the reference count for object \var{o}.  The object may be
\NULL{}, in which case the macro has no effect.
\end{cfuncdesc}

\begin{cfuncdesc}{void}{Py_DECREF}{PyObject *o}
Decrement the reference count for object \var{o}.  The object must
not be \NULL{}; if you aren't sure that it isn't \NULL{}, use
\cfunction{Py_XDECREF()}.  If the reference count reaches zero, the
object's type's deallocation function (which must not be \NULL{}) is
invoked.

\strong{Warning:} The deallocation function can cause arbitrary Python
code to be invoked (e.g. when a class instance with a \method{__del__()}
method is deallocated).  While exceptions in such code are not
propagated, the executed code has free access to all Python global
variables.  This means that any object that is reachable from a global
variable should be in a consistent state before \cfunction{Py_DECREF()} is
invoked.  For example, code to delete an object from a list should
copy a reference to the deleted object in a temporary variable, update
the list data structure, and then call \cfunction{Py_DECREF()} for the
temporary variable.
\end{cfuncdesc}

\begin{cfuncdesc}{void}{Py_XDECREF}{PyObject *o}
Decrement the reference count for object \var{o}.  The object may be
\NULL{}, in which case the macro has no effect; otherwise the effect
is the same as for \cfunction{Py_DECREF()}, and the same warning
applies.
\end{cfuncdesc}

The following functions or macros are only for internal use:
\cfunction{_Py_Dealloc()}, \cfunction{_Py_ForgetReference()},
\cfunction{_Py_NewReference()}, as well as the global variable
\cdata{_Py_RefTotal}.

XXX Should mention Py_Malloc(), Py_Realloc(), Py_Free(),
PyMem_Malloc(), PyMem_Realloc(), PyMem_Free(), PyMem_NEW(),
PyMem_RESIZE(), PyMem_DEL(), PyMem_XDEL().


\chapter{Exception Handling}
\label{exceptionHandling}

The functions in this chapter will let you handle and raise Python
exceptions.  It is important to understand some of the basics of
Python exception handling.  It works somewhat like the \UNIX{}
\cdata{errno} variable: there is a global indicator (per thread) of the
last error that occurred.  Most functions don't clear this on success,
but will set it to indicate the cause of the error on failure.  Most
functions also return an error indicator, usually \NULL{} if they are
supposed to return a pointer, or \code{-1} if they return an integer
(exception: the \cfunction{PyArg_Parse*()} functions return \code{1} for
success and \code{0} for failure).  When a function must fail because
some function it called failed, it generally doesn't set the error
indicator; the function it called already set it.

The error indicator consists of three Python objects corresponding to
the Python variables \code{sys.exc_type}, \code{sys.exc_value} and
\code{sys.exc_traceback}.  API functions exist to interact with the
error indicator in various ways.  There is a separate error indicator
for each thread.

% XXX Order of these should be more thoughtful.
% Either alphabetical or some kind of structure.

\begin{cfuncdesc}{void}{PyErr_Print}{}
Print a standard traceback to \code{sys.stderr} and clear the error
indicator.  Call this function only when the error indicator is set.
(Otherwise it will cause a fatal error!)
\end{cfuncdesc}

\begin{cfuncdesc}{PyObject*}{PyErr_Occurred}{}
Test whether the error indicator is set.  If set, return the exception
\emph{type} (the first argument to the last call to one of the
\cfunction{PyErr_Set*()} functions or to \cfunction{PyErr_Restore()}).  If
not set, return \NULL{}.  You do not own a reference to the return
value, so you do not need to \cfunction{Py_DECREF()} it.
\strong{Note:} do not compare the return value to a specific
exception; use \cfunction{PyErr_ExceptionMatches()} instead, shown
below.
\end{cfuncdesc}

\begin{cfuncdesc}{int}{PyErr_ExceptionMatches}{PyObject *exc}
Equivalent to
\samp{PyErr_GivenExceptionMatches(PyErr_Occurred(), \var{exc})}.
This should only be called when an exception is actually set.
\end{cfuncdesc}

\begin{cfuncdesc}{int}{PyErr_GivenExceptionMatches}{PyObject *given, PyObject *exc}
Return true if the \var{given} exception matches the exception in
\var{exc}.  If \var{exc} is a class object, this also returns true
when \var{given} is a subclass.  If \var{exc} is a tuple, all
exceptions in the tuple (and recursively in subtuples) are searched
for a match.  This should only be called when an exception is actually
set.
\end{cfuncdesc}

\begin{cfuncdesc}{void}{PyErr_NormalizeException}{PyObject**exc, PyObject**val, PyObject**tb}
Under certain circumstances, the values returned by
\cfunction{PyErr_Fetch()} below can be ``unnormalized'', meaning that
\code{*\var{exc}} is a class object but \code{*\var{val}} is not an
instance of the  same class.  This function can be used to instantiate
the class in that case.  If the values are already normalized, nothing
happens.
\end{cfuncdesc}

\begin{cfuncdesc}{void}{PyErr_Clear}{}
Clear the error indicator.  If the error indicator is not set, there
is no effect.
\end{cfuncdesc}

\begin{cfuncdesc}{void}{PyErr_Fetch}{PyObject **ptype, PyObject **pvalue, PyObject **ptraceback}
Retrieve the error indicator into three variables whose addresses are
passed.  If the error indicator is not set, set all three variables to
\NULL{}.  If it is set, it will be cleared and you own a reference to
each object retrieved.  The value and traceback object may be \NULL{}
even when the type object is not.  \strong{Note:} this function is
normally only used by code that needs to handle exceptions or by code
that needs to save and restore the error indicator temporarily.
\end{cfuncdesc}

\begin{cfuncdesc}{void}{PyErr_Restore}{PyObject *type, PyObject *value, PyObject *traceback}
Set  the error indicator from the three objects.  If the error
indicator is already set, it is cleared first.  If the objects are
\NULL{}, the error indicator is cleared.  Do not pass a \NULL{} type
and non-\NULL{} value or traceback.  The exception type should be a
string or class; if it is a class, the value should be an instance of
that class.  Do not pass an invalid exception type or value.
(Violating these rules will cause subtle problems later.)  This call
takes away a reference to each object, i.e. you must own a reference
to each object before the call and after the call you no longer own
these references.  (If you don't understand this, don't use this
function.  I warned you.)  \strong{Note:} this function is normally
only used by code that needs to save and restore the error indicator
temporarily.
\end{cfuncdesc}

\begin{cfuncdesc}{void}{PyErr_SetString}{PyObject *type, char *message}
This is the most common way to set the error indicator.  The first
argument specifies the exception type; it is normally one of the
standard exceptions, e.g. \cdata{PyExc_RuntimeError}.  You need not
increment its reference count.  The second argument is an error
message; it is converted to a string object.
\end{cfuncdesc}

\begin{cfuncdesc}{void}{PyErr_SetObject}{PyObject *type, PyObject *value}
This function is similar to \cfunction{PyErr_SetString()} but lets you
specify an arbitrary Python object for the ``value'' of the exception.
You need not increment its reference count.
\end{cfuncdesc}

\begin{cfuncdesc}{void}{PyErr_SetNone}{PyObject *type}
This is a shorthand for \samp{PyErr_SetObject(\var{type}, Py_None)}.
\end{cfuncdesc}

\begin{cfuncdesc}{int}{PyErr_BadArgument}{}
This is a shorthand for \samp{PyErr_SetString(PyExc_TypeError,
\var{message})}, where \var{message} indicates that a built-in operation
was invoked with an illegal argument.  It is mostly for internal use.
\end{cfuncdesc}

\begin{cfuncdesc}{PyObject*}{PyErr_NoMemory}{}
This is a shorthand for \samp{PyErr_SetNone(PyExc_MemoryError)}; it
returns \NULL{} so an object allocation function can write
\samp{return PyErr_NoMemory();} when it runs out of memory.
\end{cfuncdesc}

\begin{cfuncdesc}{PyObject*}{PyErr_SetFromErrno}{PyObject *type}
This is a convenience function to raise an exception when a \C{} library
function has returned an error and set the \C{} variable \cdata{errno}.
It constructs a tuple object whose first item is the integer
\cdata{errno} value and whose second item is the corresponding error
message (gotten from \cfunction{strerror()}), and then calls
\samp{PyErr_SetObject(\var{type}, \var{object})}.  On \UNIX{}, when
the \cdata{errno} value is \constant{EINTR}, indicating an interrupted
system call, this calls \cfunction{PyErr_CheckSignals()}, and if that set
the error indicator, leaves it set to that.  The function always
returns \NULL{}, so a wrapper function around a system call can write 
\samp{return PyErr_SetFromErrno();} when  the system call returns an
error.
\end{cfuncdesc}

\begin{cfuncdesc}{void}{PyErr_BadInternalCall}{}
This is a shorthand for \samp{PyErr_SetString(PyExc_TypeError,
\var{message})}, where \var{message} indicates that an internal
operation (e.g. a Python/C API function) was invoked with an illegal
argument.  It is mostly for internal use.
\end{cfuncdesc}

\begin{cfuncdesc}{int}{PyErr_CheckSignals}{}
This function interacts with Python's signal handling.  It checks
whether a signal has been sent to the processes and if so, invokes the
corresponding signal handler.  If the
\module{signal}\refbimodindex{signal} module is supported, this can
invoke a signal handler written in Python.  In all cases, the default
effect for \constant{SIGINT} is to raise the
\exception{KeyboadInterrupt} exception.  If an exception is raised the 
error indicator is set and the function returns \code{1}; otherwise
the function returns \code{0}.  The error indicator may or may not be
cleared if it was previously set.
\end{cfuncdesc}

\begin{cfuncdesc}{void}{PyErr_SetInterrupt}{}
This function is obsolete (XXX or platform dependent?).  It simulates
the effect of a \constant{SIGINT} signal arriving --- the next time
\cfunction{PyErr_CheckSignals()} is called,
\exception{KeyboadInterrupt} will be raised.
\end{cfuncdesc}

\begin{cfuncdesc}{PyObject*}{PyErr_NewException}{char *name,
                                                 PyObject *base,
                                                 PyObject *dict}
This utility function creates and returns a new exception object.  The
\var{name} argument must be the name of the new exception, a \C{} string
of the form \code{module.class}.  The \var{base} and \var{dict}
arguments are normally \NULL{}.  Normally, this creates a class
object derived from the root for all exceptions, the built-in name
\exception{Exception} (accessible in \C{} as \cdata{PyExc_Exception}).
In this case the \member{__module__} attribute of the new class is set to the
first part (up to the last dot) of the \var{name} argument, and the
class name is set to the last part (after the last dot).  When the
user has specified the \code{-X} command line option to use string
exceptions, for backward compatibility, or when the \var{base}
argument is not a class object (and not \NULL{}), a string object
created from the entire \var{name} argument is returned.  The
\var{base} argument can be used to specify an alternate base class.
The \var{dict} argument can be used to specify a dictionary of class
variables and methods.
\end{cfuncdesc}


\section{Standard Exceptions}
\label{standardExceptions}

All standard Python exceptions are available as global variables whose
names are \samp{PyExc_} followed by the Python exception name.
These have the type \ctype{PyObject *}; they are all either class
objects or string objects, depending on the use of the \code{-X}
option to the interpreter.  For completeness, here are all the
variables:
\cdata{PyExc_Exception},
\cdata{PyExc_StandardError},
\cdata{PyExc_ArithmeticError},
\cdata{PyExc_LookupError},
\cdata{PyExc_AssertionError},
\cdata{PyExc_AttributeError},
\cdata{PyExc_EOFError},
\cdata{PyExc_FloatingPointError},
\cdata{PyExc_IOError},
\cdata{PyExc_ImportError},
\cdata{PyExc_IndexError},
\cdata{PyExc_KeyError},
\cdata{PyExc_KeyboardInterrupt},
\cdata{PyExc_MemoryError},
\cdata{PyExc_NameError},
\cdata{PyExc_OverflowError},
\cdata{PyExc_RuntimeError},
\cdata{PyExc_SyntaxError},
\cdata{PyExc_SystemError},
\cdata{PyExc_SystemExit},
\cdata{PyExc_TypeError},
\cdata{PyExc_ValueError},
\cdata{PyExc_ZeroDivisionError}.


\chapter{Utilities}
\label{utilities}

The functions in this chapter perform various utility tasks, such as
parsing function arguments and constructing Python values from \C{}
values.

\section{OS Utilities}
\label{os}

\begin{cfuncdesc}{int}{Py_FdIsInteractive}{FILE *fp, char *filename}
Return true (nonzero) if the standard I/O file \var{fp} with name
\var{filename} is deemed interactive.  This is the case for files for
which \samp{isatty(fileno(\var{fp}))} is true.  If the global flag
\cdata{Py_InteractiveFlag} is true, this function also returns true if
the \var{name} pointer is \NULL{} or if the name is equal to one of
the strings \code{"<stdin>"} or \code{"???"}.
\end{cfuncdesc}

\begin{cfuncdesc}{long}{PyOS_GetLastModificationTime}{char *filename}
Return the time of last modification of the file \var{filename}.
The result is encoded in the same way as the timestamp returned by
the standard \C{} library function \cfunction{time()}.
\end{cfuncdesc}


\section{Process Control}
\label{processControl}

\begin{cfuncdesc}{void}{Py_FatalError}{char *message}
Print a fatal error message and kill the process.  No cleanup is
performed.  This function should only be invoked when a condition is
detected that would make it dangerous to continue using the Python
interpreter; e.g., when the object administration appears to be
corrupted.  On \UNIX{}, the standard \C{} library function
\cfunction{abort()} is called which will attempt to produce a
\file{core} file.
\end{cfuncdesc}

\begin{cfuncdesc}{void}{Py_Exit}{int status}
Exit the current process.  This calls \cfunction{Py_Finalize()} and
then calls the standard \C{} library function
\code{exit(\var{status})}.
\end{cfuncdesc}

\begin{cfuncdesc}{int}{Py_AtExit}{void (*func) ()}
Register a cleanup function to be called by \cfunction{Py_Finalize()}.
The cleanup function will be called with no arguments and should
return no value.  At most 32 cleanup functions can be registered.
When the registration is successful, \cfunction{Py_AtExit()} returns
\code{0}; on failure, it returns \code{-1}.  The cleanup function
registered last is called first.  Each cleanup function will be called
at most once.  Since Python's internal finallization will have
completed before the cleanup function, no Python APIs should be called
by \var{func}.
\end{cfuncdesc}


\section{Importing Modules}
\label{importing}

\begin{cfuncdesc}{PyObject*}{PyImport_ImportModule}{char *name}
This is a simplified interface to \cfunction{PyImport_ImportModuleEx()}
below, leaving the \var{globals} and \var{locals} arguments set to
\NULL{}.  When the \var{name} argument contains a dot (i.e., when
it specifies a submodule of a package), the \var{fromlist} argument is
set to the list \code{['*']} so that the return value is the named
module rather than the top-level package containing it as would
otherwise be the case.  (Unfortunately, this has an additional side
effect when \var{name} in fact specifies a subpackage instead of a
submodule: the submodules specified in the package's \code{__all__}
variable are loaded.)  Return a new reference to the imported module,
or \NULL{} with an exception set on failure (the module may still
be created in this case --- examine \code{sys.modules} to find out).
\end{cfuncdesc}

\begin{cfuncdesc}{PyObject*}{PyImport_ImportModuleEx}{char *name, PyObject *globals, PyObject *locals, PyObject *fromlist}
Import a module.  This is best described by referring to the built-in
Python function \function{__import__()}\bifuncindex{__import__}, as
the standard \function{__import__()} function calls this function
directly.

The return value is a new reference to the imported module or
top-level package, or \NULL{} with an exception set on failure
(the module may still be created in this case).  Like for
\function{__import__()}, the return value when a submodule of a
package was requested is normally the top-level package, unless a
non-empty \var{fromlist} was given.
\end{cfuncdesc}

\begin{cfuncdesc}{PyObject*}{PyImport_Import}{PyObject *name}
This is a higher-level interface that calls the current ``import hook
function''.  It invokes the \function{__import__()} function from the
\code{__builtins__} of the current globals.  This means that the
import is done using whatever import hooks are installed in the
current environment, e.g. by \module{rexec}\refstmodindex{rexec} or
\module{ihooks}\refstmodindex{ihooks}.
\end{cfuncdesc}

\begin{cfuncdesc}{PyObject*}{PyImport_ReloadModule}{PyObject *m}
Reload a module.  This is best described by referring to the built-in
Python function \function{reload()}\bifuncindex{reload}, as the standard
\function{reload()} function calls this function directly.  Return a
new reference to the reloaded module, or \NULL{} with an exception set
on failure (the module still exists in this case).
\end{cfuncdesc}

\begin{cfuncdesc}{PyObject*}{PyImport_AddModule}{char *name}
Return the module object corresponding to a module name.  The
\var{name} argument may be of the form \code{package.module}).  First
check the modules dictionary if there's one there, and if not, create
a new one and insert in in the modules dictionary.  Because the former
action is most common, this does not return a new reference, and you
do not own the returned reference.  Return \NULL{} with an
exception set on failure.  \strong{Note:} this function returns
a ``borrowed'' reference.  
\end{cfuncdesc}

\begin{cfuncdesc}{PyObject*}{PyImport_ExecCodeModule}{char *name, PyObject *co}
Given a module name (possibly of the form \code{package.module}) and a
code object read from a Python bytecode file or obtained from the
built-in function \function{compile()}\bifuncindex{compile}, load the
module.  Return a new reference to the module object, or \NULL{} with
an exception set if an error occurred (the module may still be created
in this case).  (This function would reload the module if it was
already imported.)
\end{cfuncdesc}

\begin{cfuncdesc}{long}{PyImport_GetMagicNumber}{}
Return the magic number for Python bytecode files (a.k.a. \file{.pyc}
and \file{.pyo} files).  The magic number should be present in the
first four bytes of the bytecode file, in little-endian byte order.
\end{cfuncdesc}

\begin{cfuncdesc}{PyObject*}{PyImport_GetModuleDict}{}
Return the dictionary used for the module administration
(a.k.a. \code{sys.modules}).  Note that this is a per-interpreter
variable.
\end{cfuncdesc}

\begin{cfuncdesc}{void}{_PyImport_Init}{}
Initialize the import mechanism.  For internal use only.
\end{cfuncdesc}

\begin{cfuncdesc}{void}{PyImport_Cleanup}{}
Empty the module table.  For internal use only.
\end{cfuncdesc}

\begin{cfuncdesc}{void}{_PyImport_Fini}{}
Finalize the import mechanism.  For internal use only.
\end{cfuncdesc}

\begin{cfuncdesc}{PyObject*}{_PyImport_FindExtension}{char *, char *}
For internal use only.
\end{cfuncdesc}

\begin{cfuncdesc}{PyObject*}{_PyImport_FixupExtension}{char *, char *}
For internal use only.
\end{cfuncdesc}

\begin{cfuncdesc}{int}{PyImport_ImportFrozenModule}{char *}
Load a frozen module.  Return \code{1} for success, \code{0} if the
module is not found, and \code{-1} with an exception set if the
initialization failed.  To access the imported module on a successful
load, use \cfunction{PyImport_ImportModule()}.
(Note the misnomer --- this function would reload the module if it was
already imported.)
\end{cfuncdesc}

\begin{ctypedesc}{struct _frozen}
This is the structure type definition for frozen module descriptors,
as generated by the \program{freeze}\index{freeze utility} utility
(see \file{Tools/freeze/} in the Python source distribution).  Its
definition is:

\begin{verbatim}
struct _frozen {
    char *name;
    unsigned char *code;
    int size;
};
\end{verbatim}
\end{ctypedesc}

\begin{cvardesc}{struct _frozen*}{PyImport_FrozenModules}
This pointer is initialized to point to an array of \ctype{struct
_frozen} records, terminated by one whose members are all \NULL{}
or zero.  When a frozen module is imported, it is searched in this
table.  Third-party code could play tricks with this to provide a
dynamically created collection of frozen modules.
\end{cvardesc}


\chapter{Abstract Objects Layer}
\label{abstract}

The functions in this chapter interact with Python objects regardless
of their type, or with wide classes of object types (e.g. all
numerical types, or all sequence types).  When used on object types
for which they do not apply, they will flag a Python exception.

\section{Object Protocol}
\label{object}

\begin{cfuncdesc}{int}{PyObject_Print}{PyObject *o, FILE *fp, int flags}
Print an object \var{o}, on file \var{fp}.  Returns \code{-1} on error
The flags argument is used to enable certain printing
options. The only option currently supported is
\constant{Py_PRINT_RAW}.
\end{cfuncdesc}

\begin{cfuncdesc}{int}{PyObject_HasAttrString}{PyObject *o, char *attr_name}
Returns \code{1} if \var{o} has the attribute \var{attr_name}, and
\code{0} otherwise.  This is equivalent to the Python expression
\samp{hasattr(\var{o}, \var{attr_name})}.
This function always succeeds.
\end{cfuncdesc}

\begin{cfuncdesc}{PyObject*}{PyObject_GetAttrString}{PyObject *o, char *attr_name}
Retrieve an attribute named \var{attr_name} from object \var{o}.
Returns the attribute value on success, or \NULL{} on failure.
This is the equivalent of the Python expression
\samp{\var{o}.\var{attr_name}}.
\end{cfuncdesc}


\begin{cfuncdesc}{int}{PyObject_HasAttr}{PyObject *o, PyObject *attr_name}
Returns \code{1} if \var{o} has the attribute \var{attr_name}, and
\code{0} otherwise.  This is equivalent to the Python expression
\samp{hasattr(\var{o}, \var{attr_name})}. 
This function always succeeds.
\end{cfuncdesc}


\begin{cfuncdesc}{PyObject*}{PyObject_GetAttr}{PyObject *o, PyObject *attr_name}
Retrieve an attribute named \var{attr_name} from object \var{o}.
Returns the attribute value on success, or \NULL{} on failure.
This is the equivalent of the Python expression
\samp{\var{o}.\var{attr_name}}.
\end{cfuncdesc}


\begin{cfuncdesc}{int}{PyObject_SetAttrString}{PyObject *o, char *attr_name, PyObject *v}
Set the value of the attribute named \var{attr_name}, for object
\var{o}, to the value \var{v}. Returns \code{-1} on failure.  This is
the equivalent of the Python statement \samp{\var{o}.\var{attr_name} =
\var{v}}.
\end{cfuncdesc}


\begin{cfuncdesc}{int}{PyObject_SetAttr}{PyObject *o, PyObject *attr_name, PyObject *v}
Set the value of the attribute named \var{attr_name}, for
object \var{o},
to the value \var{v}. Returns \code{-1} on failure.  This is
the equivalent of the Python statement \samp{\var{o}.\var{attr_name} =
\var{v}}.
\end{cfuncdesc}


\begin{cfuncdesc}{int}{PyObject_DelAttrString}{PyObject *o, char *attr_name}
Delete attribute named \var{attr_name}, for object \var{o}. Returns
\code{-1} on failure.  This is the equivalent of the Python
statement: \samp{del \var{o}.\var{attr_name}}.
\end{cfuncdesc}


\begin{cfuncdesc}{int}{PyObject_DelAttr}{PyObject *o, PyObject *attr_name}
Delete attribute named \var{attr_name}, for object \var{o}. Returns
\code{-1} on failure.  This is the equivalent of the Python
statement \samp{del \var{o}.\var{attr_name}}.
\end{cfuncdesc}


\begin{cfuncdesc}{int}{PyObject_Cmp}{PyObject *o1, PyObject *o2, int *result}
Compare the values of \var{o1} and \var{o2} using a routine provided
by \var{o1}, if one exists, otherwise with a routine provided by
\var{o2}.  The result of the comparison is returned in \var{result}.
Returns \code{-1} on failure.  This is the equivalent of the Python
statement \samp{\var{result} = cmp(\var{o1}, \var{o2})}.
\end{cfuncdesc}


\begin{cfuncdesc}{int}{PyObject_Compare}{PyObject *o1, PyObject *o2}
Compare the values of \var{o1} and \var{o2} using a routine provided
by \var{o1}, if one exists, otherwise with a routine provided by
\var{o2}.  Returns the result of the comparison on success.  On error,
the value returned is undefined; use \cfunction{PyErr_Occurred()} to
detect an error.  This is equivalent to the
Python expression \samp{cmp(\var{o1}, \var{o2})}.
\end{cfuncdesc}


\begin{cfuncdesc}{PyObject*}{PyObject_Repr}{PyObject *o}
Compute the string representation of object, \var{o}.  Returns the
string representation on success, \NULL{} on failure.  This is
the equivalent of the Python expression \samp{repr(\var{o})}.
Called by the \function{repr()}\bifuncindex{repr} built-in function
and by reverse quotes.
\end{cfuncdesc}


\begin{cfuncdesc}{PyObject*}{PyObject_Str}{PyObject *o}
Compute the string representation of object \var{o}.  Returns the
string representation on success, \NULL{} on failure.  This is
the equivalent of the Python expression \samp{str(\var{o})}.
Called by the \function{str()}\bifuncindex{str} built-in function and
by the \keyword{print} statement.
\end{cfuncdesc}


\begin{cfuncdesc}{int}{PyCallable_Check}{PyObject *o}
Determine if the object \var{o}, is callable.  Return \code{1} if the
object is callable and \code{0} otherwise.
This function always succeeds.
\end{cfuncdesc}


\begin{cfuncdesc}{PyObject*}{PyObject_CallObject}{PyObject *callable_object, PyObject *args}
Call a callable Python object \var{callable_object}, with
arguments given by the tuple \var{args}.  If no arguments are
needed, then args may be \NULL{}.  Returns the result of the
call on success, or \NULL{} on failure.  This is the equivalent
of the Python expression \samp{apply(\var{o}, \var{args})}.
\end{cfuncdesc}

\begin{cfuncdesc}{PyObject*}{PyObject_CallFunction}{PyObject *callable_object, char *format, ...}
Call a callable Python object \var{callable_object}, with a
variable number of \C{} arguments. The \C{} arguments are described
using a \cfunction{Py_BuildValue()} style format string. The format may
be \NULL{}, indicating that no arguments are provided.  Returns the
result of the call on success, or \NULL{} on failure.  This is
the equivalent of the Python expression \samp{apply(\var{o},
\var{args})}.
\end{cfuncdesc}


\begin{cfuncdesc}{PyObject*}{PyObject_CallMethod}{PyObject *o, char *m, char *format, ...}
Call the method named \var{m} of object \var{o} with a variable number
of C arguments.  The \C{} arguments are described by a
\cfunction{Py_BuildValue()} format string.  The format may be \NULL{},
indicating that no arguments are provided. Returns the result of the
call on success, or \NULL{} on failure.  This is the equivalent of the
Python expression \samp{\var{o}.\var{method}(\var{args})}.
Note that Special method names, such as \method{__add__()},
\method{__getitem__()}, and so on are not supported. The specific
abstract-object routines for these must be used.
\end{cfuncdesc}


\begin{cfuncdesc}{int}{PyObject_Hash}{PyObject *o}
Compute and return the hash value of an object \var{o}.  On
failure, return \code{-1}.  This is the equivalent of the Python
expression \samp{hash(\var{o})}.
\end{cfuncdesc}


\begin{cfuncdesc}{int}{PyObject_IsTrue}{PyObject *o}
Returns \code{1} if the object \var{o} is considered to be true, and
\code{0} otherwise. This is equivalent to the Python expression
\samp{not not \var{o}}.
This function always succeeds.
\end{cfuncdesc}


\begin{cfuncdesc}{PyObject*}{PyObject_Type}{PyObject *o}
On success, returns a type object corresponding to the object
type of object \var{o}. On failure, returns \NULL{}.  This is
equivalent to the Python expression \samp{type(\var{o})}.
\bifuncindex{type}
\end{cfuncdesc}

\begin{cfuncdesc}{int}{PyObject_Length}{PyObject *o}
Return the length of object \var{o}.  If the object \var{o} provides
both sequence and mapping protocols, the sequence length is
returned. On error, \code{-1} is returned.  This is the equivalent
to the Python expression \samp{len(\var{o})}.
\end{cfuncdesc}


\begin{cfuncdesc}{PyObject*}{PyObject_GetItem}{PyObject *o, PyObject *key}
Return element of \var{o} corresponding to the object \var{key} or
\NULL{} on failure. This is the equivalent of the Python expression
\samp{\var{o}[\var{key}]}.
\end{cfuncdesc}


\begin{cfuncdesc}{int}{PyObject_SetItem}{PyObject *o, PyObject *key, PyObject *v}
Map the object \var{key} to the value \var{v}.
Returns \code{-1} on failure.  This is the equivalent
of the Python statement \samp{\var{o}[\var{key}] = \var{v}}.
\end{cfuncdesc}


\begin{cfuncdesc}{int}{PyObject_DelItem}{PyObject *o, PyObject *key, PyObject *v}
Delete the mapping for \var{key} from \var{o}.  Returns \code{-1} on
failure. This is the equivalent of the Python statement \samp{del
\var{o}[\var{key}]}.
\end{cfuncdesc}


\section{Number Protocol}
\label{number}

\begin{cfuncdesc}{int}{PyNumber_Check}{PyObject *o}
Returns \code{1} if the object \var{o} provides numeric protocols, and
false otherwise. 
This function always succeeds.
\end{cfuncdesc}


\begin{cfuncdesc}{PyObject*}{PyNumber_Add}{PyObject *o1, PyObject *o2}
Returns the result of adding \var{o1} and \var{o2}, or \NULL{} on
failure.  This is the equivalent of the Python expression
\samp{\var{o1} + \var{o2}}.
\end{cfuncdesc}


\begin{cfuncdesc}{PyObject*}{PyNumber_Subtract}{PyObject *o1, PyObject *o2}
Returns the result of subtracting \var{o2} from \var{o1}, or \NULL{}
on failure.  This is the equivalent of the Python expression
\samp{\var{o1} - \var{o2}}.
\end{cfuncdesc}


\begin{cfuncdesc}{PyObject*}{PyNumber_Multiply}{PyObject *o1, PyObject *o2}
Returns the result of multiplying \var{o1} and \var{o2}, or \NULL{} on
failure.  This is the equivalent of the Python expression
\samp{\var{o1} * \var{o2}}.
\end{cfuncdesc}


\begin{cfuncdesc}{PyObject*}{PyNumber_Divide}{PyObject *o1, PyObject *o2}
Returns the result of dividing \var{o1} by \var{o2}, or \NULL{} on
failure. 
This is the equivalent of the Python expression \samp{\var{o1} /
\var{o2}}.
\end{cfuncdesc}


\begin{cfuncdesc}{PyObject*}{PyNumber_Remainder}{PyObject *o1, PyObject *o2}
Returns the remainder of dividing \var{o1} by \var{o2}, or \NULL{} on
failure.  This is the equivalent of the Python expression
\samp{\var{o1} \% \var{o2}}.
\end{cfuncdesc}


\begin{cfuncdesc}{PyObject*}{PyNumber_Divmod}{PyObject *o1, PyObject *o2}
See the built-in function \function{divmod()}\bifuncindex{divmod}.
Returns \NULL{} on failure.  This is the equivalent of the Python
expression \samp{divmod(\var{o1}, \var{o2})}.
\end{cfuncdesc}


\begin{cfuncdesc}{PyObject*}{PyNumber_Power}{PyObject *o1, PyObject *o2, PyObject *o3}
See the built-in function \function{pow()}\bifuncindex{pow}.  Returns
\NULL{} on failure. This is the equivalent of the Python expression
\samp{pow(\var{o1}, \var{o2}, \var{o3})}, where \var{o3} is optional.
If \var{o3} is to be ignored, pass \cdata{Py_None} in its place.
\end{cfuncdesc}


\begin{cfuncdesc}{PyObject*}{PyNumber_Negative}{PyObject *o}
Returns the negation of \var{o} on success, or \NULL{} on failure.
This is the equivalent of the Python expression \samp{-\var{o}}.
\end{cfuncdesc}


\begin{cfuncdesc}{PyObject*}{PyNumber_Positive}{PyObject *o}
Returns \var{o} on success, or \NULL{} on failure.
This is the equivalent of the Python expression \samp{+\var{o}}.
\end{cfuncdesc}


\begin{cfuncdesc}{PyObject*}{PyNumber_Absolute}{PyObject *o}
Returns the absolute value of \var{o}, or \NULL{} on failure.  This is
the equivalent of the Python expression \samp{abs(\var{o})}.
\end{cfuncdesc}


\begin{cfuncdesc}{PyObject*}{PyNumber_Invert}{PyObject *o}
Returns the bitwise negation of \var{o} on success, or \NULL{} on
failure.  This is the equivalent of the Python expression
\samp{\~\var{o}}.
\end{cfuncdesc}


\begin{cfuncdesc}{PyObject*}{PyNumber_Lshift}{PyObject *o1, PyObject *o2}
Returns the result of left shifting \var{o1} by \var{o2} on success,
or \NULL{} on failure.  This is the equivalent of the Python
expression \samp{\var{o1} << \var{o2}}.
\end{cfuncdesc}


\begin{cfuncdesc}{PyObject*}{PyNumber_Rshift}{PyObject *o1, PyObject *o2}
Returns the result of right shifting \var{o1} by \var{o2} on success,
or \NULL{} on failure.  This is the equivalent of the Python
expression \samp{\var{o1} >> \var{o2}}.
\end{cfuncdesc}


\begin{cfuncdesc}{PyObject*}{PyNumber_And}{PyObject *o1, PyObject *o2}
Returns the result of ``anding'' \var{o2} and \var{o2} on success and
\NULL{} on failure. This is the equivalent of the Python
expression \samp{\var{o1} and \var{o2}}.
\end{cfuncdesc}


\begin{cfuncdesc}{PyObject*}{PyNumber_Xor}{PyObject *o1, PyObject *o2}
Returns the bitwise exclusive or of \var{o1} by \var{o2} on success,
or \NULL{} on failure.  This is the equivalent of the Python
expression \samp{\var{o1} \^{ }\var{o2}}.
\end{cfuncdesc}

\begin{cfuncdesc}{PyObject*}{PyNumber_Or}{PyObject *o1, PyObject *o2}
Returns the result of \var{o1} and \var{o2} on success, or \NULL{} on
failure.  This is the equivalent of the Python expression
\samp{\var{o1} or \var{o2}}.
\end{cfuncdesc}


\begin{cfuncdesc}{PyObject*}{PyNumber_Coerce}{PyObject **p1, PyObject **p2}
This function takes the addresses of two variables of type
\ctype{PyObject*}.

If the objects pointed to by \code{*\var{p1}} and \code{*\var{p2}}
have the same type, increment their reference count and return
\code{0} (success). If the objects can be converted to a common
numeric type, replace \code{*p1} and \code{*p2} by their converted
value (with 'new' reference counts), and return \code{0}.
If no conversion is possible, or if some other error occurs,
return \code{-1} (failure) and don't increment the reference counts.
The call \code{PyNumber_Coerce(\&o1, \&o2)} is equivalent to the
Python statement \samp{\var{o1}, \var{o2} = coerce(\var{o1},
\var{o2})}.
\end{cfuncdesc}


\begin{cfuncdesc}{PyObject*}{PyNumber_Int}{PyObject *o}
Returns the \var{o} converted to an integer object on success, or
\NULL{} on failure.  This is the equivalent of the Python
expression \samp{int(\var{o})}.
\end{cfuncdesc}


\begin{cfuncdesc}{PyObject*}{PyNumber_Long}{PyObject *o}
Returns the \var{o} converted to a long integer object on success,
or \NULL{} on failure.  This is the equivalent of the Python
expression \samp{long(\var{o})}.
\end{cfuncdesc}


\begin{cfuncdesc}{PyObject*}{PyNumber_Float}{PyObject *o}
Returns the \var{o} converted to a float object on success, or \NULL{}
on failure.  This is the equivalent of the Python expression
\samp{float(\var{o})}.
\end{cfuncdesc}


\section{Sequence Protocol}
\label{sequence}

\begin{cfuncdesc}{int}{PySequence_Check}{PyObject *o}
Return \code{1} if the object provides sequence protocol, and \code{0}
otherwise.  
This function always succeeds.
\end{cfuncdesc}


\begin{cfuncdesc}{PyObject*}{PySequence_Concat}{PyObject *o1, PyObject *o2}
Return the concatenation of \var{o1} and \var{o2} on success, and \NULL{} on
failure.   This is the equivalent of the Python
expression \samp{\var{o1} + \var{o2}}.
\end{cfuncdesc}


\begin{cfuncdesc}{PyObject*}{PySequence_Repeat}{PyObject *o, int count}
Return the result of repeating sequence object \var{o} \var{count}
times, or \NULL{} on failure.  This is the equivalent of the Python
expression \samp{\var{o} * \var{count}}.
\end{cfuncdesc}


\begin{cfuncdesc}{PyObject*}{PySequence_GetItem}{PyObject *o, int i}
Return the \var{i}th element of \var{o}, or \NULL{} on failure. This
is the equivalent of the Python expression \samp{\var{o}[\var{i}]}.
\end{cfuncdesc}


\begin{cfuncdesc}{PyObject*}{PySequence_GetSlice}{PyObject *o, int i1, int i2}
Return the slice of sequence object \var{o} between \var{i1} and
\var{i2}, or \NULL{} on failure. This is the equivalent of the Python
expression \samp{\var{o}[\var{i1}:\var{i2}]}.
\end{cfuncdesc}


\begin{cfuncdesc}{int}{PySequence_SetItem}{PyObject *o, int i, PyObject *v}
Assign object \var{v} to the \var{i}th element of \var{o}.
Returns \code{-1} on failure.  This is the equivalent of the Python
statement \samp{\var{o}[\var{i}] = \var{v}}.
\end{cfuncdesc}

\begin{cfuncdesc}{int}{PySequence_DelItem}{PyObject *o, int i}
Delete the \var{i}th element of object \var{v}.  Returns
\code{-1} on failure.  This is the equivalent of the Python
statement \samp{del \var{o}[\var{i}]}.
\end{cfuncdesc}

\begin{cfuncdesc}{int}{PySequence_SetSlice}{PyObject *o, int i1, int i2, PyObject *v}
Assign the sequence object \var{v} to the slice in sequence
object \var{o} from \var{i1} to \var{i2}.  This is the equivalent of
the Python statement \samp{\var{o}[\var{i1}:\var{i2}] = \var{v}}.
\end{cfuncdesc}

\begin{cfuncdesc}{int}{PySequence_DelSlice}{PyObject *o, int i1, int i2}
Delete the slice in sequence object \var{o} from \var{i1} to \var{i2}.
Returns \code{-1} on failure. This is the equivalent of the Python
statement \samp{del \var{o}[\var{i1}:\var{i2}]}.
\end{cfuncdesc}

\begin{cfuncdesc}{PyObject*}{PySequence_Tuple}{PyObject *o}
Returns the \var{o} as a tuple on success, and \NULL{} on failure.
This is equivalent to the Python expression \code{tuple(\var{o})}.
\end{cfuncdesc}

\begin{cfuncdesc}{int}{PySequence_Count}{PyObject *o, PyObject *value}
Return the number of occurrences of \var{value} in \var{o}, that is,
return the number of keys for which \code{\var{o}[\var{key}] ==
\var{value}}.  On failure, return \code{-1}.  This is equivalent to
the Python expression \samp{\var{o}.count(\var{value})}.
\end{cfuncdesc}

\begin{cfuncdesc}{int}{PySequence_In}{PyObject *o, PyObject *value}
Determine if \var{o} contains \var{value}.  If an item in \var{o} is
equal to \var{value}, return \code{1}, otherwise return \code{0}.  On
error, return \code{-1}.  This is equivalent to the Python expression
\samp{\var{value} in \var{o}}.
\end{cfuncdesc}

\begin{cfuncdesc}{int}{PySequence_Index}{PyObject *o, PyObject *value}
Return the first index \var{i} for which \code{\var{o}[\var{i}] ==
\var{value}}.  On error, return \code{-1}.    This is equivalent to
the Python expression \samp{\var{o}.index(\var{value})}.
\end{cfuncdesc}


\section{Mapping Protocol}
\label{mapping}

\begin{cfuncdesc}{int}{PyMapping_Check}{PyObject *o}
Return \code{1} if the object provides mapping protocol, and \code{0}
otherwise.  
This function always succeeds.
\end{cfuncdesc}


\begin{cfuncdesc}{int}{PyMapping_Length}{PyObject *o}
Returns the number of keys in object \var{o} on success, and \code{-1}
on failure.  For objects that do not provide sequence protocol,
this is equivalent to the Python expression \samp{len(\var{o})}.
\end{cfuncdesc}


\begin{cfuncdesc}{int}{PyMapping_DelItemString}{PyObject *o, char *key}
Remove the mapping for object \var{key} from the object \var{o}.
Return \code{-1} on failure.  This is equivalent to
the Python statement \samp{del \var{o}[\var{key}]}.
\end{cfuncdesc}


\begin{cfuncdesc}{int}{PyMapping_DelItem}{PyObject *o, PyObject *key}
Remove the mapping for object \var{key} from the object \var{o}.
Return \code{-1} on failure.  This is equivalent to
the Python statement \samp{del \var{o}[\var{key}]}.
\end{cfuncdesc}


\begin{cfuncdesc}{int}{PyMapping_HasKeyString}{PyObject *o, char *key}
On success, return \code{1} if the mapping object has the key \var{key}
and \code{0} otherwise.  This is equivalent to the Python expression
\samp{\var{o}.has_key(\var{key})}. 
This function always succeeds.
\end{cfuncdesc}


\begin{cfuncdesc}{int}{PyMapping_HasKey}{PyObject *o, PyObject *key}
Return \code{1} if the mapping object has the key \var{key} and
\code{0} otherwise.  This is equivalent to the Python expression
\samp{\var{o}.has_key(\var{key})}. 
This function always succeeds.
\end{cfuncdesc}


\begin{cfuncdesc}{PyObject*}{PyMapping_Keys}{PyObject *o}
On success, return a list of the keys in object \var{o}.  On
failure, return \NULL{}. This is equivalent to the Python
expression \samp{\var{o}.keys()}.
\end{cfuncdesc}


\begin{cfuncdesc}{PyObject*}{PyMapping_Values}{PyObject *o}
On success, return a list of the values in object \var{o}.  On
failure, return \NULL{}. This is equivalent to the Python
expression \samp{\var{o}.values()}.
\end{cfuncdesc}


\begin{cfuncdesc}{PyObject*}{PyMapping_Items}{PyObject *o}
On success, return a list of the items in object \var{o}, where
each item is a tuple containing a key-value pair.  On
failure, return \NULL{}. This is equivalent to the Python
expression \samp{\var{o}.items()}.
\end{cfuncdesc}

\begin{cfuncdesc}{int}{PyMapping_Clear}{PyObject *o}
Make object \var{o} empty.  Returns \code{1} on success and \code{0}
on failure.  This is equivalent to the Python statement
\samp{for key in \var{o}.keys(): del \var{o}[key]}.
\end{cfuncdesc}


\begin{cfuncdesc}{PyObject*}{PyMapping_GetItemString}{PyObject *o, char *key}
Return element of \var{o} corresponding to the object \var{key} or
\NULL{} on failure. This is the equivalent of the Python expression
\samp{\var{o}[\var{key}]}.
\end{cfuncdesc}

\begin{cfuncdesc}{PyObject*}{PyMapping_SetItemString}{PyObject *o, char *key, PyObject *v}
Map the object \var{key} to the value \var{v} in object \var{o}.
Returns \code{-1} on failure.  This is the equivalent of the Python
statement \samp{\var{o}[\var{key}] = \var{v}}.
\end{cfuncdesc}


\section{Constructors}

\begin{cfuncdesc}{PyObject*}{PyFile_FromString}{char *file_name, char *mode}
On success, returns a new file object that is opened on the
file given by \var{file_name}, with a file mode given by \var{mode},
where \var{mode} has the same semantics as the standard \C{} routine
\cfunction{fopen()}.  On failure, return \code{-1}.
\end{cfuncdesc}

\begin{cfuncdesc}{PyObject*}{PyFile_FromFile}{FILE *fp, char *file_name, char *mode, int close_on_del}
Return a new file object for an already opened standard \C{} file
pointer, \var{fp}.  A file name, \var{file_name}, and open mode,
\var{mode}, must be provided as well as a flag, \var{close_on_del},
that indicates whether the file is to be closed when the file object
is destroyed.  On failure, return \code{-1}.
\end{cfuncdesc}

\begin{cfuncdesc}{PyObject*}{PyFloat_FromDouble}{double v}
Returns a new float object with the value \var{v} on success, and
\NULL{} on failure.
\end{cfuncdesc}

\begin{cfuncdesc}{PyObject*}{PyInt_FromLong}{long v}
Returns a new int object with the value \var{v} on success, and
\NULL{} on failure.
\end{cfuncdesc}

\begin{cfuncdesc}{PyObject*}{PyList_New}{int len}
Returns a new list of length \var{len} on success, and \NULL{} on
failure.
\end{cfuncdesc}

\begin{cfuncdesc}{PyObject*}{PyLong_FromLong}{long v}
Returns a new long object with the value \var{v} on success, and
\NULL{} on failure.
\end{cfuncdesc}

\begin{cfuncdesc}{PyObject*}{PyLong_FromDouble}{double v}
Returns a new long object with the value \var{v} on success, and
\NULL{} on failure.
\end{cfuncdesc}

\begin{cfuncdesc}{PyObject*}{PyDict_New}{}
Returns a new empty dictionary on success, and \NULL{} on
failure.
\end{cfuncdesc}

\begin{cfuncdesc}{PyObject*}{PyString_FromString}{char *v}
Returns a new string object with the value \var{v} on success, and
\NULL{} on failure.
\end{cfuncdesc}

\begin{cfuncdesc}{PyObject*}{PyString_FromStringAndSize}{char *v, int len}
Returns a new string object with the value \var{v} and length
\var{len} on success, and \NULL{} on failure.  If \var{v} is \NULL{},
the contents of the string are uninitialized.
\end{cfuncdesc}

\begin{cfuncdesc}{PyObject*}{PyTuple_New}{int len}
Returns a new tuple of length \var{len} on success, and \NULL{} on
failure.
\end{cfuncdesc}


\chapter{Concrete Objects Layer}
\label{concrete}

The functions in this chapter are specific to certain Python object
types.  Passing them an object of the wrong type is not a good idea;
if you receive an object from a Python program and you are not sure
that it has the right type, you must perform a type check first;
e.g. to check that an object is a dictionary, use
\cfunction{PyDict_Check()}.  The chapter is structured like the
``family tree'' of Python object types.


\section{Fundamental Objects}
\label{fundamental}

This section describes Python type objects and the singleton object 
\code{None}.


\subsection{Type Objects}
\label{typeObjects}

\begin{ctypedesc}{PyTypeObject}

\end{ctypedesc}

\begin{cvardesc}{PyObject *}{PyType_Type}

\end{cvardesc}


\subsection{The None Object}
\label{noneObject}

\begin{cvardesc}{PyObject *}{Py_None}
The Python \code{None} object, denoting lack of value.  This object has
no methods.
\end{cvardesc}


\section{Sequence Objects}
\label{sequenceObjects}

Generic operations on sequence objects were discussed in the previous 
chapter; this section deals with the specific kinds of sequence 
objects that are intrinsic to the Python language.


\subsection{String Objects}
\label{stringObjects}

\begin{ctypedesc}{PyStringObject}
This subtype of \ctype{PyObject} represents a Python string object.
\end{ctypedesc}

\begin{cvardesc}{PyTypeObject}{PyString_Type}
This instance of \ctype{PyTypeObject} represents the Python string type.
\end{cvardesc}

\begin{cfuncdesc}{int}{PyString_Check}{PyObject *o}
Returns true if the object \var{o} is a string object.
\end{cfuncdesc}

\begin{cfuncdesc}{PyObject*}{PyString_FromStringAndSize}{const char *v,
                                                          int len}
Returns a new string object with the value \var{v} and length
\var{len} on success, and \NULL{} on failure.  If \var{v} is \NULL{},
the contents of the string are uninitialized.
\end{cfuncdesc}

\begin{cfuncdesc}{PyObject*}{PyString_FromString}{const char *v}
Returns a new string object with the value \var{v} on success, and
\NULL{} on failure.
\end{cfuncdesc}

\begin{cfuncdesc}{int}{PyString_Size}{PyObject *string}
Returns the length of the string in string object \var{string}.
\end{cfuncdesc}

\begin{cfuncdesc}{char*}{PyString_AsString}{PyObject *string}
Resturns a \NULL{} terminated representation of the contents of \var{string}.
\end{cfuncdesc}

\begin{cfuncdesc}{void}{PyString_Concat}{PyObject **string,
                                         PyObject *newpart}
Creates a new string object in \var{*string} containing the contents
of \var{newpart} appended to \var{string}.
\end{cfuncdesc}

\begin{cfuncdesc}{void}{PyString_ConcatAndDel}{PyObject **string,
                                               PyObject *newpart}
Creates a new string object in \var{*string} containing the contents
of \var{newpart} appended to \var{string}.  This version decrements
the reference count of \var{newpart}.
\end{cfuncdesc}

\begin{cfuncdesc}{int}{_PyString_Resize}{PyObject **string, int newsize}
A way to resize a string object even though it is ``immutable''.  
Only use this to build up a brand new string object; don't use this if
the string may already be known in other parts of the code.
\end{cfuncdesc}

\begin{cfuncdesc}{PyObject*}{PyString_Format}{PyObject *format,
                                              PyObject *args}
Returns a new string object from \var{format} and \var{args}.  Analogous
to \code{\var{format} \% \var{args}}.  The \var{args} argument must be
a tuple.
\end{cfuncdesc}

\begin{cfuncdesc}{void}{PyString_InternInPlace}{PyObject **string}
Intern the argument \var{*string} in place.  The argument must be the
address of a pointer variable pointing to a Python string object.
If there is an existing interned string that is the same as
\var{*string}, it sets \var{*string} to it (decrementing the reference 
count of the old string object and incrementing the reference count of
the interned string object), otherwise it leaves \var{*string} alone
and interns it (incrementing its reference count).  (Clarification:
even though there is a lot of talk about reference counts, think of
this function as reference-count-neutral; you own the object after
the call if and only if you owned it before the call.)
\end{cfuncdesc}

\begin{cfuncdesc}{PyObject*}{PyString_InternFromString}{const char *v}
A combination of \cfunction{PyString_FromString()} and
\cfunction{PyString_InternInPlace()}, returning either a new string object
that has been interned, or a new (``owned'') reference to an earlier
interned string object with the same value.
\end{cfuncdesc}

\begin{cfuncdesc}{char*}{PyString_AS_STRING}{PyObject *string}
Macro form of \cfunction{PyString_AsString()} but without error checking.
\end{cfuncdesc}

\begin{cfuncdesc}{int}{PyString_GET_SIZE}{PyObject *string}
Macro form of \cfunction{PyString_GetSize()} but without error checking.
\end{cfuncdesc}



\subsection{Tuple Objects}
\label{tupleObjects}

\begin{ctypedesc}{PyTupleObject}
This subtype of \ctype{PyObject} represents a Python tuple object.
\end{ctypedesc}

\begin{cvardesc}{PyTypeObject}{PyTuple_Type}
This instance of \ctype{PyTypeObject} represents the Python tuple type.
\end{cvardesc}

\begin{cfuncdesc}{int}{PyTuple_Check}{PyObject *p}
Return true if the argument is a tuple object.
\end{cfuncdesc}

\begin{cfuncdesc}{PyObject*}{PyTuple_New}{int s}
Return a new tuple object of size \var{s}.
\end{cfuncdesc}

\begin{cfuncdesc}{int}{PyTuple_Size}{PyTupleObject *p}
Takes a pointer to a tuple object, and returns the size
of that tuple.
\end{cfuncdesc}

\begin{cfuncdesc}{PyObject*}{PyTuple_GetItem}{PyTupleObject *p, int pos}
Returns the object at position \var{pos} in the tuple pointed
to by \var{p}.  If \var{pos} is out of bounds, returns \NULL{} and
sets an \exception{IndexError} exception.  \strong{Note:} this
function returns a ``borrowed'' reference.
\end{cfuncdesc}

\begin{cfuncdesc}{PyObject*}{PyTuple_GET_ITEM}{PyTupleObject *p, int pos}
Does the same, but does no checking of its arguments.
\end{cfuncdesc}

\begin{cfuncdesc}{PyObject*}{PyTuple_GetSlice}{PyTupleObject *p,
            int low,
            int high}
Takes a slice of the tuple pointed to by \var{p} from
\var{low} to \var{high} and returns it as a new tuple.
\end{cfuncdesc}

\begin{cfuncdesc}{int}{PyTuple_SetItem}{PyTupleObject *p,
            int pos,
            PyObject *o}
Inserts a reference to object \var{o} at position \var{pos} of
the tuple pointed to by \var{p}. It returns \code{0} on success.
\end{cfuncdesc}

\begin{cfuncdesc}{void}{PyTuple_SET_ITEM}{PyTupleObject *p,
            int pos,
            PyObject *o}

Does the same, but does no error checking, and
should \emph{only} be used to fill in brand new tuples.
\end{cfuncdesc}

\begin{cfuncdesc}{int}{_PyTuple_Resize}{PyTupleObject *p,
            int new,
            int last_is_sticky}
Can be used to resize a tuple. Because tuples are
\emph{supposed} to be immutable, this should only be used if there is only
one module referencing the object. Do \emph{not} use this if the tuple may
already be known to some other part of the code. \var{last_is_sticky} is
a flag --- if set, the tuple will grow or shrink at the front, otherwise
it will grow or shrink at the end. Think of this as destroying the old
tuple and creating a new one, only more efficiently.
\end{cfuncdesc}


\subsection{List Objects}
\label{listObjects}

\begin{ctypedesc}{PyListObject}
This subtype of \ctype{PyObject} represents a Python list object.
\end{ctypedesc}

\begin{cvardesc}{PyTypeObject}{PyList_Type}
This instance of \ctype{PyTypeObject} represents the Python list type.
\end{cvardesc}

\begin{cfuncdesc}{int}{PyList_Check}{PyObject *p}
Returns true if its argument is a \ctype{PyListObject}.
\end{cfuncdesc}

\begin{cfuncdesc}{PyObject*}{PyList_New}{int size}
Returns a new list of length \var{len} on success, and \NULL{} on
failure.
\end{cfuncdesc}

\begin{cfuncdesc}{int}{PyList_Size}{PyObject *list}
Returns the length of the list object in \var{list}.
\end{cfuncdesc}

\begin{cfuncdesc}{PyObject*}{PyList_GetItem}{PyObject *list, int index}
Returns the object at position \var{pos} in the list pointed
to by \var{p}.  If \var{pos} is out of bounds, returns \NULL{} and
sets an \exception{IndexError} exception.  \strong{Note:} this
function returns a ``borrowed'' reference.
\end{cfuncdesc}

\begin{cfuncdesc}{int}{PyList_SetItem}{PyObject *list, int index,
                                       PyObject *item}
Sets the item at index \var{index} in list to \var{item}.
\end{cfuncdesc}

\begin{cfuncdesc}{int}{PyList_Insert}{PyObject *list, int index,
                                      PyObject *item}
Inserts the item \var{item} into list \var{list} in front of index
\var{index}.  Returns 0 if successful; returns -1 and sets an
exception if unsuccessful.  Analogous to \code{list.insert(index, item)}.
\end{cfuncdesc}

\begin{cfuncdesc}{int}{PyList_Append}{PyObject *list, PyObject *item}
Appends the object \var{item} at the end of list \var{list}.  Returns
0 if successful; returns -1 and sets an exception if unsuccessful.
Analogous to \code{list.append(item)}.
\end{cfuncdesc}

\begin{cfuncdesc}{PyObject*}{PyList_GetSlice}{PyObject *list,
                                              int low, int high}
Returns a list of the objects in \var{list} containing the objects 
\emph{between} \var{low} and \var{high}.  Returns NULL and sets an
exception if unsuccessful.
Analogous to \code{list[low:high]}.
\end{cfuncdesc}

\begin{cfuncdesc}{int}{PyList_SetSlice}{PyObject *list,
                                        int low, int high,
                                        PyObject *itemlist}
Sets the slice of \var{list} between \var{low} and \var{high} to the contents
of \var{itemlist}.  Analogous to \code{list[low:high]=itemlist}.  Returns 0
on success, -1 on failure.
\end{cfuncdesc}

\begin{cfuncdesc}{int}{PyList_Sort}{PyObject *list}
Sorts the items of \var{list} in place.  Returns 0 on success, -1 on failure.
\end{cfuncdesc}

\begin{cfuncdesc}{int}{PyList_Reverse}{PyObject *list}
Reverses the items of \var{list} in place.  Returns 0 on success, -1 on failure.
\end{cfuncdesc}

\begin{cfuncdesc}{PyObject*}{PyList_AsTuple}{PyObject *list}
Returns a new tuple object containing the contents of \var{list}.
\end{cfuncdesc}

\begin{cfuncdesc}{PyObject*}{PyList_GET_ITEM}{PyObject *list, int i}
Macro form of \cfunction{PyList_GetItem()} without error checking.
\end{cfuncdesc}

\begin{cfuncdesc}{PyObject*}{PyList_SET_ITEM}{PyObject *list, int i,
                                              PyObject *o}
Macro form of \cfunction{PyList_SetItem()} without error checking.
\end{cfuncdesc}

\begin{cfuncdesc}{int}{PyList_GET_SIZE}{PyObject *list}
Macro form of \cfunction{PyList_GetSize()} without error checking.
\end{cfuncdesc}


\section{Mapping Objects}
\label{mapObjects}

\subsection{Dictionary Objects}
\label{dictObjects}

\begin{ctypedesc}{PyDictObject}
This subtype of \ctype{PyObject} represents a Python dictionary object.
\end{ctypedesc}

\begin{cvardesc}{PyTypeObject}{PyDict_Type}
This instance of \ctype{PyTypeObject} represents the Python dictionary type.
\end{cvardesc}

\begin{cfuncdesc}{int}{PyDict_Check}{PyObject *p}
Returns true if its argument is a \ctype{PyDictObject}.
\end{cfuncdesc}

\begin{cfuncdesc}{PyObject*}{PyDict_New}{}
Returns a new empty dictionary.
\end{cfuncdesc}

\begin{cfuncdesc}{void}{PyDict_Clear}{PyDictObject *p}
Empties an existing dictionary of all key/value pairs.
\end{cfuncdesc}

\begin{cfuncdesc}{int}{PyDict_SetItem}{PyDictObject *p,
            PyObject *key,
            PyObject *val}
Inserts \var{value} into the dictionary with a key of \var{key}.  Both
\var{key} and \var{value} should be PyObjects, and \var{key} should be
hashable.
\end{cfuncdesc}

\begin{cfuncdesc}{int}{PyDict_SetItemString}{PyDictObject *p,
            char *key,
            PyObject *val}
Inserts \var{value} into the dictionary using \var{key}
as a key. \var{key} should be a \ctype{char *}.  The key object is
created using \code{PyString_FromString(\var{key})}.
\end{cfuncdesc}

\begin{cfuncdesc}{int}{PyDict_DelItem}{PyDictObject *p, PyObject *key}
Removes the entry in dictionary \var{p} with key \var{key}.
\var{key} is a PyObject.
\end{cfuncdesc}

\begin{cfuncdesc}{int}{PyDict_DelItemString}{PyDictObject *p, char *key}
Removes the entry in dictionary \var{p} which has a key
specified by the \ctype{char *}\var{key}.
\end{cfuncdesc}

\begin{cfuncdesc}{PyObject*}{PyDict_GetItem}{PyDictObject *p, PyObject *key}
Returns the object from dictionary \var{p} which has a key
\var{key}.  Returns \NULL{} if the key \var{key} is not present, but
without (!) setting an exception.  \strong{Note:}  this function
returns a ``borrowed'' reference.
\end{cfuncdesc}

\begin{cfuncdesc}{PyObject*}{PyDict_GetItemString}{PyDictObject *p, char *key}
This is the same as \cfunction{PyDict_GetItem()}, but \var{key} is
specified as a \ctype{char *}, rather than a \ctype{PyObject *}.
\end{cfuncdesc}

\begin{cfuncdesc}{PyObject*}{PyDict_Items}{PyDictObject *p}
Returns a \ctype{PyListObject} containing all the items 
from the dictionary, as in the dictinoary method \method{items()} (see
the \emph{Python Library Reference}).
\end{cfuncdesc}

\begin{cfuncdesc}{PyObject*}{PyDict_Keys}{PyDictObject *p}
Returns a \ctype{PyListObject} containing all the keys 
from the dictionary, as in the dictionary method \method{keys()} (see the
\emph{Python Library Reference}).
\end{cfuncdesc}

\begin{cfuncdesc}{PyObject*}{PyDict_Values}{PyDictObject *p}
Returns a \ctype{PyListObject} containing all the values 
from the dictionary \var{p}, as in the dictionary method
\method{values()} (see the \emph{Python Library Reference}).
\end{cfuncdesc}

\begin{cfuncdesc}{int}{PyDict_Size}{PyDictObject *p}
Returns the number of items in the dictionary.
\end{cfuncdesc}

\begin{cfuncdesc}{int}{PyDict_Next}{PyDictObject *p,
            int ppos,
            PyObject **pkey,
            PyObject **pvalue}

\end{cfuncdesc}


\section{Numeric Objects}
\label{numericObjects}

\subsection{Plain Integer Objects}
\label{intObjects}

\begin{ctypedesc}{PyIntObject}
This subtype of \ctype{PyObject} represents a Python integer object.
\end{ctypedesc}

\begin{cvardesc}{PyTypeObject}{PyInt_Type}
This instance of \ctype{PyTypeObject} represents the Python plain 
integer type.
\end{cvardesc}

\begin{cfuncdesc}{int}{PyInt_Check}{PyObject *}

\end{cfuncdesc}

\begin{cfuncdesc}{PyObject*}{PyInt_FromLong}{long ival}
Creates a new integer object with a value of \var{ival}.

The current implementation keeps an array of integer objects for all
integers between \code{-1} and \code{100}, when you create an int in
that range you actually just get back a reference to the existing
object. So it should be possible to change the value of \code{1}. I
suspect the behaviour of Python in this case is undefined. :-)
\end{cfuncdesc}

\begin{cfuncdesc}{long}{PyInt_AS_LONG}{PyIntObject *io}
Returns the value of the object \var{io}.  No error checking is
performed.
\end{cfuncdesc}

\begin{cfuncdesc}{long}{PyInt_AsLong}{PyObject *io}
Will first attempt to cast the object to a \ctype{PyIntObject}, if
it is not already one, and then return its value.
\end{cfuncdesc}

\begin{cfuncdesc}{long}{PyInt_GetMax}{}
Returns the systems idea of the largest integer it can handle
(\constant{LONG_MAX}, as defined in the system header files).
\end{cfuncdesc}


\subsection{Long Integer Objects}
\label{longObjects}

\begin{ctypedesc}{PyLongObject}
This subtype of \ctype{PyObject} represents a Python long integer
object.
\end{ctypedesc}

\begin{cvardesc}{PyTypeObject}{PyLong_Type}
This instance of \ctype{PyTypeObject} represents the Python long
integer type.
\end{cvardesc}

\begin{cfuncdesc}{int}{PyLong_Check}{PyObject *p}
Returns true if its argument is a \ctype{PyLongObject}.
\end{cfuncdesc}

\begin{cfuncdesc}{PyObject*}{PyLong_FromLong}{long v}
Returns a new \ctype{PyLongObject} object from \var{v}.
\end{cfuncdesc}

\begin{cfuncdesc}{PyObject*}{PyLong_FromUnsignedLong}{unsigned long v}
Returns a new \ctype{PyLongObject} object from an unsigned \C{} long.
\end{cfuncdesc}

\begin{cfuncdesc}{PyObject*}{PyLong_FromDouble}{double v}
Returns a new \ctype{PyLongObject} object from the integer part of \var{v}.
\end{cfuncdesc}

\begin{cfuncdesc}{long}{PyLong_AsLong}{PyObject *pylong}
Returns a \C{} \ctype{long} representation of the contents of \var{pylong}.  
WHAT HAPPENS IF \var{pylong} is greater than \constant{LONG_MAX}?
\end{cfuncdesc}

\begin{cfuncdesc}{unsigned long}{PyLong_AsUnsignedLong}{PyObject *pylong}
Returns a \C{} \ctype{unsigned long} representation of the contents of 
\var{pylong}.  WHAT HAPPENS IF \var{pylong} is greater than
\constant{ULONG_MAX}?
\end{cfuncdesc}

\begin{cfuncdesc}{double}{PyLong_AsDouble}{PyObject *pylong}
Returns a \C{} \ctype{double} representation of the contents of \var{pylong}.
\end{cfuncdesc}

\begin{cfuncdesc}{PyObject*}{PyLong_FromString}{char *str, char **pend,
                                                int base}
\end{cfuncdesc}


\subsection{Floating Point Objects}
\label{floatObjects}

\begin{ctypedesc}{PyFloatObject}
This subtype of \ctype{PyObject} represents a Python floating point
object.
\end{ctypedesc}

\begin{cvardesc}{PyTypeObject}{PyFloat_Type}
This instance of \ctype{PyTypeObject} represents the Python floating
point type.
\end{cvardesc}

\begin{cfuncdesc}{int}{PyFloat_Check}{PyObject *p}
Returns true if its argument is a \ctype{PyFloatObject}.
\end{cfuncdesc}

\begin{cfuncdesc}{PyObject*}{PyFloat_FromDouble}{double v}
Creates a \ctype{PyFloatObject} object from \var{v}.
\end{cfuncdesc}

\begin{cfuncdesc}{double}{PyFloat_AsDouble}{PyObject *pyfloat}
Returns a \C{} \ctype{double} representation of the contents of \var{pyfloat}.
\end{cfuncdesc}

\begin{cfuncdesc}{double}{PyFloat_AS_DOUBLE}{PyObject *pyfloat}
Returns a \C{} \ctype{double} representation of the contents of
\var{pyfloat}, but without error checking.
\end{cfuncdesc}


\subsection{Complex Number Objects}
\label{complexObjects}

\begin{ctypedesc}{Py_complex}
The \C{} structure which corresponds to the value portion of a Python
complex number object.  Most of the functions for dealing with complex
number objects use structures of this type as input or output values,
as appropriate.  It is defined as:

\begin{verbatim}
typedef struct {
   double real;
   double imag;
} Py_complex;
\end{verbatim}
\end{ctypedesc}

\begin{ctypedesc}{PyComplexObject}
This subtype of \ctype{PyObject} represents a Python complex number object.
\end{ctypedesc}

\begin{cvardesc}{PyTypeObject}{PyComplex_Type}
This instance of \ctype{PyTypeObject} represents the Python complex 
number type.
\end{cvardesc}

\begin{cfuncdesc}{int}{PyComplex_Check}{PyObject *p}
Returns true if its argument is a \ctype{PyComplexObject}.
\end{cfuncdesc}

\begin{cfuncdesc}{Py_complex}{_Py_c_sum}{Py_complex left, Py_complex right}
\end{cfuncdesc}

\begin{cfuncdesc}{Py_complex}{_Py_c_diff}{Py_complex left, Py_complex right}
\end{cfuncdesc}

\begin{cfuncdesc}{Py_complex}{_Py_c_neg}{Py_complex complex}
\end{cfuncdesc}

\begin{cfuncdesc}{Py_complex}{_Py_c_prod}{Py_complex left, Py_complex right}
\end{cfuncdesc}

\begin{cfuncdesc}{Py_complex}{_Py_c_quot}{Py_complex dividend,
                                          Py_complex divisor}
\end{cfuncdesc}

\begin{cfuncdesc}{Py_complex}{_Py_c_pow}{Py_complex num, Py_complex exp}
\end{cfuncdesc}

\begin{cfuncdesc}{PyObject*}{PyComplex_FromCComplex}{Py_complex v}
\end{cfuncdesc}

\begin{cfuncdesc}{PyObject*}{PyComplex_FromDoubles}{double real, double imag}
Returns a new \ctype{PyComplexObject} object from \var{real} and \var{imag}.
\end{cfuncdesc}

\begin{cfuncdesc}{double}{PyComplex_RealAsDouble}{PyObject *op}
Returns the real part of \var{op} as a \C{} \ctype{double}.
\end{cfuncdesc}

\begin{cfuncdesc}{double}{PyComplex_ImagAsDouble}{PyObject *op}
Returns the imaginary part of \var{op} as a \C{} \ctype{double}.
\end{cfuncdesc}

\begin{cfuncdesc}{Py_complex}{PyComplex_AsCComplex}{PyObject *op}
\end{cfuncdesc}



\section{Other Objects}
\label{otherObjects}

\subsection{File Objects}
\label{fileObjects}

\begin{ctypedesc}{PyFileObject}
This subtype of \ctype{PyObject} represents a Python file object.
\end{ctypedesc}

\begin{cvardesc}{PyTypeObject}{PyFile_Type}
This instance of \ctype{PyTypeObject} represents the Python file type.
\end{cvardesc}

\begin{cfuncdesc}{int}{PyFile_Check}{PyObject *p}
Returns true if its argument is a \ctype{PyFileObject}.
\end{cfuncdesc}

\begin{cfuncdesc}{PyObject*}{PyFile_FromString}{char *name, char *mode}
Creates a new \ctype{PyFileObject} pointing to the file
specified in \var{name} with the mode specified in \var{mode}.
\end{cfuncdesc}

\begin{cfuncdesc}{PyObject*}{PyFile_FromFile}{FILE *fp,
              char *name, char *mode, int (*close)}
Creates a new \ctype{PyFileObject} from the already-open \var{fp}.
The function \var{close} will be called when the file should be
closed.
\end{cfuncdesc}

\begin{cfuncdesc}{FILE *}{PyFile_AsFile}{PyFileObject *p}
Returns the file object associated with \var{p} as a \ctype{FILE *}.
\end{cfuncdesc}

\begin{cfuncdesc}{PyObject*}{PyFile_GetLine}{PyObject *p, int n}
undocumented as yet
\end{cfuncdesc}

\begin{cfuncdesc}{PyObject*}{PyFile_Name}{PyObject *p}
Returns the name of the file specified by \var{p} as a 
\ctype{PyStringObject}.
\end{cfuncdesc}

\begin{cfuncdesc}{void}{PyFile_SetBufSize}{PyFileObject *p, int n}
Available on systems with \cfunction{setvbuf()} only.  This should
only be called immediately after file object creation.
\end{cfuncdesc}

\begin{cfuncdesc}{int}{PyFile_SoftSpace}{PyFileObject *p, int newflag}
Sets the \member{softspace} attribute of \var{p} to \var{newflag}.
Returns the previous value.  This function clears any errors, and will
return \code{0} as the previous value if the attribute either does not
exist or if there were errors in retrieving it.  There is no way to
detect errors from this function, but doing so should not be needed.
\end{cfuncdesc}

\begin{cfuncdesc}{int}{PyFile_WriteObject}{PyObject *obj, PyFileObject *p,
                                           int flags}
Writes object \var{obj} to file object \var{p}.
\end{cfuncdesc}

\begin{cfuncdesc}{int}{PyFile_WriteString}{char *s, PyFileObject *p,
                                           int flags}
Writes string \var{s} to file object \var{p}.
\end{cfuncdesc}


\subsection{CObjects}
\label{cObjects}

\begin{ctypedesc}{PyCObject}
This subtype of \ctype{PyObject} represents an opaque value, useful for
\C{} extension modules who need to pass an opaque value (as a
\ctype{void *} pointer) through Python code to other \C{} code.  It is
often used to make a C function pointer defined in one module
available to other modules, so the regular import mechanism can be
used to access C APIs defined in dynamically loaded modules.
\end{ctypedesc}

\begin{cfuncdesc}{PyObject *}{PyCObject_FromVoidPtr}{void* cobj, 
	void (*destr)(void *)}
Creates a \ctype{PyCObject} from the \code{void *} \var{cobj}.  The
\var{destr} function will be called when the object is reclaimed.
\end{cfuncdesc}

\begin{cfuncdesc}{PyObject *}{PyCObject_FromVoidPtrAndDesc}{void* cobj,
	void* desc, void (*destr)(void *, void *) }
Creates a \ctype{PyCObject} from the \ctype{void *}\var{cobj}.  The
\var{destr} function will be called when the object is reclaimed.  The
\var{desc} argument can be used to pass extra callback data for the
destructor function.
\end{cfuncdesc}

\begin{cfuncdesc}{void *}{PyCObject_AsVoidPtr}{PyObject* self}
Returns the object \ctype{void *} that the \ctype{PyCObject} \var{self}
was created with.
\end{cfuncdesc}

\begin{cfuncdesc}{void *}{PyCObject_GetDesc}{PyObject* self}
Returns the description \ctype{void *} that the \ctype{PyCObject}
\var{self} was created with.
\end{cfuncdesc}

\chapter{Initialization, Finalization, and Threads}
\label{initialization}

\begin{cfuncdesc}{void}{Py_Initialize}{}
Initialize the Python interpreter.  In an application embedding 
Python, this should be called before using any other Python/C API 
functions; with the exception of \cfunction{Py_SetProgramName()},
\cfunction{PyEval_InitThreads()}, \cfunction{PyEval_ReleaseLock()},
and \cfunction{PyEval_AcquireLock()}.  This initializes the table of
loaded modules (\code{sys.modules}), and creates the fundamental
modules \module{__builtin__}\refbimodindex{__builtin__},
\module{__main__}\refbimodindex{__main__} and
\module{sys}\refbimodindex{sys}.  It also initializes the module
search path (\code{sys.path}).%
\indexiii{module}{search}{path}
It does not set \code{sys.argv}; use \cfunction{PySys_SetArgv()} for
that.  This is a no-op when called for a second time (without calling
\cfunction{Py_Finalize()} first).  There is no return value; it is a
fatal error if the initialization fails.
\end{cfuncdesc}

\begin{cfuncdesc}{int}{Py_IsInitialized}{}
Return true (nonzero) when the Python interpreter has been
initialized, false (zero) if not.  After \cfunction{Py_Finalize()} is
called, this returns false until \cfunction{Py_Initialize()} is called
again.
\end{cfuncdesc}

\begin{cfuncdesc}{void}{Py_Finalize}{}
Undo all initializations made by \cfunction{Py_Initialize()} and
subsequent use of Python/C API functions, and destroy all
sub-interpreters (see \cfunction{Py_NewInterpreter()} below) that were
created and not yet destroyed since the last call to
\cfunction{Py_Initialize()}.  Ideally, this frees all memory allocated
by the Python interpreter.  This is a no-op when called for a second
time (without calling \cfunction{Py_Initialize()} again first).  There
is no return value; errors during finalization are ignored.

This function is provided for a number of reasons.  An embedding 
application might want to restart Python without having to restart the 
application itself.  An application that has loaded the Python 
interpreter from a dynamically loadable library (or DLL) might want to 
free all memory allocated by Python before unloading the DLL. During a 
hunt for memory leaks in an application a developer might want to free 
all memory allocated by Python before exiting from the application.

\strong{Bugs and caveats:} The destruction of modules and objects in 
modules is done in random order; this may cause destructors 
(\method{__del__()} methods) to fail when they depend on other objects 
(even functions) or modules.  Dynamically loaded extension modules 
loaded by Python are not unloaded.  Small amounts of memory allocated 
by the Python interpreter may not be freed (if you find a leak, please 
report it).  Memory tied up in circular references between objects is 
not freed.  Some memory allocated by extension modules may not be 
freed.  Some extension may not work properly if their initialization 
routine is called more than once; this can happen if an applcation 
calls \cfunction{Py_Initialize()} and \cfunction{Py_Finalize()} more
than once.
\end{cfuncdesc}

\begin{cfuncdesc}{PyThreadState*}{Py_NewInterpreter}{}
Create a new sub-interpreter.  This is an (almost) totally separate
environment for the execution of Python code.  In particular, the new
interpreter has separate, independent versions of all imported
modules, including the fundamental modules
\module{__builtin__}\refbimodindex{__builtin__},
\module{__main__}\refbimodindex{__main__} and
\module{sys}\refbimodindex{sys}.  The table of loaded modules
(\code{sys.modules}) and the module search path (\code{sys.path}) are
also separate.  The new environment has no \code{sys.argv} variable.
It has new standard I/O stream file objects \code{sys.stdin},
\code{sys.stdout} and \code{sys.stderr} (however these refer to the
same underlying \ctype{FILE} structures in the \C{} library).

The return value points to the first thread state created in the new 
sub-interpreter.  This thread state is made the current thread state.  
Note that no actual thread is created; see the discussion of thread 
states below.  If creation of the new interpreter is unsuccessful, 
\NULL{} is returned; no exception is set since the exception state 
is stored in the current thread state and there may not be a current 
thread state.  (Like all other Python/C API functions, the global 
interpreter lock must be held before calling this function and is 
still held when it returns; however, unlike most other Python/C API 
functions, there needn't be a current thread state on entry.)

Extension modules are shared between (sub-)interpreters as follows: 
the first time a particular extension is imported, it is initialized 
normally, and a (shallow) copy of its module's dictionary is 
squirreled away.  When the same extension is imported by another 
(sub-)interpreter, a new module is initialized and filled with the 
contents of this copy; the extension's \code{init} function is not
called.  Note that this is different from what happens when an
extension is imported after the interpreter has been completely
re-initialized by calling \cfunction{Py_Finalize()} and
\cfunction{Py_Initialize()}; in that case, the extension's \code{init}
function \emph{is} called again.

\strong{Bugs and caveats:} Because sub-interpreters (and the main 
interpreter) are part of the same process, the insulation between them 
isn't perfect --- for example, using low-level file operations like 
\function{os.close()} they can (accidentally or maliciously) affect each 
other's open files.  Because of the way extensions are shared between 
(sub-)interpreters, some extensions may not work properly; this is 
especially likely when the extension makes use of (static) global 
variables, or when the extension manipulates its module's dictionary 
after its initialization.  It is possible to insert objects created in 
one sub-interpreter into a namespace of another sub-interpreter; this 
should be done with great care to avoid sharing user-defined 
functions, methods, instances or classes between sub-interpreters, 
since import operations executed by such objects may affect the 
wrong (sub-)interpreter's dictionary of loaded modules.  (XXX This is 
a hard-to-fix bug that will be addressed in a future release.)
\end{cfuncdesc}

\begin{cfuncdesc}{void}{Py_EndInterpreter}{PyThreadState *tstate}
Destroy the (sub-)interpreter represented by the given thread state.  
The given thread state must be the current thread state.  See the 
discussion of thread states below.  When the call returns, the current 
thread state is \NULL{}.  All thread states associated with this 
interpreted are destroyed.  (The global interpreter lock must be held 
before calling this function and is still held when it returns.)  
\cfunction{Py_Finalize()} will destroy all sub-interpreters that haven't 
been explicitly destroyed at that point.
\end{cfuncdesc}

\begin{cfuncdesc}{void}{Py_SetProgramName}{char *name}
This function should be called before \cfunction{Py_Initialize()} is called 
for the first time, if it is called at all.  It tells the interpreter 
the value of the \code{argv[0]} argument to the \cfunction{main()} function 
of the program.  This is used by \cfunction{Py_GetPath()} and some other 
functions below to find the Python run-time libraries relative to the 
interpreter executable.  The default value is \code{"python"}.  The 
argument should point to a zero-terminated character string in static 
storage whose contents will not change for the duration of the 
program's execution.  No code in the Python interpreter will change 
the contents of this storage.
\end{cfuncdesc}

\begin{cfuncdesc}{char*}{Py_GetProgramName}{}
Return the program name set with \cfunction{Py_SetProgramName()}, or the 
default.  The returned string points into static storage; the caller 
should not modify its value.
\end{cfuncdesc}

\begin{cfuncdesc}{char*}{Py_GetPrefix}{}
Return the \emph{prefix} for installed platform-independent files.  This 
is derived through a number of complicated rules from the program name 
set with \cfunction{Py_SetProgramName()} and some environment variables; 
for example, if the program name is \code{"/usr/local/bin/python"}, 
the prefix is \code{"/usr/local"}.  The returned string points into 
static storage; the caller should not modify its value.  This 
corresponds to the \makevar{prefix} variable in the top-level 
\file{Makefile} and the \code{-}\code{-prefix} argument to the 
\program{configure} script at build time.  The value is available to 
Python code as \code{sys.prefix}.  It is only useful on \UNIX{}.  See 
also the next function.
\end{cfuncdesc}

\begin{cfuncdesc}{char*}{Py_GetExecPrefix}{}
Return the \emph{exec-prefix} for installed platform-\emph{de}pendent 
files.  This is derived through a number of complicated rules from the 
program name set with \cfunction{Py_SetProgramName()} and some environment 
variables; for example, if the program name is 
\code{"/usr/local/bin/python"}, the exec-prefix is 
\code{"/usr/local"}.  The returned string points into static storage; 
the caller should not modify its value.  This corresponds to the 
\makevar{exec_prefix} variable in the top-level \file{Makefile} and the 
\code{-}\code{-exec_prefix} argument to the \program{configure} script
at build  time.  The value is available to Python code as 
\code{sys.exec_prefix}.  It is only useful on \UNIX{}.

Background: The exec-prefix differs from the prefix when platform 
dependent files (such as executables and shared libraries) are 
installed in a different directory tree.  In a typical installation, 
platform dependent files may be installed in the 
\code{"/usr/local/plat"} subtree while platform independent may be 
installed in \code{"/usr/local"}.

Generally speaking, a platform is a combination of hardware and 
software families, e.g.  Sparc machines running the Solaris 2.x 
operating system are considered the same platform, but Intel machines 
running Solaris 2.x are another platform, and Intel machines running 
Linux are yet another platform.  Different major revisions of the same 
operating system generally also form different platforms.  Non-\UNIX{} 
operating systems are a different story; the installation strategies 
on those systems are so different that the prefix and exec-prefix are 
meaningless, and set to the empty string.  Note that compiled Python 
bytecode files are platform independent (but not independent from the 
Python version by which they were compiled!).

System administrators will know how to configure the \program{mount} or 
\program{automount} programs to share \code{"/usr/local"} between platforms 
while having \code{"/usr/local/plat"} be a different filesystem for each 
platform.
\end{cfuncdesc}

\begin{cfuncdesc}{char*}{Py_GetProgramFullPath}{}
Return the full program name of the Python executable; this is 
computed as a side-effect of deriving the default module search path 
from the program name (set by \cfunction{Py_SetProgramName()} above).  The 
returned string points into static storage; the caller should not 
modify its value.  The value is available to Python code as 
\code{sys.executable}.
\end{cfuncdesc}

\begin{cfuncdesc}{char*}{Py_GetPath}{}
\indexiii{module}{search}{path}
Return the default module search path; this is computed from the 
program name (set by \cfunction{Py_SetProgramName()} above) and some 
environment variables.  The returned string consists of a series of 
directory names separated by a platform dependent delimiter character.  
The delimiter character is \character{:} on \UNIX{}, \character{;} on
DOS/Windows, and \character{\\n} (the \ASCII{} newline character) on
Macintosh.  The returned string points into static storage; the caller
should not modify its value.  The value is available to Python code 
as the list \code{sys.path}, which may be modified to change the 
future search path for loaded modules.

% XXX should give the exact rules
\end{cfuncdesc}

\begin{cfuncdesc}{const char*}{Py_GetVersion}{}
Return the version of this Python interpreter.  This is a string that 
looks something like

\begin{verbatim}
"1.5 (#67, Dec 31 1997, 22:34:28) [GCC 2.7.2.2]"
\end{verbatim}

The first word (up to the first space character) is the current Python 
version; the first three characters are the major and minor version 
separated by a period.  The returned string points into static storage; 
the caller should not modify its value.  The value is available to 
Python code as the list \code{sys.version}.
\end{cfuncdesc}

\begin{cfuncdesc}{const char*}{Py_GetPlatform}{}
Return the platform identifier for the current platform.  On \UNIX{}, 
this is formed from the ``official'' name of the operating system, 
converted to lower case, followed by the major revision number; e.g., 
for Solaris 2.x, which is also known as SunOS 5.x, the value is 
\code{"sunos5"}.  On Macintosh, it is \code{"mac"}.  On Windows, it 
is \code{"win"}.  The returned string points into static storage; 
the caller should not modify its value.  The value is available to 
Python code as \code{sys.platform}.
\end{cfuncdesc}

\begin{cfuncdesc}{const char*}{Py_GetCopyright}{}
Return the official copyright string for the current Python version, 
for example

\code{"Copyright 1991-1995 Stichting Mathematisch Centrum, Amsterdam"}

The returned string points into static storage; the caller should not 
modify its value.  The value is available to Python code as the list 
\code{sys.copyright}.
\end{cfuncdesc}

\begin{cfuncdesc}{const char*}{Py_GetCompiler}{}
Return an indication of the compiler used to build the current Python 
version, in square brackets, for example:

\begin{verbatim}
"[GCC 2.7.2.2]"
\end{verbatim}

The returned string points into static storage; the caller should not 
modify its value.  The value is available to Python code as part of 
the variable \code{sys.version}.
\end{cfuncdesc}

\begin{cfuncdesc}{const char*}{Py_GetBuildInfo}{}
Return information about the sequence number and build date and time 
of the current Python interpreter instance, for example

\begin{verbatim}
"#67, Aug  1 1997, 22:34:28"
\end{verbatim}

The returned string points into static storage; the caller should not 
modify its value.  The value is available to Python code as part of 
the variable \code{sys.version}.
\end{cfuncdesc}

\begin{cfuncdesc}{int}{PySys_SetArgv}{int argc, char **argv}
% XXX
\end{cfuncdesc}

% XXX Other PySys thingies (doesn't really belong in this chapter)

\section{Thread State and the Global Interpreter Lock}
\label{threads}

The Python interpreter is not fully thread safe.  In order to support
multi-threaded Python programs, there's a global lock that must be
held by the current thread before it can safely access Python objects.
Without the lock, even the simplest operations could cause problems in
a multi-threaded program: for example, when two threads simultaneously
increment the reference count of the same object, the reference count
could end up being incremented only once instead of twice.

Therefore, the rule exists that only the thread that has acquired the
global interpreter lock may operate on Python objects or call Python/C
API functions.  In order to support multi-threaded Python programs,
the interpreter regularly release and reacquires the lock --- by
default, every ten bytecode instructions (this can be changed with
\function{sys.setcheckinterval()}).  The lock is also released and
reacquired around potentially blocking I/O operations like reading or
writing a file, so that other threads can run while the thread that
requests the I/O is waiting for the I/O operation to complete.

The Python interpreter needs to keep some bookkeeping information
separate per thread --- for this it uses a data structure called
\ctype{PyThreadState}.  This is new in Python 1.5; in earlier versions,
such state was stored in global variables, and switching threads could
cause problems.  In particular, exception handling is now thread safe,
when the application uses \function{sys.exc_info()} to access the
exception last raised in the current thread.

There's one global variable left, however: the pointer to the current
\ctype{PyThreadState} structure.  While most thread packages have a way
to store ``per-thread global data,'' Python's internal platform
independent thread abstraction doesn't support this yet.  Therefore,
the current thread state must be manipulated explicitly.

This is easy enough in most cases.  Most code manipulating the global
interpreter lock has the following simple structure:

\begin{verbatim}
Save the thread state in a local variable.
Release the interpreter lock.
...Do some blocking I/O operation...
Reacquire the interpreter lock.
Restore the thread state from the local variable.
\end{verbatim}

This is so common that a pair of macros exists to simplify it:

\begin{verbatim}
Py_BEGIN_ALLOW_THREADS
...Do some blocking I/O operation...
Py_END_ALLOW_THREADS
\end{verbatim}

The \code{Py_BEGIN_ALLOW_THREADS} macro opens a new block and declares
a hidden local variable; the \code{Py_END_ALLOW_THREADS} macro closes
the block.  Another advantage of using these two macros is that when
Python is compiled without thread support, they are defined empty,
thus saving the thread state and lock manipulations.

When thread support is enabled, the block above expands to the
following code:

\begin{verbatim}
{
    PyThreadState *_save;
    _save = PyEval_SaveThread();
    ...Do some blocking I/O operation...
    PyEval_RestoreThread(_save);
}
\end{verbatim}

Using even lower level primitives, we can get roughly the same effect
as follows:

\begin{verbatim}
{
    PyThreadState *_save;
    _save = PyThreadState_Swap(NULL);
    PyEval_ReleaseLock();
    ...Do some blocking I/O operation...
    PyEval_AcquireLock();
    PyThreadState_Swap(_save);
}
\end{verbatim}

There are some subtle differences; in particular,
\cfunction{PyEval_RestoreThread()} saves and restores the value of the
global variable \cdata{errno}, since the lock manipulation does not
guarantee that \cdata{errno} is left alone.  Also, when thread support
is disabled, \cfunction{PyEval_SaveThread()} and
\cfunction{PyEval_RestoreThread()} don't manipulate the lock; in this
case, \cfunction{PyEval_ReleaseLock()} and
\cfunction{PyEval_AcquireLock()} are not available.  This is done so
that dynamically loaded extensions compiled with thread support
enabled can be loaded by an interpreter that was compiled with
disabled thread support.

The global interpreter lock is used to protect the pointer to the
current thread state.  When releasing the lock and saving the thread
state, the current thread state pointer must be retrieved before the
lock is released (since another thread could immediately acquire the
lock and store its own thread state in the global variable).
Reversely, when acquiring the lock and restoring the thread state, the
lock must be acquired before storing the thread state pointer.

Why am I going on with so much detail about this?  Because when
threads are created from \C{}, they don't have the global interpreter
lock, nor is there a thread state data structure for them.  Such
threads must bootstrap themselves into existence, by first creating a
thread state data structure, then acquiring the lock, and finally
storing their thread state pointer, before they can start using the
Python/C API.  When they are done, they should reset the thread state
pointer, release the lock, and finally free their thread state data
structure.

When creating a thread data structure, you need to provide an
interpreter state data structure.  The interpreter state data
structure hold global data that is shared by all threads in an
interpreter, for example the module administration
(\code{sys.modules}).  Depending on your needs, you can either create
a new interpreter state data structure, or share the interpreter state
data structure used by the Python main thread (to access the latter,
you must obtain the thread state and access its \member{interp} member;
this must be done by a thread that is created by Python or by the main
thread after Python is initialized).

XXX More?

\begin{ctypedesc}{PyInterpreterState}
This data structure represents the state shared by a number of
cooperating threads.  Threads belonging to the same interpreter
share their module administration and a few other internal items.
There are no public members in this structure.

Threads belonging to different interpreters initially share nothing,
except process state like available memory, open file descriptors and
such.  The global interpreter lock is also shared by all threads,
regardless of to which interpreter they belong.
\end{ctypedesc}

\begin{ctypedesc}{PyThreadState}
This data structure represents the state of a single thread.  The only
public data member is \ctype{PyInterpreterState *}\member{interp},
which points to this thread's interpreter state.
\end{ctypedesc}

\begin{cfuncdesc}{void}{PyEval_InitThreads}{}
Initialize and acquire the global interpreter lock.  It should be
called in the main thread before creating a second thread or engaging
in any other thread operations such as
\cfunction{PyEval_ReleaseLock()} or
\code{PyEval_ReleaseThread(\var{tstate})}.  It is not needed before
calling \cfunction{PyEval_SaveThread()} or
\cfunction{PyEval_RestoreThread()}.

This is a no-op when called for a second time.  It is safe to call
this function before calling \cfunction{Py_Initialize()}.

When only the main thread exists, no lock operations are needed.  This
is a common situation (most Python programs do not use threads), and
the lock operations slow the interpreter down a bit.  Therefore, the
lock is not created initially.  This situation is equivalent to having
acquired the lock: when there is only a single thread, all object
accesses are safe.  Therefore, when this function initializes the
lock, it also acquires it.  Before the Python
\module{thread}\refbimodindex{thread} module creates a new thread,
knowing that either it has the lock or the lock hasn't been created
yet, it calls \cfunction{PyEval_InitThreads()}.  When this call
returns, it is guaranteed that the lock has been created and that it
has acquired it.

It is \strong{not} safe to call this function when it is unknown which
thread (if any) currently has the global interpreter lock.

This function is not available when thread support is disabled at
compile time.
\end{cfuncdesc}

\begin{cfuncdesc}{void}{PyEval_AcquireLock}{}
Acquire the global interpreter lock.  The lock must have been created
earlier.  If this thread already has the lock, a deadlock ensues.
This function is not available when thread support is disabled at
compile time.
\end{cfuncdesc}

\begin{cfuncdesc}{void}{PyEval_ReleaseLock}{}
Release the global interpreter lock.  The lock must have been created
earlier.  This function is not available when thread support is
disabled at compile time.
\end{cfuncdesc}

\begin{cfuncdesc}{void}{PyEval_AcquireThread}{PyThreadState *tstate}
Acquire the global interpreter lock and then set the current thread
state to \var{tstate}, which should not be \NULL{}.  The lock must
have been created earlier.  If this thread already has the lock,
deadlock ensues.  This function is not available when thread support
is disabled at compile time.
\end{cfuncdesc}

\begin{cfuncdesc}{void}{PyEval_ReleaseThread}{PyThreadState *tstate}
Reset the current thread state to \NULL{} and release the global
interpreter lock.  The lock must have been created earlier and must be
held by the current thread.  The \var{tstate} argument, which must not
be \NULL{}, is only used to check that it represents the current
thread state --- if it isn't, a fatal error is reported.  This
function is not available when thread support is disabled at compile
time.
\end{cfuncdesc}

\begin{cfuncdesc}{PyThreadState*}{PyEval_SaveThread}{}
Release the interpreter lock (if it has been created and thread
support is enabled) and reset the thread state to \NULL{},
returning the previous thread state (which is not \NULL{}).  If
the lock has been created, the current thread must have acquired it.
(This function is available even when thread support is disabled at
compile time.)
\end{cfuncdesc}

\begin{cfuncdesc}{void}{PyEval_RestoreThread}{PyThreadState *tstate}
Acquire the interpreter lock (if it has been created and thread
support is enabled) and set the thread state to \var{tstate}, which
must not be \NULL{}.  If the lock has been created, the current
thread must not have acquired it, otherwise deadlock ensues.  (This
function is available even when thread support is disabled at compile
time.)
\end{cfuncdesc}

% XXX These aren't really C types, but the ctypedesc macro is the simplest!
\begin{ctypedesc}{Py_BEGIN_ALLOW_THREADS}
This macro expands to
\samp{\{ PyThreadState *_save; _save = PyEval_SaveThread();}.
Note that it contains an opening brace; it must be matched with a
following \code{Py_END_ALLOW_THREADS} macro.  See above for further
discussion of this macro.  It is a no-op when thread support is
disabled at compile time.
\end{ctypedesc}

\begin{ctypedesc}{Py_END_ALLOW_THREADS}
This macro expands to
\samp{PyEval_RestoreThread(_save); \}}.
Note that it contains a closing brace; it must be matched with an
earlier \code{Py_BEGIN_ALLOW_THREADS} macro.  See above for further
discussion of this macro.  It is a no-op when thread support is
disabled at compile time.
\end{ctypedesc}

\begin{ctypedesc}{Py_BEGIN_BLOCK_THREADS}
This macro expands to \samp{PyEval_RestoreThread(_save);} i.e. it
is equivalent to \code{Py_END_ALLOW_THREADS} without the closing
brace.  It is a no-op when thread support is disabled at compile
time.
\end{ctypedesc}

\begin{ctypedesc}{Py_BEGIN_UNBLOCK_THREADS}
This macro expands to \samp{_save = PyEval_SaveThread();} i.e. it is
equivalent to \code{Py_BEGIN_ALLOW_THREADS} without the opening brace
and variable declaration.  It is a no-op when thread support is
disabled at compile time.
\end{ctypedesc}

All of the following functions are only available when thread support
is enabled at compile time, and must be called only when the
interpreter lock has been created.

\begin{cfuncdesc}{PyInterpreterState*}{PyInterpreterState_New}{}
Create a new interpreter state object.  The interpreter lock must be
held.
\end{cfuncdesc}

\begin{cfuncdesc}{void}{PyInterpreterState_Clear}{PyInterpreterState *interp}
Reset all information in an interpreter state object.  The interpreter
lock must be held.
\end{cfuncdesc}

\begin{cfuncdesc}{void}{PyInterpreterState_Delete}{PyInterpreterState *interp}
Destroy an interpreter state object.  The interpreter lock need not be
held.  The interpreter state must have been reset with a previous
call to \cfunction{PyInterpreterState_Clear()}.
\end{cfuncdesc}

\begin{cfuncdesc}{PyThreadState*}{PyThreadState_New}{PyInterpreterState *interp}
Create a new thread state object belonging to the given interpreter
object.  The interpreter lock must be held.
\end{cfuncdesc}

\begin{cfuncdesc}{void}{PyThreadState_Clear}{PyThreadState *tstate}
Reset all information in a thread state object.  The interpreter lock
must be held.
\end{cfuncdesc}

\begin{cfuncdesc}{void}{PyThreadState_Delete}{PyThreadState *tstate}
Destroy a thread state object.  The interpreter lock need not be
held.  The thread state must have been reset with a previous
call to \cfunction{PyThreadState_Clear()}.
\end{cfuncdesc}

\begin{cfuncdesc}{PyThreadState*}{PyThreadState_Get}{}
Return the current thread state.  The interpreter lock must be held.
When the current thread state is \NULL{}, this issues a fatal
error (so that the caller needn't check for \NULL{}).
\end{cfuncdesc}

\begin{cfuncdesc}{PyThreadState*}{PyThreadState_Swap}{PyThreadState *tstate}
Swap the current thread state with the thread state given by the
argument \var{tstate}, which may be \NULL{}.  The interpreter lock
must be held.
\end{cfuncdesc}


\chapter{Defining New Object Types}
\label{newTypes}

\begin{cfuncdesc}{PyObject*}{_PyObject_New}{PyTypeObject *type}
\end{cfuncdesc}

\begin{cfuncdesc}{PyObject*}{_PyObject_NewVar}{PyTypeObject *type, int size}
\end{cfuncdesc}

\begin{cfuncdesc}{TYPE}{_PyObject_NEW}{TYPE, PyTypeObject *}
\end{cfuncdesc}

\begin{cfuncdesc}{TYPE}{_PyObject_NEW_VAR}{TYPE, PyTypeObject *, int size}
\end{cfuncdesc}

Py_InitModule (!!!)

PyArg_ParseTupleAndKeywords, PyArg_ParseTuple, PyArg_Parse

Py_BuildValue

PyObject, PyVarObject

PyObject_HEAD, PyObject_HEAD_INIT, PyObject_VAR_HEAD

Typedefs:
unaryfunc, binaryfunc, ternaryfunc, inquiry, coercion, intargfunc,
intintargfunc, intobjargproc, intintobjargproc, objobjargproc,
getreadbufferproc, getwritebufferproc, getsegcountproc,
destructor, printfunc, getattrfunc, getattrofunc, setattrfunc,
setattrofunc, cmpfunc, reprfunc, hashfunc

PyNumberMethods

PySequenceMethods

PyMappingMethods

PyBufferProcs

PyTypeObject

DL_IMPORT

PyType_Type

Py*_Check

Py_None, _Py_NoneStruct


\chapter{Debugging}
\label{debugging}

XXX Explain Py_DEBUG, Py_TRACE_REFS, Py_REF_DEBUG.


\documentclass{manual}

\title{Python/C API Reference Manual}

\input{boilerplate}

\makeindex			% tell \index to actually write the .idx file


\begin{document}

\maketitle

\input{copyright}

\begin{abstract}

\noindent
This manual documents the API used by \C{} (or \Cpp{}) programmers who
want to write extension modules or embed Python.  It is a companion to
\emph{Extending and Embedding the Python Interpreter}, which describes
the general principles of extension writing but does not document the
API functions in detail.

\strong{Warning:} The current version of this document is incomplete.
I hope that it is nevertheless useful.  I will continue to work on it,
and release new versions from time to time, independent from Python
source code releases.

\end{abstract}

\tableofcontents

% XXX Consider moving all this back to ext.tex and giving api.tex
% XXX a *really* short intro only.

\chapter{Introduction}
\label{intro}

The Application Programmer's Interface to Python gives \C{} and \Cpp{}
programmers access to the Python interpreter at a variety of levels.
The API is equally usable from \Cpp{}, but for brevity it is generally
referred to as the Python/\C{} API.  There are two fundamentally
different reasons for using the Python/\C{} API.  The first reason is
to write \emph{extension modules} for specific purposes; these are
\C{} modules that extend the Python interpreter.  This is probably the
most common use.  The second reason is to use Python as a component in
a larger application; this technique is generally referred to as
\dfn{embedding} Python in an application.

Writing an extension module is a relatively well-understood process, 
where a ``cookbook'' approach works well.  There are several tools 
that automate the process to some extent.  While people have embedded 
Python in other applications since its early existence, the process of 
embedding Python is less straightforward that writing an extension.  
Python 1.5 introduces a number of new API functions as well as some 
changes to the build process that make embedding much simpler.  
This manual describes the \version\ state of affairs.
% XXX Eventually, take the historical notes out

Many API functions are useful independent of whether you're embedding 
or extending Python; moreover, most applications that embed Python 
will need to provide a custom extension as well, so it's probably a 
good idea to become familiar with writing an extension before 
attempting to embed Python in a real application.

\section{Include Files}
\label{includes}

All function, type and macro definitions needed to use the Python/C
API are included in your code by the following line:

\begin{verbatim}
#include "Python.h"
\end{verbatim}

This implies inclusion of the following standard headers:
\code{<stdio.h>}, \code{<string.h>}, \code{<errno.h>}, and
\code{<stdlib.h>} (if available).

All user visible names defined by Python.h (except those defined by
the included standard headers) have one of the prefixes \samp{Py} or
\samp{_Py}.  Names beginning with \samp{_Py} are for internal use
only.  Structure member names do not have a reserved prefix.

\strong{Important:} user code should never define names that begin
with \samp{Py} or \samp{_Py}.  This confuses the reader, and
jeopardizes the portability of the user code to future Python
versions, which may define additional names beginning with one of
these prefixes.

\section{Objects, Types and Reference Counts}
\label{objects}

Most Python/C API functions have one or more arguments as well as a
return value of type \ctype{PyObject *}.  This type is a pointer
to an opaque data type representing an arbitrary Python
object.  Since all Python object types are treated the same way by the
Python language in most situations (e.g., assignments, scope rules,
and argument passing), it is only fitting that they should be
represented by a single \C{} type.  All Python objects live on the heap:
you never declare an automatic or static variable of type
\ctype{PyObject}, only pointer variables of type \ctype{PyObject *} can 
be declared.

All Python objects (even Python integers) have a \dfn{type} and a
\dfn{reference count}.  An object's type determines what kind of object 
it is (e.g., an integer, a list, or a user-defined function; there are 
many more as explained in the \emph{Python Reference Manual}).  For 
each of the well-known types there is a macro to check whether an 
object is of that type; for instance, \samp{PyList_Check(\var{a})} is
true iff the object pointed to by \var{a} is a Python list.

\subsection{Reference Counts}
\label{refcounts}

The reference count is important because today's computers have a 
finite (and often severely limited) memory size; it counts how many 
different places there are that have a reference to an object.  Such a 
place could be another object, or a global (or static) \C{} variable, or 
a local variable in some \C{} function.  When an object's reference count 
becomes zero, the object is deallocated.  If it contains references to 
other objects, their reference count is decremented.  Those other 
objects may be deallocated in turn, if this decrement makes their 
reference count become zero, and so on.  (There's an obvious problem 
with objects that reference each other here; for now, the solution is 
``don't do that''.)

Reference counts are always manipulated explicitly.  The normal way is 
to use the macro \cfunction{Py_INCREF()} to increment an object's 
reference count by one, and \cfunction{Py_DECREF()} to decrement it by 
one.  The decref macro is considerably more complex than the incref one, 
since it must check whether the reference count becomes zero and then 
cause the object's deallocator, which is a function pointer contained 
in the object's type structure.  The type-specific deallocator takes 
care of decrementing the reference counts for other objects contained 
in the object, and so on, if this is a compound object type such as a 
list.  There's no chance that the reference count can overflow; at 
least as many bits are used to hold the reference count as there are 
distinct memory locations in virtual memory (assuming 
\code{sizeof(long) >= sizeof(char *)}).  Thus, the reference count 
increment is a simple operation.

It is not necessary to increment an object's reference count for every 
local variable that contains a pointer to an object.  In theory, the 
object's reference count goes up by one when the variable is made to 
point to it and it goes down by one when the variable goes out of 
scope.  However, these two cancel each other out, so at the end the 
reference count hasn't changed.  The only real reason to use the 
reference count is to prevent the object from being deallocated as 
long as our variable is pointing to it.  If we know that there is at 
least one other reference to the object that lives at least as long as 
our variable, there is no need to increment the reference count 
temporarily.  An important situation where this arises is in objects 
that are passed as arguments to \C{} functions in an extension module 
that are called from Python; the call mechanism guarantees to hold a 
reference to every argument for the duration of the call.

However, a common pitfall is to extract an object from a list and
hold on to it for a while without incrementing its reference count.
Some other operation might conceivably remove the object from the
list, decrementing its reference count and possible deallocating it.
The real danger is that innocent-looking operations may invoke
arbitrary Python code which could do this; there is a code path which
allows control to flow back to the user from a \cfunction{Py_DECREF()},
so almost any operation is potentially dangerous.

A safe approach is to always use the generic operations (functions 
whose name begins with \samp{PyObject_}, \samp{PyNumber_}, 
\samp{PySequence_} or \samp{PyMapping_}).  These operations always 
increment the reference count of the object they return.  This leaves 
the caller with the responsibility to call \cfunction{Py_DECREF()}
when they are done with the result; this soon becomes second nature.

\subsubsection{Reference Count Details}
\label{refcountDetails}

The reference count behavior of functions in the Python/C API is best 
expelained in terms of \emph{ownership of references}.  Note that we 
talk of owning references, never of owning objects; objects are always 
shared!  When a function owns a reference, it has to dispose of it 
properly --- either by passing ownership on (usually to its caller) or 
by calling \cfunction{Py_DECREF()} or \cfunction{Py_XDECREF()}.  When
a function passes ownership of a reference on to its caller, the
caller is said to receive a \emph{new} reference.  When no ownership
is transferred, the caller is said to \emph{borrow} the reference.
Nothing needs to be done for a borrowed reference.

Conversely, when calling a function passes it a reference to an 
object, there are two possibilities: the function \emph{steals} a 
reference to the object, or it does not.  Few functions steal 
references; the two notable exceptions are
\cfunction{PyList_SetItem()} and \cfunction{PyTuple_SetItem()}, which
steal a reference to the item (but not to the tuple or list into which
the item is put!).  These functions were designed to steal a reference
because of a common idiom for populating a tuple or list with newly
created objects; for example, the code to create the tuple \code{(1,
2, "three")} could look like this (forgetting about error handling for
the moment; a better way to code this is shown below anyway):

\begin{verbatim}
PyObject *t;

t = PyTuple_New(3);
PyTuple_SetItem(t, 0, PyInt_FromLong(1L));
PyTuple_SetItem(t, 1, PyInt_FromLong(2L));
PyTuple_SetItem(t, 2, PyString_FromString("three"));
\end{verbatim}

Incidentally, \cfunction{PyTuple_SetItem()} is the \emph{only} way to
set tuple items; \cfunction{PySequence_SetItem()} and
\cfunction{PyObject_SetItem()} refuse to do this since tuples are an
immutable data type.  You should only use
\cfunction{PyTuple_SetItem()} for tuples that you are creating
yourself.

Equivalent code for populating a list can be written using 
\cfunction{PyList_New()} and \cfunction{PyList_SetItem()}.  Such code
can also use \cfunction{PySequence_SetItem()}; this illustrates the
difference between the two (the extra \cfunction{Py_DECREF()} calls):

\begin{verbatim}
PyObject *l, *x;

l = PyList_New(3);
x = PyInt_FromLong(1L);
PySequence_SetItem(l, 0, x); Py_DECREF(x);
x = PyInt_FromLong(2L);
PySequence_SetItem(l, 1, x); Py_DECREF(x);
x = PyString_FromString("three");
PySequence_SetItem(l, 2, x); Py_DECREF(x);
\end{verbatim}

You might find it strange that the ``recommended'' approach takes more
code.  However, in practice, you will rarely use these ways of
creating and populating a tuple or list.  There's a generic function,
\cfunction{Py_BuildValue()}, that can create most common objects from
\C{} values, directed by a \dfn{format string}.  For example, the
above two blocks of code could be replaced by the following (which
also takes care of the error checking):

\begin{verbatim}
PyObject *t, *l;

t = Py_BuildValue("(iis)", 1, 2, "three");
l = Py_BuildValue("[iis]", 1, 2, "three");
\end{verbatim}

It is much more common to use \cfunction{PyObject_SetItem()} and
friends with items whose references you are only borrowing, like
arguments that were passed in to the function you are writing.  In
that case, their behaviour regarding reference counts is much saner,
since you don't have to increment a reference count so you can give a
reference away (``have it be stolen'').  For example, this function
sets all items of a list (actually, any mutable sequence) to a given
item:

\begin{verbatim}
int set_all(PyObject *target, PyObject *item)
{
    int i, n;

    n = PyObject_Length(target);
    if (n < 0)
        return -1;
    for (i = 0; i < n; i++) {
        if (PyObject_SetItem(target, i, item) < 0)
            return -1;
    }
    return 0;
}
\end{verbatim}

The situation is slightly different for function return values.  
While passing a reference to most functions does not change your 
ownership responsibilities for that reference, many functions that 
return a referece to an object give you ownership of the reference.
The reason is simple: in many cases, the returned object is created 
on the fly, and the reference you get is the only reference to the 
object.  Therefore, the generic functions that return object 
references, like \cfunction{PyObject_GetItem()} and 
\cfunction{PySequence_GetItem()}, always return a new reference (i.e.,
the  caller becomes the owner of the reference).

It is important to realize that whether you own a reference returned 
by a function depends on which function you call only --- \emph{the
plumage} (i.e., the type of the type of the object passed as an
argument to the function) \emph{doesn't enter into it!}  Thus, if you 
extract an item from a list using \cfunction{PyList_GetItem()}, you
don't own the reference --- but if you obtain the same item from the
same list using \cfunction{PySequence_GetItem()} (which happens to
take exactly the same arguments), you do own a reference to the
returned object.

Here is an example of how you could write a function that computes the
sum of the items in a list of integers; once using 
\cfunction{PyList_GetItem()}, once using
\cfunction{PySequence_GetItem()}.

\begin{verbatim}
long sum_list(PyObject *list)
{
    int i, n;
    long total = 0;
    PyObject *item;

    n = PyList_Size(list);
    if (n < 0)
        return -1; /* Not a list */
    for (i = 0; i < n; i++) {
        item = PyList_GetItem(list, i); /* Can't fail */
        if (!PyInt_Check(item)) continue; /* Skip non-integers */
        total += PyInt_AsLong(item);
    }
    return total;
}
\end{verbatim}

\begin{verbatim}
long sum_sequence(PyObject *sequence)
{
    int i, n;
    long total = 0;
    PyObject *item;
    n = PyObject_Size(list);
    if (n < 0)
        return -1; /* Has no length */
    for (i = 0; i < n; i++) {
        item = PySequence_GetItem(list, i);
        if (item == NULL)
            return -1; /* Not a sequence, or other failure */
        if (PyInt_Check(item))
            total += PyInt_AsLong(item);
        Py_DECREF(item); /* Discard reference ownership */
    }
    return total;
}
\end{verbatim}

\subsection{Types}
\label{types}

There are few other data types that play a significant role in 
the Python/C API; most are simple \C{} types such as \ctype{int}, 
\ctype{long}, \ctype{double} and \ctype{char *}.  A few structure types 
are used to describe static tables used to list the functions exported 
by a module or the data attributes of a new object type.  These will 
be discussed together with the functions that use them.

\section{Exceptions}
\label{exceptions}

The Python programmer only needs to deal with exceptions if specific 
error handling is required; unhandled exceptions are automatically 
propagated to the caller, then to the caller's caller, and so on, till 
they reach the top-level interpreter, where they are reported to the 
user accompanied by a stack traceback.

For \C{} programmers, however, error checking always has to be explicit.  
All functions in the Python/C API can raise exceptions, unless an 
explicit claim is made otherwise in a function's documentation.  In 
general, when a function encounters an error, it sets an exception, 
discards any object references that it owns, and returns an 
error indicator --- usually \NULL{} or \code{-1}.  A few functions 
return a Boolean true/false result, with false indicating an error.
Very few functions return no explicit error indicator or have an 
ambiguous return value, and require explicit testing for errors with 
\cfunction{PyErr_Occurred()}.

Exception state is maintained in per-thread storage (this is 
equivalent to using global storage in an unthreaded application).  A 
thread can be in one of two states: an exception has occurred, or not.
The function \cfunction{PyErr_Occurred()} can be used to check for
this: it returns a borrowed reference to the exception type object
when an exception has occurred, and \NULL{} otherwise.  There are a
number of functions to set the exception state:
\cfunction{PyErr_SetString()} is the most common (though not the most
general) function to set the exception state, and
\cfunction{PyErr_Clear()} clears the exception state.

The full exception state consists of three objects (all of which can 
be \NULL{}): the exception type, the corresponding exception 
value, and the traceback.  These have the same meanings as the Python 
object \code{sys.exc_type}, \code{sys.exc_value}, 
\code{sys.exc_traceback}; however, they are not the same: the Python 
objects represent the last exception being handled by a Python 
\keyword{try} \ldots\ \keyword{except} statement, while the \C{} level
exception state only exists while an exception is being passed on
between \C{} functions until it reaches the Python interpreter, which
takes care of transferring it to \code{sys.exc_type} and friends.

Note that starting with Python 1.5, the preferred, thread-safe way to 
access the exception state from Python code is to call the function 
\function{sys.exc_info()}, which returns the per-thread exception state 
for Python code.  Also, the semantics of both ways to access the 
exception state have changed so that a function which catches an 
exception will save and restore its thread's exception state so as to 
preserve the exception state of its caller.  This prevents common bugs 
in exception handling code caused by an innocent-looking function 
overwriting the exception being handled; it also reduces the often 
unwanted lifetime extension for objects that are referenced by the 
stack frames in the traceback.

As a general principle, a function that calls another function to 
perform some task should check whether the called function raised an 
exception, and if so, pass the exception state on to its caller.  It 
should discard any object references that it owns, and returns an 
error indicator, but it should \emph{not} set another exception ---
that would overwrite the exception that was just raised, and lose
important information about the exact cause of the error.

A simple example of detecting exceptions and passing them on is shown 
in the \cfunction{sum_sequence()} example above.  It so happens that
that example doesn't need to clean up any owned references when it
detects an error.  The following example function shows some error
cleanup.  First, to remind you why you like Python, we show the
equivalent Python code:

\begin{verbatim}
def incr_item(dict, key):
    try:
        item = dict[key]
    except KeyError:
        item = 0
    return item + 1
\end{verbatim}

Here is the corresponding \C{} code, in all its glory:

\begin{verbatim}
int incr_item(PyObject *dict, PyObject *key)
{
    /* Objects all initialized to NULL for Py_XDECREF */
    PyObject *item = NULL, *const_one = NULL, *incremented_item = NULL;
    int rv = -1; /* Return value initialized to -1 (failure) */

    item = PyObject_GetItem(dict, key);
    if (item == NULL) {
        /* Handle KeyError only: */
        if (!PyErr_ExceptionMatches(PyExc_KeyError)) goto error;

        /* Clear the error and use zero: */
        PyErr_Clear();
        item = PyInt_FromLong(0L);
        if (item == NULL) goto error;
    }

    const_one = PyInt_FromLong(1L);
    if (const_one == NULL) goto error;

    incremented_item = PyNumber_Add(item, const_one);
    if (incremented_item == NULL) goto error;

    if (PyObject_SetItem(dict, key, incremented_item) < 0) goto error;
    rv = 0; /* Success */
    /* Continue with cleanup code */

 error:
    /* Cleanup code, shared by success and failure path */

    /* Use Py_XDECREF() to ignore NULL references */
    Py_XDECREF(item);
    Py_XDECREF(const_one);
    Py_XDECREF(incremented_item);

    return rv; /* -1 for error, 0 for success */
}
\end{verbatim}

This example represents an endorsed use of the \keyword{goto} statement 
in \C{}!  It illustrates the use of
\cfunction{PyErr_ExceptionMatches()} and \cfunction{PyErr_Clear()} to
handle specific exceptions, and the use of \cfunction{Py_XDECREF()} to
dispose of owned references that may be \NULL{} (note the \samp{X} in
the name; \cfunction{Py_DECREF()} would crash when confronted with a
\NULL{} reference).  It is important that the variables used to hold
owned references are initialized to \NULL{} for this to work;
likewise, the proposed return value is initialized to \code{-1}
(failure) and only set to success after the final call made is
successful.


\section{Embedding Python}
\label{embedding}

The one important task that only embedders (as opposed to extension
writers) of the Python interpreter have to worry about is the
initialization, and possibly the finalization, of the Python
interpreter.  Most functionality of the interpreter can only be used
after the interpreter has been initialized.

The basic initialization function is \cfunction{Py_Initialize()}.
This initializes the table of loaded modules, and creates the
fundamental modules \module{__builtin__}\refbimodindex{__builtin__},
\module{__main__}\refbimodindex{__main__} and 
\module{sys}\refbimodindex{sys}.  It also initializes the module
search path (\code{sys.path}).%
\indexiii{module}{search}{path}

\cfunction{Py_Initialize()} does not set the ``script argument list'' 
(\code{sys.argv}).  If this variable is needed by Python code that 
will be executed later, it must be set explicitly with a call to 
\code{PySys_SetArgv(\var{argc}, \var{argv})} subsequent to the call 
to \cfunction{Py_Initialize()}.

On most systems (in particular, on \UNIX{} and Windows, although the
details are slightly different), \cfunction{Py_Initialize()}
calculates the module search path based upon its best guess for the
location of the standard Python interpreter executable, assuming that
the Python library is found in a fixed location relative to the Python
interpreter executable.  In particular, it looks for a directory named
\file{lib/python1.5} (replacing \file{1.5} with the current
interpreter version) relative to the parent directory where the
executable named \file{python} is found on the shell command search
path (the environment variable \envvar{PATH}).

For instance, if the Python executable is found in
\file{/usr/local/bin/python}, it will assume that the libraries are in
\file{/usr/local/lib/python1.5}.  (In fact, this particular path
is also the ``fallback'' location, used when no executable file named
\file{python} is found along \envvar{PATH}.)  The user can override
this behavior by setting the environment variable \envvar{PYTHONHOME},
or insert additional directories in front of the standard path by
setting \envvar{PYTHONPATH}.

The embedding application can steer the search by calling 
\code{Py_SetProgramName(\var{file})} \emph{before} calling 
\cfunction{Py_Initialize()}.  Note that \envvar{PYTHONHOME} still
overrides this and \envvar{PYTHONPATH} is still inserted in front of
the standard path.  An application that requires total control has to
provide its own implementation of \cfunction{Py_GetPath()},
\cfunction{Py_GetPrefix()}, \cfunction{Py_GetExecPrefix()},
\cfunction{Py_GetProgramFullPath()} (all defined in
\file{Modules/getpath.c}).

Sometimes, it is desirable to ``uninitialize'' Python.  For instance, 
the application may want to start over (make another call to 
\cfunction{Py_Initialize()}) or the application is simply done with its 
use of Python and wants to free all memory allocated by Python.  This
can be accomplished by calling \cfunction{Py_Finalize()}.  The function
\cfunction{Py_IsInitialized()} returns true iff Python is currently in the
initialized state.  More information about these functions is given in
a later chapter.


\chapter{The Very High Level Layer}
\label{veryhigh}

The functions in this chapter will let you execute Python source code
given in a file or a buffer, but they will not let you interact in a
more detailed way with the interpreter.

\begin{cfuncdesc}{int}{PyRun_AnyFile}{FILE *fp, char *filename}
\end{cfuncdesc}

\begin{cfuncdesc}{int}{PyRun_SimpleString}{char *command}
\end{cfuncdesc}

\begin{cfuncdesc}{int}{PyRun_SimpleFile}{FILE *fp, char *filename}
\end{cfuncdesc}

\begin{cfuncdesc}{int}{PyRun_InteractiveOne}{FILE *fp, char *filename}
\end{cfuncdesc}

\begin{cfuncdesc}{int}{PyRun_InteractiveLoop}{FILE *fp, char *filename}
\end{cfuncdesc}

\begin{cfuncdesc}{struct _node*}{PyParser_SimpleParseString}{char *str,
                                                             int start}
\end{cfuncdesc}

\begin{cfuncdesc}{struct _node*}{PyParser_SimpleParseFile}{FILE *fp,
                                 char *filename, int start}
\end{cfuncdesc}

\begin{cfuncdesc}{PyObject*}{PyRun_String}{char *str, int start,
                                           PyObject *globals,
                                           PyObject *locals}
\end{cfuncdesc}

\begin{cfuncdesc}{PyObject*}{PyRun_File}{FILE *fp, char *filename,
                                         int start, PyObject *globals,
                                         PyObject *locals}
\end{cfuncdesc}

\begin{cfuncdesc}{PyObject*}{Py_CompileString}{char *str, char *filename,
                                               int start}
\end{cfuncdesc}


\chapter{Reference Counting}
\label{countingRefs}

The macros in this section are used for managing reference counts
of Python objects.

\begin{cfuncdesc}{void}{Py_INCREF}{PyObject *o}
Increment the reference count for object \var{o}.  The object must
not be \NULL{}; if you aren't sure that it isn't \NULL{}, use
\cfunction{Py_XINCREF()}.
\end{cfuncdesc}

\begin{cfuncdesc}{void}{Py_XINCREF}{PyObject *o}
Increment the reference count for object \var{o}.  The object may be
\NULL{}, in which case the macro has no effect.
\end{cfuncdesc}

\begin{cfuncdesc}{void}{Py_DECREF}{PyObject *o}
Decrement the reference count for object \var{o}.  The object must
not be \NULL{}; if you aren't sure that it isn't \NULL{}, use
\cfunction{Py_XDECREF()}.  If the reference count reaches zero, the
object's type's deallocation function (which must not be \NULL{}) is
invoked.

\strong{Warning:} The deallocation function can cause arbitrary Python
code to be invoked (e.g. when a class instance with a \method{__del__()}
method is deallocated).  While exceptions in such code are not
propagated, the executed code has free access to all Python global
variables.  This means that any object that is reachable from a global
variable should be in a consistent state before \cfunction{Py_DECREF()} is
invoked.  For example, code to delete an object from a list should
copy a reference to the deleted object in a temporary variable, update
the list data structure, and then call \cfunction{Py_DECREF()} for the
temporary variable.
\end{cfuncdesc}

\begin{cfuncdesc}{void}{Py_XDECREF}{PyObject *o}
Decrement the reference count for object \var{o}.  The object may be
\NULL{}, in which case the macro has no effect; otherwise the effect
is the same as for \cfunction{Py_DECREF()}, and the same warning
applies.
\end{cfuncdesc}

The following functions or macros are only for internal use:
\cfunction{_Py_Dealloc()}, \cfunction{_Py_ForgetReference()},
\cfunction{_Py_NewReference()}, as well as the global variable
\cdata{_Py_RefTotal}.

XXX Should mention Py_Malloc(), Py_Realloc(), Py_Free(),
PyMem_Malloc(), PyMem_Realloc(), PyMem_Free(), PyMem_NEW(),
PyMem_RESIZE(), PyMem_DEL(), PyMem_XDEL().


\chapter{Exception Handling}
\label{exceptionHandling}

The functions in this chapter will let you handle and raise Python
exceptions.  It is important to understand some of the basics of
Python exception handling.  It works somewhat like the \UNIX{}
\cdata{errno} variable: there is a global indicator (per thread) of the
last error that occurred.  Most functions don't clear this on success,
but will set it to indicate the cause of the error on failure.  Most
functions also return an error indicator, usually \NULL{} if they are
supposed to return a pointer, or \code{-1} if they return an integer
(exception: the \cfunction{PyArg_Parse*()} functions return \code{1} for
success and \code{0} for failure).  When a function must fail because
some function it called failed, it generally doesn't set the error
indicator; the function it called already set it.

The error indicator consists of three Python objects corresponding to
the Python variables \code{sys.exc_type}, \code{sys.exc_value} and
\code{sys.exc_traceback}.  API functions exist to interact with the
error indicator in various ways.  There is a separate error indicator
for each thread.

% XXX Order of these should be more thoughtful.
% Either alphabetical or some kind of structure.

\begin{cfuncdesc}{void}{PyErr_Print}{}
Print a standard traceback to \code{sys.stderr} and clear the error
indicator.  Call this function only when the error indicator is set.
(Otherwise it will cause a fatal error!)
\end{cfuncdesc}

\begin{cfuncdesc}{PyObject*}{PyErr_Occurred}{}
Test whether the error indicator is set.  If set, return the exception
\emph{type} (the first argument to the last call to one of the
\cfunction{PyErr_Set*()} functions or to \cfunction{PyErr_Restore()}).  If
not set, return \NULL{}.  You do not own a reference to the return
value, so you do not need to \cfunction{Py_DECREF()} it.
\strong{Note:} do not compare the return value to a specific
exception; use \cfunction{PyErr_ExceptionMatches()} instead, shown
below.
\end{cfuncdesc}

\begin{cfuncdesc}{int}{PyErr_ExceptionMatches}{PyObject *exc}
Equivalent to
\samp{PyErr_GivenExceptionMatches(PyErr_Occurred(), \var{exc})}.
This should only be called when an exception is actually set.
\end{cfuncdesc}

\begin{cfuncdesc}{int}{PyErr_GivenExceptionMatches}{PyObject *given, PyObject *exc}
Return true if the \var{given} exception matches the exception in
\var{exc}.  If \var{exc} is a class object, this also returns true
when \var{given} is a subclass.  If \var{exc} is a tuple, all
exceptions in the tuple (and recursively in subtuples) are searched
for a match.  This should only be called when an exception is actually
set.
\end{cfuncdesc}

\begin{cfuncdesc}{void}{PyErr_NormalizeException}{PyObject**exc, PyObject**val, PyObject**tb}
Under certain circumstances, the values returned by
\cfunction{PyErr_Fetch()} below can be ``unnormalized'', meaning that
\code{*\var{exc}} is a class object but \code{*\var{val}} is not an
instance of the  same class.  This function can be used to instantiate
the class in that case.  If the values are already normalized, nothing
happens.
\end{cfuncdesc}

\begin{cfuncdesc}{void}{PyErr_Clear}{}
Clear the error indicator.  If the error indicator is not set, there
is no effect.
\end{cfuncdesc}

\begin{cfuncdesc}{void}{PyErr_Fetch}{PyObject **ptype, PyObject **pvalue, PyObject **ptraceback}
Retrieve the error indicator into three variables whose addresses are
passed.  If the error indicator is not set, set all three variables to
\NULL{}.  If it is set, it will be cleared and you own a reference to
each object retrieved.  The value and traceback object may be \NULL{}
even when the type object is not.  \strong{Note:} this function is
normally only used by code that needs to handle exceptions or by code
that needs to save and restore the error indicator temporarily.
\end{cfuncdesc}

\begin{cfuncdesc}{void}{PyErr_Restore}{PyObject *type, PyObject *value, PyObject *traceback}
Set  the error indicator from the three objects.  If the error
indicator is already set, it is cleared first.  If the objects are
\NULL{}, the error indicator is cleared.  Do not pass a \NULL{} type
and non-\NULL{} value or traceback.  The exception type should be a
string or class; if it is a class, the value should be an instance of
that class.  Do not pass an invalid exception type or value.
(Violating these rules will cause subtle problems later.)  This call
takes away a reference to each object, i.e. you must own a reference
to each object before the call and after the call you no longer own
these references.  (If you don't understand this, don't use this
function.  I warned you.)  \strong{Note:} this function is normally
only used by code that needs to save and restore the error indicator
temporarily.
\end{cfuncdesc}

\begin{cfuncdesc}{void}{PyErr_SetString}{PyObject *type, char *message}
This is the most common way to set the error indicator.  The first
argument specifies the exception type; it is normally one of the
standard exceptions, e.g. \cdata{PyExc_RuntimeError}.  You need not
increment its reference count.  The second argument is an error
message; it is converted to a string object.
\end{cfuncdesc}

\begin{cfuncdesc}{void}{PyErr_SetObject}{PyObject *type, PyObject *value}
This function is similar to \cfunction{PyErr_SetString()} but lets you
specify an arbitrary Python object for the ``value'' of the exception.
You need not increment its reference count.
\end{cfuncdesc}

\begin{cfuncdesc}{void}{PyErr_SetNone}{PyObject *type}
This is a shorthand for \samp{PyErr_SetObject(\var{type}, Py_None)}.
\end{cfuncdesc}

\begin{cfuncdesc}{int}{PyErr_BadArgument}{}
This is a shorthand for \samp{PyErr_SetString(PyExc_TypeError,
\var{message})}, where \var{message} indicates that a built-in operation
was invoked with an illegal argument.  It is mostly for internal use.
\end{cfuncdesc}

\begin{cfuncdesc}{PyObject*}{PyErr_NoMemory}{}
This is a shorthand for \samp{PyErr_SetNone(PyExc_MemoryError)}; it
returns \NULL{} so an object allocation function can write
\samp{return PyErr_NoMemory();} when it runs out of memory.
\end{cfuncdesc}

\begin{cfuncdesc}{PyObject*}{PyErr_SetFromErrno}{PyObject *type}
This is a convenience function to raise an exception when a \C{} library
function has returned an error and set the \C{} variable \cdata{errno}.
It constructs a tuple object whose first item is the integer
\cdata{errno} value and whose second item is the corresponding error
message (gotten from \cfunction{strerror()}), and then calls
\samp{PyErr_SetObject(\var{type}, \var{object})}.  On \UNIX{}, when
the \cdata{errno} value is \constant{EINTR}, indicating an interrupted
system call, this calls \cfunction{PyErr_CheckSignals()}, and if that set
the error indicator, leaves it set to that.  The function always
returns \NULL{}, so a wrapper function around a system call can write 
\samp{return PyErr_SetFromErrno();} when  the system call returns an
error.
\end{cfuncdesc}

\begin{cfuncdesc}{void}{PyErr_BadInternalCall}{}
This is a shorthand for \samp{PyErr_SetString(PyExc_TypeError,
\var{message})}, where \var{message} indicates that an internal
operation (e.g. a Python/C API function) was invoked with an illegal
argument.  It is mostly for internal use.
\end{cfuncdesc}

\begin{cfuncdesc}{int}{PyErr_CheckSignals}{}
This function interacts with Python's signal handling.  It checks
whether a signal has been sent to the processes and if so, invokes the
corresponding signal handler.  If the
\module{signal}\refbimodindex{signal} module is supported, this can
invoke a signal handler written in Python.  In all cases, the default
effect for \constant{SIGINT} is to raise the
\exception{KeyboadInterrupt} exception.  If an exception is raised the 
error indicator is set and the function returns \code{1}; otherwise
the function returns \code{0}.  The error indicator may or may not be
cleared if it was previously set.
\end{cfuncdesc}

\begin{cfuncdesc}{void}{PyErr_SetInterrupt}{}
This function is obsolete (XXX or platform dependent?).  It simulates
the effect of a \constant{SIGINT} signal arriving --- the next time
\cfunction{PyErr_CheckSignals()} is called,
\exception{KeyboadInterrupt} will be raised.
\end{cfuncdesc}

\begin{cfuncdesc}{PyObject*}{PyErr_NewException}{char *name,
                                                 PyObject *base,
                                                 PyObject *dict}
This utility function creates and returns a new exception object.  The
\var{name} argument must be the name of the new exception, a \C{} string
of the form \code{module.class}.  The \var{base} and \var{dict}
arguments are normally \NULL{}.  Normally, this creates a class
object derived from the root for all exceptions, the built-in name
\exception{Exception} (accessible in \C{} as \cdata{PyExc_Exception}).
In this case the \member{__module__} attribute of the new class is set to the
first part (up to the last dot) of the \var{name} argument, and the
class name is set to the last part (after the last dot).  When the
user has specified the \code{-X} command line option to use string
exceptions, for backward compatibility, or when the \var{base}
argument is not a class object (and not \NULL{}), a string object
created from the entire \var{name} argument is returned.  The
\var{base} argument can be used to specify an alternate base class.
The \var{dict} argument can be used to specify a dictionary of class
variables and methods.
\end{cfuncdesc}


\section{Standard Exceptions}
\label{standardExceptions}

All standard Python exceptions are available as global variables whose
names are \samp{PyExc_} followed by the Python exception name.
These have the type \ctype{PyObject *}; they are all either class
objects or string objects, depending on the use of the \code{-X}
option to the interpreter.  For completeness, here are all the
variables:
\cdata{PyExc_Exception},
\cdata{PyExc_StandardError},
\cdata{PyExc_ArithmeticError},
\cdata{PyExc_LookupError},
\cdata{PyExc_AssertionError},
\cdata{PyExc_AttributeError},
\cdata{PyExc_EOFError},
\cdata{PyExc_FloatingPointError},
\cdata{PyExc_IOError},
\cdata{PyExc_ImportError},
\cdata{PyExc_IndexError},
\cdata{PyExc_KeyError},
\cdata{PyExc_KeyboardInterrupt},
\cdata{PyExc_MemoryError},
\cdata{PyExc_NameError},
\cdata{PyExc_OverflowError},
\cdata{PyExc_RuntimeError},
\cdata{PyExc_SyntaxError},
\cdata{PyExc_SystemError},
\cdata{PyExc_SystemExit},
\cdata{PyExc_TypeError},
\cdata{PyExc_ValueError},
\cdata{PyExc_ZeroDivisionError}.


\chapter{Utilities}
\label{utilities}

The functions in this chapter perform various utility tasks, such as
parsing function arguments and constructing Python values from \C{}
values.

\section{OS Utilities}
\label{os}

\begin{cfuncdesc}{int}{Py_FdIsInteractive}{FILE *fp, char *filename}
Return true (nonzero) if the standard I/O file \var{fp} with name
\var{filename} is deemed interactive.  This is the case for files for
which \samp{isatty(fileno(\var{fp}))} is true.  If the global flag
\cdata{Py_InteractiveFlag} is true, this function also returns true if
the \var{name} pointer is \NULL{} or if the name is equal to one of
the strings \code{"<stdin>"} or \code{"???"}.
\end{cfuncdesc}

\begin{cfuncdesc}{long}{PyOS_GetLastModificationTime}{char *filename}
Return the time of last modification of the file \var{filename}.
The result is encoded in the same way as the timestamp returned by
the standard \C{} library function \cfunction{time()}.
\end{cfuncdesc}


\section{Process Control}
\label{processControl}

\begin{cfuncdesc}{void}{Py_FatalError}{char *message}
Print a fatal error message and kill the process.  No cleanup is
performed.  This function should only be invoked when a condition is
detected that would make it dangerous to continue using the Python
interpreter; e.g., when the object administration appears to be
corrupted.  On \UNIX{}, the standard \C{} library function
\cfunction{abort()} is called which will attempt to produce a
\file{core} file.
\end{cfuncdesc}

\begin{cfuncdesc}{void}{Py_Exit}{int status}
Exit the current process.  This calls \cfunction{Py_Finalize()} and
then calls the standard \C{} library function
\code{exit(\var{status})}.
\end{cfuncdesc}

\begin{cfuncdesc}{int}{Py_AtExit}{void (*func) ()}
Register a cleanup function to be called by \cfunction{Py_Finalize()}.
The cleanup function will be called with no arguments and should
return no value.  At most 32 cleanup functions can be registered.
When the registration is successful, \cfunction{Py_AtExit()} returns
\code{0}; on failure, it returns \code{-1}.  The cleanup function
registered last is called first.  Each cleanup function will be called
at most once.  Since Python's internal finallization will have
completed before the cleanup function, no Python APIs should be called
by \var{func}.
\end{cfuncdesc}


\section{Importing Modules}
\label{importing}

\begin{cfuncdesc}{PyObject*}{PyImport_ImportModule}{char *name}
This is a simplified interface to \cfunction{PyImport_ImportModuleEx()}
below, leaving the \var{globals} and \var{locals} arguments set to
\NULL{}.  When the \var{name} argument contains a dot (i.e., when
it specifies a submodule of a package), the \var{fromlist} argument is
set to the list \code{['*']} so that the return value is the named
module rather than the top-level package containing it as would
otherwise be the case.  (Unfortunately, this has an additional side
effect when \var{name} in fact specifies a subpackage instead of a
submodule: the submodules specified in the package's \code{__all__}
variable are loaded.)  Return a new reference to the imported module,
or \NULL{} with an exception set on failure (the module may still
be created in this case --- examine \code{sys.modules} to find out).
\end{cfuncdesc}

\begin{cfuncdesc}{PyObject*}{PyImport_ImportModuleEx}{char *name, PyObject *globals, PyObject *locals, PyObject *fromlist}
Import a module.  This is best described by referring to the built-in
Python function \function{__import__()}\bifuncindex{__import__}, as
the standard \function{__import__()} function calls this function
directly.

The return value is a new reference to the imported module or
top-level package, or \NULL{} with an exception set on failure
(the module may still be created in this case).  Like for
\function{__import__()}, the return value when a submodule of a
package was requested is normally the top-level package, unless a
non-empty \var{fromlist} was given.
\end{cfuncdesc}

\begin{cfuncdesc}{PyObject*}{PyImport_Import}{PyObject *name}
This is a higher-level interface that calls the current ``import hook
function''.  It invokes the \function{__import__()} function from the
\code{__builtins__} of the current globals.  This means that the
import is done using whatever import hooks are installed in the
current environment, e.g. by \module{rexec}\refstmodindex{rexec} or
\module{ihooks}\refstmodindex{ihooks}.
\end{cfuncdesc}

\begin{cfuncdesc}{PyObject*}{PyImport_ReloadModule}{PyObject *m}
Reload a module.  This is best described by referring to the built-in
Python function \function{reload()}\bifuncindex{reload}, as the standard
\function{reload()} function calls this function directly.  Return a
new reference to the reloaded module, or \NULL{} with an exception set
on failure (the module still exists in this case).
\end{cfuncdesc}

\begin{cfuncdesc}{PyObject*}{PyImport_AddModule}{char *name}
Return the module object corresponding to a module name.  The
\var{name} argument may be of the form \code{package.module}).  First
check the modules dictionary if there's one there, and if not, create
a new one and insert in in the modules dictionary.  Because the former
action is most common, this does not return a new reference, and you
do not own the returned reference.  Return \NULL{} with an
exception set on failure.  \strong{Note:} this function returns
a ``borrowed'' reference.  
\end{cfuncdesc}

\begin{cfuncdesc}{PyObject*}{PyImport_ExecCodeModule}{char *name, PyObject *co}
Given a module name (possibly of the form \code{package.module}) and a
code object read from a Python bytecode file or obtained from the
built-in function \function{compile()}\bifuncindex{compile}, load the
module.  Return a new reference to the module object, or \NULL{} with
an exception set if an error occurred (the module may still be created
in this case).  (This function would reload the module if it was
already imported.)
\end{cfuncdesc}

\begin{cfuncdesc}{long}{PyImport_GetMagicNumber}{}
Return the magic number for Python bytecode files (a.k.a. \file{.pyc}
and \file{.pyo} files).  The magic number should be present in the
first four bytes of the bytecode file, in little-endian byte order.
\end{cfuncdesc}

\begin{cfuncdesc}{PyObject*}{PyImport_GetModuleDict}{}
Return the dictionary used for the module administration
(a.k.a. \code{sys.modules}).  Note that this is a per-interpreter
variable.
\end{cfuncdesc}

\begin{cfuncdesc}{void}{_PyImport_Init}{}
Initialize the import mechanism.  For internal use only.
\end{cfuncdesc}

\begin{cfuncdesc}{void}{PyImport_Cleanup}{}
Empty the module table.  For internal use only.
\end{cfuncdesc}

\begin{cfuncdesc}{void}{_PyImport_Fini}{}
Finalize the import mechanism.  For internal use only.
\end{cfuncdesc}

\begin{cfuncdesc}{PyObject*}{_PyImport_FindExtension}{char *, char *}
For internal use only.
\end{cfuncdesc}

\begin{cfuncdesc}{PyObject*}{_PyImport_FixupExtension}{char *, char *}
For internal use only.
\end{cfuncdesc}

\begin{cfuncdesc}{int}{PyImport_ImportFrozenModule}{char *}
Load a frozen module.  Return \code{1} for success, \code{0} if the
module is not found, and \code{-1} with an exception set if the
initialization failed.  To access the imported module on a successful
load, use \cfunction{PyImport_ImportModule()}.
(Note the misnomer --- this function would reload the module if it was
already imported.)
\end{cfuncdesc}

\begin{ctypedesc}{struct _frozen}
This is the structure type definition for frozen module descriptors,
as generated by the \program{freeze}\index{freeze utility} utility
(see \file{Tools/freeze/} in the Python source distribution).  Its
definition is:

\begin{verbatim}
struct _frozen {
    char *name;
    unsigned char *code;
    int size;
};
\end{verbatim}
\end{ctypedesc}

\begin{cvardesc}{struct _frozen*}{PyImport_FrozenModules}
This pointer is initialized to point to an array of \ctype{struct
_frozen} records, terminated by one whose members are all \NULL{}
or zero.  When a frozen module is imported, it is searched in this
table.  Third-party code could play tricks with this to provide a
dynamically created collection of frozen modules.
\end{cvardesc}


\chapter{Abstract Objects Layer}
\label{abstract}

The functions in this chapter interact with Python objects regardless
of their type, or with wide classes of object types (e.g. all
numerical types, or all sequence types).  When used on object types
for which they do not apply, they will flag a Python exception.

\section{Object Protocol}
\label{object}

\begin{cfuncdesc}{int}{PyObject_Print}{PyObject *o, FILE *fp, int flags}
Print an object \var{o}, on file \var{fp}.  Returns \code{-1} on error
The flags argument is used to enable certain printing
options. The only option currently supported is
\constant{Py_PRINT_RAW}.
\end{cfuncdesc}

\begin{cfuncdesc}{int}{PyObject_HasAttrString}{PyObject *o, char *attr_name}
Returns \code{1} if \var{o} has the attribute \var{attr_name}, and
\code{0} otherwise.  This is equivalent to the Python expression
\samp{hasattr(\var{o}, \var{attr_name})}.
This function always succeeds.
\end{cfuncdesc}

\begin{cfuncdesc}{PyObject*}{PyObject_GetAttrString}{PyObject *o, char *attr_name}
Retrieve an attribute named \var{attr_name} from object \var{o}.
Returns the attribute value on success, or \NULL{} on failure.
This is the equivalent of the Python expression
\samp{\var{o}.\var{attr_name}}.
\end{cfuncdesc}


\begin{cfuncdesc}{int}{PyObject_HasAttr}{PyObject *o, PyObject *attr_name}
Returns \code{1} if \var{o} has the attribute \var{attr_name}, and
\code{0} otherwise.  This is equivalent to the Python expression
\samp{hasattr(\var{o}, \var{attr_name})}. 
This function always succeeds.
\end{cfuncdesc}


\begin{cfuncdesc}{PyObject*}{PyObject_GetAttr}{PyObject *o, PyObject *attr_name}
Retrieve an attribute named \var{attr_name} from object \var{o}.
Returns the attribute value on success, or \NULL{} on failure.
This is the equivalent of the Python expression
\samp{\var{o}.\var{attr_name}}.
\end{cfuncdesc}


\begin{cfuncdesc}{int}{PyObject_SetAttrString}{PyObject *o, char *attr_name, PyObject *v}
Set the value of the attribute named \var{attr_name}, for object
\var{o}, to the value \var{v}. Returns \code{-1} on failure.  This is
the equivalent of the Python statement \samp{\var{o}.\var{attr_name} =
\var{v}}.
\end{cfuncdesc}


\begin{cfuncdesc}{int}{PyObject_SetAttr}{PyObject *o, PyObject *attr_name, PyObject *v}
Set the value of the attribute named \var{attr_name}, for
object \var{o},
to the value \var{v}. Returns \code{-1} on failure.  This is
the equivalent of the Python statement \samp{\var{o}.\var{attr_name} =
\var{v}}.
\end{cfuncdesc}


\begin{cfuncdesc}{int}{PyObject_DelAttrString}{PyObject *o, char *attr_name}
Delete attribute named \var{attr_name}, for object \var{o}. Returns
\code{-1} on failure.  This is the equivalent of the Python
statement: \samp{del \var{o}.\var{attr_name}}.
\end{cfuncdesc}


\begin{cfuncdesc}{int}{PyObject_DelAttr}{PyObject *o, PyObject *attr_name}
Delete attribute named \var{attr_name}, for object \var{o}. Returns
\code{-1} on failure.  This is the equivalent of the Python
statement \samp{del \var{o}.\var{attr_name}}.
\end{cfuncdesc}


\begin{cfuncdesc}{int}{PyObject_Cmp}{PyObject *o1, PyObject *o2, int *result}
Compare the values of \var{o1} and \var{o2} using a routine provided
by \var{o1}, if one exists, otherwise with a routine provided by
\var{o2}.  The result of the comparison is returned in \var{result}.
Returns \code{-1} on failure.  This is the equivalent of the Python
statement \samp{\var{result} = cmp(\var{o1}, \var{o2})}.
\end{cfuncdesc}


\begin{cfuncdesc}{int}{PyObject_Compare}{PyObject *o1, PyObject *o2}
Compare the values of \var{o1} and \var{o2} using a routine provided
by \var{o1}, if one exists, otherwise with a routine provided by
\var{o2}.  Returns the result of the comparison on success.  On error,
the value returned is undefined; use \cfunction{PyErr_Occurred()} to
detect an error.  This is equivalent to the
Python expression \samp{cmp(\var{o1}, \var{o2})}.
\end{cfuncdesc}


\begin{cfuncdesc}{PyObject*}{PyObject_Repr}{PyObject *o}
Compute the string representation of object, \var{o}.  Returns the
string representation on success, \NULL{} on failure.  This is
the equivalent of the Python expression \samp{repr(\var{o})}.
Called by the \function{repr()}\bifuncindex{repr} built-in function
and by reverse quotes.
\end{cfuncdesc}


\begin{cfuncdesc}{PyObject*}{PyObject_Str}{PyObject *o}
Compute the string representation of object \var{o}.  Returns the
string representation on success, \NULL{} on failure.  This is
the equivalent of the Python expression \samp{str(\var{o})}.
Called by the \function{str()}\bifuncindex{str} built-in function and
by the \keyword{print} statement.
\end{cfuncdesc}


\begin{cfuncdesc}{int}{PyCallable_Check}{PyObject *o}
Determine if the object \var{o}, is callable.  Return \code{1} if the
object is callable and \code{0} otherwise.
This function always succeeds.
\end{cfuncdesc}


\begin{cfuncdesc}{PyObject*}{PyObject_CallObject}{PyObject *callable_object, PyObject *args}
Call a callable Python object \var{callable_object}, with
arguments given by the tuple \var{args}.  If no arguments are
needed, then args may be \NULL{}.  Returns the result of the
call on success, or \NULL{} on failure.  This is the equivalent
of the Python expression \samp{apply(\var{o}, \var{args})}.
\end{cfuncdesc}

\begin{cfuncdesc}{PyObject*}{PyObject_CallFunction}{PyObject *callable_object, char *format, ...}
Call a callable Python object \var{callable_object}, with a
variable number of \C{} arguments. The \C{} arguments are described
using a \cfunction{Py_BuildValue()} style format string. The format may
be \NULL{}, indicating that no arguments are provided.  Returns the
result of the call on success, or \NULL{} on failure.  This is
the equivalent of the Python expression \samp{apply(\var{o},
\var{args})}.
\end{cfuncdesc}


\begin{cfuncdesc}{PyObject*}{PyObject_CallMethod}{PyObject *o, char *m, char *format, ...}
Call the method named \var{m} of object \var{o} with a variable number
of C arguments.  The \C{} arguments are described by a
\cfunction{Py_BuildValue()} format string.  The format may be \NULL{},
indicating that no arguments are provided. Returns the result of the
call on success, or \NULL{} on failure.  This is the equivalent of the
Python expression \samp{\var{o}.\var{method}(\var{args})}.
Note that Special method names, such as \method{__add__()},
\method{__getitem__()}, and so on are not supported. The specific
abstract-object routines for these must be used.
\end{cfuncdesc}


\begin{cfuncdesc}{int}{PyObject_Hash}{PyObject *o}
Compute and return the hash value of an object \var{o}.  On
failure, return \code{-1}.  This is the equivalent of the Python
expression \samp{hash(\var{o})}.
\end{cfuncdesc}


\begin{cfuncdesc}{int}{PyObject_IsTrue}{PyObject *o}
Returns \code{1} if the object \var{o} is considered to be true, and
\code{0} otherwise. This is equivalent to the Python expression
\samp{not not \var{o}}.
This function always succeeds.
\end{cfuncdesc}


\begin{cfuncdesc}{PyObject*}{PyObject_Type}{PyObject *o}
On success, returns a type object corresponding to the object
type of object \var{o}. On failure, returns \NULL{}.  This is
equivalent to the Python expression \samp{type(\var{o})}.
\bifuncindex{type}
\end{cfuncdesc}

\begin{cfuncdesc}{int}{PyObject_Length}{PyObject *o}
Return the length of object \var{o}.  If the object \var{o} provides
both sequence and mapping protocols, the sequence length is
returned. On error, \code{-1} is returned.  This is the equivalent
to the Python expression \samp{len(\var{o})}.
\end{cfuncdesc}


\begin{cfuncdesc}{PyObject*}{PyObject_GetItem}{PyObject *o, PyObject *key}
Return element of \var{o} corresponding to the object \var{key} or
\NULL{} on failure. This is the equivalent of the Python expression
\samp{\var{o}[\var{key}]}.
\end{cfuncdesc}


\begin{cfuncdesc}{int}{PyObject_SetItem}{PyObject *o, PyObject *key, PyObject *v}
Map the object \var{key} to the value \var{v}.
Returns \code{-1} on failure.  This is the equivalent
of the Python statement \samp{\var{o}[\var{key}] = \var{v}}.
\end{cfuncdesc}


\begin{cfuncdesc}{int}{PyObject_DelItem}{PyObject *o, PyObject *key, PyObject *v}
Delete the mapping for \var{key} from \var{o}.  Returns \code{-1} on
failure. This is the equivalent of the Python statement \samp{del
\var{o}[\var{key}]}.
\end{cfuncdesc}


\section{Number Protocol}
\label{number}

\begin{cfuncdesc}{int}{PyNumber_Check}{PyObject *o}
Returns \code{1} if the object \var{o} provides numeric protocols, and
false otherwise. 
This function always succeeds.
\end{cfuncdesc}


\begin{cfuncdesc}{PyObject*}{PyNumber_Add}{PyObject *o1, PyObject *o2}
Returns the result of adding \var{o1} and \var{o2}, or \NULL{} on
failure.  This is the equivalent of the Python expression
\samp{\var{o1} + \var{o2}}.
\end{cfuncdesc}


\begin{cfuncdesc}{PyObject*}{PyNumber_Subtract}{PyObject *o1, PyObject *o2}
Returns the result of subtracting \var{o2} from \var{o1}, or \NULL{}
on failure.  This is the equivalent of the Python expression
\samp{\var{o1} - \var{o2}}.
\end{cfuncdesc}


\begin{cfuncdesc}{PyObject*}{PyNumber_Multiply}{PyObject *o1, PyObject *o2}
Returns the result of multiplying \var{o1} and \var{o2}, or \NULL{} on
failure.  This is the equivalent of the Python expression
\samp{\var{o1} * \var{o2}}.
\end{cfuncdesc}


\begin{cfuncdesc}{PyObject*}{PyNumber_Divide}{PyObject *o1, PyObject *o2}
Returns the result of dividing \var{o1} by \var{o2}, or \NULL{} on
failure. 
This is the equivalent of the Python expression \samp{\var{o1} /
\var{o2}}.
\end{cfuncdesc}


\begin{cfuncdesc}{PyObject*}{PyNumber_Remainder}{PyObject *o1, PyObject *o2}
Returns the remainder of dividing \var{o1} by \var{o2}, or \NULL{} on
failure.  This is the equivalent of the Python expression
\samp{\var{o1} \% \var{o2}}.
\end{cfuncdesc}


\begin{cfuncdesc}{PyObject*}{PyNumber_Divmod}{PyObject *o1, PyObject *o2}
See the built-in function \function{divmod()}\bifuncindex{divmod}.
Returns \NULL{} on failure.  This is the equivalent of the Python
expression \samp{divmod(\var{o1}, \var{o2})}.
\end{cfuncdesc}


\begin{cfuncdesc}{PyObject*}{PyNumber_Power}{PyObject *o1, PyObject *o2, PyObject *o3}
See the built-in function \function{pow()}\bifuncindex{pow}.  Returns
\NULL{} on failure. This is the equivalent of the Python expression
\samp{pow(\var{o1}, \var{o2}, \var{o3})}, where \var{o3} is optional.
If \var{o3} is to be ignored, pass \cdata{Py_None} in its place.
\end{cfuncdesc}


\begin{cfuncdesc}{PyObject*}{PyNumber_Negative}{PyObject *o}
Returns the negation of \var{o} on success, or \NULL{} on failure.
This is the equivalent of the Python expression \samp{-\var{o}}.
\end{cfuncdesc}


\begin{cfuncdesc}{PyObject*}{PyNumber_Positive}{PyObject *o}
Returns \var{o} on success, or \NULL{} on failure.
This is the equivalent of the Python expression \samp{+\var{o}}.
\end{cfuncdesc}


\begin{cfuncdesc}{PyObject*}{PyNumber_Absolute}{PyObject *o}
Returns the absolute value of \var{o}, or \NULL{} on failure.  This is
the equivalent of the Python expression \samp{abs(\var{o})}.
\end{cfuncdesc}


\begin{cfuncdesc}{PyObject*}{PyNumber_Invert}{PyObject *o}
Returns the bitwise negation of \var{o} on success, or \NULL{} on
failure.  This is the equivalent of the Python expression
\samp{\~\var{o}}.
\end{cfuncdesc}


\begin{cfuncdesc}{PyObject*}{PyNumber_Lshift}{PyObject *o1, PyObject *o2}
Returns the result of left shifting \var{o1} by \var{o2} on success,
or \NULL{} on failure.  This is the equivalent of the Python
expression \samp{\var{o1} << \var{o2}}.
\end{cfuncdesc}


\begin{cfuncdesc}{PyObject*}{PyNumber_Rshift}{PyObject *o1, PyObject *o2}
Returns the result of right shifting \var{o1} by \var{o2} on success,
or \NULL{} on failure.  This is the equivalent of the Python
expression \samp{\var{o1} >> \var{o2}}.
\end{cfuncdesc}


\begin{cfuncdesc}{PyObject*}{PyNumber_And}{PyObject *o1, PyObject *o2}
Returns the result of ``anding'' \var{o2} and \var{o2} on success and
\NULL{} on failure. This is the equivalent of the Python
expression \samp{\var{o1} and \var{o2}}.
\end{cfuncdesc}


\begin{cfuncdesc}{PyObject*}{PyNumber_Xor}{PyObject *o1, PyObject *o2}
Returns the bitwise exclusive or of \var{o1} by \var{o2} on success,
or \NULL{} on failure.  This is the equivalent of the Python
expression \samp{\var{o1} \^{ }\var{o2}}.
\end{cfuncdesc}

\begin{cfuncdesc}{PyObject*}{PyNumber_Or}{PyObject *o1, PyObject *o2}
Returns the result of \var{o1} and \var{o2} on success, or \NULL{} on
failure.  This is the equivalent of the Python expression
\samp{\var{o1} or \var{o2}}.
\end{cfuncdesc}


\begin{cfuncdesc}{PyObject*}{PyNumber_Coerce}{PyObject **p1, PyObject **p2}
This function takes the addresses of two variables of type
\ctype{PyObject*}.

If the objects pointed to by \code{*\var{p1}} and \code{*\var{p2}}
have the same type, increment their reference count and return
\code{0} (success). If the objects can be converted to a common
numeric type, replace \code{*p1} and \code{*p2} by their converted
value (with 'new' reference counts), and return \code{0}.
If no conversion is possible, or if some other error occurs,
return \code{-1} (failure) and don't increment the reference counts.
The call \code{PyNumber_Coerce(\&o1, \&o2)} is equivalent to the
Python statement \samp{\var{o1}, \var{o2} = coerce(\var{o1},
\var{o2})}.
\end{cfuncdesc}


\begin{cfuncdesc}{PyObject*}{PyNumber_Int}{PyObject *o}
Returns the \var{o} converted to an integer object on success, or
\NULL{} on failure.  This is the equivalent of the Python
expression \samp{int(\var{o})}.
\end{cfuncdesc}


\begin{cfuncdesc}{PyObject*}{PyNumber_Long}{PyObject *o}
Returns the \var{o} converted to a long integer object on success,
or \NULL{} on failure.  This is the equivalent of the Python
expression \samp{long(\var{o})}.
\end{cfuncdesc}


\begin{cfuncdesc}{PyObject*}{PyNumber_Float}{PyObject *o}
Returns the \var{o} converted to a float object on success, or \NULL{}
on failure.  This is the equivalent of the Python expression
\samp{float(\var{o})}.
\end{cfuncdesc}


\section{Sequence Protocol}
\label{sequence}

\begin{cfuncdesc}{int}{PySequence_Check}{PyObject *o}
Return \code{1} if the object provides sequence protocol, and \code{0}
otherwise.  
This function always succeeds.
\end{cfuncdesc}


\begin{cfuncdesc}{PyObject*}{PySequence_Concat}{PyObject *o1, PyObject *o2}
Return the concatenation of \var{o1} and \var{o2} on success, and \NULL{} on
failure.   This is the equivalent of the Python
expression \samp{\var{o1} + \var{o2}}.
\end{cfuncdesc}


\begin{cfuncdesc}{PyObject*}{PySequence_Repeat}{PyObject *o, int count}
Return the result of repeating sequence object \var{o} \var{count}
times, or \NULL{} on failure.  This is the equivalent of the Python
expression \samp{\var{o} * \var{count}}.
\end{cfuncdesc}


\begin{cfuncdesc}{PyObject*}{PySequence_GetItem}{PyObject *o, int i}
Return the \var{i}th element of \var{o}, or \NULL{} on failure. This
is the equivalent of the Python expression \samp{\var{o}[\var{i}]}.
\end{cfuncdesc}


\begin{cfuncdesc}{PyObject*}{PySequence_GetSlice}{PyObject *o, int i1, int i2}
Return the slice of sequence object \var{o} between \var{i1} and
\var{i2}, or \NULL{} on failure. This is the equivalent of the Python
expression \samp{\var{o}[\var{i1}:\var{i2}]}.
\end{cfuncdesc}


\begin{cfuncdesc}{int}{PySequence_SetItem}{PyObject *o, int i, PyObject *v}
Assign object \var{v} to the \var{i}th element of \var{o}.
Returns \code{-1} on failure.  This is the equivalent of the Python
statement \samp{\var{o}[\var{i}] = \var{v}}.
\end{cfuncdesc}

\begin{cfuncdesc}{int}{PySequence_DelItem}{PyObject *o, int i}
Delete the \var{i}th element of object \var{v}.  Returns
\code{-1} on failure.  This is the equivalent of the Python
statement \samp{del \var{o}[\var{i}]}.
\end{cfuncdesc}

\begin{cfuncdesc}{int}{PySequence_SetSlice}{PyObject *o, int i1, int i2, PyObject *v}
Assign the sequence object \var{v} to the slice in sequence
object \var{o} from \var{i1} to \var{i2}.  This is the equivalent of
the Python statement \samp{\var{o}[\var{i1}:\var{i2}] = \var{v}}.
\end{cfuncdesc}

\begin{cfuncdesc}{int}{PySequence_DelSlice}{PyObject *o, int i1, int i2}
Delete the slice in sequence object \var{o} from \var{i1} to \var{i2}.
Returns \code{-1} on failure. This is the equivalent of the Python
statement \samp{del \var{o}[\var{i1}:\var{i2}]}.
\end{cfuncdesc}

\begin{cfuncdesc}{PyObject*}{PySequence_Tuple}{PyObject *o}
Returns the \var{o} as a tuple on success, and \NULL{} on failure.
This is equivalent to the Python expression \code{tuple(\var{o})}.
\end{cfuncdesc}

\begin{cfuncdesc}{int}{PySequence_Count}{PyObject *o, PyObject *value}
Return the number of occurrences of \var{value} in \var{o}, that is,
return the number of keys for which \code{\var{o}[\var{key}] ==
\var{value}}.  On failure, return \code{-1}.  This is equivalent to
the Python expression \samp{\var{o}.count(\var{value})}.
\end{cfuncdesc}

\begin{cfuncdesc}{int}{PySequence_In}{PyObject *o, PyObject *value}
Determine if \var{o} contains \var{value}.  If an item in \var{o} is
equal to \var{value}, return \code{1}, otherwise return \code{0}.  On
error, return \code{-1}.  This is equivalent to the Python expression
\samp{\var{value} in \var{o}}.
\end{cfuncdesc}

\begin{cfuncdesc}{int}{PySequence_Index}{PyObject *o, PyObject *value}
Return the first index \var{i} for which \code{\var{o}[\var{i}] ==
\var{value}}.  On error, return \code{-1}.    This is equivalent to
the Python expression \samp{\var{o}.index(\var{value})}.
\end{cfuncdesc}


\section{Mapping Protocol}
\label{mapping}

\begin{cfuncdesc}{int}{PyMapping_Check}{PyObject *o}
Return \code{1} if the object provides mapping protocol, and \code{0}
otherwise.  
This function always succeeds.
\end{cfuncdesc}


\begin{cfuncdesc}{int}{PyMapping_Length}{PyObject *o}
Returns the number of keys in object \var{o} on success, and \code{-1}
on failure.  For objects that do not provide sequence protocol,
this is equivalent to the Python expression \samp{len(\var{o})}.
\end{cfuncdesc}


\begin{cfuncdesc}{int}{PyMapping_DelItemString}{PyObject *o, char *key}
Remove the mapping for object \var{key} from the object \var{o}.
Return \code{-1} on failure.  This is equivalent to
the Python statement \samp{del \var{o}[\var{key}]}.
\end{cfuncdesc}


\begin{cfuncdesc}{int}{PyMapping_DelItem}{PyObject *o, PyObject *key}
Remove the mapping for object \var{key} from the object \var{o}.
Return \code{-1} on failure.  This is equivalent to
the Python statement \samp{del \var{o}[\var{key}]}.
\end{cfuncdesc}


\begin{cfuncdesc}{int}{PyMapping_HasKeyString}{PyObject *o, char *key}
On success, return \code{1} if the mapping object has the key \var{key}
and \code{0} otherwise.  This is equivalent to the Python expression
\samp{\var{o}.has_key(\var{key})}. 
This function always succeeds.
\end{cfuncdesc}


\begin{cfuncdesc}{int}{PyMapping_HasKey}{PyObject *o, PyObject *key}
Return \code{1} if the mapping object has the key \var{key} and
\code{0} otherwise.  This is equivalent to the Python expression
\samp{\var{o}.has_key(\var{key})}. 
This function always succeeds.
\end{cfuncdesc}


\begin{cfuncdesc}{PyObject*}{PyMapping_Keys}{PyObject *o}
On success, return a list of the keys in object \var{o}.  On
failure, return \NULL{}. This is equivalent to the Python
expression \samp{\var{o}.keys()}.
\end{cfuncdesc}


\begin{cfuncdesc}{PyObject*}{PyMapping_Values}{PyObject *o}
On success, return a list of the values in object \var{o}.  On
failure, return \NULL{}. This is equivalent to the Python
expression \samp{\var{o}.values()}.
\end{cfuncdesc}


\begin{cfuncdesc}{PyObject*}{PyMapping_Items}{PyObject *o}
On success, return a list of the items in object \var{o}, where
each item is a tuple containing a key-value pair.  On
failure, return \NULL{}. This is equivalent to the Python
expression \samp{\var{o}.items()}.
\end{cfuncdesc}

\begin{cfuncdesc}{int}{PyMapping_Clear}{PyObject *o}
Make object \var{o} empty.  Returns \code{1} on success and \code{0}
on failure.  This is equivalent to the Python statement
\samp{for key in \var{o}.keys(): del \var{o}[key]}.
\end{cfuncdesc}


\begin{cfuncdesc}{PyObject*}{PyMapping_GetItemString}{PyObject *o, char *key}
Return element of \var{o} corresponding to the object \var{key} or
\NULL{} on failure. This is the equivalent of the Python expression
\samp{\var{o}[\var{key}]}.
\end{cfuncdesc}

\begin{cfuncdesc}{PyObject*}{PyMapping_SetItemString}{PyObject *o, char *key, PyObject *v}
Map the object \var{key} to the value \var{v} in object \var{o}.
Returns \code{-1} on failure.  This is the equivalent of the Python
statement \samp{\var{o}[\var{key}] = \var{v}}.
\end{cfuncdesc}


\section{Constructors}

\begin{cfuncdesc}{PyObject*}{PyFile_FromString}{char *file_name, char *mode}
On success, returns a new file object that is opened on the
file given by \var{file_name}, with a file mode given by \var{mode},
where \var{mode} has the same semantics as the standard \C{} routine
\cfunction{fopen()}.  On failure, return \code{-1}.
\end{cfuncdesc}

\begin{cfuncdesc}{PyObject*}{PyFile_FromFile}{FILE *fp, char *file_name, char *mode, int close_on_del}
Return a new file object for an already opened standard \C{} file
pointer, \var{fp}.  A file name, \var{file_name}, and open mode,
\var{mode}, must be provided as well as a flag, \var{close_on_del},
that indicates whether the file is to be closed when the file object
is destroyed.  On failure, return \code{-1}.
\end{cfuncdesc}

\begin{cfuncdesc}{PyObject*}{PyFloat_FromDouble}{double v}
Returns a new float object with the value \var{v} on success, and
\NULL{} on failure.
\end{cfuncdesc}

\begin{cfuncdesc}{PyObject*}{PyInt_FromLong}{long v}
Returns a new int object with the value \var{v} on success, and
\NULL{} on failure.
\end{cfuncdesc}

\begin{cfuncdesc}{PyObject*}{PyList_New}{int len}
Returns a new list of length \var{len} on success, and \NULL{} on
failure.
\end{cfuncdesc}

\begin{cfuncdesc}{PyObject*}{PyLong_FromLong}{long v}
Returns a new long object with the value \var{v} on success, and
\NULL{} on failure.
\end{cfuncdesc}

\begin{cfuncdesc}{PyObject*}{PyLong_FromDouble}{double v}
Returns a new long object with the value \var{v} on success, and
\NULL{} on failure.
\end{cfuncdesc}

\begin{cfuncdesc}{PyObject*}{PyDict_New}{}
Returns a new empty dictionary on success, and \NULL{} on
failure.
\end{cfuncdesc}

\begin{cfuncdesc}{PyObject*}{PyString_FromString}{char *v}
Returns a new string object with the value \var{v} on success, and
\NULL{} on failure.
\end{cfuncdesc}

\begin{cfuncdesc}{PyObject*}{PyString_FromStringAndSize}{char *v, int len}
Returns a new string object with the value \var{v} and length
\var{len} on success, and \NULL{} on failure.  If \var{v} is \NULL{},
the contents of the string are uninitialized.
\end{cfuncdesc}

\begin{cfuncdesc}{PyObject*}{PyTuple_New}{int len}
Returns a new tuple of length \var{len} on success, and \NULL{} on
failure.
\end{cfuncdesc}


\chapter{Concrete Objects Layer}
\label{concrete}

The functions in this chapter are specific to certain Python object
types.  Passing them an object of the wrong type is not a good idea;
if you receive an object from a Python program and you are not sure
that it has the right type, you must perform a type check first;
e.g. to check that an object is a dictionary, use
\cfunction{PyDict_Check()}.  The chapter is structured like the
``family tree'' of Python object types.


\section{Fundamental Objects}
\label{fundamental}

This section describes Python type objects and the singleton object 
\code{None}.


\subsection{Type Objects}
\label{typeObjects}

\begin{ctypedesc}{PyTypeObject}

\end{ctypedesc}

\begin{cvardesc}{PyObject *}{PyType_Type}

\end{cvardesc}


\subsection{The None Object}
\label{noneObject}

\begin{cvardesc}{PyObject *}{Py_None}
The Python \code{None} object, denoting lack of value.  This object has
no methods.
\end{cvardesc}


\section{Sequence Objects}
\label{sequenceObjects}

Generic operations on sequence objects were discussed in the previous 
chapter; this section deals with the specific kinds of sequence 
objects that are intrinsic to the Python language.


\subsection{String Objects}
\label{stringObjects}

\begin{ctypedesc}{PyStringObject}
This subtype of \ctype{PyObject} represents a Python string object.
\end{ctypedesc}

\begin{cvardesc}{PyTypeObject}{PyString_Type}
This instance of \ctype{PyTypeObject} represents the Python string type.
\end{cvardesc}

\begin{cfuncdesc}{int}{PyString_Check}{PyObject *o}
Returns true if the object \var{o} is a string object.
\end{cfuncdesc}

\begin{cfuncdesc}{PyObject*}{PyString_FromStringAndSize}{const char *v,
                                                          int len}
Returns a new string object with the value \var{v} and length
\var{len} on success, and \NULL{} on failure.  If \var{v} is \NULL{},
the contents of the string are uninitialized.
\end{cfuncdesc}

\begin{cfuncdesc}{PyObject*}{PyString_FromString}{const char *v}
Returns a new string object with the value \var{v} on success, and
\NULL{} on failure.
\end{cfuncdesc}

\begin{cfuncdesc}{int}{PyString_Size}{PyObject *string}
Returns the length of the string in string object \var{string}.
\end{cfuncdesc}

\begin{cfuncdesc}{char*}{PyString_AsString}{PyObject *string}
Resturns a \NULL{} terminated representation of the contents of \var{string}.
\end{cfuncdesc}

\begin{cfuncdesc}{void}{PyString_Concat}{PyObject **string,
                                         PyObject *newpart}
Creates a new string object in \var{*string} containing the contents
of \var{newpart} appended to \var{string}.
\end{cfuncdesc}

\begin{cfuncdesc}{void}{PyString_ConcatAndDel}{PyObject **string,
                                               PyObject *newpart}
Creates a new string object in \var{*string} containing the contents
of \var{newpart} appended to \var{string}.  This version decrements
the reference count of \var{newpart}.
\end{cfuncdesc}

\begin{cfuncdesc}{int}{_PyString_Resize}{PyObject **string, int newsize}
A way to resize a string object even though it is ``immutable''.  
Only use this to build up a brand new string object; don't use this if
the string may already be known in other parts of the code.
\end{cfuncdesc}

\begin{cfuncdesc}{PyObject*}{PyString_Format}{PyObject *format,
                                              PyObject *args}
Returns a new string object from \var{format} and \var{args}.  Analogous
to \code{\var{format} \% \var{args}}.  The \var{args} argument must be
a tuple.
\end{cfuncdesc}

\begin{cfuncdesc}{void}{PyString_InternInPlace}{PyObject **string}
Intern the argument \var{*string} in place.  The argument must be the
address of a pointer variable pointing to a Python string object.
If there is an existing interned string that is the same as
\var{*string}, it sets \var{*string} to it (decrementing the reference 
count of the old string object and incrementing the reference count of
the interned string object), otherwise it leaves \var{*string} alone
and interns it (incrementing its reference count).  (Clarification:
even though there is a lot of talk about reference counts, think of
this function as reference-count-neutral; you own the object after
the call if and only if you owned it before the call.)
\end{cfuncdesc}

\begin{cfuncdesc}{PyObject*}{PyString_InternFromString}{const char *v}
A combination of \cfunction{PyString_FromString()} and
\cfunction{PyString_InternInPlace()}, returning either a new string object
that has been interned, or a new (``owned'') reference to an earlier
interned string object with the same value.
\end{cfuncdesc}

\begin{cfuncdesc}{char*}{PyString_AS_STRING}{PyObject *string}
Macro form of \cfunction{PyString_AsString()} but without error checking.
\end{cfuncdesc}

\begin{cfuncdesc}{int}{PyString_GET_SIZE}{PyObject *string}
Macro form of \cfunction{PyString_GetSize()} but without error checking.
\end{cfuncdesc}



\subsection{Tuple Objects}
\label{tupleObjects}

\begin{ctypedesc}{PyTupleObject}
This subtype of \ctype{PyObject} represents a Python tuple object.
\end{ctypedesc}

\begin{cvardesc}{PyTypeObject}{PyTuple_Type}
This instance of \ctype{PyTypeObject} represents the Python tuple type.
\end{cvardesc}

\begin{cfuncdesc}{int}{PyTuple_Check}{PyObject *p}
Return true if the argument is a tuple object.
\end{cfuncdesc}

\begin{cfuncdesc}{PyObject*}{PyTuple_New}{int s}
Return a new tuple object of size \var{s}.
\end{cfuncdesc}

\begin{cfuncdesc}{int}{PyTuple_Size}{PyTupleObject *p}
Takes a pointer to a tuple object, and returns the size
of that tuple.
\end{cfuncdesc}

\begin{cfuncdesc}{PyObject*}{PyTuple_GetItem}{PyTupleObject *p, int pos}
Returns the object at position \var{pos} in the tuple pointed
to by \var{p}.  If \var{pos} is out of bounds, returns \NULL{} and
sets an \exception{IndexError} exception.  \strong{Note:} this
function returns a ``borrowed'' reference.
\end{cfuncdesc}

\begin{cfuncdesc}{PyObject*}{PyTuple_GET_ITEM}{PyTupleObject *p, int pos}
Does the same, but does no checking of its arguments.
\end{cfuncdesc}

\begin{cfuncdesc}{PyObject*}{PyTuple_GetSlice}{PyTupleObject *p,
            int low,
            int high}
Takes a slice of the tuple pointed to by \var{p} from
\var{low} to \var{high} and returns it as a new tuple.
\end{cfuncdesc}

\begin{cfuncdesc}{int}{PyTuple_SetItem}{PyTupleObject *p,
            int pos,
            PyObject *o}
Inserts a reference to object \var{o} at position \var{pos} of
the tuple pointed to by \var{p}. It returns \code{0} on success.
\end{cfuncdesc}

\begin{cfuncdesc}{void}{PyTuple_SET_ITEM}{PyTupleObject *p,
            int pos,
            PyObject *o}

Does the same, but does no error checking, and
should \emph{only} be used to fill in brand new tuples.
\end{cfuncdesc}

\begin{cfuncdesc}{int}{_PyTuple_Resize}{PyTupleObject *p,
            int new,
            int last_is_sticky}
Can be used to resize a tuple. Because tuples are
\emph{supposed} to be immutable, this should only be used if there is only
one module referencing the object. Do \emph{not} use this if the tuple may
already be known to some other part of the code. \var{last_is_sticky} is
a flag --- if set, the tuple will grow or shrink at the front, otherwise
it will grow or shrink at the end. Think of this as destroying the old
tuple and creating a new one, only more efficiently.
\end{cfuncdesc}


\subsection{List Objects}
\label{listObjects}

\begin{ctypedesc}{PyListObject}
This subtype of \ctype{PyObject} represents a Python list object.
\end{ctypedesc}

\begin{cvardesc}{PyTypeObject}{PyList_Type}
This instance of \ctype{PyTypeObject} represents the Python list type.
\end{cvardesc}

\begin{cfuncdesc}{int}{PyList_Check}{PyObject *p}
Returns true if its argument is a \ctype{PyListObject}.
\end{cfuncdesc}

\begin{cfuncdesc}{PyObject*}{PyList_New}{int size}
Returns a new list of length \var{len} on success, and \NULL{} on
failure.
\end{cfuncdesc}

\begin{cfuncdesc}{int}{PyList_Size}{PyObject *list}
Returns the length of the list object in \var{list}.
\end{cfuncdesc}

\begin{cfuncdesc}{PyObject*}{PyList_GetItem}{PyObject *list, int index}
Returns the object at position \var{pos} in the list pointed
to by \var{p}.  If \var{pos} is out of bounds, returns \NULL{} and
sets an \exception{IndexError} exception.  \strong{Note:} this
function returns a ``borrowed'' reference.
\end{cfuncdesc}

\begin{cfuncdesc}{int}{PyList_SetItem}{PyObject *list, int index,
                                       PyObject *item}
Sets the item at index \var{index} in list to \var{item}.
\end{cfuncdesc}

\begin{cfuncdesc}{int}{PyList_Insert}{PyObject *list, int index,
                                      PyObject *item}
Inserts the item \var{item} into list \var{list} in front of index
\var{index}.  Returns 0 if successful; returns -1 and sets an
exception if unsuccessful.  Analogous to \code{list.insert(index, item)}.
\end{cfuncdesc}

\begin{cfuncdesc}{int}{PyList_Append}{PyObject *list, PyObject *item}
Appends the object \var{item} at the end of list \var{list}.  Returns
0 if successful; returns -1 and sets an exception if unsuccessful.
Analogous to \code{list.append(item)}.
\end{cfuncdesc}

\begin{cfuncdesc}{PyObject*}{PyList_GetSlice}{PyObject *list,
                                              int low, int high}
Returns a list of the objects in \var{list} containing the objects 
\emph{between} \var{low} and \var{high}.  Returns NULL and sets an
exception if unsuccessful.
Analogous to \code{list[low:high]}.
\end{cfuncdesc}

\begin{cfuncdesc}{int}{PyList_SetSlice}{PyObject *list,
                                        int low, int high,
                                        PyObject *itemlist}
Sets the slice of \var{list} between \var{low} and \var{high} to the contents
of \var{itemlist}.  Analogous to \code{list[low:high]=itemlist}.  Returns 0
on success, -1 on failure.
\end{cfuncdesc}

\begin{cfuncdesc}{int}{PyList_Sort}{PyObject *list}
Sorts the items of \var{list} in place.  Returns 0 on success, -1 on failure.
\end{cfuncdesc}

\begin{cfuncdesc}{int}{PyList_Reverse}{PyObject *list}
Reverses the items of \var{list} in place.  Returns 0 on success, -1 on failure.
\end{cfuncdesc}

\begin{cfuncdesc}{PyObject*}{PyList_AsTuple}{PyObject *list}
Returns a new tuple object containing the contents of \var{list}.
\end{cfuncdesc}

\begin{cfuncdesc}{PyObject*}{PyList_GET_ITEM}{PyObject *list, int i}
Macro form of \cfunction{PyList_GetItem()} without error checking.
\end{cfuncdesc}

\begin{cfuncdesc}{PyObject*}{PyList_SET_ITEM}{PyObject *list, int i,
                                              PyObject *o}
Macro form of \cfunction{PyList_SetItem()} without error checking.
\end{cfuncdesc}

\begin{cfuncdesc}{int}{PyList_GET_SIZE}{PyObject *list}
Macro form of \cfunction{PyList_GetSize()} without error checking.
\end{cfuncdesc}


\section{Mapping Objects}
\label{mapObjects}

\subsection{Dictionary Objects}
\label{dictObjects}

\begin{ctypedesc}{PyDictObject}
This subtype of \ctype{PyObject} represents a Python dictionary object.
\end{ctypedesc}

\begin{cvardesc}{PyTypeObject}{PyDict_Type}
This instance of \ctype{PyTypeObject} represents the Python dictionary type.
\end{cvardesc}

\begin{cfuncdesc}{int}{PyDict_Check}{PyObject *p}
Returns true if its argument is a \ctype{PyDictObject}.
\end{cfuncdesc}

\begin{cfuncdesc}{PyObject*}{PyDict_New}{}
Returns a new empty dictionary.
\end{cfuncdesc}

\begin{cfuncdesc}{void}{PyDict_Clear}{PyDictObject *p}
Empties an existing dictionary of all key/value pairs.
\end{cfuncdesc}

\begin{cfuncdesc}{int}{PyDict_SetItem}{PyDictObject *p,
            PyObject *key,
            PyObject *val}
Inserts \var{value} into the dictionary with a key of \var{key}.  Both
\var{key} and \var{value} should be PyObjects, and \var{key} should be
hashable.
\end{cfuncdesc}

\begin{cfuncdesc}{int}{PyDict_SetItemString}{PyDictObject *p,
            char *key,
            PyObject *val}
Inserts \var{value} into the dictionary using \var{key}
as a key. \var{key} should be a \ctype{char *}.  The key object is
created using \code{PyString_FromString(\var{key})}.
\end{cfuncdesc}

\begin{cfuncdesc}{int}{PyDict_DelItem}{PyDictObject *p, PyObject *key}
Removes the entry in dictionary \var{p} with key \var{key}.
\var{key} is a PyObject.
\end{cfuncdesc}

\begin{cfuncdesc}{int}{PyDict_DelItemString}{PyDictObject *p, char *key}
Removes the entry in dictionary \var{p} which has a key
specified by the \ctype{char *}\var{key}.
\end{cfuncdesc}

\begin{cfuncdesc}{PyObject*}{PyDict_GetItem}{PyDictObject *p, PyObject *key}
Returns the object from dictionary \var{p} which has a key
\var{key}.  Returns \NULL{} if the key \var{key} is not present, but
without (!) setting an exception.  \strong{Note:}  this function
returns a ``borrowed'' reference.
\end{cfuncdesc}

\begin{cfuncdesc}{PyObject*}{PyDict_GetItemString}{PyDictObject *p, char *key}
This is the same as \cfunction{PyDict_GetItem()}, but \var{key} is
specified as a \ctype{char *}, rather than a \ctype{PyObject *}.
\end{cfuncdesc}

\begin{cfuncdesc}{PyObject*}{PyDict_Items}{PyDictObject *p}
Returns a \ctype{PyListObject} containing all the items 
from the dictionary, as in the dictinoary method \method{items()} (see
the \emph{Python Library Reference}).
\end{cfuncdesc}

\begin{cfuncdesc}{PyObject*}{PyDict_Keys}{PyDictObject *p}
Returns a \ctype{PyListObject} containing all the keys 
from the dictionary, as in the dictionary method \method{keys()} (see the
\emph{Python Library Reference}).
\end{cfuncdesc}

\begin{cfuncdesc}{PyObject*}{PyDict_Values}{PyDictObject *p}
Returns a \ctype{PyListObject} containing all the values 
from the dictionary \var{p}, as in the dictionary method
\method{values()} (see the \emph{Python Library Reference}).
\end{cfuncdesc}

\begin{cfuncdesc}{int}{PyDict_Size}{PyDictObject *p}
Returns the number of items in the dictionary.
\end{cfuncdesc}

\begin{cfuncdesc}{int}{PyDict_Next}{PyDictObject *p,
            int ppos,
            PyObject **pkey,
            PyObject **pvalue}

\end{cfuncdesc}


\section{Numeric Objects}
\label{numericObjects}

\subsection{Plain Integer Objects}
\label{intObjects}

\begin{ctypedesc}{PyIntObject}
This subtype of \ctype{PyObject} represents a Python integer object.
\end{ctypedesc}

\begin{cvardesc}{PyTypeObject}{PyInt_Type}
This instance of \ctype{PyTypeObject} represents the Python plain 
integer type.
\end{cvardesc}

\begin{cfuncdesc}{int}{PyInt_Check}{PyObject *}

\end{cfuncdesc}

\begin{cfuncdesc}{PyObject*}{PyInt_FromLong}{long ival}
Creates a new integer object with a value of \var{ival}.

The current implementation keeps an array of integer objects for all
integers between \code{-1} and \code{100}, when you create an int in
that range you actually just get back a reference to the existing
object. So it should be possible to change the value of \code{1}. I
suspect the behaviour of Python in this case is undefined. :-)
\end{cfuncdesc}

\begin{cfuncdesc}{long}{PyInt_AS_LONG}{PyIntObject *io}
Returns the value of the object \var{io}.  No error checking is
performed.
\end{cfuncdesc}

\begin{cfuncdesc}{long}{PyInt_AsLong}{PyObject *io}
Will first attempt to cast the object to a \ctype{PyIntObject}, if
it is not already one, and then return its value.
\end{cfuncdesc}

\begin{cfuncdesc}{long}{PyInt_GetMax}{}
Returns the systems idea of the largest integer it can handle
(\constant{LONG_MAX}, as defined in the system header files).
\end{cfuncdesc}


\subsection{Long Integer Objects}
\label{longObjects}

\begin{ctypedesc}{PyLongObject}
This subtype of \ctype{PyObject} represents a Python long integer
object.
\end{ctypedesc}

\begin{cvardesc}{PyTypeObject}{PyLong_Type}
This instance of \ctype{PyTypeObject} represents the Python long
integer type.
\end{cvardesc}

\begin{cfuncdesc}{int}{PyLong_Check}{PyObject *p}
Returns true if its argument is a \ctype{PyLongObject}.
\end{cfuncdesc}

\begin{cfuncdesc}{PyObject*}{PyLong_FromLong}{long v}
Returns a new \ctype{PyLongObject} object from \var{v}.
\end{cfuncdesc}

\begin{cfuncdesc}{PyObject*}{PyLong_FromUnsignedLong}{unsigned long v}
Returns a new \ctype{PyLongObject} object from an unsigned \C{} long.
\end{cfuncdesc}

\begin{cfuncdesc}{PyObject*}{PyLong_FromDouble}{double v}
Returns a new \ctype{PyLongObject} object from the integer part of \var{v}.
\end{cfuncdesc}

\begin{cfuncdesc}{long}{PyLong_AsLong}{PyObject *pylong}
Returns a \C{} \ctype{long} representation of the contents of \var{pylong}.  
WHAT HAPPENS IF \var{pylong} is greater than \constant{LONG_MAX}?
\end{cfuncdesc}

\begin{cfuncdesc}{unsigned long}{PyLong_AsUnsignedLong}{PyObject *pylong}
Returns a \C{} \ctype{unsigned long} representation of the contents of 
\var{pylong}.  WHAT HAPPENS IF \var{pylong} is greater than
\constant{ULONG_MAX}?
\end{cfuncdesc}

\begin{cfuncdesc}{double}{PyLong_AsDouble}{PyObject *pylong}
Returns a \C{} \ctype{double} representation of the contents of \var{pylong}.
\end{cfuncdesc}

\begin{cfuncdesc}{PyObject*}{PyLong_FromString}{char *str, char **pend,
                                                int base}
\end{cfuncdesc}


\subsection{Floating Point Objects}
\label{floatObjects}

\begin{ctypedesc}{PyFloatObject}
This subtype of \ctype{PyObject} represents a Python floating point
object.
\end{ctypedesc}

\begin{cvardesc}{PyTypeObject}{PyFloat_Type}
This instance of \ctype{PyTypeObject} represents the Python floating
point type.
\end{cvardesc}

\begin{cfuncdesc}{int}{PyFloat_Check}{PyObject *p}
Returns true if its argument is a \ctype{PyFloatObject}.
\end{cfuncdesc}

\begin{cfuncdesc}{PyObject*}{PyFloat_FromDouble}{double v}
Creates a \ctype{PyFloatObject} object from \var{v}.
\end{cfuncdesc}

\begin{cfuncdesc}{double}{PyFloat_AsDouble}{PyObject *pyfloat}
Returns a \C{} \ctype{double} representation of the contents of \var{pyfloat}.
\end{cfuncdesc}

\begin{cfuncdesc}{double}{PyFloat_AS_DOUBLE}{PyObject *pyfloat}
Returns a \C{} \ctype{double} representation of the contents of
\var{pyfloat}, but without error checking.
\end{cfuncdesc}


\subsection{Complex Number Objects}
\label{complexObjects}

\begin{ctypedesc}{Py_complex}
The \C{} structure which corresponds to the value portion of a Python
complex number object.  Most of the functions for dealing with complex
number objects use structures of this type as input or output values,
as appropriate.  It is defined as:

\begin{verbatim}
typedef struct {
   double real;
   double imag;
} Py_complex;
\end{verbatim}
\end{ctypedesc}

\begin{ctypedesc}{PyComplexObject}
This subtype of \ctype{PyObject} represents a Python complex number object.
\end{ctypedesc}

\begin{cvardesc}{PyTypeObject}{PyComplex_Type}
This instance of \ctype{PyTypeObject} represents the Python complex 
number type.
\end{cvardesc}

\begin{cfuncdesc}{int}{PyComplex_Check}{PyObject *p}
Returns true if its argument is a \ctype{PyComplexObject}.
\end{cfuncdesc}

\begin{cfuncdesc}{Py_complex}{_Py_c_sum}{Py_complex left, Py_complex right}
\end{cfuncdesc}

\begin{cfuncdesc}{Py_complex}{_Py_c_diff}{Py_complex left, Py_complex right}
\end{cfuncdesc}

\begin{cfuncdesc}{Py_complex}{_Py_c_neg}{Py_complex complex}
\end{cfuncdesc}

\begin{cfuncdesc}{Py_complex}{_Py_c_prod}{Py_complex left, Py_complex right}
\end{cfuncdesc}

\begin{cfuncdesc}{Py_complex}{_Py_c_quot}{Py_complex dividend,
                                          Py_complex divisor}
\end{cfuncdesc}

\begin{cfuncdesc}{Py_complex}{_Py_c_pow}{Py_complex num, Py_complex exp}
\end{cfuncdesc}

\begin{cfuncdesc}{PyObject*}{PyComplex_FromCComplex}{Py_complex v}
\end{cfuncdesc}

\begin{cfuncdesc}{PyObject*}{PyComplex_FromDoubles}{double real, double imag}
Returns a new \ctype{PyComplexObject} object from \var{real} and \var{imag}.
\end{cfuncdesc}

\begin{cfuncdesc}{double}{PyComplex_RealAsDouble}{PyObject *op}
Returns the real part of \var{op} as a \C{} \ctype{double}.
\end{cfuncdesc}

\begin{cfuncdesc}{double}{PyComplex_ImagAsDouble}{PyObject *op}
Returns the imaginary part of \var{op} as a \C{} \ctype{double}.
\end{cfuncdesc}

\begin{cfuncdesc}{Py_complex}{PyComplex_AsCComplex}{PyObject *op}
\end{cfuncdesc}



\section{Other Objects}
\label{otherObjects}

\subsection{File Objects}
\label{fileObjects}

\begin{ctypedesc}{PyFileObject}
This subtype of \ctype{PyObject} represents a Python file object.
\end{ctypedesc}

\begin{cvardesc}{PyTypeObject}{PyFile_Type}
This instance of \ctype{PyTypeObject} represents the Python file type.
\end{cvardesc}

\begin{cfuncdesc}{int}{PyFile_Check}{PyObject *p}
Returns true if its argument is a \ctype{PyFileObject}.
\end{cfuncdesc}

\begin{cfuncdesc}{PyObject*}{PyFile_FromString}{char *name, char *mode}
Creates a new \ctype{PyFileObject} pointing to the file
specified in \var{name} with the mode specified in \var{mode}.
\end{cfuncdesc}

\begin{cfuncdesc}{PyObject*}{PyFile_FromFile}{FILE *fp,
              char *name, char *mode, int (*close)}
Creates a new \ctype{PyFileObject} from the already-open \var{fp}.
The function \var{close} will be called when the file should be
closed.
\end{cfuncdesc}

\begin{cfuncdesc}{FILE *}{PyFile_AsFile}{PyFileObject *p}
Returns the file object associated with \var{p} as a \ctype{FILE *}.
\end{cfuncdesc}

\begin{cfuncdesc}{PyObject*}{PyFile_GetLine}{PyObject *p, int n}
undocumented as yet
\end{cfuncdesc}

\begin{cfuncdesc}{PyObject*}{PyFile_Name}{PyObject *p}
Returns the name of the file specified by \var{p} as a 
\ctype{PyStringObject}.
\end{cfuncdesc}

\begin{cfuncdesc}{void}{PyFile_SetBufSize}{PyFileObject *p, int n}
Available on systems with \cfunction{setvbuf()} only.  This should
only be called immediately after file object creation.
\end{cfuncdesc}

\begin{cfuncdesc}{int}{PyFile_SoftSpace}{PyFileObject *p, int newflag}
Sets the \member{softspace} attribute of \var{p} to \var{newflag}.
Returns the previous value.  This function clears any errors, and will
return \code{0} as the previous value if the attribute either does not
exist or if there were errors in retrieving it.  There is no way to
detect errors from this function, but doing so should not be needed.
\end{cfuncdesc}

\begin{cfuncdesc}{int}{PyFile_WriteObject}{PyObject *obj, PyFileObject *p,
                                           int flags}
Writes object \var{obj} to file object \var{p}.
\end{cfuncdesc}

\begin{cfuncdesc}{int}{PyFile_WriteString}{char *s, PyFileObject *p,
                                           int flags}
Writes string \var{s} to file object \var{p}.
\end{cfuncdesc}


\subsection{CObjects}
\label{cObjects}

\begin{ctypedesc}{PyCObject}
This subtype of \ctype{PyObject} represents an opaque value, useful for
\C{} extension modules who need to pass an opaque value (as a
\ctype{void *} pointer) through Python code to other \C{} code.  It is
often used to make a C function pointer defined in one module
available to other modules, so the regular import mechanism can be
used to access C APIs defined in dynamically loaded modules.
\end{ctypedesc}

\begin{cfuncdesc}{PyObject *}{PyCObject_FromVoidPtr}{void* cobj, 
	void (*destr)(void *)}
Creates a \ctype{PyCObject} from the \code{void *} \var{cobj}.  The
\var{destr} function will be called when the object is reclaimed.
\end{cfuncdesc}

\begin{cfuncdesc}{PyObject *}{PyCObject_FromVoidPtrAndDesc}{void* cobj,
	void* desc, void (*destr)(void *, void *) }
Creates a \ctype{PyCObject} from the \ctype{void *}\var{cobj}.  The
\var{destr} function will be called when the object is reclaimed.  The
\var{desc} argument can be used to pass extra callback data for the
destructor function.
\end{cfuncdesc}

\begin{cfuncdesc}{void *}{PyCObject_AsVoidPtr}{PyObject* self}
Returns the object \ctype{void *} that the \ctype{PyCObject} \var{self}
was created with.
\end{cfuncdesc}

\begin{cfuncdesc}{void *}{PyCObject_GetDesc}{PyObject* self}
Returns the description \ctype{void *} that the \ctype{PyCObject}
\var{self} was created with.
\end{cfuncdesc}

\chapter{Initialization, Finalization, and Threads}
\label{initialization}

\begin{cfuncdesc}{void}{Py_Initialize}{}
Initialize the Python interpreter.  In an application embedding 
Python, this should be called before using any other Python/C API 
functions; with the exception of \cfunction{Py_SetProgramName()},
\cfunction{PyEval_InitThreads()}, \cfunction{PyEval_ReleaseLock()},
and \cfunction{PyEval_AcquireLock()}.  This initializes the table of
loaded modules (\code{sys.modules}), and creates the fundamental
modules \module{__builtin__}\refbimodindex{__builtin__},
\module{__main__}\refbimodindex{__main__} and
\module{sys}\refbimodindex{sys}.  It also initializes the module
search path (\code{sys.path}).%
\indexiii{module}{search}{path}
It does not set \code{sys.argv}; use \cfunction{PySys_SetArgv()} for
that.  This is a no-op when called for a second time (without calling
\cfunction{Py_Finalize()} first).  There is no return value; it is a
fatal error if the initialization fails.
\end{cfuncdesc}

\begin{cfuncdesc}{int}{Py_IsInitialized}{}
Return true (nonzero) when the Python interpreter has been
initialized, false (zero) if not.  After \cfunction{Py_Finalize()} is
called, this returns false until \cfunction{Py_Initialize()} is called
again.
\end{cfuncdesc}

\begin{cfuncdesc}{void}{Py_Finalize}{}
Undo all initializations made by \cfunction{Py_Initialize()} and
subsequent use of Python/C API functions, and destroy all
sub-interpreters (see \cfunction{Py_NewInterpreter()} below) that were
created and not yet destroyed since the last call to
\cfunction{Py_Initialize()}.  Ideally, this frees all memory allocated
by the Python interpreter.  This is a no-op when called for a second
time (without calling \cfunction{Py_Initialize()} again first).  There
is no return value; errors during finalization are ignored.

This function is provided for a number of reasons.  An embedding 
application might want to restart Python without having to restart the 
application itself.  An application that has loaded the Python 
interpreter from a dynamically loadable library (or DLL) might want to 
free all memory allocated by Python before unloading the DLL. During a 
hunt for memory leaks in an application a developer might want to free 
all memory allocated by Python before exiting from the application.

\strong{Bugs and caveats:} The destruction of modules and objects in 
modules is done in random order; this may cause destructors 
(\method{__del__()} methods) to fail when they depend on other objects 
(even functions) or modules.  Dynamically loaded extension modules 
loaded by Python are not unloaded.  Small amounts of memory allocated 
by the Python interpreter may not be freed (if you find a leak, please 
report it).  Memory tied up in circular references between objects is 
not freed.  Some memory allocated by extension modules may not be 
freed.  Some extension may not work properly if their initialization 
routine is called more than once; this can happen if an applcation 
calls \cfunction{Py_Initialize()} and \cfunction{Py_Finalize()} more
than once.
\end{cfuncdesc}

\begin{cfuncdesc}{PyThreadState*}{Py_NewInterpreter}{}
Create a new sub-interpreter.  This is an (almost) totally separate
environment for the execution of Python code.  In particular, the new
interpreter has separate, independent versions of all imported
modules, including the fundamental modules
\module{__builtin__}\refbimodindex{__builtin__},
\module{__main__}\refbimodindex{__main__} and
\module{sys}\refbimodindex{sys}.  The table of loaded modules
(\code{sys.modules}) and the module search path (\code{sys.path}) are
also separate.  The new environment has no \code{sys.argv} variable.
It has new standard I/O stream file objects \code{sys.stdin},
\code{sys.stdout} and \code{sys.stderr} (however these refer to the
same underlying \ctype{FILE} structures in the \C{} library).

The return value points to the first thread state created in the new 
sub-interpreter.  This thread state is made the current thread state.  
Note that no actual thread is created; see the discussion of thread 
states below.  If creation of the new interpreter is unsuccessful, 
\NULL{} is returned; no exception is set since the exception state 
is stored in the current thread state and there may not be a current 
thread state.  (Like all other Python/C API functions, the global 
interpreter lock must be held before calling this function and is 
still held when it returns; however, unlike most other Python/C API 
functions, there needn't be a current thread state on entry.)

Extension modules are shared between (sub-)interpreters as follows: 
the first time a particular extension is imported, it is initialized 
normally, and a (shallow) copy of its module's dictionary is 
squirreled away.  When the same extension is imported by another 
(sub-)interpreter, a new module is initialized and filled with the 
contents of this copy; the extension's \code{init} function is not
called.  Note that this is different from what happens when an
extension is imported after the interpreter has been completely
re-initialized by calling \cfunction{Py_Finalize()} and
\cfunction{Py_Initialize()}; in that case, the extension's \code{init}
function \emph{is} called again.

\strong{Bugs and caveats:} Because sub-interpreters (and the main 
interpreter) are part of the same process, the insulation between them 
isn't perfect --- for example, using low-level file operations like 
\function{os.close()} they can (accidentally or maliciously) affect each 
other's open files.  Because of the way extensions are shared between 
(sub-)interpreters, some extensions may not work properly; this is 
especially likely when the extension makes use of (static) global 
variables, or when the extension manipulates its module's dictionary 
after its initialization.  It is possible to insert objects created in 
one sub-interpreter into a namespace of another sub-interpreter; this 
should be done with great care to avoid sharing user-defined 
functions, methods, instances or classes between sub-interpreters, 
since import operations executed by such objects may affect the 
wrong (sub-)interpreter's dictionary of loaded modules.  (XXX This is 
a hard-to-fix bug that will be addressed in a future release.)
\end{cfuncdesc}

\begin{cfuncdesc}{void}{Py_EndInterpreter}{PyThreadState *tstate}
Destroy the (sub-)interpreter represented by the given thread state.  
The given thread state must be the current thread state.  See the 
discussion of thread states below.  When the call returns, the current 
thread state is \NULL{}.  All thread states associated with this 
interpreted are destroyed.  (The global interpreter lock must be held 
before calling this function and is still held when it returns.)  
\cfunction{Py_Finalize()} will destroy all sub-interpreters that haven't 
been explicitly destroyed at that point.
\end{cfuncdesc}

\begin{cfuncdesc}{void}{Py_SetProgramName}{char *name}
This function should be called before \cfunction{Py_Initialize()} is called 
for the first time, if it is called at all.  It tells the interpreter 
the value of the \code{argv[0]} argument to the \cfunction{main()} function 
of the program.  This is used by \cfunction{Py_GetPath()} and some other 
functions below to find the Python run-time libraries relative to the 
interpreter executable.  The default value is \code{"python"}.  The 
argument should point to a zero-terminated character string in static 
storage whose contents will not change for the duration of the 
program's execution.  No code in the Python interpreter will change 
the contents of this storage.
\end{cfuncdesc}

\begin{cfuncdesc}{char*}{Py_GetProgramName}{}
Return the program name set with \cfunction{Py_SetProgramName()}, or the 
default.  The returned string points into static storage; the caller 
should not modify its value.
\end{cfuncdesc}

\begin{cfuncdesc}{char*}{Py_GetPrefix}{}
Return the \emph{prefix} for installed platform-independent files.  This 
is derived through a number of complicated rules from the program name 
set with \cfunction{Py_SetProgramName()} and some environment variables; 
for example, if the program name is \code{"/usr/local/bin/python"}, 
the prefix is \code{"/usr/local"}.  The returned string points into 
static storage; the caller should not modify its value.  This 
corresponds to the \makevar{prefix} variable in the top-level 
\file{Makefile} and the \code{-}\code{-prefix} argument to the 
\program{configure} script at build time.  The value is available to 
Python code as \code{sys.prefix}.  It is only useful on \UNIX{}.  See 
also the next function.
\end{cfuncdesc}

\begin{cfuncdesc}{char*}{Py_GetExecPrefix}{}
Return the \emph{exec-prefix} for installed platform-\emph{de}pendent 
files.  This is derived through a number of complicated rules from the 
program name set with \cfunction{Py_SetProgramName()} and some environment 
variables; for example, if the program name is 
\code{"/usr/local/bin/python"}, the exec-prefix is 
\code{"/usr/local"}.  The returned string points into static storage; 
the caller should not modify its value.  This corresponds to the 
\makevar{exec_prefix} variable in the top-level \file{Makefile} and the 
\code{-}\code{-exec_prefix} argument to the \program{configure} script
at build  time.  The value is available to Python code as 
\code{sys.exec_prefix}.  It is only useful on \UNIX{}.

Background: The exec-prefix differs from the prefix when platform 
dependent files (such as executables and shared libraries) are 
installed in a different directory tree.  In a typical installation, 
platform dependent files may be installed in the 
\code{"/usr/local/plat"} subtree while platform independent may be 
installed in \code{"/usr/local"}.

Generally speaking, a platform is a combination of hardware and 
software families, e.g.  Sparc machines running the Solaris 2.x 
operating system are considered the same platform, but Intel machines 
running Solaris 2.x are another platform, and Intel machines running 
Linux are yet another platform.  Different major revisions of the same 
operating system generally also form different platforms.  Non-\UNIX{} 
operating systems are a different story; the installation strategies 
on those systems are so different that the prefix and exec-prefix are 
meaningless, and set to the empty string.  Note that compiled Python 
bytecode files are platform independent (but not independent from the 
Python version by which they were compiled!).

System administrators will know how to configure the \program{mount} or 
\program{automount} programs to share \code{"/usr/local"} between platforms 
while having \code{"/usr/local/plat"} be a different filesystem for each 
platform.
\end{cfuncdesc}

\begin{cfuncdesc}{char*}{Py_GetProgramFullPath}{}
Return the full program name of the Python executable; this is 
computed as a side-effect of deriving the default module search path 
from the program name (set by \cfunction{Py_SetProgramName()} above).  The 
returned string points into static storage; the caller should not 
modify its value.  The value is available to Python code as 
\code{sys.executable}.
\end{cfuncdesc}

\begin{cfuncdesc}{char*}{Py_GetPath}{}
\indexiii{module}{search}{path}
Return the default module search path; this is computed from the 
program name (set by \cfunction{Py_SetProgramName()} above) and some 
environment variables.  The returned string consists of a series of 
directory names separated by a platform dependent delimiter character.  
The delimiter character is \character{:} on \UNIX{}, \character{;} on
DOS/Windows, and \character{\\n} (the \ASCII{} newline character) on
Macintosh.  The returned string points into static storage; the caller
should not modify its value.  The value is available to Python code 
as the list \code{sys.path}, which may be modified to change the 
future search path for loaded modules.

% XXX should give the exact rules
\end{cfuncdesc}

\begin{cfuncdesc}{const char*}{Py_GetVersion}{}
Return the version of this Python interpreter.  This is a string that 
looks something like

\begin{verbatim}
"1.5 (#67, Dec 31 1997, 22:34:28) [GCC 2.7.2.2]"
\end{verbatim}

The first word (up to the first space character) is the current Python 
version; the first three characters are the major and minor version 
separated by a period.  The returned string points into static storage; 
the caller should not modify its value.  The value is available to 
Python code as the list \code{sys.version}.
\end{cfuncdesc}

\begin{cfuncdesc}{const char*}{Py_GetPlatform}{}
Return the platform identifier for the current platform.  On \UNIX{}, 
this is formed from the ``official'' name of the operating system, 
converted to lower case, followed by the major revision number; e.g., 
for Solaris 2.x, which is also known as SunOS 5.x, the value is 
\code{"sunos5"}.  On Macintosh, it is \code{"mac"}.  On Windows, it 
is \code{"win"}.  The returned string points into static storage; 
the caller should not modify its value.  The value is available to 
Python code as \code{sys.platform}.
\end{cfuncdesc}

\begin{cfuncdesc}{const char*}{Py_GetCopyright}{}
Return the official copyright string for the current Python version, 
for example

\code{"Copyright 1991-1995 Stichting Mathematisch Centrum, Amsterdam"}

The returned string points into static storage; the caller should not 
modify its value.  The value is available to Python code as the list 
\code{sys.copyright}.
\end{cfuncdesc}

\begin{cfuncdesc}{const char*}{Py_GetCompiler}{}
Return an indication of the compiler used to build the current Python 
version, in square brackets, for example:

\begin{verbatim}
"[GCC 2.7.2.2]"
\end{verbatim}

The returned string points into static storage; the caller should not 
modify its value.  The value is available to Python code as part of 
the variable \code{sys.version}.
\end{cfuncdesc}

\begin{cfuncdesc}{const char*}{Py_GetBuildInfo}{}
Return information about the sequence number and build date and time 
of the current Python interpreter instance, for example

\begin{verbatim}
"#67, Aug  1 1997, 22:34:28"
\end{verbatim}

The returned string points into static storage; the caller should not 
modify its value.  The value is available to Python code as part of 
the variable \code{sys.version}.
\end{cfuncdesc}

\begin{cfuncdesc}{int}{PySys_SetArgv}{int argc, char **argv}
% XXX
\end{cfuncdesc}

% XXX Other PySys thingies (doesn't really belong in this chapter)

\section{Thread State and the Global Interpreter Lock}
\label{threads}

The Python interpreter is not fully thread safe.  In order to support
multi-threaded Python programs, there's a global lock that must be
held by the current thread before it can safely access Python objects.
Without the lock, even the simplest operations could cause problems in
a multi-threaded program: for example, when two threads simultaneously
increment the reference count of the same object, the reference count
could end up being incremented only once instead of twice.

Therefore, the rule exists that only the thread that has acquired the
global interpreter lock may operate on Python objects or call Python/C
API functions.  In order to support multi-threaded Python programs,
the interpreter regularly release and reacquires the lock --- by
default, every ten bytecode instructions (this can be changed with
\function{sys.setcheckinterval()}).  The lock is also released and
reacquired around potentially blocking I/O operations like reading or
writing a file, so that other threads can run while the thread that
requests the I/O is waiting for the I/O operation to complete.

The Python interpreter needs to keep some bookkeeping information
separate per thread --- for this it uses a data structure called
\ctype{PyThreadState}.  This is new in Python 1.5; in earlier versions,
such state was stored in global variables, and switching threads could
cause problems.  In particular, exception handling is now thread safe,
when the application uses \function{sys.exc_info()} to access the
exception last raised in the current thread.

There's one global variable left, however: the pointer to the current
\ctype{PyThreadState} structure.  While most thread packages have a way
to store ``per-thread global data,'' Python's internal platform
independent thread abstraction doesn't support this yet.  Therefore,
the current thread state must be manipulated explicitly.

This is easy enough in most cases.  Most code manipulating the global
interpreter lock has the following simple structure:

\begin{verbatim}
Save the thread state in a local variable.
Release the interpreter lock.
...Do some blocking I/O operation...
Reacquire the interpreter lock.
Restore the thread state from the local variable.
\end{verbatim}

This is so common that a pair of macros exists to simplify it:

\begin{verbatim}
Py_BEGIN_ALLOW_THREADS
...Do some blocking I/O operation...
Py_END_ALLOW_THREADS
\end{verbatim}

The \code{Py_BEGIN_ALLOW_THREADS} macro opens a new block and declares
a hidden local variable; the \code{Py_END_ALLOW_THREADS} macro closes
the block.  Another advantage of using these two macros is that when
Python is compiled without thread support, they are defined empty,
thus saving the thread state and lock manipulations.

When thread support is enabled, the block above expands to the
following code:

\begin{verbatim}
{
    PyThreadState *_save;
    _save = PyEval_SaveThread();
    ...Do some blocking I/O operation...
    PyEval_RestoreThread(_save);
}
\end{verbatim}

Using even lower level primitives, we can get roughly the same effect
as follows:

\begin{verbatim}
{
    PyThreadState *_save;
    _save = PyThreadState_Swap(NULL);
    PyEval_ReleaseLock();
    ...Do some blocking I/O operation...
    PyEval_AcquireLock();
    PyThreadState_Swap(_save);
}
\end{verbatim}

There are some subtle differences; in particular,
\cfunction{PyEval_RestoreThread()} saves and restores the value of the
global variable \cdata{errno}, since the lock manipulation does not
guarantee that \cdata{errno} is left alone.  Also, when thread support
is disabled, \cfunction{PyEval_SaveThread()} and
\cfunction{PyEval_RestoreThread()} don't manipulate the lock; in this
case, \cfunction{PyEval_ReleaseLock()} and
\cfunction{PyEval_AcquireLock()} are not available.  This is done so
that dynamically loaded extensions compiled with thread support
enabled can be loaded by an interpreter that was compiled with
disabled thread support.

The global interpreter lock is used to protect the pointer to the
current thread state.  When releasing the lock and saving the thread
state, the current thread state pointer must be retrieved before the
lock is released (since another thread could immediately acquire the
lock and store its own thread state in the global variable).
Reversely, when acquiring the lock and restoring the thread state, the
lock must be acquired before storing the thread state pointer.

Why am I going on with so much detail about this?  Because when
threads are created from \C{}, they don't have the global interpreter
lock, nor is there a thread state data structure for them.  Such
threads must bootstrap themselves into existence, by first creating a
thread state data structure, then acquiring the lock, and finally
storing their thread state pointer, before they can start using the
Python/C API.  When they are done, they should reset the thread state
pointer, release the lock, and finally free their thread state data
structure.

When creating a thread data structure, you need to provide an
interpreter state data structure.  The interpreter state data
structure hold global data that is shared by all threads in an
interpreter, for example the module administration
(\code{sys.modules}).  Depending on your needs, you can either create
a new interpreter state data structure, or share the interpreter state
data structure used by the Python main thread (to access the latter,
you must obtain the thread state and access its \member{interp} member;
this must be done by a thread that is created by Python or by the main
thread after Python is initialized).

XXX More?

\begin{ctypedesc}{PyInterpreterState}
This data structure represents the state shared by a number of
cooperating threads.  Threads belonging to the same interpreter
share their module administration and a few other internal items.
There are no public members in this structure.

Threads belonging to different interpreters initially share nothing,
except process state like available memory, open file descriptors and
such.  The global interpreter lock is also shared by all threads,
regardless of to which interpreter they belong.
\end{ctypedesc}

\begin{ctypedesc}{PyThreadState}
This data structure represents the state of a single thread.  The only
public data member is \ctype{PyInterpreterState *}\member{interp},
which points to this thread's interpreter state.
\end{ctypedesc}

\begin{cfuncdesc}{void}{PyEval_InitThreads}{}
Initialize and acquire the global interpreter lock.  It should be
called in the main thread before creating a second thread or engaging
in any other thread operations such as
\cfunction{PyEval_ReleaseLock()} or
\code{PyEval_ReleaseThread(\var{tstate})}.  It is not needed before
calling \cfunction{PyEval_SaveThread()} or
\cfunction{PyEval_RestoreThread()}.

This is a no-op when called for a second time.  It is safe to call
this function before calling \cfunction{Py_Initialize()}.

When only the main thread exists, no lock operations are needed.  This
is a common situation (most Python programs do not use threads), and
the lock operations slow the interpreter down a bit.  Therefore, the
lock is not created initially.  This situation is equivalent to having
acquired the lock: when there is only a single thread, all object
accesses are safe.  Therefore, when this function initializes the
lock, it also acquires it.  Before the Python
\module{thread}\refbimodindex{thread} module creates a new thread,
knowing that either it has the lock or the lock hasn't been created
yet, it calls \cfunction{PyEval_InitThreads()}.  When this call
returns, it is guaranteed that the lock has been created and that it
has acquired it.

It is \strong{not} safe to call this function when it is unknown which
thread (if any) currently has the global interpreter lock.

This function is not available when thread support is disabled at
compile time.
\end{cfuncdesc}

\begin{cfuncdesc}{void}{PyEval_AcquireLock}{}
Acquire the global interpreter lock.  The lock must have been created
earlier.  If this thread already has the lock, a deadlock ensues.
This function is not available when thread support is disabled at
compile time.
\end{cfuncdesc}

\begin{cfuncdesc}{void}{PyEval_ReleaseLock}{}
Release the global interpreter lock.  The lock must have been created
earlier.  This function is not available when thread support is
disabled at compile time.
\end{cfuncdesc}

\begin{cfuncdesc}{void}{PyEval_AcquireThread}{PyThreadState *tstate}
Acquire the global interpreter lock and then set the current thread
state to \var{tstate}, which should not be \NULL{}.  The lock must
have been created earlier.  If this thread already has the lock,
deadlock ensues.  This function is not available when thread support
is disabled at compile time.
\end{cfuncdesc}

\begin{cfuncdesc}{void}{PyEval_ReleaseThread}{PyThreadState *tstate}
Reset the current thread state to \NULL{} and release the global
interpreter lock.  The lock must have been created earlier and must be
held by the current thread.  The \var{tstate} argument, which must not
be \NULL{}, is only used to check that it represents the current
thread state --- if it isn't, a fatal error is reported.  This
function is not available when thread support is disabled at compile
time.
\end{cfuncdesc}

\begin{cfuncdesc}{PyThreadState*}{PyEval_SaveThread}{}
Release the interpreter lock (if it has been created and thread
support is enabled) and reset the thread state to \NULL{},
returning the previous thread state (which is not \NULL{}).  If
the lock has been created, the current thread must have acquired it.
(This function is available even when thread support is disabled at
compile time.)
\end{cfuncdesc}

\begin{cfuncdesc}{void}{PyEval_RestoreThread}{PyThreadState *tstate}
Acquire the interpreter lock (if it has been created and thread
support is enabled) and set the thread state to \var{tstate}, which
must not be \NULL{}.  If the lock has been created, the current
thread must not have acquired it, otherwise deadlock ensues.  (This
function is available even when thread support is disabled at compile
time.)
\end{cfuncdesc}

% XXX These aren't really C types, but the ctypedesc macro is the simplest!
\begin{ctypedesc}{Py_BEGIN_ALLOW_THREADS}
This macro expands to
\samp{\{ PyThreadState *_save; _save = PyEval_SaveThread();}.
Note that it contains an opening brace; it must be matched with a
following \code{Py_END_ALLOW_THREADS} macro.  See above for further
discussion of this macro.  It is a no-op when thread support is
disabled at compile time.
\end{ctypedesc}

\begin{ctypedesc}{Py_END_ALLOW_THREADS}
This macro expands to
\samp{PyEval_RestoreThread(_save); \}}.
Note that it contains a closing brace; it must be matched with an
earlier \code{Py_BEGIN_ALLOW_THREADS} macro.  See above for further
discussion of this macro.  It is a no-op when thread support is
disabled at compile time.
\end{ctypedesc}

\begin{ctypedesc}{Py_BEGIN_BLOCK_THREADS}
This macro expands to \samp{PyEval_RestoreThread(_save);} i.e. it
is equivalent to \code{Py_END_ALLOW_THREADS} without the closing
brace.  It is a no-op when thread support is disabled at compile
time.
\end{ctypedesc}

\begin{ctypedesc}{Py_BEGIN_UNBLOCK_THREADS}
This macro expands to \samp{_save = PyEval_SaveThread();} i.e. it is
equivalent to \code{Py_BEGIN_ALLOW_THREADS} without the opening brace
and variable declaration.  It is a no-op when thread support is
disabled at compile time.
\end{ctypedesc}

All of the following functions are only available when thread support
is enabled at compile time, and must be called only when the
interpreter lock has been created.

\begin{cfuncdesc}{PyInterpreterState*}{PyInterpreterState_New}{}
Create a new interpreter state object.  The interpreter lock must be
held.
\end{cfuncdesc}

\begin{cfuncdesc}{void}{PyInterpreterState_Clear}{PyInterpreterState *interp}
Reset all information in an interpreter state object.  The interpreter
lock must be held.
\end{cfuncdesc}

\begin{cfuncdesc}{void}{PyInterpreterState_Delete}{PyInterpreterState *interp}
Destroy an interpreter state object.  The interpreter lock need not be
held.  The interpreter state must have been reset with a previous
call to \cfunction{PyInterpreterState_Clear()}.
\end{cfuncdesc}

\begin{cfuncdesc}{PyThreadState*}{PyThreadState_New}{PyInterpreterState *interp}
Create a new thread state object belonging to the given interpreter
object.  The interpreter lock must be held.
\end{cfuncdesc}

\begin{cfuncdesc}{void}{PyThreadState_Clear}{PyThreadState *tstate}
Reset all information in a thread state object.  The interpreter lock
must be held.
\end{cfuncdesc}

\begin{cfuncdesc}{void}{PyThreadState_Delete}{PyThreadState *tstate}
Destroy a thread state object.  The interpreter lock need not be
held.  The thread state must have been reset with a previous
call to \cfunction{PyThreadState_Clear()}.
\end{cfuncdesc}

\begin{cfuncdesc}{PyThreadState*}{PyThreadState_Get}{}
Return the current thread state.  The interpreter lock must be held.
When the current thread state is \NULL{}, this issues a fatal
error (so that the caller needn't check for \NULL{}).
\end{cfuncdesc}

\begin{cfuncdesc}{PyThreadState*}{PyThreadState_Swap}{PyThreadState *tstate}
Swap the current thread state with the thread state given by the
argument \var{tstate}, which may be \NULL{}.  The interpreter lock
must be held.
\end{cfuncdesc}


\chapter{Defining New Object Types}
\label{newTypes}

\begin{cfuncdesc}{PyObject*}{_PyObject_New}{PyTypeObject *type}
\end{cfuncdesc}

\begin{cfuncdesc}{PyObject*}{_PyObject_NewVar}{PyTypeObject *type, int size}
\end{cfuncdesc}

\begin{cfuncdesc}{TYPE}{_PyObject_NEW}{TYPE, PyTypeObject *}
\end{cfuncdesc}

\begin{cfuncdesc}{TYPE}{_PyObject_NEW_VAR}{TYPE, PyTypeObject *, int size}
\end{cfuncdesc}

Py_InitModule (!!!)

PyArg_ParseTupleAndKeywords, PyArg_ParseTuple, PyArg_Parse

Py_BuildValue

PyObject, PyVarObject

PyObject_HEAD, PyObject_HEAD_INIT, PyObject_VAR_HEAD

Typedefs:
unaryfunc, binaryfunc, ternaryfunc, inquiry, coercion, intargfunc,
intintargfunc, intobjargproc, intintobjargproc, objobjargproc,
getreadbufferproc, getwritebufferproc, getsegcountproc,
destructor, printfunc, getattrfunc, getattrofunc, setattrfunc,
setattrofunc, cmpfunc, reprfunc, hashfunc

PyNumberMethods

PySequenceMethods

PyMappingMethods

PyBufferProcs

PyTypeObject

DL_IMPORT

PyType_Type

Py*_Check

Py_None, _Py_NoneStruct


\chapter{Debugging}
\label{debugging}

XXX Explain Py_DEBUG, Py_TRACE_REFS, Py_REF_DEBUG.


\input{api.ind}			% Index -- must be last

\end{document}
			% Index -- must be last

\end{document}
			% Index -- must be last

\end{document}
			% Index -- must be last

\end{document}

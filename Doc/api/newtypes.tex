\chapter{Defining New Object Types \label{newTypes}}


\section{Allocating Objects on the Heap
         \label{allocating-objects}}

\begin{cfuncdesc}{PyObject*}{_PyObject_New}{PyTypeObject *type}
\end{cfuncdesc}

\begin{cfuncdesc}{PyVarObject*}{_PyObject_NewVar}{PyTypeObject *type, int size}
\end{cfuncdesc}

\begin{cfuncdesc}{void}{_PyObject_Del}{PyObject *op}
\end{cfuncdesc}

\begin{cfuncdesc}{PyObject*}{PyObject_Init}{PyObject *op,
					    PyTypeObject *type}
  Initialize a newly-allocated object \var{op} with its type and
  initial reference.  Returns the initialized object.  If \var{type}
  indicates that the object participates in the cyclic garbage
  detector, it it added to the detector's set of observed objects.
  Other fields of the object are not affected.
\end{cfuncdesc}

\begin{cfuncdesc}{PyVarObject*}{PyObject_InitVar}{PyVarObject *op,
						  PyTypeObject *type, int size}
  This does everything \cfunction{PyObject_Init()} does, and also
  initializes the length information for a variable-size object.
\end{cfuncdesc}

\begin{cfuncdesc}{\var{TYPE}*}{PyObject_New}{TYPE, PyTypeObject *type}
  Allocate a new Python object using the C structure type \var{TYPE}
  and the Python type object \var{type}.  Fields not defined by the
  Python object header are not initialized; the object's reference
  count will be one.  The size of the memory
  allocation is determined from the \member{tp_basicsize} field of the
  type object.
\end{cfuncdesc}

\begin{cfuncdesc}{\var{TYPE}*}{PyObject_NewVar}{TYPE, PyTypeObject *type,
                                                int size}
  Allocate a new Python object using the C structure type \var{TYPE}
  and the Python type object \var{type}.  Fields not defined by the
  Python object header are not initialized.  The allocated memory
  allows for the \var{TYPE} structure plus \var{size} fields of the
  size given by the \member{tp_itemsize} field of \var{type}.  This is
  useful for implementing objects like tuples, which are able to
  determine their size at construction time.  Embedding the array of
  fields into the same allocation decreases the number of allocations,
  improving the memory management efficiency.
\end{cfuncdesc}

\begin{cfuncdesc}{void}{PyObject_Del}{PyObject *op}
  Releases memory allocated to an object using
  \cfunction{PyObject_New()} or \cfunction{PyObject_NewVar()}.  This
  is normally called from the \member{tp_dealloc} handler specified in
  the object's type.  The fields of the object should not be accessed
  after this call as the memory is no longer a valid Python object.
\end{cfuncdesc}

\begin{cfuncdesc}{\var{TYPE}*}{PyObject_NEW}{TYPE, PyTypeObject *type}
  Macro version of \cfunction{PyObject_New()}, to gain performance at
  the expense of safety.  This does not check \var{type} for a \NULL{}
  value.
\end{cfuncdesc}

\begin{cfuncdesc}{\var{TYPE}*}{PyObject_NEW_VAR}{TYPE, PyTypeObject *type,
                                                int size}
  Macro version of \cfunction{PyObject_NewVar()}, to gain performance
  at the expense of safety.  This does not check \var{type} for a
  \NULL{} value.
\end{cfuncdesc}

\begin{cfuncdesc}{void}{PyObject_DEL}{PyObject *op}
  Macro version of \cfunction{PyObject_Del()}.
\end{cfuncdesc}

\begin{cfuncdesc}{PyObject*}{Py_InitModule}{char *name,
                                            PyMethodDef *methods}
  Create a new module object based on a name and table of functions,
  returning the new module object.
\end{cfuncdesc}

\begin{cfuncdesc}{PyObject*}{Py_InitModule3}{char *name,
                                             PyMethodDef *methods,
                                             char *doc}
  Create a new module object based on a name and table of functions,
  returning the new module object.  If \var{doc} is non-\NULL, it will
  be used to define the docstring for the module.
\end{cfuncdesc}

\begin{cfuncdesc}{PyObject*}{Py_InitModule4}{char *name,
                                             PyMethodDef *methods,
                                             char *doc, PyObject *self,
                                             int apiver}
  Create a new module object based on a name and table of functions,
  returning the new module object.  If \var{doc} is non-\NULL, it will
  be used to define the docstring for the module.  If \var{self} is
  non-\NULL, it will passed to the functions of the module as their
  (otherwise \NULL) first parameter.  (This was added as an
  experimental feature, and there are no known uses in the current
  version of Python.)  For \var{apiver}, the only value which should
  be passed is defined by the constant \constant{PYTHON_API_VERSION}.

  \note{Most uses of this function should probably be using
  the \cfunction{Py_InitModule3()} instead; only use this if you are
  sure you need it.}
\end{cfuncdesc}

DL_IMPORT

\begin{cvardesc}{PyObject}{_Py_NoneStruct}
  Object which is visible in Python as \code{None}.  This should only
  be accessed using the \code{Py_None} macro, which evaluates to a
  pointer to this object.
\end{cvardesc}


\section{Common Object Structures \label{common-structs}}

PyObject, PyVarObject

PyObject_HEAD, PyObject_HEAD_INIT, PyObject_VAR_HEAD

Typedefs:
unaryfunc, binaryfunc, ternaryfunc, inquiry, coercion, intargfunc,
intintargfunc, intobjargproc, intintobjargproc, objobjargproc,
destructor, printfunc, getattrfunc, getattrofunc, setattrfunc,
setattrofunc, cmpfunc, reprfunc, hashfunc

\begin{ctypedesc}{PyCFunction}
  Type of the functions used to implement most Python callables in C.
\end{ctypedesc}

\begin{ctypedesc}{PyMethodDef}
  Structure used to describe a method of an extension type.  This
  structure has four fields:

  \begin{tableiii}{l|l|l}{member}{Field}{C Type}{Meaning}
    \lineiii{ml_name}{char *}{name of the method}
    \lineiii{ml_meth}{PyCFunction}{pointer to the C implementation}
    \lineiii{ml_flags}{int}{flag bits indicating how the call should be
                            constructed}
    \lineiii{ml_doc}{char *}{points to the contents of the docstring}
  \end{tableiii}
\end{ctypedesc}

The \member{ml_meth} is a C function pointer.  The functions may be of
different types, but they always return \ctype{PyObject*}.  If the
function is not of the \ctype{PyCFunction}, the compiler will require
a cast in the method table.  Even though \ctype{PyCFunction} defines
the first parameter as \ctype{PyObject*}, it is common that the method
implementation uses a the specific C type of the \var{self} object.

The flags can have the following values. Only \constant{METH_VARARGS}
and \constant{METH_KEYWORDS} can be combined; the others can't.

\begin{datadesc}{METH_VARARGS}
  This is the typical calling convention, where the methods have the
  type \ctype{PyMethodDef}. The function expects two
  \ctype{PyObject*} values.  The first one is the \var{self} object for
  methods; for module functions, it has the value given to
  \cfunction{Py_InitModule4()} (or \NULL{} if
  \cfunction{Py_InitModule()} was used).  The second parameter
  (often called \var{args}) is a tuple object representing all
  arguments. This parameter is typically processed using
  \cfunction{PyArg_ParseTuple()}.
\end{datadesc}

\begin{datadesc}{METH_KEYWORDS}
  Methods with these flags must be of type
  \ctype{PyCFunctionWithKeywords}.  The function expects three
  parameters: \var{self}, \var{args}, and a dictionary of all the
  keyword arguments.  The flag is typically combined with
  \constant{METH_VARARGS}, and the parameters are typically processed
  using \cfunction{PyArg_ParseTupleAndKeywords()}.
\end{datadesc}

\begin{datadesc}{METH_NOARGS}
  Methods without parameters don't need to check whether arguments are
  given if they are listed with the \constant{METH_NOARGS} flag.  They
  need to be of type \ctype{PyNoArgsFunction}: they expect a single
  single \ctype{PyObject*} as a parameter.  When used with object
  methods, this parameter is typically named \code{self} and will hold
  a reference to the object instance.
\end{datadesc}

\begin{datadesc}{METH_O}
  Methods with a single object argument can be listed with the
  \constant{METH_O} flag, instead of invoking
  \cfunction{PyArg_ParseTuple()} with a \code{"O"} argument. They have
  the type \ctype{PyCFunction}, with the \var{self} parameter, and a
  \ctype{PyObject*} parameter representing the single argument.
\end{datadesc}

\begin{datadesc}{METH_OLDARGS}
  This calling convention is deprecated.  The method must be of type
  \ctype{PyCFunction}.  The second argument is \NULL{} if no arguments
  are given, a single object if exactly one argument is given, and a
  tuple of objects if more than one argument is given.  There is no
  way for a function using this convention to distinguish between a
  call with multiple arguments and a call with a tuple as the only
  argument.
\end{datadesc}

\begin{cfuncdesc}{PyObject*}{Py_FindMethod}{PyMethodDef table[],
                                            PyObject *ob, char *name}
  Return a bound method object for an extension type implemented in
  C.  This can be useful in the implementation of a
  \member{tp_getattro} or \member{tp_getattr} handler that does not
  use the \cfunction{PyObject_GenericGetAttr()} function.
\end{cfuncdesc}


\section{Mapping Object Structures \label{mapping-structs}}

\begin{ctypedesc}{PyMappingMethods}
  Structure used to hold pointers to the functions used to implement
  the mapping protocol for an extension type.
\end{ctypedesc}


\section{Number Object Structures \label{number-structs}}

\begin{ctypedesc}{PyNumberMethods}
  Structure used to hold pointers to the functions an extension type
  uses to implement the number protocol.
\end{ctypedesc}


\section{Sequence Object Structures \label{sequence-structs}}

\begin{ctypedesc}{PySequenceMethods}
  Structure used to hold pointers to the functions which an object
  uses to implement the sequence protocol.
\end{ctypedesc}


\section{Buffer Object Structures \label{buffer-structs}}
\sectionauthor{Greg J. Stein}{greg@lyra.org}

The buffer interface exports a model where an object can expose its
internal data as a set of chunks of data, where each chunk is
specified as a pointer/length pair.  These chunks are called
\dfn{segments} and are presumed to be non-contiguous in memory.

If an object does not export the buffer interface, then its
\member{tp_as_buffer} member in the \ctype{PyTypeObject} structure
should be \NULL.  Otherwise, the \member{tp_as_buffer} will point to
a \ctype{PyBufferProcs} structure.

\note{It is very important that your \ctype{PyTypeObject} structure
uses \constant{Py_TPFLAGS_DEFAULT} for the value of the
\member{tp_flags} member rather than \code{0}.  This tells the Python
runtime that your \ctype{PyBufferProcs} structure contains the
\member{bf_getcharbuffer} slot. Older versions of Python did not have
this member, so a new Python interpreter using an old extension needs
to be able to test for its presence before using it.}

\begin{ctypedesc}{PyBufferProcs}
  Structure used to hold the function pointers which define an
  implementation of the buffer protocol.

  The first slot is \member{bf_getreadbuffer}, of type
  \ctype{getreadbufferproc}.  If this slot is \NULL, then the object
  does not support reading from the internal data.  This is
  non-sensical, so implementors should fill this in, but callers
  should test that the slot contains a non-\NULL{} value.

  The next slot is \member{bf_getwritebuffer} having type
  \ctype{getwritebufferproc}.  This slot may be \NULL{} if the object
  does not allow writing into its returned buffers.

  The third slot is \member{bf_getsegcount}, with type
  \ctype{getsegcountproc}.  This slot must not be \NULL{} and is used
  to inform the caller how many segments the object contains.  Simple
  objects such as \ctype{PyString_Type} and \ctype{PyBuffer_Type}
  objects contain a single segment.

  The last slot is \member{bf_getcharbuffer}, of type
  \ctype{getcharbufferproc}.  This slot will only be present if the
  \constant{Py_TPFLAGS_HAVE_GETCHARBUFFER} flag is present in the
  \member{tp_flags} field of the object's \ctype{PyTypeObject}.
  Before using this slot, the caller should test whether it is present
  by using the
  \cfunction{PyType_HasFeature()}\ttindex{PyType_HasFeature()}
  function.  If present, it may be \NULL, indicating that the object's
  contents cannot be used as \emph{8-bit characters}.
  The slot function may also raise an error if the object's contents
  cannot be interpreted as 8-bit characters.  For example, if the
  object is an array which is configured to hold floating point
  values, an exception may be raised if a caller attempts to use
  \member{bf_getcharbuffer} to fetch a sequence of 8-bit characters.
  This notion of exporting the internal buffers as ``text'' is used to
  distinguish between objects that are binary in nature, and those
  which have character-based content.

  \note{The current policy seems to state that these characters
  may be multi-byte characters. This implies that a buffer size of
  \var{N} does not mean there are \var{N} characters present.}
\end{ctypedesc}

\begin{datadesc}{Py_TPFLAGS_HAVE_GETCHARBUFFER}
  Flag bit set in the type structure to indicate that the
  \member{bf_getcharbuffer} slot is known.  This being set does not
  indicate that the object supports the buffer interface or that the
  \member{bf_getcharbuffer} slot is non-\NULL.
\end{datadesc}

\begin{ctypedesc}[getreadbufferproc]{int (*getreadbufferproc)
                            (PyObject *self, int segment, void **ptrptr)}
  Return a pointer to a readable segment of the buffer.  This function
  is allowed to raise an exception, in which case it must return
  \code{-1}.  The \var{segment} which is passed must be zero or
  positive, and strictly less than the number of segments returned by
  the \member{bf_getsegcount} slot function.  On success, it returns
  the length of the buffer memory, and sets \code{*\var{ptrptr}} to a
  pointer to that memory.
\end{ctypedesc}

\begin{ctypedesc}[getwritebufferproc]{int (*getwritebufferproc)
                            (PyObject *self, int segment, void **ptrptr)}
  Return a pointer to a writable memory buffer in
  \code{*\var{ptrptr}}, and the length of that segment as the function
  return value.  The memory buffer must correspond to buffer segment
  \var{segment}.  Must return \code{-1} and set an exception on
  error.  \exception{TypeError} should be raised if the object only
  supports read-only buffers, and \exception{SystemError} should be
  raised when \var{segment} specifies a segment that doesn't exist.
% Why doesn't it raise ValueError for this one?
% GJS: because you shouldn't be calling it with an invalid
%      segment. That indicates a blatant programming error in the C
%      code.
\end{ctypedesc}

\begin{ctypedesc}[getsegcountproc]{int (*getsegcountproc)
                            (PyObject *self, int *lenp)}
  Return the number of memory segments which comprise the buffer.  If
  \var{lenp} is not \NULL, the implementation must report the sum of
  the sizes (in bytes) of all segments in \code{*\var{lenp}}.
  The function cannot fail.
\end{ctypedesc}

\begin{ctypedesc}[getcharbufferproc]{int (*getcharbufferproc)
                            (PyObject *self, int segment, const char **ptrptr)}
\end{ctypedesc}


\section{Supporting the Iterator Protocol
         \label{supporting-iteration}}


\section{Supporting Cyclic Garbarge Collection
         \label{supporting-cycle-detection}}

Python's support for detecting and collecting garbage which involves
circular references requires support from object types which are
``containers'' for other objects which may also be containers.  Types
which do not store references to other objects, or which only store
references to atomic types (such as numbers or strings), do not need
to provide any explicit support for garbage collection.

To create a container type, the \member{tp_flags} field of the type
object must include the \constant{Py_TPFLAGS_HAVE_GC} and provide an
implementation of the \member{tp_traverse} handler.  If instances of the
type are mutable, a \member{tp_clear} implementation must also be
provided.

\begin{datadesc}{Py_TPFLAGS_HAVE_GC}
  Objects with a type with this flag set must conform with the rules
  documented here.  For convenience these objects will be referred to
  as container objects.
\end{datadesc}

Constructors for container types must conform to two rules:

\begin{enumerate}
\item  The memory for the object must be allocated using
       \cfunction{PyObject_GC_New()} or \cfunction{PyObject_GC_VarNew()}.

\item  Once all the fields which may contain references to other
       containers are initialized, it must call
       \cfunction{PyObject_GC_Track()}.
\end{enumerate}

\begin{cfuncdesc}{\var{TYPE}*}{PyObject_GC_New}{TYPE, PyTypeObject *type}
  Analogous to \cfunction{PyObject_New()} but for container objects with
  the \constant{Py_TPFLAGS_HAVE_GC} flag set.
\end{cfuncdesc}

\begin{cfuncdesc}{\var{TYPE}*}{PyObject_GC_NewVar}{TYPE, PyTypeObject *type,
                                                   int size}
  Analogous to \cfunction{PyObject_NewVar()} but for container objects
  with the \constant{Py_TPFLAGS_HAVE_GC} flag set.
\end{cfuncdesc}

\begin{cfuncdesc}{PyVarObject *}{PyObject_GC_Resize}{PyVarObject *op, int}
  Resize an object allocated by \cfunction{PyObject_NewVar()}.  Returns
  the resized object or \NULL{} on failure.
\end{cfuncdesc}

\begin{cfuncdesc}{void}{PyObject_GC_Track}{PyObject *op}
  Adds the object \var{op} to the set of container objects tracked by
  the collector.  The collector can run at unexpected times so objects
  must be valid while being tracked.  This should be called once all
  the fields followed by the \member{tp_traverse} handler become valid,
  usually near the end of the constructor.
\end{cfuncdesc}

\begin{cfuncdesc}{void}{_PyObject_GC_TRACK}{PyObject *op}
  A macro version of \cfunction{PyObject_GC_Track()}.  It should not be
  used for extension modules.
\end{cfuncdesc}

Similarly, the deallocator for the object must conform to a similar
pair of rules:

\begin{enumerate}
\item  Before fields which refer to other containers are invalidated,
       \cfunction{PyObject_GC_UnTrack()} must be called.

\item  The object's memory must be deallocated using
       \cfunction{PyObject_GC_Del()}.
\end{enumerate}

\begin{cfuncdesc}{void}{PyObject_GC_Del}{PyObject *op}
  Releases memory allocated to an object using
  \cfunction{PyObject_GC_New()} or \cfunction{PyObject_GC_NewVar()}.
\end{cfuncdesc}

\begin{cfuncdesc}{void}{PyObject_GC_UnTrack}{PyObject *op}
  Remove the object \var{op} from the set of container objects tracked
  by the collector.  Note that \cfunction{PyObject_GC_Track()} can be
  called again on this object to add it back to the set of tracked
  objects.  The deallocator (\member{tp_dealloc} handler) should call
  this for the object before any of the fields used by the
  \member{tp_traverse} handler become invalid.
\end{cfuncdesc}

\begin{cfuncdesc}{void}{_PyObject_GC_UNTRACK}{PyObject *op}
  A macro version of \cfunction{PyObject_GC_UnTrack()}.  It should not be
  used for extension modules.
\end{cfuncdesc}

The \member{tp_traverse} handler accepts a function parameter of this
type:

\begin{ctypedesc}[visitproc]{int (*visitproc)(PyObject *object, void *arg)}
  Type of the visitor function passed to the \member{tp_traverse}
  handler.  The function should be called with an object to traverse
  as \var{object} and the third parameter to the \member{tp_traverse}
  handler as \var{arg}.
\end{ctypedesc}

The \member{tp_traverse} handler must have the following type:

\begin{ctypedesc}[traverseproc]{int (*traverseproc)(PyObject *self,
                                visitproc visit, void *arg)}
  Traversal function for a container object.  Implementations must
  call the \var{visit} function for each object directly contained by
  \var{self}, with the parameters to \var{visit} being the contained
  object and the \var{arg} value passed to the handler.  If
  \var{visit} returns a non-zero value then an error has occurred and
  that value should be returned immediately.
\end{ctypedesc}

The \member{tp_clear} handler must be of the \ctype{inquiry} type, or
\NULL{} if the object is immutable.

\begin{ctypedesc}[inquiry]{int (*inquiry)(PyObject *self)}
  Drop references that may have created reference cycles.  Immutable
  objects do not have to define this method since they can never
  directly create reference cycles.  Note that the object must still
  be valid after calling this method (don't just call
  \cfunction{Py_DECREF()} on a reference).  The collector will call
  this method if it detects that this object is involved in a
  reference cycle.
\end{ctypedesc}


\subsection{Example Cycle Collector Support
            \label{example-cycle-support}}

This example shows only enough of the implementation of an extension
type to show how the garbage collector support needs to be added.  It
shows the definition of the object structure, the
\member{tp_traverse}, \member{tp_clear} and \member{tp_dealloc}
implementations, the type structure, and a constructor --- the module
initialization needed to export the constructor to Python is not shown
as there are no special considerations there for the collector.  To
make this interesting, assume that the module exposes ways for the
\member{container} field of the object to be modified.  Note that
since no checks are made on the type of the object used to initialize
\member{container}, we have to assume that it may be a container.

\begin{verbatim}
#include "Python.h"

typedef struct {
    PyObject_HEAD
    PyObject *container;
} MyObject;

static int
my_traverse(MyObject *self, visitproc visit, void *arg)
{
    if (self->container != NULL)
        return visit(self->container, arg);
    else
        return 0;
}

static int
my_clear(MyObject *self)
{
    Py_XDECREF(self->container);
    self->container = NULL;

    return 0;
}

static void
my_dealloc(MyObject *self)
{
    PyObject_GC_UnTrack((PyObject *) self);
    Py_XDECREF(self->container);
    PyObject_GC_Del(self);
}
\end{verbatim}

\begin{verbatim}
statichere PyTypeObject
MyObject_Type = {
    PyObject_HEAD_INIT(NULL)
    0,
    "MyObject",
    sizeof(MyObject),
    0,
    (destructor)my_dealloc,     /* tp_dealloc */
    0,                          /* tp_print */
    0,                          /* tp_getattr */
    0,                          /* tp_setattr */
    0,                          /* tp_compare */
    0,                          /* tp_repr */
    0,                          /* tp_as_number */
    0,                          /* tp_as_sequence */
    0,                          /* tp_as_mapping */
    0,                          /* tp_hash */
    0,                          /* tp_call */
    0,                          /* tp_str */
    0,                          /* tp_getattro */
    0,                          /* tp_setattro */
    0,                          /* tp_as_buffer */
    Py_TPFLAGS_DEFAULT | Py_TPFLAGS_HAVE_GC,
    0,                          /* tp_doc */
    (traverseproc)my_traverse,  /* tp_traverse */
    (inquiry)my_clear,          /* tp_clear */
    0,                          /* tp_richcompare */
    0,                          /* tp_weaklistoffset */
};

/* This constructor should be made accessible from Python. */
static PyObject *
new_object(PyObject *unused, PyObject *args)
{
    PyObject *container = NULL;
    MyObject *result = NULL;

    if (PyArg_ParseTuple(args, "|O:new_object", &container)) {
        result = PyObject_GC_New(MyObject, &MyObject_Type);
        if (result != NULL) {
            result->container = container;
            PyObject_GC_Track(result);
        }
    }
    return (PyObject *) result;
}
\end{verbatim}

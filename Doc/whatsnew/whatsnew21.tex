\documentclass{howto}

% $Id$

\title{What's New in Python 2.1}
\release{0.01}
\author{A.M. Kuchling}
\authoraddress{\email{amk1@bigfoot.com}}
\begin{document}
\maketitle\tableofcontents

\section{Introduction}

It's that time again... time for a new Python release, version 2.1.
One recent goal of the Python development team has been to accelerate
the pace of new releases, with a new release coming every 6 to 9
months. 2.1 is the first release to come out at this faster pace, with
the first alpha appearing in January, 3 months after the final version
of 2.0 was released.

This article explains the new features in 2.1.  While there aren't as
many changes in 2.1 as there were in Python 2.0, there are still some
pleasant surprises in store.  2.1 is the first release to be steered
through the use of Python Enhancement Proposals, or PEPs, so most of
the sizable changes have accompanying PEPs that provide more complete
documentation and a design rationale for the change.  This article
doesn't attempt to document the new features completely, but simply
provides an overview of the new features for Python programmers.
Refer to the Python 2.1 documentation, or to the specific PEP, for
more details about any new feature that particularly interests you.

Currently 2.1 is available in an alpha release, but the release
schedule calls for a beta release by late February 2001, and a final
release in April 2001.

% ======================================================================
\section{PEP 232: Function Attributes}

In Python 2.1, functions can have arbitrary attributes assigned to
them.  We noticed that people were often using docstrings to hold
information about functions and methods, because the \code{__doc__}
attribute was the only way of attaching any information to a function.
For example, in the Zope Web application server, functions are marked
as safe for public access by having a docstring, and in John Aycock's
SPARK parsing framework, docstrings hold parts of the BNF grammar to
be parsed.  This overloading is unfortunate, since docstrings are
really intended to hold a function's documentation, and means you
can't properly document functions intended for private use in Zope.

XXX example here

% ======================================================================
\section{PEP 207: Rich Comparisons}

xxx

% ======================================================================
\section{XXX Nested Scopes ?}

xxx

% ======================================================================
\section{PEP 230: Warning Framework}

Over its 10 years of existence, Python has accumulated a certain
number of obsolete modules and features along the way.  It's difficult
to know when a feature is safe to remove, since there's no way of
knowing how much code uses it -- perhaps no programs depend on the
feature, or perhaps many do.  To enable removing old features in a
more structured way, a warning framework was added.  When the Python
developers want to get rid of a feature, it will first trigger a
warning in the next version of Python.  The following Python version
can then drop the feature, and users will have had a full release
cycle to remove uses of the old feature.  

Python 2.1 adds the warning framework to be used in this scheme.  It
adds a \module{warnings} module that provide functions to issue
warnings, and to filter out warnings that you don't want to be
displayed. Third-party modules can also use this framework to
deprecate old features that they no longer wish to support.

For example, in Python 2.1 the \module{regex} module is deprecated,
so importing it causes a warning to be printed:

\begin{verbatim}
>>> import regex
xxx
\end{verbatim}

Warnings can be issued by calling the \function{warnings.warn} function:

\begin{verbatim}
warnings.warn("feature X no longer supported")              
\end{verbatim}

The first parameter is the warning message; an additional optional
parameters can be used to specify a particular warning category.

Filters can be added to disable certain warnings; a regular expression
pattern can be applied to the message or to the module name in order
to suppress a warning.  For example, you may have a program that uses
the \module{regex} module and not want to spare the time to convert it
to use the \module{re} module right now.  The warning can be suppressed by calling

\begin{verbatim}
XXX is this correct?
warnings.filterwarnings(module = 'regex')
\end{verbatim}

Functions were also added to Python's C API for issuing warnings;
refer to the PEP or to Python's API documentation for the details.

\seepep{5}{Guidelines for Language Evolution}{Written by Paul Prescod.}

\seepep{230}{Warning Framework}{Written and implemented by GvR.}
    
% ======================================================================
\section{PEP 229: New Build System}

When compiling Python, the user had to go in and edit the
\file{Modules/Setup} file in order to enable various additional
modules; the default set is relatively small and limited to modules
that compile on most Unix platforms.  This means that on Unix
platforms with many more features, most notably Linux, Python
installations often don't contain all useful modules they could.

Python 2.0 added the Distutils, a set of modules for distributing and
installing extensions.  In Python 2.1, the Distutils are used to
compile much of the standard library of extension modules,
autodetecting which ones are supported on the current machine.  
It's hoped that this will make Python installations easier and more featureful.

\seepep{229}{xxx}{Written and implemented by A.M. Kuchling.}

% ======================================================================
\section{PEP 217: Interactive Display Hook}

When using the Python interpreter interactively, the output of
commands is displayed using the built-in \function{repr()} function.
In Python 2.1, the variable \module{sys.displayhook} can be set to a
callable object which will be called instead of \function{repr()}.
For example, you can set it to a special pretty-printing function:

\begin{verbatim}
xxx sample transcript
\end{verbatim}

\seepep{217}{Display Hook for Interactive Use}{Written and implemented by Moshe Zadka.}

% ======================================================================
\section{PEP 208: New Coercion Model}

How numeric coercion is done at the C level was significantly
modified.  This will only affect the authors of C extensions to
Python, allowing them more flexibility in writing extension types that
support numeric operations.

Extension types can now set the type flag
\code{Py_TPFLAGS_NEWSTYLENUMBER} in their \code{PyTypeObject}
structure to indicate that they support the new coercion model.  In
such extension types, the numeric slot functions can no longer assume
that they'll be passed two arguments of the same type; instead they
may be passed two arguments of differing types, and can then perform
their own internal coercion.  If the slot function is passed a type it
can't handle, it can indicate the failure by returning a reference to
the \code{Py_NotImplemented} singleton value.  The numeric functions
of the other type will then be tried, and perhaps they can handle the
operation; if the other type also returns \code{Py_NotImplemented},
then a \exception{TypeError} will be raised.

\seepep{208}{Reworking the Coercion Model}{Written and implemented by
Neil Schemenauer, heavily based upon earlier work by Marc-Andr\'e
Lemburg.  Read this to understand the fine points of how numeric
operations will now be processed at the C level.}


% ======================================================================
\section{Minor Changes and Fixes}

There were relatively few smaller changes made in Python 2.1 due to
the shorter release cycle.  A search through the CVS change logs turns
up 57 patches applied, and 86 bugs fixed; both figures are likely to
be underestimates.  Some of the more notable changes are:

\begin{itemize}


\item The speed of line-oriented file I/O has been improved because
people often complain about its lack of speed, and because it's often
been used as a na\"ive benchmark.  The \method{readline()} method of
file objects has therefore been rewritten to be much faster.  The
exact amount of the speedup will vary from platform to platform
depending on how slow the C library's \function{getc()} was, but is
around 66\%, and potentially much faster on some particular operating
systems.

A new module and method for file objects was also added, contributed
by Jeff Epler. The new method, \method{xreadlines()}, is similar to
the existing \function{xrange()} built-in.  \function{xreadlines()}
returns an opaque sequence object that only supports being iterated
over, reading a line on every iteration but not reading the entire file into memory as
the existing \method{readline()} method.  You'd use it like this:

\begin{verbatim}
for line in sys.stdin.xreadlines():
    # ... do something for each line ...
    ...
\end{verbatim}

For a fuller discussion of the line I/O changes, see the python-dev summary for January 1-15, 2001.
 
\item \module{curses.panel}, a wrapper for the panel library, part of
ncurses and of SYSV curses, was contributed by Thomas Gellekum.  The
panel library provides windows with the additional feature of depth.
Windows can be moved higher or lower in the depth ordering, and the
panel library figures out where panels overlap and which sections are
visible.
XXX who contributed this?

\item Modules can now control which names are imported when \code{from
\var{module} import *} is used, by defining a \code{__all__} attribute
containing a list of names that will be imported.  One common
complaint is that if the module imports other modules such as
\module{sys} or \module{string}, \code{from \var{module} import *}
will add them to the importing module's namespace.  To fix this,
simply list the public names in \code{__all__}:

\begin{verbatim}
# List public names
__all__ = ['Database', 'open']
\end{verbatim}

\item The \module{ftplib} module now defaults to retrieving files in passive mode,
because passive mode is more likely to work from behind a firewall.
If passive mode is unsuitable for your application or network setup, call
\method{set_pasv(0)} on FTP objects to disable passive mode.

XXX check bug 126851 for arguments.

\end{itemize}

And there's the usual list of bugfixes, minor memory leaks, docstring
edits, and other tweaks, too lengthy to be worth itemizing; see the
CVS logs for the full details if you want them.


% ======================================================================
\section{Acknowledgements}

The author would like to thank the following people for offering
suggestions on various drafts of this article: no one yet!

\end{document}

\section{Standard Module \sectcode{sgmllib}}
\stmodindex{sgmllib}
\index{SGML}

\renewcommand{\indexsubitem}{(in module sgmllib)}

This module defines a class \code{SGMLParser} which serves as the
basis for parsing text files formatted in SGML (Standard Generalized
Mark-up Language).  In fact, it does not provide a full SGML parser
--- it only parses SGML insofar as it is used by HTML, and the module only
exists as a basis for the \code{htmllib} module.
\stmodindex{htmllib}

In particular, the parser is hardcoded to recognize the following
elements:

\begin{itemize}

\item
Opening and closing tags of the form
``\code{<\var{tag} \var{attr}="\var{value}" ...>}'' and
``\code{</\var{tag}>}'', respectively.

\item
Character references of the form ``\code{\&\#\var{name};}''.

\item
Entity references of the form ``\code{\&\var{name};}''.

\item
SGML comments of the form ``\code{<!--\var{text}>}''.

\end{itemize}

The \code{SGMLParser} class must be instantiated without arguments.
It has the following interface methods:

\begin{funcdesc}{reset}{}
Reset the instance.  Loses all unprocessed data.  This is called
implicitly at instantiation time.
\end{funcdesc}

\begin{funcdesc}{setnomoretags}{}
Stop processing tags.  Treat all following input as literal input
(CDATA).  (This is only provided so the HTML tag \code{<PLAINTEXT>}
can be implemented.)
\end{funcdesc}

\begin{funcdesc}{setliteral}{}
Enter literal mode (CDATA mode).
\end{funcdesc}

\begin{funcdesc}{feed}{data}
Feed some text to the parser.  It is processed insofar as it consists
of complete elements; incomplete data is buffered until more data is
fed or \code{close()} is called.
\end{funcdesc}

\begin{funcdesc}{close}{}
Force processing of all buffered data as if it were followed by an
end-of-file mark.  This method may be redefined by a derived class to
define additional processing at the end of the input, but the
redefined version should always call \code{SGMLParser.close()}.
\end{funcdesc}

\begin{funcdesc}{handle_charref}{ref}
This method is called to process a character reference of the form
``\code{\&\#\var{ref};}'' where \var{ref} is a decimal number in the
range 0-255.  It translates the character to \ASCII{} and calls the
method \code{handle_data()} with the character as argument.  If
\var{ref} is invalid or out of range, the method
\code{unknown_charref(\var{ref})} is called instead.
\end{funcdesc}

\begin{funcdesc}{handle_entityref}{ref}
This method is called to process an entity reference of the form
``\code{\&\var{ref};}'' where \var{ref} is an alphabetic entity
reference.  It looks for \var{ref} in the instance (or class)
variable \code{entitydefs} which should give the entity's translation.
If a translation is found, it calls the method \code{handle_data()}
with the translation; otherwise, it calls the method
\code{unknown_entityref(\var{ref})}.
\end{funcdesc}

\begin{funcdesc}{handle_data}{data}
This method is called to process arbitrary data.  It is intended to be
overridden by a derived class; the base class implementation does
nothing.
\end{funcdesc}

\begin{funcdesc}{unknown_starttag}{tag\, attributes}
This method is called to process an unknown start tag.  It is intended
to be overridden by a derived class; the base class implementation
does nothing.  The \var{attributes} argument is a list of
(\var{name}, \var{value}) pairs containing the attributes found inside
the tag's \code{<>} brackets.  The \var{name} has been translated to
lower case and double quotes and backslashes in the \var{value} have
been interpreted.  For instance, for the tag
\code{<A HREF="http://www.cwi.nl/">}, this method would be
called as \code{unknown_starttag('a', [('href', 'http://www.cwi.nl/')])}.
\end{funcdesc}

\begin{funcdesc}{unknown_endtag}{tag}
This method is called to process an unknown end tag.  It is intended
to be overridden by a derived class; the base class implementation
does nothing.
\end{funcdesc}

\begin{funcdesc}{unknown_charref}{ref}
This method is called to process an unknown character reference.  It
is intended to be overridden by a derived class; the base class
implementation does nothing.
\end{funcdesc}

\begin{funcdesc}{unknown_entityref}{ref}
This method is called to process an unknown entity reference.  It is
intended to be overridden by a derived class; the base class
implementation does nothing.
\end{funcdesc}

Apart from overriding or extending the methods listed above, derived
classes may also define methods of the following form to define
processing of specific tags.  Tag names in the input stream are case
independent; the \var{tag} occurring in method names must be in lower
case:

\begin{funcdesc}{start_\var{tag}}{attributes}
This method is called to process an opening tag \var{tag}.  It has
preference over \code{do_\var{tag}()}.  The \var{attributes} argument
has the same meaning as described for \code{unknown_tag()} above.
\end{funcdesc}

\begin{funcdesc}{do_\var{tag}}{attributes}
This method is called to process an opening tag \var{tag} that does
not come with a matching closing tag.  The \var{attributes} argument
has the same meaning as described for \code{unknown_tag()} above.
\end{funcdesc}

\begin{funcdesc}{end_\var{tag}}{}
This method is called to process a closing tag \var{tag}.
\end{funcdesc}

Note that the parser maintains a stack of opening tags for which no
matching closing tag has been found yet.  Only tags processed by
\code{start_\var{tag}()} are pushed on this stack.  Definition of a
\code{end_\var{tag}()} method is optional for these tags.  For tags
processed by \code{do_\var{tag}()} or by \code{unknown_tag()}, no
\code{end_\var{tag}()} method must be defined.

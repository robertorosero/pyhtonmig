\chapter{Defining New Types
        \label{defining-new-types}}
\sectionauthor{Michael Hudson}{mwh21@cam.ac.uk}
\sectionauthor{Dave Kuhlman}{dkuhlman@rexx.com}

As mentioned in the last chapter, Python allows the writer of an
extension module to define new types that can be manipulated from
Python code, much like strings and lists in core Python.

This is not hard; the code for all extension types follows a pattern,
but there are some details that you need to understand before you can
get started.

\section{The Basics
    \label{dnt-basics}}

The Python runtime sees all Python objects as variables of type
\ctype{PyObject*}.  A \ctype{PyObject} is not a very magnificent
object - it just contains the refcount and a pointer to the object's
``type object''.  This is where the action is; the type object
determines which (C) functions get called when, for instance, an
attribute gets looked up on an object or it is multiplied by another
object.  I call these C functions ``type methods'' to distinguish them
from things like \code{[].append} (which I will call ``object
methods'' when I get around to them).

So, if you want to define a new object type, you need to create a new
type object.

This sort of thing can only be explained by example, so here's a
minimal, but complete, module that defines a new type:

\begin{verbatim}
#include <Python.h>

staticforward PyTypeObject noddy_NoddyType;

typedef struct {
    PyObject_HEAD
} noddy_NoddyObject;

static PyObject*
noddy_new_noddy(PyObject* self, PyObject* args)
{
    noddy_NoddyObject* noddy;

    if (!PyArg_ParseTuple(args,":new_noddy")) 
        return NULL;

    noddy = PyObject_New(noddy_NoddyObject, &noddy_NoddyType);

    return (PyObject*)noddy;
}

static void
noddy_noddy_dealloc(PyObject* self)
{
    PyObject_Del(self);
}

static PyTypeObject noddy_NoddyType = {
    PyObject_HEAD_INIT(NULL)
    0,
    "Noddy",
    sizeof(noddy_NoddyObject),
    0,
    noddy_noddy_dealloc, /*tp_dealloc*/
    0,          /*tp_print*/
    0,          /*tp_getattr*/
    0,          /*tp_setattr*/
    0,          /*tp_compare*/
    0,          /*tp_repr*/
    0,          /*tp_as_number*/
    0,          /*tp_as_sequence*/
    0,          /*tp_as_mapping*/
    0,          /*tp_hash */
};

static PyMethodDef noddy_methods[] = {
    { "new_noddy", noddy_new_noddy, METH_VARARGS },
    {NULL, NULL}
};

DL_EXPORT(void)
initnoddy(void) 
{
    noddy_NoddyType.ob_type = &PyType_Type;

    Py_InitModule("noddy", noddy_methods);
}
\end{verbatim}

Now that's quite a bit to take in at once, but hopefully bits will
seem familiar from the last chapter.

The first bit that will be new is:

\begin{verbatim}
staticforward PyTypeObject noddy_NoddyType;
\end{verbatim}

This names the type object that will be defining further down in the
file.  It can't be defined here because its definition has to refer to
functions that have no yet been defined, but we need to be able to
refer to it, hence the declaration.

The \code{staticforward} is required to placate various brain dead
compilers.

\begin{verbatim}
typedef struct {
    PyObject_HEAD
} noddy_NoddyObject;
\end{verbatim}

This is what a Noddy object will contain.  In this case nothing more
than every Python object contains - a refcount and a pointer to a type
object.  These are the fields the \code{PyObject_HEAD} macro brings
in.  The reason for the macro is to standardize the layout and to
enable special debugging fields to be brought in debug builds.

For contrast

\begin{verbatim}
typedef struct {
    PyObject_HEAD
    long ob_ival;
} PyIntObject;
\end{verbatim}

is the corresponding definition for standard Python integers.

Next up is:

\begin{verbatim}
static PyObject*
noddy_new_noddy(PyObject* self, PyObject* args)
{
    noddy_NoddyObject* noddy;

    if (!PyArg_ParseTuple(args,":new_noddy")) 
        return NULL;

    noddy = PyObject_New(noddy_NoddyObject, &noddy_NoddyType);

    return (PyObject*)noddy;
}
\end{verbatim}

This is in fact just a regular module function, as described in the
last chapter.  The reason it gets special mention is that this is
where we create our Noddy object.  Defining PyTypeObject structures is
all very well, but if there's no way to actually \emph{create} one
of the wretched things it is not going to do anyone much good.

Almost always, you create objects with a call of the form:

\begin{verbatim}
PyObject_New(<type>, &<type object>);
\end{verbatim}

This allocates the memory and then initializes the object (sets
the reference count to one, makes the \cdata{ob_type} pointer point at
the right place and maybe some other stuff, depending on build options).
You \emph{can} do these steps separately if you have some reason to
--- but at this level we don't bother.

We cast the return value to a \ctype{PyObject*} because that's what
the Python runtime expects.  This is safe because of guarantees about
the layout of structures in the C standard, and is a fairly common C
programming trick.  One could declare \cfunction{noddy_new_noddy} to
return a \ctype{noddy_NoddyObject*} and then put a cast in the
definition of \cdata{noddy_methods} further down the file --- it
doesn't make much difference.

Now a Noddy object doesn't do very much and so doesn't need to
implement many type methods.  One you can't avoid is handling
deallocation, so we find

\begin{verbatim}
static void
noddy_noddy_dealloc(PyObject* self)
{
    PyObject_Del(self);
}
\end{verbatim}

This is so short as to be self explanatory.  This function will be
called when the reference count on a Noddy object reaches \code{0} (or
it is found as part of an unreachable cycle by the cyclic garbage
collector).  \cfunction{PyObject_Del()} is what you call when you want
an object to go away.  If a Noddy object held references to other
Python objects, one would decref them here.

Moving on, we come to the crunch --- the type object.

\begin{verbatim}
static PyTypeObject noddy_NoddyType = {
    PyObject_HEAD_INIT(NULL)
    0,
    "Noddy",
    sizeof(noddy_NoddyObject),
    0,
    noddy_noddy_dealloc, /*tp_dealloc*/
    0,                   /*tp_print*/
    0,                   /*tp_getattr*/
    0,                   /*tp_setattr*/
    0,                   /*tp_compare*/
    0,                   /*tp_repr*/
    0,                   /*tp_as_number*/
    0,                   /*tp_as_sequence*/
    0,                   /*tp_as_mapping*/
    0,                   /*tp_hash */
};
\end{verbatim}

Now if you go and look up the definition of \ctype{PyTypeObject} in
\file{object.h} you'll see that it has many, many more fields that the
definition above.  The remaining fields will be filled with zeros by
the C compiler, and it's common practice to not specify them
explicitly unless you need them.  

This is so important that I'm going to pick the top of it apart still
further:

\begin{verbatim}
    PyObject_HEAD_INIT(NULL)
\end{verbatim}

This line is a bit of a wart; what we'd like to write is:

\begin{verbatim}
    PyObject_HEAD_INIT(&PyType_Type)
\end{verbatim}

as the type of a type object is ``type'', but this isn't strictly
conforming C and some compilers complain.  So instead we fill in the
\cdata{ob_type} field of \cdata{noddy_NoddyType} at the earliest
oppourtunity --- in \cfunction{initnoddy()}.

\begin{verbatim}
    0,
\end{verbatim}

XXX why does the type info struct start PyObject_*VAR*_HEAD??

\begin{verbatim}
    "Noddy",
\end{verbatim}

The name of our type.  This will appear in the default textual
representation of our objects and in some error messages, for example:

\begin{verbatim}
>>> "" + noddy.new_noddy()
Traceback (most recent call last):
  File "<stdin>", line 1, in ?
TypeError: cannot add type "Noddy" to string
\end{verbatim}

\begin{verbatim}
    sizeof(noddy_NoddyObject),
\end{verbatim}

This is so that Python knows how much memory to allocate when you call
\cfunction{PyObject_New}.

\begin{verbatim}
    0,
\end{verbatim}

This has to do with variable length objects like lists and strings.
Ignore for now...

Now we get into the type methods, the things that make your objects
different from the others.  Of course, the Noddy object doesn't
implement many of these, but as mentioned above you have to implement
the deallocation function.

\begin{verbatim}
    noddy_noddy_dealloc, /*tp_dealloc*/
\end{verbatim}

From here, all the type methods are nil so I won't go over them yet -
that's for the next section!

Everything else in the file should be familiar, except for this line
in \cfunction{initnoddy}:

\begin{verbatim}
    noddy_NoddyType.ob_type = &PyType_Type;
\end{verbatim}

This was alluded to above --- the \cdata{noddy_NoddyType} object should
have type ``type'', but \code{\&PyType_Type} is not constant and so
can't be used in its initializer.  To work around this, we patch it up
in the module initialization.

That's it!  All that remains is to build it; put the above code in a
file called \file{noddymodule.c} and

\begin{verbatim}
from distutils.core import setup, Extension
setup(name = "noddy", version = "1.0",
    ext_modules = [Extension("noddy", ["noddymodule.c"])])
\end{verbatim}

in a file called \file{setup.py}; then typing

\begin{verbatim}
$ python setup.py build%$
\end{verbatim}

at a shell should produce a file \file{noddy.so} in a subdirectory;
move to that directory and fire up Python --- you should be able to
\code{import noddy} and play around with Noddy objects.

That wasn't so hard, was it?


\section{Type Methods
         \label{dnt-type-methods}}

This section aims to give a quick fly-by on the various type methods
you can implement and what they do.

Here is the definition of \ctype{PyTypeObject}, with some fields only
used in debug builds omitted:

\begin{verbatim}
typedef struct _typeobject {
    PyObject_VAR_HEAD
    char *tp_name; /* For printing */
    int tp_basicsize, tp_itemsize; /* For allocation */

    /* Methods to implement standard operations */

    destructor tp_dealloc;
    printfunc tp_print;
    getattrfunc tp_getattr;
    setattrfunc tp_setattr;
    cmpfunc tp_compare;
    reprfunc tp_repr;

    /* Method suites for standard classes */

    PyNumberMethods *tp_as_number;
    PySequenceMethods *tp_as_sequence;
    PyMappingMethods *tp_as_mapping;

    /* More standard operations (here for binary compatibility) */

    hashfunc tp_hash;
    ternaryfunc tp_call;
    reprfunc tp_str;
    getattrofunc tp_getattro;
    setattrofunc tp_setattro;

    /* Functions to access object as input/output buffer */
    PyBufferProcs *tp_as_buffer;

    /* Flags to define presence of optional/expanded features */
    long tp_flags;

    char *tp_doc; /* Documentation string */

    /* Assigned meaning in release 2.0 */
    /* call function for all accessible objects */
    traverseproc tp_traverse;

    /* delete references to contained objects */
    inquiry tp_clear;

    /* Assigned meaning in release 2.1 */
    /* rich comparisons */
    richcmpfunc tp_richcompare;

    /* weak reference enabler */
    long tp_weaklistoffset;

    /* Added in release 2.2 */
    /* Iterators */
    getiterfunc tp_iter;
    iternextfunc tp_iternext;

    /* Attribute descriptor and subclassing stuff */
    struct PyMethodDef *tp_methods;
    struct memberlist *tp_members;
    struct getsetlist *tp_getset;
    struct _typeobject *tp_base;
    PyObject *tp_dict;
    descrgetfunc tp_descr_get;
    descrsetfunc tp_descr_set;
    long tp_dictoffset;
    initproc tp_init;
    allocfunc tp_alloc;
    newfunc tp_new;
    destructor tp_free; /* Low-level free-memory routine */
    PyObject *tp_bases;
    PyObject *tp_mro; /* method resolution order */
    PyObject *tp_defined;

} PyTypeObject;
\end{verbatim}

Now that's a \emph{lot} of methods.  Don't worry too much though - if
you have a type you want to define, the chances are very good that you
will only implement a handful of these.

As you probably expect by now, we're going to go over this and give
more information about the various handlers.  We won't go in the order
they are defined in the structure, because there is a lot of
historical baggage that impacts the ordering of the fields; be sure
your type initializaion keeps the fields in the right order!  It's
often easiest to find an example that includes all the fields you need
(even if they're initialized to \code{0}) and then change the values
to suit your new type.

\begin{verbatim}
    char *tp_name; /* For printing */
\end{verbatim}

The name of the type - as mentioned in the last section, this will
appear in various places, almost entirely for diagnostic purposes.
Try to choose something that will be helpful in such a situation!

\begin{verbatim}
    int tp_basicsize, tp_itemsize; /* For allocation */
\end{verbatim}

These fields tell the runtime how much memory to allocate when new
objects of this typed are created.  Python has some builtin support
for variable length structures (think: strings, lists) which is where
the \cdata{tp_itemsize} field comes in.  This will be dealt with
later.

\begin{verbatim}
    char *tp_doc;
\end{verbatim}

Here you can put a string (or its address) that you want returned when
the Python script references \code{obj.__doc__} to retrieve the
docstring.
   
Now we come to the basic type methods---the ones most extension types
will implement.


\subsection{Finalization and De-allocation}

\begin{verbatim}
    destructor tp_dealloc;
\end{verbatim}

This function is called when the reference count of the instance of
your type is reduced to zero and the Python interpreter wants to
reclaim it.  If your type has memory to free or other clean-up to
perform, put it here.  The object itself needs to be freed here as
well.  Here is an example of this function:

\begin{verbatim}
static void
newdatatype_dealloc(newdatatypeobject * obj)
{
    free(obj->obj_UnderlyingDatatypePtr);
    PyObject_DEL(obj);
}
\end{verbatim}


\subsection{Object Representation}

In Python, there are three ways to generate a textual representation
of an object: the \function{repr()}\bifuncindex{repr} function (or
equivalent backtick syntax), the \function{str()}\bifuncindex{str}
function, and the \keyword{print} statement.  For most objects, the
\keyword{print} statement is equivalent to the \function{str()}
function, but it is possible to special-case printing to a
\ctype{FILE*} if necessary; this should only be done if efficiency is
identified as a problem and profiling suggests that creating a
temporary string object to be written to a file is too expensive.

These handlers are all optional, and most types at most need to
implement the \member{tp_str} and \member{tp_repr} handlers.

\begin{verbatim}
    reprfunc tp_repr;
    reprfunc tp_str;
    printfunc tp_print;
\end{verbatim}

The \member{tp_repr} handler should return a string object containing
a representation of the instance for which it is called.  Here is a
simple example:

\begin{verbatim}
static PyObject *
newdatatype_repr(newdatatypeobject * obj)
{
    char buf[4096];
    sprintf(buf, "Repr-ified_newdatatype{{size:%d}}",
            obj->obj_UnderlyingDatatypePtr->size);
    return PyString_FromString(buf);
}
\end{verbatim}

If no \member{tp_repr} handler is specified, the interpreter will
supply a representation that uses the type's \member{tp_name} and a
uniquely-identifying value for the object.

The \member{tp_str} handler is to \function{str()} what the
\member{tp_repr} handler described above is to \function{repr()}; that
is, it is called when Python code calls \function{str()} on an
instance of your object.  It's implementation is very similar to the
\member{tp_repr} function, but the resulting string is intended to be
human consumption.  It \member{tp_str} is not specified, the
\member{tp_repr} handler is used instead.

Here is a simple example:

\begin{verbatim}
static PyObject *
newdatatype_str(newdatatypeobject * obj)
{
    PyObject *pyString;
    char buf[4096];
    sprintf(buf, "Stringified_newdatatype{{size:%d}}",
        obj->obj_UnderlyingDatatypePtr->size
        );
    pyString = PyString_FromString(buf);
    return pyString;
}
\end{verbatim}

The print function will be called whenever Python needs to "print" an
instance of the type.  For example, if 'node' is an instance of type
TreeNode, then the print function is called when Python code calls:

\begin{verbatim}
print node
\end{verbatim}

There is a flags argument and one flag, \constant{Py_PRINT_RAW}, and
it suggests that you print without string quotes and possibly without
interpreting escape sequences.

The print function receives a file object as an argument. You will
likely want to write to that file object.

Here is a sampe print function:

\begin{verbatim}
static int
newdatatype_print(newdatatypeobject *obj, FILE *fp, int flags)
{
    if (flags & Py_PRINT_RAW) {
        fprintf(fp, "<{newdatatype object--size: %d}>",
                obj->obj_UnderlyingDatatypePtr->size);
    }
    else {
        fprintf(fp, "\"<{newdatatype object--size: %d}>\"",
                obj->obj_UnderlyingDatatypePtr->size);
    }
    return 0;
}
\end{verbatim}


\subsection{Attribute Management Functions}

\begin{verbatim}
    getattrfunc tp_getattr;
    setattrfunc tp_setattr;
\end{verbatim}

The \member{tp_getattr} handle is called when the object requires an
attribute look-up.  It is called in the same situations where the
\method{__getattr__()} method of a class would be called.

A likely way to handle this is (1) to implement a set of functions
(such as \cfunction{newdatatype_getSize()} and
\cfunction{newdatatype_setSize()} in the example below), (2) provide a
method table listing these functions, and (3) provide a getattr
function that returns the result of a lookup in that table.

Here is an example:

\begin{verbatim}
static PyMethodDef newdatatype_methods[] = {
    {"getSize", (PyCFunction)newdatatype_getSize, METH_VARARGS},
    {"setSize", (PyCFunction)newdatatype_setSize, METH_VARARGS},
    {NULL,      NULL}           /* sentinel */
};

static PyObject *
newdatatype_getattr(newdatatypeobject *obj, char *name)
{
    return Py_FindMethod(newdatatype_methods, (PyObject *)obj, name);
}
\end{verbatim}

The \member{tp_setattr} handler is called when the
\method{__setattr__()} or \method{__delattr__()} method of a class
instance would be called.  When an attribute should be deleted, the
third parameter will be \NULL.  Here is an example that simply raises
an exception; if this were really all you wanted, the
\member{tp_setattr} handler should be set to \NULL.
   
\begin{verbatim}
static int
newdatatype_setattr(newdatatypeobject *obj, char *name, PyObject *v)
{
    char buf[1024];
    sprintf(buf, "Set attribute not supported for attribute %s", name);
    PyErr_SetString(PyExc_RuntimeError, buf);
    return -1;
}
\end{verbatim}


\subsection{Object Comparison}

\begin{verbatim}
    cmpfunc tp_compare;
\end{verbatim}

The \member{tp_compare} handler is called when comparisons are needed
are the object does not implement the specific rich comparison method
which matches the requested comparison.  (It is always used if defined
and the \cfunction{PyObject_Compare()} or \cfunction{PyObject_Cmp()}
functions are used, or if \function{cmp()} is used from Python.)
It is analogous to the \method{__cmp__()} method.  This function
should return a negative integer if \var{obj1} is less than
\var{obj2}, \code{0} if they are equal, and a positive integer if
\var{obj1} is greater than
\var{obj2}.

Here is a sample implementation:

\begin{verbatim}
static int
newdatatype_compare(newdatatypeobject * obj1, newdatatypeobject * obj2)
{
    long result;
 
    if (obj1->obj_UnderlyingDatatypePtr->size <
        obj2->obj_UnderlyingDatatypePtr->size) {
        result = -1;
    }
    else if (obj1->obj_UnderlyingDatatypePtr->size >
             obj2->obj_UnderlyingDatatypePtr->size) {
        result = 1;
    }
    else {
        result = 0;
    }
    return result;
}
\end{verbatim}


\subsection{Abstract Protocol Support}

\begin{verbatim}
  tp_as_number;
  tp_as_sequence;
  tp_as_mapping;
\end{verbatim}

If you wish your object to be able to act like a number, a sequence,
or a mapping object, then you place the address of a structure that
implements the C type \ctype{PyNumberMethods},
\ctype{PySequenceMethods}, or \ctype{PyMappingMethods}, respectively.
It is up to you to fill in this structure with appropriate values. You
can find examples of the use of each of these in the \file{Objects}
directory of the Python source distribution.


\begin{verbatim}
    hashfunc tp_hash;
\end{verbatim}

This function, if you choose to provide it, should return a hash
number for an instance of your datatype. Here is a moderately
pointless example:

\begin{verbatim}
static long
newdatatype_hash(newdatatypeobject *obj)
{
    long result;
    result = obj->obj_UnderlyingDatatypePtr->size;
    result = result * 3;
    return result;
}
\end{verbatim}

\begin{verbatim}
    ternaryfunc tp_call;
\end{verbatim}

This function is called when an instance of your datatype is "called",
for example, if \code{obj1} is an instance of your datatype and the Python
script contains \code{obj1('hello')}, the \member{tp_call} handler is
invoked.

This function takes three arguments:

\begin{enumerate}
  \item
    \var{arg1} is the instance of the datatype which is the subject of
    the call. If the call is \code{obj1('hello')}, then \var{arg1} is
    \code{obj1}.

  \item
    \var{arg2} is a tuple containing the arguments to the call.  You
    can use \cfunction{PyArg_ParseTuple()} to extract the arguments.

  \item
    \var{arg3} is a dictionary of keyword arguments that were passed.
    If this is non-\NULL{} and you support keyword arguments, use
    \cfunction{PyArg_ParseTupleAndKeywords()} to extract the
    arguments.  If you do not want to support keyword arguments and
    this is non-\NULL, raise a \exception{TypeError} with a message
    saying that keyword arguments are not supported.
\end{enumerate}
       
Here is a desultory example of the implementation of call function.

\begin{verbatim}
/* Implement the call function.
 *    obj1 is the instance receiving the call.
 *    obj2 is a tuple containing the arguments to the call, in this
 *         case 3 strings.
 */
static PyObject *
newdatatype_call(newdatatypeobject *obj, PyObject *args, PyObject *other)
{
    PyObject *result;
    char *arg1;
    char *arg2;
    char *arg3;
    char buf[4096];
    if (!PyArg_ParseTuple(args, "sss:call", &arg1, &arg2, &arg3)) {
        return NULL;
    }
    sprintf(buf,
            "Returning -- value: [%d] arg1: [%s] arg2: [%s] arg3: [%s]\n",
            obj->obj_UnderlyingDatatypePtr->size,
            arg1, arg2, arg3);
    printf(buf);
    return PyString_FromString(buf);
}
\end{verbatim}


\subsection{More Suggestions}

Remember that you can omit most of these functions, in which case you
provide \code{0} as a value.

In the \file{Objects} directory of the Python source distribution,
there is a file \file{xxobject.c}, which is intended to be used as a
template for the implementation of new types.  One useful strategy
for implementing a new type is to copy and rename this file, then
read the instructions at the top of it.

There are type definitions for each of the functions you must
provide.  They are in \file{object.h} in the Python include
directory that comes with the source distribution of Python.

In order to learn how to implement any specific method for your new
datatype, do the following: Download and unpack the Python source
distribution.  Go the the \file{Objects} directory, then search the
C source files for \code{tp_} plus the function you want (for
example, \code{tp_print} or \code{tp_compare}).  You will find
examples of the function you want to implement.

When you need to verify that the type of an object is indeed the
object you are implementing and if you use xxobject.c as an starting
template for your implementation, then there is a macro defined for
this purpose. The macro definition will look something like this:

\begin{verbatim}
#define is_newdatatypeobject(v)  ((v)->ob_type == &Newdatatypetype)
\end{verbatim}

And, a sample of its use might be something like the following:

\begin{verbatim}
    if (!is_newdatatypeobject(objp1) {
        PyErr_SetString(PyExc_TypeError, "arg #1 not a newdatatype");
        return NULL;
    }
\end{verbatim}

%For a reasonably extensive example, from which most of the snippits
%above were taken, see \file{newdatatype.c} and \file{newdatatype.h}.

\documentclass{manual}

% XXX PM explain how to add new types to Python

\title{Extending and Embedding the Python Interpreter}

\author{
	Guido van Rossum \\
	Dept. AA, CWI, P.O. Box 94079 \\
	1090 GB Amsterdam, The Netherlands \\
	E-mail: {\tt guido@cwi.nl}
}

\date{17 March 1995 \\ Release 1.2-proof-2} % XXX update before release!


% Tell \index to actually write the .idx file
\makeindex

\begin{document}

\maketitle

\ifhtml
\chapter*{Front Matter\label{front}}
\fi

\strong{BEOPEN.COM TERMS AND CONDITIONS FOR PYTHON 2.0}

\centerline{\strong{BEOPEN PYTHON OPEN SOURCE LICENSE AGREEMENT VERSION 1}}

\begin{enumerate}

\item
This LICENSE AGREEMENT is between BeOpen.com (``BeOpen''), having an
office at 160 Saratoga Avenue, Santa Clara, CA 95051, and the
Individual or Organization (``Licensee'') accessing and otherwise
using this software in source or binary form and its associated
documentation (``the Software'').

\item
Subject to the terms and conditions of this BeOpen Python License
Agreement, BeOpen hereby grants Licensee a non-exclusive,
royalty-free, world-wide license to reproduce, analyze, test, perform
and/or display publicly, prepare derivative works, distribute, and
otherwise use the Software alone or in any derivative version,
provided, however, that the BeOpen Python License is retained in the
Software, alone or in any derivative version prepared by Licensee.

\item
BeOpen is making the Software available to Licensee on an ``AS IS''
basis.  BEOPEN MAKES NO REPRESENTATIONS OR WARRANTIES, EXPRESS OR
IMPLIED.  BY WAY OF EXAMPLE, BUT NOT LIMITATION, BEOPEN MAKES NO AND
DISCLAIMS ANY REPRESENTATION OR WARRANTY OF MERCHANTABILITY OR FITNESS
FOR ANY PARTICULAR PURPOSE OR THAT THE USE OF THE SOFTWARE WILL NOT
INFRINGE ANY THIRD PARTY RIGHTS.

\item
BEOPEN SHALL NOT BE LIABLE TO LICENSEE OR ANY OTHER USERS OF THE
SOFTWARE FOR ANY INCIDENTAL, SPECIAL, OR CONSEQUENTIAL DAMAGES OR LOSS
AS A RESULT OF USING, MODIFYING OR DISTRIBUTING THE SOFTWARE, OR ANY
DERIVATIVE THEREOF, EVEN IF ADVISED OF THE POSSIBILITY THEREOF.

\item
This License Agreement will automatically terminate upon a material
breach of its terms and conditions.

\item
This License Agreement shall be governed by and interpreted in all
respects by the law of the State of California, excluding conflict of
law provisions.  Nothing in this License Agreement shall be deemed to
create any relationship of agency, partnership, or joint venture
between BeOpen and Licensee.  This License Agreement does not grant
permission to use BeOpen trademarks or trade names in a trademark
sense to endorse or promote products or services of Licensee, or any
third party.  As an exception, the ``BeOpen Python'' logos available
at http://www.pythonlabs.com/logos.html may be used according to the
permissions granted on that web page.

\item
By copying, installing or otherwise using the software, Licensee
agrees to be bound by the terms and conditions of this License
Agreement.
\end{enumerate}


\centerline{\strong{CNRI OPEN SOURCE LICENSE AGREEMENT}}

Python 1.6 is made available subject to the terms and conditions in
CNRI's License Agreement.  This Agreement together with Python 1.6 may
be located on the Internet using the following unique, persistent
identifier (known as a handle): 1895.22/1012.  This Agreement may also
be obtained from a proxy server on the Internet using the following
URL: \url{http://hdl.handle.net/1895.22/1012}.


\centerline{\strong{CWI PERMISSIONS STATEMENT AND DISCLAIMER}}

Copyright \copyright{} 1991 - 1995, Stichting Mathematisch Centrum
Amsterdam, The Netherlands.  All rights reserved.

Permission to use, copy, modify, and distribute this software and its
documentation for any purpose and without fee is hereby granted,
provided that the above copyright notice appear in all copies and that
both that copyright notice and this permission notice appear in
supporting documentation, and that the name of Stichting Mathematisch
Centrum or CWI not be used in advertising or publicity pertaining to
distribution of the software without specific, written prior
permission.

STICHTING MATHEMATISCH CENTRUM DISCLAIMS ALL WARRANTIES WITH REGARD TO
THIS SOFTWARE, INCLUDING ALL IMPLIED WARRANTIES OF MERCHANTABILITY AND
FITNESS, IN NO EVENT SHALL STICHTING MATHEMATISCH CENTRUM BE LIABLE
FOR ANY SPECIAL, INDIRECT OR CONSEQUENTIAL DAMAGES OR ANY DAMAGES
WHATSOEVER RESULTING FROM LOSS OF USE, DATA OR PROFITS, WHETHER IN AN
ACTION OF CONTRACT, NEGLIGENCE OR OTHER TORTIOUS ACTION, ARISING OUT
OF OR IN CONNECTION WITH THE USE OR PERFORMANCE OF THIS SOFTWARE.



\begin{abstract}

\noindent
Python is an interpreted, object-oriented programming language.  This
document describes how to write modules in C or \Cpp{} to extend the
Python interpreter with new modules.  Those modules can define new
functions but also new object types and their methods.  The document
also describes how to embed the Python interpreter in another
application, for use as an extension language.  Finally, it shows how
to compile and link extension modules so that they can be loaded
dynamically (at run time) into the interpreter, if the underlying
operating system supports this feature.

This document assumes basic knowledge about Python.  For an informal
introduction to the language, see the
\citetitle[../tut/tut.html]{Python Tutorial}.  The
\citetitle[../ref/ref.html]{Python Reference Manual} gives a more
formal definition of the language.  The
\citetitle[../lib/lib.html]{Python Library Reference} documents the
existing object types, functions and modules (both built-in and
written in Python) that give the language its wide application range.

For a detailed description of the whole Python/C API, see the separate
\citetitle[../api/api.html]{Python/C API Reference Manual}.

\end{abstract}

\tableofcontents


\chapter{Extending Python with \C{} or \Cpp{} \label{intro}}


It is quite easy to add new built-in modules to Python, if you know
how to program in C.  Such \dfn{extension modules} can do two things
that can't be done directly in Python: they can implement new built-in
object types, and they can call C library functions and system calls.

To support extensions, the Python API (Application Programmers
Interface) defines a set of functions, macros and variables that
provide access to most aspects of the Python run-time system.  The
Python API is incorporated in a C source file by including the header
\code{"Python.h"}.

The compilation of an extension module depends on its intended use as
well as on your system setup; details are given in later chapters.


\section{A Simple Example
         \label{simpleExample}}

Let's create an extension module called \samp{spam} (the favorite food
of Monty Python fans...) and let's say we want to create a Python
interface to the C library function \cfunction{system()}.\footnote{An
interface for this function already exists in the standard module
\module{os} --- it was chosen as a simple and straightfoward example.}
This function takes a null-terminated character string as argument and
returns an integer.  We want this function to be callable from Python
as follows:

\begin{verbatim}
>>> import spam
>>> status = spam.system("ls -l")
\end{verbatim}

Begin by creating a file \file{spammodule.c}.  (Historically, if a
module is called \samp{spam}, the C file containing its implementation
is called \file{spammodule.c}; if the module name is very long, like
\samp{spammify}, the module name can be just \file{spammify.c}.)

The first line of our file can be:

\begin{verbatim}
#include <Python.h>
\end{verbatim}

which pulls in the Python API (you can add a comment describing the
purpose of the module and a copyright notice if you like).

\begin{notice}[warning]
  Since Python may define some pre-processor definitions which affect
  the standard headers on some systems, you \emph{must} include
  \file{Python.h} before any standard headers are included.
\end{notice}

All user-visible symbols defined by \file{Python.h} have a prefix of
\samp{Py} or \samp{PY}, except those defined in standard header files.
For convenience, and since they are used extensively by the Python
interpreter, \code{"Python.h"} includes a few standard header files:
\code{<stdio.h>}, \code{<string.h>}, \code{<errno.h>}, and
\code{<stdlib.h>}.  If the latter header file does not exist on your
system, it declares the functions \cfunction{malloc()},
\cfunction{free()} and \cfunction{realloc()} directly.

The next thing we add to our module file is the C function that will
be called when the Python expression \samp{spam.system(\var{string})}
is evaluated (we'll see shortly how it ends up being called):

\begin{verbatim}
static PyObject *
spam_system(PyObject *self, PyObject *args)
{
    const char *command;
    int sts;

    if (!PyArg_ParseTuple(args, "s", &command))
        return NULL;
    sts = system(command);
    return Py_BuildValue("i", sts);
}
\end{verbatim}

There is a straightforward translation from the argument list in
Python (for example, the single expression \code{"ls -l"}) to the
arguments passed to the C function.  The C function always has two
arguments, conventionally named \var{self} and \var{args}.

The \var{self} argument is only used when the C function implements a
built-in method, not a function. In the example, \var{self} will
always be a \NULL{} pointer, since we are defining a function, not a
method.  (This is done so that the interpreter doesn't have to
understand two different types of C functions.)

The \var{args} argument will be a pointer to a Python tuple object
containing the arguments.  Each item of the tuple corresponds to an
argument in the call's argument list.  The arguments are Python
objects --- in order to do anything with them in our C function we have
to convert them to C values.  The function \cfunction{PyArg_ParseTuple()}
in the Python API checks the argument types and converts them to C
values.  It uses a template string to determine the required types of
the arguments as well as the types of the C variables into which to
store the converted values.  More about this later.

\cfunction{PyArg_ParseTuple()} returns true (nonzero) if all arguments have
the right type and its components have been stored in the variables
whose addresses are passed.  It returns false (zero) if an invalid
argument list was passed.  In the latter case it also raises an
appropriate exception so the calling function can return
\NULL{} immediately (as we saw in the example).


\section{Intermezzo: Errors and Exceptions
         \label{errors}}

An important convention throughout the Python interpreter is the
following: when a function fails, it should set an exception condition
and return an error value (usually a \NULL{} pointer).  Exceptions
are stored in a static global variable inside the interpreter; if this
variable is \NULL{} no exception has occurred.  A second global
variable stores the ``associated value'' of the exception (the second
argument to \keyword{raise}).  A third variable contains the stack
traceback in case the error originated in Python code.  These three
variables are the C equivalents of the Python variables
\code{sys.exc_type}, \code{sys.exc_value} and \code{sys.exc_traceback} (see
the section on module \module{sys} in the
\citetitle[../lib/lib.html]{Python Library Reference}).  It is
important to know about them to understand how errors are passed
around.

The Python API defines a number of functions to set various types of
exceptions.

The most common one is \cfunction{PyErr_SetString()}.  Its arguments
are an exception object and a C string.  The exception object is
usually a predefined object like \cdata{PyExc_ZeroDivisionError}.  The
C string indicates the cause of the error and is converted to a
Python string object and stored as the ``associated value'' of the
exception.

Another useful function is \cfunction{PyErr_SetFromErrno()}, which only
takes an exception argument and constructs the associated value by
inspection of the global variable \cdata{errno}.  The most
general function is \cfunction{PyErr_SetObject()}, which takes two object
arguments, the exception and its associated value.  You don't need to
\cfunction{Py_INCREF()} the objects passed to any of these functions.

You can test non-destructively whether an exception has been set with
\cfunction{PyErr_Occurred()}.  This returns the current exception object,
or \NULL{} if no exception has occurred.  You normally don't need
to call \cfunction{PyErr_Occurred()} to see whether an error occurred in a
function call, since you should be able to tell from the return value.

When a function \var{f} that calls another function \var{g} detects
that the latter fails, \var{f} should itself return an error value
(usually \NULL{} or \code{-1}).  It should \emph{not} call one of the
\cfunction{PyErr_*()} functions --- one has already been called by \var{g}.
\var{f}'s caller is then supposed to also return an error indication
to \emph{its} caller, again \emph{without} calling \cfunction{PyErr_*()},
and so on --- the most detailed cause of the error was already
reported by the function that first detected it.  Once the error
reaches the Python interpreter's main loop, this aborts the currently
executing Python code and tries to find an exception handler specified
by the Python programmer.

(There are situations where a module can actually give a more detailed
error message by calling another \cfunction{PyErr_*()} function, and in
such cases it is fine to do so.  As a general rule, however, this is
not necessary, and can cause information about the cause of the error
to be lost: most operations can fail for a variety of reasons.)

To ignore an exception set by a function call that failed, the exception
condition must be cleared explicitly by calling \cfunction{PyErr_Clear()}. 
The only time C code should call \cfunction{PyErr_Clear()} is if it doesn't
want to pass the error on to the interpreter but wants to handle it
completely by itself (possibly by trying something else, or pretending
nothing went wrong).

Every failing \cfunction{malloc()} call must be turned into an
exception --- the direct caller of \cfunction{malloc()} (or
\cfunction{realloc()}) must call \cfunction{PyErr_NoMemory()} and
return a failure indicator itself.  All the object-creating functions
(for example, \cfunction{PyInt_FromLong()}) already do this, so this
note is only relevant to those who call \cfunction{malloc()} directly.

Also note that, with the important exception of
\cfunction{PyArg_ParseTuple()} and friends, functions that return an
integer status usually return a positive value or zero for success and
\code{-1} for failure, like \UNIX{} system calls.

Finally, be careful to clean up garbage (by making
\cfunction{Py_XDECREF()} or \cfunction{Py_DECREF()} calls for objects
you have already created) when you return an error indicator!

The choice of which exception to raise is entirely yours.  There are
predeclared C objects corresponding to all built-in Python exceptions,
such as \cdata{PyExc_ZeroDivisionError}, which you can use directly.
Of course, you should choose exceptions wisely --- don't use
\cdata{PyExc_TypeError} to mean that a file couldn't be opened (that
should probably be \cdata{PyExc_IOError}).  If something's wrong with
the argument list, the \cfunction{PyArg_ParseTuple()} function usually
raises \cdata{PyExc_TypeError}.  If you have an argument whose value
must be in a particular range or must satisfy other conditions,
\cdata{PyExc_ValueError} is appropriate.

You can also define a new exception that is unique to your module.
For this, you usually declare a static object variable at the
beginning of your file:

\begin{verbatim}
static PyObject *SpamError;
\end{verbatim}

and initialize it in your module's initialization function
(\cfunction{initspam()}) with an exception object (leaving out
the error checking for now):

\begin{verbatim}
PyMODINIT_FUNC
initspam(void)
{
    PyObject *m;

    m = Py_InitModule("spam", SpamMethods);

    SpamError = PyErr_NewException("spam.error", NULL, NULL);
    Py_INCREF(SpamError);
    PyModule_AddObject(m, "error", SpamError);
}
\end{verbatim}

Note that the Python name for the exception object is
\exception{spam.error}.  The \cfunction{PyErr_NewException()} function
may create a class with the base class being \exception{Exception}
(unless another class is passed in instead of \NULL), described in the
\citetitle[../lib/lib.html]{Python Library Reference} under ``Built-in
Exceptions.''

Note also that the \cdata{SpamError} variable retains a reference to
the newly created exception class; this is intentional!  Since the
exception could be removed from the module by external code, an owned
reference to the class is needed to ensure that it will not be
discarded, causing \cdata{SpamError} to become a dangling pointer.
Should it become a dangling pointer, C code which raises the exception
could cause a core dump or other unintended side effects.

We discuss the use of PyMODINIT_FUNC as a function return type later in this
sample.

\section{Back to the Example
         \label{backToExample}}

Going back to our example function, you should now be able to
understand this statement:

\begin{verbatim}
    if (!PyArg_ParseTuple(args, "s", &command))
        return NULL;
\end{verbatim}

It returns \NULL{} (the error indicator for functions returning
object pointers) if an error is detected in the argument list, relying
on the exception set by \cfunction{PyArg_ParseTuple()}.  Otherwise the
string value of the argument has been copied to the local variable
\cdata{command}.  This is a pointer assignment and you are not supposed
to modify the string to which it points (so in Standard C, the variable
\cdata{command} should properly be declared as \samp{const char
*command}).

The next statement is a call to the \UNIX{} function
\cfunction{system()}, passing it the string we just got from
\cfunction{PyArg_ParseTuple()}:

\begin{verbatim}
    sts = system(command);
\end{verbatim}

Our \function{spam.system()} function must return the value of
\cdata{sts} as a Python object.  This is done using the function
\cfunction{Py_BuildValue()}, which is something like the inverse of
\cfunction{PyArg_ParseTuple()}: it takes a format string and an
arbitrary number of C values, and returns a new Python object.
More info on \cfunction{Py_BuildValue()} is given later.

\begin{verbatim}
    return Py_BuildValue("i", sts);
\end{verbatim}

In this case, it will return an integer object.  (Yes, even integers
are objects on the heap in Python!)

If you have a C function that returns no useful argument (a function
returning \ctype{void}), the corresponding Python function must return
\code{None}.   You need this idiom to do so (which is implemented by the
\csimplemacro{Py_RETURN_NONE} macro):

\begin{verbatim}
    Py_INCREF(Py_None);
    return Py_None;
\end{verbatim}

\cdata{Py_None} is the C name for the special Python object
\code{None}.  It is a genuine Python object rather than a \NULL{}
pointer, which means ``error'' in most contexts, as we have seen.


\section{The Module's Method Table and Initialization Function
         \label{methodTable}}

I promised to show how \cfunction{spam_system()} is called from Python
programs.  First, we need to list its name and address in a ``method
table'':

\begin{verbatim}
static PyMethodDef SpamMethods[] = {
    ...
    {"system",  spam_system, METH_VARARGS,
     "Execute a shell command."},
    ...
    {NULL, NULL, 0, NULL}        /* Sentinel */
};
\end{verbatim}

Note the third entry (\samp{METH_VARARGS}).  This is a flag telling
the interpreter the calling convention to be used for the C
function.  It should normally always be \samp{METH_VARARGS} or
\samp{METH_VARARGS | METH_KEYWORDS}; a value of \code{0} means that an
obsolete variant of \cfunction{PyArg_ParseTuple()} is used.

When using only \samp{METH_VARARGS}, the function should expect
the Python-level parameters to be passed in as a tuple acceptable for
parsing via \cfunction{PyArg_ParseTuple()}; more information on this
function is provided below.

The \constant{METH_KEYWORDS} bit may be set in the third field if
keyword arguments should be passed to the function.  In this case, the
C function should accept a third \samp{PyObject *} parameter which
will be a dictionary of keywords.  Use
\cfunction{PyArg_ParseTupleAndKeywords()} to parse the arguments to
such a function.

The method table must be passed to the interpreter in the module's
initialization function.  The initialization function must be named
\cfunction{init\var{name}()}, where \var{name} is the name of the
module, and should be the only non-\keyword{static} item defined in
the module file:

\begin{verbatim}
PyMODINIT_FUNC
initspam(void)
{
    (void) Py_InitModule("spam", SpamMethods);
}
\end{verbatim}

Note that PyMODINIT_FUNC declares the function as \code{void} return type, 
declares any special linkage declarations required by the platform, and for 
\Cpp{} declares the function as \code{extern "C"}.

When the Python program imports module \module{spam} for the first
time, \cfunction{initspam()} is called. (See below for comments about
embedding Python.)  It calls
\cfunction{Py_InitModule()}, which creates a ``module object'' (which
is inserted in the dictionary \code{sys.modules} under the key
\code{"spam"}), and inserts built-in function objects into the newly
created module based upon the table (an array of \ctype{PyMethodDef}
structures) that was passed as its second argument.
\cfunction{Py_InitModule()} returns a pointer to the module object
that it creates (which is unused here).  It aborts with a fatal error
if the module could not be initialized satisfactorily, so the caller
doesn't need to check for errors.

When embedding Python, the \cfunction{initspam()} function is not
called automatically unless there's an entry in the
\cdata{_PyImport_Inittab} table.  The easiest way to handle this is to 
statically initialize your statically-linked modules by directly
calling \cfunction{initspam()} after the call to
\cfunction{Py_Initialize()} or \cfunction{PyMac_Initialize()}:

\begin{verbatim}
int
main(int argc, char *argv[])
{
    /* Pass argv[0] to the Python interpreter */
    Py_SetProgramName(argv[0]);

    /* Initialize the Python interpreter.  Required. */
    Py_Initialize();

    /* Add a static module */
    initspam();
\end{verbatim}

An example may be found in the file \file{Demo/embed/demo.c} in the
Python source distribution.

\note{Removing entries from \code{sys.modules} or importing
compiled modules into multiple interpreters within a process (or
following a \cfunction{fork()} without an intervening
\cfunction{exec()}) can create problems for some extension modules.
Extension module authors should exercise caution when initializing
internal data structures.
Note also that the \function{reload()} function can be used with
extension modules, and will call the module initialization function
(\cfunction{initspam()} in the example), but will not load the module
again if it was loaded from a dynamically loadable object file
(\file{.so} on \UNIX, \file{.dll} on Windows).}

A more substantial example module is included in the Python source
distribution as \file{Modules/xxmodule.c}.  This file may be used as a 
template or simply read as an example.  The \program{modulator.py}
script included in the source distribution or Windows install provides 
a simple graphical user interface for declaring the functions and
objects which a module should implement, and can generate a template
which can be filled in.  The script lives in the
\file{Tools/modulator/} directory; see the \file{README} file there
for more information.


\section{Compilation and Linkage
         \label{compilation}}

There are two more things to do before you can use your new extension:
compiling and linking it with the Python system.  If you use dynamic
loading, the details may depend on the style of dynamic loading your
system uses; see the chapters about building extension modules
(chapter \ref{building}) and additional information that pertains only
to building on Windows (chapter \ref{building-on-windows}) for more
information about this.
% XXX Add information about Mac OS

If you can't use dynamic loading, or if you want to make your module a
permanent part of the Python interpreter, you will have to change the
configuration setup and rebuild the interpreter.  Luckily, this is
very simple on \UNIX: just place your file (\file{spammodule.c} for
example) in the \file{Modules/} directory of an unpacked source
distribution, add a line to the file \file{Modules/Setup.local}
describing your file:

\begin{verbatim}
spam spammodule.o
\end{verbatim}

and rebuild the interpreter by running \program{make} in the toplevel
directory.  You can also run \program{make} in the \file{Modules/}
subdirectory, but then you must first rebuild \file{Makefile}
there by running `\program{make} Makefile'.  (This is necessary each
time you change the \file{Setup} file.)

If your module requires additional libraries to link with, these can
be listed on the line in the configuration file as well, for instance:

\begin{verbatim}
spam spammodule.o -lX11
\end{verbatim}

\section{Calling Python Functions from C
         \label{callingPython}}

So far we have concentrated on making C functions callable from
Python.  The reverse is also useful: calling Python functions from C.
This is especially the case for libraries that support so-called
``callback'' functions.  If a C interface makes use of callbacks, the
equivalent Python often needs to provide a callback mechanism to the
Python programmer; the implementation will require calling the Python
callback functions from a C callback.  Other uses are also imaginable.

Fortunately, the Python interpreter is easily called recursively, and
there is a standard interface to call a Python function.  (I won't
dwell on how to call the Python parser with a particular string as
input --- if you're interested, have a look at the implementation of
the \programopt{-c} command line option in \file{Python/pythonmain.c}
from the Python source code.)

Calling a Python function is easy.  First, the Python program must
somehow pass you the Python function object.  You should provide a
function (or some other interface) to do this.  When this function is
called, save a pointer to the Python function object (be careful to
\cfunction{Py_INCREF()} it!) in a global variable --- or wherever you
see fit. For example, the following function might be part of a module
definition:

\begin{verbatim}
static PyObject *my_callback = NULL;

static PyObject *
my_set_callback(PyObject *dummy, PyObject *args)
{
    PyObject *result = NULL;
    PyObject *temp;

    if (PyArg_ParseTuple(args, "O:set_callback", &temp)) {
        if (!PyCallable_Check(temp)) {
            PyErr_SetString(PyExc_TypeError, "parameter must be callable");
            return NULL;
        }
        Py_XINCREF(temp);         /* Add a reference to new callback */
        Py_XDECREF(my_callback);  /* Dispose of previous callback */
        my_callback = temp;       /* Remember new callback */
        /* Boilerplate to return "None" */
        Py_INCREF(Py_None);
        result = Py_None;
    }
    return result;
}
\end{verbatim}

This function must be registered with the interpreter using the
\constant{METH_VARARGS} flag; this is described in section
\ref{methodTable}, ``The Module's Method Table and Initialization
Function.''  The \cfunction{PyArg_ParseTuple()} function and its
arguments are documented in section~\ref{parseTuple}, ``Extracting
Parameters in Extension Functions.''

The macros \cfunction{Py_XINCREF()} and \cfunction{Py_XDECREF()}
increment/decrement the reference count of an object and are safe in
the presence of \NULL{} pointers (but note that \var{temp} will not be 
\NULL{} in this context).  More info on them in
section~\ref{refcounts}, ``Reference Counts.''

Later, when it is time to call the function, you call the C function
\cfunction{PyEval_CallObject()}.\ttindex{PyEval_CallObject()}  This
function has two arguments, both pointers to arbitrary Python objects:
the Python function, and the argument list.  The argument list must
always be a tuple object, whose length is the number of arguments.  To
call the Python function with no arguments, pass an empty tuple; to
call it with one argument, pass a singleton tuple.
\cfunction{Py_BuildValue()} returns a tuple when its format string
consists of zero or more format codes between parentheses.  For
example:

\begin{verbatim}
    int arg;
    PyObject *arglist;
    PyObject *result;
    ...
    arg = 123;
    ...
    /* Time to call the callback */
    arglist = Py_BuildValue("(i)", arg);
    result = PyEval_CallObject(my_callback, arglist);
    Py_DECREF(arglist);
\end{verbatim}

\cfunction{PyEval_CallObject()} returns a Python object pointer: this is
the return value of the Python function.  \cfunction{PyEval_CallObject()} is
``reference-count-neutral'' with respect to its arguments.  In the
example a new tuple was created to serve as the argument list, which
is \cfunction{Py_DECREF()}-ed immediately after the call.

The return value of \cfunction{PyEval_CallObject()} is ``new'': either it
is a brand new object, or it is an existing object whose reference
count has been incremented.  So, unless you want to save it in a
global variable, you should somehow \cfunction{Py_DECREF()} the result,
even (especially!) if you are not interested in its value.

Before you do this, however, it is important to check that the return
value isn't \NULL.  If it is, the Python function terminated by
raising an exception.  If the C code that called
\cfunction{PyEval_CallObject()} is called from Python, it should now
return an error indication to its Python caller, so the interpreter
can print a stack trace, or the calling Python code can handle the
exception.  If this is not possible or desirable, the exception should
be cleared by calling \cfunction{PyErr_Clear()}.  For example:

\begin{verbatim}
    if (result == NULL)
        return NULL; /* Pass error back */
    ...use result...
    Py_DECREF(result); 
\end{verbatim}

Depending on the desired interface to the Python callback function,
you may also have to provide an argument list to
\cfunction{PyEval_CallObject()}.  In some cases the argument list is
also provided by the Python program, through the same interface that
specified the callback function.  It can then be saved and used in the
same manner as the function object.  In other cases, you may have to
construct a new tuple to pass as the argument list.  The simplest way
to do this is to call \cfunction{Py_BuildValue()}.  For example, if
you want to pass an integral event code, you might use the following
code:

\begin{verbatim}
    PyObject *arglist;
    ...
    arglist = Py_BuildValue("(l)", eventcode);
    result = PyEval_CallObject(my_callback, arglist);
    Py_DECREF(arglist);
    if (result == NULL)
        return NULL; /* Pass error back */
    /* Here maybe use the result */
    Py_DECREF(result);
\end{verbatim}

Note the placement of \samp{Py_DECREF(arglist)} immediately after the
call, before the error check!  Also note that strictly spoken this
code is not complete: \cfunction{Py_BuildValue()} may run out of
memory, and this should be checked.


\section{Extracting Parameters in Extension Functions
         \label{parseTuple}}

\ttindex{PyArg_ParseTuple()}

The \cfunction{PyArg_ParseTuple()} function is declared as follows:

\begin{verbatim}
int PyArg_ParseTuple(PyObject *arg, char *format, ...);
\end{verbatim}

The \var{arg} argument must be a tuple object containing an argument
list passed from Python to a C function.  The \var{format} argument
must be a format string, whose syntax is explained in
``\ulink{Parsing arguments and building
values}{../api/arg-parsing.html}'' in the
\citetitle[../api/api.html]{Python/C API Reference Manual}.  The
remaining arguments must be addresses of variables whose type is
determined by the format string.

Note that while \cfunction{PyArg_ParseTuple()} checks that the Python
arguments have the required types, it cannot check the validity of the
addresses of C variables passed to the call: if you make mistakes
there, your code will probably crash or at least overwrite random bits
in memory.  So be careful!

Note that any Python object references which are provided to the
caller are \emph{borrowed} references; do not decrement their
reference count!

Some example calls:

\begin{verbatim}
    int ok;
    int i, j;
    long k, l;
    const char *s;
    int size;

    ok = PyArg_ParseTuple(args, ""); /* No arguments */
        /* Python call: f() */
\end{verbatim}

\begin{verbatim}
    ok = PyArg_ParseTuple(args, "s", &s); /* A string */
        /* Possible Python call: f('whoops!') */
\end{verbatim}

\begin{verbatim}
    ok = PyArg_ParseTuple(args, "lls", &k, &l, &s); /* Two longs and a string */
        /* Possible Python call: f(1, 2, 'three') */
\end{verbatim}

\begin{verbatim}
    ok = PyArg_ParseTuple(args, "(ii)s#", &i, &j, &s, &size);
        /* A pair of ints and a string, whose size is also returned */
        /* Possible Python call: f((1, 2), 'three') */
\end{verbatim}

\begin{verbatim}
    {
        const char *file;
        const char *mode = "r";
        int bufsize = 0;
        ok = PyArg_ParseTuple(args, "s|si", &file, &mode, &bufsize);
        /* A string, and optionally another string and an integer */
        /* Possible Python calls:
           f('spam')
           f('spam', 'w')
           f('spam', 'wb', 100000) */
    }
\end{verbatim}

\begin{verbatim}
    {
        int left, top, right, bottom, h, v;
        ok = PyArg_ParseTuple(args, "((ii)(ii))(ii)",
                 &left, &top, &right, &bottom, &h, &v);
        /* A rectangle and a point */
        /* Possible Python call:
           f(((0, 0), (400, 300)), (10, 10)) */
    }
\end{verbatim}

\begin{verbatim}
    {
        Py_complex c;
        ok = PyArg_ParseTuple(args, "D:myfunction", &c);
        /* a complex, also providing a function name for errors */
        /* Possible Python call: myfunction(1+2j) */
    }
\end{verbatim}


\section{Keyword Parameters for Extension Functions
         \label{parseTupleAndKeywords}}

\ttindex{PyArg_ParseTupleAndKeywords()}

The \cfunction{PyArg_ParseTupleAndKeywords()} function is declared as
follows:

\begin{verbatim}
int PyArg_ParseTupleAndKeywords(PyObject *arg, PyObject *kwdict,
                                char *format, char *kwlist[], ...);
\end{verbatim}

The \var{arg} and \var{format} parameters are identical to those of the
\cfunction{PyArg_ParseTuple()} function.  The \var{kwdict} parameter
is the dictionary of keywords received as the third parameter from the
Python runtime.  The \var{kwlist} parameter is a \NULL-terminated
list of strings which identify the parameters; the names are matched
with the type information from \var{format} from left to right.  On
success, \cfunction{PyArg_ParseTupleAndKeywords()} returns true,
otherwise it returns false and raises an appropriate exception.

\note{Nested tuples cannot be parsed when using keyword
arguments!  Keyword parameters passed in which are not present in the
\var{kwlist} will cause \exception{TypeError} to be raised.}

Here is an example module which uses keywords, based on an example by
Geoff Philbrick (\email{philbrick@hks.com}):%
\index{Philbrick, Geoff}

\begin{verbatim}
#include "Python.h"

static PyObject *
keywdarg_parrot(PyObject *self, PyObject *args, PyObject *keywds)
{  
    int voltage;
    char *state = "a stiff";
    char *action = "voom";
    char *type = "Norwegian Blue";

    static char *kwlist[] = {"voltage", "state", "action", "type", NULL};

    if (!PyArg_ParseTupleAndKeywords(args, keywds, "i|sss", kwlist, 
                                     &voltage, &state, &action, &type))
        return NULL; 
  
    printf("-- This parrot wouldn't %s if you put %i Volts through it.\n", 
           action, voltage);
    printf("-- Lovely plumage, the %s -- It's %s!\n", type, state);

    Py_INCREF(Py_None);

    return Py_None;
}

static PyMethodDef keywdarg_methods[] = {
    /* The cast of the function is necessary since PyCFunction values
     * only take two PyObject* parameters, and keywdarg_parrot() takes
     * three.
     */
    {"parrot", (PyCFunction)keywdarg_parrot, METH_VARARGS | METH_KEYWORDS,
     "Print a lovely skit to standard output."},
    {NULL, NULL, 0, NULL}   /* sentinel */
};
\end{verbatim}

\begin{verbatim}
void
initkeywdarg(void)
{
  /* Create the module and add the functions */
  Py_InitModule("keywdarg", keywdarg_methods);
}
\end{verbatim}


\section{Building Arbitrary Values
         \label{buildValue}}

This function is the counterpart to \cfunction{PyArg_ParseTuple()}.  It is
declared as follows:

\begin{verbatim}
PyObject *Py_BuildValue(char *format, ...);
\end{verbatim}

It recognizes a set of format units similar to the ones recognized by
\cfunction{PyArg_ParseTuple()}, but the arguments (which are input to the
function, not output) must not be pointers, just values.  It returns a
new Python object, suitable for returning from a C function called
from Python.

One difference with \cfunction{PyArg_ParseTuple()}: while the latter
requires its first argument to be a tuple (since Python argument lists
are always represented as tuples internally),
\cfunction{Py_BuildValue()} does not always build a tuple.  It builds
a tuple only if its format string contains two or more format units.
If the format string is empty, it returns \code{None}; if it contains
exactly one format unit, it returns whatever object is described by
that format unit.  To force it to return a tuple of size 0 or one,
parenthesize the format string.

Examples (to the left the call, to the right the resulting Python value):

\begin{verbatim}
    Py_BuildValue("")                        None
    Py_BuildValue("i", 123)                  123
    Py_BuildValue("iii", 123, 456, 789)      (123, 456, 789)
    Py_BuildValue("s", "hello")              'hello'
    Py_BuildValue("ss", "hello", "world")    ('hello', 'world')
    Py_BuildValue("s#", "hello", 4)          'hell'
    Py_BuildValue("()")                      ()
    Py_BuildValue("(i)", 123)                (123,)
    Py_BuildValue("(ii)", 123, 456)          (123, 456)
    Py_BuildValue("(i,i)", 123, 456)         (123, 456)
    Py_BuildValue("[i,i]", 123, 456)         [123, 456]
    Py_BuildValue("{s:i,s:i}",
                  "abc", 123, "def", 456)    {'abc': 123, 'def': 456}
    Py_BuildValue("((ii)(ii)) (ii)",
                  1, 2, 3, 4, 5, 6)          (((1, 2), (3, 4)), (5, 6))
\end{verbatim}


\section{Reference Counts
         \label{refcounts}}

In languages like C or \Cpp, the programmer is responsible for
dynamic allocation and deallocation of memory on the heap.  In C,
this is done using the functions \cfunction{malloc()} and
\cfunction{free()}.  In \Cpp, the operators \keyword{new} and
\keyword{delete} are used with essentially the same meaning and
we'll restrict the following discussion to the C case.

Every block of memory allocated with \cfunction{malloc()} should
eventually be returned to the pool of available memory by exactly one
call to \cfunction{free()}.  It is important to call
\cfunction{free()} at the right time.  If a block's address is
forgotten but \cfunction{free()} is not called for it, the memory it
occupies cannot be reused until the program terminates.  This is
called a \dfn{memory leak}.  On the other hand, if a program calls
\cfunction{free()} for a block and then continues to use the block, it
creates a conflict with re-use of the block through another
\cfunction{malloc()} call.  This is called \dfn{using freed memory}.
It has the same bad consequences as referencing uninitialized data ---
core dumps, wrong results, mysterious crashes.

Common causes of memory leaks are unusual paths through the code.  For
instance, a function may allocate a block of memory, do some
calculation, and then free the block again.  Now a change in the
requirements for the function may add a test to the calculation that
detects an error condition and can return prematurely from the
function.  It's easy to forget to free the allocated memory block when
taking this premature exit, especially when it is added later to the
code.  Such leaks, once introduced, often go undetected for a long
time: the error exit is taken only in a small fraction of all calls,
and most modern machines have plenty of virtual memory, so the leak
only becomes apparent in a long-running process that uses the leaking
function frequently.  Therefore, it's important to prevent leaks from
happening by having a coding convention or strategy that minimizes
this kind of errors.

Since Python makes heavy use of \cfunction{malloc()} and
\cfunction{free()}, it needs a strategy to avoid memory leaks as well
as the use of freed memory.  The chosen method is called
\dfn{reference counting}.  The principle is simple: every object
contains a counter, which is incremented when a reference to the
object is stored somewhere, and which is decremented when a reference
to it is deleted.  When the counter reaches zero, the last reference
to the object has been deleted and the object is freed.

An alternative strategy is called \dfn{automatic garbage collection}.
(Sometimes, reference counting is also referred to as a garbage
collection strategy, hence my use of ``automatic'' to distinguish the
two.)  The big advantage of automatic garbage collection is that the
user doesn't need to call \cfunction{free()} explicitly.  (Another claimed
advantage is an improvement in speed or memory usage --- this is no
hard fact however.)  The disadvantage is that for C, there is no
truly portable automatic garbage collector, while reference counting
can be implemented portably (as long as the functions \cfunction{malloc()}
and \cfunction{free()} are available --- which the C Standard guarantees).
Maybe some day a sufficiently portable automatic garbage collector
will be available for C.  Until then, we'll have to live with
reference counts.

While Python uses the traditional reference counting implementation,
it also offers a cycle detector that works to detect reference
cycles.  This allows applications to not worry about creating direct
or indirect circular references; these are the weakness of garbage
collection implemented using only reference counting.  Reference
cycles consist of objects which contain (possibly indirect) references
to themselves, so that each object in the cycle has a reference count
which is non-zero.  Typical reference counting implementations are not
able to reclaim the memory belonging to any objects in a reference
cycle, or referenced from the objects in the cycle, even though there
are no further references to the cycle itself.

The cycle detector is able to detect garbage cycles and can reclaim
them so long as there are no finalizers implemented in Python
(\method{__del__()} methods).  When there are such finalizers, the
detector exposes the cycles through the \ulink{\module{gc}
module}{../lib/module-gc.html} (specifically, the \code{garbage}
variable in that module).  The \module{gc} module also exposes a way
to run the detector (the \function{collect()} function), as well as
configuration interfaces and the ability to disable the detector at
runtime.  The cycle detector is considered an optional component;
though it is included by default, it can be disabled at build time
using the \longprogramopt{without-cycle-gc} option to the
\program{configure} script on \UNIX{} platforms (including Mac OS X)
or by removing the definition of \code{WITH_CYCLE_GC} in the
\file{pyconfig.h} header on other platforms.  If the cycle detector is
disabled in this way, the \module{gc} module will not be available.


\subsection{Reference Counting in Python
            \label{refcountsInPython}}

There are two macros, \code{Py_INCREF(x)} and \code{Py_DECREF(x)},
which handle the incrementing and decrementing of the reference count.
\cfunction{Py_DECREF()} also frees the object when the count reaches zero.
For flexibility, it doesn't call \cfunction{free()} directly --- rather, it
makes a call through a function pointer in the object's \dfn{type
object}.  For this purpose (and others), every object also contains a
pointer to its type object.

The big question now remains: when to use \code{Py_INCREF(x)} and
\code{Py_DECREF(x)}?  Let's first introduce some terms.  Nobody
``owns'' an object; however, you can \dfn{own a reference} to an
object.  An object's reference count is now defined as the number of
owned references to it.  The owner of a reference is responsible for
calling \cfunction{Py_DECREF()} when the reference is no longer
needed.  Ownership of a reference can be transferred.  There are three
ways to dispose of an owned reference: pass it on, store it, or call
\cfunction{Py_DECREF()}.  Forgetting to dispose of an owned reference
creates a memory leak.

It is also possible to \dfn{borrow}\footnote{The metaphor of
``borrowing'' a reference is not completely correct: the owner still
has a copy of the reference.} a reference to an object.  The borrower
of a reference should not call \cfunction{Py_DECREF()}.  The borrower must
not hold on to the object longer than the owner from which it was
borrowed.  Using a borrowed reference after the owner has disposed of
it risks using freed memory and should be avoided
completely.\footnote{Checking that the reference count is at least 1
\strong{does not work} --- the reference count itself could be in
freed memory and may thus be reused for another object!}

The advantage of borrowing over owning a reference is that you don't
need to take care of disposing of the reference on all possible paths
through the code --- in other words, with a borrowed reference you
don't run the risk of leaking when a premature exit is taken.  The
disadvantage of borrowing over leaking is that there are some subtle
situations where in seemingly correct code a borrowed reference can be
used after the owner from which it was borrowed has in fact disposed
of it.

A borrowed reference can be changed into an owned reference by calling
\cfunction{Py_INCREF()}.  This does not affect the status of the owner from
which the reference was borrowed --- it creates a new owned reference,
and gives full owner responsibilities (the new owner must
dispose of the reference properly, as well as the previous owner).


\subsection{Ownership Rules
            \label{ownershipRules}}

Whenever an object reference is passed into or out of a function, it
is part of the function's interface specification whether ownership is
transferred with the reference or not.

Most functions that return a reference to an object pass on ownership
with the reference.  In particular, all functions whose function it is
to create a new object, such as \cfunction{PyInt_FromLong()} and
\cfunction{Py_BuildValue()}, pass ownership to the receiver.  Even if
the object is not actually new, you still receive ownership of a new
reference to that object.  For instance, \cfunction{PyInt_FromLong()}
maintains a cache of popular values and can return a reference to a
cached item.

Many functions that extract objects from other objects also transfer
ownership with the reference, for instance
\cfunction{PyObject_GetAttrString()}.  The picture is less clear, here,
however, since a few common routines are exceptions:
\cfunction{PyTuple_GetItem()}, \cfunction{PyList_GetItem()},
\cfunction{PyDict_GetItem()}, and \cfunction{PyDict_GetItemString()}
all return references that you borrow from the tuple, list or
dictionary.

The function \cfunction{PyImport_AddModule()} also returns a borrowed
reference, even though it may actually create the object it returns:
this is possible because an owned reference to the object is stored in
\code{sys.modules}.

When you pass an object reference into another function, in general,
the function borrows the reference from you --- if it needs to store
it, it will use \cfunction{Py_INCREF()} to become an independent
owner.  There are exactly two important exceptions to this rule:
\cfunction{PyTuple_SetItem()} and \cfunction{PyList_SetItem()}.  These
functions take over ownership of the item passed to them --- even if
they fail!  (Note that \cfunction{PyDict_SetItem()} and friends don't
take over ownership --- they are ``normal.'')

When a C function is called from Python, it borrows references to its
arguments from the caller.  The caller owns a reference to the object,
so the borrowed reference's lifetime is guaranteed until the function
returns.  Only when such a borrowed reference must be stored or passed
on, it must be turned into an owned reference by calling
\cfunction{Py_INCREF()}.

The object reference returned from a C function that is called from
Python must be an owned reference --- ownership is tranferred from the
function to its caller.


\subsection{Thin Ice
            \label{thinIce}}

There are a few situations where seemingly harmless use of a borrowed
reference can lead to problems.  These all have to do with implicit
invocations of the interpreter, which can cause the owner of a
reference to dispose of it.

The first and most important case to know about is using
\cfunction{Py_DECREF()} on an unrelated object while borrowing a
reference to a list item.  For instance:

\begin{verbatim}
void
bug(PyObject *list)
{
    PyObject *item = PyList_GetItem(list, 0);

    PyList_SetItem(list, 1, PyInt_FromLong(0L));
    PyObject_Print(item, stdout, 0); /* BUG! */
}
\end{verbatim}

This function first borrows a reference to \code{list[0]}, then
replaces \code{list[1]} with the value \code{0}, and finally prints
the borrowed reference.  Looks harmless, right?  But it's not!

Let's follow the control flow into \cfunction{PyList_SetItem()}.  The list
owns references to all its items, so when item 1 is replaced, it has
to dispose of the original item 1.  Now let's suppose the original
item 1 was an instance of a user-defined class, and let's further
suppose that the class defined a \method{__del__()} method.  If this
class instance has a reference count of 1, disposing of it will call
its \method{__del__()} method.

Since it is written in Python, the \method{__del__()} method can execute
arbitrary Python code.  Could it perhaps do something to invalidate
the reference to \code{item} in \cfunction{bug()}?  You bet!  Assuming
that the list passed into \cfunction{bug()} is accessible to the
\method{__del__()} method, it could execute a statement to the effect of
\samp{del list[0]}, and assuming this was the last reference to that
object, it would free the memory associated with it, thereby
invalidating \code{item}.

The solution, once you know the source of the problem, is easy:
temporarily increment the reference count.  The correct version of the
function reads:

\begin{verbatim}
void
no_bug(PyObject *list)
{
    PyObject *item = PyList_GetItem(list, 0);

    Py_INCREF(item);
    PyList_SetItem(list, 1, PyInt_FromLong(0L));
    PyObject_Print(item, stdout, 0);
    Py_DECREF(item);
}
\end{verbatim}

This is a true story.  An older version of Python contained variants
of this bug and someone spent a considerable amount of time in a C
debugger to figure out why his \method{__del__()} methods would fail...

The second case of problems with a borrowed reference is a variant
involving threads.  Normally, multiple threads in the Python
interpreter can't get in each other's way, because there is a global
lock protecting Python's entire object space.  However, it is possible
to temporarily release this lock using the macro
\csimplemacro{Py_BEGIN_ALLOW_THREADS}, and to re-acquire it using
\csimplemacro{Py_END_ALLOW_THREADS}.  This is common around blocking
I/O calls, to let other threads use the processor while waiting for
the I/O to complete.  Obviously, the following function has the same
problem as the previous one:

\begin{verbatim}
void
bug(PyObject *list)
{
    PyObject *item = PyList_GetItem(list, 0);
    Py_BEGIN_ALLOW_THREADS
    ...some blocking I/O call...
    Py_END_ALLOW_THREADS
    PyObject_Print(item, stdout, 0); /* BUG! */
}
\end{verbatim}


\subsection{NULL Pointers
            \label{nullPointers}}

In general, functions that take object references as arguments do not
expect you to pass them \NULL{} pointers, and will dump core (or
cause later core dumps) if you do so.  Functions that return object
references generally return \NULL{} only to indicate that an
exception occurred.  The reason for not testing for \NULL{}
arguments is that functions often pass the objects they receive on to
other function --- if each function were to test for \NULL,
there would be a lot of redundant tests and the code would run more
slowly.

It is better to test for \NULL{} only at the ``source:'' when a
pointer that may be \NULL{} is received, for example, from
\cfunction{malloc()} or from a function that may raise an exception.

The macros \cfunction{Py_INCREF()} and \cfunction{Py_DECREF()}
do not check for \NULL{} pointers --- however, their variants
\cfunction{Py_XINCREF()} and \cfunction{Py_XDECREF()} do.

The macros for checking for a particular object type
(\code{Py\var{type}_Check()}) don't check for \NULL{} pointers ---
again, there is much code that calls several of these in a row to test
an object against various different expected types, and this would
generate redundant tests.  There are no variants with \NULL{}
checking.

The C function calling mechanism guarantees that the argument list
passed to C functions (\code{args} in the examples) is never
\NULL{} --- in fact it guarantees that it is always a tuple.\footnote{
These guarantees don't hold when you use the ``old'' style
calling convention --- this is still found in much existing code.}

It is a severe error to ever let a \NULL{} pointer ``escape'' to
the Python user.

% Frank Stajano:
% A pedagogically buggy example, along the lines of the previous listing, 
% would be helpful here -- showing in more concrete terms what sort of 
% actions could cause the problem. I can't very well imagine it from the 
% description.


\section{Writing Extensions in \Cpp
         \label{cplusplus}}

It is possible to write extension modules in \Cpp.  Some restrictions
apply.  If the main program (the Python interpreter) is compiled and
linked by the C compiler, global or static objects with constructors
cannot be used.  This is not a problem if the main program is linked
by the \Cpp{} compiler.  Functions that will be called by the
Python interpreter (in particular, module initalization functions)
have to be declared using \code{extern "C"}.
It is unnecessary to enclose the Python header files in
\code{extern "C" \{...\}} --- they use this form already if the symbol
\samp{__cplusplus} is defined (all recent \Cpp{} compilers define this
symbol).


\section{Providing a C API for an Extension Module
         \label{using-cobjects}}
\sectionauthor{Konrad Hinsen}{hinsen@cnrs-orleans.fr}

Many extension modules just provide new functions and types to be
used from Python, but sometimes the code in an extension module can
be useful for other extension modules. For example, an extension
module could implement a type ``collection'' which works like lists
without order. Just like the standard Python list type has a C API
which permits extension modules to create and manipulate lists, this
new collection type should have a set of C functions for direct
manipulation from other extension modules.

At first sight this seems easy: just write the functions (without
declaring them \keyword{static}, of course), provide an appropriate
header file, and document the C API. And in fact this would work if
all extension modules were always linked statically with the Python
interpreter. When modules are used as shared libraries, however, the
symbols defined in one module may not be visible to another module.
The details of visibility depend on the operating system; some systems
use one global namespace for the Python interpreter and all extension
modules (Windows, for example), whereas others require an explicit
list of imported symbols at module link time (AIX is one example), or
offer a choice of different strategies (most Unices). And even if
symbols are globally visible, the module whose functions one wishes to
call might not have been loaded yet!

Portability therefore requires not to make any assumptions about
symbol visibility. This means that all symbols in extension modules
should be declared \keyword{static}, except for the module's
initialization function, in order to avoid name clashes with other
extension modules (as discussed in section~\ref{methodTable}). And it
means that symbols that \emph{should} be accessible from other
extension modules must be exported in a different way.

Python provides a special mechanism to pass C-level information
(pointers) from one extension module to another one: CObjects.
A CObject is a Python data type which stores a pointer (\ctype{void
*}).  CObjects can only be created and accessed via their C API, but
they can be passed around like any other Python object. In particular, 
they can be assigned to a name in an extension module's namespace.
Other extension modules can then import this module, retrieve the
value of this name, and then retrieve the pointer from the CObject.

There are many ways in which CObjects can be used to export the C API
of an extension module. Each name could get its own CObject, or all C
API pointers could be stored in an array whose address is published in
a CObject. And the various tasks of storing and retrieving the pointers
can be distributed in different ways between the module providing the
code and the client modules.

The following example demonstrates an approach that puts most of the
burden on the writer of the exporting module, which is appropriate
for commonly used library modules. It stores all C API pointers
(just one in the example!) in an array of \ctype{void} pointers which
becomes the value of a CObject. The header file corresponding to
the module provides a macro that takes care of importing the module
and retrieving its C API pointers; client modules only have to call
this macro before accessing the C API.

The exporting module is a modification of the \module{spam} module from
section~\ref{simpleExample}. The function \function{spam.system()}
does not call the C library function \cfunction{system()} directly,
but a function \cfunction{PySpam_System()}, which would of course do
something more complicated in reality (such as adding ``spam'' to
every command). This function \cfunction{PySpam_System()} is also
exported to other extension modules.

The function \cfunction{PySpam_System()} is a plain C function,
declared \keyword{static} like everything else:

\begin{verbatim}
static int
PySpam_System(const char *command)
{
    return system(command);
}
\end{verbatim}

The function \cfunction{spam_system()} is modified in a trivial way:

\begin{verbatim}
static PyObject *
spam_system(PyObject *self, PyObject *args)
{
    const char *command;
    int sts;

    if (!PyArg_ParseTuple(args, "s", &command))
        return NULL;
    sts = PySpam_System(command);
    return Py_BuildValue("i", sts);
}
\end{verbatim}

In the beginning of the module, right after the line

\begin{verbatim}
#include "Python.h"
\end{verbatim}

two more lines must be added:

\begin{verbatim}
#define SPAM_MODULE
#include "spammodule.h"
\end{verbatim}

The \code{\#define} is used to tell the header file that it is being
included in the exporting module, not a client module. Finally,
the module's initialization function must take care of initializing
the C API pointer array:

\begin{verbatim}
PyMODINIT_FUNC
initspam(void)
{
    PyObject *m;
    static void *PySpam_API[PySpam_API_pointers];
    PyObject *c_api_object;

    m = Py_InitModule("spam", SpamMethods);

    /* Initialize the C API pointer array */
    PySpam_API[PySpam_System_NUM] = (void *)PySpam_System;

    /* Create a CObject containing the API pointer array's address */
    c_api_object = PyCObject_FromVoidPtr((void *)PySpam_API, NULL);

    if (c_api_object != NULL)
        PyModule_AddObject(m, "_C_API", c_api_object);
}
\end{verbatim}

Note that \code{PySpam_API} is declared \keyword{static}; otherwise
the pointer array would disappear when \function{initspam()} terminates!

The bulk of the work is in the header file \file{spammodule.h},
which looks like this:

\begin{verbatim}
#ifndef Py_SPAMMODULE_H
#define Py_SPAMMODULE_H
#ifdef __cplusplus
extern "C" {
#endif

/* Header file for spammodule */

/* C API functions */
#define PySpam_System_NUM 0
#define PySpam_System_RETURN int
#define PySpam_System_PROTO (char *command)

/* Total number of C API pointers */
#define PySpam_API_pointers 1


#ifdef SPAM_MODULE
/* This section is used when compiling spammodule.c */

static PySpam_System_RETURN PySpam_System PySpam_System_PROTO;

#else
/* This section is used in modules that use spammodule's API */

static void **PySpam_API;

#define PySpam_System \
 (*(PySpam_System_RETURN (*)PySpam_System_PROTO) PySpam_API[PySpam_System_NUM])

/* Return -1 and set exception on error, 0 on success. */
static int
import_spam(void)
{
    PyObject *module = PyImport_ImportModule("spam");

    if (module != NULL) {
        PyObject *c_api_object = PyObject_GetAttrString(module, "_C_API");
        if (c_api_object == NULL)
            return -1;
        if (PyCObject_Check(c_api_object))
            PySpam_API = (void **)PyCObject_AsVoidPtr(c_api_object);
        Py_DECREF(c_api_object);
    }
    return 0;
}

#endif

#ifdef __cplusplus
}
#endif

#endif /* !defined(Py_SPAMMODULE_H) */
\end{verbatim}

All that a client module must do in order to have access to the
function \cfunction{PySpam_System()} is to call the function (or
rather macro) \cfunction{import_spam()} in its initialization
function:

\begin{verbatim}
PyMODINIT_FUNC
initclient(void)
{
    PyObject *m;

    Py_InitModule("client", ClientMethods);
    if (import_spam() < 0)
        return;
    /* additional initialization can happen here */
}
\end{verbatim}

The main disadvantage of this approach is that the file
\file{spammodule.h} is rather complicated. However, the
basic structure is the same for each function that is
exported, so it has to be learned only once.

Finally it should be mentioned that CObjects offer additional
functionality, which is especially useful for memory allocation and
deallocation of the pointer stored in a CObject. The details
are described in the \citetitle[../api/api.html]{Python/C API
Reference Manual} in the section
``\ulink{CObjects}{../api/cObjects.html}'' and in the implementation
of CObjects (files \file{Include/cobject.h} and
\file{Objects/cobject.c} in the Python source code distribution).

\chapter{Object Implementation Support \label{newTypes}}


This chapter describes the functions, types, and macros used when
defining new object types.


\section{Allocating Objects on the Heap
         \label{allocating-objects}}

\begin{cfuncdesc}{PyObject*}{_PyObject_New}{PyTypeObject *type}
\end{cfuncdesc}

\begin{cfuncdesc}{PyVarObject*}{_PyObject_NewVar}{PyTypeObject *type, int size}
\end{cfuncdesc}

\begin{cfuncdesc}{void}{_PyObject_Del}{PyObject *op}
\end{cfuncdesc}

\begin{cfuncdesc}{PyObject*}{PyObject_Init}{PyObject *op,
					    PyTypeObject *type}
  Initialize a newly-allocated object \var{op} with its type and
  initial reference.  Returns the initialized object.  If \var{type}
  indicates that the object participates in the cyclic garbage
  detector, it it added to the detector's set of observed objects.
  Other fields of the object are not affected.
\end{cfuncdesc}

\begin{cfuncdesc}{PyVarObject*}{PyObject_InitVar}{PyVarObject *op,
						  PyTypeObject *type, int size}
  This does everything \cfunction{PyObject_Init()} does, and also
  initializes the length information for a variable-size object.
\end{cfuncdesc}

\begin{cfuncdesc}{\var{TYPE}*}{PyObject_New}{TYPE, PyTypeObject *type}
  Allocate a new Python object using the C structure type \var{TYPE}
  and the Python type object \var{type}.  Fields not defined by the
  Python object header are not initialized; the object's reference
  count will be one.  The size of the memory
  allocation is determined from the \member{tp_basicsize} field of the
  type object.
\end{cfuncdesc}

\begin{cfuncdesc}{\var{TYPE}*}{PyObject_NewVar}{TYPE, PyTypeObject *type,
                                                int size}
  Allocate a new Python object using the C structure type \var{TYPE}
  and the Python type object \var{type}.  Fields not defined by the
  Python object header are not initialized.  The allocated memory
  allows for the \var{TYPE} structure plus \var{size} fields of the
  size given by the \member{tp_itemsize} field of \var{type}.  This is
  useful for implementing objects like tuples, which are able to
  determine their size at construction time.  Embedding the array of
  fields into the same allocation decreases the number of allocations,
  improving the memory management efficiency.
\end{cfuncdesc}

\begin{cfuncdesc}{void}{PyObject_Del}{PyObject *op}
  Releases memory allocated to an object using
  \cfunction{PyObject_New()} or \cfunction{PyObject_NewVar()}.  This
  is normally called from the \member{tp_dealloc} handler specified in
  the object's type.  The fields of the object should not be accessed
  after this call as the memory is no longer a valid Python object.
\end{cfuncdesc}

\begin{cfuncdesc}{\var{TYPE}*}{PyObject_NEW}{TYPE, PyTypeObject *type}
  Macro version of \cfunction{PyObject_New()}, to gain performance at
  the expense of safety.  This does not check \var{type} for a \NULL{}
  value.
\end{cfuncdesc}

\begin{cfuncdesc}{\var{TYPE}*}{PyObject_NEW_VAR}{TYPE, PyTypeObject *type,
                                                int size}
  Macro version of \cfunction{PyObject_NewVar()}, to gain performance
  at the expense of safety.  This does not check \var{type} for a
  \NULL{} value.
\end{cfuncdesc}

\begin{cfuncdesc}{void}{PyObject_DEL}{PyObject *op}
  Macro version of \cfunction{PyObject_Del()}.
\end{cfuncdesc}

\begin{cfuncdesc}{PyObject*}{Py_InitModule}{char *name,
                                            PyMethodDef *methods}
  Create a new module object based on a name and table of functions,
  returning the new module object.

  \versionchanged[Older versions of Python did not support \NULL{} as
                  the value for the \var{methods} argument]{2.3}
\end{cfuncdesc}

\begin{cfuncdesc}{PyObject*}{Py_InitModule3}{char *name,
                                             PyMethodDef *methods,
                                             char *doc}
  Create a new module object based on a name and table of functions,
  returning the new module object.  If \var{doc} is non-\NULL, it will
  be used to define the docstring for the module.

  \versionchanged[Older versions of Python did not support \NULL{} as
                  the value for the \var{methods} argument]{2.3}
\end{cfuncdesc}

\begin{cfuncdesc}{PyObject*}{Py_InitModule4}{char *name,
                                             PyMethodDef *methods,
                                             char *doc, PyObject *self,
                                             int apiver}
  Create a new module object based on a name and table of functions,
  returning the new module object.  If \var{doc} is non-\NULL, it will
  be used to define the docstring for the module.  If \var{self} is
  non-\NULL, it will passed to the functions of the module as their
  (otherwise \NULL) first parameter.  (This was added as an
  experimental feature, and there are no known uses in the current
  version of Python.)  For \var{apiver}, the only value which should
  be passed is defined by the constant \constant{PYTHON_API_VERSION}.

  \note{Most uses of this function should probably be using
  the \cfunction{Py_InitModule3()} instead; only use this if you are
  sure you need it.}

  \versionchanged[Older versions of Python did not support \NULL{} as
                  the value for the \var{methods} argument]{2.3}
\end{cfuncdesc}

DL_IMPORT

\begin{cvardesc}{PyObject}{_Py_NoneStruct}
  Object which is visible in Python as \code{None}.  This should only
  be accessed using the \code{Py_None} macro, which evaluates to a
  pointer to this object.
\end{cvardesc}


\section{Common Object Structures \label{common-structs}}

There are a large number of structures which are used in the
definition of object types for Python.  This section describes these
structures and how they are used.

All Python objects ultimately share a small number of fields at the
beginning of the object's representation in memory.  These are
represented by the \ctype{PyObject} and \ctype{PyVarObject} types,
which are defined, in turn, by the expansions of some macros also
used, whether directly or indirectly, in the definition of all other
Python objects.

\begin{ctypedesc}{PyObject}
  All object types are extensions of this type.  This is a type which
  contains the information Python needs to treat a pointer to an
  object as an object.  In a normal ``release'' build, it contains
  only the objects reference count and a pointer to the corresponding
  type object.  It corresponds to the fields defined by the
  expansion of the \code{PyObject_VAR_HEAD} macro.
\end{ctypedesc}

\begin{ctypedesc}{PyVarObject}
  This is an extension of \ctype{PyObject} that adds the
  \member{ob_size} field.  This is only used for objects that have
  some notion of \emph{length}.  This type does not often appear in
  the Python/C API.  It corresponds to the fields defined by the
  expansion of the \code{PyObject_VAR_HEAD} macro.
\end{ctypedesc}

These macros are used in the definition of \ctype{PyObject} and
\ctype{PyVarObject}:

\begin{csimplemacrodesc}{PyObject_HEAD}
  This is a macro which expands to the declarations of the fields of
  the \ctype{PyObject} type; it is used when declaring new types which
  represent objects without a varying length.  The specific fields it
  expands to depends on the definition of
  \csimplemacro{Py_TRACE_REFS}.  By default, that macro is not
  defined, and \csimplemacro{PyObject_HEAD} expands to:
  \begin{verbatim}
    int ob_refcnt;
    PyTypeObject *ob_type;
  \end{verbatim}
  When \csimplemacro{Py_TRACE_REFS} is defined, it expands to:
  \begin{verbatim}
    PyObject *_ob_next, *_ob_prev;
    int ob_refcnt;
    PyTypeObject *ob_type;
  \end{verbatim}
\end{csimplemacrodesc}

\begin{csimplemacrodesc}{PyObject_VAR_HEAD}
  This is a macro which expands to the declarations of the fields of
  the \ctype{PyVarObject} type; it is used when declaring new types which
  represent objects with a length that varies from instance to
  instance.  This macro always expands to:
  \begin{verbatim}
    PyObject_HEAD
    int ob_size;
  \end{verbatim}
  Note that \csimplemacro{PyObject_HEAD} is part of the expansion, and
  that it's own expansion varies depending on the definition of
  \csimplemacro{Py_TRACE_REFS}.
\end{csimplemacrodesc}

PyObject_HEAD_INIT

\begin{ctypedesc}{PyCFunction}
  Type of the functions used to implement most Python callables in C.
  Functions of this type take two \ctype{PyObject*} parameters and
  return one such value.  If the return value is \NULL, an exception
  shall have been set.  If not \NULL, the return value is interpreted
  as the return value of the function as exposed in Python.  The
  function must return a new reference.
\end{ctypedesc}

\begin{ctypedesc}{PyMethodDef}
  Structure used to describe a method of an extension type.  This
  structure has four fields:

  \begin{tableiii}{l|l|l}{member}{Field}{C Type}{Meaning}
    \lineiii{ml_name}{char *}{name of the method}
    \lineiii{ml_meth}{PyCFunction}{pointer to the C implementation}
    \lineiii{ml_flags}{int}{flag bits indicating how the call should be
                            constructed}
    \lineiii{ml_doc}{char *}{points to the contents of the docstring}
  \end{tableiii}
\end{ctypedesc}

The \member{ml_meth} is a C function pointer.  The functions may be of
different types, but they always return \ctype{PyObject*}.  If the
function is not of the \ctype{PyCFunction}, the compiler will require
a cast in the method table.  Even though \ctype{PyCFunction} defines
the first parameter as \ctype{PyObject*}, it is common that the method
implementation uses a the specific C type of the \var{self} object.

The \member{ml_flags} field is a bitfield which can include the
following flags.  The individual flags indicate either a calling
convention or a binding convention.  Of the calling convention flags,
only \constant{METH_VARARGS} and \constant{METH_KEYWORDS} can be
combined (but note that \constant{METH_KEYWORDS} alone is equivalent
to \code{\constant{METH_VARARGS} | \constant{METH_KEYWORDS}}).
Any of the calling convention flags can be combined with a
binding flag.

\begin{datadesc}{METH_VARARGS}
  This is the typical calling convention, where the methods have the
  type \ctype{PyCFunction}. The function expects two
  \ctype{PyObject*} values.  The first one is the \var{self} object for
  methods; for module functions, it has the value given to
  \cfunction{Py_InitModule4()} (or \NULL{} if
  \cfunction{Py_InitModule()} was used).  The second parameter
  (often called \var{args}) is a tuple object representing all
  arguments. This parameter is typically processed using
  \cfunction{PyArg_ParseTuple()} or \cfunction{PyArg_UnpackTuple}.
\end{datadesc}

\begin{datadesc}{METH_KEYWORDS}
  Methods with these flags must be of type
  \ctype{PyCFunctionWithKeywords}.  The function expects three
  parameters: \var{self}, \var{args}, and a dictionary of all the
  keyword arguments.  The flag is typically combined with
  \constant{METH_VARARGS}, and the parameters are typically processed
  using \cfunction{PyArg_ParseTupleAndKeywords()}.
\end{datadesc}

\begin{datadesc}{METH_NOARGS}
  Methods without parameters don't need to check whether arguments are
  given if they are listed with the \constant{METH_NOARGS} flag.  They
  need to be of type \ctype{PyCFunction}.  When used with object
  methods, the first parameter is typically named \code{self} and will
  hold a reference to the object instance.  In all cases the second
  parameter will be \NULL.
\end{datadesc}

\begin{datadesc}{METH_O}
  Methods with a single object argument can be listed with the
  \constant{METH_O} flag, instead of invoking
  \cfunction{PyArg_ParseTuple()} with a \code{"O"} argument. They have
  the type \ctype{PyCFunction}, with the \var{self} parameter, and a
  \ctype{PyObject*} parameter representing the single argument.
\end{datadesc}

\begin{datadesc}{METH_OLDARGS}
  This calling convention is deprecated.  The method must be of type
  \ctype{PyCFunction}.  The second argument is \NULL{} if no arguments
  are given, a single object if exactly one argument is given, and a
  tuple of objects if more than one argument is given.  There is no
  way for a function using this convention to distinguish between a
  call with multiple arguments and a call with a tuple as the only
  argument.
\end{datadesc}

These two constants are not used to indicate the calling convention
but the binding when use with methods of classes.  These may not be
used for functions defined for modules.  At most one of these flags
may be set for any given method.

\begin{datadesc}{METH_CLASS}
  The method will be passed the type object as the first parameter
  rather than an instance of the type.  This is used to create
  \emph{class methods}, similar to what is created when using the
  \function{classmethod()}\bifuncindex{classmethod} built-in
  function.
  \versionadded{2.3}
\end{datadesc}

\begin{datadesc}{METH_STATIC}
  The method will be passed \NULL{} as the first parameter rather than
  an instance of the type.  This is used to create \emph{static
  methods}, similar to what is created when using the
  \function{staticmethod()}\bifuncindex{staticmethod} built-in
  function.
  \versionadded{2.3}
\end{datadesc}


\begin{cfuncdesc}{PyObject*}{Py_FindMethod}{PyMethodDef table[],
                                            PyObject *ob, char *name}
  Return a bound method object for an extension type implemented in
  C.  This can be useful in the implementation of a
  \member{tp_getattro} or \member{tp_getattr} handler that does not
  use the \cfunction{PyObject_GenericGetAttr()} function.
\end{cfuncdesc}


\section{Type Objects \label{type-structs}}

Perhaps one of the most important structures of the Python object
system is the structure that defines a new type: the
\ctype{PyTypeObject} structure.  Type objects can be handled using any
of the \cfunction{PyObject_*()} or \cfunction{PyType_*()} functions,
but do not offer much that's interesting to most Python applications.
These objects are fundamental to how objects behave, so they are very
important to the interpreter itself and to any extension module that
implements new types.

Type objects are fairly large compared to most of the standard types.
The reason for the size is that each type object stores a large number
of values, mostly C function pointers, each of which implements a
small part of the type's functionality.  The fields of the type object
are examined in detail in this section.  The fields will be described
in the order in which they occur in the structure.

Typedefs:
unaryfunc, binaryfunc, ternaryfunc, inquiry, coercion, intargfunc,
intintargfunc, intobjargproc, intintobjargproc, objobjargproc,
destructor, freefunc, printfunc, getattrfunc, getattrofunc, setattrfunc,
setattrofunc, cmpfunc, reprfunc, hashfunc

The structure definition for \ctype{PyTypeObject} can be found in
\file{Include/object.h}.  For convenience of reference, this repeats
the definition found there:

\verbatiminput{typestruct.h}

The type object structure extends the \ctype{PyVarObject} structure.
The \member{ob_size} field is used for dynamic types (created
by  \function{type_new()}, usually called from a class statement).
Note that \cdata{PyType_Type} (the metatype) initializes
\member{tp_itemsize}, which means that its instances (i.e. type
objects) \emph{must} have the \member{ob_size} field.

\begin{cmemberdesc}{PyObject}{PyObject*}{_ob_next}
\cmemberline{PyObject}{PyObject*}{_ob_prev}
  These fields are only present when the macro \code{Py_TRACE_REFS} is
  defined.  Their initialization to \NULL{} is taken care of by the
  \code{PyObject_HEAD_INIT} macro.  For statically allocated objects,
  these fields always remain \NULL.  For dynamically allocated
  objects, these two fields are used to link the object into a
  doubly-linked list of \emph{all} live objects on the heap.  This
  could be used for various debugging purposes; currently the only use
  is to print the objects that are still alive at the end of a run
  when the environment variable \envvar{PYTHONDUMPREFS} is set.

  These fields are not inherited by subtypes.
\end{cmemberdesc}

\begin{cmemberdesc}{PyObject}{int}{ob_refcnt}
  This is the type object's reference count, initialized to \code{1}
  by the \code{PyObject_HEAD_INIT} macro.  Note that for statically
  allocated type objects, the type's instances (objects whose
  \member{ob_type} points back to the type) do \emph{not} count as
  references.  But for dynamically allocated type objects, the
  instances \emph{do} count as references.

  This field is not inherited by subtypes.
\end{cmemberdesc}

\begin{cmemberdesc}{PyObject}{PyTypeObject*}{ob_type}
  This is the type's type, in other words its metatype.  It is
  initialized by the argument to the \code{PyObject_HEAD_INIT} macro,
  and its value should normally be \code{\&PyType_Type}.  However, for
  dynamically loadable extension modules that must be usable on
  Windows (at least), the compiler complains that this is not a valid
  initializer.  Therefore, the convention is to pass \NULL{} to the
  \code{PyObject_HEAD_INIT} macro and to initialize this field
  explicitly at the start of the module's initialization function,
  before doing anything else.  This is typically done like this:

\begin{verbatim}
Foo_Type.ob_type = &PyType_Type;
\end{verbatim}

  This should be done before any instances of the type are created.
  \cfunction{PyType_Ready()} checks if \member{ob_type} is \NULL, and
  if so, initializes it: in Python 2.2, it is set to
  \code{\&PyType_Type}; in Python 2.2.1 and later it will be
  initialized to the \member{ob_type} field of the base class.
  \cfunction{PyType_Ready()} will not change this field if it is
  non-zero.

  In Python 2.2, this field is not inherited by subtypes.  In 2.2.1,
  and in 2.3 and beyond, it is inherited by subtypes.
\end{cmemberdesc}

\begin{cmemberdesc}{PyVarObject}{int}{ob_size}
  For statically allocated type objects, this should be initialized
  to zero.  For dynamically allocated type objects, this field has a
  special internal meaning.

  This field is not inherited by subtypes.
\end{cmemberdesc}

\begin{cmemberdesc}{PyTypeObject}{char*}{tp_name}
  Pointer to a NUL-terminated string containing the name of the type.
  For types that are accessible as module globals, the string should
  be the full module name, followed by a dot, followed by the type
  name; for built-in types, it should be just the type name.  If the
  module is a submodule of a package, the full package name is part of
  the full module name.  For example, a type named \class{T} defined
  in module \module{M} in subpackage \module{Q} in package \module{P}
  should have the \member{tp_name} initializer \code{"P.Q.M.T"}.

  For dynamically allocated type objects, this may be just the type
  name, if the module name is explicitly stored in the type dict as
  the value for key \code{'__module__'}.

  If the tp_name field contains a dot, everything before the last dot
  is made accessible as the \member{__module__} attribute, and
  everything after the last dot is made accessible as the
  \member{__name__} attribute.  If no dot is present, the entire
  \member{tp_name} field is made accessible as the \member{__name__}
  attribute, and the \member{__module__} attribute is undefined
  (unless explicitly set in the dictionary, as explained above).

  This field is not inherited by subtypes.
\end{cmemberdesc}

\begin{cmemberdesc}{PyTypeObject}{int}{tp_basicsize}
\cmemberline{PyTypeObject}{int}{tp_itemsize}
  These fields allow calculating the size in bytes of instances of
  the type.

  There are two kinds of types: types with fixed-length instances have
  a zero \member{tp_itemsize} field, types with variable-length
  instances have a non-zero \member{tp_itemsize} field.  For a type
  with fixed-length instances, all instances have the same size,
  given in \member{tp_basicsize}.

  For a type with variable-length instances, the instances must have
  an \member{ob_size} field, and the instance size is
  \member{tp_basicsize} plus N times \member{tp_itemsize}, where N is
  the ``length'' of the object.  The value of N is typically stored in
  the instance's \member{ob_size} field.  There are exceptions:  for
  example, long ints use a negative \member{ob_size} to indicate a
  negative number, and N is \code{abs(\member{ob_size})} there.  Also,
  the presence of an \member{ob_size} field in the instance layout
  doesn't mean that the type is variable-length (for example, the list
  type has fixed-length instances, yet those instances have a
  meaningful \member{ob_size} field).

  The basic size includes the fields in the instance declared by the
  macro \csimplemacro{PyObject_HEAD} or
  \csimplemacro{PyObject_VAR_HEAD} (whichever is used to declare the
  instance struct) and this in turn includes the \member{_ob_prev} and
  \member{_ob_next} fields if they are present.  This means that the
  only correct way to get an initializer for the \member{tp_basicsize}
  is to use the \keyword{sizeof} operator on the struct used to
  declare the instance layout.  The basic size does not include the GC
  header size (this is new in Python 2.2; in 2.1 and 2.0, the GC
  header size was included in \member{tp_basicsize}).

  These fields are inherited separately by subtypes.  If the base type
  has a non-zero \member{tp_itemsize}, it is generally not safe to set
  \member{tp_itemsize} to a different non-zero value in a subtype
  (though this depends on the implementation of the base type).

  A note about alignment: if the variable items require a particular
  alignment, this should be taken care of by the value of
  \member{tp_basicsize}.  Example: suppose a type implements an array
  of \code{double}. \member{tp_itemsize} is \code{sizeof(double)}.
  It is the programmer's responsibility that \member{tp_basicsize} is
  a multiple of \code{sizeof(double)} (assuming this is the alignment
  requirement for \code{double}).
\end{cmemberdesc}

\begin{cmemberdesc}{PyTypeObject}{destructor}{tp_dealloc}
  A pointer to the instance destructor function.  This function must
  be defined unless the type guarantees that its instances will never
  be deallocated (as is the case for the singletons \code{None} and
  \code{Ellipsis}).

  The destructor function is called by the \cfunction{Py_DECREF()} and
  \cfunction{Py_XDECREF()} macros when the new reference count is
  zero.  At this point, the instance is still in existance, but there
  are no references to it.  The destructor function should free all
  references which the instance owns, free all memory buffers owned by
  the instance (using the freeing function corresponding to the
  allocation function used to allocate the buffer), and finally (as
  its last action) call the type's \member{tp_free} function.  If the
  type is not subtypable (doesn't have the
  \constant{Py_TPFLAGS_BASETYPE} flag bit set), it is permissible to
  call the object deallocator directly instead of via
  \member{tp_free}.  The object deallocator should be the one used to
  allocate the instance; this is normally \cfunction{PyObject_Del()}
  if the instance was allocated using \cfunction{PyObject_New()} or
  \cfunction{PyOject_VarNew()}, or \cfunction{PyObject_GC_Del()} if
  the instance was allocated using \cfunction{PyObject_GC_New()} or
  \cfunction{PyObject_GC_VarNew()}.

  This field is inherited by subtypes.
\end{cmemberdesc}

\begin{cmemberdesc}{PyTypeObject}{printfunc}{tp_print}
  An optional pointer to the instance print function.

  The print function is only called when the instance is printed to a
  \emph{real} file; when it is printed to a pseudo-file (like a
  \class{StringIO} instance), the instance's \member{tp_repr} or
  \member{tp_str} function is called to convert it to a string.  These
  are also called when the type's \member{tp_print} field is \NULL.  A
  type should never implement \member{tp_print} in a way that produces
  different output than \member{tp_repr} or \member{tp_str} would.

  The print function is called with the same signature as
  \cfunction{PyObject_Print()}: \code{int tp_print(PyObject *self, FILE
  *file, int flags)}.  The \var{self} argument is the instance to be
  printed.  The \var{file} argument is the stdio file to which it is
  to be printed.  The \var{flags} argument is composed of flag bits.
  The only flag bit currently defined is \constant{Py_PRINT_RAW}.
  When the \constant{Py_PRINT_RAW} flag bit is set, the instance
  should be printed the same way as \member{tp_str} would format it;
  when the \constant{Py_PRINT_RAW} flag bit is clear, the instance
  should be printed the same was as \member{tp_repr} would format it.
  It should return \code{-1} and set an exception condition when an
  error occurred during the comparison.

  It is possible that the \member{tp_print} field will be deprecated.
  In any case, it is recommended not to define \member{tp_print}, but
  instead to rely on \member{tp_repr} and \member{tp_str} for
  printing.

  This field is inherited by subtypes.
\end{cmemberdesc}

\begin{cmemberdesc}{PyTypeObject}{getattrfunc}{tp_getattr}
  An optional pointer to the get-attribute-string function.

  This field is deprecated.  When it is defined, it should point to a
  function that acts the same as the \member{tp_getattro} function,
  but taking a C string instead of a Python string object to give the
  attribute name.  The signature is the same as for
  \cfunction{PyObject_GetAttrString()}.

  This field is inherited by subtypes together with
  \member{tp_getattro}: a subtype inherits both \member{tp_getattr}
  and \member{tp_getattro} from its base type when the subtype's
  \member{tp_getattr} and \member{tp_getattro} are both \NULL.
\end{cmemberdesc}

\begin{cmemberdesc}{PyTypeObject}{setattrfunc}{tp_setattr}
  An optional pointer to the set-attribute-string function.

  This field is deprecated.  When it is defined, it should point to a
  function that acts the same as the \member{tp_setattro} function,
  but taking a C string instead of a Python string object to give the
  attribute name.  The signature is the same as for
  \cfunction{PyObject_SetAttrString()}.

  This field is inherited by subtypes together with
  \member{tp_setattro}: a subtype inherits both \member{tp_setattr}
  and \member{tp_setattro} from its base type when the subtype's
  \member{tp_setattr} and \member{tp_setattro} are both \NULL.
\end{cmemberdesc}

\begin{cmemberdesc}{PyTypeObject}{cmpfunc}{tp_compare}
  An optional pointer to the three-way comparison function.

  The signature is the same as for \cfunction{PyObject_Compare()}.
  The function should return \code{1} if \var{self} greater than
  \var{other}, \code{0} if \var{self} is equal to \var{other}, and
  \code{-1} if \var{self} less than \var{other}.  It should return
  \code{-1} and set an exception condition when an error occurred
  during the comparison.

  This field is inherited by subtypes together with
  \member{tp_richcompare} and \member{tp_hash}: a subtypes inherits
  all three of \member{tp_compare}, \member{tp_richcompare}, and
  \member{tp_hash} when the subtype's \member{tp_compare},
  \member{tp_richcompare}, and \member{tp_hash} are all \NULL.
\end{cmemberdesc}

\begin{cmemberdesc}{PyTypeObject}{reprfunc}{tp_repr}
  An optional pointer to a function that implements the built-in
  function \function{repr()}.\bifuncindex{repr}

  The signature is the same as for \cfunction{PyObject_Repr()}; it
  must return a string or a Unicode object.  Ideally, this function
  should return a string that, when passed to \function{eval()}, given
  a suitable environment, returns an object with the same value.  If
  this is not feasible, it should return a string starting with
  \character{\textless} and ending with \character{\textgreater} from
  which both the type and the value of the object can be deduced.

  When this field is not set, a string of the form \samp{<\%s object
  at \%p>} is returned, where \code{\%s} is replaced by the type name,
  and \code{\%p} by the object's memory address.

  This field is inherited by subtypes.
\end{cmemberdesc}

PyNumberMethods *tp_as_number;

    XXX

PySequenceMethods *tp_as_sequence;

    XXX

PyMappingMethods *tp_as_mapping;

    XXX

\begin{cmemberdesc}{PyTypeObject}{hashfunc}{tp_hash}
  An optional pointer to a function that implements the built-in
  function \function{hash()}.\bifuncindex{hash}

  The signature is the same as for \cfunction{PyObject_Hash()}; it
  must return a C long.  The value \code{-1} should not be returned as
  a normal return value; when an error occurs during the computation
  of the hash value, the function should set an exception and return
  \code{-1}.

  When this field is not set, two possibilities exist: if the
  \member{tp_compare} and \member{tp_richcompare} fields are both
  \NULL, a default hash value based on the object's address is
  returned; otherwise, a \exception{TypeError} is raised.

  This field is inherited by subtypes together with
  \member{tp_richcompare} and \member{tp_compare}: a subtypes inherits
  all three of \member{tp_compare}, \member{tp_richcompare}, and
  \member{tp_hash}, when the subtype's \member{tp_compare},
  \member{tp_richcompare} and \member{tp_hash} are all \NULL.
\end{cmemberdesc}

\begin{cmemberdesc}{PyTypeObject}{ternaryfunc}{tp_call}
  An optional pointer to a function that implements calling the
  object.  This should be \NULL{} if the object is not callable.  The
  signature is the same as for \cfunction{PyObject_Call()}.

  This field is inherited by subtypes.
\end{cmemberdesc}

\begin{cmemberdesc}{PyTypeObject}{reprfunc}{tp_str}
  An optional pointer to a function that implements the built-in
  operation \function{str()}.  (Note that \class{str} is a type now,
  and \function{str()} calls the constructor for that type.  This
  constructor calls \cfunction{PyObject_Str()} to do the actual work,
  and \cfunction{PyObject_Str()} will call this handler.)

  The signature is the same as for \cfunction{PyObject_Str()}; it must
  return a string or a Unicode object.  This function should return a
  ``friendly'' string representation of the object, as this is the
  representation that will be used by the print statement.

  When this field is not set, \cfunction{PyObject_Repr()} is called to
  return a string representation.

  This field is inherited by subtypes.
\end{cmemberdesc}

\begin{cmemberdesc}{PyTypeObject}{getattrofunc}{tp_getattro}
  An optional pointer to the get-attribute function.

  The signature is the same as for \cfunction{PyObject_GetAttr()}.  It
  is usually convenient to set this field to
  \cfunction{PyObject_GenericGetAttr()}, which implements the normal
  way of looking for object attributes.

  This field is inherited by subtypes together with
  \member{tp_getattr}: a subtype inherits both \member{tp_getattr} and
  \member{tp_getattro} from its base type when the subtype's
  \member{tp_getattr} and \member{tp_getattro} are both \NULL.
\end{cmemberdesc}

\begin{cmemberdesc}{PyTypeObject}{setattrofunc}{tp_setattro}
  An optional pointer to the set-attribute function.

  The signature is the same as for \cfunction{PyObject_SetAttr()}.  It
  is usually convenient to set this field to
  \cfunction{PyObject_GenericSetAttr()}, which implements the normal
  way of setting object attributes.

  This field is inherited by subtypes together with
  \member{tp_setattr}: a subtype inherits both \member{tp_setattr} and
  \member{tp_setattro} from its base type when the subtype's
  \member{tp_setattr} and \member{tp_setattro} are both \NULL.
\end{cmemberdesc}

\begin{cmemberdesc}{PyTypeObject}{PyBufferProcs*}{tp_as_buffer}
  Pointer to an additional structure contains fields relevant only to
  objects which implement the buffer interface.  These fields are
  documented in ``Buffer Object Structures'' (section
  \ref{buffer-structs}).

  The \member{tp_as_buffer} field is not inherited, but the contained
  fields are inherited individually.
\end{cmemberdesc}

\begin{cmemberdesc}{PyTypeObject}{long}{tp_flags}
  This field is a bit mask of various flags.  Some flags indicate
  variant semantics for certain situations; others are used to
  indicate that certain fields in the type object (or in the extension
  structures referenced via \member{tp_as_number},
  \member{tp_as_sequence}, \member{tp_as_mapping}, and
  \member{tp_as_buffer}) that were historically not always present are
  valid; if such a flag bit is clear, the type fields it guards must
  not be accessed and must be considered to have a zero or \NULL{}
  value instead.

  Inheritance of this field is complicated.  Most flag bits are
  inherited individually, i.e. if the base type has a flag bit set,
  the subtype inherits this flag bit.  The flag bits that pertain to
  extension structures are strictly inherited if the extension
  structure is inherited, i.e. the base type's value of the flag bit
  is copied into the subtype together with a pointer to the extension
  structure.  The \constant{Py_TPFLAGS_HAVE_GC} flag bit is inherited
  together with the \member{tp_traverse} and \member{tp_clear} fields,
  i.e. if the \constant{Py_TPFLAGS_HAVE_GC} flag bit is clear in the
  subtype and the \member{tp_traverse} and \member{tp_clear} fields in
  the subtype exist (as indicated by the
  \constant{Py_TPFLAGS_HAVE_RICHCOMPARE} flag bit) and have \NULL{}
  values.

  The following bit masks are currently defined; these can be or-ed
  together using the \code{|} operator to form the value of the
  \member{tp_flags} field.  The macro \cfunction{PyType_HasFeature()}
  takes a type and a flags value, \var{tp} and \var{f}, and checks
  whether \code{\var{tp}->tp_flags \& \var{f}} is non-zero.

  \begin{datadesc}{Py_TPFLAGS_HAVE_GETCHARBUFFER}
    If this bit is set, the \ctype{PyBufferProcs} struct referenced by
    \member{tp_as_buffer} has the \member{bf_getcharbuffer} field.
  \end{datadesc}

  \begin{datadesc}{Py_TPFLAGS_HAVE_SEQUENCE_IN}
    If this bit is set, the \ctype{PySequenceMethods} struct
    referenced by \member{tp_as_sequence} has the \member{sq_contains}
    field.
  \end{datadesc}

  \begin{datadesc}{Py_TPFLAGS_GC}
    This bit is obsolete.  The bit it used to name is no longer in
    use.  The symbol is now defined as zero.
  \end{datadesc}

  \begin{datadesc}{Py_TPFLAGS_HAVE_INPLACEOPS}
    If this bit is set, the \ctype{PySequenceMethods} struct
    referenced by \member{tp_as_sequence} and the
    \ctype{PyNumberMethods} structure referenced by
    \member{tp_as_number} contain the fields for in-place operators.
    In particular, this means that the \ctype{PyNumberMethods}
    structure has the fields \member{nb_inplace_add},
    \member{nb_inplace_subtract}, \member{nb_inplace_multiply},
    \member{nb_inplace_divide}, \member{nb_inplace_remainder},
    \member{nb_inplace_power}, \member{nb_inplace_lshift},
    \member{nb_inplace_rshift}, \member{nb_inplace_and},
    \member{nb_inplace_xor}, and \member{nb_inplace_or}; and the
    \ctype{PySequenceMethods} struct has the fields
    \member{sq_inplace_concat} and \member{sq_inplace_repeat}.
  \end{datadesc}

  \begin{datadesc}{Py_TPFLAGS_CHECKTYPES}
    If this bit is set, the binary and ternary operations in the
    \ctype{PyNumberMethods} structure referenced by
    \member{tp_as_number} accept arguments of arbitrary object types,
    and do their own type conversions if needed.  If this bit is
    clear, those operations require that all arguments have the
    current type as their type, and the caller is supposed to perform
    a coercion operation first.  This applies to \member{nb_add},
    \member{nb_subtract}, \member{nb_multiply}, \member{nb_divide},
    \member{nb_remainder}, \member{nb_divmod}, \member{nb_power},
    \member{nb_lshift}, \member{nb_rshift}, \member{nb_and},
    \member{nb_xor}, and \member{nb_or}.
  \end{datadesc}

  \begin{datadesc}{Py_TPFLAGS_HAVE_RICHCOMPARE}
    If this bit is set, the type object has the
    \member{tp_richcompare} field, as well as the \member{tp_traverse}
    and the \member{tp_clear} fields.
  \end{datadesc}

  \begin{datadesc}{Py_TPFLAGS_HAVE_WEAKREFS}
    If this bit is set, the \member{tp_weaklistoffset} field is
    defined.  Instances of a type are weakly referenceable if the
    type's \member{tp_weaklistoffset} field has a value greater than
    zero.
  \end{datadesc}

  \begin{datadesc}{Py_TPFLAGS_HAVE_ITER}
    If this bit is set, the type object has the \member{tp_iter} and
    \member{tp_iternext} fields.
  \end{datadesc}

  \begin{datadesc}{Py_TPFLAGS_HAVE_CLASS}
    If this bit is set, the type object has several new fields defined
    starting in Python 2.2: \member{tp_methods}, \member{tp_members},
    \member{tp_getset}, \member{tp_base}, \member{tp_dict},
    \member{tp_descr_get}, \member{tp_descr_set},
    \member{tp_dictoffset}, \member{tp_init}, \member{tp_alloc},
    \member{tp_new}, \member{tp_free}, \member{tp_is_gc},
    \member{tp_bases}, \member{tp_mro}, \member{tp_cache},
    \member{tp_subclasses}, and \member{tp_weaklist}.
  \end{datadesc}

  \begin{datadesc}{Py_TPFLAGS_HEAPTYPE}
    This bit is set when the type object itself is allocated on the
    heap.  In this case, the \member{ob_type} field of its instances
    is considered a reference to the type, and the type object is
    INCREF'ed when a new instance is created, and DECREF'ed when an
    instance is destroyed (this does not apply to instances of
    subtypes; only the type referenced by the instance's ob_type gets
    INCREF'ed or DECREF'ed).
  \end{datadesc}

  \begin{datadesc}{Py_TPFLAGS_BASETYPE}
    This bit is set when the type can be used as the base type of
    another type.  If this bit is clear, the type cannot be subtyped
    (similar to a "final" class in Java).
  \end{datadesc}

  \begin{datadesc}{Py_TPFLAGS_READY}
    This bit is set when the type object has been fully initialized by
    \cfunction{PyType_Ready()}.
  \end{datadesc}

  \begin{datadesc}{Py_TPFLAGS_READYING}
    This bit is set while \cfunction{PyType_Ready()} is in the process
    of initializing the type object.
  \end{datadesc}

  \begin{datadesc}{Py_TPFLAGS_HAVE_GC}
    This bit is set when the object supports garbage collection.  If
    this bit is set, instances must be created using
    \cfunction{PyObject_GC_New()} and destroyed using
    \cfunction{PyObject_GC_Del()}.  More information in section XXX
    about garbage collection.  This bit also implies that the
    GC-related fields \member{tp_traverse} and \member{tp_clear} are
    present in the type object; but those fields also exist when
    \constant{Py_TPFLAGS_HAVE_GC} is clear but
    \constant{Py_TPFLAGS_HAVE_RICHCOMPARE} is set).
  \end{datadesc}

  \begin{datadesc}{Py_TPFLAGS_DEFAULT}
    This is a bitmask of all the bits that pertain to the existence of
    certain fields in the type object and its extension structures.
    Currently, it includes the following bits:
    \constant{Py_TPFLAGS_HAVE_GETCHARBUFFER},
    \constant{Py_TPFLAGS_HAVE_SEQUENCE_IN},
    \constant{Py_TPFLAGS_HAVE_INPLACEOPS},
    \constant{Py_TPFLAGS_HAVE_RICHCOMPARE},
    \constant{Py_TPFLAGS_HAVE_WEAKREFS},
    \constant{Py_TPFLAGS_HAVE_ITER}, and
    \constant{Py_TPFLAGS_HAVE_CLASS}.
  \end{datadesc}
\end{cmemberdesc}

\begin{cmemberdesc}{PyTypeObject}{char*}{tp_doc}
  An optional pointer to a NUL-terminated C string giving the
  docstring for this type object.  This is exposed as the
  \member{__doc__} attribute on the type and instances of the type.

  This field is \emph{not} inherited by subtypes.
\end{cmemberdesc}

The following three fields only exist if the
\constant{Py_TPFLAGS_HAVE_RICHCOMPARE} flag bit is set.

\begin{cmemberdesc}{PyTypeObject}{traverseproc}{tp_traverse}
  An optional pointer to a traversal function for the garbage
  collector.  This is only used if the \constant{Py_TPFLAGS_HAVE_GC}
  flag bit is set.  More information in section XXX about garbage
  collection.

  This field is inherited by subtypes together with \member{tp_clear}
  and the \constant{Py_TPFLAGS_HAVE_GC} flag bit: the flag bit,
  \member{tp_traverse}, and \member{tp_clear} are all inherited from
  the base type if they are all zero in the subtype \emph{and} the
  subtype has the \constant{Py_TPFLAGS_HAVE_RICHCOMPARE} flag bit set.
\end{cmemberdesc}

\begin{cmemberdesc}{PyTypeObject}{inquiry}{tp_clear}
  An optional pointer to a clear function for the garbage collector.
  This is only used if the \constant{Py_TPFLAGS_HAVE_GC} flag bit is
  set.  More information in section XXX about garbage collection.

  This field is inherited by subtypes together with \member{tp_clear}
  and the \constant{Py_TPFLAGS_HAVE_GC} flag bit: the flag bit,
  \member{tp_traverse}, and \member{tp_clear} are all inherited from
  the base type if they are all zero in the subtype \emph{and} the
  subtype has the \constant{Py_TPFLAGS_HAVE_RICHCOMPARE} flag bit set.
\end{cmemberdesc}

\begin{cmemberdesc}{PyTypeObject}{richcmpfunc}{tp_richcompare}
  An optional pointer to the rich comparison function.

  The signature is the same as for \cfunction{PyObject_RichCompare()}.
  The function should return \code{1} if the requested comparison
  returns true, \code{0} if it returns false.  It should return
  \code{-1} and set an exception condition when an error occurred
  during the comparison.

  This field is inherited by subtypes together with
  \member{tp_compare} and \member{tp_hash}: a subtype inherits all
  three of \member{tp_compare}, \member{tp_richcompare}, and
  \member{tp_hash}, when the subtype's \member{tp_compare},
  \member{tp_richcompare}, and \member{tp_hash} are all \NULL.

  The following constants are defined to be used as the third argument
  for \member{tp_richcompare} and for \cfunction{PyObject_RichCompare()}:

  \begin{tableii}{l|c}{constant}{Constant}{Comparison}
    \lineii{Py_LT}{\code{<}}
    \lineii{Py_LE}{\code{<=}}
    \lineii{Py_EQ}{\code{==}}
    \lineii{Py_NE}{\code{!=}}
    \lineii{Py_GT}{\code{>}}
    \lineii{Py_GE}{\code{>=}}
  \end{tableii}
\end{cmemberdesc}

The next field only exists if the \constant{Py_TPFLAGS_HAVE_WEAKREFS}
flag bit is set.

\begin{cmemberdesc}{PyTypeObject}{long}{tp_weaklistoffset}
  If the instances of this type are weakly referenceable, this field
  is greater than zero and contains the offset in the instance
  structure of the weak reference list head (ignoring the GC header,
  if present); this offset is used by
  \cfunction{PyObject_ClearWeakRefs()} and the
  \cfunction{PyWeakref_*()} functions.  The instance structure needs
  to include a field of type \ctype{PyObject*} which is initialized to
  \NULL.

  Do not confuse this field with \member{tp_weaklist}; that is the
  list head for weak references to the type object itself.

  This field is inherited by subtypes, but see the rules listed below.
  A subtype may override this offset; this means that the subtype uses
  a different weak reference list head than the base type.  Since the
  list head is always found via \member{tp_weaklistoffset}, this
  should not be a problem.

  When a type defined by a class statement has no \member{__slots__}
  declaration, and none of its base types are weakly referenceable,
  the type is made weakly referenceable by adding a weak reference
  list head slot to the instance layout and setting the
  \member{tp_weaklistoffset} of that slot's offset.

  When a type's \member{__slots__} declaration contains a slot named
  \member{__weakref__}, that slot becomes the weak reference list head
  for instances of the type, and the slot's offset is stored in the
  type's \member{tp_weaklistoffset}.

  When a type's \member{__slots__} declaration does not contain a slot
  named \member{__weakref__}, the type inherits its
  \member{tp_weaklistoffset} from its base type.
\end{cmemberdesc}

The next two fields only exist if the
\constant{Py_TPFLAGS_HAVE_CLASS} flag bit is set.

\begin{cmemberdesc}{PyTypeObject}{getiterfunc}{tp_iter}
  An optional pointer to a function that returns an iterator for the
  object.  Its presence normally signals that the instances of this
  type are iterable (although sequences may be iterable without this
  function, and classic instances always have this function, even if
  they don't define an \method{__iter__()} method).

  This function has the same signature as
  \cfunction{PyObject_GetIter()}.

  This field is inherited by subtypes.
\end{cmemberdesc}

\begin{cmemberdesc}{PyTypeObject}{iternextfunc}{tp_iternext}
  An optional pointer to a function that returns the next item in an
  iterator, or raises \exception{StopIteration} when the iterator is
  exhausted.  Its presence normally signals that the instances of this
  type are iterators (although classic instances always have this
  function, even if they don't define a \method{next()} method).

  Iterator types should also define the \member{tp_iter} function, and
  that function should return the iterator instance itself (not a new
  iterator instance).

  This function has the same signature as \cfunction{PyIter_Next()}.

  This field is inherited by subtypes.
\end{cmemberdesc}

The next fields, up to and including \member{tp_weaklist}, only exist
if the \constant{Py_TPFLAGS_HAVE_CLASS} flag bit is set.

\begin{cmemberdesc}{PyTypeObject}{struct PyMethodDef*}{tp_methods}
  An optional pointer to a static \NULL-terminated array of
  \ctype{PyMethodDef} structures, declaring regular methods of this
  type.

  For each entry in the array, an entry is added to the type's
  dictionary (see \member{tp_dict} below) containing a method
  descriptor.

  This field is not inherited by subtypes (methods are
  inherited through a different mechanism).
\end{cmemberdesc}

\begin{cmemberdesc}{PyTypeObject}{struct PyMemberDef*}{tp_members}
  An optional pointer to a static \NULL-terminated array of
  \ctype{PyMemberDef} structures, declaring regular data members
  (fields or slots) of instances of this type.

  For each entry in the array, an entry is added to the type's
  dictionary (see \member{tp_dict} below) containing a member
  descriptor.

  This field is not inherited by subtypes (members are inherited
  through a different mechanism).
\end{cmemberdesc}

\begin{cmemberdesc}{PyTypeObject}{struct PyGetSetDef*}{tp_getset}
  An optional pointer to a static \NULL-terminated array of
  \ctype{PyGetSetDef} structures, declaring computed attributes of
  instances of this type.

  For each entry in the array, an entry is added to the type's
  dictionary (see \member{tp_dict} below) containing a getset
  descriptor.

  This field is not inherited by subtypes (computed attributes are
  inherited through a different mechanism).

  Docs for PyGetSetDef (XXX belong elsewhere):

\begin{verbatim}
typedef PyObject *(*getter)(PyObject *, void *);
typedef int (*setter)(PyObject *, PyObject *, void *);

typedef struct PyGetSetDef {
    char *name;    /* attribute name */
    getter get;    /* C function to get the attribute */
    setter set;    /* C function to set the attribute */
    char *doc;     /* optional doc string */
    void *closure; /* optional additional data for getter and setter */
} PyGetSetDef;
\end{verbatim}
\end{cmemberdesc}

\begin{cmemberdesc}{PyTypeObject}{PyTypeObject*}{tp_base}
  An optional pointer to a base type from which type properties are
  inherited.  At this level, only single inheritance is supported;
  multiple inheritance require dynamically creating a type object by
  calling the metatype.

  This field is not inherited by subtypes (obviously), but it defaults
  to \code{\&PyBaseObject_Type} (which to Python programmers is known
  as the type \class{object}).
\end{cmemberdesc}

\begin{cmemberdesc}{PyTypeObject}{PyObject*}{tp_dict}
  The type's dictionary is stored here by \cfunction{PyType_Ready()}.

  This field should normally be initialized to \NULL{} before
  PyType_Ready is called; it may also be initialized to a dictionary
  containing initial attributes for the type.  Once
  \cfunction{PyType_Ready()} has initialized the type, extra
  attributes for the type may be added to this dictionary only if they
  don't correspond to overloaded operations (like \method{__add__()}).

  This field is not inherited by subtypes (though the attributes
  defined in here are inherited through a different mechanism).
\end{cmemberdesc}

\begin{cmemberdesc}{PyTypeObject}{descrgetfunc}{tp_descr_get}
  An optional pointer to a "descriptor get" function.

  XXX blah, blah.

  This field is inherited by subtypes.
\end{cmemberdesc}

\begin{cmemberdesc}{PyTypeObject}{descrsetfunc}{tp_descr_set}
  An optional pointer to a "descriptor set" function.

  XXX blah, blah.

  This field is inherited by subtypes.
\end{cmemberdesc}

\begin{cmemberdesc}{PyTypeObject}{long}{tp_dictoffset}
  If the instances of this type have a dictionary containing instance
  variables, this field is non-zero and contains the offset in the
  instances of the type of the instance variable dictionary; this
  offset is used by \cfunction{PyObject_GenericGetAttr()}.

  Do not confuse this field with \member{tp_dict}; that is the
  dictionary for attributes of the type object itself.

  If the value of this field is greater than zero, it specifies the
  offset from the start of the instance structure.  If the value is
  less than zero, it specifies the offset from the *end* of the
  instance structure.  A negative offset is more expensive to use, and
  should only be used when the instance structure contains a
  variable-length part.  This is used for example to add an instance
  variable dictionary to subtypes of \class{str} or \class{tuple}.
  Note that the \member{tp_basicsize} field should account for the
  dictionary added to the end in that case, even though the dictionary
  is not included in the basic object layout.  On a system with a
  pointer size of 4 bytes, \member{tp_dictoffset} should be set to
  \code{-4} to indicate that the dictionary is at the very end of the
  structure.

  The real dictionary offset in an instance can be computed from a
  negative \member{tp_dictoffset} as follows:

\begin{verbatim}
dictoffset = tp_basicsize + abs(ob_size)*tp_itemsize + tp_dictoffset
if dictoffset is not aligned on sizeof(void*):
    round up to sizeof(void*)
\end{verbatim}

  where \member{tp_basicsize}, \member{tp_itemsize} and
  \member{tp_dictoffset} are taken from the type object, and
  \member{ob_size} is taken from the instance.  The absolute value is
  taken because long ints use the sign of \member{ob_size} to store
  the sign of the number.  (There's never a need to do this
  calculation yourself; it is done for you by
  \cfunction{_PyObject_GetDictPtr()}.)

  This field is inherited by subtypes, but see the rules listed below.
  A subtype may override this offset; this means that the subtype
  instances store the dictionary at a difference offset than the base
  type.  Since the dictionary is always found via
  \member{tp_dictoffset}, this should not be a problem.

  When a type defined by a class statement has no \member{__slots__}
  declaration, and none of its base types has an instance variable
  dictionary, a dictionary slot is added to the instance layout and
  the \member{tp_dictoffset} is set to that slot's offset.

  When a type defined by a class statement has a \member{__slots__}
  declaration, the type inherits its \member{tp_dictoffset} from its
  base type.

  (Adding a slot named \member{__dict__} to the \member{__slots__}
  declaration does not have the expected effect, it just causes
  confusion.  Maybe this should be added as a feature just like
  \member{__weakref__} though.)
\end{cmemberdesc}

\begin{cmemberdesc}{PyTypeObject}{initproc}{tp_init}
  An optional pointer to an instance initialization function.

  This function corresponds to the \method{__init__()} method of
  classes.  Like \method{__init__()}, it is possible to create an
  instance without calling \method{__init__()}, and it is possible to
  reinitialize an instance by calling its \method{__init__()} method
  again.

  The function signature is

\begin{verbatim}
int tp_init(PyObject *self, PyObject *args, PyObject *kwds)
\end{verbatim}

  The self argument is the instance to be initialized; the \var{args}
  and \var{kwds} arguments represent positional and keyword arguments
  of the call to \method{__init__()}.

  The \member{tp_init} function, if not \NULL, is called when an
  instance is created normally by calling its type, after the type's
  \member{tp_new} function has returned an instance of the type.  If
  the \member{tp_new} function returns an instance of some other type
  that is not a subtype of the original type, no \member{tp_init}
  function is called; if \member{tp_new} returns an instance of a
  subtype of the original type, the subtype's \member{tp_init} is
  called.  (VERSION NOTE: described here is what is implemented in
  Python 2.2.1 and later.  In Python 2.2, the \member{tp_init} of the
  type of the object returned by \member{tp_new} was always called, if
  not \NULL.)

  This field is inherited by subtypes.
\end{cmemberdesc}

\begin{cmemberdesc}{PyTypeObject}{allocfunc}{tp_alloc}
  An optional pointer to an instance allocation function.

  The function signature is

\begin{verbatim}
PyObject *tp_alloc(PyTypeObject *self, int nitems)
\end{verbatim}

  The purpose of this function is to separate memory allocation from
  memory initialization.  It should return a pointer to a block of
  memory of adequate length for the instance, suitably aligned, and
  initialized to zeros, but with \member{ob_refcnt} set to \code{1}
  and \member{ob_type} set to the type argument.  If the type's
  \member{tp_itemsize} is non-zero, the object's \member{ob_size} field
  should be initialized to \var{nitems} and the length of the
  allocated memory block should be \code{tp_basicsize +
  \var{nitems}*tp_itemsize}, rounded up to a multiple of
  \code{sizeof(void*)}; otherwise, \var{nitems} is not used and the
  length of the block should be \member{tp_basicsize}.

  Do not use this function to do any other instance initialization,
  not even to allocate additional memory; that should be done by
  \member{tp_new}.

  This field is inherited by static subtypes, but not by dynamic
  subtypes (subtypes created by a class statement); in the latter,
  this field is always set to \cfunction{PyType_GenericAlloc()}, to
  force a standard heap allocation strategy.  That is also the
  recommended value for statically defined types.
\end{cmemberdesc}

\begin{cmemberdesc}{PyTypeObject}{newfunc}{tp_new}
  An optional pointer to an instance creation function.

  If this function is \NULL{} for a particular type, that type cannot
  be called to create new instances; presumably there is some other
  way to create instances, like a factory function.

  The function signature is

\begin{verbatim}
PyObject *tp_new(PyTypeObject *subtype, PyObject *args, PyObject *kwds)
\end{verbatim}

  The subtype argument is the type of the object being created; the
  \var{args} and \var{kwds} arguments represent positional and keyword
  arguments of the call to the type.  Note that subtype doesn't have
  to equal the type whose \member{tp_new} function is called; it may
  be a subtype of that type (but not an unrelated type).

  The \member{tp_new} function should call
  \code{\var{subtype}->tp_alloc(\var{subtype}, \var{nitems})} to
  allocate space for the object, and then do only as much further
  initialization as is absolutely necessary.  Initialization that can
  safely be ignored or repeated should be placed in the
  \member{tp_init} handler.  A good rule of thumb is that for
  immutable types, all initialization should take place in
  \member{tp_new}, while for mutable types, most initialization should
  be deferred to \member{tp_init}.

  This field is inherited by subtypes, except it is not inherited by
  static types whose \member{tp_base} is \NULL{} or
  \code{\&PyBaseObject_Type}.  The latter exception is a precaution so
  that old extension types don't become callable simply by being
  linked with Python 2.2.
\end{cmemberdesc}

\begin{cmemberdesc}{PyTypeObject}{destructor}{tp_free}
  An optional pointer to an instance deallocation function.

  The signature of this function has changed slightly: in Python
  2.2 and 2.2.1, its signature is \ctype{destructor}:

\begin{verbatim}
void tp_free(PyObject *)
\end{verbatim}

  In Python 2.3 and beyond, its signature is \ctype{freefunc}:

\begin{verbatim}
void tp_free(void *)
\end{verbatim}

  The only initializer that is compatible with both versions is
  \code{_PyObject_Del}, whose definition has suitably adapted in
  Python 2.3.

  This field is inherited by static subtypes, but not by dynamic
  subtypes (subtypes created by a class statement); in the latter,
  this field is set to a deallocator suitable to match
  \cfunction{PyType_GenericAlloc()} and the value of the
  \constant{Py_TPFLAGS_HAVE_GC} flag bit.
\end{cmemberdesc}

\begin{cmemberdesc}{PyTypeObject}{inquiry}{tp_is_gc}
  An optional pointer to a function called by the garbage collector.

  The garbage collector needs to know whether a particular object is
  collectible or not.  Normally, it is sufficient to look at the
  object's type's \member{tp_flags} field, and check the
  \constant{Py_TPFLAGS_HAVE_GC} flag bit.  But some types have a
  mixture of statically and dynamically allocated instances, and the
  statically allocated instances are not collectible.  Such types
  should define this function; it should return \code{1} for a
  collectible instance, and \code{0} for a non-collectible instance.
  The signature is

\begin{verbatim}
int tp_is_gc(PyObject *self)
\end{verbatim}

  (The only example of this are types themselves.  The metatype,
  \cdata{PyType_Type}, defines this function to distinguish between
  statically and dynamically allocated types.)

  This field is inherited by subtypes.  (VERSION NOTE: in Python
  2.2, it was not inherited.  It is inherited in 2.2.1 and later
  versions.)
\end{cmemberdesc}

\begin{cmemberdesc}{PyTypeObject}{PyObject*}{tp_bases}
  Tuple of base types.

  This is set for types created by a class statement.  It should be
  \NULL{} for statically defined types.

  This field is not inherited.
\end{cmemberdesc}

\begin{cmemberdesc}{PyTypeObject}{PyObject*}{tp_mro}
  Tuple containing the expanded set of base types, starting with the
  type itself and ending with \class{object}, in Method Resolution
  Order.

  This field is not inherited; it is calculated fresh by
  \cfunction{PyType_Ready()}.
\end{cmemberdesc}

\begin{cmemberdesc}{PyTypeObject}{PyObject*}{tp_cache}
  Unused.  Not inherited.  Internal use only.
\end{cmemberdesc}

\begin{cmemberdesc}{PyTypeObject}{PyObject*}{tp_subclasses}
  List of weak references to subclasses.  Not inherited.  Internal
  use only.
\end{cmemberdesc}

\begin{cmemberdesc}{PyTypeObject}{PyObject*}{tp_weaklist}
  Weak reference list head, for weak references to this type
  object.  Not inherited.  Internal use only.
\end{cmemberdesc}

The remaining fields are only defined if the feature test macro
\constant{COUNT_ALLOCS} is defined, and are for internal use only.
They are documented here for completeness.  None of these fields are
inherited by subtypes.

\begin{cmemberdesc}{PyTypeObject}{int}{tp_allocs}
  Number of allocations.
\end{cmemberdesc}

\begin{cmemberdesc}{PyTypeObject}{int}{tp_frees}
  Number of frees.
\end{cmemberdesc}

\begin{cmemberdesc}{PyTypeObject}{int}{tp_maxalloc}
  Maximum simultaneously allocated objects.
\end{cmemberdesc}

\begin{cmemberdesc}{PyTypeObject}{PyTypeObject*}{tp_next}
  Pointer to the next type object with a non-zero \member{tp_allocs}
  field.
\end{cmemberdesc}


\section{Mapping Object Structures \label{mapping-structs}}

\begin{ctypedesc}{PyMappingMethods}
  Structure used to hold pointers to the functions used to implement
  the mapping protocol for an extension type.
\end{ctypedesc}


\section{Number Object Structures \label{number-structs}}

\begin{ctypedesc}{PyNumberMethods}
  Structure used to hold pointers to the functions an extension type
  uses to implement the number protocol.
\end{ctypedesc}


\section{Sequence Object Structures \label{sequence-structs}}

\begin{ctypedesc}{PySequenceMethods}
  Structure used to hold pointers to the functions which an object
  uses to implement the sequence protocol.
\end{ctypedesc}


\section{Buffer Object Structures \label{buffer-structs}}
\sectionauthor{Greg J. Stein}{greg@lyra.org}

The buffer interface exports a model where an object can expose its
internal data as a set of chunks of data, where each chunk is
specified as a pointer/length pair.  These chunks are called
\dfn{segments} and are presumed to be non-contiguous in memory.

If an object does not export the buffer interface, then its
\member{tp_as_buffer} member in the \ctype{PyTypeObject} structure
should be \NULL.  Otherwise, the \member{tp_as_buffer} will point to
a \ctype{PyBufferProcs} structure.

\note{It is very important that your \ctype{PyTypeObject} structure
uses \constant{Py_TPFLAGS_DEFAULT} for the value of the
\member{tp_flags} member rather than \code{0}.  This tells the Python
runtime that your \ctype{PyBufferProcs} structure contains the
\member{bf_getcharbuffer} slot. Older versions of Python did not have
this member, so a new Python interpreter using an old extension needs
to be able to test for its presence before using it.}

\begin{ctypedesc}{PyBufferProcs}
  Structure used to hold the function pointers which define an
  implementation of the buffer protocol.

  The first slot is \member{bf_getreadbuffer}, of type
  \ctype{getreadbufferproc}.  If this slot is \NULL, then the object
  does not support reading from the internal data.  This is
  non-sensical, so implementors should fill this in, but callers
  should test that the slot contains a non-\NULL{} value.

  The next slot is \member{bf_getwritebuffer} having type
  \ctype{getwritebufferproc}.  This slot may be \NULL{} if the object
  does not allow writing into its returned buffers.

  The third slot is \member{bf_getsegcount}, with type
  \ctype{getsegcountproc}.  This slot must not be \NULL{} and is used
  to inform the caller how many segments the object contains.  Simple
  objects such as \ctype{PyString_Type} and \ctype{PyBuffer_Type}
  objects contain a single segment.

  The last slot is \member{bf_getcharbuffer}, of type
  \ctype{getcharbufferproc}.  This slot will only be present if the
  \constant{Py_TPFLAGS_HAVE_GETCHARBUFFER} flag is present in the
  \member{tp_flags} field of the object's \ctype{PyTypeObject}.
  Before using this slot, the caller should test whether it is present
  by using the
  \cfunction{PyType_HasFeature()}\ttindex{PyType_HasFeature()}
  function.  If present, it may be \NULL, indicating that the object's
  contents cannot be used as \emph{8-bit characters}.
  The slot function may also raise an error if the object's contents
  cannot be interpreted as 8-bit characters.  For example, if the
  object is an array which is configured to hold floating point
  values, an exception may be raised if a caller attempts to use
  \member{bf_getcharbuffer} to fetch a sequence of 8-bit characters.
  This notion of exporting the internal buffers as ``text'' is used to
  distinguish between objects that are binary in nature, and those
  which have character-based content.

  \note{The current policy seems to state that these characters
  may be multi-byte characters. This implies that a buffer size of
  \var{N} does not mean there are \var{N} characters present.}
\end{ctypedesc}

\begin{datadesc}{Py_TPFLAGS_HAVE_GETCHARBUFFER}
  Flag bit set in the type structure to indicate that the
  \member{bf_getcharbuffer} slot is known.  This being set does not
  indicate that the object supports the buffer interface or that the
  \member{bf_getcharbuffer} slot is non-\NULL.
\end{datadesc}

\begin{ctypedesc}[getreadbufferproc]{int (*getreadbufferproc)
                            (PyObject *self, int segment, void **ptrptr)}
  Return a pointer to a readable segment of the buffer.  This function
  is allowed to raise an exception, in which case it must return
  \code{-1}.  The \var{segment} which is passed must be zero or
  positive, and strictly less than the number of segments returned by
  the \member{bf_getsegcount} slot function.  On success, it returns
  the length of the buffer memory, and sets \code{*\var{ptrptr}} to a
  pointer to that memory.
\end{ctypedesc}

\begin{ctypedesc}[getwritebufferproc]{int (*getwritebufferproc)
                            (PyObject *self, int segment, void **ptrptr)}
  Return a pointer to a writable memory buffer in
  \code{*\var{ptrptr}}, and the length of that segment as the function
  return value.  The memory buffer must correspond to buffer segment
  \var{segment}.  Must return \code{-1} and set an exception on
  error.  \exception{TypeError} should be raised if the object only
  supports read-only buffers, and \exception{SystemError} should be
  raised when \var{segment} specifies a segment that doesn't exist.
% Why doesn't it raise ValueError for this one?
% GJS: because you shouldn't be calling it with an invalid
%      segment. That indicates a blatant programming error in the C
%      code.
\end{ctypedesc}

\begin{ctypedesc}[getsegcountproc]{int (*getsegcountproc)
                            (PyObject *self, int *lenp)}
  Return the number of memory segments which comprise the buffer.  If
  \var{lenp} is not \NULL, the implementation must report the sum of
  the sizes (in bytes) of all segments in \code{*\var{lenp}}.
  The function cannot fail.
\end{ctypedesc}

\begin{ctypedesc}[getcharbufferproc]{int (*getcharbufferproc)
                            (PyObject *self, int segment, const char **ptrptr)}
  Return the size of the memory buffer in \var{ptrptr} for segment
  \var{segment}.  \code{*\var{ptrptr}} is set to the memory buffer.
\end{ctypedesc}


\section{Supporting the Iterator Protocol
         \label{supporting-iteration}}


\section{Supporting Cyclic Garbarge Collection
         \label{supporting-cycle-detection}}

Python's support for detecting and collecting garbage which involves
circular references requires support from object types which are
``containers'' for other objects which may also be containers.  Types
which do not store references to other objects, or which only store
references to atomic types (such as numbers or strings), do not need
to provide any explicit support for garbage collection.

An example showing the use of these interfaces can be found in
``\ulink{Supporting the Cycle
Collector}{../ext/example-cycle-support.html}'' in
\citetitle[../ext/ext.html]{Extending and Embedding the Python
Interpreter}.

To create a container type, the \member{tp_flags} field of the type
object must include the \constant{Py_TPFLAGS_HAVE_GC} and provide an
implementation of the \member{tp_traverse} handler.  If instances of the
type are mutable, a \member{tp_clear} implementation must also be
provided.

\begin{datadesc}{Py_TPFLAGS_HAVE_GC}
  Objects with a type with this flag set must conform with the rules
  documented here.  For convenience these objects will be referred to
  as container objects.
\end{datadesc}

Constructors for container types must conform to two rules:

\begin{enumerate}
\item  The memory for the object must be allocated using
       \cfunction{PyObject_GC_New()} or \cfunction{PyObject_GC_VarNew()}.

\item  Once all the fields which may contain references to other
       containers are initialized, it must call
       \cfunction{PyObject_GC_Track()}.
\end{enumerate}

\begin{cfuncdesc}{\var{TYPE}*}{PyObject_GC_New}{TYPE, PyTypeObject *type}
  Analogous to \cfunction{PyObject_New()} but for container objects with
  the \constant{Py_TPFLAGS_HAVE_GC} flag set.
\end{cfuncdesc}

\begin{cfuncdesc}{\var{TYPE}*}{PyObject_GC_NewVar}{TYPE, PyTypeObject *type,
                                                   int size}
  Analogous to \cfunction{PyObject_NewVar()} but for container objects
  with the \constant{Py_TPFLAGS_HAVE_GC} flag set.
\end{cfuncdesc}

\begin{cfuncdesc}{PyVarObject *}{PyObject_GC_Resize}{PyVarObject *op, int}
  Resize an object allocated by \cfunction{PyObject_NewVar()}.  Returns
  the resized object or \NULL{} on failure.
\end{cfuncdesc}

\begin{cfuncdesc}{void}{PyObject_GC_Track}{PyObject *op}
  Adds the object \var{op} to the set of container objects tracked by
  the collector.  The collector can run at unexpected times so objects
  must be valid while being tracked.  This should be called once all
  the fields followed by the \member{tp_traverse} handler become valid,
  usually near the end of the constructor.
\end{cfuncdesc}

\begin{cfuncdesc}{void}{_PyObject_GC_TRACK}{PyObject *op}
  A macro version of \cfunction{PyObject_GC_Track()}.  It should not be
  used for extension modules.
\end{cfuncdesc}

Similarly, the deallocator for the object must conform to a similar
pair of rules:

\begin{enumerate}
\item  Before fields which refer to other containers are invalidated,
       \cfunction{PyObject_GC_UnTrack()} must be called.

\item  The object's memory must be deallocated using
       \cfunction{PyObject_GC_Del()}.
\end{enumerate}

\begin{cfuncdesc}{void}{PyObject_GC_Del}{PyObject *op}
  Releases memory allocated to an object using
  \cfunction{PyObject_GC_New()} or \cfunction{PyObject_GC_NewVar()}.
\end{cfuncdesc}

\begin{cfuncdesc}{void}{PyObject_GC_UnTrack}{PyObject *op}
  Remove the object \var{op} from the set of container objects tracked
  by the collector.  Note that \cfunction{PyObject_GC_Track()} can be
  called again on this object to add it back to the set of tracked
  objects.  The deallocator (\member{tp_dealloc} handler) should call
  this for the object before any of the fields used by the
  \member{tp_traverse} handler become invalid.
\end{cfuncdesc}

\begin{cfuncdesc}{void}{_PyObject_GC_UNTRACK}{PyObject *op}
  A macro version of \cfunction{PyObject_GC_UnTrack()}.  It should not be
  used for extension modules.
\end{cfuncdesc}

The \member{tp_traverse} handler accepts a function parameter of this
type:

\begin{ctypedesc}[visitproc]{int (*visitproc)(PyObject *object, void *arg)}
  Type of the visitor function passed to the \member{tp_traverse}
  handler.  The function should be called with an object to traverse
  as \var{object} and the third parameter to the \member{tp_traverse}
  handler as \var{arg}.
\end{ctypedesc}

The \member{tp_traverse} handler must have the following type:

\begin{ctypedesc}[traverseproc]{int (*traverseproc)(PyObject *self,
                                visitproc visit, void *arg)}
  Traversal function for a container object.  Implementations must
  call the \var{visit} function for each object directly contained by
  \var{self}, with the parameters to \var{visit} being the contained
  object and the \var{arg} value passed to the handler.  If
  \var{visit} returns a non-zero value then an error has occurred and
  that value should be returned immediately.
\end{ctypedesc}

The \member{tp_clear} handler must be of the \ctype{inquiry} type, or
\NULL{} if the object is immutable.

\begin{ctypedesc}[inquiry]{int (*inquiry)(PyObject *self)}
  Drop references that may have created reference cycles.  Immutable
  objects do not have to define this method since they can never
  directly create reference cycles.  Note that the object must still
  be valid after calling this method (don't just call
  \cfunction{Py_DECREF()} on a reference).  The collector will call
  this method if it detects that this object is involved in a
  reference cycle.
\end{ctypedesc}

\chapter{Building C and \Cpp{} Extensions with distutils
     \label{building}}

\sectionauthor{Martin v. L\"owis}{martin@v.loewis.de}

Starting in Python 1.4, Python provides, on \UNIX{}, a special make
file for building make files for building dynamically-linked
extensions and custom interpreters.  Starting with Python 2.0, this
mechanism (known as related to Makefile.pre.in, and Setup files) is no
longer supported. Building custom interpreters was rarely used, and
extensions modules can be build using distutils.

Building an extension module using distutils requires that distutils
is installed on the build machine, which is included in Python 2.x and
available separately for Python 1.5. Since distutils also supports
creation of binary packages, users don't necessarily need a compiler
and distutils to install the extension.

A distutils package contains a driver script, \file{setup.py}. This is
a plain Python file, which, in the most simple case, could look like
this:

\begin{verbatim}
from distutils.core import setup, Extension

module1 = Extension('demo',
                    sources = ['demo.c'])

setup (name = 'PackageName',
       version = '1.0',
       description = 'This is a demo package',
       ext_modules = [module1])

\end{verbatim}

With this \file{setup.py}, and a file \file{demo.c}, running

\begin{verbatim}
python setup.py build 
\end{verbatim}

will compile \file{demo.c}, and produce an extension module named
\samp{demo} in the \file{build} directory. Depending on the system,
the module file will end up in a subdirectory \file{build/lib.system},
and may have a name like \file{demo.so} or \file{demo.pyd}.

In the \file{setup.py}, all execution is performed by calling the
\samp{setup} function. This takes a variable number of keyword 
arguments, of which the example above uses only a
subset. Specifically, the example specifies meta-information to build
packages, and it specifies the contents of the package.  Normally, a
package will contain of addition modules, like Python source modules,
documentation, subpackages, etc. Please refer to the distutils
documentation in \citetitle[../dist/dist.html]{Distributing Python
Modules} to learn more about the features of distutils; this section
explains building extension modules only.

It is common to pre-compute arguments to \function{setup}, to better
structure the driver script. In the example above,
the\samp{ext_modules} argument to \function{setup} is a list of
extension modules, each of which is an instance of the
\class{Extension}. In the example, the instance defines an extension
named \samp{demo} which is build by compiling a single source file,
\file{demo.c}.

In many cases, building an extension is more complex, since additional
preprocessor defines and libraries may be needed. This is demonstrated
in the example below.

\begin{verbatim}
from distutils.core import setup, Extension

module1 = Extension('demo',
                    define_macros = [('MAJOR_VERSION', '1'),
                                     ('MINOR_VERSION', '0')],
                    include_dirs = ['/usr/local/include'],
                    libraries = ['tcl83'],
                    library_dirs = ['/usr/local/lib'],
                    sources = ['demo.c'])

setup (name = 'PackageName',
       version = '1.0',
       description = 'This is a demo package',
       author = 'Martin v. Loewis',
       author_email = 'martin@v.loewis.de',
       url = 'http://www.python.org/doc/current/ext/building.html',
       long_description = '''
This is really just a demo package.
''',
       ext_modules = [module1])

\end{verbatim}

In this example, \function{setup} is called with additional
meta-information, which is recommended when distribution packages have
to be built. For the extension itself, it specifies preprocessor
defines, include directories, library directories, and libraries.
Depending on the compiler, distutils passes this information in
different ways to the compiler. For example, on \UNIX{}, this may
result in the compilation commands

\begin{verbatim}
gcc -DNDEBUG -g -O3 -Wall -Wstrict-prototypes -fPIC -DMAJOR_VERSION=1 -DMINOR_VERSION=0 -I/usr/local/include -I/usr/local/include/python2.2 -c demo.c -o build/temp.linux-i686-2.2/demo.o

gcc -shared build/temp.linux-i686-2.2/demo.o -L/usr/local/lib -ltcl83 -o build/lib.linux-i686-2.2/demo.so
\end{verbatim}

These lines are for demonstration purposes only; distutils users
should trust that distutils gets the invocations right.

\section{Distributing your extension modules
     \label{distributing}}

When an extension has been successfully build, there are three ways to
use it.

End-users will typically want to install the module, they do so by
running

\begin{verbatim}
python setup.py install
\end{verbatim}

Module maintainers should produce source packages; to do so, they run

\begin{verbatim}
python setup.py sdist
\end{verbatim}

In some cases, additional files need to be included in a source
distribution; this is done through a \file{MANIFEST.in} file; see the
distutils documentation for details.

If the source distribution has been build successfully, maintainers
can also create binary distributions. Depending on the platform, one
of the following commands can be used to do so.

\begin{verbatim}
python setup.py bdist_wininst
python setup.py bdist_rpm
python setup.py bdist_dumb
\end{verbatim}


\chapter{MS Windows Specific Modules}


This chapter describes modules that are only available on MS Windows
platforms.


\localmoduletable

\chapter{Embedding Python in Another Application
     \label{embedding}}

The previous chapters discussed how to extend Python, that is, how to
extend the functionality of Python by attaching a library of C
functions to it.  It is also possible to do it the other way around:
enrich your C/\Cpp{} application by embedding Python in it.  Embedding
provides your application with the ability to implement some of the
functionality of your application in Python rather than C or \Cpp.
This can be used for many purposes; one example would be to allow
users to tailor the application to their needs by writing some scripts
in Python.  You can also use it yourself if some of the functionality
can be written in Python more easily.

Embedding Python is similar to extending it, but not quite.  The
difference is that when you extend Python, the main program of the
application is still the Python interpreter, while if you embed
Python, the main program may have nothing to do with Python ---
instead, some parts of the application occasionally call the Python
interpreter to run some Python code.

So if you are embedding Python, you are providing your own main
program.  One of the things this main program has to do is initialize
the Python interpreter.  At the very least, you have to call the
function \cfunction{Py_Initialize()} (on Mac OS, call
\cfunction{PyMac_Initialize()} instead).  There are optional calls to
pass command line arguments to Python.  Then later you can call the
interpreter from any part of the application.

There are several different ways to call the interpreter: you can pass
a string containing Python statements to
\cfunction{PyRun_SimpleString()}, or you can pass a stdio file pointer
and a file name (for identification in error messages only) to
\cfunction{PyRun_SimpleFile()}.  You can also call the lower-level
operations described in the previous chapters to construct and use
Python objects.

A simple demo of embedding Python can be found in the directory
\file{Demo/embed/} of the source distribution.


\begin{seealso}
  \seetitle[../api/api.html]{Python/C API Reference Manual}{The
            details of Python's C interface are given in this manual.
            A great deal of necessary information can be found here.}
\end{seealso}


\section{Very High Level Embedding
         \label{high-level-embedding}}

The simplest form of embedding Python is the use of the very
high level interface. This interface is intended to execute a
Python script without needing to interact with the application
directly. This can for example be used to perform some operation
on a file.

\begin{verbatim}
#include <Python.h>

int
main(int argc, char *argv[])
{
  Py_Initialize();
  PyRun_SimpleString("from time import time,ctime\n"
                     "print 'Today is',ctime(time())\n");
  Py_Finalize();
  return 0;
}
\end{verbatim}

The above code first initializes the Python interpreter with
\cfunction{Py_Initialize()}, followed by the execution of a hard-coded
Python script that print the date and time.  Afterwards, the
\cfunction{Py_Finalize()} call shuts the interpreter down, followed by
the end of the program.  In a real program, you may want to get the
Python script from another source, perhaps a text-editor routine, a
file, or a database.  Getting the Python code from a file can better
be done by using the \cfunction{PyRun_SimpleFile()} function, which
saves you the trouble of allocating memory space and loading the file
contents.


\section{Beyond Very High Level Embedding: An overview
         \label{lower-level-embedding}}

The high level interface gives you the ability to execute
arbitrary pieces of Python code from your application, but
exchanging data values is quite cumbersome to say the least. If
you want that, you should use lower level calls. At the cost of
having to write more C code, you can achieve almost anything.

It should be noted that extending Python and embedding Python
is quite the same activity, despite the different intent. Most
topics discussed in the previous chapters are still valid. To
show this, consider what the extension code from Python to C
really does:

\begin{enumerate}
    \item Convert data values from Python to C,
    \item Perform a function call to a C routine using the
        converted values, and
    \item Convert the data values from the call from C to Python.
\end{enumerate}

When embedding Python, the interface code does:

\begin{enumerate}
    \item Convert data values from C to Python,
    \item Perform a function call to a Python interface routine
        using the converted values, and
    \item Convert the data values from the call from Python to C.
\end{enumerate}

As you can see, the data conversion steps are simply swapped to
accomodate the different direction of the cross-language transfer.
The only difference is the routine that you call between both
data conversions. When extending, you call a C routine, when
embedding, you call a Python routine.

This chapter will not discuss how to convert data from Python
to C and vice versa.  Also, proper use of references and dealing
with errors is assumed to be understood.  Since these aspects do not
differ from extending the interpreter, you can refer to earlier
chapters for the required information.


\section{Pure Embedding
         \label{pure-embedding}}

The first program aims to execute a function in a Python
script. Like in the section about the very high level interface,
the Python interpreter does not directly interact with the
application (but that will change in th next section).

The code to run a function defined in a Python script is:

\verbatiminput{run-func.c}

This code loads a Python script using \code{argv[1]}, and calls the
function named in \code{argv[2]}.  Its integer arguments are the other
values of the \code{argv} array.  If you compile and link this
program (let's call the finished executable \program{call}), and use
it to execute a Python script, such as:

\begin{verbatim}
def multiply(a,b):
    print "Thy shall add", a, "times", b
    c = 0
    for i in range(0, a):
        c = c + b
    return c
\end{verbatim}

then the result should be:

\begin{verbatim}
$ call multiply 3 2
Thy shall add 3 times 2
Result of call: 6
\end{verbatim} % $

Although the program is quite large for its functionality, most of the
code is for data conversion between Python and C, and for error
reporting.  The interesting part with respect to embedding Python
starts with

\begin{verbatim}
    Py_Initialize();
    pName = PyString_FromString(argv[1]);
    /* Error checking of pName left out */
    pModule = PyImport_Import(pName);
\end{verbatim}

After initializing the interpreter, the script is loaded using
\cfunction{PyImport_Import()}.  This routine needs a Python string
as its argument, which is constructed using the
\cfunction{PyString_FromString()} data conversion routine.

\begin{verbatim}
    pFunc = PyObject_GetAttrString(pModule, argv[2]);
    /* pFunc is a new reference */

    if (pFunc && PyCallable_Check(pFunc)) {
        ...
    }
    Py_XDECREF(pFunc);
\end{verbatim}

Once the script is loaded, the name we're looking for is retrieved
using \cfunction{PyObject_GetAttrString()}.  If the name exists, and
the object retunred is callable, you can safely assume that it is a
function.  The program then proceeds by constructing a tuple of
arguments as normal.  The call to the Python function is then made
with:

\begin{verbatim}
    pValue = PyObject_CallObject(pFunc, pArgs);
\end{verbatim}

Upon return of the function, \code{pValue} is either \NULL{} or it
contains a reference to the return value of the function.  Be sure to
release the reference after examining the value.


\section{Extending Embedded Python
         \label{extending-with-embedding}}

Until now, the embedded Python interpreter had no access to
functionality from the application itself.  The Python API allows this
by extending the embedded interpreter.  That is, the embedded
interpreter gets extended with routines provided by the application.
While it sounds complex, it is not so bad.  Simply forget for a while
that the application starts the Python interpreter.  Instead, consider
the application to be a set of subroutines, and write some glue code
that gives Python access to those routines, just like you would write
a normal Python extension.  For example:

\begin{verbatim}
static int numargs=0;

/* Return the number of arguments of the application command line */
static PyObject*
emb_numargs(PyObject *self, PyObject *args)
{
    if(!PyArg_ParseTuple(args, ":numargs"))
        return NULL;
    return Py_BuildValue("i", numargs);
}

static PyMethodDef EmbMethods[] = {
    {"numargs", emb_numargs, METH_VARARGS,
     "Return the number of arguments received by the process."},
    {NULL, NULL, 0, NULL}
};
\end{verbatim}

Insert the above code just above the \cfunction{main()} function.
Also, insert the following two statements directly after
\cfunction{Py_Initialize()}:

\begin{verbatim}
    numargs = argc;
    Py_InitModule("emb", EmbMethods);
\end{verbatim}

These two lines initialize the \code{numargs} variable, and make the
\function{emb.numargs()} function accessible to the embedded Python
interpreter.  With these extensions, the Python script can do things
like

\begin{verbatim}
import emb
print "Number of arguments", emb.numargs()
\end{verbatim}

In a real application, the methods will expose an API of the
application to Python.


%\section{For the future}
%
%You don't happen to have a nice library to get textual
%equivalents of numeric values do you :-) ?
%Callbacks here ? (I may be using information from that section
%?!)
%threads
%code examples do not really behave well if errors happen
% (what to watch out for)


\section{Embedding Python in \Cpp
     \label{embeddingInCplusplus}}

It is also possible to embed Python in a \Cpp{} program; precisely how this
is done will depend on the details of the \Cpp{} system used; in general you
will need to write the main program in \Cpp, and use the \Cpp{} compiler
to compile and link your program.  There is no need to recompile Python
itself using \Cpp.


\section{Linking Requirements
         \label{link-reqs}}

While the \program{configure} script shipped with the Python sources
will correctly build Python to export the symbols needed by
dynamically linked extensions, this is not automatically inherited by
applications which embed the Python library statically, at least on
\UNIX.  This is an issue when the application is linked to the static
runtime library (\file{libpython.a}) and needs to load dynamic
extensions (implemented as \file{.so} files).

The problem is that some entry points are defined by the Python
runtime solely for extension modules to use.  If the embedding
application does not use any of these entry points, some linkers will
not include those entries in the symbol table of the finished
executable.  Some additional options are needed to inform the linker
not to remove these symbols.

Determining the right options to use for any given platform can be
quite difficult, but fortunately the Python configuration already has
those values.  To retrieve them from an installed Python interpreter,
start an interactive interpreter and have a short session like this:

\begin{verbatim}
>>> import distutils.sysconfig
>>> distutils.sysconfig.get_config_var('LINKFORSHARED')
'-Xlinker -export-dynamic'
\end{verbatim}
\refstmodindex{distutils.sysconfig}

The contents of the string presented will be the options that should
be used.  If the string is empty, there's no need to add any
additional options.  The \constant{LINKFORSHARED} definition
corresponds to the variable of the same name in Python's top-level
\file{Makefile}.



\appendix
\chapter{Reporting Bugs}
\label{reporting-bugs}

Python is a mature programming language which has established a
reputation for stability.  In order to maintain this reputation, the
developers would like to know of any deficiencies you find in Python
or its documentation.

All bug reports should be submitted via the Python Bug Tracker on
SourceForge (\url{http://sourceforge.net/bugs/?group_id=5470}).  The
bug tracker offers a Web form which allows pertinent information to be
entered and submitted to the developers.

Before submitting a report, please log into SourceForge if you are a
member; this will make it possible for the developers to contact you
for additional information if needed.  If you are not a SourceForge
member but would not mind the developers contacting you, you may
include your email address in your bug description.  In this case,
please realize that the information is publically available and cannot
be protected.

The first step in filing a report is to determine whether the problem
has already been reported.  The advantage in doing so, aside from
saving the developers time, is that you learn what has been done to
fix it; it may be that the problem has already been fixed for the next
release, or additional information is needed (in which case you are
welcome to provide it if you can!).  To do this, search the bug
database using the search box near the bottom of the page.

If the problem you're reporting is not already in the bug tracker, go
back to the Python Bug Tracker
(\url{http://sourceforge.net/bugs/?group_id=5470}).  Select the
``Submit a Bug'' link at the top of the page to open the bug reporting
form.

The submission form has a number of fields.  The only fields that are
required are the ``Summary'' and ``Details'' fields.  For the summary,
enter a \emph{very} short description of the problem; less than ten
words is good.  In the Details field, describe the problem in detail,
including what you expected to happen and what did happen.  Be sure to
include the version of Python you used, whether any extension modules
were involved, and what hardware and software platform you were using
(including version information as appropriate).

The only other field that you may want to set is the ``Category''
field, which allows you to place the bug report into a broad category
(such as ``Documentation'' or ``Library'').

Each bug report will be assigned to a developer who will determine
what needs to be done to correct the problem.  If you have a
SourceForge account and logged in to report the problem, you will
receive an update each time action is taken on the bug.


\begin{seealso}
  \seetitle[http://www-mice.cs.ucl.ac.uk/multimedia/software/documentation/ReportingBugs.html]{How
        to Report Bugs Effectively}{Article which goes into some
        detail about how to create a useful bug report.  This
        describes what kind of information is useful and why it is
        useful.}

  \seetitle[http://www.mozilla.org/quality/bug-writing-guidelines.html]{Bug
        Writing Guidelines}{Information about writing a good bug
        report.  Some of this is specific to the Mozilla project, but
        describes general good practices.}
\end{seealso}


\chapter{History and License}
\section{History of the software}

Python was created in the early 1990s by Guido van Rossum at Stichting
Mathematisch Centrum (CWI, see \url{http://www.cwi.nl/}) in the Netherlands
as a successor of a language called ABC.  Guido remains Python's
principal author, although it includes many contributions from others.

In 1995, Guido continued his work on Python at the Corporation for
National Research Initiatives (CNRI, see \url{http://www.cnri.reston.va.us/})
in Reston, Virginia where he released several versions of the
software.

In May 2000, Guido and the Python core development team moved to
BeOpen.com to form the BeOpen PythonLabs team.  In October of the same
year, the PythonLabs team moved to Digital Creations (now Zope
Corporation; see \url{http://www.zope.com/}).  In 2001, the Python
Software Foundation (PSF, see \url{http://www.python.org/psf/}) was
formed, a non-profit organization created specifically to own
Python-related Intellectual Property.  Zope Corporation is a
sponsoring member of the PSF.

All Python releases are Open Source (see
\url{http://www.opensource.org/} for the Open Source Definition).
Historically, most, but not all, Python releases have also been
GPL-compatible; the table below summarizes the various releases.

\begin{tablev}{c|c|c|c|c}{textrm}%
  {Release}{Derived from}{Year}{Owner}{GPL compatible?}
  \linev{0.9.0 thru 1.2}{n/a}{1991-1995}{CWI}{yes}
  \linev{1.3 thru 1.5.2}{1.2}{1995-1999}{CNRI}{yes}
  \linev{1.6}{1.5.2}{2000}{CNRI}{no}
  \linev{2.0}{1.6}{2000}{BeOpen.com}{no}
  \linev{1.6.1}{1.6}{2001}{CNRI}{no}
  \linev{2.1}{2.0+1.6.1}{2001}{PSF}{no}
  \linev{2.0.1}{2.0+1.6.1}{2001}{PSF}{yes}
  \linev{2.1.1}{2.1+2.0.1}{2001}{PSF}{yes}
  \linev{2.2}{2.1.1}{2001}{PSF}{yes}
  \linev{2.1.2}{2.1.1}{2002}{PSF}{yes}
  \linev{2.1.3}{2.1.2}{2002}{PSF}{yes}
  \linev{2.2.1}{2.2}{2002}{PSF}{yes}
  \linev{2.2.2}{2.2.1}{2002}{PSF}{yes}
  \linev{2.2.3}{2.2.2}{2002-2003}{PSF}{yes}
  \linev{2.3}{2.2.2}{2002-2003}{PSF}{yes}
  \linev{2.3.1}{2.3}{2002-2003}{PSF}{yes}
  \linev{2.3.2}{2.3.1}{2003}{PSF}{yes}
  \linev{2.3.3}{2.3.2}{2003}{PSF}{yes}
  \linev{2.3.4}{2.3.3}{2004}{PSF}{yes}
  \linev{2.3.5}{2.3.4}{2005}{PSF}{yes}
  \linev{2.4}{2.3}{2004}{PSF}{yes}
\end{tablev}

\note{GPL-compatible doesn't mean that we're distributing
Python under the GPL.  All Python licenses, unlike the GPL, let you
distribute a modified version without making your changes open source.
The GPL-compatible licenses make it possible to combine Python with
other software that is released under the GPL; the others don't.}

Thanks to the many outside volunteers who have worked under Guido's
direction to make these releases possible.


\section{Terms and conditions for accessing or otherwise using Python}

\centerline{\strong{PSF LICENSE AGREEMENT FOR PYTHON \version}}

\begin{enumerate}
\item
This LICENSE AGREEMENT is between the Python Software Foundation
(``PSF''), and the Individual or Organization (``Licensee'') accessing
and otherwise using Python \version{} software in source or binary
form and its associated documentation.

\item
Subject to the terms and conditions of this License Agreement, PSF
hereby grants Licensee a nonexclusive, royalty-free, world-wide
license to reproduce, analyze, test, perform and/or display publicly,
prepare derivative works, distribute, and otherwise use Python
\version{} alone or in any derivative version, provided, however, that
PSF's License Agreement and PSF's notice of copyright, i.e.,
``Copyright \copyright{} 2001-2004 Python Software Foundation; All
Rights Reserved'' are retained in Python \version{} alone or in any
derivative version prepared by Licensee.

\item
In the event Licensee prepares a derivative work that is based on
or incorporates Python \version{} or any part thereof, and wants to
make the derivative work available to others as provided herein, then
Licensee hereby agrees to include in any such work a brief summary of
the changes made to Python \version.

\item
PSF is making Python \version{} available to Licensee on an ``AS IS''
basis.  PSF MAKES NO REPRESENTATIONS OR WARRANTIES, EXPRESS OR
IMPLIED.  BY WAY OF EXAMPLE, BUT NOT LIMITATION, PSF MAKES NO AND
DISCLAIMS ANY REPRESENTATION OR WARRANTY OF MERCHANTABILITY OR FITNESS
FOR ANY PARTICULAR PURPOSE OR THAT THE USE OF PYTHON \version{} WILL
NOT INFRINGE ANY THIRD PARTY RIGHTS.

\item
PSF SHALL NOT BE LIABLE TO LICENSEE OR ANY OTHER USERS OF PYTHON
\version{} FOR ANY INCIDENTAL, SPECIAL, OR CONSEQUENTIAL DAMAGES OR
LOSS AS A RESULT OF MODIFYING, DISTRIBUTING, OR OTHERWISE USING PYTHON
\version, OR ANY DERIVATIVE THEREOF, EVEN IF ADVISED OF THE
POSSIBILITY THEREOF.

\item
This License Agreement will automatically terminate upon a material
breach of its terms and conditions.

\item
Nothing in this License Agreement shall be deemed to create any
relationship of agency, partnership, or joint venture between PSF and
Licensee.  This License Agreement does not grant permission to use PSF
trademarks or trade name in a trademark sense to endorse or promote
products or services of Licensee, or any third party.

\item
By copying, installing or otherwise using Python \version, Licensee
agrees to be bound by the terms and conditions of this License
Agreement.
\end{enumerate}


\centerline{\strong{BEOPEN.COM LICENSE AGREEMENT FOR PYTHON 2.0}}

\centerline{\strong{BEOPEN PYTHON OPEN SOURCE LICENSE AGREEMENT VERSION 1}}

\begin{enumerate}
\item
This LICENSE AGREEMENT is between BeOpen.com (``BeOpen''), having an
office at 160 Saratoga Avenue, Santa Clara, CA 95051, and the
Individual or Organization (``Licensee'') accessing and otherwise
using this software in source or binary form and its associated
documentation (``the Software'').

\item
Subject to the terms and conditions of this BeOpen Python License
Agreement, BeOpen hereby grants Licensee a non-exclusive,
royalty-free, world-wide license to reproduce, analyze, test, perform
and/or display publicly, prepare derivative works, distribute, and
otherwise use the Software alone or in any derivative version,
provided, however, that the BeOpen Python License is retained in the
Software, alone or in any derivative version prepared by Licensee.

\item
BeOpen is making the Software available to Licensee on an ``AS IS''
basis.  BEOPEN MAKES NO REPRESENTATIONS OR WARRANTIES, EXPRESS OR
IMPLIED.  BY WAY OF EXAMPLE, BUT NOT LIMITATION, BEOPEN MAKES NO AND
DISCLAIMS ANY REPRESENTATION OR WARRANTY OF MERCHANTABILITY OR FITNESS
FOR ANY PARTICULAR PURPOSE OR THAT THE USE OF THE SOFTWARE WILL NOT
INFRINGE ANY THIRD PARTY RIGHTS.

\item
BEOPEN SHALL NOT BE LIABLE TO LICENSEE OR ANY OTHER USERS OF THE
SOFTWARE FOR ANY INCIDENTAL, SPECIAL, OR CONSEQUENTIAL DAMAGES OR LOSS
AS A RESULT OF USING, MODIFYING OR DISTRIBUTING THE SOFTWARE, OR ANY
DERIVATIVE THEREOF, EVEN IF ADVISED OF THE POSSIBILITY THEREOF.

\item
This License Agreement will automatically terminate upon a material
breach of its terms and conditions.

\item
This License Agreement shall be governed by and interpreted in all
respects by the law of the State of California, excluding conflict of
law provisions.  Nothing in this License Agreement shall be deemed to
create any relationship of agency, partnership, or joint venture
between BeOpen and Licensee.  This License Agreement does not grant
permission to use BeOpen trademarks or trade names in a trademark
sense to endorse or promote products or services of Licensee, or any
third party.  As an exception, the ``BeOpen Python'' logos available
at http://www.pythonlabs.com/logos.html may be used according to the
permissions granted on that web page.

\item
By copying, installing or otherwise using the software, Licensee
agrees to be bound by the terms and conditions of this License
Agreement.
\end{enumerate}


\centerline{\strong{CNRI LICENSE AGREEMENT FOR PYTHON 1.6.1}}

\begin{enumerate}
\item
This LICENSE AGREEMENT is between the Corporation for National
Research Initiatives, having an office at 1895 Preston White Drive,
Reston, VA 20191 (``CNRI''), and the Individual or Organization
(``Licensee'') accessing and otherwise using Python 1.6.1 software in
source or binary form and its associated documentation.

\item
Subject to the terms and conditions of this License Agreement, CNRI
hereby grants Licensee a nonexclusive, royalty-free, world-wide
license to reproduce, analyze, test, perform and/or display publicly,
prepare derivative works, distribute, and otherwise use Python 1.6.1
alone or in any derivative version, provided, however, that CNRI's
License Agreement and CNRI's notice of copyright, i.e., ``Copyright
\copyright{} 1995-2001 Corporation for National Research Initiatives;
All Rights Reserved'' are retained in Python 1.6.1 alone or in any
derivative version prepared by Licensee.  Alternately, in lieu of
CNRI's License Agreement, Licensee may substitute the following text
(omitting the quotes): ``Python 1.6.1 is made available subject to the
terms and conditions in CNRI's License Agreement.  This Agreement
together with Python 1.6.1 may be located on the Internet using the
following unique, persistent identifier (known as a handle):
1895.22/1013.  This Agreement may also be obtained from a proxy server
on the Internet using the following URL:
\url{http://hdl.handle.net/1895.22/1013}.''

\item
In the event Licensee prepares a derivative work that is based on
or incorporates Python 1.6.1 or any part thereof, and wants to make
the derivative work available to others as provided herein, then
Licensee hereby agrees to include in any such work a brief summary of
the changes made to Python 1.6.1.

\item
CNRI is making Python 1.6.1 available to Licensee on an ``AS IS''
basis.  CNRI MAKES NO REPRESENTATIONS OR WARRANTIES, EXPRESS OR
IMPLIED.  BY WAY OF EXAMPLE, BUT NOT LIMITATION, CNRI MAKES NO AND
DISCLAIMS ANY REPRESENTATION OR WARRANTY OF MERCHANTABILITY OR FITNESS
FOR ANY PARTICULAR PURPOSE OR THAT THE USE OF PYTHON 1.6.1 WILL NOT
INFRINGE ANY THIRD PARTY RIGHTS.

\item
CNRI SHALL NOT BE LIABLE TO LICENSEE OR ANY OTHER USERS OF PYTHON
1.6.1 FOR ANY INCIDENTAL, SPECIAL, OR CONSEQUENTIAL DAMAGES OR LOSS AS
A RESULT OF MODIFYING, DISTRIBUTING, OR OTHERWISE USING PYTHON 1.6.1,
OR ANY DERIVATIVE THEREOF, EVEN IF ADVISED OF THE POSSIBILITY THEREOF.

\item
This License Agreement will automatically terminate upon a material
breach of its terms and conditions.

\item
This License Agreement shall be governed by the federal
intellectual property law of the United States, including without
limitation the federal copyright law, and, to the extent such
U.S. federal law does not apply, by the law of the Commonwealth of
Virginia, excluding Virginia's conflict of law provisions.
Notwithstanding the foregoing, with regard to derivative works based
on Python 1.6.1 that incorporate non-separable material that was
previously distributed under the GNU General Public License (GPL), the
law of the Commonwealth of Virginia shall govern this License
Agreement only as to issues arising under or with respect to
Paragraphs 4, 5, and 7 of this License Agreement.  Nothing in this
License Agreement shall be deemed to create any relationship of
agency, partnership, or joint venture between CNRI and Licensee.  This
License Agreement does not grant permission to use CNRI trademarks or
trade name in a trademark sense to endorse or promote products or
services of Licensee, or any third party.

\item
By clicking on the ``ACCEPT'' button where indicated, or by copying,
installing or otherwise using Python 1.6.1, Licensee agrees to be
bound by the terms and conditions of this License Agreement.
\end{enumerate}

\centerline{ACCEPT}



\centerline{\strong{CWI LICENSE AGREEMENT FOR PYTHON 0.9.0 THROUGH 1.2}}

Copyright \copyright{} 1991 - 1995, Stichting Mathematisch Centrum
Amsterdam, The Netherlands.  All rights reserved.

Permission to use, copy, modify, and distribute this software and its
documentation for any purpose and without fee is hereby granted,
provided that the above copyright notice appear in all copies and that
both that copyright notice and this permission notice appear in
supporting documentation, and that the name of Stichting Mathematisch
Centrum or CWI not be used in advertising or publicity pertaining to
distribution of the software without specific, written prior
permission.

STICHTING MATHEMATISCH CENTRUM DISCLAIMS ALL WARRANTIES WITH REGARD TO
THIS SOFTWARE, INCLUDING ALL IMPLIED WARRANTIES OF MERCHANTABILITY AND
FITNESS, IN NO EVENT SHALL STICHTING MATHEMATISCH CENTRUM BE LIABLE
FOR ANY SPECIAL, INDIRECT OR CONSEQUENTIAL DAMAGES OR ANY DAMAGES
WHATSOEVER RESULTING FROM LOSS OF USE, DATA OR PROFITS, WHETHER IN AN
ACTION OF CONTRACT, NEGLIGENCE OR OTHER TORTIOUS ACTION, ARISING OUT
OF OR IN CONNECTION WITH THE USE OR PERFORMANCE OF THIS SOFTWARE.


\section{Licenses and Acknowledgements for Incorporated Software}

This section is an incomplete, but growing list of licenses and
acknowledgements for third-party software incorporated in the
Python distribution.


\subsection{Mersenne Twister}

The \module{_random} module includes code based on a download from
\url{http://www.math.keio.ac.jp/~matumoto/MT2002/emt19937ar.html}.
The following are the verbatim comments from the original code:

\begin{verbatim}
A C-program for MT19937, with initialization improved 2002/1/26.
Coded by Takuji Nishimura and Makoto Matsumoto.

Before using, initialize the state by using init_genrand(seed)
or init_by_array(init_key, key_length).

Copyright (C) 1997 - 2002, Makoto Matsumoto and Takuji Nishimura,
All rights reserved.

Redistribution and use in source and binary forms, with or without
modification, are permitted provided that the following conditions
are met:

 1. Redistributions of source code must retain the above copyright
    notice, this list of conditions and the following disclaimer.

 2. Redistributions in binary form must reproduce the above copyright
    notice, this list of conditions and the following disclaimer in the
    documentation and/or other materials provided with the distribution.

 3. The names of its contributors may not be used to endorse or promote
    products derived from this software without specific prior written
    permission.

THIS SOFTWARE IS PROVIDED BY THE COPYRIGHT HOLDERS AND CONTRIBUTORS
"AS IS" AND ANY EXPRESS OR IMPLIED WARRANTIES, INCLUDING, BUT NOT
LIMITED TO, THE IMPLIED WARRANTIES OF MERCHANTABILITY AND FITNESS FOR
A PARTICULAR PURPOSE ARE DISCLAIMED.  IN NO EVENT SHALL THE COPYRIGHT OWNER OR
CONTRIBUTORS BE LIABLE FOR ANY DIRECT, INDIRECT, INCIDENTAL, SPECIAL,
EXEMPLARY, OR CONSEQUENTIAL DAMAGES (INCLUDING, BUT NOT LIMITED TO,
PROCUREMENT OF SUBSTITUTE GOODS OR SERVICES; LOSS OF USE, DATA, OR
PROFITS; OR BUSINESS INTERRUPTION) HOWEVER CAUSED AND ON ANY THEORY OF
LIABILITY, WHETHER IN CONTRACT, STRICT LIABILITY, OR TORT (INCLUDING
NEGLIGENCE OR OTHERWISE) ARISING IN ANY WAY OUT OF THE USE OF THIS
SOFTWARE, EVEN IF ADVISED OF THE POSSIBILITY OF SUCH DAMAGE.


Any feedback is very welcome.
http://www.math.keio.ac.jp/matumoto/emt.html
email: matumoto@math.keio.ac.jp
\end{verbatim}



\subsection{Sockets}

The \module{socket} module uses the functions, \function{getaddrinfo},
and \function{getnameinfo}, which are coded in separate source files
from the WIDE Project, \url{http://www.wide.ad.jp/about/index.html}.

\begin{verbatim}      
Copyright (C) 1995, 1996, 1997, and 1998 WIDE Project.
All rights reserved.
 
Redistribution and use in source and binary forms, with or without
modification, are permitted provided that the following conditions
are met:
1. Redistributions of source code must retain the above copyright
   notice, this list of conditions and the following disclaimer.
2. Redistributions in binary form must reproduce the above copyright
   notice, this list of conditions and the following disclaimer in the
   documentation and/or other materials provided with the distribution.
3. Neither the name of the project nor the names of its contributors
   may be used to endorse or promote products derived from this software
   without specific prior written permission.

THIS SOFTWARE IS PROVIDED BY THE PROJECT AND CONTRIBUTORS ``AS IS'' AND
GAI_ANY EXPRESS OR IMPLIED WARRANTIES, INCLUDING, BUT NOT LIMITED TO, THE
IMPLIED WARRANTIES OF MERCHANTABILITY AND FITNESS FOR A PARTICULAR PURPOSE
ARE DISCLAIMED.  IN NO EVENT SHALL THE PROJECT OR CONTRIBUTORS BE LIABLE
FOR GAI_ANY DIRECT, INDIRECT, INCIDENTAL, SPECIAL, EXEMPLARY, OR CONSEQUENTIAL
DAMAGES (INCLUDING, BUT NOT LIMITED TO, PROCUREMENT OF SUBSTITUTE GOODS
OR SERVICES; LOSS OF USE, DATA, OR PROFITS; OR BUSINESS INTERRUPTION)
HOWEVER CAUSED AND ON GAI_ANY THEORY OF LIABILITY, WHETHER IN CONTRACT, STRICT
LIABILITY, OR TORT (INCLUDING NEGLIGENCE OR OTHERWISE) ARISING IN GAI_ANY WAY
OUT OF THE USE OF THIS SOFTWARE, EVEN IF ADVISED OF THE POSSIBILITY OF
SUCH DAMAGE.
\end{verbatim}



\subsection{Floating point exception control}

The source for the \module{fpectl} module includes the following notice:

\begin{verbatim}
     ---------------------------------------------------------------------  
    /                       Copyright (c) 1996.                           \ 
   |          The Regents of the University of California.                 |
   |                        All rights reserved.                           |
   |                                                                       |
   |   Permission to use, copy, modify, and distribute this software for   |
   |   any purpose without fee is hereby granted, provided that this en-   |
   |   tire notice is included in all copies of any software which is or   |
   |   includes  a  copy  or  modification  of  this software and in all   |
   |   copies of the supporting documentation for such software.           |
   |                                                                       |
   |   This  work was produced at the University of California, Lawrence   |
   |   Livermore National Laboratory under  contract  no.  W-7405-ENG-48   |
   |   between  the  U.S.  Department  of  Energy and The Regents of the   |
   |   University of California for the operation of UC LLNL.              |
   |                                                                       |
   |                              DISCLAIMER                               |
   |                                                                       |
   |   This  software was prepared as an account of work sponsored by an   |
   |   agency of the United States Government. Neither the United States   |
   |   Government  nor the University of California nor any of their em-   |
   |   ployees, makes any warranty, express or implied, or  assumes  any   |
   |   liability  or  responsibility  for the accuracy, completeness, or   |
   |   usefulness of any information,  apparatus,  product,  or  process   |
   |   disclosed,   or  represents  that  its  use  would  not  infringe   |
   |   privately-owned rights. Reference herein to any specific  commer-   |
   |   cial  products,  process,  or  service  by trade name, trademark,   |
   |   manufacturer, or otherwise, does not  necessarily  constitute  or   |
   |   imply  its endorsement, recommendation, or favoring by the United   |
   |   States Government or the University of California. The views  and   |
   |   opinions  of authors expressed herein do not necessarily state or   |
   |   reflect those of the United States Government or  the  University   |
   |   of  California,  and shall not be used for advertising or product   |
    \  endorsement purposes.                                              / 
     ---------------------------------------------------------------------
\end{verbatim}



\subsection{MD5 message digest algorithm}

The source code for the \module{md5} module contains the following notice:

\begin{verbatim}
Copyright (C) 1991-2, RSA Data Security, Inc. Created 1991. All
rights reserved.

License to copy and use this software is granted provided that it
is identified as the "RSA Data Security, Inc. MD5 Message-Digest
Algorithm" in all material mentioning or referencing this software
or this function.

License is also granted to make and use derivative works provided
that such works are identified as "derived from the RSA Data
Security, Inc. MD5 Message-Digest Algorithm" in all material
mentioning or referencing the derived work.

RSA Data Security, Inc. makes no representations concerning either
the merchantability of this software or the suitability of this
software for any particular purpose. It is provided "as is"
without express or implied warranty of any kind.

These notices must be retained in any copies of any part of this
documentation and/or software.
\end{verbatim}



\subsection{Asynchronous socket services}

The \module{asynchat} and \module{asyncore} modules contain the
following notice:

\begin{verbatim}      
 Copyright 1996 by Sam Rushing

                         All Rights Reserved

 Permission to use, copy, modify, and distribute this software and
 its documentation for any purpose and without fee is hereby
 granted, provided that the above copyright notice appear in all
 copies and that both that copyright notice and this permission
 notice appear in supporting documentation, and that the name of Sam
 Rushing not be used in advertising or publicity pertaining to
 distribution of the software without specific, written prior
 permission.

 SAM RUSHING DISCLAIMS ALL WARRANTIES WITH REGARD TO THIS SOFTWARE,
 INCLUDING ALL IMPLIED WARRANTIES OF MERCHANTABILITY AND FITNESS, IN
 NO EVENT SHALL SAM RUSHING BE LIABLE FOR ANY SPECIAL, INDIRECT OR
 CONSEQUENTIAL DAMAGES OR ANY DAMAGES WHATSOEVER RESULTING FROM LOSS
 OF USE, DATA OR PROFITS, WHETHER IN AN ACTION OF CONTRACT,
 NEGLIGENCE OR OTHER TORTIOUS ACTION, ARISING OUT OF OR IN
 CONNECTION WITH THE USE OR PERFORMANCE OF THIS SOFTWARE.
\end{verbatim}


\subsection{Cookie management}

The \module{Cookie} module contains the following notice:

\begin{verbatim}
 Copyright 2000 by Timothy O'Malley <timo@alum.mit.edu>

                All Rights Reserved

 Permission to use, copy, modify, and distribute this software
 and its documentation for any purpose and without fee is hereby
 granted, provided that the above copyright notice appear in all
 copies and that both that copyright notice and this permission
 notice appear in supporting documentation, and that the name of
 Timothy O'Malley  not be used in advertising or publicity
 pertaining to distribution of the software without specific, written
 prior permission.

 Timothy O'Malley DISCLAIMS ALL WARRANTIES WITH REGARD TO THIS
 SOFTWARE, INCLUDING ALL IMPLIED WARRANTIES OF MERCHANTABILITY
 AND FITNESS, IN NO EVENT SHALL Timothy O'Malley BE LIABLE FOR
 ANY SPECIAL, INDIRECT OR CONSEQUENTIAL DAMAGES OR ANY DAMAGES
 WHATSOEVER RESULTING FROM LOSS OF USE, DATA OR PROFITS,
 WHETHER IN AN ACTION OF CONTRACT, NEGLIGENCE OR OTHER TORTIOUS
 ACTION, ARISING OUT OF OR IN CONNECTION WITH THE USE OR
 PERFORMANCE OF THIS SOFTWARE.
\end{verbatim}      



\subsection{Profiling}

The \module{profile} and \module{pstats} modules contain
the following notice:

\begin{verbatim}
 Copyright 1994, by InfoSeek Corporation, all rights reserved.
 Written by James Roskind

 Permission to use, copy, modify, and distribute this Python software
 and its associated documentation for any purpose (subject to the
 restriction in the following sentence) without fee is hereby granted,
 provided that the above copyright notice appears in all copies, and
 that both that copyright notice and this permission notice appear in
 supporting documentation, and that the name of InfoSeek not be used in
 advertising or publicity pertaining to distribution of the software
 without specific, written prior permission.  This permission is
 explicitly restricted to the copying and modification of the software
 to remain in Python, compiled Python, or other languages (such as C)
 wherein the modified or derived code is exclusively imported into a
 Python module.

 INFOSEEK CORPORATION DISCLAIMS ALL WARRANTIES WITH REGARD TO THIS
 SOFTWARE, INCLUDING ALL IMPLIED WARRANTIES OF MERCHANTABILITY AND
 FITNESS. IN NO EVENT SHALL INFOSEEK CORPORATION BE LIABLE FOR ANY
 SPECIAL, INDIRECT OR CONSEQUENTIAL DAMAGES OR ANY DAMAGES WHATSOEVER
 RESULTING FROM LOSS OF USE, DATA OR PROFITS, WHETHER IN AN ACTION OF
 CONTRACT, NEGLIGENCE OR OTHER TORTIOUS ACTION, ARISING OUT OF OR IN
 CONNECTION WITH THE USE OR PERFORMANCE OF THIS SOFTWARE.
\end{verbatim}



\subsection{Execution tracing}

The \module{trace} module contains the following notice:

\begin{verbatim}
 portions copyright 2001, Autonomous Zones Industries, Inc., all rights...
 err...  reserved and offered to the public under the terms of the
 Python 2.2 license.
 Author: Zooko O'Whielacronx
 http://zooko.com/
 mailto:zooko@zooko.com

 Copyright 2000, Mojam Media, Inc., all rights reserved.
 Author: Skip Montanaro

 Copyright 1999, Bioreason, Inc., all rights reserved.
 Author: Andrew Dalke

 Copyright 1995-1997, Automatrix, Inc., all rights reserved.
 Author: Skip Montanaro

 Copyright 1991-1995, Stichting Mathematisch Centrum, all rights reserved.


 Permission to use, copy, modify, and distribute this Python software and
 its associated documentation for any purpose without fee is hereby
 granted, provided that the above copyright notice appears in all copies,
 and that both that copyright notice and this permission notice appear in
 supporting documentation, and that the name of neither Automatrix,
 Bioreason or Mojam Media be used in advertising or publicity pertaining to
 distribution of the software without specific, written prior permission.
\end{verbatim} 



\subsection{UUencode and UUdecode functions}

The \module{uu} module contains the following notice:

\begin{verbatim}
 Copyright 1994 by Lance Ellinghouse
 Cathedral City, California Republic, United States of America.
                        All Rights Reserved
 Permission to use, copy, modify, and distribute this software and its
 documentation for any purpose and without fee is hereby granted,
 provided that the above copyright notice appear in all copies and that
 both that copyright notice and this permission notice appear in
 supporting documentation, and that the name of Lance Ellinghouse
 not be used in advertising or publicity pertaining to distribution
 of the software without specific, written prior permission.
 LANCE ELLINGHOUSE DISCLAIMS ALL WARRANTIES WITH REGARD TO
 THIS SOFTWARE, INCLUDING ALL IMPLIED WARRANTIES OF MERCHANTABILITY AND
 FITNESS, IN NO EVENT SHALL LANCE ELLINGHOUSE CENTRUM BE LIABLE
 FOR ANY SPECIAL, INDIRECT OR CONSEQUENTIAL DAMAGES OR ANY DAMAGES
 WHATSOEVER RESULTING FROM LOSS OF USE, DATA OR PROFITS, WHETHER IN AN
 ACTION OF CONTRACT, NEGLIGENCE OR OTHER TORTIOUS ACTION, ARISING OUT
 OF OR IN CONNECTION WITH THE USE OR PERFORMANCE OF THIS SOFTWARE.

 Modified by Jack Jansen, CWI, July 1995:
 - Use binascii module to do the actual line-by-line conversion
   between ascii and binary. This results in a 1000-fold speedup. The C
   version is still 5 times faster, though.
 - Arguments more compliant with python standard
\end{verbatim}



\subsection{XML Remote Procedure Calls}

The \module{xmlrpclib} module contains the following notice:

\begin{verbatim}
     The XML-RPC client interface is

 Copyright (c) 1999-2002 by Secret Labs AB
 Copyright (c) 1999-2002 by Fredrik Lundh

 By obtaining, using, and/or copying this software and/or its
 associated documentation, you agree that you have read, understood,
 and will comply with the following terms and conditions:

 Permission to use, copy, modify, and distribute this software and
 its associated documentation for any purpose and without fee is
 hereby granted, provided that the above copyright notice appears in
 all copies, and that both that copyright notice and this permission
 notice appear in supporting documentation, and that the name of
 Secret Labs AB or the author not be used in advertising or publicity
 pertaining to distribution of the software without specific, written
 prior permission.

 SECRET LABS AB AND THE AUTHOR DISCLAIMS ALL WARRANTIES WITH REGARD
 TO THIS SOFTWARE, INCLUDING ALL IMPLIED WARRANTIES OF MERCHANT-
 ABILITY AND FITNESS.  IN NO EVENT SHALL SECRET LABS AB OR THE AUTHOR
 BE LIABLE FOR ANY SPECIAL, INDIRECT OR CONSEQUENTIAL DAMAGES OR ANY
 DAMAGES WHATSOEVER RESULTING FROM LOSS OF USE, DATA OR PROFITS,
 WHETHER IN AN ACTION OF CONTRACT, NEGLIGENCE OR OTHER TORTIOUS
 ACTION, ARISING OUT OF OR IN CONNECTION WITH THE USE OR PERFORMANCE
 OF THIS SOFTWARE.
\end{verbatim}


\end{document}

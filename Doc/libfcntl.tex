% Manual text by Jaap Vermeulen
\section{Built-in Module \sectcode{fcntl}}
\label{module-fcntl}
\bimodindex{fcntl}
\indexii{UNIX@\UNIX{}}{file control}
\indexii{UNIX@\UNIX{}}{I/O control}

This module performs file control and I/O control on file descriptors.
It is an interface to the \cfunction{fcntl()} and \cfunction{ioctl()}
\UNIX{} routines.  File descriptors can be obtained with the
\method{fileno()} method of a file or socket object.

The module defines the following functions:


\begin{funcdesc}{fcntl}{fd, op\optional{, arg}}
  Perform the requested operation on file descriptor \var{fd}.
  The operation is defined by \var{op} and is operating system
  dependent.  Typically these codes can be retrieved from the library
  module \module{FCNTL}\refstmodindex{FCNTL}. The argument \var{arg}
  is optional, and defaults to the integer value \code{0}.  When
  present, it can either be an integer value, or a string.  With
  the argument missing or an integer value, the return value of this
  function is the integer return value of the \C{} \cfunction{fcntl()}
  call.  When the argument is a string it represents a binary
  structure, e.g.\ created by \function{struct.pack()}. The binary
  data is copied to a buffer whose address is passed to the \C{}
  \cfunction{fcntl()} call.  The return value after a successful call
  is the contents of the buffer, converted to a string object.  In
  case the \cfunction{fcntl()} fails, an \exception{IOError} is
  raised.
\end{funcdesc}

\begin{funcdesc}{ioctl}{fd, op, arg}
  This function is identical to the \function{fcntl()} function, except
  that the operations are typically defined in the library module
  \module{IOCTL}.
\end{funcdesc}

\begin{funcdesc}{flock}{fd, op}
Perform the lock operation \var{op} on file descriptor \var{fd}.
See the \UNIX{} manual \manpage{flock}{3} for details.  (On some
systems, this function is emulated using \cfunction{fcntl()}.)
\end{funcdesc}

\begin{funcdesc}{lockf}{fd, code, \optional{len, \optional{start, \optional{whence}}}}
This is a wrapper around the \constant{FCNTL.F_SETLK} and
\constant{FCNTL.F_SETLKW} \function{fcntl()} calls.  See the \UNIX{}
manual for details.
\end{funcdesc}

If the library modules \module{FCNTL}\refstmodindex{FCNTL} or
\module{IOCTL}\refstmodindex{IOCTL} are missing, you can find the
opcodes in the \C{} include files \code{<sys/fcntl.h>} and
\code{<sys/ioctl.h>}.  You can create the modules yourself with the
\program{h2py} script, found in the \file{Tools/scripts/} directory.


Examples (all on a SVR4 compliant system):

\begin{verbatim}
import struct, FCNTL

file = open(...)
rv = fcntl(file.fileno(), FCNTL.O_NDELAY, 1)

lockdata = struct.pack('hhllhh', FCNTL.F_WRLCK, 0, 0, 0, 0, 0)
rv = fcntl(file.fileno(), FCNTL.F_SETLKW, lockdata)
\end{verbatim}

Note that in the first example the return value variable \code{rv} will
hold an integer value; in the second example it will hold a string
value.  The structure lay-out for the \var{lockdata} variable is
system dependent --- therefore using the \function{flock()} call may be
better.

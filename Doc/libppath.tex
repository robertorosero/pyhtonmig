\section{Standard Module \sectcode{posixpath}}
\stmodindex{posixpath}

This module implements some useful functions on POSIX pathnames.

\strong{Do not import this module directly.}  Instead, import the
module \code{os} and use \code{os.path}.
\stmodindex{os}

\renewcommand{\indexsubitem}{(in module posixpath)}

\begin{funcdesc}{basename}{p}
Return the base name of pathname
\var{p}.
This is the second half of the pair returned by
\code{posixpath.split(\var{p})}.
\end{funcdesc}

\begin{funcdesc}{commonprefix}{list}
Return the longest string that is a prefix of all strings in
\var{list}.
If
\var{list}
is empty, return the empty string (\code{''}).
\end{funcdesc}

\begin{funcdesc}{exists}{p}
Return true if
\var{p}
refers to an existing path.
\end{funcdesc}

\begin{funcdesc}{expanduser}{p}
Return the argument with an initial component of \samp{\~} or
\samp{\~\var{user}} replaced by that \var{user}'s home directory.  An
initial \samp{\~{}} is replaced by the environment variable \code{\${}HOME};
an initial \samp{\~\var{user}} is looked up in the password directory through
the built-in module \code{pwd}.  If the expansion fails, or if the
path does not begin with a tilde, the path is returned unchanged.
\end{funcdesc}

\begin{funcdesc}{expandvars}{p}
Return the argument with environment variables expanded.  Substrings
of the form \samp{\$\var{name}} or \samp{\$\{\var{name}\}} are
replaced by the value of environment variable \var{name}.  Malformed
variable names and references to non-existing variables are left
unchanged.
\end{funcdesc}

\begin{funcdesc}{isabs}{p}
Return true if \var{p} is an absolute pathname (begins with a slash).
\end{funcdesc}

\begin{funcdesc}{isfile}{p}
Return true if \var{p} is an existing regular file.  This follows
symbolic links, so both \code{islink()} and \code{isfile()} can be true for the same
path.
\end{funcdesc}

\begin{funcdesc}{isdir}{p}
Return true if \var{p} is an existing directory.  This follows
symbolic links, so both \code{islink()} and \code{isdir()} can be true for the same
path.
\end{funcdesc}

\begin{funcdesc}{islink}{p}
Return true if
\var{p}
refers to a directory entry that is a symbolic link.
Always false if symbolic links are not supported.
\end{funcdesc}

\begin{funcdesc}{ismount}{p}
Return true if pathname \var{p} is a \dfn{mount point}: a point in a
file system where a different file system has been mounted.  The
function checks whether \var{p}'s parent, \file{\var{p}/..}, is on a
different device than \var{p}, or whether \file{\var{p}/..} and
\var{p} point to the same i-node on the same device --- this should
detect mount points for all \UNIX{} and POSIX variants.
\end{funcdesc}

\begin{funcdesc}{join}{p\, q}
Join the paths
\var{p}
and
\var{q} intelligently:
If
\var{q}
is an absolute path, the return value is
\var{q}.
Otherwise, the concatenation of
\var{p}
and
\var{q}
is returned, with a slash (\code{'/'}) inserted unless
\var{p}
is empty or ends in a slash.
\end{funcdesc}

\begin{funcdesc}{normcase}{p}
Normalize the case of a pathname.  This returns the path unchanged;
however, a similar function in \code{macpath} converts upper case to
lower case.
\end{funcdesc}

\begin{funcdesc}{samefile}{p\, q}
Return true if both pathname arguments refer to the same file or directory
(as indicated by device number and i-node number).
Raise an exception if a stat call on either pathname fails.
\end{funcdesc}

\begin{funcdesc}{split}{p}
Split the pathname \var{p} in a pair \code{(\var{head}, \var{tail})},
where \var{tail} is the last pathname component and \var{head} is
everything leading up to that.  The \var{tail} part will never contain
a slash; if \var{p} ends in a slash, \var{tail} will be empty.  If
there is no slash in \var{p}, \var{head} will be empty.  If \var{p} is
empty, both \var{head} and \var{tail} are empty.  Trailing slashes are
stripped from \var{head} unless it is the root (one or more slashes
only).  In nearly all cases, \code{join(\var{head}, \var{tail})}
equals \var{p} (the only exception being when there were multiple
slashes separating \var{head} from \var{tail}).
\end{funcdesc}

\begin{funcdesc}{splitext}{p}
Split the pathname \var{p} in a pair \code{(\var{root}, \var{ext})}
such that \code{\var{root} + \var{ext} == \var{p}},
the last component of \var{root} contains no periods,
and \var{ext} is empty or begins with a period.
\end{funcdesc}

\begin{funcdesc}{walk}{p\, visit\, arg}
Calls the function \var{visit} with arguments
\code{(\var{arg}, \var{dirname}, \var{names})} for each directory in the
directory tree rooted at \var{p} (including \var{p} itself, if it is a
directory).  The argument \var{dirname} specifies the visited directory,
the argument \var{names} lists the files in the directory (gotten from
\code{posix.listdir(\var{dirname})}, so including \samp{.} and
\samp{..}).  The \var{visit} function may modify \var{names} to
influence the set of directories visited below \var{dirname}, e.g., to
avoid visiting certain parts of the tree.  (The object referred to by
\var{names} must be modified in place, using \code{del} or slice
assignment.)
\end{funcdesc}

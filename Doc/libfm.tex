\section{Built-in Module \sectcode{fm}}
\bimodindex{fm}

This module provides access to the IRIS {\em Font Manager} library.
It is available only on Silicon Graphics machines.
See also: 4Sight User's Guide, Section 1, Chapter 5: Using the IRIS
Font Manager.

This is not yet a full interface to the IRIS Font Manager.
Among the unsupported features are: matrix operations; cache
operations; character operations (use string operations instead); some
details of font info; individual glyph metrics; and printer matching.

It supports the following operations:

\renewcommand{\indexsubitem}{(in module fm)}
\begin{funcdesc}{init}{}
Initialization function.
Calls \code{fminit()}.
It is normally not necessary to call this function, since it is called
automatically the first time the \code{fm} module is imported.
\end{funcdesc}

\begin{funcdesc}{findfont}{fontname}
Return a font handle object.
Calls \code{fmfindfont(\var{fontname})}.
\end{funcdesc}

\begin{funcdesc}{enumerate}{}
Returns a list of available font names.
This is an interface to \code{fmenumerate()}.
\end{funcdesc}

\begin{funcdesc}{prstr}{string}
Render a string using the current font (see the \code{setfont()} font
handle method below).
Calls \code{fmprstr(\var{string})}.
\end{funcdesc}

\begin{funcdesc}{setpath}{string}
Sets the font search path.
Calls \code{fmsetpath(string)}.
(XXX Does not work!?!)
\end{funcdesc}

\begin{funcdesc}{fontpath}{}
Returns the current font search path.
\end{funcdesc}

Font handle objects support the following operations:

\renewcommand{\indexsubitem}{(font handle method)}
\begin{funcdesc}{scalefont}{factor}
Returns a handle for a scaled version of this font.
Calls \code{fmscalefont(\var{fh}, \var{factor})}.
\end{funcdesc}

\begin{funcdesc}{setfont}{}
Makes this font the current font.
Note: the effect is undone silently when the font handle object is
deleted.
Calls \code{fmsetfont(\var{fh})}.
\end{funcdesc}

\begin{funcdesc}{getfontname}{}
Returns this font's name.
Calls \code{fmgetfontname(\var{fh})}.
\end{funcdesc}

\begin{funcdesc}{getcomment}{}
Returns the comment string associated with this font.
Raises an exception if there is none.
Calls \code{fmgetcomment(\var{fh})}.
\end{funcdesc}

\begin{funcdesc}{getfontinfo}{}
Returns a tuple giving some pertinent data about this font.
This is an interface to \code{fmgetfontinfo()}.
The returned tuple contains the following numbers:
\code{(\var{printermatched}, \var{fixed_width}, \var{xorig}, \var{yorig}, \var{xsize}, \var{ysize}, \var{height}, \var{nglyphs})}.
\end{funcdesc}

\begin{funcdesc}{getstrwidth}{string}
Returns the width, in pixels, of the string when drawn in this font.
Calls \code{fmgetstrwidth(\var{fh}, \var{string})}.
\end{funcdesc}

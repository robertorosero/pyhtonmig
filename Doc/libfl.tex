\section{Built-in Module \sectcode{fl}}
\bimodindex{fl}

This module provides an interface to the FORMS Library by Mark
Overmars.  The source for the library can be retrieved by anonymous
ftp from host \samp{ftp.cs.ruu.nl}, directory \file{SGI/FORMS}.  It
was last tested with version 2.0b.

Most functions are literal translations of their C equivalents,
dropping the initial \samp{fl_} from their name.  Constants used by
the library are defined in module \code{FL} described below.

The creation of objects is a little different in Python than in C:
instead of the `current form' maintained by the library to which new
FORMS objects are added, all functions that add a FORMS object to a
form are methods of the Python object representing the form.
Consequently, there are no Python equivalents for the C functions
\code{fl_addto_form} and \code{fl_end_form}, and the equivalent of
\code{fl_bgn_form} is called \code{fl.make_form}.

Watch out for the somewhat confusing terminology: FORMS uses the word
\dfn{object} for the buttons, sliders etc. that you can place in a form.
In Python, `object' means any value.  The Python interface to FORMS
introduces two new Python object types: form objects (representing an
entire form) and FORMS objects (representing one button, slider etc.).
Hopefully this isn't too confusing...

There are no `free objects' in the Python interface to FORMS, nor is
there an easy way to add object classes written in Python.  The FORMS
interface to GL event handling is available, though, so you can mix
FORMS with pure GL windows.

\strong{Please note:} importing \code{fl} implies a call to the GL function
\code{foreground()} and to the FORMS routine \code{fl_init()}.

\subsection{Functions Defined in Module \sectcode{fl}}

Module \code{fl} defines the following functions.  For more information
about what they do, see the description of the equivalent C function
in the FORMS documentation:

\renewcommand{\indexsubitem}{(in module fl)}
\begin{funcdesc}{make_form}{type\, width\, height}
Create a form with given type, width and height.  This returns a
\dfn{form} object, whose methods are described below.
\end{funcdesc}

\begin{funcdesc}{do_forms}{}
The standard FORMS main loop.  Returns a Python object representing
the FORMS object needing interaction, or the special value
\code{FL.EVENT}.
\end{funcdesc}

\begin{funcdesc}{check_forms}{}
Check for FORMS events.  Returns what \code{do_forms} above returns,
or \code{None} if there is no event that immediately needs
interaction.
\end{funcdesc}

\begin{funcdesc}{set_event_call_back}{function}
Set the event callback function.
\end{funcdesc}

\begin{funcdesc}{set_graphics_mode}{rgbmode\, doublebuffering}
Set the graphics modes.
\end{funcdesc}

\begin{funcdesc}{get_rgbmode}{}
Return the current rgb mode.  This is the value of the C global
variable \code{fl_rgbmode}.
\end{funcdesc}

\begin{funcdesc}{show_message}{str1\, str2\, str3}
Show a dialog box with a three-line message and an OK button.
\end{funcdesc}

\begin{funcdesc}{show_question}{str1\, str2\, str3}
Show a dialog box with a three-line message and YES and NO buttons.
It returns \code{1} if the user pressed YES, \code{0} if NO.
\end{funcdesc}

\begin{funcdesc}{show_choice}{str1\, str2\, str3\, but1\optional{\, but2\,
but3}}
Show a dialog box with a three-line message and up to three buttons.
It returns the number of the button clicked by the user
(\code{1}, \code{2} or \code{3}).
\end{funcdesc}

\begin{funcdesc}{show_input}{prompt\, default}
Show a dialog box with a one-line prompt message and text field in
which the user can enter a string.  The second argument is the default
input string.  It returns the string value as edited by the user.
\end{funcdesc}

\begin{funcdesc}{show_file_selector}{message\, directory\, pattern\, default}
Show a dialog box in which the user can select a file.  It returns
the absolute filename selected by the user, or \code{None} if the user
presses Cancel.
\end{funcdesc}

\begin{funcdesc}{get_directory}{}
\funcline{get_pattern}{}
\funcline{get_filename}{}
These functions return the directory, pattern and filename (the tail
part only) selected by the user in the last \code{show_file_selector}
call.
\end{funcdesc}

\begin{funcdesc}{qdevice}{dev}
\funcline{unqdevice}{dev}
\funcline{isqueued}{dev}
\funcline{qtest}{}
\funcline{qread}{}
%\funcline{blkqread}{?}
\funcline{qreset}{}
\funcline{qenter}{dev\, val}
\funcline{get_mouse}{}
\funcline{tie}{button\, valuator1\, valuator2}
These functions are the FORMS interfaces to the corresponding GL
functions.  Use these if you want to handle some GL events yourself
when using \code{fl.do_events}.  When a GL event is detected that
FORMS cannot handle, \code{fl.do_forms()} returns the special value
\code{FL.EVENT} and you should call \code{fl.qread()} to read the
event from the queue.  Don't use the equivalent GL functions!
\end{funcdesc}

\begin{funcdesc}{color}{}
\funcline{mapcolor}{}
\funcline{getmcolor}{}
See the description in the FORMS documentation of \code{fl_color},
\code{fl_mapcolor} and \code{fl_getmcolor}.
\end{funcdesc}

\subsection{Form Objects}

Form objects (returned by \code{fl.make_form()} above) have the
following methods.  Each method corresponds to a C function whose name
is prefixed with \samp{fl_}; and whose first argument is a form
pointer; please refer to the official FORMS documentation for
descriptions.

All the \samp{add_{\rm \ldots}} functions return a Python object representing
the FORMS object.  Methods of FORMS objects are described below.  Most
kinds of FORMS object also have some methods specific to that kind;
these methods are listed here.

\begin{flushleft}
\renewcommand{\indexsubitem}{(form object method)}
\begin{funcdesc}{show_form}{placement\, bordertype\, name}
  Show the form.
\end{funcdesc}

\begin{funcdesc}{hide_form}{}
  Hide the form.
\end{funcdesc}

\begin{funcdesc}{redraw_form}{}
  Redraw the form.
\end{funcdesc}

\begin{funcdesc}{set_form_position}{x\, y}
Set the form's position.
\end{funcdesc}

\begin{funcdesc}{freeze_form}{}
Freeze the form.
\end{funcdesc}

\begin{funcdesc}{unfreeze_form}{}
  Unfreeze the form.
\end{funcdesc}

\begin{funcdesc}{activate_form}{}
  Activate the form.
\end{funcdesc}

\begin{funcdesc}{deactivate_form}{}
  Deactivate the form.
\end{funcdesc}

\begin{funcdesc}{bgn_group}{}
  Begin a new group of objects; return a group object.
\end{funcdesc}

\begin{funcdesc}{end_group}{}
  End the current group of objects.
\end{funcdesc}

\begin{funcdesc}{find_first}{}
  Find the first object in the form.
\end{funcdesc}

\begin{funcdesc}{find_last}{}
  Find the last object in the form.
\end{funcdesc}

%---

\begin{funcdesc}{add_box}{type\, x\, y\, w\, h\, name}
Add a box object to the form.
No extra methods.
\end{funcdesc}

\begin{funcdesc}{add_text}{type\, x\, y\, w\, h\, name}
Add a text object to the form.
No extra methods.
\end{funcdesc}

%\begin{funcdesc}{add_bitmap}{type\, x\, y\, w\, h\, name}
%Add a bitmap object to the form.
%\end{funcdesc}

\begin{funcdesc}{add_clock}{type\, x\, y\, w\, h\, name}
Add a clock object to the form. \\
Method:
\code{get_clock}.
\end{funcdesc}

%---

\begin{funcdesc}{add_button}{type\, x\, y\, w\, h\,  name}
Add a button object to the form. \\
Methods:
\code{get_button},
\code{set_button}.
\end{funcdesc}

\begin{funcdesc}{add_lightbutton}{type\, x\, y\, w\, h\, name}
Add a lightbutton object to the form. \\
Methods:
\code{get_button},
\code{set_button}.
\end{funcdesc}

\begin{funcdesc}{add_roundbutton}{type\, x\, y\, w\, h\, name}
Add a roundbutton object to the form. \\
Methods:
\code{get_button},
\code{set_button}.
\end{funcdesc}

%---

\begin{funcdesc}{add_slider}{type\, x\, y\, w\, h\, name}
Add a slider object to the form. \\
Methods:
\code{set_slider_value},
\code{get_slider_value},
\code{set_slider_bounds},
\code{get_slider_bounds},
\code{set_slider_return},
\code{set_slider_size},
\code{set_slider_precision},
\code{set_slider_step}.
\end{funcdesc}

\begin{funcdesc}{add_valslider}{type\, x\, y\, w\, h\, name}
Add a valslider object to the form. \\
Methods:
\code{set_slider_value},
\code{get_slider_value},
\code{set_slider_bounds},
\code{get_slider_bounds},
\code{set_slider_return},
\code{set_slider_size},
\code{set_slider_precision},
\code{set_slider_step}.
\end{funcdesc}

\begin{funcdesc}{add_dial}{type\, x\, y\, w\, h\, name}
Add a dial object to the form. \\
Methods:
\code{set_dial_value},
\code{get_dial_value},
\code{set_dial_bounds},
\code{get_dial_bounds}.
\end{funcdesc}

\begin{funcdesc}{add_positioner}{type\, x\, y\, w\, h\, name}
Add a positioner object to the form. \\
Methods:
\code{set_positioner_xvalue},
\code{set_positioner_yvalue},
\code{set_positioner_xbounds},
\code{set_positioner_ybounds},
\code{get_positioner_xvalue},
\code{get_positioner_yvalue},
\code{get_positioner_xbounds},
\code{get_positioner_ybounds}.
\end{funcdesc}

\begin{funcdesc}{add_counter}{type\, x\, y\, w\, h\, name}
Add a counter object to the form. \\
Methods:
\code{set_counter_value},
\code{get_counter_value},
\code{set_counter_bounds},
\code{set_counter_step},
\code{set_counter_precision},
\code{set_counter_return}.
\end{funcdesc}

%---

\begin{funcdesc}{add_input}{type\, x\, y\, w\, h\, name}
Add a input object to the form. \\
Methods:
\code{set_input},
\code{get_input},
\code{set_input_color},
\code{set_input_return}.
\end{funcdesc}

%---

\begin{funcdesc}{add_menu}{type\, x\, y\, w\, h\, name}
Add a menu object to the form. \\
Methods:
\code{set_menu},
\code{get_menu},
\code{addto_menu}.
\end{funcdesc}

\begin{funcdesc}{add_choice}{type\, x\, y\, w\, h\, name}
Add a choice object to the form. \\
Methods:
\code{set_choice},
\code{get_choice},
\code{clear_choice},
\code{addto_choice},
\code{replace_choice},
\code{delete_choice},
\code{get_choice_text},
\code{set_choice_fontsize},
\code{set_choice_fontstyle}.
\end{funcdesc}

\begin{funcdesc}{add_browser}{type\, x\, y\, w\, h\, name}
Add a browser object to the form. \\
Methods:
\code{set_browser_topline},
\code{clear_browser},
\code{add_browser_line},
\code{addto_browser},
\code{insert_browser_line},
\code{delete_browser_line},
\code{replace_browser_line},
\code{get_browser_line},
\code{load_browser},
\code{get_browser_maxline},
\code{select_browser_line},
\code{deselect_browser_line},
\code{deselect_browser},
\code{isselected_browser_line},
\code{get_browser},
\code{set_browser_fontsize},
\code{set_browser_fontstyle},
\code{set_browser_specialkey}.
\end{funcdesc}

%---

\begin{funcdesc}{add_timer}{type\, x\, y\, w\, h\, name}
Add a timer object to the form. \\
Methods:
\code{set_timer},
\code{get_timer}.
\end{funcdesc}
\end{flushleft}

Form objects have the following data attributes; see the FORMS
documentation:

\begin{tableiii}{|l|c|l|}{code}{Name}{Type}{Meaning}
  \lineiii{window}{int (read-only)}{GL window id}
  \lineiii{w}{float}{form width}
  \lineiii{h}{float}{form height}
  \lineiii{x}{float}{form x origin}
  \lineiii{y}{float}{form y origin}
  \lineiii{deactivated}{int}{nonzero if form is deactivated}
  \lineiii{visible}{int}{nonzero if form is visible}
  \lineiii{frozen}{int}{nonzero if form is frozen}
  \lineiii{doublebuf}{int}{nonzero if double buffering on}
\end{tableiii}

\subsection{FORMS Objects}

Besides methods specific to particular kinds of FORMS objects, all
FORMS objects also have the following methods:

\renewcommand{\indexsubitem}{(FORMS object method)}
\begin{funcdesc}{set_call_back}{function\, argument}
Set the object's callback function and argument.  When the object
needs interaction, the callback function will be called with two
arguments: the object, and the callback argument.  (FORMS objects
without a callback function are returned by \code{fl.do_forms()} or
\code{fl.check_forms()} when they need interaction.)  Call this method
without arguments to remove the callback function.
\end{funcdesc}

\begin{funcdesc}{delete_object}{}
  Delete the object.
\end{funcdesc}

\begin{funcdesc}{show_object}{}
  Show the object.
\end{funcdesc}

\begin{funcdesc}{hide_object}{}
  Hide the object.
\end{funcdesc}

\begin{funcdesc}{redraw_object}{}
  Redraw the object.
\end{funcdesc}

\begin{funcdesc}{freeze_object}{}
  Freeze the object.
\end{funcdesc}

\begin{funcdesc}{unfreeze_object}{}
  Unfreeze the object.
\end{funcdesc}

%\begin{funcdesc}{handle_object}{} XXX
%\end{funcdesc}

%\begin{funcdesc}{handle_object_direct}{} XXX
%\end{funcdesc}

FORMS objects have these data attributes; see the FORMS documentation:

\begin{tableiii}{|l|c|l|}{code}{Name}{Type}{Meaning}
  \lineiii{objclass}{int (read-only)}{object class}
  \lineiii{type}{int (read-only)}{object type}
  \lineiii{boxtype}{int}{box type}
  \lineiii{x}{float}{x origin}
  \lineiii{y}{float}{y origin}
  \lineiii{w}{float}{width}
  \lineiii{h}{float}{height}
  \lineiii{col1}{int}{primary color}
  \lineiii{col2}{int}{secondary color}
  \lineiii{align}{int}{alignment}
  \lineiii{lcol}{int}{label color}
  \lineiii{lsize}{float}{label font size}
  \lineiii{label}{string}{label string}
  \lineiii{lstyle}{int}{label style}
  \lineiii{pushed}{int (read-only)}{(see FORMS docs)}
  \lineiii{focus}{int (read-only)}{(see FORMS docs)}
  \lineiii{belowmouse}{int (read-only)}{(see FORMS docs)}
  \lineiii{frozen}{int (read-only)}{(see FORMS docs)}
  \lineiii{active}{int (read-only)}{(see FORMS docs)}
  \lineiii{input}{int (read-only)}{(see FORMS docs)}
  \lineiii{visible}{int (read-only)}{(see FORMS docs)}
  \lineiii{radio}{int (read-only)}{(see FORMS docs)}
  \lineiii{automatic}{int (read-only)}{(see FORMS docs)}
\end{tableiii}

\section{Standard Module \sectcode{FL}}
\nodename{FL (uppercase)}
\stmodindex{FL}

This module defines symbolic constants needed to use the built-in
module \code{fl} (see above); they are equivalent to those defined in
the C header file \file{<forms.h>} except that the name prefix
\samp{FL_} is omitted.  Read the module source for a complete list of
the defined names.  Suggested use:

\bcode\begin{verbatim}
import fl
from FL import *
\end{verbatim}\ecode

\section{Standard Module \sectcode{flp}}
\stmodindex{flp}

This module defines functions that can read form definitions created
by the `form designer' (\code{fdesign}) program that comes with the
FORMS library (see module \code{fl} above).

For now, see the file \file{flp.doc} in the Python library source
directory for a description.

XXX A complete description should be inserted here!

\documentstyle[twoside,11pt,myformat]{report}

% Things to do:
% Add a section on file I/O
% Write a chapter entitled ``Some Useful Modules''
%  --regex, math+cmath
% Should really move the Python startup file info to an appendix
%

\title{Python Tutorial -- DRAFT of \today}

\author{
	Guido van Rossum \\
	Dept. AA, CWI, P.O. Box 94079 \\
	1090 GB Amsterdam, The Netherlands \\
	E-mail: {\tt guido@cwi.nl}
}

\date{17 March 1995 \\ Release 1.2-proof-2} % XXX update before release!


\begin{document}

\pagenumbering{roman}

\maketitle

\strong{BEOPEN.COM TERMS AND CONDITIONS FOR PYTHON 2.0}

\centerline{\strong{BEOPEN PYTHON OPEN SOURCE LICENSE AGREEMENT VERSION 1}}

\begin{enumerate}

\item
This LICENSE AGREEMENT is between BeOpen.com (``BeOpen''), having an
office at 160 Saratoga Avenue, Santa Clara, CA 95051, and the
Individual or Organization (``Licensee'') accessing and otherwise
using this software in source or binary form and its associated
documentation (``the Software'').

\item
Subject to the terms and conditions of this BeOpen Python License
Agreement, BeOpen hereby grants Licensee a non-exclusive,
royalty-free, world-wide license to reproduce, analyze, test, perform
and/or display publicly, prepare derivative works, distribute, and
otherwise use the Software alone or in any derivative version,
provided, however, that the BeOpen Python License is retained in the
Software, alone or in any derivative version prepared by Licensee.

\item
BeOpen is making the Software available to Licensee on an ``AS IS''
basis.  BEOPEN MAKES NO REPRESENTATIONS OR WARRANTIES, EXPRESS OR
IMPLIED.  BY WAY OF EXAMPLE, BUT NOT LIMITATION, BEOPEN MAKES NO AND
DISCLAIMS ANY REPRESENTATION OR WARRANTY OF MERCHANTABILITY OR FITNESS
FOR ANY PARTICULAR PURPOSE OR THAT THE USE OF THE SOFTWARE WILL NOT
INFRINGE ANY THIRD PARTY RIGHTS.

\item
BEOPEN SHALL NOT BE LIABLE TO LICENSEE OR ANY OTHER USERS OF THE
SOFTWARE FOR ANY INCIDENTAL, SPECIAL, OR CONSEQUENTIAL DAMAGES OR LOSS
AS A RESULT OF USING, MODIFYING OR DISTRIBUTING THE SOFTWARE, OR ANY
DERIVATIVE THEREOF, EVEN IF ADVISED OF THE POSSIBILITY THEREOF.

\item
This License Agreement will automatically terminate upon a material
breach of its terms and conditions.

\item
This License Agreement shall be governed by and interpreted in all
respects by the law of the State of California, excluding conflict of
law provisions.  Nothing in this License Agreement shall be deemed to
create any relationship of agency, partnership, or joint venture
between BeOpen and Licensee.  This License Agreement does not grant
permission to use BeOpen trademarks or trade names in a trademark
sense to endorse or promote products or services of Licensee, or any
third party.  As an exception, the ``BeOpen Python'' logos available
at http://www.pythonlabs.com/logos.html may be used according to the
permissions granted on that web page.

\item
By copying, installing or otherwise using the software, Licensee
agrees to be bound by the terms and conditions of this License
Agreement.
\end{enumerate}


\centerline{\strong{CNRI OPEN SOURCE LICENSE AGREEMENT}}

Python 1.6 is made available subject to the terms and conditions in
CNRI's License Agreement.  This Agreement together with Python 1.6 may
be located on the Internet using the following unique, persistent
identifier (known as a handle): 1895.22/1012.  This Agreement may also
be obtained from a proxy server on the Internet using the following
URL: \url{http://hdl.handle.net/1895.22/1012}.


\centerline{\strong{CWI PERMISSIONS STATEMENT AND DISCLAIMER}}

Copyright \copyright{} 1991 - 1995, Stichting Mathematisch Centrum
Amsterdam, The Netherlands.  All rights reserved.

Permission to use, copy, modify, and distribute this software and its
documentation for any purpose and without fee is hereby granted,
provided that the above copyright notice appear in all copies and that
both that copyright notice and this permission notice appear in
supporting documentation, and that the name of Stichting Mathematisch
Centrum or CWI not be used in advertising or publicity pertaining to
distribution of the software without specific, written prior
permission.

STICHTING MATHEMATISCH CENTRUM DISCLAIMS ALL WARRANTIES WITH REGARD TO
THIS SOFTWARE, INCLUDING ALL IMPLIED WARRANTIES OF MERCHANTABILITY AND
FITNESS, IN NO EVENT SHALL STICHTING MATHEMATISCH CENTRUM BE LIABLE
FOR ANY SPECIAL, INDIRECT OR CONSEQUENTIAL DAMAGES OR ANY DAMAGES
WHATSOEVER RESULTING FROM LOSS OF USE, DATA OR PROFITS, WHETHER IN AN
ACTION OF CONTRACT, NEGLIGENCE OR OTHER TORTIOUS ACTION, ARISING OUT
OF OR IN CONNECTION WITH THE USE OR PERFORMANCE OF THIS SOFTWARE.


\begin{abstract}

\noindent
Python is a simple, yet powerful programming language that bridges the
gap between C and shell programming, and is thus ideally suited for
``throw-away programming'' and rapid prototyping.  Its syntax is put
together from constructs borrowed from a variety of other languages;
most prominent are influences from ABC, C, Modula-3 and Icon.

The Python interpreter is easily extended with new functions and data
types implemented in C.  Python is also suitable as an extension
language for highly customizable C applications such as editors or
window managers.

Python is available for many operating systems:
several flavors of \UNIX{}, the Apple Macintosh, MS-DOS, Windows
(3.1(1), '95 and NT flavors), OS/2, and others.

This tutorial introduces the reader informally to the basic concepts
and features of the Python language and system.  It helps to have a
Python interpreter handy for hands-on experience, but as the examples
are self-contained, the tutorial can be read off-line as well.

For a description of standard objects and modules, see the {\em Python
Library Reference} document.  The {\em Python Reference Manual} gives
a more formal definition of the language.

\end{abstract}

\pagebreak
{
\parskip = 0mm
\tableofcontents
}

\pagebreak

\pagenumbering{arabic}


\chapter{Whetting Your Appetite}

\section{Disclaimer}

Now that there are several books out on Python, this tutorial has lost
its role as the only introduction to Python for most new users.  This
tutorial does not attempt to be comprehensive and cover every single
feature, or even every commonly used feature.  Instead, it introduces
many of Python's most noteworthy features, and will give you a good
idea of the language's flavor and style.

%It takes time to keep a document like this up to date in the face of
%additions to the language, and I simply don't have enough time to do a
%good job.  Therefore, this version of the tutorial is almost unchanged
%since the previous release.  This doesn't mean that the tutorial is
%out of date --- all the examples still work exactly as before.  There
%are simply some new areas of the language that aren't covered.

%To make up for this, there are some chapters at the end that cover
%important changes in recent Python releases, and these are up to date
%with the current release.

\section{Introduction}

If you ever wrote a large shell script, you probably know this
feeling: you'd love to add yet another feature, but it's already so
slow, and so big, and so complicated; or the feature involves a system
call or other function that is only accessible from C \ldots Usually
the problem at hand isn't serious enough to warrant rewriting the
script in C; perhaps the problem requires variable-length strings or
other data types (like sorted lists of file names) that are easy in
the shell but lots of work to implement in C, or perhaps you're not
sufficiently familiar with C.

Another situation: perhaps you have to work with several C libraries,
and the usual C write/compile/test/re-compile cycle is too slow.  You
need to develop software more quickly.  Possibly perhaps you've
written a program that could use an extension language, and you don't
want to design a language, write and debug an interpreter for it, then
tie it into your application.

In such cases, Python may be just the language for you.  Python is
simple to use, but it is a real programming language, offering much
more structure and support for large programs than the shell has.  On
the other hand, it also offers much more error checking than C, and,
being a {\em very-high-level language}, it has high-level data types
built in, such as flexible arrays and dictionaries that would cost you
days to implement efficiently in C.  Because of its more general data
types Python is applicable to a much larger problem domain than {\em
Awk} or even {\em Perl}, yet many things are at least as easy in
Python as in those languages.

Python allows you to split up your program in modules that can be
reused in other Python programs.  It comes with a large collection of
standard modules that you can use as the basis of your programs --- or
as examples to start learning to program in Python.  There are also
built-in modules that provide things like file I/O, system calls,
sockets, and even interfaces to GUI toolkits like Tk.  

Python is an interpreted language, which can save you considerable time
during program development because no compilation and linking is
necessary.  The interpreter can be used interactively, which makes it
easy to experiment with features of the language, to write throw-away
programs, or to test functions during bottom-up program development.
It is also a handy desk calculator.

Python allows writing very compact and readable programs.  Programs
written in Python are typically much shorter than equivalent C
programs, for several reasons:
\begin{itemize}
\item
the high-level data types allow you to express complex operations in a
single statement;
\item
statement grouping is done by indentation instead of begin/end
brackets;
\item
no variable or argument declarations are necessary.
\end{itemize}

Python is {\em extensible}: if you know how to program in C it is easy
to add a new built-in function or module to the interpreter, either to
perform critical operations at maximum speed, or to link Python
programs to libraries that may only be available in binary form (such
as a vendor-specific graphics library).  Once you are really hooked,
you can link the Python interpreter into an application written in C
and use it as an extension or command language for that application.

By the way, the language is named after the BBC show ``Monty Python's
Flying Circus'' and has nothing to do with nasty reptiles.  Making
references to Monty Python skits in documentation is not only allowed,
it is encouraged.  

\section{Where From Here}

Now that you are all excited about Python, you'll want to examine it
in some more detail.  Since the best way to learn a language is
using it, you are invited here to do so.

In the next chapter, the mechanics of using the interpreter are
explained.  This is rather mundane information, but essential for
trying out the examples shown later.

The rest of the tutorial introduces various features of the Python
language and system though examples, beginning with simple
expressions, statements and data types, through functions and modules,
and finally touching upon advanced concepts like exceptions
and user-defined classes.

\chapter{Using the Python Interpreter}

\section{Invoking the Interpreter}

The Python interpreter is usually installed as {\tt /usr/local/bin/python}
on those machines where it is available; putting {\tt /usr/local/bin} in
your \UNIX{} shell's search path makes it possible to start it by
typing the command

\bcode\begin{verbatim}
python
\end{verbatim}\ecode
%
to the shell.  Since the choice of the directory where the interpreter
lives is an installation option, other places are possible; check with
your local Python guru or system administrator.  (E.g., {\tt
/usr/local/python} is a popular alternative location.)

Typing an EOF character (Control-D on \UNIX{}, Control-Z or F6 on DOS
or Windows) at the primary prompt causes the interpreter to exit with
a zero exit status.  If that doesn't work, you can exit the
interpreter by typing the following commands: \code{import sys ;
sys.exit()}.

The interpreter's line-editing features usually aren't very
sophisticated.  On Unix, whoever installed the interpreter may have
enabled support for the GNU readline library, which adds more
elaborate interactive editing and history features. Perhaps the
quickest check to see whether command line editing is supported is
typing Control-P to the first Python prompt you get.  If it beeps, you
have command line editing; see Appendix A for an introduction to the
keys.  If nothing appears to happen, or if \verb/^P/ is echoed,
command line editing isn't available; you'll only be able to use
backspace to remove characters from the current line.

The interpreter operates somewhat like the \UNIX{} shell: when called
with standard input connected to a tty device, it reads and executes
commands interactively; when called with a file name argument or with
a file as standard input, it reads and executes a {\em script} from
that file. 

A third way of starting the interpreter is
``{\tt python -c command [arg] ...}'', which
executes the statement(s) in {\tt command}, analogous to the shell's
{\tt -c} option.  Since Python statements often contain spaces or other
characters that are special to the shell, it is best to quote {\tt
command} in its entirety with double quotes.

Note that there is a difference between ``{\tt python file}'' and
``{\tt python <file}''.  In the latter case, input requests from the
program, such as calls to {\tt input()} and {\tt raw_input()}, are
satisfied from {\em file}.  Since this file has already been read
until the end by the parser before the program starts executing, the
program will encounter EOF immediately.  In the former case (which is
usually what you want) they are satisfied from whatever file or device
is connected to standard input of the Python interpreter.

When a script file is used, it is sometimes useful to be able to run
the script and enter interactive mode afterwards.  This can be done by
passing {\tt -i} before the script.  (This does not work if the script
is read from standard input, for the same reason as explained in the
previous paragraph.)

\subsection{Argument Passing}

When known to the interpreter, the script name and additional
arguments thereafter are passed to the script in the variable {\tt
sys.argv}, which is a list of strings.  Its length is at least one;
when no script and no arguments are given, {\tt sys.argv[0]} is an
empty string.  When the script name is given as {\tt '-'} (meaning
standard input), {\tt sys.argv[0]} is set to {\tt '-'}.  When {\tt -c
command} is used, {\tt sys.argv[0]} is set to {\tt '-c'}.  Options
found after {\tt -c command} are not consumed by the Python
interpreter's option processing but left in {\tt sys.argv} for the
command to handle.

\subsection{Interactive Mode}

When commands are read from a tty, the interpreter is said to be in
{\em interactive\ mode}.  In this mode it prompts for the next command
with the {\em primary\ prompt}, usually three greater-than signs ({\tt
>>>}); for continuation lines it prompts with the
{\em secondary\ prompt},
by default three dots ({\tt ...}).  

The interpreter prints a welcome message stating its version number
and a copyright notice before printing the first prompt, e.g.:

\bcode\begin{verbatim}
python
Python 1.4 (Oct 25 1996)  [GCC 2.7.2]
Copyright 1991-1996 Stichting Mathematisch Centrum, Amsterdam
>>>
\end{verbatim}\ecode

\section{The Interpreter and its Environment}

\subsection{Error Handling}

When an error occurs, the interpreter prints an error
message and a stack trace.  In interactive mode, it then returns to
the primary prompt; when input came from a file, it exits with a
nonzero exit status after printing
the stack trace.  (Exceptions handled by an {\tt except} clause in a
{\tt try} statement are not errors in this context.)  Some errors are
unconditionally fatal and cause an exit with a nonzero exit; this
applies to internal inconsistencies and some cases of running out of
memory.  All error messages are written to the standard error stream;
normal output from the executed commands is written to standard
output.

Typing the interrupt character (usually Control-C or DEL) to the
primary or secondary prompt cancels the input and returns to the
primary prompt.%
\footnote{
        A problem with the GNU Readline package may prevent this.
}
Typing an interrupt while a command is executing raises the {\tt
KeyboardInterrupt} exception, which may be handled by a {\tt try}
statement.

\subsection{Executable Python scripts}

On BSD'ish \UNIX{} systems, Python scripts can be made directly
executable, like shell scripts, by putting the line

\bcode\begin{verbatim}
#! /usr/local/bin/python
\end{verbatim}\ecode
%
(assuming that's the name of the interpreter) at the beginning of the
script and giving the file an executable mode.  The {\tt \#!} must be
the first two characters of the file.

\subsection{The Interactive Startup File}

XXX This should probably be dumped in an appendix, since most people
don't use Python interactively in non-trivial ways.

When you use Python interactively, it is frequently handy to have some
standard commands executed every time the interpreter is started.  You
can do this by setting an environment variable named {\tt
PYTHONSTARTUP} to the name of a file containing your start-up
commands.  This is similar to the {\tt .profile} feature of the \UNIX{}
shells.

This file is only read in interactive sessions, not when Python reads
commands from a script, and not when {\tt /dev/tty} is given as the
explicit source of commands (which otherwise behaves like an
interactive session).  It is executed in the same name space where
interactive commands are executed, so that objects that it defines or
imports can be used without qualification in the interactive session.
You can also change the prompts {\tt sys.ps1} and {\tt sys.ps2} in
this file.

If you want to read an additional start-up file from the current
directory, you can program this in the global start-up file, e.g.
\verb\execfile('.pythonrc')\.  If you want to use the startup file
in a script, you must write this explicitly in the script, e.g.
\verb\import os;\ \verb\execfile(os.environ['PYTHONSTARTUP'])\.

\chapter{An Informal Introduction to Python}

In the following examples, input and output are distinguished by the
presence or absence of prompts ({\tt >>>} and {\tt ...}): to repeat
the example, you must type everything after the prompt, when the
prompt appears; lines that do not begin with a prompt are output from
the interpreter.%
%\footnote{
%        I'd prefer to use different fonts to distinguish input
%        from output, but the amount of LaTeX hacking that would require
%        is currently beyond my ability.
%}
Note that a secondary prompt on a line by itself in an example means
you must type a blank line; this is used to end a multi-line command.

\section{Using Python as a Calculator}

Let's try some simple Python commands.  Start the interpreter and wait
for the primary prompt, {\tt >>>}.  (It shouldn't take long.)

\subsection{Numbers}

The interpreter acts as a simple calculator: you can type an
expression at it and it will write the value.  Expression syntax is
straightforward: the operators {\tt +}, {\tt -}, {\tt *} and {\tt /}
work just like in most other languages (e.g., Pascal or C); parentheses
can be used for grouping.  For example:

\bcode\begin{verbatim}
>>> 2+2
4
>>> # This is a comment
... 2+2
4
>>> 2+2  # and a comment on the same line as code
4
>>> (50-5*6)/4
5
>>> # Integer division returns the floor:
... 7/3
2
>>> 7/-3
-3
>>> 
\end{verbatim}\ecode
%
Like in C, the equal sign ({\tt =}) is used to assign a value to a
variable.  The value of an assignment is not written:

\bcode\begin{verbatim}
>>> width = 20
>>> height = 5*9
>>> width * height
900
>>> 
\end{verbatim}\ecode
%
A value can be assigned to several variables simultaneously:

\bcode\begin{verbatim}
>>> x = y = z = 0  # Zero x, y and z
>>> x
0
>>> y
0
>>> z
0
>>> 
\end{verbatim}\ecode
%
There is full support for floating point; operators with mixed type
operands convert the integer operand to floating point:

\bcode\begin{verbatim}
>>> 4 * 2.5 / 3.3
3.0303030303
>>> 7.0 / 2
3.5
\end{verbatim}\ecode
%
Complex numbers are also supported; imaginary numbers are written with
a suffix of \code{'j'} or \code{'J'}.  Complex numbers with a nonzero
real component are written as \code{(\var{real}+\var{imag}j)}, or can
be created with the \code{complex(\var{real}, \var{imag})} function.

\bcode\begin{verbatim}
>>> 1j * 1J
(-1+0j)
>>> 1j * complex(0,1)
(-1+0j)
>>> 3+1j*3
(3+3j)
>>> (3+1j)*3
(9+3j)
>>> (1+2j)/(1+1j)
(1.5+0.5j)
\end{verbatim}\ecode
%
Complex numbers are always represented as two floating point numbers,
the real and imaginary part.  To extract these parts from a complex
number \code{z}, use \code{z.real} and \code{z.imag}.  

\bcode\begin{verbatim}
>>> a=1.5+0.5j
>>> a.real
1.5
>>> a.imag
0.5
\end{verbatim}\ecode
%
The conversion functions to floating point and integer
(\code{float()}, \code{int()} and \code{long()}) don't work for
complex numbers --- there is no one correct way to convert a complex
number to a real number.  Use \code{abs(z)} to get its magnitude (as a
float) or \code{z.real} to get its real part.

\bcode\begin{verbatim}
>>> a=1.5+0.5j
>>> float(a)
Traceback (innermost last):
  File "<stdin>", line 1, in ?
TypeError: can't convert complex to float; use e.g. abs(z)
>>> a.real
1.5
>>> abs(a)
1.58113883008
\end{verbatim}\ecode
%
In interactive mode, the last printed expression is assigned to the
variable \code{_}.  This means that when you are using Python as a
desk calculator, it is somewhat easier to continue calculations, for
example:

\begin{verbatim}
>>> tax = 17.5 / 100
>>> price = 3.50
>>> price * tax
0.6125
>>> price + _
4.1125
>>> round(_, 2)
4.11
\end{verbatim}

This variable should be treated as read-only by the user.  Don't
explicitly assign a value to it --- you would create an independent
local variable with the same name masking the built-in variable with
its magic behavior.

\subsection{Strings}

Besides numbers, Python can also manipulate strings, which can be
expressed in several ways.  They can be enclosed in single quotes or
double quotes:

\bcode\begin{verbatim}
>>> 'spam eggs'
'spam eggs'
>>> 'doesn\'t'
"doesn't"
>>> "doesn't"
"doesn't"
>>> '"Yes," he said.'
'"Yes," he said.'
>>> "\"Yes,\" he said."
'"Yes," he said.'
>>> '"Isn\'t," she said.'
'"Isn\'t," she said.'
>>> 
\end{verbatim}\ecode
%
String literals can span multiple lines in several ways.  Newlines can be escaped with backslashes, e.g.

\begin{verbatim}
hello = "This is a rather long string containing\n\
several lines of text just as you would do in C.\n\
    Note that whitespace at the beginning of the line is\
 significant.\n"
print hello
\end{verbatim}

which would print the following:
\begin{verbatim}
This is a rather long string containing
several lines of text just as you would do in C.
    Note that whitespace at the beginning of the line is significant.
\end{verbatim}

Or, strings can be surrounded in a pair of matching triple-quotes:
\code{"""} or \code {'''}.  End of lines do not need to be escaped
when using triple-quotes, but they will be included in the string.

\begin{verbatim}
print """
Usage: thingy [OPTIONS] 
     -h                        Display this usage message
     -H hostname               Hostname to connect to
"""
\end{verbatim}

produces the following output:

\bcode\begin{verbatim}
Usage: thingy [OPTIONS] 
     -h                        Display this usage message
     -H hostname               Hostname to connect to
\end{verbatim}\ecode
%
The interpreter prints the result of string operations in the same way
as they are typed for input: inside quotes, and with quotes and other
funny characters escaped by backslashes, to show the precise
value.  The string is enclosed in double quotes if the string contains
a single quote and no double quotes, else it's enclosed in single
quotes.  (The {\tt print} statement, described later, can be used to
write strings without quotes or escapes.)

Strings can be concatenated (glued together) with the {\tt +}
operator, and repeated with {\tt *}:

\bcode\begin{verbatim}
>>> word = 'Help' + 'A'
>>> word
'HelpA'
>>> '<' + word*5 + '>'
'<HelpAHelpAHelpAHelpAHelpA>'
>>> 
\end{verbatim}\ecode
%
Two string literals next to each other are automatically concatenated;
the first line above could also have been written \code{word = 'Help'
'A'}; this only works with two literals, not with arbitrary string expressions.

Strings can be subscripted (indexed); like in C, the first character
of a string has subscript (index) 0.  There is no separate character
type; a character is simply a string of size one.  Like in Icon,
substrings can be specified with the {\em slice} notation: two indices
separated by a colon.

\bcode\begin{verbatim}
>>> word[4]
'A'
>>> word[0:2]
'He'
>>> word[2:4]
'lp'
>>> 
\end{verbatim}\ecode
%
Slice indices have useful defaults; an omitted first index defaults to
zero, an omitted second index defaults to the size of the string being
sliced.

\bcode\begin{verbatim}
>>> word[:2]    # The first two characters
'He'
>>> word[2:]    # All but the first two characters
'lpA'
>>>
\end{verbatim}\ecode
%
Here's a useful invariant of slice operations: \verb\s[:i] + s[i:]\
equals \verb\s\.

\bcode\begin{verbatim}
>>> word[:2] + word[2:]
'HelpA'
>>> word[:3] + word[3:]
'HelpA'
>>> 
\end{verbatim}\ecode
%
Degenerate slice indices are handled gracefully: an index that is too
large is replaced by the string size, an upper bound smaller than the
lower bound returns an empty string.

\bcode\begin{verbatim}
>>> word[1:100]
'elpA'
>>> word[10:]
''
>>> word[2:1]
''
>>> 
\end{verbatim}\ecode
%
Indices may be negative numbers, to start counting from the right.
For example:

\bcode\begin{verbatim}
>>> word[-1]     # The last character
'A'
>>> word[-2]     # The last-but-one character
'p'
>>> word[-2:]    # The last two characters
'pA'
>>> word[:-2]    # All but the last two characters
'Hel'
>>>
\end{verbatim}\ecode
%
But note that -0 is really the same as 0, so it does not count from
the right!

\bcode\begin{verbatim}
>>> word[-0]     # (since -0 equals 0)
'H'
>>>
\end{verbatim}\ecode
%
Out-of-range negative slice indices are truncated, but don't try this
for single-element (non-slice) indices:

\bcode\begin{verbatim}
>>> word[-100:]
'HelpA'
>>> word[-10]    # error
Traceback (innermost last):
  File "<stdin>", line 1
IndexError: string index out of range
>>> 
\end{verbatim}\ecode
%
The best way to remember how slices work is to think of the indices as
pointing {\em between} characters, with the left edge of the first
character numbered 0.  Then the right edge of the last character of a
string of {\tt n} characters has index {\tt n}, for example:

\bcode\begin{verbatim}
 +---+---+---+---+---+ 
 | H | e | l | p | A |
 +---+---+---+---+---+ 
 0   1   2   3   4   5 
-5  -4  -3  -2  -1
\end{verbatim}\ecode
%
The first row of numbers gives the position of the indices 0...5 in
the string; the second row gives the corresponding negative indices.
The slice from \verb\i\ to \verb\j\ consists of all characters between
the edges labeled \verb\i\ and \verb\j\, respectively.

For nonnegative indices, the length of a slice is the difference of
the indices, if both are within bounds, e.g., the length of
\verb\word[1:3]\ is 2.

The built-in function {\tt len()} returns the length of a string:

\bcode\begin{verbatim}
>>> s = 'supercalifragilisticexpialidocious'
>>> len(s)
34
>>> 
\end{verbatim}\ecode

\subsection{Lists}

Python knows a number of {\em compound} data types, used to group
together other values.  The most versatile is the {\em list}, which
can be written as a list of comma-separated values (items) between
square brackets.  List items need not all have the same type.

\bcode\begin{verbatim}
>>> a = ['spam', 'eggs', 100, 1234]
>>> a
['spam', 'eggs', 100, 1234]
>>> 
\end{verbatim}\ecode
%
Like string indices, list indices start at 0, and lists can be sliced,
concatenated and so on:

\bcode\begin{verbatim}
>>> a[0]
'spam'
>>> a[3]
1234
>>> a[-2]
100
>>> a[1:-1]
['eggs', 100]
>>> a[:2] + ['bacon', 2*2]
['spam', 'eggs', 'bacon', 4]
>>> 3*a[:3] + ['Boe!']
['spam', 'eggs', 100, 'spam', 'eggs', 100, 'spam', 'eggs', 100, 'Boe!']
>>> 
\end{verbatim}\ecode
%
Unlike strings, which are {\em immutable}, it is possible to change
individual elements of a list:

\bcode\begin{verbatim}
>>> a
['spam', 'eggs', 100, 1234]
>>> a[2] = a[2] + 23
>>> a
['spam', 'eggs', 123, 1234]
>>>
\end{verbatim}\ecode
%
Assignment to slices is also possible, and this can even change the size
of the list:

\bcode\begin{verbatim}
>>> # Replace some items:
... a[0:2] = [1, 12]
>>> a
[1, 12, 123, 1234]
>>> # Remove some:
... a[0:2] = []
>>> a
[123, 1234]
>>> # Insert some:
... a[1:1] = ['bletch', 'xyzzy']
>>> a
[123, 'bletch', 'xyzzy', 1234]
>>> a[:0] = a     # Insert (a copy of) itself at the beginning
>>> a
[123, 'bletch', 'xyzzy', 1234, 123, 'bletch', 'xyzzy', 1234]
>>> 
\end{verbatim}\ecode
%
The built-in function {\tt len()} also applies to lists:

\bcode\begin{verbatim}
>>> len(a)
8
>>> 
\end{verbatim}\ecode
%
It is possible to nest lists (create lists containing other lists),
for example:

\bcode\begin{verbatim}
>>> q = [2, 3]
>>> p = [1, q, 4]
>>> len(p)
3
>>> p[1]
[2, 3]
>>> p[1][0]
2
>>> p[1].append('xtra')     # See section 5.1
>>> p
[1, [2, 3, 'xtra'], 4]
>>> q
[2, 3, 'xtra']
>>>
\end{verbatim}\ecode
%
Note that in the last example, {\tt p[1]} and {\tt q} really refer to
the same object!  We'll come back to {\em object semantics} later.

\section{First Steps Towards Programming}

Of course, we can use Python for more complicated tasks than adding
two and two together.  For instance, we can write an initial
subsequence of the {\em Fibonacci} series as follows:

\bcode\begin{verbatim}
>>> # Fibonacci series:
... # the sum of two elements defines the next
... a, b = 0, 1
>>> while b < 10:
...       print b
...       a, b = b, a+b
... 
1
1
2
3
5
8
>>> 
\end{verbatim}\ecode
%
This example introduces several new features.

\begin{itemize}

\item
The first line contains a {\em multiple assignment}: the variables
{\tt a} and {\tt b} simultaneously get the new values 0 and 1.  On the
last line this is used again, demonstrating that the expressions on
the right-hand side are all evaluated first before any of the
assignments take place.

\item
The {\tt while} loop executes as long as the condition (here: {\tt b <
10}) remains true.  In Python, like in C, any non-zero integer value is
true; zero is false.  The condition may also be a string or list value,
in fact any sequence; anything with a non-zero length is true, empty
sequences are false.  The test used in the example is a simple
comparison.  The standard comparison operators are written the same as
in C: {\tt <}, {\tt >}, {\tt ==}, {\tt <=}, {\tt >=} and {\tt !=}.

\item
The {\em body} of the loop is {\em indented}: indentation is Python's
way of grouping statements.  Python does not (yet!) provide an
intelligent input line editing facility, so you have to type a tab or
space(s) for each indented line.  In practice you will prepare more
complicated input for Python with a text editor; most text editors have
an auto-indent facility.  When a compound statement is entered
interactively, it must be followed by a blank line to indicate
completion (since the parser cannot guess when you have typed the last
line).

\item
The {\tt print} statement writes the value of the expression(s) it is
given.  It differs from just writing the expression you want to write
(as we did earlier in the calculator examples) in the way it handles
multiple expressions and strings.  Strings are printed without quotes,
and a space is inserted between items, so you can format things nicely,
like this:

\bcode\begin{verbatim}
>>> i = 256*256
>>> print 'The value of i is', i
The value of i is 65536
>>> 
\end{verbatim}\ecode
%
A trailing comma avoids the newline after the output:

\bcode\begin{verbatim}
>>> a, b = 0, 1
>>> while b < 1000:
...     print b,
...     a, b = b, a+b
... 
1 1 2 3 5 8 13 21 34 55 89 144 233 377 610 987
>>> 
\end{verbatim}\ecode
%
Note that the interpreter inserts a newline before it prints the next
prompt if the last line was not completed.

\end{itemize}


\chapter{More Control Flow Tools}

Besides the {\tt while} statement just introduced, Python knows the
usual control flow statements known from other languages, with some
twists.

\section{If Statements}

Perhaps the most well-known statement type is the {\tt if} statement.
For example:

\bcode\begin{verbatim}
>>> if x < 0:
...      x = 0
...      print 'Negative changed to zero'
... elif x == 0:
...      print 'Zero'
... elif x == 1:
...      print 'Single'
... else:
...      print 'More'
... 
\end{verbatim}\ecode
%
There can be zero or more {\tt elif} parts, and the {\tt else} part is
optional.  The keyword `{\tt elif}' is short for `{\tt else if}', and is
useful to avoid excessive indentation.  An {\tt if...elif...elif...}
sequence is a substitute for the {\em switch} or {\em case} statements
found in other languages.

\section{For Statements}

The {\tt for} statement in Python differs a bit from what you may be
used to in C or Pascal.  Rather than always iterating over an
arithmetic progression of numbers (like in Pascal), or leaving the user
completely free in the iteration test and step (as C), Python's {\tt
for} statement iterates over the items of any sequence (e.g., a list
or a string), in the order that they appear in the sequence.  For
example (no pun intended):

\bcode\begin{verbatim}
>>> # Measure some strings:
... a = ['cat', 'window', 'defenestrate']
>>> for x in a:
...     print x, len(x)
... 
cat 3
window 6
defenestrate 12
>>> 
\end{verbatim}\ecode
%
It is not safe to modify the sequence being iterated over in the loop
(this can only happen for mutable sequence types, i.e., lists).  If
you need to modify the list you are iterating over, e.g., duplicate
selected items, you must iterate over a copy.  The slice notation
makes this particularly convenient:

\bcode\begin{verbatim}
>>> for x in a[:]: # make a slice copy of the entire list
...    if len(x) > 6: a.insert(0, x)
... 
>>> a
['defenestrate', 'cat', 'window', 'defenestrate']
>>> 
\end{verbatim}\ecode

\section{The {\tt range()} Function}

If you do need to iterate over a sequence of numbers, the built-in
function {\tt range()} comes in handy.  It generates lists containing
arithmetic progressions, e.g.:

\bcode\begin{verbatim}
>>> range(10)
[0, 1, 2, 3, 4, 5, 6, 7, 8, 9]
>>> 
\end{verbatim}\ecode
%
The given end point is never part of the generated list; {\tt range(10)}
generates a list of 10 values, exactly the legal indices for items of a
sequence of length 10.  It is possible to let the range start at another
number, or to specify a different increment (even negative):

\bcode\begin{verbatim}
>>> range(5, 10)
[5, 6, 7, 8, 9]
>>> range(0, 10, 3)
[0, 3, 6, 9]
>>> range(-10, -100, -30)
[-10, -40, -70]
>>> 
\end{verbatim}\ecode
%
To iterate over the indices of a sequence, combine {\tt range()} and
{\tt len()} as follows:

\bcode\begin{verbatim}
>>> a = ['Mary', 'had', 'a', 'little', 'lamb']
>>> for i in range(len(a)):
...     print i, a[i]
... 
0 Mary
1 had
2 a
3 little
4 lamb
>>> 
\end{verbatim}\ecode

\section{Break and Continue Statements, and Else Clauses on Loops}

The {\tt break} statement, like in C, breaks out of the smallest
enclosing {\tt for} or {\tt while} loop.

The {\tt continue} statement, also borrowed from C, continues with the
next iteration of the loop.

Loop statements may have an {\tt else} clause; it is executed when the
loop terminates through exhaustion of the list (with {\tt for}) or when
the condition becomes false (with {\tt while}), but not when the loop is
terminated by a {\tt break} statement.  This is exemplified by the
following loop, which searches for prime numbers:

\bcode\begin{verbatim}
>>> for n in range(2, 10):
...     for x in range(2, n):
...         if n % x == 0:
...            print n, 'equals', x, '*', n/x
...            break
...     else:
...          print n, 'is a prime number'
... 
2 is a prime number
3 is a prime number
4 equals 2 * 2
5 is a prime number
6 equals 2 * 3
7 is a prime number
8 equals 2 * 4
9 equals 3 * 3
>>> 
\end{verbatim}\ecode

\section{Pass Statements}

The {\tt pass} statement does nothing.
It can be used when a statement is required syntactically but the
program requires no action.
For example:

\bcode\begin{verbatim}
>>> while 1:
...       pass # Busy-wait for keyboard interrupt
... 
\end{verbatim}\ecode

\section{Defining Functions}

We can create a function that writes the Fibonacci series to an
arbitrary boundary:

\bcode\begin{verbatim}
>>> def fib(n):    # write Fibonacci series up to n
...     "Print a Fibonacci series up to n"
...     a, b = 0, 1
...     while b < n:
...           print b,
...           a, b = b, a+b
... 
>>> # Now call the function we just defined:
... fib(2000)
1 1 2 3 5 8 13 21 34 55 89 144 233 377 610 987 1597
>>> 
\end{verbatim}\ecode
%
The keyword {\tt def} introduces a function {\em definition}.  It must
be followed by the function name and the parenthesized list of formal
parameters.  The statements that form the body of the function start
at the next line, indented by a tab stop.  The first statement of the
function body can optionally be a string literal; this string literal
is the function's documentation string, or \dfn{docstring}.  There are
tools which use docstrings to automatically produce printed
documentation, or to let the user interactively browse through code;
it's good practice to include docstrings in code that you write, so
try to make a habit of it.

The {\em execution} of a function introduces a new symbol table used
for the local variables of the function.  More precisely, all variable
assignments in a function store the value in the local symbol table;
whereas variable references first look in the local symbol table, then
in the global symbol table, and then in the table of built-in names.
Thus, 
global variables cannot be directly assigned a value within a
function (unless named in a {\tt global} statement), although
they may be referenced.

The actual parameters (arguments) to a function call are introduced in
the local symbol table of the called function when it is called; thus,
arguments are passed using {\em call\ by\ value}.%
\footnote{
         Actually, {\em call  by  object reference} would be a better
         description, since if a mutable object is passed, the caller
         will see any changes the callee makes to it (e.g., items
         inserted into a list).
}
When a function calls another function, a new local symbol table is
created for that call.

A function definition introduces the function name in the
current
symbol table.  The value
of the function name
has a type that is recognized by the interpreter as a user-defined
function.  This value can be assigned to another name which can then
also be used as a function.  This serves as a general renaming
mechanism:

\bcode\begin{verbatim}
>>> fib
<function object at 10042ed0>
>>> f = fib
>>> f(100)
1 1 2 3 5 8 13 21 34 55 89
>>> 
\end{verbatim}\ecode
%
You might object that {\tt fib} is not a function but a procedure.  In
Python, like in C, procedures are just functions that don't return a
value.  In fact, technically speaking, procedures do return a value,
albeit a rather boring one.  This value is called {\tt None} (it's a
built-in name).  Writing the value {\tt None} is normally suppressed by
the interpreter if it would be the only value written.  You can see it
if you really want to:

\bcode\begin{verbatim}
>>> print fib(0)
None
>>> 
\end{verbatim}\ecode
%
It is simple to write a function that returns a list of the numbers of
the Fibonacci series, instead of printing it:

\bcode\begin{verbatim}
>>> def fib2(n): # return Fibonacci series up to n
...     "Return a list containing the Fibonacci series up to n"
...     result = []
...     a, b = 0, 1
...     while b < n:
...           result.append(b)    # see below
...           a, b = b, a+b
...     return result
... 
>>> f100 = fib2(100)    # call it
>>> f100                # write the result
[1, 1, 2, 3, 5, 8, 13, 21, 34, 55, 89]
>>> 
\end{verbatim}\ecode
%
This example, as usual, demonstrates some new Python features:

\begin{itemize}

\item
The {\tt return} statement returns with a value from a function.  {\tt
return} without an expression argument is used to return from the middle
of a procedure (falling off the end also returns from a procedure), in
which case the {\tt None} value is returned.

\item
The statement {\tt result.append(b)} calls a {\em method} of the list
object {\tt result}.  A method is a function that `belongs' to an
object and is named {\tt obj.methodname}, where {\tt obj} is some
object (this may be an expression), and {\tt methodname} is the name
of a method that is defined by the object's type.  Different types
define different methods.  Methods of different types may have the
same name without causing ambiguity.  (It is possible to define your
own object types and methods, using {\em classes}, as discussed later
in this tutorial.)
The method {\tt append} shown in the example, is defined for
list objects; it adds a new element at the end of the list.  In this
example
it is equivalent to {\tt result = result + [b]}, but more efficient.

\end{itemize}

\section{More on Defining Functions}

It is also possible to define functions with a variable number of
arguments.  There are three forms, which can be combined.

\subsection{Default Argument Values}

The most useful form is to specify a default value for one or more
arguments.  This creates a function that can be called with fewer
arguments than it is defined, e.g.

\begin{verbatim}
        def ask_ok(prompt, retries = 4, complaint = 'Yes or no, please!'):
                while 1:
                        ok = raw_input(prompt)
                        if ok in ('y', 'ye', 'yes'): return 1
                        if ok in ('n', 'no', 'nop', 'nope'): return 0
                        retries = retries - 1
                        if retries < 0: raise IOError, 'refusenik user'
                        print complaint
\end{verbatim}

This function can be called either like this:
\verb\ask_ok('Do you really want to quit?')\ or like this:
\verb\ask_ok('OK to overwrite the file?', 2)\.

The default values are evaluated at the point of function definition
in the {\em defining} scope, so that e.g.

\begin{verbatim}
        i = 5
        def f(arg = i): print arg
        i = 6
        f()
\end{verbatim}

will print \verb\5\.

\subsection{Keyword Arguments}

Functions can also be called using
keyword arguments of the form \code{\var{keyword} = \var{value}}.  For
instance, the following function:

\begin{verbatim}
def parrot(voltage, state='a stiff', action='voom', type='Norwegian Blue'):
    print "-- This parrot wouldn't", action,
    print "if you put", voltage, "Volts through it."
    print "-- Lovely plumage, the", type
    print "-- It's", state, "!"
\end{verbatim}

could be called in any of the following ways:

\begin{verbatim}
parrot(1000)
parrot(action = 'VOOOOOM', voltage = 1000000)
parrot('a thousand', state = 'pushing up the daisies')
parrot('a million', 'bereft of life', 'jump')
\end{verbatim}

but the following calls would all be invalid:

\begin{verbatim}
parrot()                     # required argument missing
parrot(voltage=5.0, 'dead')  # non-keyword argument following keyword
parrot(110, voltage=220)     # duplicate value for argument
parrot(actor='John Cleese')  # unknown keyword
\end{verbatim}

In general, an argument list must have any positional arguments
followed by any keyword arguments, where the keywords must be chosen
from the formal parameter names.  It's not important whether a formal
parameter has a default value or not.  No argument must receive a
value more than once --- formal parameter names corresponding to
positional arguments cannot be used as keywords in the same calls.

When a final formal parameter of the form \code{**\var{name}} is
present, it receives a dictionary containing all keyword arguments
whose keyword doesn't correspond to a formal parameter.  This may be
combined with a formal parameter of the form \code{*\var{name}}
(described in the next subsection) which receives a tuple containing
the positional arguments beyond the formal parameter list.
(\code{*\var{name}} must occur before \code{**\var{name}}.)  For
example, if we define a function like this:

\begin{verbatim}
def cheeseshop(kind, *arguments, **keywords):
    print "-- Do you have any", kind, '?'
    print "-- I'm sorry, we're all out of", kind
    for arg in arguments: print arg
    print '-'*40
    for kw in keywords.keys(): print kw, ':', keywords[kw]
\end{verbatim}

It could be called like this:

\begin{verbatim}
cheeseshop('Limburger', "It's very runny, sir.",
           "It's really very, VERY runny, sir.",
           client='John Cleese',
           shopkeeper='Michael Palin',
           sketch='Cheese Shop Sketch')
\end{verbatim}

and of course it would print:

\begin{verbatim}
-- Do you have any Limburger ?
-- I'm sorry, we're all out of Limburger
It's very runny, sir.
It's really very, VERY runny, sir.
----------------------------------------
client : John Cleese
shopkeeper : Michael Palin
sketch : Cheese Shop Sketch
\end{verbatim}

\subsection{Arbitrary Argument Lists}

Finally, the least frequently used option is to specify that a
function can be called with an arbitrary number of arguments.  These
arguments will be wrapped up in a tuple.  Before the variable number
of arguments, zero or more normal arguments may occur.

\begin{verbatim}
        def fprintf(file, format, *args):
                file.write(format % args)
\end{verbatim}

\chapter{Data Structures}

This chapter describes some things you've learned about already in
more detail, and adds some new things as well.

\section{More on Lists}

The list data type has some more methods.  Here are all of the methods
of lists objects:

\begin{description}

\item[{\tt insert(i, x)}]
Insert an item at a given position.  The first argument is the index of
the element before which to insert, so {\tt a.insert(0, x)} inserts at
the front of the list, and {\tt a.insert(len(a), x)} is equivalent to
{\tt a.append(x)}.

\item[{\tt append(x)}]
Equivalent to {\tt a.insert(len(a), x)}.

\item[{\tt index(x)}]
Return the index in the list of the first item whose value is {\tt x}.
It is an error if there is no such item.

\item[{\tt remove(x)}]
Remove the first item from the list whose value is {\tt x}.
It is an error if there is no such item.

\item[{\tt sort()}]
Sort the items of the list, in place.

\item[{\tt reverse()}]
Reverse the elements of the list, in place.

\item[{\tt count(x)}]
Return the number of times {\tt x} appears in the list.

\end{description}

An example that uses all list methods:

\bcode\begin{verbatim}
>>> a = [66.6, 333, 333, 1, 1234.5]
>>> print a.count(333), a.count(66.6), a.count('x')
2 1 0
>>> a.insert(2, -1)
>>> a.append(333)
>>> a
[66.6, 333, -1, 333, 1, 1234.5, 333]
>>> a.index(333)
1
>>> a.remove(333)
>>> a
[66.6, -1, 333, 1, 1234.5, 333]
>>> a.reverse()
>>> a
[333, 1234.5, 1, 333, -1, 66.6]
>>> a.sort()
>>> a
[-1, 1, 66.6, 333, 333, 1234.5]
>>>
\end{verbatim}\ecode

\subsection{Functional Programming Tools}

There are three built-in functions that are very useful when used with
lists: \verb\filter\, \verb\map\, and \verb\reduce\.

\verb\filter(function, sequence)\ returns a sequence (of the same
type, if possible) consisting of those items from the sequence for
which \verb\function(item)\ is true.  For example, to compute some
primes:

\begin{verbatim}
        >>> def f(x): return x%2 != 0 and x%3 != 0
        ...
        >>> filter(f, range(2, 25))
        [5, 7, 11, 13, 17, 19, 23]
        >>>
\end{verbatim}

\verb\map(function, sequence)\ calls \verb\function(item)\ for each of
the sequence's items and returns a list of the return values.  For
example, to compute some cubes:

\begin{verbatim}
	>>> def cube(x): return x*x*x
        ...
        >>> map(cube, range(1, 11))
        [1, 8, 27, 64, 125, 216, 343, 512, 729, 1000]
        >>>
\end{verbatim}

More than one sequence may be passed; the function must then have as
many arguments as there are sequences and is called with the
corresponding item from each sequence (or \verb\None\ if some sequence
is shorter than another).  If \verb\None\ is passed for the function,
a function returning its argument(s) is substituted.

Combining these two special cases, we see that
\verb\map(None, list1, list2)\  is a convenient way of turning a pair
of lists into a list of pairs.  For example:

\begin{verbatim}
        >>> seq = range(8)
	>>> def square(x): return x*x
        ...
        >>> map(None, seq, map(square, seq))
        [(0, 0), (1, 1), (2, 4), (3, 9), (4, 16), (5, 25), (6, 36), (7, 49)]
        >>> 
\end{verbatim}

\verb\reduce(func, sequence)\ returns a single value constructed
by calling the binary function \verb\func\ on the first two items of the
sequence, then on the result and the next item, and so on.  For
example, to compute the sum of the numbers 1 through 10:

\begin{verbatim}
	>>> def add(x,y): return x+y
        ...
        >>> reduce(add, range(1, 11))
        55
        >>> 
\end{verbatim}

If there's only one item in the sequence, its value is returned; if
the sequence is empty, an exception is raised.

A third argument can be passed to indicate the starting value.  In this
case the starting value is returned for an empty sequence, and the
function is first applied to the starting value and the first sequence
item, then to the result and the next item, and so on.  For example,

\begin{verbatim}
        >>> def sum(seq):
        ...     def add(x,y): return x+y
        ...     return reduce(add, seq, 0)
        ... 
        >>> sum(range(1, 11))
        55
        >>> sum([])
        0
        >>> 
\end{verbatim}

\section{The {\tt del} statement}

There is a way to remove an item from a list given its index instead
of its value: the {\tt del} statement.  This can also be used to
remove slices from a list (which we did earlier by assignment of an
empty list to the slice).  For example:

\bcode\begin{verbatim}
>>> a
[-1, 1, 66.6, 333, 333, 1234.5]
>>> del a[0]
>>> a
[1, 66.6, 333, 333, 1234.5]
>>> del a[2:4]
>>> a
[1, 66.6, 1234.5]
>>>
\end{verbatim}\ecode
%
{\tt del} can also be used to delete entire variables:

\bcode\begin{verbatim}
>>> del a
>>>
\end{verbatim}\ecode
%
Referencing the name {\tt a} hereafter is an error (at least until
another value is assigned to it).  We'll find other uses for {\tt del}
later.

\section{Tuples and Sequences}

We saw that lists and strings have many common properties, e.g.,
indexing and slicing operations.  They are two examples of {\em
sequence} data types.  Since Python is an evolving language, other
sequence data types may be added.  There is also another standard
sequence data type: the {\em tuple}.

A tuple consists of a number of values separated by commas, for
instance:

\bcode\begin{verbatim}
>>> t = 12345, 54321, 'hello!'
>>> t[0]
12345
>>> t
(12345, 54321, 'hello!')
>>> # Tuples may be nested:
... u = t, (1, 2, 3, 4, 5)
>>> u
((12345, 54321, 'hello!'), (1, 2, 3, 4, 5))
>>>
\end{verbatim}\ecode
%
As you see, on output tuples are alway enclosed in parentheses, so
that nested tuples are interpreted correctly; they may be input with
or without surrounding parentheses, although often parentheses are
necessary anyway (if the tuple is part of a larger expression).

Tuples have many uses, e.g., (x, y) coordinate pairs, employee records
from a database, etc.  Tuples, like strings, are immutable: it is not
possible to assign to the individual items of a tuple (you can
simulate much of the same effect with slicing and concatenation,
though).

A special problem is the construction of tuples containing 0 or 1
items: the syntax has some extra quirks to accommodate these.  Empty
tuples are constructed by an empty pair of parentheses; a tuple with
one item is constructed by following a value with a comma
(it is not sufficient to enclose a single value in parentheses).
Ugly, but effective.  For example:

\bcode\begin{verbatim}
>>> empty = ()
>>> singleton = 'hello',    # <-- note trailing comma
>>> len(empty)
0
>>> len(singleton)
1
>>> singleton
('hello',)
>>>
\end{verbatim}\ecode
%
The statement {\tt t = 12345, 54321, 'hello!'} is an example of {\em
tuple packing}: the values {\tt 12345}, {\tt 54321} and {\tt 'hello!'}
are packed together in a tuple.  The reverse operation is also
possible, e.g.:

\bcode\begin{verbatim}
>>> x, y, z = t
>>>
\end{verbatim}\ecode
%
This is called, appropriately enough, {\em tuple unpacking}.  Tuple
unpacking requires that the list of variables on the left has the same
number of elements as the length of the tuple.  Note that multiple
assignment is really just a combination of tuple packing and tuple
unpacking!

Occasionally, the corresponding operation on lists is useful: {\em list
unpacking}.  This is supported by enclosing the list of variables in
square brackets:

\bcode\begin{verbatim}
>>> a = ['spam', 'eggs', 100, 1234]
>>> [a1, a2, a3, a4] = a
>>>
\end{verbatim}\ecode

\section{Dictionaries}

Another useful data type built into Python is the {\em dictionary}.
Dictionaries are sometimes found in other languages as ``associative
memories'' or ``associative arrays''.  Unlike sequences, which are
indexed by a range of numbers, dictionaries are indexed by {\em keys},
which can be any non-mutable type; strings and numbers can always be
keys.  Tuples can be used as keys if they contain only strings,
numbers, or tuples.  You can't use lists as keys, since lists can be
modified in place using their \code{append()} method.

It is best to think of a dictionary as an unordered set of
{\em key:value} pairs, with the requirement that the keys are unique
(within one dictionary).
A pair of braces creates an empty dictionary: \verb/{}/.
Placing a comma-separated list of key:value pairs within the
braces adds initial key:value pairs to the dictionary; this is also the
way dictionaries are written on output.

The main operations on a dictionary are storing a value with some key
and extracting the value given the key.  It is also possible to delete
a key:value pair
with {\tt del}.
If you store using a key that is already in use, the old value
associated with that key is forgotten.  It is an error to extract a
value using a non-existent key.

The {\tt keys()} method of a dictionary object returns a list of all the
keys used in the dictionary, in random order (if you want it sorted,
just apply the {\tt sort()} method to the list of keys).  To check
whether a single key is in the dictionary, use the \verb/has_key()/
method of the dictionary.

Here is a small example using a dictionary:

\bcode\begin{verbatim}
>>> tel = {'jack': 4098, 'sape': 4139}
>>> tel['guido'] = 4127
>>> tel
{'sape': 4139, 'guido': 4127, 'jack': 4098}
>>> tel['jack']
4098
>>> del tel['sape']
>>> tel['irv'] = 4127
>>> tel
{'guido': 4127, 'irv': 4127, 'jack': 4098}
>>> tel.keys()
['guido', 'irv', 'jack']
>>> tel.has_key('guido')
1
>>> 
\end{verbatim}\ecode

\section{More on Conditions}

The conditions used in {\tt while} and {\tt if} statements above can
contain other operators besides comparisons.

The comparison operators {\tt in} and {\tt not in} check whether a value
occurs (does not occur) in a sequence.  The operators {\tt is} and {\tt
is not} compare whether two objects are really the same object; this
only matters for mutable objects like lists.  All comparison operators
have the same priority, which is lower than that of all numerical
operators.

Comparisons can be chained: e.g., {\tt a < b == c} tests whether {\tt a}
is less than {\tt b} and moreover {\tt b} equals {\tt c}.

Comparisons may be combined by the Boolean operators {\tt and} and {\tt
or}, and the outcome of a comparison (or of any other Boolean
expression) may be negated with {\tt not}.  These all have lower
priorities than comparison operators again; between them, {\tt not} has
the highest priority, and {\tt or} the lowest, so that
{\tt A and not B or C} is equivalent to {\tt (A and (not B)) or C}.  Of
course, parentheses can be used to express the desired composition.

The Boolean operators {\tt and} and {\tt or} are so-called {\em
shortcut} operators: their arguments are evaluated from left to right,
and evaluation stops as soon as the outcome is determined.  E.g., if
{\tt A} and {\tt C} are true but {\tt B} is false, {\tt A and B and C}
does not evaluate the expression C.  In general, the return value of a
shortcut operator, when used as a general value and not as a Boolean, is
the last evaluated argument.

It is possible to assign the result of a comparison or other Boolean
expression to a variable.  For example,

\bcode\begin{verbatim}
>>> string1, string2, string3 = '', 'Trondheim', 'Hammer Dance'
>>> non_null = string1 or string2 or string3
>>> non_null
'Trondheim'
>>> 
\end{verbatim}\ecode
%
Note that in Python, unlike C, assignment cannot occur inside expressions.

\section{Comparing Sequences and Other Types}

Sequence objects may be compared to other objects with the same
sequence type.  The comparison uses {\em lexicographical} ordering:
first the first two items are compared, and if they differ this
determines the outcome of the comparison; if they are equal, the next
two items are compared, and so on, until either sequence is exhausted.
If two items to be compared are themselves sequences of the same type,
the lexicographical comparison is carried out recursively.  If all
items of two sequences compare equal, the sequences are considered
equal.  If one sequence is an initial subsequence of the other, the
shorted sequence is the smaller one.  Lexicographical ordering for
strings uses the \ASCII{} ordering for individual characters.  Some
examples of comparisons between sequences with the same types:

\bcode\begin{verbatim}
(1, 2, 3)              < (1, 2, 4)
[1, 2, 3]              < [1, 2, 4]
'ABC' < 'C' < 'Pascal' < 'Python'
(1, 2, 3, 4)           < (1, 2, 4)
(1, 2)                 < (1, 2, -1)
(1, 2, 3)              = (1.0, 2.0, 3.0)
(1, 2, ('aa', 'ab'))   < (1, 2, ('abc', 'a'), 4)
\end{verbatim}\ecode
%
Note that comparing objects of different types is legal.  The outcome
is deterministic but arbitrary: the types are ordered by their name.
Thus, a list is always smaller than a string, a string is always
smaller than a tuple, etc.  Mixed numeric types are compared according
to their numeric value, so 0 equals 0.0, etc.%
\footnote{
        The rules for comparing objects of different types should
        not be relied upon; they may change in a future version of
        the language.
}


\chapter{Modules}

If you quit from the Python interpreter and enter it again, the
definitions you have made (functions and variables) are lost.
Therefore, if you want to write a somewhat longer program, you are
better off using a text editor to prepare the input for the interpreter
and running it with that file as input instead.  This is known as creating a
{\em script}.  As your program gets longer, you may want to split it
into several files for easier maintenance.  You may also want to use a
handy function that you've written in several programs without copying
its definition into each program.

To support this, Python has a way to put definitions in a file and use
them in a script or in an interactive instance of the interpreter.
Such a file is called a {\em module}; definitions from a module can be
{\em imported} into other modules or into the {\em main} module (the
collection of variables that you have access to in a script
executed at the top level
and in calculator mode).

A module is a file containing Python definitions and statements.  The
file name is the module name with the suffix {\tt .py} appended.  Within
a module, the module's name (as a string) is available as the value of
the global variable {\tt __name__}.  For instance, use your favorite text
editor to create a file called {\tt fibo.py} in the current directory
with the following contents:

\bcode\begin{verbatim}
# Fibonacci numbers module

def fib(n):    # write Fibonacci series up to n
    a, b = 0, 1
    while b < n:
          print b,
          a, b = b, a+b

def fib2(n): # return Fibonacci series up to n
    result = []
    a, b = 0, 1
    while b < n:
          result.append(b)
          a, b = b, a+b
    return result
\end{verbatim}\ecode
%
Now enter the Python interpreter and import this module with the
following command:

\bcode\begin{verbatim}
>>> import fibo
>>> 
\end{verbatim}\ecode
%
This does not enter the names of the functions defined in
{\tt fibo}
directly in the current symbol table; it only enters the module name
{\tt fibo}
there.
Using the module name you can access the functions:

\bcode\begin{verbatim}
>>> fibo.fib(1000)
1 1 2 3 5 8 13 21 34 55 89 144 233 377 610 987
>>> fibo.fib2(100)
[1, 1, 2, 3, 5, 8, 13, 21, 34, 55, 89]
>>> fibo.__name__
'fibo'
>>> 
\end{verbatim}\ecode
%
If you intend to use a function often you can assign it to a local name:

\bcode\begin{verbatim}
>>> fib = fibo.fib
>>> fib(500)
1 1 2 3 5 8 13 21 34 55 89 144 233 377
>>> 
\end{verbatim}\ecode


\section{More on Modules}

A module can contain executable statements as well as function
definitions.
These statements are intended to initialize the module.
They are executed only the
{\em first}
time the module is imported somewhere.%
\footnote{
        In fact function definitions are also `statements' that are
        `executed'; the execution enters the function name in the
        module's global symbol table.
}

Each module has its own private symbol table, which is used as the
global symbol table by all functions defined in the module.
Thus, the author of a module can use global variables in the module
without worrying about accidental clashes with a user's global
variables.
On the other hand, if you know what you are doing you can touch a
module's global variables with the same notation used to refer to its
functions,
{\tt modname.itemname}.

Modules can import other modules.
It is customary but not required to place all
{\tt import}
statements at the beginning of a module (or script, for that matter).
The imported module names are placed in the importing module's global
symbol table.

There is a variant of the
{\tt import}
statement that imports names from a module directly into the importing
module's symbol table.
For example:

\bcode\begin{verbatim}
>>> from fibo import fib, fib2
>>> fib(500)
1 1 2 3 5 8 13 21 34 55 89 144 233 377
>>> 
\end{verbatim}\ecode
%
This does not introduce the module name from which the imports are taken
in the local symbol table (so in the example, {\tt fibo} is not
defined).

There is even a variant to import all names that a module defines:

\bcode\begin{verbatim}
>>> from fibo import *
>>> fib(500)
1 1 2 3 5 8 13 21 34 55 89 144 233 377
>>> 
\end{verbatim}\ecode
%
This imports all names except those beginning with an underscore
({\tt _}).

\subsection{The Module Search Path}

When a module named {\tt spam} is imported, the interpreter searches
for a file named {\tt spam.py} in the current directory,
and then in the list of directories specified by
the environment variable {\tt PYTHONPATH}.  This has the same syntax as
the \UNIX{} shell variable {\tt PATH}, i.e., a list of colon-separated
directory names.  When {\tt PYTHONPATH} is not set, or when the file
is not found there, the search continues in an installation-dependent
default path, usually {\tt .:/usr/local/lib/python}.

Actually, modules are searched in the list of directories given by the 
variable {\tt sys.path} which is initialized from the directory 
containing the input script (or the current directory), {\tt 
PYTHONPATH} and the installation-dependent default.  This allows 
Python programs that know what they're doing to modify or replace the 
module search path.  See the section on Standard Modules later.

\subsection{``Compiled'' Python files}

As an important speed-up of the start-up time for short programs that
use a lot of standard modules, if a file called {\tt spam.pyc} exists
in the directory where {\tt spam.py} is found, this is assumed to
contain an already-``compiled'' version of the module {\tt spam}.  The
modification time of the version of {\tt spam.py} used to create {\tt
spam.pyc} is recorded in {\tt spam.pyc}, and the file is ignored if
these don't match.

Normally, you don't need to do anything to create the {\tt spam.pyc} file.
Whenever {\tt spam.py} is successfully compiled, an attempt is made to
write the compiled version to {\tt spam.pyc}.  It is not an error if
this attempt fails; if for any reason the file is not written
completely, the resulting {\tt spam.pyc} file will be recognized as
invalid and thus ignored later.  The contents of the {\tt spam.pyc}
file is platform independent, so a Python module directory can be
shared by machines of different architectures.  (Tip for experts:
the module {\tt compileall} creates {\tt .pyc} files for all modules.)

XXX Should optimization with -O be covered here?

\section{Standard Modules}

Python comes with a library of standard modules, described in a separate
document (Python Library Reference).  Some modules are built into the
interpreter; these provide access to operations that are not part of the
core of the language but are nevertheless built in, either for
efficiency or to provide access to operating system primitives such as
system calls.  The set of such modules is a configuration option; e.g.,
the {\tt amoeba} module is only provided on systems that somehow support
Amoeba primitives.  One particular module deserves some attention: {\tt
sys}, which is built into every Python interpreter.  The variables {\tt
sys.ps1} and {\tt sys.ps2} define the strings used as primary and
secondary prompts:

\bcode\begin{verbatim}
>>> import sys
>>> sys.ps1
'>>> '
>>> sys.ps2
'... '
>>> sys.ps1 = 'C> '
C> print 'Yuck!'
Yuck!
C> 
\end{verbatim}\ecode
%
These two variables are only defined if the interpreter is in
interactive mode.

The variable
{\tt sys.path}
is a list of strings that determine the interpreter's search path for
modules.
It is initialized to a default path taken from the environment variable
{\tt PYTHONPATH},
or from a built-in default if
{\tt PYTHONPATH}
is not set.
You can modify it using standard list operations, e.g.:

\bcode\begin{verbatim}
>>> import sys
>>> sys.path.append('/ufs/guido/lib/python')
>>> 
\end{verbatim}\ecode

\section{The {\tt dir()} function}

The built-in function {\tt dir} is used to find out which names a module
defines.  It returns a sorted list of strings:

\bcode\begin{verbatim}
>>> import fibo, sys
>>> dir(fibo)
['__name__', 'fib', 'fib2']
>>> dir(sys)
['__name__', 'argv', 'builtin_module_names', 'copyright', 'exit',
'maxint', 'modules', 'path', 'ps1', 'ps2', 'setprofile', 'settrace',
'stderr', 'stdin', 'stdout', 'version']
>>>
\end{verbatim}\ecode
%
Without arguments, {\tt dir()} lists the names you have defined currently:

\bcode\begin{verbatim}
>>> a = [1, 2, 3, 4, 5]
>>> import fibo, sys
>>> fib = fibo.fib
>>> dir()
['__name__', 'a', 'fib', 'fibo', 'sys']
>>>
\end{verbatim}\ecode
%
Note that it lists all types of names: variables, modules, functions, etc.

{\tt dir()} does not list the names of built-in functions and variables.
If you want a list of those, they are defined in the standard module
{\tt __builtin__}:

\bcode\begin{verbatim}
>>> import __builtin__
>>> dir(__builtin__)
['AccessError', 'AttributeError', 'ConflictError', 'EOFError', 'IOError',
'ImportError', 'IndexError', 'KeyError', 'KeyboardInterrupt',
'MemoryError', 'NameError', 'None', 'OverflowError', 'RuntimeError',
'SyntaxError', 'SystemError', 'SystemExit', 'TypeError', 'ValueError',
'ZeroDivisionError', '__name__', 'abs', 'apply', 'chr', 'cmp', 'coerce',
'compile', 'dir', 'divmod', 'eval', 'execfile', 'filter', 'float',
'getattr', 'hasattr', 'hash', 'hex', 'id', 'input', 'int', 'len', 'long',
'map', 'max', 'min', 'oct', 'open', 'ord', 'pow', 'range', 'raw_input',
'reduce', 'reload', 'repr', 'round', 'setattr', 'str', 'type', 'xrange']
>>>
\end{verbatim}\ecode


\chapter{Input and Output}

There are several ways to present the output of a program; data can be
printed in a human-readable form, or written to a file for future use.
This chapter will discuss some of the possibilities.

\section{Fancier Output Formatting}
So far we've encountered two ways of writing values: {\em expression
statements} and the {\tt print} statement.  (A third way is using the
{\tt write} method of file objects; the standard output file can be
referenced as {\tt sys.stdout}.  See the Library Reference for more
information on this.)

Often you'll want more control over the formatting of your output than
simply printing space-separated values.  There are two ways to format
your output; the first way is to do all the string handling yourself;
using string slicing and concatenation operations you can create any
lay-out you can imagine.  The standard module {\tt string} contains
some useful operations for padding strings to a given column width;
these will be discussed shortly.  The second way is to use the
\code{\%} operator with a string as the left argument.  \code{\%}
interprets the left argument as a \C\ \code{sprintf()}-style format
string to be applied to the right argument, and returns the string
resulting from this formatting operation.

One question remains, of course: how do you convert values to strings?
Luckily, Python has a way to convert any value to a string: pass it to
the \verb/repr()/ function, or just write the value between reverse
quotes (\verb/``/).  Some examples:

\bcode\begin{verbatim}
>>> x = 10 * 3.14
>>> y = 200*200
>>> s = 'The value of x is ' + `x` + ', and y is ' + `y` + '...'
>>> print s
The value of x is 31.4, and y is 40000...
>>> # Reverse quotes work on other types besides numbers:
... p = [x, y]
>>> ps = repr(p)
>>> ps
'[31.4, 40000]'
>>> # Converting a string adds string quotes and backslashes:
... hello = 'hello, world\n'
>>> hellos = `hello`
>>> print hellos
'hello, world\012'
>>> # The argument of reverse quotes may be a tuple:
... `x, y, ('spam', 'eggs')`
"(31.4, 40000, ('spam', 'eggs'))"
>>>
\end{verbatim}\ecode
%
Here are two ways to write a table of squares and cubes:

\bcode\begin{verbatim}
>>> import string
>>> for x in range(1, 11):
...     print string.rjust(`x`, 2), string.rjust(`x*x`, 3),
...     # Note trailing comma on previous line
...     print string.rjust(`x*x*x`, 4)
...
 1   1    1
 2   4    8
 3   9   27
 4  16   64
 5  25  125
 6  36  216
 7  49  343
 8  64  512
 9  81  729
10 100 1000
>>> for x in range(1,11):
...     print '%2d %3d %4d' % (x, x*x, x*x*x)
... 
 1   1    1
 2   4    8
 3   9   27
 4  16   64
 5  25  125
 6  36  216
 7  49  343
 8  64  512
 9  81  729
10 100 1000
>>>
\end{verbatim}\ecode
%
(Note that one space between each column was added by the way {\tt print}
works: it always adds spaces between its arguments.)

This example demonstrates the function {\tt string.rjust()}, which
right-justifies a string in a field of a given width by padding it with
spaces on the left.  There are similar functions {\tt string.ljust()}
and {\tt string.center()}.  These functions do not write anything, they
just return a new string.  If the input string is too long, they don't
truncate it, but return it unchanged; this will mess up your column
lay-out but that's usually better than the alternative, which would be
lying about a value.  (If you really want truncation you can always add
a slice operation, as in {\tt string.ljust(x,~n)[0:n]}.)

There is another function, {\tt string.zfill}, which pads a numeric
string on the left with zeros.  It understands about plus and minus
signs:

\bcode\begin{verbatim}
>>> string.zfill('12', 5)
'00012'
>>> string.zfill('-3.14', 7)
'-003.14'
>>> string.zfill('3.14159265359', 5)
'3.14159265359'
>>>
\end{verbatim}\ecode
%
Using the \code{\%} operator looks like this:

\begin{verbatim}
        >>> import math
        >>> print 'The value of PI is approximately %5.3f.' % math.pi
        The value of PI is approximately 3.142.
        >>> 
\end{verbatim}

If there is more than one format in the string you pass a tuple as
right operand, e.g.

\begin{verbatim}
        >>> table = {'Sjoerd': 4127, 'Jack': 4098, 'Dcab': 8637678}
        >>> for name, phone in table.items():
        ...     print '%-10s ==> %10d' % (name, phone)
        ... 
        Jack       ==>       4098
        Dcab       ==>    8637678
        Sjoerd     ==>       4127
        >>> 
\end{verbatim}

Most formats work exactly as in C and require that you pass the proper
type; however, if you don't you get an exception, not a core dump.
The \verb\%s\ format is more relaxed: if the corresponding argument is
not a string object, it is converted to string using the \verb\str()\
built-in function.  Using \verb\*\ to pass the width or precision in
as a separate (integer) argument is supported.  The C formats
\verb\%n\ and \verb\%p\ are not supported.

If you have a really long format string that you don't want to split
up, it would be nice if you could reference the variables to be
formatted by name instead of by position.  This can be done by using
an extension of C formats using the form \verb\%(name)format\, e.g.

\begin{verbatim}
        >>> table = {'Sjoerd': 4127, 'Jack': 4098, 'Dcab': 8637678}
        >>> print 'Jack: %(Jack)d; Sjoerd: %(Sjoerd)d; Dcab: %(Dcab)d' % table
        Jack: 4098; Sjoerd: 4127; Dcab: 8637678
        >>> 
\end{verbatim}

This is particularly useful in combination with the new built-in
\verb\vars()\ function, which returns a dictionary containing all
local variables.

\section{Reading and Writing Files}
% Opening files 
\code{open()} returns a file object, and is most commonly used with
two arguments: \code{open(\var{filename},\var{mode})}.  

\bcode\begin{verbatim}
>>> f=open('/tmp/workfile', 'w')
>>> print f
<open file '/tmp/workfile', mode 'w' at 80a0960>
\end{verbatim}\ecode
%
The first argument is a string containing the filename.  The second
argument is another string containing a few characters describing the
way in which the file will be used.  \var{mode} can be \code{'r'} when
the file will only be read, \code{'w'} for only writing (an existing
file with the same name will be erased), and \code{'a'} opens the file
for appending; any data written to the file is automatically added to
the end.  \code{'r+'} opens the file for both reading and writing.
The \var{mode} argument is optional; \code{'r'} will be assumed if
it's omitted.

On Windows, (XXX does the Mac need this too?) \code{'b'} appended to the
mode opens the file in binary mode, so there are also modes like
\code{'rb'}, \code{'wb'}, and \code{'r+b'}.  Windows makes a
distinction between text and binary files; the end-of-line characters
in text files are automatically altered slightly when data is read or
written.  This behind-the-scenes modification to file data is fine for
ASCII text files, but it'll corrupt binary data like that in JPEGs or
.EXE files.  Be very careful to use binary mode when reading and
writing such files.

\subsection{Methods of file objects}

The rest of the examples in this section will assume that a file
object called \code{f} has already been created.

To read a file's contents, call \code{f.read(\var{size})}, which reads
some quantity of data and returns it as a string.  \var{size} is an
optional numeric argument.  When \var{size} is omitted or negative,
the entire contents of the file will be read and returned; it's your
problem if the file is twice as large as your machine's memory.
Otherwise, at most \var{size} bytes are read and returned.  If the end
of the file has been reached, \code{f.read()} will return an empty
string (\code {""}).
\bcode\begin{verbatim}
>>> f.read()
'This is the entire file.\012'
>>> f.read()
''
\end{verbatim}\ecode
%
\code{f.readline()} reads a single line from the file; a newline
character (\verb/\n/) is left at the end of the string, and is only
omitted on the last line of the file if the file doesn't end in a
newline.  This makes the return value unambiguous; if
\code{f.readline()} returns an empty string, the end of the file has
been reached, while a blank line is represented by \verb/'\n'/, a
string containing only a single newline.  

\bcode\begin{verbatim}
>>> f.readline()
'This is the first line of the file.\012'
>>> f.readline()
'Second line of the file\012'
>>> f.readline()
''
\end{verbatim}\ecode
%
\code{f.readlines()} uses {\code{f.readline()} repeatedly, and returns
a list containing all the lines of data in the file.

\bcode\begin{verbatim}
>>> f.readlines()
['This is the first line of the file.\012', 'Second line of the file\012']
\end{verbatim}\ecode
%
\code{f.write(\var{string})} writes the contents of \var{string} to
the file, returning \code{None}.  

\bcode\begin{verbatim}
>>> f.write('This is a test\n')
\end{verbatim}\ecode
%
\code{f.tell()} returns an integer giving the file object's current
position in the file, measured in bytes from the beginning of the
file.  To change the file object's position, use
\code{f.seek(\var{offset}, \var{from_what})}.  The position is
computed from adding \var{offset} to a reference point; the reference
point is selected by the \var{from_what} argument.  A \var{from_what}
value of 0 measures from the beginning of the file, 1 uses the current
file position, and 2 uses the end of the file as the reference point.
\var{from_what}
can be omitted and defaults to 0, using the beginning of the file as the reference point.

\bcode\begin{verbatim}
>>> f=open('/tmp/workfile', 'r+')
>>> f.write('0123456789abcdef')
>>> f.seek(5)     # Go to the 5th byte in the file
>>> f.read(1)        
'5'
>>> f.seek(-3, 2) # Go to the 3rd byte before the end
>>> f.read(1)
'd'
\end{verbatim}\ecode
%
When you're done with a file, call \code{f.close()} to close it and
free up any system resources taken up by the open file.  After calling
\code{f.close()}, attempts to use the file object will automatically fail.

\bcode\begin{verbatim}
>>> f.close()
>>> f.read()
Traceback (innermost last):
  File "<stdin>", line 1, in ?
ValueError: I/O operation on closed file
\end{verbatim}\ecode
%
File objects have some additional methods, such as \code{isatty()} and
\code{truncate()} which are less frequently used; consult the Library
Reference for a complete guide to file objects.

\subsection{The pickle module}

Strings can easily be written to and read from a file. Numbers take a
bit more effort, since the \code{read()} method only returns strings,
which will have to be passed to a function like \code{string.atoi()},
which takes a string like \code{'123'} and returns its numeric value
123.  However, when you want to save more complex data types like
lists, dictionaries, or class instances, things get a lot more
complicated.  

Rather than have users be constantly writing and debugging code to
save complicated data types, Python provides a standard module called
\code{pickle}.  code{pickle} is an amazing module that can take almost
any Python object (even some forms of Python code!), and convert it to
a string representation; this process is called \dfn{pickling}.  
Reconstructing the object from the string representation is called
\dfn{unpickling}.  Between pickling and unpickling, the string
representing the object may have been stored in a file or data, or
sent over a network connection to some distant machine.

If you have an object \code{x}, and a file object \code{f} that's been
opened for writing, the simplest way to pickle the object takes only
one line of code:

\bcode\begin{verbatim}
pickle.dump(x, f)
\end{verbatim}\ecode
%
To unpickle the object again, if \code{f} is a file object which has been
opened for reading:

\bcode\begin{verbatim}
x = pickle.load(f)
\end{verbatim}\ecode
%
(There are other variants of this, used when pickling many objects or
when you don't want to write the pickled data to a file; consult the
complete documentation for \code{pickle} in the Library Reference.)

\code{pickle} is the standard way to make Python objects which can be
stored and reused by other programs or by a future invocation of the
same program; the technical term for this is a \dfn{persistent}
object.  Because \code{pickle} is so widely used, many authors who
write Python extensions take care to ensure that new data types such
as matrices, XXX more examples needed XXX, can be properly pickled and
unpickled.



\chapter{Errors and Exceptions}

Until now error messages haven't been more than mentioned, but if you
have tried out the examples you have probably seen some.  There are
(at least) two distinguishable kinds of errors: {\em syntax\ errors}
and {\em exceptions}.

\section{Syntax Errors}

Syntax errors, also known as parsing errors, are perhaps the most common
kind of complaint you get while you are still learning Python:

\bcode\begin{verbatim}
>>> while 1 print 'Hello world'
  File "<stdin>", line 1
    while 1 print 'Hello world'
                ^
SyntaxError: invalid syntax
>>> 
\end{verbatim}\ecode
%
The parser repeats the offending line and displays a little `arrow'
pointing at the earliest point in the line where the error was detected.
The error is caused by (or at least detected at) the token
{\em preceding}
the arrow: in the example, the error is detected at the keyword
{\tt print}, since a colon ({\tt :}) is missing before it.
File name and line number are printed so you know where to look in case
the input came from a script.

\section{Exceptions}

Even if a statement or expression is syntactically correct, it may
cause an error when an attempt is made to execute it.
Errors detected during execution are called {\em exceptions} and are
not unconditionally fatal: you will soon learn how to handle them in
Python programs.  Most exceptions are not handled by programs,
however, and result in error messages as shown here:

\bcode\small\begin{verbatim}
>>> 10 * (1/0)
Traceback (innermost last):
  File "<stdin>", line 1
ZeroDivisionError: integer division or modulo
>>> 4 + spam*3
Traceback (innermost last):
  File "<stdin>", line 1
NameError: spam
>>> '2' + 2
Traceback (innermost last):
  File "<stdin>", line 1
TypeError: illegal argument type for built-in operation
>>> 
\end{verbatim}\normalsize\ecode
%
The last line of the error message indicates what happened.
Exceptions come in different types, and the type is printed as part of
the message: the types in the example are
{\tt ZeroDivisionError},
{\tt NameError}
and
{\tt TypeError}.
The string printed as the exception type is the name of the built-in
name for the exception that occurred.  This is true for all built-in
exceptions, but need not be true for user-defined exceptions (although
it is a useful convention).
Standard exception names are built-in identifiers (not reserved
keywords).

The rest of the line is a detail whose interpretation depends on the
exception type; its meaning is dependent on the exception type.

The preceding part of the error message shows the context where the
exception happened, in the form of a stack backtrace.
In general it contains a stack backtrace listing source lines; however,
it will not display lines read from standard input.

The Python Library Reference Manual lists the built-in exceptions and
their meanings.

\section{Handling Exceptions}

It is possible to write programs that handle selected exceptions.
Look at the following example, which prints a table of inverses of
some floating point numbers:

\bcode\begin{verbatim}
>>> numbers = [0.3333, 2.5, 0, 10]
>>> for x in numbers:
...     print x,
...     try:
...         print 1.0 / x
...     except ZeroDivisionError:
...         print '*** has no inverse ***'
...     
0.3333 3.00030003
2.5 0.4
0 *** has no inverse ***
10 0.1
>>> 
\end{verbatim}\ecode
%
The {\tt try} statement works as follows.
\begin{itemize}
\item
First, the
{\em try\ clause}
(the statement(s) between the {\tt try} and {\tt except} keywords) is
executed.
\item
If no exception occurs, the
{\em except\ clause}
is skipped and execution of the {\tt try} statement is finished.
\item
If an exception occurs during execution of the try clause,
the rest of the clause is skipped.  Then if
its type matches the exception named after the {\tt except} keyword,
the rest of the try clause is skipped, the except clause is executed,
and then execution continues after the {\tt try} statement.
\item
If an exception occurs which does not match the exception named in the
except clause, it is passed on to outer try statements; if no handler is
found, it is an
{\em unhandled\ exception}
and execution stops with a message as shown above.
\end{itemize}
A {\tt try} statement may have more than one except clause, to specify
handlers for different exceptions.
At most one handler will be executed.
Handlers only handle exceptions that occur in the corresponding try
clause, not in other handlers of the same {\tt try} statement.
An except clause may name multiple exceptions as a parenthesized list,
e.g.:

\bcode\begin{verbatim}
... except (RuntimeError, TypeError, NameError):
...     pass
\end{verbatim}\ecode
%
The last except clause may omit the exception name(s), to serve as a
wildcard.
Use this with extreme caution, since it is easy to mask a real
programming error in this way!

The \verb\try...except\ statement has an optional \verb\else\ clause,
which must follow all \verb\except\ clauses.  It is useful to place
code that must be executed if the \verb\try\ clause does not raise an
exception.  For example:

\begin{verbatim}
        for arg in sys.argv:
                try:
                        f = open(arg, 'r')
                except IOError:
                        print 'cannot open', arg
                else:
                        print arg, 'has', len(f.readlines()), 'lines'
                        f.close()
\end{verbatim}


When an exception occurs, it may have an associated value, also known as
the exceptions's
{\em argument}.
The presence and type of the argument depend on the exception type.
For exception types which have an argument, the except clause may
specify a variable after the exception name (or list) to receive the
argument's value, as follows:

\bcode\begin{verbatim}
>>> try:
...     spam()
... except NameError, x:
...     print 'name', x, 'undefined'
... 
name spam undefined
>>> 
\end{verbatim}\ecode
%
If an exception has an argument, it is printed as the last part
(`detail') of the message for unhandled exceptions.

Exception handlers don't just handle exceptions if they occur
immediately in the try clause, but also if they occur inside functions
that are called (even indirectly) in the try clause.
For example:

\bcode\begin{verbatim}
>>> def this_fails():
...     x = 1/0
... 
>>> try:
...     this_fails()
... except ZeroDivisionError, detail:
...     print 'Handling run-time error:', detail
... 
Handling run-time error: integer division or modulo
>>>
\end{verbatim}\ecode
%

\section{Raising Exceptions}

The {\tt raise} statement allows the programmer to force a specified
exception to occur.
For example:

\bcode\begin{verbatim}
>>> raise NameError, 'HiThere'
Traceback (innermost last):
  File "<stdin>", line 1
NameError: HiThere
>>> 
\end{verbatim}\ecode
%
The first argument to {\tt raise} names the exception to be raised.
The optional second argument specifies the exception's argument.

%

\section{User-defined Exceptions}

Programs may name their own exceptions by assigning a string to a
variable.
For example:

\bcode\begin{verbatim}
>>> my_exc = 'my_exc'
>>> try:
...     raise my_exc, 2*2
... except my_exc, val:
...     print 'My exception occurred, value:', val
... 
My exception occurred, value: 4
>>> raise my_exc, 1
Traceback (innermost last):
  File "<stdin>", line 1
my_exc: 1
>>> 
\end{verbatim}\ecode
%
Many standard modules use this to report errors that may occur in
functions they define.

%

\section{Defining Clean-up Actions}

The {\tt try} statement has another optional clause which is intended to
define clean-up actions that must be executed under all circumstances.
For example:

\bcode\begin{verbatim}
>>> try:
...     raise KeyboardInterrupt
... finally:
...     print 'Goodbye, world!'
... 
Goodbye, world!
Traceback (innermost last):
  File "<stdin>", line 2
KeyboardInterrupt
>>> 
\end{verbatim}\ecode
%
A {\tt finally} clause is executed whether or not an exception has
occurred in the {\tt try} clause.  When an exception has occurred, it
is re-raised after the {\tt finally} clause is executed.  The
{\tt finally} clause is also executed ``on the way out'' when the
{\tt try} statement is left via a {\tt break} or {\tt return}
statement.

A {\tt try} statement must either have one or more {\tt except}
clauses or one {\tt finally} clause, but not both.

\chapter{Classes}

Python's class mechanism adds classes to the language with a minimum
of new syntax and semantics.  It is a mixture of the class mechanisms
found in \Cpp{} and Modula-3.  As is true for modules, classes in Python
do not put an absolute barrier between definition and user, but rather
rely on the politeness of the user not to ``break into the
definition.''  The most important features of classes are retained
with full power, however: the class inheritance mechanism allows
multiple base classes, a derived class can override any methods of its
base class(es), a method can call the method of a base class with the
same name.  Objects can contain an arbitrary amount of private data.

In \Cpp{} terminology, all class members (including the data members) are
{\em public}, and all member functions are {\em virtual}.  There are
no special constructors or destructors.  As in Modula-3, there are no
shorthands for referencing the object's members from its methods: the
method function is declared with an explicit first argument
representing the object, which is provided implicitly by the call.  As
in Smalltalk, classes themselves are objects, albeit in the wider
sense of the word: in Python, all data types are objects.  This
provides semantics for importing and renaming.  But, just like in \Cpp{}
or Modula-3, built-in types cannot be used as base classes for
extension by the user.  Also, like in \Cpp{} but unlike in Modula-3, most
built-in operators with special syntax (arithmetic operators,
subscripting etc.) can be redefined for class members.

\section{A word about terminology}

Lacking universally accepted terminology to talk about classes, I'll
make occasional use of Smalltalk and \Cpp{} terms.  (I'd use Modula-3
terms, since its object-oriented semantics are closer to those of
Python than \Cpp{}, but I expect that few readers have heard of it...)

I also have to warn you that there's a terminological pitfall for
object-oriented readers: the word ``object'' in Python does not
necessarily mean a class instance.  Like \Cpp{} and Modula-3, and unlike
Smalltalk, not all types in Python are classes: the basic built-in
types like integers and lists aren't, and even somewhat more exotic
types like files aren't.  However, {\em all} Python types share a little
bit of common semantics that is best described by using the word
object.

Objects have individuality, and multiple names (in multiple scopes)
can be bound to the same object.  This is known as aliasing in other
languages.  This is usually not appreciated on a first glance at
Python, and can be safely ignored when dealing with immutable basic
types (numbers, strings, tuples).  However, aliasing has an
(intended!) effect on the semantics of Python code involving mutable
objects such as lists, dictionaries, and most types representing
entities outside the program (files, windows, etc.).  This is usually
used to the benefit of the program, since aliases behave like pointers
in some respects.  For example, passing an object is cheap since only
a pointer is passed by the implementation; and if a function modifies
an object passed as an argument, the caller will see the change --- this
obviates the need for two different argument passing mechanisms as in
Pascal.


\section{Python scopes and name spaces}

Before introducing classes, I first have to tell you something about
Python's scope rules.  Class definitions play some neat tricks with
name spaces, and you need to know how scopes and name spaces work to
fully understand what's going on.  Incidentally, knowledge about this
subject is useful for any advanced Python programmer.

Let's begin with some definitions.

A {\em name space} is a mapping from names to objects.  Most name
spaces are currently implemented as Python dictionaries, but that's
normally not noticeable in any way (except for performance), and it
may change in the future.  Examples of name spaces are: the set of
built-in names (functions such as \verb\abs()\, and built-in exception
names); the global names in a module; and the local names in a
function invocation.  In a sense the set of attributes of an object
also form a name space.  The important thing to know about name
spaces is that there is absolutely no relation between names in
different name spaces; for instance, two different modules may both
define a function ``maximize'' without confusion --- users of the
modules must prefix it with the module name.

By the way, I use the word {\em attribute} for any name following a
dot --- for example, in the expression \verb\z.real\, \verb\real\ is
an attribute of the object \verb\z\.  Strictly speaking, references to
names in modules are attribute references: in the expression
\verb\modname.funcname\, \verb\modname\ is a module object and
\verb\funcname\ is an attribute of it.  In this case there happens to
be a straightforward mapping between the module's attributes and the
global names defined in the module: they share the same name space!%
\footnote{
        Except for one thing.  Module objects have a secret read-only
        attribute called {\tt __dict__} which returns the dictionary
        used to implement the module's name space; the name
        {\tt __dict__} is an attribute but not a global name.
        Obviously, using this violates the abstraction of name space
        implementation, and should be restricted to things like
        post-mortem debuggers...
}

Attributes may be read-only or writable.  In the latter case,
assignment to attributes is possible.  Module attributes are writable:
you can write \verb\modname.the_answer = 42\.  Writable attributes may
also be deleted with the del statement, e.g.
\verb\del modname.the_answer\.

Name spaces are created at different moments and have different
lifetimes.  The name space containing the built-in names is created
when the Python interpreter starts up, and is never deleted.  The
global name space for a module is created when the module definition
is read in; normally, module name spaces also last until the
interpreter quits.  The statements executed by the top-level
invocation of the interpreter, either read from a script file or
interactively, are considered part of a module called \verb\__main__\,
so they have their own global name space.  (The built-in names
actually also live in a module; this is called \verb\__builtin__\.)

The local name space for a function is created when the function is
called, and deleted when the function returns or raises an exception
that is not handled within the function.  (Actually, forgetting would
be a better way to describe what actually happens.)  Of course,
recursive invocations each have their own local name space.

A {\em scope} is a textual region of a Python program where a name space
is directly accessible.  ``Directly accessible'' here means that an
unqualified reference to a name attempts to find the name in the name
space.

Although scopes are determined statically, they are used dynamically.
At any time during execution, exactly three nested scopes are in use
(i.e., exactly three name spaces are directly accessible): the
innermost scope, which is searched first, contains the local names,
the middle scope, searched next, contains the current module's global
names, and the outermost scope (searched last) is the name space
containing built-in names.

Usually, the local scope references the local names of the (textually)
current function.  Outside of functions, the local scope references
the same name space as the global scope: the module's name space.
Class definitions place yet another name space in the local scope.

It is important to realize that scopes are determined textually: the
global scope of a function defined in a module is that module's name
space, no matter from where or by what alias the function is called.
On the other hand, the actual search for names is done dynamically, at
run time --- however, the language definition is evolving towards
static name resolution, at ``compile'' time, so don't rely on dynamic
name resolution!  (In fact, local variables are already determined
statically.)

A special quirk of Python is that assignments always go into the
innermost scope.  Assignments do not copy data --- they just
bind names to objects.  The same is true for deletions: the statement
\verb\del x\ removes the binding of x from the name space referenced by the
local scope.  In fact, all operations that introduce new names use the
local scope: in particular, import statements and function definitions
bind the module or function name in the local scope.  (The
\verb\global\ statement can be used to indicate that particular
variables live in the global scope.)


\section{A first look at classes}

Classes introduce a little bit of new syntax, three new object types,
and some new semantics.


\subsection{Class definition syntax}

The simplest form of class definition looks like this:

\begin{verbatim}
        class ClassName:
                <statement-1>
                .
                .
                .
                <statement-N>
\end{verbatim}

Class definitions, like function definitions (\verb\def\ statements)
must be executed before they have any effect.  (You could conceivably
place a class definition in a branch of an \verb\if\ statement, or
inside a function.)

In practice, the statements inside a class definition will usually be
function definitions, but other statements are allowed, and sometimes
useful --- we'll come back to this later.  The function definitions
inside a class normally have a peculiar form of argument list,
dictated by the calling conventions for methods --- again, this is
explained later.

When a class definition is entered, a new name space is created, and
used as the local scope --- thus, all assignments to local variables
go into this new name space.  In particular, function definitions bind
the name of the new function here.

When a class definition is left normally (via the end), a {\em class
object} is created.  This is basically a wrapper around the contents
of the name space created by the class definition; we'll learn more
about class objects in the next section.  The original local scope
(the one in effect just before the class definitions was entered) is
reinstated, and the class object is bound here to class name given in
the class definition header (ClassName in the example).


\subsection{Class objects}

Class objects support two kinds of operations: attribute references
and instantiation.

{\em Attribute references} use the standard syntax used for all
attribute references in Python: \verb\obj.name\.  Valid attribute
names are all the names that were in the class's name space when the
class object was created.  So, if the class definition looked like
this:

\begin{verbatim}
        class MyClass:
                "A simple example class"
                i = 12345
                def f(x):
                        return 'hello world'
\end{verbatim}

then \verb\MyClass.i\ and \verb\MyClass.f\ are valid attribute
references, returning an integer and a function object, respectively.
Class attributes can also be assigned to, so you can change the value
of \verb\MyClass.i\ by assignment.  \verb\__doc__\ is also a valid
attribute that's read-only, returning the docstring belonging to
the class: \verb\"A simple example class"\).  

Class {\em instantiation} uses function notation.  Just pretend that
the class object is a parameterless function that returns a new
instance of the class.  For example, (assuming the above class):

\begin{verbatim}
        x = MyClass()
\end{verbatim}

creates a new {\em instance} of the class and assigns this object to
the local variable \verb\x\.


\subsection{Instance objects}

Now what can we do with instance objects?  The only operations
understood by instance objects are attribute references.  There are
two kinds of valid attribute names.

The first I'll call {\em data attributes}.  These correspond to
``instance variables'' in Smalltalk, and to ``data members'' in \Cpp{}.
Data attributes need not be declared; like local variables, they
spring into existence when they are first assigned to.  For example,
if \verb\x\ in the instance of \verb\MyClass\ created above, the
following piece of code will print the value 16, without leaving a
trace:

\begin{verbatim}
        x.counter = 1
        while x.counter < 10:
                x.counter = x.counter * 2
        print x.counter
        del x.counter
\end{verbatim}

The second kind of attribute references understood by instance objects
are {\em methods}.  A method is a function that ``belongs to'' an
object.  (In Python, the term method is not unique to class instances:
other object types can have methods as well, e.g., list objects have
methods called append, insert, remove, sort, and so on.  However,
below, we'll use the term method exclusively to mean methods of class
instance objects, unless explicitly stated otherwise.)

Valid method names of an instance object depend on its class.  By
definition, all attributes of a class that are (user-defined) function
objects define corresponding methods of its instances.  So in our
example, \verb\x.f\ is a valid method reference, since
\verb\MyClass.f\ is a function, but \verb\x.i\ is not, since
\verb\MyClass.i\ is not.  But \verb\x.f\ is not the
same thing as \verb\MyClass.f\ --- it is a {\em method object}, not a
function object.


\subsection{Method objects}

Usually, a method is called immediately, e.g.:

\begin{verbatim}
        x.f()
\end{verbatim}

In our example, this will return the string \verb\'hello world'\.
However, it is not necessary to call a method right away: \verb\x.f\
is a method object, and can be stored away and called at a later
moment, for example:

\begin{verbatim}
        xf = x.f
        while 1:
                print xf()
\end{verbatim}

will continue to print \verb\hello world\ until the end of time.

What exactly happens when a method is called?  You may have noticed
that \verb\x.f()\ was called without an argument above, even though
the function definition for \verb\f\ specified an argument.  What
happened to the argument?  Surely Python raises an exception when a
function that requires an argument is called without any --- even if
the argument isn't actually used...

Actually, you may have guessed the answer: the special thing about
methods is that the object is passed as the first argument of the
function.  In our example, the call \verb\x.f()\ is exactly equivalent
to \verb\MyClass.f(x)\.  In general, calling a method with a list of
{\em n} arguments is equivalent to calling the corresponding function
with an argument list that is created by inserting the method's object
before the first argument.

If you still don't understand how methods work, a look at the
implementation can perhaps clarify matters.  When an instance
attribute is referenced that isn't a data attribute, its class is
searched.  If the name denotes a valid class attribute that is a
function object, a method object is created by packing (pointers to)
the instance object and the function object just found together in an
abstract object: this is the method object.  When the method object is
called with an argument list, it is unpacked again, a new argument
list is constructed from the instance object and the original argument
list, and the function object is called with this new argument list.


\section{Random remarks}


[These should perhaps be placed more carefully...]


Data attributes override method attributes with the same name; to
avoid accidental name conflicts, which may cause hard-to-find bugs in
large programs, it is wise to use some kind of convention that
minimizes the chance of conflicts, e.g., capitalize method names,
prefix data attribute names with a small unique string (perhaps just
an underscore), or use verbs for methods and nouns for data attributes.


Data attributes may be referenced by methods as well as by ordinary
users (``clients'') of an object.  In other words, classes are not
usable to implement pure abstract data types.  In fact, nothing in
Python makes it possible to enforce data hiding --- it is all based
upon convention.  (On the other hand, the Python implementation,
written in C, can completely hide implementation details and control
access to an object if necessary; this can be used by extensions to
Python written in C.)


Clients should use data attributes with care --- clients may mess up
invariants maintained by the methods by stamping on their data
attributes.  Note that clients may add data attributes of their own to
an instance object without affecting the validity of the methods, as
long as name conflicts are avoided --- again, a naming convention can
save a lot of headaches here.


There is no shorthand for referencing data attributes (or other
methods!) from within methods.  I find that this actually increases
the readability of methods: there is no chance of confusing local
variables and instance variables when glancing through a method.


Conventionally, the first argument of methods is often called
\verb\self\.  This is nothing more than a convention: the name
\verb\self\ has absolutely no special meaning to Python.  (Note,
however, that by not following the convention your code may be less
readable by other Python programmers, and it is also conceivable that
a {\em class browser} program be written which relies upon such a
convention.)


Any function object that is a class attribute defines a method for
instances of that class.  It is not necessary that the function
definition is textually enclosed in the class definition: assigning a
function object to a local variable in the class is also ok.  For
example:

\begin{verbatim}
        # Function defined outside the class
        def f1(self, x, y):
                return min(x, x+y)

        class C:
                f = f1
                def g(self):
                        return 'hello world'
                h = g
\end{verbatim}

Now \verb\f\, \verb\g\ and \verb\h\ are all attributes of class
\verb\C\ that refer to function objects, and consequently they are all
methods of instances of \verb\C\ --- \verb\h\ being exactly equivalent
to \verb\g\.  Note that this practice usually only serves to confuse
the reader of a program.


Methods may call other methods by using method attributes of the
\verb\self\ argument, e.g.:

\begin{verbatim}
        class Bag:
                def empty(self):
                        self.data = []
                def add(self, x):
                        self.data.append(x)
                def addtwice(self, x):
                        self.add(x)
                        self.add(x)
\end{verbatim}


The instantiation operation (``calling'' a class object) creates an
empty object.  Many classes like to create objects in a known initial
state.  Therefore a class may define a special method named
\verb\__init__\, like this:

\begin{verbatim}
                def __init__(self):
                        self.empty()
\end{verbatim}

When a class defines an \verb\__init__\ method, class instantiation
automatically invokes \verb\__init__\ for the newly-created class
instance.  So in the \verb\Bag\ example, a new and initialized instance
can be obtained by:

\begin{verbatim}
        x = Bag()
\end{verbatim}

Of course, the \verb\__init__\ method may have arguments for greater
flexibility.  In that case, arguments given to the class instantiation
operator are passed on to \verb\__init__\.  For example,

\bcode\begin{verbatim}
>>> class Complex:
...     def __init__(self, realpart, imagpart):
...         self.r = realpart
...         self.i = imagpart
... 
>>> x = Complex(3.0,-4.5)
>>> x.r, x.i
(3.0, -4.5)
>>> 
\end{verbatim}\ecode
%
Methods may reference global names in the same way as ordinary
functions.  The global scope associated with a method is the module
containing the class definition.  (The class itself is never used as a
global scope!)  While one rarely encounters a good reason for using
global data in a method, there are many legitimate uses of the global
scope: for one thing, functions and modules imported into the global
scope can be used by methods, as well as functions and classes defined
in it.  Usually, the class containing the method is itself defined in
this global scope, and in the next section we'll find some good
reasons why a method would want to reference its own class!


\section{Inheritance}

Of course, a language feature would not be worthy of the name ``class''
without supporting inheritance.  The syntax for a derived class
definition looks as follows:

\begin{verbatim}
        class DerivedClassName(BaseClassName):
                <statement-1>
                .
                .
                .
                <statement-N>
\end{verbatim}

The name \verb\BaseClassName\ must be defined in a scope containing
the derived class definition.  Instead of a base class name, an
expression is also allowed.  This is useful when the base class is
defined in another module, e.g.,

\begin{verbatim}
        class DerivedClassName(modname.BaseClassName):
\end{verbatim}

Execution of a derived class definition proceeds the same as for a
base class.  When the class object is constructed, the base class is
remembered.  This is used for resolving attribute references: if a
requested attribute is not found in the class, it is searched in the
base class.  This rule is applied recursively if the base class itself
is derived from some other class.

There's nothing special about instantiation of derived classes:
\verb\DerivedClassName()\ creates a new instance of the class.  Method
references are resolved as follows: the corresponding class attribute
is searched, descending down the chain of base classes if necessary,
and the method reference is valid if this yields a function object.

Derived classes may override methods of their base classes.  Because
methods have no special privileges when calling other methods of the
same object, a method of a base class that calls another method
defined in the same base class, may in fact end up calling a method of
a derived class that overrides it.  (For \Cpp{} programmers: all methods
in Python are ``virtual functions''.)

An overriding method in a derived class may in fact want to extend
rather than simply replace the base class method of the same name.
There is a simple way to call the base class method directly: just
call \verb\BaseClassName.methodname(self, arguments)\.  This is
occasionally useful to clients as well.  (Note that this only works if
the base class is defined or imported directly in the global scope.)


\subsection{Multiple inheritance}

Python supports a limited form of multiple inheritance as well.  A
class definition with multiple base classes looks as follows:

\begin{verbatim}
        class DerivedClassName(Base1, Base2, Base3):
                <statement-1>
                .
                .
                .
                <statement-N>
\end{verbatim}

The only rule necessary to explain the semantics is the resolution
rule used for class attribute references.  This is depth-first,
left-to-right.  Thus, if an attribute is not found in
\verb\DerivedClassName\, it is searched in \verb\Base1\, then
(recursively) in the base classes of \verb\Base1\, and only if it is
not found there, it is searched in \verb\Base2\, and so on.

(To some people breadth first---searching \verb\Base2\ and
\verb\Base3\ before the base classes of \verb\Base1\---looks more
natural.  However, this would require you to know whether a particular
attribute of \verb\Base1\ is actually defined in \verb\Base1\ or in
one of its base classes before you can figure out the consequences of
a name conflict with an attribute of \verb\Base2\.  The depth-first
rule makes no differences between direct and inherited attributes of
\verb\Base1\.)

It is clear that indiscriminate use of multiple inheritance is a
maintenance nightmare, given the reliance in Python on conventions to
avoid accidental name conflicts.  A well-known problem with multiple
inheritance is a class derived from two classes that happen to have a
common base class.  While it is easy enough to figure out what happens
in this case (the instance will have a single copy of ``instance
variables'' or data attributes used by the common base class), it is
not clear that these semantics are in any way useful.


\section{Private variables through name mangling}

There is now limited support for class-private
identifiers.  Any identifier of the form \code{__spam} (at least two
leading underscores, at most one trailing underscore) is now textually
replaced with \code{_classname__spam}, where \code{classname} is the
current class name with leading underscore(s) stripped.  This mangling
is done without regard of the syntactic position of the identifier, so
it can be used to define class-private instance and class variables,
methods, as well as globals, and even to store instance variables
private to this class on instances of {\em other} classes.  Truncation
may occur when the mangled name would be longer than 255 characters.
Outside classes, or when the class name consists of only underscores,
no mangling occurs.

Name mangling is intended to give classes an easy way to define
``private'' instance variables and methods, without having to worry
about instance variables defined by derived classes, or mucking with
instance variables by code outside the class.  Note that the mangling
rules are designed mostly to avoid accidents; it still is possible for
a determined soul to access or modify a variable that is considered
private.  This can even be useful, e.g. for the debugger, and that's
one reason why this loophole is not closed.  (Buglet: derivation of a
class with the same name as the base class makes use of private
variables of the base class possible.)

Notice that code passed to \code{exec}, \code{eval()} or
\code{evalfile()} does not consider the classname of the invoking 
class to be the current class; this is similar to the effect of the 
\code{global} statement, the effect of which is likewise restricted to 
code that is byte-compiled together.  The same restriction applies to
\code{getattr()}, \code{setattr()} and \code{delattr()}, as well as
when referencing \code{__dict__} directly.

Here's an example of a class that implements its own
\code{__getattr__} and \code{__setattr__} methods and stores all
attributes in a private variable, in a way that works in Python 1.4 as
well as in previous versions:

\begin{verbatim}
class VirtualAttributes:
    __vdict = None
    __vdict_name = locals().keys()[0]
     
    def __init__(self):
        self.__dict__[self.__vdict_name] = {}
    
    def __getattr__(self, name):
        return self.__vdict[name]
    
    def __setattr__(self, name, value):
        self.__vdict[name] = value
\end{verbatim}

%{\em Warning: this is an experimental feature.}  To avoid all
%potential problems, refrain from using identifiers starting with
%double underscore except for predefined uses like \code{__init__}.  To
%use private names while maintaining future compatibility: refrain from
%using the same private name in classes related via subclassing; avoid
%explicit (manual) mangling/unmangling; and assume that at some point
%in the future, leading double underscore will revert to being just a
%naming convention.  Discussion on extensive compile-time declarations
%are currently underway, and it is impossible to predict what solution
%will eventually be chosen for private names.  Double leading
%underscore is still a candidate, of course --- just not the only one.
%It is placed in the distribution in the belief that it is useful, and
%so that widespread experience with its use can be gained.  It will not
%be removed without providing a better solution and a migration path.

\section{Odds and ends}

Sometimes it is useful to have a data type similar to the Pascal
``record'' or C ``struct'', bundling together a couple of named data
items.  An empty class definition will do nicely, e.g.:

\begin{verbatim}
        class Employee:
                pass

        john = Employee() # Create an empty employee record

        # Fill the fields of the record
        john.name = 'John Doe'
        john.dept = 'computer lab'
        john.salary = 1000
\end{verbatim}


A piece of Python code that expects a particular abstract data type
can often be passed a class that emulates the methods of that data
type instead.  For instance, if you have a function that formats some
data from a file object, you can define a class with methods
\verb\read()\ and \verb\readline()\ that gets the data from a string
buffer instead, and pass it as an argument.  (Unfortunately, this
technique has its limitations: a class can't define operations that
are accessed by special syntax such as sequence subscripting or
arithmetic operators, and assigning such a ``pseudo-file'' to
\verb\sys.stdin\ will not cause the interpreter to read further input
from it.)


Instance method objects have attributes, too: \verb\m.im_self\ is the
object of which the method is an instance, and \verb\m.im_func\ is the
function object corresponding to the method.

\subsection{Exceptions Can Be Classes}

User-defined exceptions are no longer limited to being string objects
--- they can be identified by classes as well.  Using this mechanism it
is possible to create extensible hierarchies of exceptions.

There are two new valid (semantic) forms for the raise statement:

\begin{verbatim}
raise Class, instance

raise instance
\end{verbatim}

In the first form, \code{instance} must be an instance of \code{Class}
or of a class derived from it.  The second form is a shorthand for

\begin{verbatim}
raise instance.__class__, instance
\end{verbatim}

An except clause may list classes as well as string objects.  A class
in an except clause is compatible with an exception if it is the same
class or a base class thereof (but not the other way around --- an
except clause listing a derived class is not compatible with a base
class).  For example, the following code will print B, C, D in that
order:

\begin{verbatim}
class B:
    pass
class C(B):
    pass
class D(C):
    pass

for c in [B, C, D]:
    try:
        raise c()
    except D:
        print "D"
    except C:
        print "C"
    except B:
        print "B"
\end{verbatim}

Note that if the except clauses were reversed (with ``\code{except B}''
first), it would have printed B, B, B --- the first matching except
clause is triggered.

When an error message is printed for an unhandled exception which is a
class, the class name is printed, then a colon and a space, and
finally the instance converted to a string using the built-in function
\code{str()}.

In this release, the built-in exceptions are still strings.

\chapter{What Now?}

Hopefully reading this tutorial has reinforced your interest in using
Python.  Now what should you do?

You should read, or at least page through, the Library Reference,
which gives complete (though terse) reference material about types,
functions, and modules that can save you a lot of time when writing
Python programs.  The standard Python distribution includes a
\emph{lot} of code in both C and Python; there are modules to read
Unix mailboxes, retrieve documents via HTTP, generate random numbers,
parse command-line options, write CGI programs, compress data, and a
lot more; skimming through the Library Reference will give you an idea
of what's available.

The major Python Web site is \code{http://www.python.org}; it contains
code, documentation, and pointers to Python-related pages around the
Web.  \code{www.python.org} is mirrored in various places around the
world, such as Europe, Japan, and Australia; a mirror may be faster
than the main site, depending on your geographical location.  A more
informal site is \code{http://starship.skyport.net}, which contains a
bunch of Python-related personal home pages; many people have
downloadable software here.

For Python-related questions and problem reports, you can post to the
newsgroup \code{comp.lang.python}, or send them to the mailing list at
\code{python-list@cwi.nl}.  The newsgroup and mailing list are
gatewayed, so messages posted to one will automatically be forwarded
to the other.  There are around 20--30 postings a day, asking (and
answering) questions, suggesting new features, and announcing new
modules.  But before posting, be sure to check the list of Frequently
Asked Questions (also called the FAQ), at
\code{http://www.python.org/doc/FAQ.html}, or look for it in the
\code{Misc/} directory of the Python source distribution.  The FAQ
answers many of the questions that come up again and again, and may
already contain the solution for your problem.

You can support the Python community by joining the Python Software
Activity, which runs the python.org web, ftp and email servers, and
organizes Python workshops.  See \code{http://www.python.org/psa/} for
information on how to join.


\chapter{Recent Additions as of Release 1.1}

XXX Should the stuff in this chapter be deleted, or can a home be found or it elsewhere in the Tutorial?

\section{Lambda Forms}

XXX Where to put this?  Or just leave it out?

By popular demand, a few features commonly found in functional
programming languages and Lisp have been added to Python.  With the
\verb\lambda\ keyword, small anonymous functions can be created.
Here's a function that returns the sum of its two arguments:
\verb\lambda a, b: a+b\.  Lambda forms can be used wherever function
objects are required.  They are syntactically restricted to a single
expression.  Semantically, they are just syntactic sugar for a normal
function definition.  Like nested function definitions, lambda forms
cannot reference variables from the containing scope, but this can be
overcome through the judicious use of default argument values, e.g.

\begin{verbatim}
        def make_incrementor(n):
                return lambda x, incr=n: x+incr
\end{verbatim}

\section{Documentation Strings}

XXX Where to put this?  Or just leave it out?

There are emerging conventions about the content and formatting of
documentation strings.

The first line should always be a short, concise summary of the
object's purpose.  For brevity, it should not explicitly state the
object's name or type, since these are available by other means
(except if the name happens to be a verb describing a function's
operation).  This line should begin with a capital letter and end with
a period.

If there are more lines in the documentation string, the second line
should be blank, visually separating the summary from the rest of the
description.  The following lines should be one of more of paragraphs
describing the objects calling conventions, its side effects, etc.

Some people like to copy the Emacs convention of using UPPER CASE for
function parameters --- this often saves a few words or lines.

The Python parser does not strip indentation from multi-line string
literals in Python, so tools that process documentation have to strip
indentation.  This is done using the following convention.  The first
non-blank line {\em after} the first line of the string determines the
amount of indentation for the entire documentation string.  (We can't
use the first line since it is generally adjacent to the string's
opening quotes so its indentation is not apparent in the string
literal.)  Whitespace ``equivalent'' to this indentation is then
stripped from the start of all lines of the string.  Lines that are
indented less should not occur, but if they occur all their leading
whitespace should be stripped.  Equivalence of whitespace should be
tested after expansion of tabs (to 8 spaces, normally).


\appendix\chapter{Interactive Input Editing and History Substitution}

Some versions of the Python interpreter support editing of the current
input line and history substitution, similar to facilities found in
the Korn shell and the GNU Bash shell.  This is implemented using the
{\em GNU\ Readline} library, which supports Emacs-style and vi-style
editing.  This library has its own documentation which I won't
duplicate here; however, the basics are easily explained.

\subsection{Line Editing}

If supported, input line editing is active whenever the interpreter
prints a primary or secondary prompt.  The current line can be edited
using the conventional Emacs control characters.  The most important
of these are: C-A (Control-A) moves the cursor to the beginning of the
line, C-E to the end, C-B moves it one position to the left, C-F to
the right.  Backspace erases the character to the left of the cursor,
C-D the character to its right.  C-K kills (erases) the rest of the
line to the right of the cursor, C-Y yanks back the last killed
string.  C-underscore undoes the last change you made; it can be
repeated for cumulative effect.

\subsection{History Substitution}

History substitution works as follows.  All non-empty input lines
issued are saved in a history buffer, and when a new prompt is given
you are positioned on a new line at the bottom of this buffer.  C-P
moves one line up (back) in the history buffer, C-N moves one down.
Any line in the history buffer can be edited; an asterisk appears in
front of the prompt to mark a line as modified.  Pressing the Return
key passes the current line to the interpreter.  C-R starts an
incremental reverse search; C-S starts a forward search.

\subsection{Key Bindings}

The key bindings and some other parameters of the Readline library can
be customized by placing commands in an initialization file called
{\tt \$HOME/.inputrc}.  Key bindings have the form

\bcode\begin{verbatim}
key-name: function-name
\end{verbatim}\ecode
%
or

\bcode\begin{verbatim}
"string": function-name
\end{verbatim}\ecode
%
and options can be set with

\bcode\begin{verbatim}
set option-name value
\end{verbatim}\ecode
%
For example:

\bcode\begin{verbatim}
# I prefer vi-style editing:
set editing-mode vi
# Edit using a single line:
set horizontal-scroll-mode On
# Rebind some keys:
Meta-h: backward-kill-word
"\C-u": universal-argument
"\C-x\C-r": re-read-init-file
\end{verbatim}\ecode
%
Note that the default binding for TAB in Python is to insert a TAB
instead of Readline's default filename completion function.  If you
insist, you can override this by putting

\bcode\begin{verbatim}
TAB: complete
\end{verbatim}\ecode
%
in your {\tt \$HOME/.inputrc}.  (Of course, this makes it hard to type
indented continuation lines...)

\subsection{Commentary}

This facility is an enormous step forward compared to previous
versions of the interpreter; however, some wishes are left: It would
be nice if the proper indentation were suggested on continuation lines
(the parser knows if an indent token is required next).  The
completion mechanism might use the interpreter's symbol table.  A
command to check (or even suggest) matching parentheses, quotes etc.
would also be useful.

XXX Lele Gaifax's readline module, which adds name completion...

\end{document}


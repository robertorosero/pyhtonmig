\section{Standard Module \sectcode{binhex}}
\label{module-binhex}
\stmodindex{binhex}

This module encodes and decodes files in binhex4 format, a format
allowing representation of Macintosh files in ASCII. On the macintosh,
both forks of a file and the finder information are encoded (or
decoded), on other platforms only the data fork is handled.

The \code{binhex} module defines the following functions:

\setindexsubitem{(in module binhex)}

\begin{funcdesc}{binhex}{input\, output}
Convert a binary file with filename \var{input} to binhex file
\var{output}. The \var{output} parameter can either be a filename or a
file-like object (any object supporting a \var{write} and \var{close}
method).
\end{funcdesc}

\begin{funcdesc}{hexbin}{input\optional{\, output}}
Decode a binhex file \var{input}. \var{input} may be a filename or a
file-like object supporting \var{read} and \var{close} methods.
The resulting file is written to a file named \var{output}, unless the
argument is empty in which case the output filename is read from the
binhex file.
\end{funcdesc}

\subsection{notes}
There is an alternative, more powerful interface to the coder and
decoder, see the source for details.

If you code or decode textfiles on non-Macintosh platforms they will
still use the macintosh newline convention (carriage-return as end of
line).

As of this writing, \var{hexbin} appears to not work in all cases.

\section{Standard Module \sectcode{uu}}
\stmodindex{uu}

This module encodes and decodes files in uuencode format, allowing
arbitrary binary data to be transferred over ascii-only connections.
Wherever a file argument is expected, the methods accept a file-like
object.  For backwards compatibility, a string containing a pathname
is also accepted, and the corresponding file will be opened for
reading and writing; the pathname \code{'-'} is understood to mean the
standard input or output.  However, this interface is deprecated; it's
better for the caller to open the file itself, and be sure that, when
required, the mode is \code{'rb'} or \code{'wb'} on Windows or DOS.

This code was contributed by Lance Ellinghouse, and modified by Jack
Jansen.

The \code{uu} module defines the following functions:

\setindexsubitem{(in module uu)}

\begin{funcdesc}{encode}{in_file\, out_file\optional{\, name\, mode}}
Uuencode file \var{in_file} into file \var{out_file}.  The uuencoded
file will have the header specifying \var{name} and \var{mode} as the
defaults for the results of decoding the file. The default defaults
are taken from \var{in_file}, or \code{'-'} and \code{0666}
respectively. 
\end{funcdesc}

\begin{funcdesc}{decode}{in_file\optional{\, out_file\, mode}}
This call decodes uuencoded file \var{in_file} placing the result on
file \var{out_file}. If \var{out_file} is a pathname the \var{mode} is
also set. Defaults for \var{out_file} and \var{mode} are taken from
the uuencode header.
\end{funcdesc}

\section{Built-in Module \sectcode{binascii}}	% If implemented in C
\bimodindex{binascii}

The binascii module contains a number of methods to convert between
binary and various ascii-encoded binary representations. Normally, you
will not use these modules directly but use wrapper modules like
\var{uu} or \var{hexbin} in stead, this module solely exists because
bit-manipuation of large amounts of data is slow in python.

The \code{binascii} module defines the following functions:

\setindexsubitem{(in module binascii)}

\begin{funcdesc}{a2b_uu}{string}
Convert a single line of uuencoded data back to binary and return the
binary data. Lines normally contain 45 (binary) bytes, except for the
last line. Line data may be followed by whitespace.
\end{funcdesc}

\begin{funcdesc}{b2a_uu}{data}
Convert binary data to a line of ascii characters, the return value is
the converted line, including a newline char. The length of \var{data}
should be at most 45.
\end{funcdesc}

\begin{funcdesc}{a2b_base64}{string}
Convert a block of base64 data back to binary and return the
binary data. More than one line may be passed at a time.
\end{funcdesc}

\begin{funcdesc}{b2a_base64}{data}
Convert binary data to a line of ascii characters in base64 coding.
The return value is the converted line, including a newline char.
The length of \var{data} should be at most 57 to adhere to the base64
standard.
\end{funcdesc}

\begin{funcdesc}{a2b_hqx}{string}
Convert binhex4 formatted ascii data to binary, without doing
rle-decompression. The string should contain a complete number of
binary bytes, or (in case of the last portion of the binhex4 data)
have the remaining bits zero.
\end{funcdesc}

\begin{funcdesc}{rledecode_hqx}{data}
Perform RLE-decompression on the data, as per the binhex4
standard. The algorithm uses \code{0x90} after a byte as a repeat
indicator, followed by a count. A count of \code{0} specifies a byte
value of \code{0x90}. The routine returns the decompressed data,
unless data input data ends in an orphaned repeat indicator, in which
case the \var{Incomplete} exception is raised.
\end{funcdesc}

\begin{funcdesc}{rlecode_hqx}{data}
Perform binhex4 style RLE-compression on \var{data} and return the
result.
\end{funcdesc}

\begin{funcdesc}{b2a_hqx}{data}
Perform hexbin4 binary-to-ascii translation and return the resulting
string. The argument should already be rle-coded, and have a length
divisible by 3 (except possibly the last fragment).
\end{funcdesc}

\begin{funcdesc}{crc_hqx}{data, crc}
Compute the binhex4 crc value of \var{data}, starting with an initial
\var{crc} and returning the result.
\end{funcdesc}
 
\begin{excdesc}{Error}
Exception raised on errors. These are usually programming errors.
\end{excdesc}

\begin{excdesc}{Incomplete}
Exception raised on incomplete data. These are usually not programming
errors, but handled by reading a little more data and trying again.
\end{excdesc}

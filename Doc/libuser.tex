\section{Standard Module \sectcode{user}}
\label{module-user}
\stmodindex{user}
\kwindex{.pythonrc.py}

As a policy, Python doesn't run user-specified code on startup of
Python programs.  (Only interactive sessions execute the script
specified in the \code{PYTHONSTARTUP} environment variable if it exists).

However, some programs or sites may find it convenient to allow users
to have a standard customization file, which gets run when a program
requests it.  This module implements such a mechanism.  A program
that wishes to use the mechanism must execute the statement

\bcode\begin{verbatim}
import user
\end{verbatim}\ecode

The user module looks for a file \file{.pythonrc.py} in the user's
home directory and if it can be opened, exececutes it (using
\code{execfile()}) in its own (i.e. the module \code{user}'s) global
namespace.  Errors during this phase are not caught; that's up to the
program that imports the user module, if it wishes.  The home
directory is assumed to be named by the \code{HOME} environment
variable; if this is not set, the current directory is used.

The user's \file{.pythonrc.py} could conceivably test for
\code{sys.version} if it wishes to do different things depending on
the Python version.

A warning to users: be very conservative in what you place in your
\file{.pythonrc.py} file.  Since you don't know which programs will
use it, changing the behavior of standard modules or functions is
generally not a good idea.

A suggestion for programmers who wish to use this mechanism: a simple
way to let users specify options for your package is to have them
define variables in their \var{.pythonrc.py} file that you test in
your module.  For example, a module \code{spam} that has a verbosity
level can look for a variable \code{user.spam_verbose}, as follows:

\bcode\begin{verbatim}
import user
try:
    verbose = user.spam_verbose  # user's verbosity preference
except AttributeError:
    verbose = 0                  # default verbosity
\end{verbatim}\ecode

Programs with extensive customization needs are better off reading a
program-specific customization file.

Programs with security or privacy concerns should \emph{not} import
this module; a user can easily break into a a program by placing
arbitrary code in the \file{.pythonrc.py} file.

Modules for general use should \emph{not} import this module; it may
interfere with the operation of the importing program.

For a site-wide customization mechanism, see module \code{site}.

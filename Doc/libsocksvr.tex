\section{Standard Module \module{SocketServer}}
\label{module-SocketServer}
\stmodindex{SocketServer}

The \module{SocketServer} module simplifies the task of writing network
servers.

There are four basic server classes: \class{TCPServer} uses the
Internet TCP protocol, which provides for continuous streams of data
between the client and server.  \class{UDPServer} uses datagrams, which
are discrete packets of information that may arrive out of order or be
lost while in transit.  The more infrequently used
\class{UnixStreamServer} and \class{UnixDatagramServer} classes are
similar, but use \UNIX{} domain sockets; they're not available on
non-\UNIX{} platforms.  For more details on network programming, consult
a book such as W. Richard Steven's \emph{UNIX Network Programming}
or Ralph Davis's \emph{Win32 Network Programming}.

These four classes process requests \dfn{synchronously}; each request
must be completed before the next request can be started.  This isn't
suitable if each request takes a long time to complete, because it
requires a lot of computation, or because it returns a lot of data
which the client is slow to process.  The solution is to create a
separate process or thread to handle each request; the
\class{ForkingMixIn} and \class{ThreadingMixIn} mix-in classes can be
used to support asynchronous behaviour.

Creating a server requires several steps.  First, you must create a
request handler class by subclassing the \class{BaseRequestHandler}
class and overriding its \method{handle()} method; this method will
process incoming requests.  Second, you must instantiate one of the
server classes, passing it the server's address and the request
handler class.  Finally, call the \method{handle_request()} or
\method{serve_forever()} method of the server object to process one or
many requests.

Server classes have the same external methods and attributes, no
matter what network protocol they use:

\setindexsubitem{(SocketServer protocol)}

%XXX should data and methods be intermingled, or separate?
% how should the distinction between class and instance variables be
% drawn?

\begin{funcdesc}{fileno}{}
Return an integer file descriptor for the socket on which the server
is listening.  This function is most commonly passed to
\function{select.select()}, to allow monitoring multiple servers in the
same process.
\end{funcdesc}

\begin{funcdesc}{handle_request}{}
Process a single request.  This function calls the following methods
in order: \method{get_request()}, \method{verify_request()}, and
\method{process_request()}.  If the user-provided \method{handle()}
method of the handler class raises an exception, the server's
\method{handle_error()} method will be called.
\end{funcdesc}

\begin{funcdesc}{serve_forever}{}
Handle an infinite number of requests.  This simply calls
\method{handle_request()} inside an infinite loop.
\end{funcdesc}

\begin{datadesc}{address_family}
The family of protocols to which the server's socket belongs.
\constant{socket.AF_INET} and \constant{socket.AF_UNIX} are two
possible values.
\end{datadesc}

\begin{datadesc}{RequestHandlerClass}
The user-provided request handler class; an instance of this class is
created for each request.
\end{datadesc}

\begin{datadesc}{server_address}
The address on which the server is listening.  The format of addresses
varies depending on the protocol family; see the documentation for the
socket module for details.  For Internet protocols, this is a tuple
containing a string giving the address, and an integer port number:
\code{('127.0.0.1', 80)}, for example.
\end{datadesc}

\begin{datadesc}{socket}
The socket object on which the server will listen for incoming requests.
\end{datadesc}

% XXX should class variables be covered before instance variables, or
% vice versa?

The server classes support the following class variables:

\begin{datadesc}{request_queue_size}
The size of the request queue.  If it takes a long time to process a
single request, any requests that arrive while the server is busy are
placed into a queue, up to \member{request_queue_size} requests.  Once
the queue is full, further requests from clients will get a
``Connection denied'' error.  The default value is usually 5, but this
can be overridden by subclasses.
\end{datadesc}

\begin{datadesc}{socket_type}
The type of socket used by the server; \constant{socket.SOCK_STREAM}
and \constant{socket.SOCK_DGRAM} are two possible values.
\end{datadesc}

There are various server methods that can be overridden by subclasses
of base server classes like \class{TCPServer}; these methods aren't
useful to external users of the server object.

% should the default implementations of these be documented, or should
% it be assumed that the user will look at SocketServer.py?

\begin{funcdesc}{finish_request}{}
Actually processes the request by instantiating
\member{RequestHandlerClass} and calling its \method{handle()} method.
\end{funcdesc}

\begin{funcdesc}{get_request}{}
Must accept a request from the socket, and return a 2-tuple containing
the \emph{new} socket object to be used to communicate with the
client, and the client's address.
\end{funcdesc}

\begin{funcdesc}{handle_error}{request, client_address}
This function is called if the \member{RequestHandlerClass}'s
\method{handle()} method raises an exception.  The default action is
to print the traceback to standard output and continue handling
further requests.
\end{funcdesc}

\begin{funcdesc}{process_request}{request, client_address}
Calls \method{finish_request()} to create an instance of the
\member{RequestHandlerClass}.  If desired, this function can create a
new process or thread to handle the request; the \class{ForkingMixIn}
and \class{ThreadingMixIn} classes do this.
\end{funcdesc}

% Is there any point in documenting the following two functions?
% What would the purpose of overriding them be: initializing server
% instance variables, adding new network families?

\begin{funcdesc}{server_activate}{}
Called by the server's constructor to activate the server.
May be overridden.
\end{funcdesc}

\begin{funcdesc}{server_bind}{}
Called by the server's constructor to bind the socket to the desired
address.  May be overridden.
\end{funcdesc}

\begin{funcdesc}{verify_request}{request, client_address}
Must return a Boolean value; if the value is true, the request will be
processed, and if it's false, the request will be denied.
This function can be overridden to implement access controls for a server.
The default implementation always return true.
\end{funcdesc}

The request handler class must define a new \method{handle()} method,
and can override any of the following methods.  A new instance is
created for each request.

\begin{funcdesc}{finish}{}
Called after the \method{handle()} method to perform any clean-up
actions required.  The default implementation does nothing.  If
\method{setup()} or \method{handle()} raise an exception, this
function will not be called.
\end{funcdesc}

\begin{funcdesc}{handle}{}
This function must do all the work required to service a request.
Several instance attributes are available to it; the request is
available as \member{self.request}; the client address as
\member{self.client_request}; and the server instance as
\member{self.server}, in case it needs access to per-server
information.

The type of \member{self.request} is different for datagram or stream
services.  For stream services, \member{self.request} is a socket
object; for datagram services, \member{self.request} is a string.
However, this can be hidden by using the mix-in request handler
classes
\class{StreamRequestHandler} or \class{DatagramRequestHandler}, which
override the \method{setup()} and \method{finish()} methods, and
provides \member{self.rfile} and \member{self.wfile} attributes.
\member{self.rfile} and \member{self.wfile} can be read or written,
respectively, to get the request data or return data to the client.
\end{funcdesc}

\begin{funcdesc}{setup}{}
Called before the \method{handle()} method to perform any
initialization actions required.  The default implementation does
nothing.
\end{funcdesc}

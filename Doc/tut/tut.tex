\documentclass{manual}
\usepackage[T1]{fontenc}

% Things to do:
% Should really move the Python startup file info to an appendix

\title{Python Tutorial}

\author{
	Guido van Rossum \\
	Dept. AA, CWI, P.O. Box 94079 \\
	1090 GB Amsterdam, The Netherlands \\
	E-mail: {\tt guido@cwi.nl}
}

\date{17 March 1995 \\ Release 1.2-proof-2} % XXX update before release!


\makeindex

\begin{document}

\maketitle

\ifhtml
\chapter*{Front Matter\label{front}}
\fi

\strong{BEOPEN.COM TERMS AND CONDITIONS FOR PYTHON 2.0}

\centerline{\strong{BEOPEN PYTHON OPEN SOURCE LICENSE AGREEMENT VERSION 1}}

\begin{enumerate}

\item
This LICENSE AGREEMENT is between BeOpen.com (``BeOpen''), having an
office at 160 Saratoga Avenue, Santa Clara, CA 95051, and the
Individual or Organization (``Licensee'') accessing and otherwise
using this software in source or binary form and its associated
documentation (``the Software'').

\item
Subject to the terms and conditions of this BeOpen Python License
Agreement, BeOpen hereby grants Licensee a non-exclusive,
royalty-free, world-wide license to reproduce, analyze, test, perform
and/or display publicly, prepare derivative works, distribute, and
otherwise use the Software alone or in any derivative version,
provided, however, that the BeOpen Python License is retained in the
Software, alone or in any derivative version prepared by Licensee.

\item
BeOpen is making the Software available to Licensee on an ``AS IS''
basis.  BEOPEN MAKES NO REPRESENTATIONS OR WARRANTIES, EXPRESS OR
IMPLIED.  BY WAY OF EXAMPLE, BUT NOT LIMITATION, BEOPEN MAKES NO AND
DISCLAIMS ANY REPRESENTATION OR WARRANTY OF MERCHANTABILITY OR FITNESS
FOR ANY PARTICULAR PURPOSE OR THAT THE USE OF THE SOFTWARE WILL NOT
INFRINGE ANY THIRD PARTY RIGHTS.

\item
BEOPEN SHALL NOT BE LIABLE TO LICENSEE OR ANY OTHER USERS OF THE
SOFTWARE FOR ANY INCIDENTAL, SPECIAL, OR CONSEQUENTIAL DAMAGES OR LOSS
AS A RESULT OF USING, MODIFYING OR DISTRIBUTING THE SOFTWARE, OR ANY
DERIVATIVE THEREOF, EVEN IF ADVISED OF THE POSSIBILITY THEREOF.

\item
This License Agreement will automatically terminate upon a material
breach of its terms and conditions.

\item
This License Agreement shall be governed by and interpreted in all
respects by the law of the State of California, excluding conflict of
law provisions.  Nothing in this License Agreement shall be deemed to
create any relationship of agency, partnership, or joint venture
between BeOpen and Licensee.  This License Agreement does not grant
permission to use BeOpen trademarks or trade names in a trademark
sense to endorse or promote products or services of Licensee, or any
third party.  As an exception, the ``BeOpen Python'' logos available
at http://www.pythonlabs.com/logos.html may be used according to the
permissions granted on that web page.

\item
By copying, installing or otherwise using the software, Licensee
agrees to be bound by the terms and conditions of this License
Agreement.
\end{enumerate}


\centerline{\strong{CNRI OPEN SOURCE LICENSE AGREEMENT}}

Python 1.6 is made available subject to the terms and conditions in
CNRI's License Agreement.  This Agreement together with Python 1.6 may
be located on the Internet using the following unique, persistent
identifier (known as a handle): 1895.22/1012.  This Agreement may also
be obtained from a proxy server on the Internet using the following
URL: \url{http://hdl.handle.net/1895.22/1012}.


\centerline{\strong{CWI PERMISSIONS STATEMENT AND DISCLAIMER}}

Copyright \copyright{} 1991 - 1995, Stichting Mathematisch Centrum
Amsterdam, The Netherlands.  All rights reserved.

Permission to use, copy, modify, and distribute this software and its
documentation for any purpose and without fee is hereby granted,
provided that the above copyright notice appear in all copies and that
both that copyright notice and this permission notice appear in
supporting documentation, and that the name of Stichting Mathematisch
Centrum or CWI not be used in advertising or publicity pertaining to
distribution of the software without specific, written prior
permission.

STICHTING MATHEMATISCH CENTRUM DISCLAIMS ALL WARRANTIES WITH REGARD TO
THIS SOFTWARE, INCLUDING ALL IMPLIED WARRANTIES OF MERCHANTABILITY AND
FITNESS, IN NO EVENT SHALL STICHTING MATHEMATISCH CENTRUM BE LIABLE
FOR ANY SPECIAL, INDIRECT OR CONSEQUENTIAL DAMAGES OR ANY DAMAGES
WHATSOEVER RESULTING FROM LOSS OF USE, DATA OR PROFITS, WHETHER IN AN
ACTION OF CONTRACT, NEGLIGENCE OR OTHER TORTIOUS ACTION, ARISING OUT
OF OR IN CONNECTION WITH THE USE OR PERFORMANCE OF THIS SOFTWARE.


\begin{abstract}

\noindent
Python is an easy to learn, powerful programming language.  It has
efficient high-level data structures and a simple but effective
approach to object-oriented programming.  Python's elegant syntax and
dynamic typing, together with its interpreted nature, make it an ideal 
language for scripting and rapid application development in many areas 
on most platforms.

The Python interpreter and the extensive standard library are freely
available in source or binary form for all major platforms from the
Python Web site, \url{http://www.python.org/}, and can be freely
distributed.  The same site also contains distributions of and
pointers to many free third party Python modules, programs and tools,
and additional documentation.

The Python interpreter is easily extended with new functions and data
types implemented in C or \Cpp{} (or other languages callable from C).
Python is also suitable as an extension language for customizable
applications.

This tutorial introduces the reader informally to the basic concepts
and features of the Python language and system.  It helps to have a
Python interpreter handy for hands-on experience, but all examples are
self-contained, so the tutorial can be read off-line as well.

For a description of standard objects and modules, see the
\citetitle[../lib/lib.html]{Python Library Reference} document.  The
\citetitle[../ref/ref.html]{Python Reference Manual} gives a more
formal definition of the language.  To write extensions in C or
\Cpp, read \citetitle[../ext/ext.html]{Extending and Embedding the
Python Interpreter} and \citetitle[../api/api.html]{Python/C API
Reference}.  There are also several books covering Python in depth.

This tutorial does not attempt to be comprehensive and cover every
single feature, or even every commonly used feature.  Instead, it
introduces many of Python's most noteworthy features, and will give
you a good idea of the language's flavor and style.  After reading it,
you will be able to read and write Python modules and programs, and
you will be ready to learn more about the various Python library
modules described in the \citetitle[../lib/lib.html]{Python Library
Reference}.

\end{abstract}

\tableofcontents


\chapter{Whetting Your Appetite \label{intro}}

If you ever wrote a large shell script, you probably know this
feeling: you'd love to add yet another feature, but it's already so
slow, and so big, and so complicated; or the feature involves a system
call or other function that is only accessible from C \ldots Usually
the problem at hand isn't serious enough to warrant rewriting the
script in C; perhaps the problem requires variable-length strings or
other data types (like sorted lists of file names) that are easy in
the shell but lots of work to implement in C, or perhaps you're not
sufficiently familiar with C.

Another situation: perhaps you have to work with several C libraries,
and the usual C write/compile/test/re-compile cycle is too slow.  You
need to develop software more quickly.  Possibly perhaps you've
written a program that could use an extension language, and you don't
want to design a language, write and debug an interpreter for it, then
tie it into your application.

In such cases, Python may be just the language for you.  Python is
simple to use, but it is a real programming language, offering much
more structure and support for large programs than the shell has.  On
the other hand, it also offers much more error checking than C, and,
being a \emph{very-high-level language}, it has high-level data types
built in, such as flexible arrays and dictionaries that would cost you
days to implement efficiently in C.  Because of its more general data
types Python is applicable to a much larger problem domain than
\emph{Awk} or even \emph{Perl}, yet many things are at least as easy
in Python as in those languages.

Python allows you to split up your program in modules that can be
reused in other Python programs.  It comes with a large collection of
standard modules that you can use as the basis of your programs --- or
as examples to start learning to program in Python.  There are also
built-in modules that provide things like file I/O, system calls,
sockets, and even interfaces to graphical user interface toolkits like Tk.  

Python is an interpreted language, which can save you considerable time
during program development because no compilation and linking is
necessary.  The interpreter can be used interactively, which makes it
easy to experiment with features of the language, to write throw-away
programs, or to test functions during bottom-up program development.
It is also a handy desk calculator.

Python allows writing very compact and readable programs.  Programs
written in Python are typically much shorter than equivalent C or
\Cpp{} programs, for several reasons:
\begin{itemize}
\item
the high-level data types allow you to express complex operations in a
single statement;
\item
statement grouping is done by indentation instead of beginning and ending
brackets;
\item
no variable or argument declarations are necessary.
\end{itemize}

Python is \emph{extensible}: if you know how to program in C it is easy
to add a new built-in function or module to the interpreter, either to
perform critical operations at maximum speed, or to link Python
programs to libraries that may only be available in binary form (such
as a vendor-specific graphics library).  Once you are really hooked,
you can link the Python interpreter into an application written in C
and use it as an extension or command language for that application.

By the way, the language is named after the BBC show ``Monty Python's
Flying Circus'' and has nothing to do with nasty reptiles.  Making
references to Monty Python skits in documentation is not only allowed,
it is encouraged!

%\section{Where From Here \label{where}}

Now that you are all excited about Python, you'll want to examine it
in some more detail.  Since the best way to learn a language is
using it, you are invited here to do so.

In the next chapter, the mechanics of using the interpreter are
explained.  This is rather mundane information, but essential for
trying out the examples shown later.

The rest of the tutorial introduces various features of the Python
language and system through examples, beginning with simple
expressions, statements and data types, through functions and modules,
and finally touching upon advanced concepts like exceptions
and user-defined classes.

\chapter{Using the Python Interpreter \label{using}}

\section{Invoking the Interpreter \label{invoking}}

The Python interpreter is usually installed as
\file{/usr/local/bin/python} on those machines where it is available;
putting \file{/usr/local/bin} in your \UNIX{} shell's search path
makes it possible to start it by typing the command

\begin{verbatim}
python
\end{verbatim}

to the shell.  Since the choice of the directory where the interpreter
lives is an installation option, other places are possible; check with
your local Python guru or system administrator.  (E.g.,
\file{/usr/local/python} is a popular alternative location.)

Typing an end-of-file character (\kbd{Control-D} on \UNIX,
\kbd{Control-Z} on Windows) at the primary prompt causes the
interpreter to exit with a zero exit status.  If that doesn't work,
you can exit the interpreter by typing the following commands:
\samp{import sys; sys.exit()}.

The interpreter's line-editing features usually aren't very
sophisticated.  On \UNIX, whoever installed the interpreter may have
enabled support for the GNU readline library, which adds more
elaborate interactive editing and history features. Perhaps the
quickest check to see whether command line editing is supported is
typing Control-P to the first Python prompt you get.  If it beeps, you
have command line editing; see Appendix \ref{interacting} for an
introduction to the keys.  If nothing appears to happen, or if
\code{\^P} is echoed, command line editing isn't available; you'll
only be able to use backspace to remove characters from the current
line.

The interpreter operates somewhat like the \UNIX{} shell: when called
with standard input connected to a tty device, it reads and executes
commands interactively; when called with a file name argument or with
a file as standard input, it reads and executes a \emph{script} from
that file. 

A second way of starting the interpreter is
\samp{\program{python} \programopt{-c} \var{command} [arg] ...}, which
executes the statement(s) in \var{command}, analogous to the shell's
\programopt{-c} option.  Since Python statements often contain spaces
or other characters that are special to the shell, it is best to quote 
\var{command} in its entirety with double quotes.

Note that there is a difference between \samp{python file} and
\samp{python <file}.  In the latter case, input requests from the
program, such as calls to \function{input()} and \function{raw_input()}, are
satisfied from \emph{file}.  Since this file has already been read
until the end by the parser before the program starts executing, the
program will encounter end-of-file immediately.  In the former case
(which is usually what you want) they are satisfied from whatever file
or device is connected to standard input of the Python interpreter.

When a script file is used, it is sometimes useful to be able to run
the script and enter interactive mode afterwards.  This can be done by
passing \programopt{-i} before the script.  (This does not work if the
script is read from standard input, for the same reason as explained
in the previous paragraph.)

\subsection{Argument Passing \label{argPassing}}

When known to the interpreter, the script name and additional
arguments thereafter are passed to the script in the variable
\code{sys.argv}, which is a list of strings.  Its length is at least
one; when no script and no arguments are given, \code{sys.argv[0]} is
an empty string.  When the script name is given as \code{'-'} (meaning 
standard input), \code{sys.argv[0]} is set to \code{'-'}.  When
\programopt{-c} \var{command} is used, \code{sys.argv[0]} is set to
\code{'-c'}.  Options found after \programopt{-c} \var{command} are
not consumed by the Python interpreter's option processing but left in
\code{sys.argv} for the command to handle.

\subsection{Interactive Mode \label{interactive}}

When commands are read from a tty, the interpreter is said to be in
\emph{interactive mode}.  In this mode it prompts for the next command
with the \emph{primary prompt}, usually three greater-than signs
(\samp{>\code{>}>~}); for continuation lines it prompts with the
\emph{secondary prompt}, by default three dots (\samp{...~}).
The interpreter prints a welcome message stating its version number
and a copyright notice before printing the first prompt:

\begin{verbatim}
python
Python 1.5.2b2 (#1, Feb 28 1999, 00:02:06)  [GCC 2.8.1] on sunos5
Copyright 1991-1995 Stichting Mathematisch Centrum, Amsterdam
>>>
\end{verbatim}

Continuation lines are needed when entering a multi-line construct.
As an example, take a look at this \keyword{if} statement:

\begin{verbatim}
>>> the_world_is_flat = 1
>>> if the_world_is_flat:
...     print "Be careful not to fall off!"
... 
Be careful not to fall off!
\end{verbatim}


\section{The Interpreter and Its Environment \label{interp}}

\subsection{Error Handling \label{error}}

When an error occurs, the interpreter prints an error
message and a stack trace.  In interactive mode, it then returns to
the primary prompt; when input came from a file, it exits with a
nonzero exit status after printing
the stack trace.  (Exceptions handled by an \keyword{except} clause in a
\keyword{try} statement are not errors in this context.)  Some errors are
unconditionally fatal and cause an exit with a nonzero exit; this
applies to internal inconsistencies and some cases of running out of
memory.  All error messages are written to the standard error stream;
normal output from the executed commands is written to standard
output.

Typing the interrupt character (usually Control-C or DEL) to the
primary or secondary prompt cancels the input and returns to the
primary prompt.\footnote{
        A problem with the GNU Readline package may prevent this.
}
Typing an interrupt while a command is executing raises the
\exception{KeyboardInterrupt} exception, which may be handled by a
\keyword{try} statement.

\subsection{Executable Python Scripts \label{scripts}}

On BSD'ish \UNIX{} systems, Python scripts can be made directly
executable, like shell scripts, by putting the line

\begin{verbatim}
#! /usr/bin/env python
\end{verbatim}

(assuming that the interpreter is on the user's \envvar{PATH}) at the
beginning of the script and giving the file an executable mode.  The
\samp{\#!} must be the first two characters of the file.  On some
platforms, this first line must end with a \UNIX-style line ending
(\character{\e n}), not a Mac OS (\character{\e r}) or Windows
(\character{\e r\e n}) line ending.  Note that
the hash, or pound, character, \character{\#}, is used to start a
comment in Python.

The script can be given a executable mode, or permission, using the
\program{chmod} command:

\begin{verbatim}
$ chmod +x myscript.py
\end{verbatim} % $ <-- bow to font-lock


\subsection{Source Code Encoding}

It is possible to use encodings different than \ASCII{} in Python source
files. The best way to do it is to put one more special comment line
right after the \code{\#!} line to define the source file encoding:

\begin{verbatim}
# -*- coding: iso-8859-1 -*- 
\end{verbatim}

With that declaration, all characters in the source file will be treated as
{}\code{iso-8859-1}, and it will be
possible to directly write Unicode string literals in the selected
encoding.  The list of possible encodings can be found in the
\citetitle[../lib/lib.html]{Python Library Reference}, in the section
on \ulink{\module{codecs}}{../lib/module-codecs.html}.

If your editor supports saving files as \code{UTF-8} with a UTF-8
\emph{byte order mark} (aka BOM), you can use that instead of an
encoding declaration. IDLE supports this capability if
\code{Options/General/Default Source Encoding/UTF-8} is set. Notice
that this signature is not understood in older Python releases (2.2
and earlier), and also not understood by the operating system for
\code{\#!} files.

By using UTF-8 (either through the signature or an encoding
declaration), characters of most languages in the world can be used
simultaneously in string literals and comments. Using non-\ASCII
characters in identifiers is not supported. To display all these
characters properly, your editor must recognize that the file is
UTF-8, and it must use a font that supports all the characters in the
file.

\subsection{The Interactive Startup File \label{startup}}

% XXX This should probably be dumped in an appendix, since most people
% don't use Python interactively in non-trivial ways.

When you use Python interactively, it is frequently handy to have some
standard commands executed every time the interpreter is started.  You
can do this by setting an environment variable named
\envvar{PYTHONSTARTUP} to the name of a file containing your start-up
commands.  This is similar to the \file{.profile} feature of the
\UNIX{} shells.

This file is only read in interactive sessions, not when Python reads
commands from a script, and not when \file{/dev/tty} is given as the
explicit source of commands (which otherwise behaves like an
interactive session).  It is executed in the same namespace where
interactive commands are executed, so that objects that it defines or
imports can be used without qualification in the interactive session.
You can also change the prompts \code{sys.ps1} and \code{sys.ps2} in
this file.

If you want to read an additional start-up file from the current
directory, you can program this in the global start-up file using code
like \samp{if os.path.isfile('.pythonrc.py'):
execfile('.pythonrc.py')}.  If you want to use the startup file in a
script, you must do this explicitly in the script:

\begin{verbatim}
import os
filename = os.environ.get('PYTHONSTARTUP')
if filename and os.path.isfile(filename):
    execfile(filename)
\end{verbatim}


\chapter{An Informal Introduction to Python \label{informal}}

In the following examples, input and output are distinguished by the
presence or absence of prompts (\samp{>\code{>}>~} and \samp{...~}): to repeat
the example, you must type everything after the prompt, when the
prompt appears; lines that do not begin with a prompt are output from
the interpreter. %
%\footnote{
%        I'd prefer to use different fonts to distinguish input
%        from output, but the amount of LaTeX hacking that would require
%        is currently beyond my ability.
%}
Note that a secondary prompt on a line by itself in an example means
you must type a blank line; this is used to end a multi-line command.

Many of the examples in this manual, even those entered at the
interactive prompt, include comments.  Comments in Python start with
the hash character, \character{\#}, and extend to the end of the
physical line.  A comment may appear at the start of a line or
following whitespace or code, but not within a string literal.  A hash 
character within a string literal is just a hash character.

Some examples:

\begin{verbatim}
# this is the first comment
SPAM = 1                 # and this is the second comment
                         # ... and now a third!
STRING = "# This is not a comment."
\end{verbatim}


\section{Using Python as a Calculator \label{calculator}}

Let's try some simple Python commands.  Start the interpreter and wait
for the primary prompt, \samp{>\code{>}>~}.  (It shouldn't take long.)

\subsection{Numbers \label{numbers}}

The interpreter acts as a simple calculator: you can type an
expression at it and it will write the value.  Expression syntax is
straightforward: the operators \code{+}, \code{-}, \code{*} and
\code{/} work just like in most other languages (for example, Pascal
or C); parentheses can be used for grouping.  For example:

\begin{verbatim}
>>> 2+2
4
>>> # This is a comment
... 2+2
4
>>> 2+2  # and a comment on the same line as code
4
>>> (50-5*6)/4
5
>>> # Integer division returns the floor:
... 7/3
2
>>> 7/-3
-3
\end{verbatim}

Like in C, the equal sign (\character{=}) is used to assign a value to a
variable.  The value of an assignment is not written:

\begin{verbatim}
>>> width = 20
>>> height = 5*9
>>> width * height
900
\end{verbatim}

A value can be assigned to several variables simultaneously:

\begin{verbatim}
>>> x = y = z = 0  # Zero x, y and z
>>> x
0
>>> y
0
>>> z
0
\end{verbatim}

There is full support for floating point; operators with mixed type
operands convert the integer operand to floating point:

\begin{verbatim}
>>> 3 * 3.75 / 1.5
7.5
>>> 7.0 / 2
3.5
\end{verbatim}

Complex numbers are also supported; imaginary numbers are written with
a suffix of \samp{j} or \samp{J}.  Complex numbers with a nonzero
real component are written as \samp{(\var{real}+\var{imag}j)}, or can
be created with the \samp{complex(\var{real}, \var{imag})} function.

\begin{verbatim}
>>> 1j * 1J
(-1+0j)
>>> 1j * complex(0,1)
(-1+0j)
>>> 3+1j*3
(3+3j)
>>> (3+1j)*3
(9+3j)
>>> (1+2j)/(1+1j)
(1.5+0.5j)
\end{verbatim}

Complex numbers are always represented as two floating point numbers,
the real and imaginary part.  To extract these parts from a complex
number \var{z}, use \code{\var{z}.real} and \code{\var{z}.imag}.  

\begin{verbatim}
>>> a=1.5+0.5j
>>> a.real
1.5
>>> a.imag
0.5
\end{verbatim}

The conversion functions to floating point and integer
(\function{float()}, \function{int()} and \function{long()}) don't
work for complex numbers --- there is no one correct way to convert a
complex number to a real number.  Use \code{abs(\var{z})} to get its
magnitude (as a float) or \code{z.real} to get its real part.

\begin{verbatim}
>>> a=3.0+4.0j
>>> float(a)
Traceback (most recent call last):
  File "<stdin>", line 1, in ?
TypeError: can't convert complex to float; use abs(z)
>>> a.real
3.0
>>> a.imag
4.0
>>> abs(a)  # sqrt(a.real**2 + a.imag**2)
5.0
>>>
\end{verbatim}

In interactive mode, the last printed expression is assigned to the
variable \code{_}.  This means that when you are using Python as a
desk calculator, it is somewhat easier to continue calculations, for
example:

\begin{verbatim}
>>> tax = 12.5 / 100
>>> price = 100.50
>>> price * tax
12.5625
>>> price + _
113.0625
>>> round(_, 2)
113.06
>>>
\end{verbatim}

This variable should be treated as read-only by the user.  Don't
explicitly assign a value to it --- you would create an independent
local variable with the same name masking the built-in variable with
its magic behavior.

\subsection{Strings \label{strings}}

Besides numbers, Python can also manipulate strings, which can be
expressed in several ways.  They can be enclosed in single quotes or
double quotes:

\begin{verbatim}
>>> 'spam eggs'
'spam eggs'
>>> 'doesn\'t'
"doesn't"
>>> "doesn't"
"doesn't"
>>> '"Yes," he said.'
'"Yes," he said.'
>>> "\"Yes,\" he said."
'"Yes," he said.'
>>> '"Isn\'t," she said.'
'"Isn\'t," she said.'
\end{verbatim}

String literals can span multiple lines in several ways.  Continuation
lines can be used, with a backslash as the last character on the line
indicating that the next line is a logical continuation of the line:

\begin{verbatim}
hello = "This is a rather long string containing\n\
several lines of text just as you would do in C.\n\
    Note that whitespace at the beginning of the line is\
 significant."

print hello
\end{verbatim}

Note that newlines would still need to be embedded in the string using
\code{\e n}; the newline following the trailing backslash is
discarded.  This example would print the following:

\begin{verbatim}
This is a rather long string containing
several lines of text just as you would do in C.
    Note that whitespace at the beginning of the line is significant.
\end{verbatim}

If we make the string literal a ``raw'' string, however, the
\code{\e n} sequences are not converted to newlines, but the backslash
at the end of the line, and the newline character in the source, are
both included in the string as data.  Thus, the example:

\begin{verbatim}
hello = r"This is a rather long string containing\n\
several lines of text much as you would do in C."

print hello
\end{verbatim}

would print:

\begin{verbatim}
This is a rather long string containing\n\
several lines of text much as you would do in C.
\end{verbatim}

Or, strings can be surrounded in a pair of matching triple-quotes:
\code{"""} or \code{'\code{'}'}.  End of lines do not need to be escaped
when using triple-quotes, but they will be included in the string.

\begin{verbatim}
print """
Usage: thingy [OPTIONS] 
     -h                        Display this usage message
     -H hostname               Hostname to connect to
"""
\end{verbatim}

produces the following output:

\begin{verbatim}
Usage: thingy [OPTIONS] 
     -h                        Display this usage message
     -H hostname               Hostname to connect to
\end{verbatim}

The interpreter prints the result of string operations in the same way
as they are typed for input: inside quotes, and with quotes and other
funny characters escaped by backslashes, to show the precise
value.  The string is enclosed in double quotes if the string contains
a single quote and no double quotes, else it's enclosed in single
quotes.  (The \keyword{print} statement, described later, can be used
to write strings without quotes or escapes.)

Strings can be concatenated (glued together) with the
\code{+} operator, and repeated with \code{*}:

\begin{verbatim}
>>> word = 'Help' + 'A'
>>> word
'HelpA'
>>> '<' + word*5 + '>'
'<HelpAHelpAHelpAHelpAHelpA>'
\end{verbatim}

Two string literals next to each other are automatically concatenated;
the first line above could also have been written \samp{word = 'Help'
'A'}; this only works with two literals, not with arbitrary string
expressions:

\begin{verbatim}
>>> 'str' 'ing'                   #  <-  This is ok
'string'
>>> 'str'.strip() + 'ing'   #  <-  This is ok
'string'
>>> 'str'.strip() 'ing'     #  <-  This is invalid
  File "<stdin>", line 1, in ?
    'str'.strip() 'ing'
                      ^
SyntaxError: invalid syntax
\end{verbatim}

Strings can be subscripted (indexed); like in C, the first character
of a string has subscript (index) 0.  There is no separate character
type; a character is simply a string of size one.  Like in Icon,
substrings can be specified with the \emph{slice notation}: two indices
separated by a colon.

\begin{verbatim}
>>> word[4]
'A'
>>> word[0:2]
'He'
>>> word[2:4]
'lp'
\end{verbatim}

Slice indices have useful defaults; an omitted first index defaults to
zero, an omitted second index defaults to the size of the string being
sliced.

\begin{verbatim}
>>> word[:2]    # The first two characters
'He'
>>> word[2:]    # Everything except the first two characters
'lpA'
\end{verbatim}

Unlike a C string, Python strings cannot be changed.  Assigning to an 
indexed position in the string results in an error:

\begin{verbatim}
>>> word[0] = 'x'
Traceback (most recent call last):
  File "<stdin>", line 1, in ?
TypeError: object doesn't support item assignment
>>> word[:1] = 'Splat'
Traceback (most recent call last):
  File "<stdin>", line 1, in ?
TypeError: object doesn't support slice assignment
\end{verbatim}

However, creating a new string with the combined content is easy and
efficient:

\begin{verbatim}
>>> 'x' + word[1:]
'xelpA'
>>> 'Splat' + word[4]
'SplatA'
\end{verbatim}

Here's a useful invariant of slice operations:
\code{s[:i] + s[i:]} equals \code{s}.

\begin{verbatim}
>>> word[:2] + word[2:]
'HelpA'
>>> word[:3] + word[3:]
'HelpA'
\end{verbatim}

Degenerate slice indices are handled gracefully: an index that is too
large is replaced by the string size, an upper bound smaller than the
lower bound returns an empty string.

\begin{verbatim}
>>> word[1:100]
'elpA'
>>> word[10:]
''
>>> word[2:1]
''
\end{verbatim}

Indices may be negative numbers, to start counting from the right.
For example:

\begin{verbatim}
>>> word[-1]     # The last character
'A'
>>> word[-2]     # The last-but-one character
'p'
>>> word[-2:]    # The last two characters
'pA'
>>> word[:-2]    # Everything except the last two characters
'Hel'
\end{verbatim}

But note that -0 is really the same as 0, so it does not count from
the right!

\begin{verbatim}
>>> word[-0]     # (since -0 equals 0)
'H'
\end{verbatim}

Out-of-range negative slice indices are truncated, but don't try this
for single-element (non-slice) indices:

\begin{verbatim}
>>> word[-100:]
'HelpA'
>>> word[-10]    # error
Traceback (most recent call last):
  File "<stdin>", line 1, in ?
IndexError: string index out of range
\end{verbatim}

The best way to remember how slices work is to think of the indices as
pointing \emph{between} characters, with the left edge of the first
character numbered 0.  Then the right edge of the last character of a
string of \var{n} characters has index \var{n}, for example:

\begin{verbatim}
 +---+---+---+---+---+ 
 | H | e | l | p | A |
 +---+---+---+---+---+ 
 0   1   2   3   4   5 
-5  -4  -3  -2  -1
\end{verbatim}

The first row of numbers gives the position of the indices 0...5 in
the string; the second row gives the corresponding negative indices.
The slice from \var{i} to \var{j} consists of all characters between
the edges labeled \var{i} and \var{j}, respectively.

For non-negative indices, the length of a slice is the difference of
the indices, if both are within bounds.  For example, the length of
\code{word[1:3]} is 2.

The built-in function \function{len()} returns the length of a string:

\begin{verbatim}
>>> s = 'supercalifragilisticexpialidocious'
>>> len(s)
34
\end{verbatim}


\begin{seealso}
  \seetitle[../lib/typesseq.html]{Sequence Types}%
           {Strings, and the Unicode strings described in the next
            section, are examples of \emph{sequence types}, and
            support the common operations supported by such types.}
  \seetitle[../lib/string-methods.html]{String Methods}%
           {Both strings and Unicode strings support a large number of
            methods for basic transformations and searching.}
  \seetitle[../lib/typesseq-strings.html]{String Formatting Operations}%
           {The formatting operations invoked when strings and Unicode
            strings are the left operand of the \code{\%} operator are
            described in more detail here.}
\end{seealso}


\subsection{Unicode Strings \label{unicodeStrings}}
\sectionauthor{Marc-Andre Lemburg}{mal@lemburg.com}

Starting with Python 2.0 a new data type for storing text data is
available to the programmer: the Unicode object. It can be used to
store and manipulate Unicode data (see \url{http://www.unicode.org/})
and integrates well with the existing string objects providing
auto-conversions where necessary.

Unicode has the advantage of providing one ordinal for every character
in every script used in modern and ancient texts. Previously, there
were only 256 possible ordinals for script characters and texts were
typically bound to a code page which mapped the ordinals to script
characters. This lead to very much confusion especially with respect
to internationalization (usually written as \samp{i18n} ---
\character{i} + 18 characters + \character{n}) of software.  Unicode
solves these problems by defining one code page for all scripts.

Creating Unicode strings in Python is just as simple as creating
normal strings:

\begin{verbatim}
>>> u'Hello World !'
u'Hello World !'
\end{verbatim}

The small \character{u} in front of the quote indicates that an
Unicode string is supposed to be created. If you want to include
special characters in the string, you can do so by using the Python
\emph{Unicode-Escape} encoding. The following example shows how:

\begin{verbatim}
>>> u'Hello\u0020World !'
u'Hello World !'
\end{verbatim}

The escape sequence \code{\e u0020} indicates to insert the Unicode
character with the ordinal value 0x0020 (the space character) at the
given position.

Other characters are interpreted by using their respective ordinal
values directly as Unicode ordinals.  If you have literal strings
in the standard Latin-1 encoding that is used in many Western countries,
you will find it convenient that the lower 256 characters
of Unicode are the same as the 256 characters of Latin-1.

For experts, there is also a raw mode just like the one for normal
strings. You have to prefix the opening quote with 'ur' to have
Python use the \emph{Raw-Unicode-Escape} encoding. It will only apply
the above \code{\e uXXXX} conversion if there is an uneven number of
backslashes in front of the small 'u'.

\begin{verbatim}
>>> ur'Hello\u0020World !'
u'Hello World !'
>>> ur'Hello\\u0020World !'
u'Hello\\\\u0020World !'
\end{verbatim}

The raw mode is most useful when you have to enter lots of
backslashes, as can be necessary in regular expressions.

Apart from these standard encodings, Python provides a whole set of
other ways of creating Unicode strings on the basis of a known
encoding. 

The built-in function \function{unicode()}\bifuncindex{unicode} provides
access to all registered Unicode codecs (COders and DECoders). Some of
the more well known encodings which these codecs can convert are
\emph{Latin-1}, \emph{ASCII}, \emph{UTF-8}, and \emph{UTF-16}.
The latter two are variable-length encodings that store each Unicode
character in one or more bytes. The default encoding is
normally set to \ASCII, which passes through characters in the range
0 to 127 and rejects any other characters with an error.
When a Unicode string is printed, written to a file, or converted
with \function{str()}, conversion takes place using this default encoding.

\begin{verbatim}
>>> u"abc"
u'abc'
>>> str(u"abc")
'abc'
>>> u"���"
u'\xe4\xf6\xfc'
>>> str(u"���")
Traceback (most recent call last):
  File "<stdin>", line 1, in ?
UnicodeEncodeError: 'ascii' codec can't encode characters in position 0-2: ordinal not in range(128)
\end{verbatim}

To convert a Unicode string into an 8-bit string using a specific
encoding, Unicode objects provide an \function{encode()} method
that takes one argument, the name of the encoding.  Lowercase names
for encodings are preferred.

\begin{verbatim}
>>> u"���".encode('utf-8')
'\xc3\xa4\xc3\xb6\xc3\xbc'
\end{verbatim}

If you have data in a specific encoding and want to produce a
corresponding Unicode string from it, you can use the
\function{unicode()} function with the encoding name as the second
argument.

\begin{verbatim}
>>> unicode('\xc3\xa4\xc3\xb6\xc3\xbc', 'utf-8')
u'\xe4\xf6\xfc'
\end{verbatim}

\subsection{Lists \label{lists}}

Python knows a number of \emph{compound} data types, used to group
together other values.  The most versatile is the \emph{list}, which
can be written as a list of comma-separated values (items) between
square brackets.  List items need not all have the same type.

\begin{verbatim}
>>> a = ['spam', 'eggs', 100, 1234]
>>> a
['spam', 'eggs', 100, 1234]
\end{verbatim}

Like string indices, list indices start at 0, and lists can be sliced,
concatenated and so on:

\begin{verbatim}
>>> a[0]
'spam'
>>> a[3]
1234
>>> a[-2]
100
>>> a[1:-1]
['eggs', 100]
>>> a[:2] + ['bacon', 2*2]
['spam', 'eggs', 'bacon', 4]
>>> 3*a[:3] + ['Boe!']
['spam', 'eggs', 100, 'spam', 'eggs', 100, 'spam', 'eggs', 100, 'Boe!']
\end{verbatim}

Unlike strings, which are \emph{immutable}, it is possible to change
individual elements of a list:

\begin{verbatim}
>>> a
['spam', 'eggs', 100, 1234]
>>> a[2] = a[2] + 23
>>> a
['spam', 'eggs', 123, 1234]
\end{verbatim}

Assignment to slices is also possible, and this can even change the size
of the list:

\begin{verbatim}
>>> # Replace some items:
... a[0:2] = [1, 12]
>>> a
[1, 12, 123, 1234]
>>> # Remove some:
... a[0:2] = []
>>> a
[123, 1234]
>>> # Insert some:
... a[1:1] = ['bletch', 'xyzzy']
>>> a
[123, 'bletch', 'xyzzy', 1234]
>>> a[:0] = a     # Insert (a copy of) itself at the beginning
>>> a
[123, 'bletch', 'xyzzy', 1234, 123, 'bletch', 'xyzzy', 1234]
\end{verbatim}

The built-in function \function{len()} also applies to lists:

\begin{verbatim}
>>> len(a)
8
\end{verbatim}

It is possible to nest lists (create lists containing other lists),
for example:

\begin{verbatim}
>>> q = [2, 3]
>>> p = [1, q, 4]
>>> len(p)
3
>>> p[1]
[2, 3]
>>> p[1][0]
2
>>> p[1].append('xtra')     # See section 5.1
>>> p
[1, [2, 3, 'xtra'], 4]
>>> q
[2, 3, 'xtra']
\end{verbatim}

Note that in the last example, \code{p[1]} and \code{q} really refer to
the same object!  We'll come back to \emph{object semantics} later.

\section{First Steps Towards Programming \label{firstSteps}}

Of course, we can use Python for more complicated tasks than adding
two and two together.  For instance, we can write an initial
sub-sequence of the \emph{Fibonacci} series as follows:

\begin{verbatim}
>>> # Fibonacci series:
... # the sum of two elements defines the next
... a, b = 0, 1
>>> while b < 10:
...       print b
...       a, b = b, a+b
... 
1
1
2
3
5
8
\end{verbatim}

This example introduces several new features.

\begin{itemize}

\item
The first line contains a \emph{multiple assignment}: the variables
\code{a} and \code{b} simultaneously get the new values 0 and 1.  On the
last line this is used again, demonstrating that the expressions on
the right-hand side are all evaluated first before any of the
assignments take place.  The right-hand side expressions are evaluated 
from the left to the right.

\item
The \keyword{while} loop executes as long as the condition (here:
\code{b < 10}) remains true.  In Python, like in C, any non-zero
integer value is true; zero is false.  The condition may also be a
string or list value, in fact any sequence; anything with a non-zero
length is true, empty sequences are false.  The test used in the
example is a simple comparison.  The standard comparison operators are
written the same as in C: \code{<} (less than), \code{>} (greater than),
\code{==} (equal to), \code{<=} (less than or equal to),
\code{>=} (greater than or equal to) and \code{!=} (not equal to).

\item
The \emph{body} of the loop is \emph{indented}: indentation is Python's
way of grouping statements.  Python does not (yet!) provide an
intelligent input line editing facility, so you have to type a tab or
space(s) for each indented line.  In practice you will prepare more
complicated input for Python with a text editor; most text editors have
an auto-indent facility.  When a compound statement is entered
interactively, it must be followed by a blank line to indicate
completion (since the parser cannot guess when you have typed the last
line).  Note that each line within a basic block must be indented by
the same amount.

\item
The \keyword{print} statement writes the value of the expression(s) it is
given.  It differs from just writing the expression you want to write
(as we did earlier in the calculator examples) in the way it handles
multiple expressions and strings.  Strings are printed without quotes,
and a space is inserted between items, so you can format things nicely,
like this:

\begin{verbatim}
>>> i = 256*256
>>> print 'The value of i is', i
The value of i is 65536
\end{verbatim}

A trailing comma avoids the newline after the output:

\begin{verbatim}
>>> a, b = 0, 1
>>> while b < 1000:
...     print b,
...     a, b = b, a+b
... 
1 1 2 3 5 8 13 21 34 55 89 144 233 377 610 987
\end{verbatim}

Note that the interpreter inserts a newline before it prints the next
prompt if the last line was not completed.

\end{itemize}


\chapter{More Control Flow Tools \label{moreControl}}

Besides the \keyword{while} statement just introduced, Python knows
the usual control flow statements known from other languages, with
some twists.

\section{\keyword{if} Statements \label{if}}

Perhaps the most well-known statement type is the
\keyword{if} statement.  For example:

\begin{verbatim}
>>> x = int(raw_input("Please enter an integer: "))
>>> if x < 0:
...      x = 0
...      print 'Negative changed to zero'
... elif x == 0:
...      print 'Zero'
... elif x == 1:
...      print 'Single'
... else:
...      print 'More'
... 
\end{verbatim}

There can be zero or more \keyword{elif} parts, and the
\keyword{else} part is optional.  The keyword `\keyword{elif}' is
short for `else if', and is useful to avoid excessive indentation.  An 
\keyword{if} \ldots\ \keyword{elif} \ldots\ \keyword{elif} \ldots\ sequence
%    Weird spacings happen here if the wrapping of the source text
%    gets changed in the wrong way.
is a substitute for the \keyword{switch} or
\keyword{case} statements found in other languages.


\section{\keyword{for} Statements \label{for}}

The \keyword{for}\stindex{for} statement in Python differs a bit from
what you may be used to in C or Pascal.  Rather than always
iterating over an arithmetic progression of numbers (like in Pascal),
or giving the user the ability to define both the iteration step and
halting condition (as C), Python's
\keyword{for}\stindex{for} statement iterates over the items of any
sequence (a list or a string), in the order that they appear in
the sequence.  For example (no pun intended):
% One suggestion was to give a real C example here, but that may only
% serve to confuse non-C programmers.

\begin{verbatim}
>>> # Measure some strings:
... a = ['cat', 'window', 'defenestrate']
>>> for x in a:
...     print x, len(x)
... 
cat 3
window 6
defenestrate 12
\end{verbatim}

It is not safe to modify the sequence being iterated over in the loop
(this can only happen for mutable sequence types, such as lists).  If
you need to modify the list you are iterating over (for example, to
duplicate selected items) you must iterate over a copy.  The slice
notation makes this particularly convenient:

\begin{verbatim}
>>> for x in a[:]: # make a slice copy of the entire list
...    if len(x) > 6: a.insert(0, x)
... 
>>> a
['defenestrate', 'cat', 'window', 'defenestrate']
\end{verbatim}


\section{The \function{range()} Function \label{range}}

If you do need to iterate over a sequence of numbers, the built-in
function \function{range()} comes in handy.  It generates lists
containing arithmetic progressions:

\begin{verbatim}
>>> range(10)
[0, 1, 2, 3, 4, 5, 6, 7, 8, 9]
\end{verbatim}

The given end point is never part of the generated list;
\code{range(10)} generates a list of 10 values, exactly the legal
indices for items of a sequence of length 10.  It is possible to let
the range start at another number, or to specify a different increment
(even negative; sometimes this is called the `step'):

\begin{verbatim}
>>> range(5, 10)
[5, 6, 7, 8, 9]
>>> range(0, 10, 3)
[0, 3, 6, 9]
>>> range(-10, -100, -30)
[-10, -40, -70]
\end{verbatim}

To iterate over the indices of a sequence, combine
\function{range()} and \function{len()} as follows:

\begin{verbatim}
>>> a = ['Mary', 'had', 'a', 'little', 'lamb']
>>> for i in range(len(a)):
...     print i, a[i]
... 
0 Mary
1 had
2 a
3 little
4 lamb
\end{verbatim}


\section{\keyword{break} and \keyword{continue} Statements, and
         \keyword{else} Clauses on Loops
         \label{break}}

The \keyword{break} statement, like in C, breaks out of the smallest
enclosing \keyword{for} or \keyword{while} loop.

The \keyword{continue} statement, also borrowed from C, continues
with the next iteration of the loop.

Loop statements may have an \code{else} clause; it is executed when
the loop terminates through exhaustion of the list (with
\keyword{for}) or when the condition becomes false (with
\keyword{while}), but not when the loop is terminated by a
\keyword{break} statement.  This is exemplified by the following loop,
which searches for prime numbers:

\begin{verbatim}
>>> for n in range(2, 10):
...     for x in range(2, n):
...         if n % x == 0:
...             print n, 'equals', x, '*', n/x
...             break
...     else:
...         # loop fell through without finding a factor
...         print n, 'is a prime number'
... 
2 is a prime number
3 is a prime number
4 equals 2 * 2
5 is a prime number
6 equals 2 * 3
7 is a prime number
8 equals 2 * 4
9 equals 3 * 3
\end{verbatim}


\section{\keyword{pass} Statements \label{pass}}

The \keyword{pass} statement does nothing.
It can be used when a statement is required syntactically but the
program requires no action.
For example:

\begin{verbatim}
>>> while True:
...       pass # Busy-wait for keyboard interrupt
... 
\end{verbatim}


\section{Defining Functions \label{functions}}

We can create a function that writes the Fibonacci series to an
arbitrary boundary:

\begin{verbatim}
>>> def fib(n):    # write Fibonacci series up to n
...     """Print a Fibonacci series up to n."""
...     a, b = 0, 1
...     while b < n:
...         print b,
...         a, b = b, a+b
... 
>>> # Now call the function we just defined:
... fib(2000)
1 1 2 3 5 8 13 21 34 55 89 144 233 377 610 987 1597
\end{verbatim}

The keyword \keyword{def} introduces a function \emph{definition}.  It
must be followed by the function name and the parenthesized list of
formal parameters.  The statements that form the body of the function
start at the next line, and must be indented.  The first statement of
the function body can optionally be a string literal; this string
literal is the function's \index{documentation strings}documentation
string, or \dfn{docstring}.\index{docstrings}\index{strings, documentation}

There are tools which use docstrings to automatically produce online
or printed documentation, or to let the user interactively browse
through code; it's good practice to include docstrings in code that
you write, so try to make a habit of it.

The \emph{execution} of a function introduces a new symbol table used
for the local variables of the function.  More precisely, all variable
assignments in a function store the value in the local symbol table;
whereas variable references first look in the local symbol table, then
in the global symbol table, and then in the table of built-in names.
Thus,  global variables cannot be directly assigned a value within a
function (unless named in a \keyword{global} statement), although
they may be referenced.

The actual parameters (arguments) to a function call are introduced in
the local symbol table of the called function when it is called; thus,
arguments are passed using \emph{call by value} (where the
\emph{value} is always an object \emph{reference}, not the value of
the object).\footnote{
         Actually, \emph{call by object reference} would be a better
         description, since if a mutable object is passed, the caller
         will see any changes the callee makes to it (items
         inserted into a list).
} When a function calls another function, a new local symbol table is
created for that call.

A function definition introduces the function name in the current
symbol table.  The value of the function name
has a type that is recognized by the interpreter as a user-defined
function.  This value can be assigned to another name which can then
also be used as a function.  This serves as a general renaming
mechanism:

\begin{verbatim}
>>> fib
<function fib at 10042ed0>
>>> f = fib
>>> f(100)
1 1 2 3 5 8 13 21 34 55 89
\end{verbatim}

You might object that \code{fib} is not a function but a procedure.  In
Python, like in C, procedures are just functions that don't return a
value.  In fact, technically speaking, procedures do return a value,
albeit a rather boring one.  This value is called \code{None} (it's a
built-in name).  Writing the value \code{None} is normally suppressed by
the interpreter if it would be the only value written.  You can see it
if you really want to:

\begin{verbatim}
>>> print fib(0)
None
\end{verbatim}

It is simple to write a function that returns a list of the numbers of
the Fibonacci series, instead of printing it:

\begin{verbatim}
>>> def fib2(n): # return Fibonacci series up to n
...     """Return a list containing the Fibonacci series up to n."""
...     result = []
...     a, b = 0, 1
...     while b < n:
...         result.append(b)    # see below
...         a, b = b, a+b
...     return result
... 
>>> f100 = fib2(100)    # call it
>>> f100                # write the result
[1, 1, 2, 3, 5, 8, 13, 21, 34, 55, 89]
\end{verbatim}

This example, as usual, demonstrates some new Python features:

\begin{itemize}

\item
The \keyword{return} statement returns with a value from a function.
\keyword{return} without an expression argument returns \code{None}.
Falling off the end of a procedure also returns \code{None}.

\item
The statement \code{result.append(b)} calls a \emph{method} of the list
object \code{result}.  A method is a function that `belongs' to an
object and is named \code{obj.methodname}, where \code{obj} is some
object (this may be an expression), and \code{methodname} is the name
of a method that is defined by the object's type.  Different types
define different methods.  Methods of different types may have the
same name without causing ambiguity.  (It is possible to define your
own object types and methods, using \emph{classes}, as discussed later
in this tutorial.)
The method \method{append()} shown in the example, is defined for
list objects; it adds a new element at the end of the list.  In this
example it is equivalent to \samp{result = result + [b]}, but more
efficient.

\end{itemize}

\section{More on Defining Functions \label{defining}}

It is also possible to define functions with a variable number of
arguments.  There are three forms, which can be combined.

\subsection{Default Argument Values \label{defaultArgs}}

The most useful form is to specify a default value for one or more
arguments.  This creates a function that can be called with fewer
arguments than it is defined to allow.  For example:

\begin{verbatim}
def ask_ok(prompt, retries=4, complaint='Yes or no, please!'):
    while True:
        ok = raw_input(prompt)
        if ok in ('y', 'ye', 'yes'): return True
        if ok in ('n', 'no', 'nop', 'nope'): return False
        retries = retries - 1
        if retries < 0: raise IOError, 'refusenik user'
        print complaint
\end{verbatim}

This function can be called either like this:
\code{ask_ok('Do you really want to quit?')} or like this:
\code{ask_ok('OK to overwrite the file?', 2)}.

This example also introduces the \keyword{in} keyword. This tests
whether or not a sequence contains a certain value.

The default values are evaluated at the point of function definition
in the \emph{defining} scope, so that

\begin{verbatim}
i = 5

def f(arg=i):
    print arg

i = 6
f()
\end{verbatim}

will print \code{5}.

\strong{Important warning:}  The default value is evaluated only once.
This makes a difference when the default is a mutable object such as a
list, dictionary, or instances of most classes.  For example, the
following function accumulates the arguments passed to it on
subsequent calls:

\begin{verbatim}
def f(a, L=[]):
    L.append(a)
    return L

print f(1)
print f(2)
print f(3)
\end{verbatim}

This will print

\begin{verbatim}
[1]
[1, 2]
[1, 2, 3]
\end{verbatim}

If you don't want the default to be shared between subsequent calls,
you can write the function like this instead:

\begin{verbatim}
def f(a, L=None):
    if L is None:
        L = []
    L.append(a)
    return L
\end{verbatim}

\subsection{Keyword Arguments \label{keywordArgs}}

Functions can also be called using
keyword arguments of the form \samp{\var{keyword} = \var{value}}.  For
instance, the following function:

\begin{verbatim}
def parrot(voltage, state='a stiff', action='voom', type='Norwegian Blue'):
    print "-- This parrot wouldn't", action,
    print "if you put", voltage, "Volts through it."
    print "-- Lovely plumage, the", type
    print "-- It's", state, "!"
\end{verbatim}

could be called in any of the following ways:

\begin{verbatim}
parrot(1000)
parrot(action = 'VOOOOOM', voltage = 1000000)
parrot('a thousand', state = 'pushing up the daisies')
parrot('a million', 'bereft of life', 'jump')
\end{verbatim}

but the following calls would all be invalid:

\begin{verbatim}
parrot()                     # required argument missing
parrot(voltage=5.0, 'dead')  # non-keyword argument following keyword
parrot(110, voltage=220)     # duplicate value for argument
parrot(actor='John Cleese')  # unknown keyword
\end{verbatim}

In general, an argument list must have any positional arguments
followed by any keyword arguments, where the keywords must be chosen
from the formal parameter names.  It's not important whether a formal
parameter has a default value or not.  No argument may receive a
value more than once --- formal parameter names corresponding to
positional arguments cannot be used as keywords in the same calls.
Here's an example that fails due to this restriction:

\begin{verbatim}
>>> def function(a):
...     pass
... 
>>> function(0, a=0)
Traceback (most recent call last):
  File "<stdin>", line 1, in ?
TypeError: function() got multiple values for keyword argument 'a'
\end{verbatim}

When a final formal parameter of the form \code{**\var{name}} is
present, it receives a \ulink{dictionary}{../lib/typesmapping.html} containing all keyword arguments
whose keyword doesn't correspond to a formal parameter.  This may be
combined with a formal parameter of the form
\code{*\var{name}} (described in the next subsection) which receives a
tuple containing the positional arguments beyond the formal parameter
list.  (\code{*\var{name}} must occur before \code{**\var{name}}.)
For example, if we define a function like this:

\begin{verbatim}
def cheeseshop(kind, *arguments, **keywords):
    print "-- Do you have any", kind, '?'
    print "-- I'm sorry, we're all out of", kind
    for arg in arguments: print arg
    print '-'*40
    keys = keywords.keys()
    keys.sort()
    for kw in keys: print kw, ':', keywords[kw]
\end{verbatim}

It could be called like this:

\begin{verbatim}
cheeseshop('Limburger', "It's very runny, sir.",
           "It's really very, VERY runny, sir.",
           client='John Cleese',
           shopkeeper='Michael Palin',
           sketch='Cheese Shop Sketch')
\end{verbatim}

and of course it would print:

\begin{verbatim}
-- Do you have any Limburger ?
-- I'm sorry, we're all out of Limburger
It's very runny, sir.
It's really very, VERY runny, sir.
----------------------------------------
client : John Cleese
shopkeeper : Michael Palin
sketch : Cheese Shop Sketch
\end{verbatim}

Note that the \method{sort()} method of the list of keyword argument
names is called before printing the contents of the \code{keywords}
dictionary; if this is not done, the order in which the arguments are
printed is undefined.


\subsection{Arbitrary Argument Lists \label{arbitraryArgs}}

Finally, the least frequently used option is to specify that a
function can be called with an arbitrary number of arguments.  These
arguments will be wrapped up in a tuple.  Before the variable number
of arguments, zero or more normal arguments may occur.

\begin{verbatim}
def fprintf(file, format, *args):
    file.write(format % args)
\end{verbatim}


\subsection{Unpacking Argument Lists \label{unpacking-arguments}}

The reverse situation occurs when the arguments are already in a list
or tuple but need to be unpacked for a function call requiring separate
positional arguments.  For instance, the built-in \function{range()}
function expects separate \var{start} and \var{stop} arguments.  If they
are not available separately, write the function call with the 
\code{*}-operator to unpack the arguments out of a list or tuple:

\begin{verbatim}
>>> range(3, 6)             # normal call with separate arguments
[3, 4, 5]
>>> args = [3, 6]
>>> range(*args)            # call with arguments unpacked from a list
[3, 4, 5]
\end{verbatim}


\subsection{Lambda Forms \label{lambda}}

By popular demand, a few features commonly found in functional
programming languages and Lisp have been added to Python.  With the
\keyword{lambda} keyword, small anonymous functions can be created.
Here's a function that returns the sum of its two arguments:
\samp{lambda a, b: a+b}.  Lambda forms can be used wherever function
objects are required.  They are syntactically restricted to a single
expression.  Semantically, they are just syntactic sugar for a normal
function definition.  Like nested function definitions, lambda forms
can reference variables from the containing scope:

\begin{verbatim}
>>> def make_incrementor(n):
...     return lambda x: x + n
...
>>> f = make_incrementor(42)
>>> f(0)
42
>>> f(1)
43
\end{verbatim}


\subsection{Documentation Strings \label{docstrings}}

There are emerging conventions about the content and formatting of
documentation strings.
\index{docstrings}\index{documentation strings}
\index{strings, documentation}

The first line should always be a short, concise summary of the
object's purpose.  For brevity, it should not explicitly state the
object's name or type, since these are available by other means
(except if the name happens to be a verb describing a function's
operation).  This line should begin with a capital letter and end with
a period.

If there are more lines in the documentation string, the second line
should be blank, visually separating the summary from the rest of the
description.  The following lines should be one or more paragraphs
describing the object's calling conventions, its side effects, etc.

The Python parser does not strip indentation from multi-line string
literals in Python, so tools that process documentation have to strip
indentation if desired.  This is done using the following convention.
The first non-blank line \emph{after} the first line of the string
determines the amount of indentation for the entire documentation
string.  (We can't use the first line since it is generally adjacent
to the string's opening quotes so its indentation is not apparent in
the string literal.)  Whitespace ``equivalent'' to this indentation is
then stripped from the start of all lines of the string.  Lines that
are indented less should not occur, but if they occur all their
leading whitespace should be stripped.  Equivalence of whitespace
should be tested after expansion of tabs (to 8 spaces, normally).

Here is an example of a multi-line docstring:

\begin{verbatim}
>>> def my_function():
...     """Do nothing, but document it.
... 
...     No, really, it doesn't do anything.
...     """
...     pass
... 
>>> print my_function.__doc__
Do nothing, but document it.

    No, really, it doesn't do anything.
    
\end{verbatim}



\chapter{Data Structures \label{structures}}

This chapter describes some things you've learned about already in
more detail, and adds some new things as well.


\section{More on Lists \label{moreLists}}

The list data type has some more methods.  Here are all of the methods
of list objects:

\begin{methoddesc}[list]{append}{x}
Add an item to the end of the list;
equivalent to \code{a[len(a):] = [\var{x}]}.
\end{methoddesc}

\begin{methoddesc}[list]{extend}{L}
Extend the list by appending all the items in the given list;
equivalent to \code{a[len(a):] = \var{L}}.
\end{methoddesc}

\begin{methoddesc}[list]{insert}{i, x}
Insert an item at a given position.  The first argument is the index
of the element before which to insert, so \code{a.insert(0, \var{x})}
inserts at the front of the list, and \code{a.insert(len(a), \var{x})}
is equivalent to \code{a.append(\var{x})}.
\end{methoddesc}

\begin{methoddesc}[list]{remove}{x}
Remove the first item from the list whose value is \var{x}.
It is an error if there is no such item.
\end{methoddesc}

\begin{methoddesc}[list]{pop}{\optional{i}}
Remove the item at the given position in the list, and return it.  If
no index is specified, \code{a.pop()} returns the last item in the
list.  The item is also removed from the list.  (The square brackets
around the \var{i} in the method signature denote that the parameter
is optional, not that you should type square brackets at that
position.  You will see this notation frequently in the
\citetitle[../lib/lib.html]{Python Library Reference}.)
\end{methoddesc}

\begin{methoddesc}[list]{index}{x}
Return the index in the list of the first item whose value is \var{x}.
It is an error if there is no such item.
\end{methoddesc}

\begin{methoddesc}[list]{count}{x}
Return the number of times \var{x} appears in the list.
\end{methoddesc}

\begin{methoddesc}[list]{sort}{}
Sort the items of the list, in place.
\end{methoddesc}

\begin{methoddesc}[list]{reverse}{}
Reverse the elements of the list, in place.
\end{methoddesc}

An example that uses most of the list methods:

\begin{verbatim}
>>> a = [66.6, 333, 333, 1, 1234.5]
>>> print a.count(333), a.count(66.6), a.count('x')
2 1 0
>>> a.insert(2, -1)
>>> a.append(333)
>>> a
[66.6, 333, -1, 333, 1, 1234.5, 333]
>>> a.index(333)
1
>>> a.remove(333)
>>> a
[66.6, -1, 333, 1, 1234.5, 333]
>>> a.reverse()
>>> a
[333, 1234.5, 1, 333, -1, 66.6]
>>> a.sort()
>>> a
[-1, 1, 66.6, 333, 333, 1234.5]
\end{verbatim}


\subsection{Using Lists as Stacks \label{lists-as-stacks}}
\sectionauthor{Ka-Ping Yee}{ping@lfw.org}

The list methods make it very easy to use a list as a stack, where the
last element added is the first element retrieved (``last-in,
first-out'').  To add an item to the top of the stack, use
\method{append()}.  To retrieve an item from the top of the stack, use
\method{pop()} without an explicit index.  For example:

\begin{verbatim}
>>> stack = [3, 4, 5]
>>> stack.append(6)
>>> stack.append(7)
>>> stack
[3, 4, 5, 6, 7]
>>> stack.pop()
7
>>> stack
[3, 4, 5, 6]
>>> stack.pop()
6
>>> stack.pop()
5
>>> stack
[3, 4]
\end{verbatim}


\subsection{Using Lists as Queues \label{lists-as-queues}}
\sectionauthor{Ka-Ping Yee}{ping@lfw.org}

You can also use a list conveniently as a queue, where the first
element added is the first element retrieved (``first-in,
first-out'').  To add an item to the back of the queue, use
\method{append()}.  To retrieve an item from the front of the queue,
use \method{pop()} with \code{0} as the index.  For example:

\begin{verbatim}
>>> queue = ["Eric", "John", "Michael"]
>>> queue.append("Terry")           # Terry arrives
>>> queue.append("Graham")          # Graham arrives
>>> queue.pop(0)
'Eric'
>>> queue.pop(0)
'John'
>>> queue
['Michael', 'Terry', 'Graham']
\end{verbatim}


\subsection{Functional Programming Tools \label{functional}}

There are three built-in functions that are very useful when used with
lists: \function{filter()}, \function{map()}, and \function{reduce()}.

\samp{filter(\var{function}, \var{sequence})} returns a sequence (of
the same type, if possible) consisting of those items from the
sequence for which \code{\var{function}(\var{item})} is true.  For
example, to compute some primes:

\begin{verbatim}
>>> def f(x): return x % 2 != 0 and x % 3 != 0
...
>>> filter(f, range(2, 25))
[5, 7, 11, 13, 17, 19, 23]
\end{verbatim}

\samp{map(\var{function}, \var{sequence})} calls
\code{\var{function}(\var{item})} for each of the sequence's items and
returns a list of the return values.  For example, to compute some
cubes:

\begin{verbatim}
>>> def cube(x): return x*x*x
...
>>> map(cube, range(1, 11))
[1, 8, 27, 64, 125, 216, 343, 512, 729, 1000]
\end{verbatim}

More than one sequence may be passed; the function must then have as
many arguments as there are sequences and is called with the
corresponding item from each sequence (or \code{None} if some sequence
is shorter than another).  For example:

\begin{verbatim}
>>> seq = range(8)
>>> def add(x, y): return x+y
...
>>> map(add, seq, seq)
[0, 2, 4, 6, 8, 10, 12, 14]
\end{verbatim}

\samp{reduce(\var{func}, \var{sequence})} returns a single value
constructed by calling the binary function \var{func} on the first two
items of the sequence, then on the result and the next item, and so
on.  For example, to compute the sum of the numbers 1 through 10:

\begin{verbatim}
>>> def add(x,y): return x+y
...
>>> reduce(add, range(1, 11))
55
\end{verbatim}

If there's only one item in the sequence, its value is returned; if
the sequence is empty, an exception is raised.

A third argument can be passed to indicate the starting value.  In this
case the starting value is returned for an empty sequence, and the
function is first applied to the starting value and the first sequence
item, then to the result and the next item, and so on.  For example,

\begin{verbatim}
>>> def sum(seq):
...     def add(x,y): return x+y
...     return reduce(add, seq, 0)
... 
>>> sum(range(1, 11))
55
>>> sum([])
0
\end{verbatim}

Don't use this example's definition of \function{sum()}: since summing
numbers is such a common need, a built-in function
\code{sum(\var{sequence})} is already provided, and works exactly like
this.
\versionadded{2.3}

\subsection{List Comprehensions}

List comprehensions provide a concise way to create lists without resorting
to use of \function{map()}, \function{filter()} and/or \keyword{lambda}.
The resulting list definition tends often to be clearer than lists built
using those constructs.  Each list comprehension consists of an expression
followed by a \keyword{for} clause, then zero or more \keyword{for} or
\keyword{if} clauses.  The result will be a list resulting from evaluating
the expression in the context of the \keyword{for} and \keyword{if} clauses
which follow it.  If the expression would evaluate to a tuple, it must be
parenthesized.

\begin{verbatim}
>>> freshfruit = ['  banana', '  loganberry ', 'passion fruit  ']
>>> [weapon.strip() for weapon in freshfruit]
['banana', 'loganberry', 'passion fruit']
>>> vec = [2, 4, 6]
>>> [3*x for x in vec]
[6, 12, 18]
>>> [3*x for x in vec if x > 3]
[12, 18]
>>> [3*x for x in vec if x < 2]
[]
>>> [[x,x**2] for x in vec]
[[2, 4], [4, 16], [6, 36]]
>>> [x, x**2 for x in vec]	# error - parens required for tuples
  File "<stdin>", line 1, in ?
    [x, x**2 for x in vec]
               ^
SyntaxError: invalid syntax
>>> [(x, x**2) for x in vec]
[(2, 4), (4, 16), (6, 36)]
>>> vec1 = [2, 4, 6]
>>> vec2 = [4, 3, -9]
>>> [x*y for x in vec1 for y in vec2]
[8, 6, -18, 16, 12, -36, 24, 18, -54]
>>> [x+y for x in vec1 for y in vec2]
[6, 5, -7, 8, 7, -5, 10, 9, -3]
>>> [vec1[i]*vec2[i] for i in range(len(vec1))]
[8, 12, -54]
\end{verbatim}

List comprehensions are much more flexible than \function{map()} and can be
applied to functions with more than one argument and to nested functions:

\begin{verbatim}
>>> [str(round(355/113.0, i)) for i in range(1,6)]
['3.1', '3.14', '3.142', '3.1416', '3.14159']
\end{verbatim}


\section{The \keyword{del} statement \label{del}}

There is a way to remove an item from a list given its index instead
of its value: the \keyword{del} statement.  This can also be used to
remove slices from a list (which we did earlier by assignment of an
empty list to the slice).  For example:

\begin{verbatim}
>>> a = [-1, 1, 66.6, 333, 333, 1234.5]
>>> del a[0]
>>> a
[1, 66.6, 333, 333, 1234.5]
>>> del a[2:4]
>>> a
[1, 66.6, 1234.5]
\end{verbatim}

\keyword{del} can also be used to delete entire variables:

\begin{verbatim}
>>> del a
\end{verbatim}

Referencing the name \code{a} hereafter is an error (at least until
another value is assigned to it).  We'll find other uses for
\keyword{del} later.


\section{Tuples and Sequences \label{tuples}}

We saw that lists and strings have many common properties, such as
indexing and slicing operations.  They are two examples of
\ulink{\emph{sequence} data types}{../lib/typesseq.html}.  Since
Python is an evolving language, other sequence data types may be
added.  There is also another standard sequence data type: the
\emph{tuple}.

A tuple consists of a number of values separated by commas, for
instance:

\begin{verbatim}
>>> t = 12345, 54321, 'hello!'
>>> t[0]
12345
>>> t
(12345, 54321, 'hello!')
>>> # Tuples may be nested:
... u = t, (1, 2, 3, 4, 5)
>>> u
((12345, 54321, 'hello!'), (1, 2, 3, 4, 5))
\end{verbatim}

As you see, on output tuples are alway enclosed in parentheses, so
that nested tuples are interpreted correctly; they may be input with
or without surrounding parentheses, although often parentheses are
necessary anyway (if the tuple is part of a larger expression).

Tuples have many uses.  For example: (x, y) coordinate pairs, employee
records from a database, etc.  Tuples, like strings, are immutable: it
is not possible to assign to the individual items of a tuple (you can
simulate much of the same effect with slicing and concatenation,
though).  It is also possible to create tuples which contain mutable
objects, such as lists.

A special problem is the construction of tuples containing 0 or 1
items: the syntax has some extra quirks to accommodate these.  Empty
tuples are constructed by an empty pair of parentheses; a tuple with
one item is constructed by following a value with a comma
(it is not sufficient to enclose a single value in parentheses).
Ugly, but effective.  For example:

\begin{verbatim}
>>> empty = ()
>>> singleton = 'hello',    # <-- note trailing comma
>>> len(empty)
0
>>> len(singleton)
1
>>> singleton
('hello',)
\end{verbatim}

The statement \code{t = 12345, 54321, 'hello!'} is an example of
\emph{tuple packing}: the values \code{12345}, \code{54321} and
\code{'hello!'} are packed together in a tuple.  The reverse operation
is also possible:

\begin{verbatim}
>>> x, y, z = t
\end{verbatim}

This is called, appropriately enough, \emph{sequence unpacking}.
Sequence unpacking requires that the list of variables on the left
have the same number of elements as the length of the sequence.  Note
that multiple assignment is really just a combination of tuple packing
and sequence unpacking!

There is a small bit of asymmetry here:  packing multiple values
always creates a tuple, and unpacking works for any sequence.

% XXX Add a bit on the difference between tuples and lists.


\section{Sets \label{sets}}

Python also includes a data type for \emph{sets}.  A set is an unordered
collection with no duplicate elements.  Basic uses include membership
testing and eliminating duplicate entries.  Set objects also support
mathematical operations like union, intersection, difference, and
symmetric difference.

Here is a brief demonstration:

\begin{verbatim}
>>> basket = ['apple', 'orange', 'apple', 'pear', 'orange', 'banana']
>>> fruits = set(basket)               # create a set without duplicates
>>> fruits
set(['orange', 'pear', 'apple', 'banana'])
>>> 'orange' in fruits                 # fast membership testing
True
>>> 'crabgrass' in fruits
False

>>> # Demonstrate set operations on unique letters from two words
...
>>> a = set('abracadabra')
>>> b = set('alacazam')
>>> a                                  # unique letters in a
set(['a', 'r', 'b', 'c', 'd'])
>>> a - b                              # letters in a but not in b
set(['r', 'd', 'b'])
>>> a | b                              # letters in either a or b
set(['a', 'c', 'r', 'd', 'b', 'm', 'z', 'l'])
>>> a & b                              # letters in both a and b
set(['a', 'c'])
>>> a ^ b                              # letters in a or b but not both
set(['r', 'd', 'b', 'm', 'z', 'l'])
\end{verbatim}


\section{Dictionaries \label{dictionaries}}

Another useful data type built into Python is the
\ulink{\emph{dictionary}}{../lib/typesmapping.html}.
Dictionaries are sometimes found in other languages as ``associative
memories'' or ``associative arrays''.  Unlike sequences, which are
indexed by a range of numbers, dictionaries are indexed by \emph{keys},
which can be any immutable type; strings and numbers can always be
keys.  Tuples can be used as keys if they contain only strings,
numbers, or tuples; if a tuple contains any mutable object either
directly or indirectly, it cannot be used as a key.  You can't use
lists as keys, since lists can be modified in place using their
\method{append()} and \method{extend()} methods, as well as slice and
indexed assignments.

It is best to think of a dictionary as an unordered set of
\emph{key: value} pairs, with the requirement that the keys are unique
(within one dictionary).
A pair of braces creates an empty dictionary: \code{\{\}}.
Placing a comma-separated list of key:value pairs within the
braces adds initial key:value pairs to the dictionary; this is also the
way dictionaries are written on output.

The main operations on a dictionary are storing a value with some key
and extracting the value given the key.  It is also possible to delete
a key:value pair
with \code{del}.
If you store using a key that is already in use, the old value
associated with that key is forgotten.  It is an error to extract a
value using a non-existent key.

The \method{keys()} method of a dictionary object returns a list of all
the keys used in the dictionary, in arbitrary order (if you want it
sorted, just apply the \method{sort()} method to the list of keys).  To
check whether a single key is in the dictionary, use the
\method{has_key()} method of the dictionary.

Here is a small example using a dictionary:

\begin{verbatim}
>>> tel = {'jack': 4098, 'sape': 4139}
>>> tel['guido'] = 4127
>>> tel
{'sape': 4139, 'guido': 4127, 'jack': 4098}
>>> tel['jack']
4098
>>> del tel['sape']
>>> tel['irv'] = 4127
>>> tel
{'guido': 4127, 'irv': 4127, 'jack': 4098}
>>> tel.keys()
['guido', 'irv', 'jack']
>>> tel.has_key('guido')
True
\end{verbatim}

The \function{dict()} constructor builds dictionaries directly from
lists of key-value pairs stored as tuples.  When the pairs form a
pattern, list comprehensions can compactly specify the key-value list.

\begin{verbatim}
>>> dict([('sape', 4139), ('guido', 4127), ('jack', 4098)])
{'sape': 4139, 'jack': 4098, 'guido': 4127}
>>> dict([(x, x**2) for x in vec])     # use a list comprehension
{2: 4, 4: 16, 6: 36}
\end{verbatim}


\section{Looping Techniques \label{loopidioms}}

When looping through dictionaries, the key and corresponding value can
be retrieved at the same time using the \method{iteritems()} method.

\begin{verbatim}
>>> knights = {'gallahad': 'the pure', 'robin': 'the brave'}
>>> for k, v in knights.iteritems():
...     print k, v
...
gallahad the pure
robin the brave
\end{verbatim}
 
When looping through a sequence, the position index and corresponding
value can be retrieved at the same time using the
\function{enumerate()} function.

\begin{verbatim} 
>>> for i, v in enumerate(['tic', 'tac', 'toe']):
...     print i, v
...
0 tic
1 tac
2 toe
\end{verbatim}

To loop over two or more sequences at the same time, the entries
can be paired with the \function{zip()} function.

\begin{verbatim}
>>> questions = ['name', 'quest', 'favorite color']
>>> answers = ['lancelot', 'the holy grail', 'blue']
>>> for q, a in zip(questions, answers):
...     print 'What is your %s?  It is %s.' % (q, a)
...	
What is your name?  It is lancelot.
What is your quest?  It is the holy grail.
What is your favorite color?  It is blue.
\end{verbatim}

To loop over a sequence in reverse, first specify the sequence
in a forward direction and then call the \function{reversed()}
function.

\begin{verbatim}
>>> for i in reversed(xrange(1,10,2)):
...     print i
...
9
7
5
3
1
\end{verbatim}

To loop over a sequence in sorted order, use the \function{sorted()}
function which returns a new sorted list while leaving the source
unaltered.

\begin{verbatim}
>>> basket = ['apple', 'orange', 'apple', 'pear', 'orange', 'banana']
>>> for f in sorted(set(basket)):
...     print f
... 	
apple
banana
orange
pear
\end{verbatim}

\section{More on Conditions \label{conditions}}

The conditions used in \code{while} and \code{if} statements can
contain any operators, not just comparisons.

The comparison operators \code{in} and \code{not in} check whether a value
occurs (does not occur) in a sequence.  The operators \code{is} and
\code{is not} compare whether two objects are really the same object; this
only matters for mutable objects like lists.  All comparison operators
have the same priority, which is lower than that of all numerical
operators.

Comparisons can be chained.  For example, \code{a < b == c} tests
whether \code{a} is less than \code{b} and moreover \code{b} equals
\code{c}.

Comparisons may be combined by the Boolean operators \code{and} and
\code{or}, and the outcome of a comparison (or of any other Boolean
expression) may be negated with \code{not}.  These have lower
priorities than comparison operators; between them, \code{not} has
the highest priority and \code{or} the lowest, so that
\code{A and not B or C} is equivalent to \code{(A and (not B)) or C}.
As always, parentheses can be used to express the desired composition.

The Boolean operators \code{and} and \code{or} are so-called
\emph{short-circuit} operators: their arguments are evaluated from
left to right, and evaluation stops as soon as the outcome is
determined.  For example, if \code{A} and \code{C} are true but
\code{B} is false, \code{A and B and C} does not evaluate the
expression \code{C}.  In general, the return value of a short-circuit
operator, when used as a general value and not as a Boolean, is the
last evaluated argument.

It is possible to assign the result of a comparison or other Boolean
expression to a variable.  For example,

\begin{verbatim}
>>> string1, string2, string3 = '', 'Trondheim', 'Hammer Dance'
>>> non_null = string1 or string2 or string3
>>> non_null
'Trondheim'
\end{verbatim}

Note that in Python, unlike C, assignment cannot occur inside expressions.
C programmers may grumble about this, but it avoids a common class of
problems encountered in C programs: typing \code{=} in an expression when
\code{==} was intended.


\section{Comparing Sequences and Other Types \label{comparing}}

Sequence objects may be compared to other objects with the same
sequence type.  The comparison uses \emph{lexicographical} ordering:
first the first two items are compared, and if they differ this
determines the outcome of the comparison; if they are equal, the next
two items are compared, and so on, until either sequence is exhausted.
If two items to be compared are themselves sequences of the same type,
the lexicographical comparison is carried out recursively.  If all
items of two sequences compare equal, the sequences are considered
equal.  If one sequence is an initial sub-sequence of the other, the
shorter sequence is the smaller (lesser) one.  Lexicographical
ordering for strings uses the \ASCII{} ordering for individual
characters.  Some examples of comparisons between sequences with the
same types:

\begin{verbatim}
(1, 2, 3)              < (1, 2, 4)
[1, 2, 3]              < [1, 2, 4]
'ABC' < 'C' < 'Pascal' < 'Python'
(1, 2, 3, 4)           < (1, 2, 4)
(1, 2)                 < (1, 2, -1)
(1, 2, 3)             == (1.0, 2.0, 3.0)
(1, 2, ('aa', 'ab'))   < (1, 2, ('abc', 'a'), 4)
\end{verbatim}

Note that comparing objects of different types is legal.  The outcome
is deterministic but arbitrary: the types are ordered by their name.
Thus, a list is always smaller than a string, a string is always
smaller than a tuple, etc.  \footnote{
        The rules for comparing objects of different types should
        not be relied upon; they may change in a future version of
        the language.
} Mixed numeric types are compared according to their numeric value, so
0 equals 0.0, etc.


\chapter{Modules \label{modules}}

If you quit from the Python interpreter and enter it again, the
definitions you have made (functions and variables) are lost.
Therefore, if you want to write a somewhat longer program, you are
better off using a text editor to prepare the input for the interpreter
and running it with that file as input instead.  This is known as creating a
\emph{script}.  As your program gets longer, you may want to split it
into several files for easier maintenance.  You may also want to use a
handy function that you've written in several programs without copying
its definition into each program.

To support this, Python has a way to put definitions in a file and use
them in a script or in an interactive instance of the interpreter.
Such a file is called a \emph{module}; definitions from a module can be
\emph{imported} into other modules or into the \emph{main} module (the
collection of variables that you have access to in a script
executed at the top level
and in calculator mode).

A module is a file containing Python definitions and statements.  The
file name is the module name with the suffix \file{.py} appended.  Within
a module, the module's name (as a string) is available as the value of
the global variable \code{__name__}.  For instance, use your favorite text
editor to create a file called \file{fibo.py} in the current directory
with the following contents:

\begin{verbatim}
# Fibonacci numbers module

def fib(n):    # write Fibonacci series up to n
    a, b = 0, 1
    while b < n:
        print b,
        a, b = b, a+b

def fib2(n): # return Fibonacci series up to n
    result = []
    a, b = 0, 1
    while b < n:
        result.append(b)
        a, b = b, a+b
    return result
\end{verbatim}

Now enter the Python interpreter and import this module with the
following command:

\begin{verbatim}
>>> import fibo
\end{verbatim}

This does not enter the names of the functions defined in \code{fibo} 
directly in the current symbol table; it only enters the module name
\code{fibo} there.
Using the module name you can access the functions:

\begin{verbatim}
>>> fibo.fib(1000)
1 1 2 3 5 8 13 21 34 55 89 144 233 377 610 987
>>> fibo.fib2(100)
[1, 1, 2, 3, 5, 8, 13, 21, 34, 55, 89]
>>> fibo.__name__
'fibo'
\end{verbatim}

If you intend to use a function often you can assign it to a local name:

\begin{verbatim}
>>> fib = fibo.fib
>>> fib(500)
1 1 2 3 5 8 13 21 34 55 89 144 233 377
\end{verbatim}


\section{More on Modules \label{moreModules}}

A module can contain executable statements as well as function
definitions.
These statements are intended to initialize the module.
They are executed only the
\emph{first} time the module is imported somewhere.\footnote{
        In fact function definitions are also `statements' that are
        `executed'; the execution enters the function name in the
        module's global symbol table.
}

Each module has its own private symbol table, which is used as the
global symbol table by all functions defined in the module.
Thus, the author of a module can use global variables in the module
without worrying about accidental clashes with a user's global
variables.
On the other hand, if you know what you are doing you can touch a
module's global variables with the same notation used to refer to its
functions,
\code{modname.itemname}.

Modules can import other modules.  It is customary but not required to
place all \keyword{import} statements at the beginning of a module (or
script, for that matter).  The imported module names are placed in the
importing module's global symbol table.

There is a variant of the \keyword{import} statement that imports
names from a module directly into the importing module's symbol
table.  For example:

\begin{verbatim}
>>> from fibo import fib, fib2
>>> fib(500)
1 1 2 3 5 8 13 21 34 55 89 144 233 377
\end{verbatim}

This does not introduce the module name from which the imports are taken
in the local symbol table (so in the example, \code{fibo} is not
defined).

There is even a variant to import all names that a module defines:

\begin{verbatim}
>>> from fibo import *
>>> fib(500)
1 1 2 3 5 8 13 21 34 55 89 144 233 377
\end{verbatim}

This imports all names except those beginning with an underscore
(\code{_}).


\subsection{The Module Search Path \label{searchPath}}

\indexiii{module}{search}{path}
When a module named \module{spam} is imported, the interpreter searches
for a file named \file{spam.py} in the current directory,
and then in the list of directories specified by
the environment variable \envvar{PYTHONPATH}.  This has the same syntax as
the shell variable \envvar{PATH}, that is, a list of
directory names.  When \envvar{PYTHONPATH} is not set, or when the file
is not found there, the search continues in an installation-dependent
default path; on \UNIX, this is usually \file{.:/usr/local/lib/python}.

Actually, modules are searched in the list of directories given by the 
variable \code{sys.path} which is initialized from the directory 
containing the input script (or the current directory),
\envvar{PYTHONPATH} and the installation-dependent default.  This allows
Python programs that know what they're doing to modify or replace the 
module search path.  Note that because the directory containing the
script being run is on the search path, it is important that the
script not have the same name as a standard module, or Python will
attempt to load the script as a module when that module is imported.
This will generally be an error.  See section~\ref{standardModules},
``Standard Modules,'' for more information.


\subsection{``Compiled'' Python files}

As an important speed-up of the start-up time for short programs that
use a lot of standard modules, if a file called \file{spam.pyc} exists
in the directory where \file{spam.py} is found, this is assumed to
contain an already-``byte-compiled'' version of the module \module{spam}.
The modification time of the version of \file{spam.py} used to create
\file{spam.pyc} is recorded in \file{spam.pyc}, and the
\file{.pyc} file is ignored if these don't match.

Normally, you don't need to do anything to create the
\file{spam.pyc} file.  Whenever \file{spam.py} is successfully
compiled, an attempt is made to write the compiled version to
\file{spam.pyc}.  It is not an error if this attempt fails; if for any
reason the file is not written completely, the resulting
\file{spam.pyc} file will be recognized as invalid and thus ignored
later.  The contents of the \file{spam.pyc} file are platform
independent, so a Python module directory can be shared by machines of
different architectures.

Some tips for experts:

\begin{itemize}

\item
When the Python interpreter is invoked with the \programopt{-O} flag,
optimized code is generated and stored in \file{.pyo} files.  The
optimizer currently doesn't help much; it only removes
\keyword{assert} statements.  When \programopt{-O} is used, \emph{all}
bytecode is optimized; \code{.pyc} files are ignored and \code{.py}
files are compiled to optimized bytecode.

\item
Passing two \programopt{-O} flags to the Python interpreter
(\programopt{-OO}) will cause the bytecode compiler to perform
optimizations that could in some rare cases result in malfunctioning
programs.  Currently only \code{__doc__} strings are removed from the
bytecode, resulting in more compact \file{.pyo} files.  Since some
programs may rely on having these available, you should only use this
option if you know what you're doing.

\item
A program doesn't run any faster when it is read from a \file{.pyc} or
\file{.pyo} file than when it is read from a \file{.py} file; the only
thing that's faster about \file{.pyc} or \file{.pyo} files is the
speed with which they are loaded.

\item
When a script is run by giving its name on the command line, the
bytecode for the script is never written to a \file{.pyc} or
\file{.pyo} file.  Thus, the startup time of a script may be reduced
by moving most of its code to a module and having a small bootstrap
script that imports that module.  It is also possible to name a
\file{.pyc} or \file{.pyo} file directly on the command line.

\item
It is possible to have a file called \file{spam.pyc} (or
\file{spam.pyo} when \programopt{-O} is used) without a file
\file{spam.py} for the same module.  This can be used to distribute a
library of Python code in a form that is moderately hard to reverse
engineer.

\item
The module \ulink{\module{compileall}}{../lib/module-compileall.html}%
{} \refstmodindex{compileall} can create \file{.pyc} files (or
\file{.pyo} files when \programopt{-O} is used) for all modules in a
directory.

\end{itemize}


\section{Standard Modules \label{standardModules}}

Python comes with a library of standard modules, described in a separate
document, the \citetitle[../lib/lib.html]{Python Library Reference}
(``Library Reference'' hereafter).  Some modules are built into the
interpreter; these provide access to operations that are not part of
the core of the language but are nevertheless built in, either for
efficiency or to provide access to operating system primitives such as
system calls.  The set of such modules is a configuration option which
also depends on the underlying platform  For example,
the \module{amoeba} module is only provided on systems that somehow
support Amoeba primitives.  One particular module deserves some
attention: \ulink{\module{sys}}{../lib/module-sys.html}%
\refstmodindex{sys}, which is built into every 
Python interpreter.  The variables \code{sys.ps1} and
\code{sys.ps2} define the strings used as primary and secondary
prompts:

\begin{verbatim}
>>> import sys
>>> sys.ps1
'>>> '
>>> sys.ps2
'... '
>>> sys.ps1 = 'C> '
C> print 'Yuck!'
Yuck!
C>

\end{verbatim}

These two variables are only defined if the interpreter is in
interactive mode.

The variable \code{sys.path} is a list of strings that determine the
interpreter's search path for modules. It is initialized to a default
path taken from the environment variable \envvar{PYTHONPATH}, or from
a built-in default if \envvar{PYTHONPATH} is not set.  You can modify
it using standard list operations: 

\begin{verbatim}
>>> import sys
>>> sys.path.append('/ufs/guido/lib/python')
\end{verbatim}

\section{The \function{dir()} Function \label{dir}}

The built-in function \function{dir()} is used to find out which names
a module defines.  It returns a sorted list of strings:

\begin{verbatim}
>>> import fibo, sys
>>> dir(fibo)
['__name__', 'fib', 'fib2']
>>> dir(sys)
['__displayhook__', '__doc__', '__excepthook__', '__name__', '__stderr__',
 '__stdin__', '__stdout__', '_getframe', 'api_version', 'argv', 
 'builtin_module_names', 'byteorder', 'callstats', 'copyright',
 'displayhook', 'exc_clear', 'exc_info', 'exc_type', 'excepthook',
 'exec_prefix', 'executable', 'exit', 'getdefaultencoding', 'getdlopenflags',
 'getrecursionlimit', 'getrefcount', 'hexversion', 'maxint', 'maxunicode',
 'meta_path', 'modules', 'path', 'path_hooks', 'path_importer_cache',
 'platform', 'prefix', 'ps1', 'ps2', 'setcheckinterval', 'setdlopenflags',
 'setprofile', 'setrecursionlimit', 'settrace', 'stderr', 'stdin', 'stdout',
 'version', 'version_info', 'warnoptions']
\end{verbatim}

Without arguments, \function{dir()} lists the names you have defined
currently:

\begin{verbatim}
>>> a = [1, 2, 3, 4, 5]
>>> import fibo, sys
>>> fib = fibo.fib
>>> dir()
['__name__', 'a', 'fib', 'fibo', 'sys']
\end{verbatim}

Note that it lists all types of names: variables, modules, functions, etc.

\function{dir()} does not list the names of built-in functions and
variables.  If you want a list of those, they are defined in the
standard module \module{__builtin__}\refbimodindex{__builtin__}:

\begin{verbatim}
>>> import __builtin__
>>> dir(__builtin__)
['ArithmeticError', 'AssertionError', 'AttributeError',
 'DeprecationWarning', 'EOFError', 'Ellipsis', 'EnvironmentError',
 'Exception', 'False', 'FloatingPointError', 'IOError', 'ImportError',
 'IndentationError', 'IndexError', 'KeyError', 'KeyboardInterrupt',
 'LookupError', 'MemoryError', 'NameError', 'None', 'NotImplemented',
 'NotImplementedError', 'OSError', 'OverflowError', 'OverflowWarning',
 'PendingDeprecationWarning', 'ReferenceError',
 'RuntimeError', 'RuntimeWarning', 'StandardError', 'StopIteration',
 'SyntaxError', 'SyntaxWarning', 'SystemError', 'SystemExit', 'TabError',
 'True', 'TypeError', 'UnboundLocalError', 'UnicodeError', 'UserWarning',
 'ValueError', 'Warning', 'ZeroDivisionError', '__debug__', '__doc__',
 '__import__', '__name__', 'abs', 'apply', 'bool', 'buffer',
 'callable', 'chr', 'classmethod', 'cmp', 'coerce', 'compile', 'complex',
 'copyright', 'credits', 'delattr', 'dict', 'dir', 'divmod',
 'enumerate', 'eval', 'execfile', 'exit', 'file', 'filter', 'float',
 'getattr', 'globals', 'hasattr', 'hash', 'help', 'hex', 'id',
 'input', 'int', 'intern', 'isinstance', 'issubclass', 'iter',
 'len', 'license', 'list', 'locals', 'long', 'map', 'max', 'min',
 'object', 'oct', 'open', 'ord', 'pow', 'property', 'quit',
 'range', 'raw_input', 'reduce', 'reload', 'repr', 'round',
 'setattr', 'slice', 'staticmethod', 'str', 'string', 'sum', 'super',
 'tuple', 'type', 'unichr', 'unicode', 'vars', 'xrange', 'zip']
\end{verbatim}


\section{Packages \label{packages}}

Packages are a way of structuring Python's module namespace
by using ``dotted module names''.  For example, the module name
\module{A.B} designates a submodule named \samp{B} in a package named
\samp{A}.  Just like the use of modules saves the authors of different
modules from having to worry about each other's global variable names,
the use of dotted module names saves the authors of multi-module
packages like NumPy or the Python Imaging Library from having to worry
about each other's module names.

Suppose you want to design a collection of modules (a ``package'') for
the uniform handling of sound files and sound data.  There are many
different sound file formats (usually recognized by their extension,
for example: \file{.wav}, \file{.aiff}, \file{.au}), so you may need
to create and maintain a growing collection of modules for the
conversion between the various file formats.  There are also many
different operations you might want to perform on sound data (such as
mixing, adding echo, applying an equalizer function, creating an
artificial stereo effect), so in addition you will be writing a
never-ending stream of modules to perform these operations.  Here's a
possible structure for your package (expressed in terms of a
hierarchical filesystem):

\begin{verbatim}
Sound/                          Top-level package
      __init__.py               Initialize the sound package
      Formats/                  Subpackage for file format conversions
              __init__.py
              wavread.py
              wavwrite.py
              aiffread.py
              aiffwrite.py
              auread.py
              auwrite.py
              ...
      Effects/                  Subpackage for sound effects
              __init__.py
              echo.py
              surround.py
              reverse.py
              ...
      Filters/                  Subpackage for filters
              __init__.py
              equalizer.py
              vocoder.py
              karaoke.py
              ...
\end{verbatim}

When importing the package, Python searches through the directories
on \code{sys.path} looking for the package subdirectory.

The \file{__init__.py} files are required to make Python treat the
directories as containing packages; this is done to prevent
directories with a common name, such as \samp{string}, from
unintentionally hiding valid modules that occur later on the module
search path. In the simplest case, \file{__init__.py} can just be an
empty file, but it can also execute initialization code for the
package or set the \code{__all__} variable, described later.

Users of the package can import individual modules from the
package, for example:

\begin{verbatim}
import Sound.Effects.echo
\end{verbatim}

This loads the submodule \module{Sound.Effects.echo}.  It must be referenced
with its full name.

\begin{verbatim}
Sound.Effects.echo.echofilter(input, output, delay=0.7, atten=4)
\end{verbatim}

An alternative way of importing the submodule is:

\begin{verbatim}
from Sound.Effects import echo
\end{verbatim}

This also loads the submodule \module{echo}, and makes it available without
its package prefix, so it can be used as follows:

\begin{verbatim}
echo.echofilter(input, output, delay=0.7, atten=4)
\end{verbatim}

Yet another variation is to import the desired function or variable directly:

\begin{verbatim}
from Sound.Effects.echo import echofilter
\end{verbatim}

Again, this loads the submodule \module{echo}, but this makes its function
\function{echofilter()} directly available:

\begin{verbatim}
echofilter(input, output, delay=0.7, atten=4)
\end{verbatim}

Note that when using \code{from \var{package} import \var{item}}, the
item can be either a submodule (or subpackage) of the package, or some 
other name defined in the package, like a function, class or
variable.  The \code{import} statement first tests whether the item is
defined in the package; if not, it assumes it is a module and attempts
to load it.  If it fails to find it, an
\exception{ImportError} exception is raised.

Contrarily, when using syntax like \code{import
\var{item.subitem.subsubitem}}, each item except for the last must be
a package; the last item can be a module or a package but can't be a
class or function or variable defined in the previous item.

\subsection{Importing * From a Package \label{pkg-import-star}}
%The \code{__all__} Attribute

\ttindex{__all__}
Now what happens when the user writes \code{from Sound.Effects import
*}?  Ideally, one would hope that this somehow goes out to the
filesystem, finds which submodules are present in the package, and
imports them all.  Unfortunately, this operation does not work very
well on Mac and Windows platforms, where the filesystem does not
always have accurate information about the case of a filename!  On
these platforms, there is no guaranteed way to know whether a file
\file{ECHO.PY} should be imported as a module \module{echo},
\module{Echo} or \module{ECHO}.  (For example, Windows 95 has the
annoying practice of showing all file names with a capitalized first
letter.)  The DOS 8+3 filename restriction adds another interesting
problem for long module names.

The only solution is for the package author to provide an explicit
index of the package.  The import statement uses the following
convention: if a package's \file{__init__.py} code defines a list
named \code{__all__}, it is taken to be the list of module names that
should be imported when \code{from \var{package} import *} is
encountered.  It is up to the package author to keep this list
up-to-date when a new version of the package is released.  Package
authors may also decide not to support it, if they don't see a use for
importing * from their package.  For example, the file
\file{Sounds/Effects/__init__.py} could contain the following code:

\begin{verbatim}
__all__ = ["echo", "surround", "reverse"]
\end{verbatim}

This would mean that \code{from Sound.Effects import *} would
import the three named submodules of the \module{Sound} package.

If \code{__all__} is not defined, the statement \code{from Sound.Effects
import *} does \emph{not} import all submodules from the package
\module{Sound.Effects} into the current namespace; it only ensures that the
package \module{Sound.Effects} has been imported (possibly running its
initialization code, \file{__init__.py}) and then imports whatever names are
defined in the package.  This includes any names defined (and
submodules explicitly loaded) by \file{__init__.py}.  It also includes any
submodules of the package that were explicitly loaded by previous
import statements.  Consider this code:

\begin{verbatim}
import Sound.Effects.echo
import Sound.Effects.surround
from Sound.Effects import *
\end{verbatim}

In this example, the echo and surround modules are imported in the
current namespace because they are defined in the
\module{Sound.Effects} package when the \code{from...import} statement
is executed.  (This also works when \code{__all__} is defined.)

Note that in general the practice of importing \code{*} from a module or
package is frowned upon, since it often causes poorly readable code.
However, it is okay to use it to save typing in interactive sessions,
and certain modules are designed to export only names that follow
certain patterns.

Remember, there is nothing wrong with using \code{from Package
import specific_submodule}!  In fact, this is the
recommended notation unless the importing module needs to use
submodules with the same name from different packages.


\subsection{Intra-package References}

The submodules often need to refer to each other.  For example, the
\module{surround} module might use the \module{echo} module.  In fact,
such references
are so common that the \keyword{import} statement first looks in the
containing package before looking in the standard module search path.
Thus, the surround module can simply use \code{import echo} or
\code{from echo import echofilter}.  If the imported module is not
found in the current package (the package of which the current module
is a submodule), the \keyword{import} statement looks for a top-level
module with the given name.

When packages are structured into subpackages (as with the
\module{Sound} package in the example), there's no shortcut to refer
to submodules of sibling packages - the full name of the subpackage
must be used.  For example, if the module
\module{Sound.Filters.vocoder} needs to use the \module{echo} module
in the \module{Sound.Effects} package, it can use \code{from
Sound.Effects import echo}.

\subsection{Packages in Multiple Directories}

Packages support one more special attribute, \member{__path__}.  This
is initialized to be a list containing the name of the directory
holding the package's \file{__init__.py} before the code in that file
is executed.  This variable can be modified; doing so affects future
searches for modules and subpackages contained in the package.

While this feature is not often needed, it can be used to extend the
set of modules found in a package.



\chapter{Input and Output \label{io}}

There are several ways to present the output of a program; data can be
printed in a human-readable form, or written to a file for future use.
This chapter will discuss some of the possibilities.


\section{Fancier Output Formatting \label{formatting}}

So far we've encountered two ways of writing values: \emph{expression
statements} and the \keyword{print} statement.  (A third way is using
the \method{write()} method of file objects; the standard output file
can be referenced as \code{sys.stdout}.  See the Library Reference for
more information on this.)

Often you'll want more control over the formatting of your output than
simply printing space-separated values.  There are two ways to format
your output; the first way is to do all the string handling yourself;
using string slicing and concatenation operations you can create any
lay-out you can imagine.  The standard module
\module{string}\refstmodindex{string} contains some useful operations
for padding strings to a given column width; these will be discussed
shortly.  The second way is to use the \code{\%} operator with a
string as the left argument.  The \code{\%} operator interprets the
left argument much like a \cfunction{sprintf()}-style format
string to be applied to the right argument, and returns the string
resulting from this formatting operation.

One question remains, of course: how do you convert values to strings?
Luckily, Python has ways to convert any value to a string: pass it to
the \function{repr()}  or \function{str()} functions.  Reverse quotes
(\code{``}) are equivalent to \function{repr()}, but their use is
discouraged.

The \function{str()} function is meant to return representations of
values which are fairly human-readable, while \function{repr()} is
meant to generate representations which can be read by the interpreter
(or will force a \exception{SyntaxError} if there is not equivalent
syntax).  For objects which don't have a particular representation for
human consumption, \function{str()} will return the same value as
\function{repr()}.  Many values, such as numbers or structures like
lists and dictionaries, have the same representation using either
function.  Strings and floating point numbers, in particular, have two
distinct representations.

Some examples:

\begin{verbatim}
>>> s = 'Hello, world.'
>>> str(s)
'Hello, world.'
>>> repr(s)
"'Hello, world.'"
>>> str(0.1)
'0.1'
>>> repr(0.1)
'0.10000000000000001'
>>> x = 10 * 3.25
>>> y = 200 * 200
>>> s = 'The value of x is ' + repr(x) + ', and y is ' + repr(y) + '...'
>>> print s
The value of x is 32.5, and y is 40000...
>>> # The repr() of a string adds string quotes and backslashes:
... hello = 'hello, world\n'
>>> hellos = repr(hello)
>>> print hellos
'hello, world\n'
>>> # The argument to repr() may be any Python object:
... repr((x, y, ('spam', 'eggs')))
"(32.5, 40000, ('spam', 'eggs'))"
>>> # reverse quotes are convenient in interactive sessions:
... `x, y, ('spam', 'eggs')`
"(32.5, 40000, ('spam', 'eggs'))"
\end{verbatim}

Here are two ways to write a table of squares and cubes:

\begin{verbatim}
>>> for x in range(1, 11):
...     print repr(x).rjust(2), repr(x*x).rjust(3),
...     # Note trailing comma on previous line
...     print repr(x*x*x).rjust(4)
...
 1   1    1
 2   4    8
 3   9   27
 4  16   64
 5  25  125
 6  36  216
 7  49  343
 8  64  512
 9  81  729
10 100 1000
>>> for x in range(1,11):
...     print '%2d %3d %4d' % (x, x*x, x*x*x)
... 
 1   1    1
 2   4    8
 3   9   27
 4  16   64
 5  25  125
 6  36  216
 7  49  343
 8  64  512
 9  81  729
10 100 1000
\end{verbatim}

(Note that one space between each column was added by the way
\keyword{print} works: it always adds spaces between its arguments.)

This example demonstrates the \method{rjust()} method of string objects,
which right-justifies a string in a field of a given width by padding
it with spaces on the left.  There are similar methods
\method{ljust()} and \method{center()}.  These
methods do not write anything, they just return a new string.  If
the input string is too long, they don't truncate it, but return it
unchanged; this will mess up your column lay-out but that's usually
better than the alternative, which would be lying about a value.  (If
you really want truncation you can always add a slice operation, as in
\samp{x.ljust(~n)[:n]}.)

There is another method, \method{zfill()}, which pads a
numeric string on the left with zeros.  It understands about plus and
minus signs:

\begin{verbatim}
>>> '12'.zfill(5)
'00012'
>>> '-3.14'.zfill(7)
'-003.14'
>>> '3.14159265359'.zfill(5)
'3.14159265359'
\end{verbatim}

Using the \code{\%} operator looks like this:

\begin{verbatim}
>>> import math
>>> print 'The value of PI is approximately %5.3f.' % math.pi
The value of PI is approximately 3.142.
\end{verbatim}

If there is more than one format in the string, you need to pass a
tuple as right operand, as in this example:

\begin{verbatim}
>>> table = {'Sjoerd': 4127, 'Jack': 4098, 'Dcab': 7678}
>>> for name, phone in table.items():
...     print '%-10s ==> %10d' % (name, phone)
... 
Jack       ==>       4098
Dcab       ==>       7678
Sjoerd     ==>       4127
\end{verbatim}

Most formats work exactly as in C and require that you pass the proper
type; however, if you don't you get an exception, not a core dump.
The \code{\%s} format is more relaxed: if the corresponding argument is
not a string object, it is converted to string using the
\function{str()} built-in function.  Using \code{*} to pass the width
or precision in as a separate (integer) argument is supported.  The
C formats \code{\%n} and \code{\%p} are not supported.

If you have a really long format string that you don't want to split
up, it would be nice if you could reference the variables to be
formatted by name instead of by position.  This can be done by using
form \code{\%(name)format}, as shown here:

\begin{verbatim}
>>> table = {'Sjoerd': 4127, 'Jack': 4098, 'Dcab': 8637678}
>>> print 'Jack: %(Jack)d; Sjoerd: %(Sjoerd)d; Dcab: %(Dcab)d' % table
Jack: 4098; Sjoerd: 4127; Dcab: 8637678
\end{verbatim}

This is particularly useful in combination with the new built-in
\function{vars()} function, which returns a dictionary containing all
local variables.

\section{Reading and Writing Files \label{files}}

% Opening files 
\function{open()}\bifuncindex{open} returns a file
object\obindex{file}, and is most commonly used with two arguments:
\samp{open(\var{filename}, \var{mode})}.

\begin{verbatim}
>>> f=open('/tmp/workfile', 'w')
>>> print f
<open file '/tmp/workfile', mode 'w' at 80a0960>
\end{verbatim}

The first argument is a string containing the filename.  The second
argument is another string containing a few characters describing the
way in which the file will be used.  \var{mode} can be \code{'r'} when
the file will only be read, \code{'w'} for only writing (an existing
file with the same name will be erased), and \code{'a'} opens the file
for appending; any data written to the file is automatically added to
the end.  \code{'r+'} opens the file for both reading and writing.
The \var{mode} argument is optional; \code{'r'} will be assumed if
it's omitted.

On Windows and the Macintosh, \code{'b'} appended to the
mode opens the file in binary mode, so there are also modes like
\code{'rb'}, \code{'wb'}, and \code{'r+b'}.  Windows makes a
distinction between text and binary files; the end-of-line characters
in text files are automatically altered slightly when data is read or
written.  This behind-the-scenes modification to file data is fine for
\ASCII{} text files, but it'll corrupt binary data like that in JPEGs or
\file{.EXE} files.  Be very careful to use binary mode when reading and
writing such files.  (Note that the precise semantics of text mode on
the Macintosh depends on the underlying C library being used.)

\subsection{Methods of File Objects \label{fileMethods}}

The rest of the examples in this section will assume that a file
object called \code{f} has already been created.

To read a file's contents, call \code{f.read(\var{size})}, which reads
some quantity of data and returns it as a string.  \var{size} is an
optional numeric argument.  When \var{size} is omitted or negative,
the entire contents of the file will be read and returned; it's your
problem if the file is twice as large as your machine's memory.
Otherwise, at most \var{size} bytes are read and returned.  If the end
of the file has been reached, \code{f.read()} will return an empty
string (\code {""}).
\begin{verbatim}
>>> f.read()
'This is the entire file.\n'
>>> f.read()
''
\end{verbatim}

\code{f.readline()} reads a single line from the file; a newline
character (\code{\e n}) is left at the end of the string, and is only
omitted on the last line of the file if the file doesn't end in a
newline.  This makes the return value unambiguous; if
\code{f.readline()} returns an empty string, the end of the file has
been reached, while a blank line is represented by \code{'\e n'}, a
string containing only a single newline.  

\begin{verbatim}
>>> f.readline()
'This is the first line of the file.\n'
>>> f.readline()
'Second line of the file\n'
>>> f.readline()
''
\end{verbatim}

\code{f.readlines()} returns a list containing all the lines of data
in the file.  If given an optional parameter \var{sizehint}, it reads
that many bytes from the file and enough more to complete a line, and
returns the lines from that.  This is often used to allow efficient
reading of a large file by lines, but without having to load the
entire file in memory.  Only complete lines will be returned.

\begin{verbatim}
>>> f.readlines()
['This is the first line of the file.\n', 'Second line of the file\n']
\end{verbatim}

\code{f.write(\var{string})} writes the contents of \var{string} to
the file, returning \code{None}.  

\begin{verbatim}
>>> f.write('This is a test\n')
\end{verbatim}

\code{f.tell()} returns an integer giving the file object's current
position in the file, measured in bytes from the beginning of the
file.  To change the file object's position, use
\samp{f.seek(\var{offset}, \var{from_what})}.  The position is
computed from adding \var{offset} to a reference point; the reference
point is selected by the \var{from_what} argument.  A
\var{from_what} value of 0 measures from the beginning of the file, 1
uses the current file position, and 2 uses the end of the file as the
reference point.  \var{from_what} can be omitted and defaults to 0,
using the beginning of the file as the reference point.

\begin{verbatim}
>>> f=open('/tmp/workfile', 'r+')
>>> f.write('0123456789abcdef')
>>> f.seek(5)     # Go to the 6th byte in the file
>>> f.read(1)        
'5'
>>> f.seek(-3, 2) # Go to the 3rd byte before the end
>>> f.read(1)
'd'
\end{verbatim}

When you're done with a file, call \code{f.close()} to close it and
free up any system resources taken up by the open file.  After calling
\code{f.close()}, attempts to use the file object will automatically fail.

\begin{verbatim}
>>> f.close()
>>> f.read()
Traceback (most recent call last):
  File "<stdin>", line 1, in ?
ValueError: I/O operation on closed file
\end{verbatim}

File objects have some additional methods, such as
\method{isatty()} and \method{truncate()} which are less frequently
used; consult the Library Reference for a complete guide to file
objects.

\subsection{The \module{pickle} Module \label{pickle}}
\refstmodindex{pickle}

Strings can easily be written to and read from a file. Numbers take a
bit more effort, since the \method{read()} method only returns
strings, which will have to be passed to a function like
\function{int()}, which takes a string like \code{'123'} and
returns its numeric value 123.  However, when you want to save more
complex data types like lists, dictionaries, or class instances,
things get a lot more complicated.

Rather than have users be constantly writing and debugging code to
save complicated data types, Python provides a standard module called
\ulink{\module{pickle}}{../lib/module-pickle.html}.  This is an
amazing module that can take almost
any Python object (even some forms of Python code!), and convert it to
a string representation; this process is called \dfn{pickling}.  
Reconstructing the object from the string representation is called
\dfn{unpickling}.  Between pickling and unpickling, the string
representing the object may have been stored in a file or data, or
sent over a network connection to some distant machine.

If you have an object \code{x}, and a file object \code{f} that's been
opened for writing, the simplest way to pickle the object takes only
one line of code:

\begin{verbatim}
pickle.dump(x, f)
\end{verbatim}

To unpickle the object again, if \code{f} is a file object which has
been opened for reading:

\begin{verbatim}
x = pickle.load(f)
\end{verbatim}

(There are other variants of this, used when pickling many objects or
when you don't want to write the pickled data to a file; consult the
complete documentation for
\ulink{\module{pickle}}{../lib/module-pickle.html} in the
\citetitle[../lib/]{Python Library Reference}.)

\ulink{\module{pickle}}{../lib/module-pickle.html} is the standard way
to make Python objects which can be stored and reused by other
programs or by a future invocation of the same program; the technical
term for this is a \dfn{persistent} object.  Because
\ulink{\module{pickle}}{../lib/module-pickle.html} is so widely used,
many authors who write Python extensions take care to ensure that new
data types such as matrices can be properly pickled and unpickled.



\chapter{Errors and Exceptions \label{errors}}

Until now error messages haven't been more than mentioned, but if you
have tried out the examples you have probably seen some.  There are
(at least) two distinguishable kinds of errors:
\emph{syntax errors} and \emph{exceptions}.

\section{Syntax Errors \label{syntaxErrors}}

Syntax errors, also known as parsing errors, are perhaps the most common
kind of complaint you get while you are still learning Python:

\begin{verbatim}
>>> while True print 'Hello world'
  File "<stdin>", line 1, in ?
    while True print 'Hello world'
                   ^
SyntaxError: invalid syntax
\end{verbatim}

The parser repeats the offending line and displays a little `arrow'
pointing at the earliest point in the line where the error was
detected.  The error is caused by (or at least detected at) the token
\emph{preceding} the arrow: in the example, the error is detected at
the keyword \keyword{print}, since a colon (\character{:}) is missing
before it.  File name and line number are printed so you know where to
look in case the input came from a script.

\section{Exceptions \label{exceptions}}

Even if a statement or expression is syntactically correct, it may
cause an error when an attempt is made to execute it.
Errors detected during execution are called \emph{exceptions} and are
not unconditionally fatal: you will soon learn how to handle them in
Python programs.  Most exceptions are not handled by programs,
however, and result in error messages as shown here:

\begin{verbatim}
>>> 10 * (1/0)
Traceback (most recent call last):
  File "<stdin>", line 1, in ?
ZeroDivisionError: integer division or modulo by zero
>>> 4 + spam*3
Traceback (most recent call last):
  File "<stdin>", line 1, in ?
NameError: name 'spam' is not defined
>>> '2' + 2
Traceback (most recent call last):
  File "<stdin>", line 1, in ?
TypeError: cannot concatenate 'str' and 'int' objects
\end{verbatim}

The last line of the error message indicates what happened.
Exceptions come in different types, and the type is printed as part of
the message: the types in the example are
\exception{ZeroDivisionError}, \exception{NameError} and
\exception{TypeError}.
The string printed as the exception type is the name of the built-in
exception that occurred.  This is true for all built-in
exceptions, but need not be true for user-defined exceptions (although
it is a useful convention).
Standard exception names are built-in identifiers (not reserved
keywords).

The rest of the line is a detail whose interpretation depends on the
exception type; its meaning is dependent on the exception type.

The preceding part of the error message shows the context where the
exception happened, in the form of a stack backtrace.
In general it contains a stack backtrace listing source lines; however,
it will not display lines read from standard input.

The \citetitle[../lib/module-exceptions.html]{Python Library
Reference} lists the built-in exceptions and their meanings.


\section{Handling Exceptions \label{handling}}

It is possible to write programs that handle selected exceptions.
Look at the following example, which asks the user for input until a
valid integer has been entered, but allows the user to interrupt the
program (using \kbd{Control-C} or whatever the operating system
supports); note that a user-generated interruption is signalled by
raising the \exception{KeyboardInterrupt} exception.

\begin{verbatim}
>>> while True:
...     try:
...         x = int(raw_input("Please enter a number: "))
...         break
...     except ValueError:
...         print "Oops! That was no valid number.  Try again..."
...     
\end{verbatim}

The \keyword{try} statement works as follows.

\begin{itemize}
\item
First, the \emph{try clause} (the statement(s) between the
\keyword{try} and \keyword{except} keywords) is executed.

\item
If no exception occurs, the \emph{except\ clause} is skipped and
execution of the \keyword{try} statement is finished.

\item
If an exception occurs during execution of the try clause, the rest of
the clause is skipped.  Then if its type matches the exception named
after the \keyword{except} keyword, the rest of the try clause is
skipped, the except clause is executed, and then execution continues
after the \keyword{try} statement.

\item
If an exception occurs which does not match the exception named in the
except clause, it is passed on to outer \keyword{try} statements; if
no handler is found, it is an \emph{unhandled exception} and execution
stops with a message as shown above.

\end{itemize}

A \keyword{try} statement may have more than one except clause, to
specify handlers for different exceptions.  At most one handler will
be executed.  Handlers only handle exceptions that occur in the
corresponding try clause, not in other handlers of the same
\keyword{try} statement.  An except clause may name multiple exceptions
as a parenthesized list, for example:

\begin{verbatim}
... except (RuntimeError, TypeError, NameError):
...     pass
\end{verbatim}

The last except clause may omit the exception name(s), to serve as a
wildcard.  Use this with extreme caution, since it is easy to mask a
real programming error in this way!  It can also be used to print an
error message and then re-raise the exception (allowing a caller to
handle the exception as well):

\begin{verbatim}
import sys

try:
    f = open('myfile.txt')
    s = f.readline()
    i = int(s.strip())
except IOError, (errno, strerror):
    print "I/O error(%s): %s" % (errno, strerror)
except ValueError:
    print "Could not convert data to an integer."
except:
    print "Unexpected error:", sys.exc_info()[0]
    raise
\end{verbatim}

The \keyword{try} \ldots\ \keyword{except} statement has an optional
\emph{else clause}, which, when present, must follow all except
clauses.  It is useful for code that must be executed if the try
clause does not raise an exception.  For example:

\begin{verbatim}
for arg in sys.argv[1:]:
    try:
        f = open(arg, 'r')
    except IOError:
        print 'cannot open', arg
    else:
        print arg, 'has', len(f.readlines()), 'lines'
        f.close()
\end{verbatim}

The use of the \keyword{else} clause is better than adding additional
code to the \keyword{try} clause because it avoids accidentally
catching an exception that wasn't raised by the code being protected
by the \keyword{try} \ldots\ \keyword{except} statement.


When an exception occurs, it may have an associated value, also known as
the exception's \emph{argument}.
The presence and type of the argument depend on the exception type.

The except clause may specify a variable after the exception name (or list).
The variable is bound to an exception instance with the arguments stored
in \code{instance.args}.  For convenience, the exception instance
defines \method{__getitem__} and \method{__str__} so the arguments can
be accessed or printed directly without having to reference \code{.args}.

\begin{verbatim}
>>> try:
...    raise Exception('spam', 'eggs')
... except Exception, inst:
...    print type(inst)     # the exception instance
...    print inst.args      # arguments stored in .args
...    print inst           # __str__ allows args to printed directly
...    x, y = inst          # __getitem__ allows args to be unpacked directly
...    print 'x =', x
...    print 'y =', y
...
<type 'instance'>
('spam', 'eggs')
('spam', 'eggs')
x = spam
y = eggs
\end{verbatim}

If an exception has an argument, it is printed as the last part
(`detail') of the message for unhandled exceptions.

Exception handlers don't just handle exceptions if they occur
immediately in the try clause, but also if they occur inside functions
that are called (even indirectly) in the try clause.
For example:

\begin{verbatim}
>>> def this_fails():
...     x = 1/0
... 
>>> try:
...     this_fails()
... except ZeroDivisionError, detail:
...     print 'Handling run-time error:', detail
... 
Handling run-time error: integer division or modulo
\end{verbatim}


\section{Raising Exceptions \label{raising}}

The \keyword{raise} statement allows the programmer to force a
specified exception to occur.
For example:

\begin{verbatim}
>>> raise NameError, 'HiThere'
Traceback (most recent call last):
  File "<stdin>", line 1, in ?
NameError: HiThere
\end{verbatim}

The first argument to \keyword{raise} names the exception to be
raised.  The optional second argument specifies the exception's
argument.

If you need to determine whether an exception was raised but don't
intend to handle it, a simpler form of the \keyword{raise} statement
allows you to re-raise the exception:

\begin{verbatim}
>>> try:
...     raise NameError, 'HiThere'
... except NameError:
...     print 'An exception flew by!'
...     raise
...
An exception flew by!
Traceback (most recent call last):
  File "<stdin>", line 2, in ?
NameError: HiThere
\end{verbatim}


\section{User-defined Exceptions \label{userExceptions}}

Programs may name their own exceptions by creating a new exception
class.  Exceptions should typically be derived from the
\exception{Exception} class, either directly or indirectly.  For
example:

\begin{verbatim}
>>> class MyError(Exception):
...     def __init__(self, value):
...         self.value = value
...     def __str__(self):
...         return repr(self.value)
... 
>>> try:
...     raise MyError(2*2)
... except MyError, e:
...     print 'My exception occurred, value:', e.value
... 
My exception occurred, value: 4
>>> raise MyError, 'oops!'
Traceback (most recent call last):
  File "<stdin>", line 1, in ?
__main__.MyError: 'oops!'
\end{verbatim}

Exception classes can be defined which do anything any other class can
do, but are usually kept simple, often only offering a number of
attributes that allow information about the error to be extracted by
handlers for the exception.  When creating a module which can raise
several distinct errors, a common practice is to create a base class
for exceptions defined by that module, and subclass that to create
specific exception classes for different error conditions:

\begin{verbatim}
class Error(Exception):
    """Base class for exceptions in this module."""
    pass

class InputError(Error):
    """Exception raised for errors in the input.

    Attributes:
        expression -- input expression in which the error occurred
        message -- explanation of the error
    """

    def __init__(self, expression, message):
        self.expression = expression
        self.message = message

class TransitionError(Error):
    """Raised when an operation attempts a state transition that's not
    allowed.

    Attributes:
        previous -- state at beginning of transition
        next -- attempted new state
        message -- explanation of why the specific transition is not allowed
    """

    def __init__(self, previous, next, message):
        self.previous = previous
        self.next = next
        self.message = message
\end{verbatim}

Most exceptions are defined with names that end in ``Error,'' similar
to the naming of the standard exceptions.

Many standard modules define their own exceptions to report errors
that may occur in functions they define.  More information on classes
is presented in chapter \ref{classes}, ``Classes.''


\section{Defining Clean-up Actions \label{cleanup}}

The \keyword{try} statement has another optional clause which is
intended to define clean-up actions that must be executed under all
circumstances.  For example:

\begin{verbatim}
>>> try:
...     raise KeyboardInterrupt
... finally:
...     print 'Goodbye, world!'
... 
Goodbye, world!
Traceback (most recent call last):
  File "<stdin>", line 2, in ?
KeyboardInterrupt
\end{verbatim}

A \emph{finally clause} is executed whether or not an exception has
occurred in the try clause.  When an exception has occurred, it is
re-raised after the finally clause is executed.  The finally clause is
also executed ``on the way out'' when the \keyword{try} statement is
left via a \keyword{break} or \keyword{return} statement.

The code in the finally clause is useful for releasing external
resources (such as files or network connections), regardless of
whether or not the use of the resource was successful.

A \keyword{try} statement must either have one or more except clauses
or one finally clause, but not both.


\chapter{Classes \label{classes}}

Python's class mechanism adds classes to the language with a minimum
of new syntax and semantics.  It is a mixture of the class mechanisms
found in \Cpp{} and Modula-3.  As is true for modules, classes in Python
do not put an absolute barrier between definition and user, but rather
rely on the politeness of the user not to ``break into the
definition.''  The most important features of classes are retained
with full power, however: the class inheritance mechanism allows
multiple base classes, a derived class can override any methods of its
base class or classes, a method can call the method of a base class with the
same name.  Objects can contain an arbitrary amount of private data.

In \Cpp{} terminology, all class members (including the data members) are
\emph{public}, and all member functions are \emph{virtual}.  There are
no special constructors or destructors.  As in Modula-3, there are no
shorthands for referencing the object's members from its methods: the
method function is declared with an explicit first argument
representing the object, which is provided implicitly by the call.  As
in Smalltalk, classes themselves are objects, albeit in the wider
sense of the word: in Python, all data types are objects.  This
provides semantics for importing and renaming.  Unlike 
\Cpp{} and Modula-3, built-in types can be used as base classes for
extension by the user.  Also, like in \Cpp{} but unlike in Modula-3, most
built-in operators with special syntax (arithmetic operators,
subscripting etc.) can be redefined for class instances.

\section{A Word About Terminology \label{terminology}}

Lacking universally accepted terminology to talk about classes, I will
make occasional use of Smalltalk and \Cpp{} terms.  (I would use Modula-3
terms, since its object-oriented semantics are closer to those of
Python than \Cpp, but I expect that few readers have heard of it.)

I also have to warn you that there's a terminological pitfall for
object-oriented readers: the word ``object'' in Python does not
necessarily mean a class instance.  Like \Cpp{} and Modula-3, and
unlike Smalltalk, not all types in Python are classes: the basic
built-in types like integers and lists are not, and even somewhat more
exotic types like files aren't.  However, \emph{all} Python types
share a little bit of common semantics that is best described by using
the word object.

Objects have individuality, and multiple names (in multiple scopes)
can be bound to the same object.  This is known as aliasing in other
languages.  This is usually not appreciated on a first glance at
Python, and can be safely ignored when dealing with immutable basic
types (numbers, strings, tuples).  However, aliasing has an
(intended!) effect on the semantics of Python code involving mutable
objects such as lists, dictionaries, and most types representing
entities outside the program (files, windows, etc.).  This is usually
used to the benefit of the program, since aliases behave like pointers
in some respects.  For example, passing an object is cheap since only
a pointer is passed by the implementation; and if a function modifies
an object passed as an argument, the caller will see the change --- this
eliminates the need for two different argument passing mechanisms as in
Pascal.


\section{Python Scopes and Name Spaces \label{scopes}}

Before introducing classes, I first have to tell you something about
Python's scope rules.  Class definitions play some neat tricks with
namespaces, and you need to know how scopes and namespaces work to
fully understand what's going on.  Incidentally, knowledge about this
subject is useful for any advanced Python programmer.

Let's begin with some definitions.

A \emph{namespace} is a mapping from names to objects.  Most
namespaces are currently implemented as Python dictionaries, but
that's normally not noticeable in any way (except for performance),
and it may change in the future.  Examples of namespaces are: the set
of built-in names (functions such as \function{abs()}, and built-in
exception names); the global names in a module; and the local names in
a function invocation.  In a sense the set of attributes of an object
also form a namespace.  The important thing to know about namespaces
is that there is absolutely no relation between names in different
namespaces; for instance, two different modules may both define a
function ``maximize'' without confusion --- users of the modules must
prefix it with the module name.

By the way, I use the word \emph{attribute} for any name following a
dot --- for example, in the expression \code{z.real}, \code{real} is
an attribute of the object \code{z}.  Strictly speaking, references to
names in modules are attribute references: in the expression
\code{modname.funcname}, \code{modname} is a module object and
\code{funcname} is an attribute of it.  In this case there happens to
be a straightforward mapping between the module's attributes and the
global names defined in the module: they share the same namespace!
\footnote{
        Except for one thing.  Module objects have a secret read-only
        attribute called \member{__dict__} which returns the dictionary
        used to implement the module's namespace; the name
        \member{__dict__} is an attribute but not a global name.
        Obviously, using this violates the abstraction of namespace
        implementation, and should be restricted to things like
        post-mortem debuggers.
}

Attributes may be read-only or writable.  In the latter case,
assignment to attributes is possible.  Module attributes are writable:
you can write \samp{modname.the_answer = 42}.  Writable attributes may
also be deleted with the \keyword{del} statement.  For example,
\samp{del modname.the_answer} will remove the attribute
\member{the_answer} from the object named by \code{modname}.

Name spaces are created at different moments and have different
lifetimes.  The namespace containing the built-in names is created
when the Python interpreter starts up, and is never deleted.  The
global namespace for a module is created when the module definition
is read in; normally, module namespaces also last until the
interpreter quits.  The statements executed by the top-level
invocation of the interpreter, either read from a script file or
interactively, are considered part of a module called
\module{__main__}, so they have their own global namespace.  (The
built-in names actually also live in a module; this is called
\module{__builtin__}.)

The local namespace for a function is created when the function is
called, and deleted when the function returns or raises an exception
that is not handled within the function.  (Actually, forgetting would
be a better way to describe what actually happens.)  Of course,
recursive invocations each have their own local namespace.

A \emph{scope} is a textual region of a Python program where a
namespace is directly accessible.  ``Directly accessible'' here means
that an unqualified reference to a name attempts to find the name in
the namespace.

Although scopes are determined statically, they are used dynamically.
At any time during execution, there are at least three nested scopes whose
namespaces are directly accessible: the innermost scope, which is searched
first, contains the local names; the namespaces of any enclosing
functions, which are searched starting with the nearest enclosing scope;
the middle scope, searched next, contains the current module's global names;
and the outermost scope (searched last) is the namespace containing built-in
names.

If a name is declared global, then all references and assignments go
directly to the middle scope containing the module's global names.
Otherwise, all variables found outside of the innermost scope are read-only.

Usually, the local scope references the local names of the (textually)
current function.  Outside of functions, the local scope references
the same namespace as the global scope: the module's namespace.
Class definitions place yet another namespace in the local scope.

It is important to realize that scopes are determined textually: the
global scope of a function defined in a module is that module's
namespace, no matter from where or by what alias the function is
called.  On the other hand, the actual search for names is done
dynamically, at run time --- however, the language definition is
evolving towards static name resolution, at ``compile'' time, so don't
rely on dynamic name resolution!  (In fact, local variables are
already determined statically.)

A special quirk of Python is that assignments always go into the
innermost scope.  Assignments do not copy data --- they just
bind names to objects.  The same is true for deletions: the statement
\samp{del x} removes the binding of \code{x} from the namespace
referenced by the local scope.  In fact, all operations that introduce
new names use the local scope: in particular, import statements and
function definitions bind the module or function name in the local
scope.  (The \keyword{global} statement can be used to indicate that
particular variables live in the global scope.)


\section{A First Look at Classes \label{firstClasses}}

Classes introduce a little bit of new syntax, three new object types,
and some new semantics.


\subsection{Class Definition Syntax \label{classDefinition}}

The simplest form of class definition looks like this:

\begin{verbatim}
class ClassName:
    <statement-1>
    .
    .
    .
    <statement-N>
\end{verbatim}

Class definitions, like function definitions
(\keyword{def} statements) must be executed before they have any
effect.  (You could conceivably place a class definition in a branch
of an \keyword{if} statement, or inside a function.)

In practice, the statements inside a class definition will usually be
function definitions, but other statements are allowed, and sometimes
useful --- we'll come back to this later.  The function definitions
inside a class normally have a peculiar form of argument list,
dictated by the calling conventions for methods --- again, this is
explained later.

When a class definition is entered, a new namespace is created, and
used as the local scope --- thus, all assignments to local variables
go into this new namespace.  In particular, function definitions bind
the name of the new function here.

When a class definition is left normally (via the end), a \emph{class
object} is created.  This is basically a wrapper around the contents
of the namespace created by the class definition; we'll learn more
about class objects in the next section.  The original local scope
(the one in effect just before the class definitions was entered) is
reinstated, and the class object is bound here to the class name given
in the class definition header (\class{ClassName} in the example).


\subsection{Class Objects \label{classObjects}}

Class objects support two kinds of operations: attribute references
and instantiation.

\emph{Attribute references} use the standard syntax used for all
attribute references in Python: \code{obj.name}.  Valid attribute
names are all the names that were in the class's namespace when the
class object was created.  So, if the class definition looked like
this:

\begin{verbatim}
class MyClass:
    "A simple example class"
    i = 12345
    def f(self):
        return 'hello world'
\end{verbatim}

then \code{MyClass.i} and \code{MyClass.f} are valid attribute
references, returning an integer and a method object, respectively.
Class attributes can also be assigned to, so you can change the value
of \code{MyClass.i} by assignment.  \member{__doc__} is also a valid
attribute, returning the docstring belonging to the class: \code{"A
simple example class"}. 

Class \emph{instantiation} uses function notation.  Just pretend that
the class object is a parameterless function that returns a new
instance of the class.  For example (assuming the above class):

\begin{verbatim}
x = MyClass()
\end{verbatim}

creates a new \emph{instance} of the class and assigns this object to
the local variable \code{x}.

The instantiation operation (``calling'' a class object) creates an
empty object.  Many classes like to create objects in a known initial
state.  Therefore a class may define a special method named
\method{__init__()}, like this:

\begin{verbatim}
    def __init__(self):
        self.data = []
\end{verbatim}

When a class defines an \method{__init__()} method, class
instantiation automatically invokes \method{__init__()} for the
newly-created class instance.  So in this example, a new, initialized
instance can be obtained by:

\begin{verbatim}
x = MyClass()
\end{verbatim}

Of course, the \method{__init__()} method may have arguments for
greater flexibility.  In that case, arguments given to the class
instantiation operator are passed on to \method{__init__()}.  For
example,

\begin{verbatim}
>>> class Complex:
...     def __init__(self, realpart, imagpart):
...         self.r = realpart
...         self.i = imagpart
... 
>>> x = Complex(3.0, -4.5)
>>> x.r, x.i
(3.0, -4.5)
\end{verbatim}


\subsection{Instance Objects \label{instanceObjects}}

Now what can we do with instance objects?  The only operations
understood by instance objects are attribute references.  There are
two kinds of valid attribute names.

The first I'll call \emph{data attributes}.  These correspond to
``instance variables'' in Smalltalk, and to ``data members'' in
\Cpp.  Data attributes need not be declared; like local variables,
they spring into existence when they are first assigned to.  For
example, if \code{x} is the instance of \class{MyClass} created above,
the following piece of code will print the value \code{16}, without
leaving a trace:

\begin{verbatim}
x.counter = 1
while x.counter < 10:
    x.counter = x.counter * 2
print x.counter
del x.counter
\end{verbatim}

The second kind of attribute references understood by instance objects
are \emph{methods}.  A method is a function that ``belongs to'' an
object.  (In Python, the term method is not unique to class instances:
other object types can have methods as well.  For example, list objects have
methods called append, insert, remove, sort, and so on.  However,
below, we'll use the term method exclusively to mean methods of class
instance objects, unless explicitly stated otherwise.)

Valid method names of an instance object depend on its class.  By
definition, all attributes of a class that are (user-defined) function 
objects define corresponding methods of its instances.  So in our
example, \code{x.f} is a valid method reference, since
\code{MyClass.f} is a function, but \code{x.i} is not, since
\code{MyClass.i} is not.  But \code{x.f} is not the same thing as
\code{MyClass.f} --- it is a \obindex{method}\emph{method object}, not
a function object.


\subsection{Method Objects \label{methodObjects}}

Usually, a method is called immediately:

\begin{verbatim}
x.f()
\end{verbatim}

In our example, this will return the string \code{'hello world'}.
However, it is not necessary to call a method right away:
\code{x.f} is a method object, and can be stored away and called at a
later time.  For example:

\begin{verbatim}
xf = x.f
while True:
    print xf()
\end{verbatim}

will continue to print \samp{hello world} until the end of time.

What exactly happens when a method is called?  You may have noticed
that \code{x.f()} was called without an argument above, even though
the function definition for \method{f} specified an argument.  What
happened to the argument?  Surely Python raises an exception when a
function that requires an argument is called without any --- even if
the argument isn't actually used...

Actually, you may have guessed the answer: the special thing about
methods is that the object is passed as the first argument of the
function.  In our example, the call \code{x.f()} is exactly equivalent
to \code{MyClass.f(x)}.  In general, calling a method with a list of
\var{n} arguments is equivalent to calling the corresponding function
with an argument list that is created by inserting the method's object
before the first argument.

If you still don't understand how methods work, a look at the
implementation can perhaps clarify matters.  When an instance
attribute is referenced that isn't a data attribute, its class is
searched.  If the name denotes a valid class attribute that is a
function object, a method object is created by packing (pointers to)
the instance object and the function object just found together in an
abstract object: this is the method object.  When the method object is
called with an argument list, it is unpacked again, a new argument
list is constructed from the instance object and the original argument
list, and the function object is called with this new argument list.


\section{Random Remarks \label{remarks}}

% [These should perhaps be placed more carefully...]


Data attributes override method attributes with the same name; to
avoid accidental name conflicts, which may cause hard-to-find bugs in
large programs, it is wise to use some kind of convention that
minimizes the chance of conflicts.  Possible conventions include
capitalizing method names, prefixing data attribute names with a small
unique string (perhaps just an underscore), or using verbs for methods
and nouns for data attributes.


Data attributes may be referenced by methods as well as by ordinary
users (``clients'') of an object.  In other words, classes are not
usable to implement pure abstract data types.  In fact, nothing in
Python makes it possible to enforce data hiding --- it is all based
upon convention.  (On the other hand, the Python implementation,
written in C, can completely hide implementation details and control
access to an object if necessary; this can be used by extensions to
Python written in C.)


Clients should use data attributes with care --- clients may mess up
invariants maintained by the methods by stamping on their data
attributes.  Note that clients may add data attributes of their own to
an instance object without affecting the validity of the methods, as
long as name conflicts are avoided --- again, a naming convention can
save a lot of headaches here.


There is no shorthand for referencing data attributes (or other
methods!) from within methods.  I find that this actually increases
the readability of methods: there is no chance of confusing local
variables and instance variables when glancing through a method.


Conventionally, the first argument of methods is often called
\code{self}.  This is nothing more than a convention: the name
\code{self} has absolutely no special meaning to Python.  (Note,
however, that by not following the convention your code may be less
readable by other Python programmers, and it is also conceivable that
a \emph{class browser} program be written which relies upon such a
convention.)


Any function object that is a class attribute defines a method for
instances of that class.  It is not necessary that the function
definition is textually enclosed in the class definition: assigning a
function object to a local variable in the class is also ok.  For
example:

\begin{verbatim}
# Function defined outside the class
def f1(self, x, y):
    return min(x, x+y)

class C:
    f = f1
    def g(self):
        return 'hello world'
    h = g
\end{verbatim}

Now \code{f}, \code{g} and \code{h} are all attributes of class
\class{C} that refer to function objects, and consequently they are all
methods of instances of \class{C} --- \code{h} being exactly equivalent
to \code{g}.  Note that this practice usually only serves to confuse
the reader of a program.


Methods may call other methods by using method attributes of the
\code{self} argument:

\begin{verbatim}
class Bag:
    def __init__(self):
        self.data = []
    def add(self, x):
        self.data.append(x)
    def addtwice(self, x):
        self.add(x)
        self.add(x)
\end{verbatim}

Methods may reference global names in the same way as ordinary
functions.  The global scope associated with a method is the module
containing the class definition.  (The class itself is never used as a
global scope!)  While one rarely encounters a good reason for using
global data in a method, there are many legitimate uses of the global
scope: for one thing, functions and modules imported into the global
scope can be used by methods, as well as functions and classes defined
in it.  Usually, the class containing the method is itself defined in
this global scope, and in the next section we'll find some good
reasons why a method would want to reference its own class!


\section{Inheritance \label{inheritance}}

Of course, a language feature would not be worthy of the name ``class''
without supporting inheritance.  The syntax for a derived class
definition looks as follows:

\begin{verbatim}
class DerivedClassName(BaseClassName):
    <statement-1>
    .
    .
    .
    <statement-N>
\end{verbatim}

The name \class{BaseClassName} must be defined in a scope containing
the derived class definition.  Instead of a base class name, an
expression is also allowed.  This is useful when the base class is
defined in another module,

\begin{verbatim}
class DerivedClassName(modname.BaseClassName):
\end{verbatim}

Execution of a derived class definition proceeds the same as for a
base class.  When the class object is constructed, the base class is
remembered.  This is used for resolving attribute references: if a
requested attribute is not found in the class, it is searched in the
base class.  This rule is applied recursively if the base class itself
is derived from some other class.

There's nothing special about instantiation of derived classes:
\code{DerivedClassName()} creates a new instance of the class.  Method
references are resolved as follows: the corresponding class attribute
is searched, descending down the chain of base classes if necessary,
and the method reference is valid if this yields a function object.

Derived classes may override methods of their base classes.  Because
methods have no special privileges when calling other methods of the
same object, a method of a base class that calls another method
defined in the same base class, may in fact end up calling a method of
a derived class that overrides it.  (For \Cpp{} programmers: all methods
in Python are effectively \keyword{virtual}.)

An overriding method in a derived class may in fact want to extend
rather than simply replace the base class method of the same name.
There is a simple way to call the base class method directly: just
call \samp{BaseClassName.methodname(self, arguments)}.  This is
occasionally useful to clients as well.  (Note that this only works if
the base class is defined or imported directly in the global scope.)


\subsection{Multiple Inheritance \label{multiple}}

Python supports a limited form of multiple inheritance as well.  A
class definition with multiple base classes looks as follows:

\begin{verbatim}
class DerivedClassName(Base1, Base2, Base3):
    <statement-1>
    .
    .
    .
    <statement-N>
\end{verbatim}

The only rule necessary to explain the semantics is the resolution
rule used for class attribute references.  This is depth-first,
left-to-right.  Thus, if an attribute is not found in
\class{DerivedClassName}, it is searched in \class{Base1}, then
(recursively) in the base classes of \class{Base1}, and only if it is
not found there, it is searched in \class{Base2}, and so on.

(To some people breadth first --- searching \class{Base2} and
\class{Base3} before the base classes of \class{Base1} --- looks more
natural.  However, this would require you to know whether a particular
attribute of \class{Base1} is actually defined in \class{Base1} or in
one of its base classes before you can figure out the consequences of
a name conflict with an attribute of \class{Base2}.  The depth-first
rule makes no differences between direct and inherited attributes of
\class{Base1}.)

It is clear that indiscriminate use of multiple inheritance is a
maintenance nightmare, given the reliance in Python on conventions to
avoid accidental name conflicts.  A well-known problem with multiple
inheritance is a class derived from two classes that happen to have a
common base class.  While it is easy enough to figure out what happens
in this case (the instance will have a single copy of ``instance
variables'' or data attributes used by the common base class), it is
not clear that these semantics are in any way useful.


\section{Private Variables \label{private}}

There is limited support for class-private
identifiers.  Any identifier of the form \code{__spam} (at least two
leading underscores, at most one trailing underscore) is textually
replaced with \code{_classname__spam}, where \code{classname} is the
current class name with leading underscore(s) stripped.  This mangling
is done without regard of the syntactic position of the identifier, so
it can be used to define class-private instance and class variables,
methods, as well as globals, and even to store instance variables
private to this class on instances of \emph{other} classes.  Truncation
may occur when the mangled name would be longer than 255 characters.
Outside classes, or when the class name consists of only underscores,
no mangling occurs.

Name mangling is intended to give classes an easy way to define
``private'' instance variables and methods, without having to worry
about instance variables defined by derived classes, or mucking with
instance variables by code outside the class.  Note that the mangling
rules are designed mostly to avoid accidents; it still is possible for
a determined soul to access or modify a variable that is considered
private.  This can even be useful in special circumstances, such as in
the debugger, and that's one reason why this loophole is not closed.
(Buglet: derivation of a class with the same name as the base class
makes use of private variables of the base class possible.)

Notice that code passed to \code{exec}, \code{eval()} or
\code{evalfile()} does not consider the classname of the invoking 
class to be the current class; this is similar to the effect of the 
\code{global} statement, the effect of which is likewise restricted to 
code that is byte-compiled together.  The same restriction applies to
\code{getattr()}, \code{setattr()} and \code{delattr()}, as well as
when referencing \code{__dict__} directly.


\section{Odds and Ends \label{odds}}

Sometimes it is useful to have a data type similar to the Pascal
``record'' or C ``struct'', bundling together a couple of named data
items.  An empty class definition will do nicely:

\begin{verbatim}
class Employee:
    pass

john = Employee() # Create an empty employee record

# Fill the fields of the record
john.name = 'John Doe'
john.dept = 'computer lab'
john.salary = 1000
\end{verbatim}

A piece of Python code that expects a particular abstract data type
can often be passed a class that emulates the methods of that data
type instead.  For instance, if you have a function that formats some
data from a file object, you can define a class with methods
\method{read()} and \method{readline()} that gets the data from a string
buffer instead, and pass it as an argument.%  (Unfortunately, this
%technique has its limitations: a class can't define operations that
%are accessed by special syntax such as sequence subscripting or
%arithmetic operators, and assigning such a ``pseudo-file'' to
%\code{sys.stdin} will not cause the interpreter to read further input
%from it.)


Instance method objects have attributes, too: \code{m.im_self} is the
object of which the method is an instance, and \code{m.im_func} is the
function object corresponding to the method.


\section{Exceptions Are Classes Too\label{exceptionClasses}}

User-defined exceptions are identified by classes as well.  Using this
mechanism it is possible to create extensible hierarchies of exceptions.

There are two new valid (semantic) forms for the raise statement:

\begin{verbatim}
raise Class, instance

raise instance
\end{verbatim}

In the first form, \code{instance} must be an instance of
\class{Class} or of a class derived from it.  The second form is a
shorthand for:

\begin{verbatim}
raise instance.__class__, instance
\end{verbatim}

A class in an except clause is compatible with an exception if it is the same
class or a base class thereof (but not the other way around --- an
except clause listing a derived class is not compatible with a base
class).  For example, the following code will print B, C, D in that
order:

\begin{verbatim}
class B:
    pass
class C(B):
    pass
class D(C):
    pass

for c in [B, C, D]:
    try:
        raise c()
    except D:
        print "D"
    except C:
        print "C"
    except B:
        print "B"
\end{verbatim}

Note that if the except clauses were reversed (with
\samp{except B} first), it would have printed B, B, B --- the first
matching except clause is triggered.

When an error message is printed for an unhandled exception which is a
class, the class name is printed, then a colon and a space, and
finally the instance converted to a string using the built-in function
\function{str()}.


\section{Iterators\label{iterators}}

By now, you've probably noticed that most container objects can be looped
over using a \keyword{for} statement:

\begin{verbatim}
for element in [1, 2, 3]:
    print element
for element in (1, 2, 3):
    print element
for key in {'one':1, 'two':2}:
    print key
for char in "123":
    print char
for line in open("myfile.txt"):
    print line
\end{verbatim}

This style of access is clear, concise, and convenient.  The use of iterators
pervades and unifies Python.  Behind the scenes, the \keyword{for}
statement calls \function{iter()} on the container object.  The
function returns an iterator object that defines the method
\method{next()} which accesses elements in the container one at a
time.  When there are no more elements, \method{next()} raises a
\exception{StopIteration} exception which tells the \keyword{for} loop
to terminate.  This example shows how it all works:

\begin{verbatim}
>>> s = 'abc'
>>> it = iter(s)
>>> it
<iterator object at 0x00A1DB50>
>>> it.next()
'a'
>>> it.next()
'b'
>>> it.next()
'c'
>>> it.next()

Traceback (most recent call last):
  File "<pyshell#6>", line 1, in -toplevel-
    it.next()
StopIteration
\end{verbatim}

Having seen the mechanics behind the iterator protocol, it is easy to add
iterator behavior to your classes.  Define a \method{__iter__()} method
which returns an object with a \method{next()} method.  If the class defines
\method{next()}, then \method{__iter__()} can just return \code{self}:

\begin{verbatim}
class Reverse:
    "Iterator for looping over a sequence backwards"
    def __init__(self, data):
        self.data = data
        self.index = len(data)
    def __iter__(self):
        return self
    def next(self):
        if self.index == 0:
            raise StopIteration
        self.index = self.index - 1
        return self.data[self.index]

>>> for char in Reverse('spam'):
...     print char
...
m
a
p
s
\end{verbatim}


\section{Generators\label{generators}}

Generators are a simple and powerful tool for creating iterators.  They are
written like regular functions but use the \keyword{yield} statement whenever
they want to return data.  Each time \method{next()} is called, the
generator resumes where it left-off (it remembers all the data values and
which statement was last executed).  An example shows that generators can
be trivially easy to create:

\begin{verbatim}
def reverse(data):
    for index in range(len(data)-1, -1, -1):
        yield data[index]
	
>>> for char in reverse('golf'):
...     print char
...
f
l
o
g	
\end{verbatim}

Anything that can be done with generators can also be done with class based
iterators as described in the previous section.  What makes generators so
compact is that the \method{__iter__()} and \method{next()} methods are
created automatically.

Another key feature is that the local variables and execution state
are automatically saved between calls.  This made the function easier to write
and much more clear than an approach using class variables like
\code{self.index} and \code{self.data}.

In addition to automatic method creation and saving program state, when
generators terminate, they automatically raise \exception{StopIteration}.
In combination, these features make it easy to create iterators with no
more effort than writing a regular function.

\section{Generator Expressions\label{genexps}}

Some simple generators can be coded succinctly as expressions using a syntax
similar to list comprehensions but with parentheses instead of brackets.  These
expressions are designed for situations where the generator is used right
away by an enclosing function.  Generator expressions are more compact but
less versatile than full generator definitions and tend to be more memory
friendly than equivalent list comprehensions.

Examples:

\begin{verbatim}
>>> sum(i*i for i in range(10))                 # sum of squares
285

>>> xvec = [10, 20, 30]
>>> yvec = [7, 5, 3]
>>> sum(x*y for x,y in zip(xvec, yvec))         # dot product
260

>>> from math import pi, sin
>>> sine_table = dict((x, sin(x*pi/180)) for x in range(0, 91))

>>> unique_words = set(word  for line in page  for word in line.split())

>>> valedictorian = max((student.gpa, student.name) for student in graduates)

>>> data = 'golf'
>>> list(data[i] for i in range(len(data)-1,-1,-1))
['f', 'l', 'o', 'g']

\end{verbatim}



\chapter{Brief Tour of the Standard Library \label{briefTour}}


\section{Operating System Interface\label{os-interface}}

The \ulink{\module{os}}{../lib/module-os.html}
module provides dozens of functions for interacting with the
operating system:

\begin{verbatim}
>>> import os
>>> os.system('time 0:02')
0
>>> os.getcwd()      # Return the current working directory
'C:\\Python24'
>>> os.chdir('/server/accesslogs')
\end{verbatim}

Be sure to use the \samp{import os} style instead of
\samp{from os import *}.  This will keep \function{os.open()} from
shadowing the builtin \function{open()} function which operates much
differently.

The builtin \function{dir()} and \function{help()} functions are useful
as interactive aids for working with large modules like \module{os}:

\begin{verbatim}
>>> import os
>>> dir(os)
<returns a list of all module functions>
>>> help(os)
<returns an extensive manual page created from the module's docstrings>
\end{verbatim}

For daily file and directory management tasks, the 
\ulink{\module{shutil}}{../lib/module-shutil.html}
module provides a higher level interface that is easier to use:

\begin{verbatim}
>>> import shutil
>>> shutil.copyfile('data.db', 'archive.db')
>>> shutil.move('/build/executables', 'installdir')
\end{verbatim}


\section{File Wildcards\label{file-wildcards}}

The \ulink{\module{glob}}{../lib/module-glob.html}
module provides a function for making file lists from directory
wildcard searches:

\begin{verbatim}
>>> import glob
>>> glob.glob('*.py')
['primes.py', 'random.py', 'quote.py']
\end{verbatim}


\section{Command Line Arguments\label{command-line-arguments}}

Common utility scripts often invoke processing command line arguments.
These arguments are stored in the
\ulink{\module{sys}}{../lib/module-sys.html}\ module's \var{argv}
attribute as a list.  For instance the following output results from
running \samp{python demo.py one two three} at the command line:

\begin{verbatim}
>>> import sys
>>> print sys.argv
['demo.py', 'one', 'two', 'three']
\end{verbatim}

The \ulink{\module{getopt}}{../lib/module-getopt.html}
module processes \var{sys.argv} using the conventions of the \UNIX{}
\function{getopt()} function.  More powerful and flexible command line
processing is provided by the
\ulink{\module{optparse}}{../lib/module-optparse.html} module.


\section{Error Output Redirection and Program Termination\label{stderr}}

The \ulink{\module{sys}}{../lib/module-sys.html}
module also has attributes for \var{stdin}, \var{stdout}, and
\var{stderr}.  The latter is useful for emitting warnings and error
messages to make them visible even when \var{stdout} has been redirected:

\begin{verbatim}
>>> sys.stderr.write('Warning, log file not found starting a new one')
Warning, log file not found starting a new one
\end{verbatim}

The most direct way to terminate a script is to use \samp{sys.exit()}.


\section{String Pattern Matching\label{string-pattern-matching}}

The \ulink{\module{re}}{../lib/module-re.html}
module provides regular expression tools for advanced string processing.
For complex matching and manipulation, regular expressions offer succinct,
optimized solutions:

\begin{verbatim}
>>> import re
>>> re.findall(r'\bf[a-z]*', 'which foot or hand fell fastest')
['foot', 'fell', 'fastest']
>>> re.sub(r'(\b[a-z]+) \1', r'\1', 'cat in the the hat')
'cat in the hat'
\end{verbatim}

When only simple capabilities are needed, string methods are preferred
because they are easier to read and debug:

\begin{verbatim}
>>> 'tea for too'.replace('too', 'two')
'tea for two'
\end{verbatim}

\section{Mathematics\label{mathematics}}

The \ulink{\module{math}}{../lib/module-math.html} module gives
access to the underlying C library functions for floating point math:

\begin{verbatim}
>>> import math
>>> math.cos(math.pi / 4.0)
0.70710678118654757
>>> math.log(1024, 2)
10.0
\end{verbatim}

The \ulink{\module{random}}{../lib/module-random.html}
module provides tools for making random selections:

\begin{verbatim}
>>> import random
>>> random.choice(['apple', 'pear', 'banana'])
'apple'
>>> random.sample(xrange(100), 10)   # sampling without replacement
[30, 83, 16, 4, 8, 81, 41, 50, 18, 33]
>>> random.random()    # random float
0.17970987693706186
>>> random.randrange(6)    # random integer chosen from range(6)
4   
\end{verbatim}


\section{Internet Access\label{internet-access}}

There are a number of modules for accessing the internet and processing
internet protocols. Two of the simplest are
\ulink{\module{urllib2}}{../lib/module-urllib2.html}
for retrieving data from urls and
\ulink{\module{smtplib}}{../lib/module-smtplib.html} 
for sending mail:

\begin{verbatim}
>>> import urllib2
>>> for line in urllib2.urlopen('http://tycho.usno.navy.mil/cgi-bin/timer.pl'):
...     if 'EST' in line:      # look for Eastern Standard Time
...         print line
    
<BR>Nov. 25, 09:43:32 PM EST

>>> import smtplib
>>> server = smtplib.SMTP('localhost')
>>> server.sendmail('soothsayer@example.org', 'jceasar@example.org',
"""To: jceasar@example.org
From: soothsayer@example.org

Beware the Ides of March.
""")
>>> server.quit()
\end{verbatim}


\section{Dates and Times\label{dates-and-times}}

The \ulink{\module{datetime}}{../lib/module-datetime.html} module
supplies classes for manipulating dates and times in both simple
and complex ways. While date and time arithmetic is supported, the
focus of the implementation is on efficient member extraction for
output formatting and manipulation.  The module also supports objects
that are time zone aware.

\begin{verbatim}
# dates are easily constructed and formatted
>>> from datetime import date
>>> now = date.today()
>>> now
datetime.date(2003, 12, 2)
>>> now.strftime("%m-%d-%y or %d%b %Y is a %A on the %d day of %B")
'12-02-03 or 02Dec 2003 is a Tuesday on the 02 day of December'

# dates support calendar arithmetic
>>> birthday = date(1964, 7, 31)
>>> age = now - birthday
>>> age.days
14368
\end{verbatim}


\section{Data Compression\label{data-compression}}

Common data archiving and compression formats are directly supported
by modules including:
\ulink{\module{zlib}}{../lib/module-zlib.html},
\ulink{\module{gzip}}{../lib/module-gzip.html},
\ulink{\module{bz2}}{../lib/module-bz2.html},
\ulink{\module{zipfile}}{../lib/module-zipfile.html}, and
\ulink{\module{tarfile}}{../lib/module-tarfile.html}.

\begin{verbatim}
>>> import zlib
>>> s = 'witch which has which witches wrist watch'
>>> len(s)
41
>>> t = zlib.compress(s)
>>> len(t)
37
>>> zlib.decompress(t)
'witch which has which witches wrist watch'
>>> zlib.crc32(t)
-1438085031
\end{verbatim}


\section{Performance Measurement\label{performance-measurement}}

Some Python users develop a deep interest in knowing the relative
performance between different approaches to the same problem.
Python provides a measurement tool that answers those questions
immediately.

For example, it may be tempting to use the tuple packing and unpacking
feature instead of the traditional approach to swapping arguments.
The \ulink{\module{timeit}}{../lib/module-timeit.html} module
quickly demonstrates a modest performance advantage:

\begin{verbatim}
>>> from timeit import Timer
>>> Timer('t=a; a=b; b=t', 'a=1; b=2').timeit()
0.57535828626024577
>>> Timer('a,b = b,a', 'a=1; b=2').timeit()
0.54962537085770791
\end{verbatim}

In contrast to \module{timeit}'s fine level of granularity, the
\ulink{\module{profile}}{../lib/module-profile.html} and \module{pstats}
modules provide tools for identifying time critical sections in larger
blocks of code.


\section{Quality Control\label{quality-control}}

One approach for developing high quality software is to write tests for
each function as it is developed and to run those tests frequently during
the development process.

The \ulink{\module{doctest}}{../lib/module-doctest.html} module provides
a tool for scanning a module and validating tests embedded in a program's
docstrings.  Test construction is as simple as cutting-and-pasting a
typical call along with its results into the docstring.  This improves
the documentation by providing the user with an example and it allows the
doctest module to make sure the code remains true to the documentation:

\begin{verbatim}
def average(values):
    """Computes the arithmetic mean of a list of numbers.

    >>> print average([20, 30, 70])
    40.0
    """
    return sum(values, 0.0) / len(values)

import doctest
doctest.testmod()   # automatically validate the embedded tests
\end{verbatim}

The \ulink{\module{unittest}}{../lib/module-unittest.html} module is not
as effortless as the \module{doctest} module, but it allows a more
comprehensive set of tests to be maintained in a separate file:

\begin{verbatim}
import unittest

class TestStatisticalFunctions(unittest.TestCase):

    def test_average(self):
        self.assertEqual(average([20, 30, 70]), 40.0)
        self.assertEqual(round(average([1, 5, 7]), 1), 4.3)
        self.assertRaises(ZeroDivisionError, average, [])
        self.assertRaises(TypeError, average, 20, 30, 70)

unittest.main() # Calling from the command line invokes all tests
\end{verbatim}

\section{Batteries Included\label{batteries-included}}

Python has a ``batteries included'' philosophy.  This is best seen
through the sophisticated and robust capabilities of its larger
packages. For example:

* The \ulink{\module{xmlrpclib}}{../lib/module-xmlrpclib.html} and
\ulink{\module{SimpleXMLRPCServer}}{../lib/module-SimpleXMLRPCServer.html}
modules make implementing remote procedure calls into an almost trivial
task.  Despite the names, no direct knowledge or handling of XML is needed.

* The \ulink{\module{email}}{../lib/module-email.html}
package is a library for managing email messages,
including MIME and other RFC 2822-based message documents.  Unlike
\module{smtplib} and \module{poplib} which actually send and receive
messages, the email package has a complete toolset for building or
decoding complex message structures (including attachments)
and for implementing internet encoding and header protocols.

* The \ulink{\module{xml.dom}}{../lib/module-xml.dom.html} and
\ulink{\module{xml.sax}}{../lib/module-xml.sax.html} packages provide
robust support for parsing this popular data interchange format.  Likewise,
the \module{csv} module supports direct reads and writes in a common
database format.  Together, these modules and packages greatly simplify
data interchange between python applications and other tools.

* Internationalization is supported by a number of modules including
\ulink{\module{gettext}}{../lib/module-gettext.html},
\ulink{\module{locale}}{../lib/module-locale.html}, and the
\ulink{\module{codecs}}{../lib/module-codecs.html} package.



\chapter{Brief Tour of the Standard Library -- Part II\label{briefTourTwo}}

This second tour covers more advanced modules that support professional
programming needs.  These modules rarely occur in small scripts.


\section{Output Formatting\label{output-formatting}}

The \ulink{\module{repr}}{../lib/module-repr.html} module provides an
version of \function{repr()} for abbreviated displays of large or deeply
nested containers:

\begin{verbatim}
    >>> import repr   
    >>> repr.repr(set('supercalifragilisticexpialidocious'))
    "set(['a', 'c', 'd', 'e', 'f', 'g', ...])"
\end{verbatim}

The \ulink{\module{pprint}}{../lib/module-pprint.html} module offers
more sophisticated control over printing both built-in and user defined
objects in a way that is readable by the interpreter.  When the result
is longer than one line, the ``pretty printer'' adds line breaks and
indentation to more clearly reveal data structure:

\begin{verbatim}
    >>> import pprint
    >>> t = [[[['black', 'cyan'], 'white', ['green', 'red']], [['magenta',
    ...     'yellow'], 'blue']]]
    ...
    >>> pprint.pprint(t, width=30)
    [[[['black', 'cyan'],
       'white',
       ['green', 'red']],
      [['magenta', 'yellow'],
       'blue']]]
\end{verbatim}

The \ulink{\module{textwrap}}{../lib/module-textwrap.html} module
formats paragraphs of text to fit a given screen width:

\begin{verbatim}
    >>> import textwrap
    >>> doc = """The wrap() method is just like fill() except that it returns
    ... a list of strings instead of one big string with newlines to separate
    ... the wrapped lines."""
    ...
    >>> print textwrap.fill(doc, width=40)
    The wrap() method is just like fill()
    except that it returns a list of strings
    instead of one big string with newlines
    to separate the wrapped lines.
\end{verbatim}

The \ulink{\module{locale}}{../lib/module-locale.html} module accesses
a database of culture specific data formats.  The grouping attribute
of locale's format function provides a direct way of formatting numbers
with group separators:

\begin{verbatim}
    >>> import locale
    >>> locale.setlocale(locale.LC_ALL, 'English_United States.1252')
    'English_United States.1252'
    >>> conv = locale.localeconv()          # get a mapping of conventions
    >>> x = 1234567.8
    >>> locale.format("%d", x, grouping=True)
    '1,234,567'
    >>> locale.format("%s%.*f", (conv['currency_symbol'],
    ...	      conv['int_frac_digits'], x), grouping=True)
    '$1,234,567.80'
\end{verbatim}


\section{Working with Binary Data Record Layouts\label{binary-formats}}

The \ulink{\module{struct}}{../lib/module-struct.html} module provides
\function{pack()} and \function{unpack()} functions for working with
variable length binary record formats.  The following example shows how
to loop through header information in a ZIP file (with pack codes
\code{"H"} and \code{"L"} representing two and four byte unsigned
numbers respectively):

\begin{verbatim}
    import struct

    data = open('myfile.zip', 'rb').read()
    start = 0
    for i in range(3):                      # show the first 3 file headers
        start += 14
        fields = struct.unpack('LLLHH', data[start:start+16])
        crc32, comp_size, uncomp_size, filenamesize, extra_size =  fields

        start += 16
        filename = data[start:start+filenamesize]
        start += filenamesize
        extra = data[start:start+extra_size]
        print filename, hex(crc32), comp_size, uncomp_size

        start += extra_size + comp_size     # skip to the next header
\end{verbatim}


\section{Multi-threading\label{multi-threading}}

Threading is a technique for decoupling tasks which are not sequentially
dependent.  Threads can be used to improve the responsiveness of
applications that accept user input while other tasks run in the
background.  A related use case is running I/O in parallel with
computations in another thread.

The following code shows how the high level
\ulink{\module{threading}}{../lib/module-threading.html} module can run
tasks in background while the main program continues to run:

\begin{verbatim}
    import threading, zipfile

    class AsyncZip(threading.Thread):
        def __init__(self, infile, outfile):
            threading.Thread.__init__(self)        
            self.infile = infile
            self.outfile = outfile
        def run(self):
            f = zipfile.ZipFile(self.outfile, 'w', zipfile.ZIP_DEFLATED)
            f.write(self.infile)
            f.close()
            print 'Finished background zip of: ', self.infile

    background = AsyncZip('mydata.txt', 'myarchive.zip')
    background.start()
    print 'The main program continues to run in foreground.'
    
    background.join()    # Wait for the background task to finish
    print 'Main program waited until background was done.'
\end{verbatim}

The principal challenge of multi-threaded applications is coordinating
threads that share data or other resources.  To that end, the threading
module provides a number of synchronization primitives including locks,
events, condition variables, and semaphores.

While those tools are powerful, minor design errors can result in
problems that are difficult to reproduce.  So, the preferred approach
to task coordination is to concentrate all access to a resource
in a single thread and then using the
\ulink{\module{Queue}}{../lib/module-Queue.html} module to feed that
thread with requests from other threads.  Applications using
\class{Queue} objects for inter-thread communication and coordination
are easier to design, more readable, and more reliable.


\section{Logging\label{logging}}

The \ulink{\module{logging}}{../lib/module-logging.html} module offers
a full featured and flexible logging system.  At its simplest, log
messages are sent to a file or to \code{sys.stderr}:

\begin{verbatim}
    import logging
    logging.debug('Debugging information')
    logging.info('Informational message')
    logging.warning('Warning:config file %s not found', 'server.conf')
    logging.error('Error occurred')
    logging.critical('Critical error -- shutting down')
\end{verbatim}

This produces the following output: 

\begin{verbatim}
    WARNING:root:Warning:config file server.conf not found
    ERROR:root:Error occurred
    CRITICAL:root:Critical error -- shutting down
\end{verbatim}

By default, informational and debugging messages are suppressed and the
output is sent to standard error.  Other output options include routing
messages through email, datagrams, sockets, or to an HTTP Server.  New
filters can select different routing based on message priority:
\constant{DEBUG}, \constant{INFO}, \constant{WARNING}, \constant{ERROR},
and \constant{CRITICAL}.

The logging system can be configured directly from Python or can be
loaded from a user editable configuration file for customized logging
without altering the application.


\section{Weak References\label{weak-references}}

Python does automatic memory management (reference counting for most
objects and garbage collection to eliminate cycles).  The memory is
freed shortly after the last reference to it has been eliminated.

This approach works fine for most applications but occasionally there
is a need to track objects only as long as they are being used by
something else.  Unfortunately, just tracking them creates a reference
that makes them permanent.  The
\ulink{\module{weakref}}{../lib/module-weakref.html} module provides
tools for tracking objects without creating a reference.  When the
object is no longer needed, it is automatically removed from a weakref
table and a callback is triggered for weakref objects.  Typical
applications include caching objects that are expensive to create:

\begin{verbatim}
    >>> import weakref, gc
    >>> class A:
    ...     def __init__(self, value):
    ...             self.value = value
    ...     def __repr__(self):
    ...             return str(self.value)
    ...
    >>> a = A(10)                   # create a reference
    >>> d = weakref.WeakValueDictionary()
    >>> d['primary'] = a            # does not create a reference
    >>> d['primary']                # fetch the object if it is still alive
    10
    >>> del a                       # remove the one reference
    >>> gc.collect()                # run garbage collection right away
    0
    >>> d['primary']                # entry was automatically removed
    Traceback (most recent call last):
      File "<pyshell#108>", line 1, in -toplevel-
        d['primary']                # entry was automatically removed
      File "C:/PY24/lib/weakref.py", line 46, in __getitem__
        o = self.data[key]()
    KeyError: 'primary'
\end{verbatim}

\section{Tools for Working with Lists\label{list-tools}}

Many data structure needs can be met with the built-in list type.
However, sometimes there is a need for alternative implementations
with different performance trade-offs.

The \ulink{\module{array}}{../lib/module-array.html} module provides an
\class{array()} object that is like a list that stores only homogenous
data but stores it more compactly.  The following example shows an array
of numbers stored as two byte unsigned binary numbers (typecode
\code{"H"}) rather than the usual 16 bytes per entry for regular lists
of python int objects:

\begin{verbatim}
    >>> from array import array
    >>> a = array('H', [4000, 10, 700, 22222])
    >>> sum(a)
    26932
    >>> a[1:3]
    array('H', [10, 700])
\end{verbatim}

The \ulink{\module{collections}}{../lib/module-collections.html} module
provides a \class{deque()} object that is like a list with faster
appends and pops from the left side but slower lookups in the middle.
These objects are well suited for implementing queues and breadth first
tree searches:

\begin{verbatim}
    >>> from collections import deque
    >>> d = deque(["task1", "task2", "task3"])
    >>> d.append("task4")
    >>> print "Handling", d.popleft()
    Handling task1

    unsearched = deque([starting_node])
    def breadth_first_search(unsearched):
        node = unsearched.popleft()
        for m in gen_moves(node):
            if is_goal(m):
                return m
            unsearched.append(m)
\end{verbatim}

In addition to alternative list implementations, the library also offers
other tools such as the \ulink{\module{bisect}}{../lib/module-bisect.html}
module with functions for manipulating sorted lists:

\begin{verbatim}
    >>> import bisect
    >>> scores = [(100, 'perl'), (200, 'tcl'), (400, 'lua'), (500, 'python')]
    >>> bisect.insort(scores, (300, 'ruby'))
    >>> scores
    [(100, 'perl'), (200, 'tcl'), (300, 'ruby'), (400, 'lua'), (500, 'python')]
\end{verbatim}

The \ulink{\module{heapq}}{../lib/module-heapq.html} module provides
functions for implementing heaps based on regular lists.  The lowest
valued entry is always kept at position zero.  This is useful for
applications which repeatedly access the smallest element but do not
want to run a full list sort:

\begin{verbatim}
    >>> from heapq import heapify, heappop, heappush
    >>> data = [1, 3, 5, 7, 9, 2, 4, 6, 8, 0]
    >>> heapify(data)                      # rearrange the list into heap order
    >>> heappush(data, -5)                 # add a new entry
    >>> [heappop(data) for i in range(3)]  # fetch the three smallest entries
    [-5, 0, 1]
\end{verbatim}


\section{Decimal Floating Point Arithmetic\label{decimal-fp}}

The \ulink{\module{decimal}}{../lib/module-decimal.html} module offers a
\class{Decimal} datatype for decimal floating point arithmetic.  Compared to
the built-in \class{float} implementation of binary floating point, the new
class is especially helpful for financial applications and other uses which
require exact decimal representation, control over precision, control over
rounding to meet legal or regulatory requirements, tracking of significant
decimal places, or for applications where the user expects the results to
match calculations done by hand.

For example, calculating a 5\%{} tax on a 70 cent phone charge gives
different results in decimal floating point and binary floating point.
The difference becomes significant if the results are rounded to the
nearest cent:

\begin{verbatim}
>>> from decimal import *       
>>> Decimal('0.70') * Decimal('1.05')
Decimal("0.7350")
>>> .70 * 1.05
0.73499999999999999       
\end{verbatim}

The \class{Decimal} result keeps a trailing zero, automatically inferring four
place significance from the two digit multiplicands.  Decimal reproduces
mathematics as done by hand and avoids issues that can arise when binary
floating point cannot exactly represent decimal quantities.

Exact representation enables the \class{Decimal} class to perform
modulo calculations and equality tests that are unsuitable for binary
floating point:

\begin{verbatim}
>>> Decimal('1.00') % Decimal('.10')
Decimal("0.00")
>>> 1.00 % 0.10
0.09999999999999995
       
>>> sum([Decimal('0.1')]*10) == Decimal('1.0')
True
>>> sum([0.1]*10) == 1.0
False      
\end{verbatim}

The \module{decimal} module provides arithmetic with as much precision as
needed:

\begin{verbatim}
>>> getcontext().prec = 36
>>> Decimal(1) / Decimal(7)
Decimal("0.142857142857142857142857142857142857")
\end{verbatim}



\chapter{What Now? \label{whatNow}}

Reading this tutorial has probably reinforced your interest in using
Python --- you should be eager to apply Python to solve your
real-world problems.  Now what should you do?

You should read, or at least page through, the
\citetitle[../lib/lib.html]{Python Library Reference},
which gives complete (though terse) reference material about types,
functions, and modules that can save you a lot of time when writing
Python programs.  The standard Python distribution includes a
\emph{lot} of code in both C and Python; there are modules to read
\UNIX{} mailboxes, retrieve documents via HTTP, generate random
numbers, parse command-line options, write CGI programs, compress
data, and a lot more; skimming through the Library Reference will give
you an idea of what's available.

The major Python Web site is \url{http://www.python.org/}; it contains
code, documentation, and pointers to Python-related pages around the
Web.  This Web site is mirrored in various places around the
world, such as Europe, Japan, and Australia; a mirror may be faster
than the main site, depending on your geographical location.  A more
informal site is \url{http://starship.python.net/}, which contains a
bunch of Python-related personal home pages; many people have
downloadable software there. Many more user-created Python modules
can be found in the \ulink{Python Package
Index}{http://www.python.org/pypi} (PyPI).

For Python-related questions and problem reports, you can post to the
newsgroup \newsgroup{comp.lang.python}, or send them to the mailing
list at \email{python-list@python.org}.  The newsgroup and mailing list
are gatewayed, so messages posted to one will automatically be
forwarded to the other.  There are around 120 postings a day (with peaks
up to several hundred),
% Postings figure based on average of last six months activity as
% reported by www.egroups.com; Jan. 2000 - June 2000: 21272 msgs / 182
% days = 116.9 msgs / day and steadily increasing.
asking (and answering) questions, suggesting new features, and
announcing new modules.  Before posting, be sure to check the list of
\ulink{Frequently Asked Questions}{http://www.python.org/doc/faq/} (also called the FAQ), or look for it in the
\file{Misc/} directory of the Python source distribution.  Mailing
list archives are available at \url{http://www.python.org/pipermail/}.
The FAQ answers many of the questions that come up again and again,
and may already contain the solution for your problem.


\appendix

\chapter{Interactive Input Editing and History Substitution\label{interacting}}

Some versions of the Python interpreter support editing of the current
input line and history substitution, similar to facilities found in
the Korn shell and the GNU Bash shell.  This is implemented using the
\emph{GNU Readline} library, which supports Emacs-style and vi-style
editing.  This library has its own documentation which I won't
duplicate here; however, the basics are easily explained.  The
interactive editing and history described here are optionally
available in the \UNIX{} and CygWin versions of the interpreter.

This chapter does \emph{not} document the editing facilities of Mark
Hammond's PythonWin package or the Tk-based environment, IDLE,
distributed with Python.  The command line history recall which
operates within DOS boxes on NT and some other DOS and Windows flavors 
is yet another beast.

\section{Line Editing \label{lineEditing}}

If supported, input line editing is active whenever the interpreter
prints a primary or secondary prompt.  The current line can be edited
using the conventional Emacs control characters.  The most important
of these are: \kbd{C-A} (Control-A) moves the cursor to the beginning
of the line, \kbd{C-E} to the end, \kbd{C-B} moves it one position to
the left, \kbd{C-F} to the right.  Backspace erases the character to
the left of the cursor, \kbd{C-D} the character to its right.
\kbd{C-K} kills (erases) the rest of the line to the right of the
cursor, \kbd{C-Y} yanks back the last killed string.
\kbd{C-underscore} undoes the last change you made; it can be repeated
for cumulative effect.

\section{History Substitution \label{history}}

History substitution works as follows.  All non-empty input lines
issued are saved in a history buffer, and when a new prompt is given
you are positioned on a new line at the bottom of this buffer.
\kbd{C-P} moves one line up (back) in the history buffer,
\kbd{C-N} moves one down.  Any line in the history buffer can be
edited; an asterisk appears in front of the prompt to mark a line as
modified.  Pressing the \kbd{Return} key passes the current line to
the interpreter.  \kbd{C-R} starts an incremental reverse search;
\kbd{C-S} starts a forward search.

\section{Key Bindings \label{keyBindings}}

The key bindings and some other parameters of the Readline library can
be customized by placing commands in an initialization file called
\file{\~{}/.inputrc}.  Key bindings have the form

\begin{verbatim}
key-name: function-name
\end{verbatim}

or

\begin{verbatim}
"string": function-name
\end{verbatim}

and options can be set with

\begin{verbatim}
set option-name value
\end{verbatim}

For example:

\begin{verbatim}
# I prefer vi-style editing:
set editing-mode vi

# Edit using a single line:
set horizontal-scroll-mode On

# Rebind some keys:
Meta-h: backward-kill-word
"\C-u": universal-argument
"\C-x\C-r": re-read-init-file
\end{verbatim}

Note that the default binding for \kbd{Tab} in Python is to insert a
\kbd{Tab} character instead of Readline's default filename completion
function.  If you insist, you can override this by putting

\begin{verbatim}
Tab: complete
\end{verbatim}

in your \file{\~{}/.inputrc}.  (Of course, this makes it harder to
type indented continuation lines if you're accustomed to using
\kbd{Tab} for that purpose.)

Automatic completion of variable and module names is optionally
available.  To enable it in the interpreter's interactive mode, add
the following to your startup file:\footnote{
  Python will execute the contents of a file identified by the
  \envvar{PYTHONSTARTUP} environment variable when you start an
  interactive interpreter.}
\refstmodindex{rlcompleter}\refbimodindex{readline}

\begin{verbatim}
import rlcompleter, readline
readline.parse_and_bind('tab: complete')
\end{verbatim}

This binds the \kbd{Tab} key to the completion function, so hitting
the \kbd{Tab} key twice suggests completions; it looks at Python
statement names, the current local variables, and the available module
names.  For dotted expressions such as \code{string.a}, it will
evaluate the expression up to the final \character{.} and then
suggest completions from the attributes of the resulting object.  Note
that this may execute application-defined code if an object with a
\method{__getattr__()} method is part of the expression.

A more capable startup file might look like this example.  Note that
this deletes the names it creates once they are no longer needed; this
is done since the startup file is executed in the same namespace as
the interactive commands, and removing the names avoids creating side
effects in the interactive environments.  You may find it convenient
to keep some of the imported modules, such as
\ulink{\module{os}}{../lib/module-os.html}, which turn
out to be needed in most sessions with the interpreter.

\begin{verbatim}
# Add auto-completion and a stored history file of commands to your Python
# interactive interpreter. Requires Python 2.0+, readline. Autocomplete is
# bound to the Esc key by default (you can change it - see readline docs).
#
# Store the file in ~/.pystartup, and set an environment variable to point
# to it:  "export PYTHONSTARTUP=/max/home/itamar/.pystartup" in bash.
#
# Note that PYTHONSTARTUP does *not* expand "~", so you have to put in the
# full path to your home directory.

import atexit
import os
import readline
import rlcompleter

historyPath = os.path.expanduser("~/.pyhistory")

def save_history(historyPath=historyPath):
    import readline
    readline.write_history_file(historyPath)

if os.path.exists(historyPath):
    readline.read_history_file(historyPath)

atexit.register(save_history)
del os, atexit, readline, rlcompleter, save_history, historyPath
\end{verbatim}


\section{Commentary \label{commentary}}

This facility is an enormous step forward compared to earlier versions
of the interpreter; however, some wishes are left: It would be nice if
the proper indentation were suggested on continuation lines (the
parser knows if an indent token is required next).  The completion
mechanism might use the interpreter's symbol table.  A command to
check (or even suggest) matching parentheses, quotes, etc., would also
be useful.


\chapter{Floating Point Arithmetic:  Issues and Limitations\label{fp-issues}}
\sectionauthor{Tim Peters}{tim_one@users.sourceforge.net}

Floating-point numbers are represented in computer hardware as
base 2 (binary) fractions.  For example, the decimal fraction

\begin{verbatim}
0.125
\end{verbatim}

has value 1/10 + 2/100 + 5/1000, and in the same way the binary fraction

\begin{verbatim}
0.001
\end{verbatim}

has value 0/2 + 0/4 + 1/8.  These two fractions have identical values,
the only real difference being that the first is written in base 10
fractional notation, and the second in base 2.

Unfortunately, most decimal fractions cannot be represented exactly as
binary fractions.  A consequence is that, in general, the decimal
floating-point numbers you enter are only approximated by the binary
floating-point numbers actually stored in the machine.

The problem is easier to understand at first in base 10.  Consider the
fraction 1/3.  You can approximate that as a base 10 fraction:

\begin{verbatim}
0.3
\end{verbatim}

or, better,

\begin{verbatim}
0.33
\end{verbatim}

or, better,

\begin{verbatim}
0.333
\end{verbatim}

and so on.  No matter how many digits you're willing to write down, the
result will never be exactly 1/3, but will be an increasingly better
approximation to 1/3.

In the same way, no matter how many base 2 digits you're willing to
use, the decimal value 0.1 cannot be represented exactly as a base 2
fraction.  In base 2, 1/10 is the infinitely repeating fraction

\begin{verbatim}
0.0001100110011001100110011001100110011001100110011...
\end{verbatim}

Stop at any finite number of bits, and you get an approximation.  This
is why you see things like:

\begin{verbatim}
>>> 0.1
0.10000000000000001
\end{verbatim}

On most machines today, that is what you'll see if you enter 0.1 at
a Python prompt.  You may not, though, because the number of bits
used by the hardware to store floating-point values can vary across
machines, and Python only prints a decimal approximation to the true
decimal value of the binary approximation stored by the machine.  On
most machines, if Python were to print the true decimal value of
the binary approximation stored for 0.1, it would have to display

\begin{verbatim}
>>> 0.1
0.1000000000000000055511151231257827021181583404541015625
\end{verbatim}

instead!  The Python prompt (implicitly) uses the builtin
\function{repr()} function to obtain a string version of everything it
displays.  For floats, \code{repr(\var{float})} rounds the true
decimal value to 17 significant digits, giving

\begin{verbatim}
0.10000000000000001
\end{verbatim}

\code{repr(\var{float})} produces 17 significant digits because it
turns out that's enough (on most machines) so that
\code{eval(repr(\var{x})) == \var{x}} exactly for all finite floats
\var{x}, but rounding to 16 digits is not enough to make that true.

Note that this is in the very nature of binary floating-point: this is
not a bug in Python, it is not a bug in your code either, and you'll
see the same kind of thing in all languages that support your
hardware's floating-point arithmetic (although some languages may
not \emph{display} the difference by default, or in all output modes).

Python's builtin \function{str()} function produces only 12
significant digits, and you may wish to use that instead.  It's
unusual for \code{eval(str(\var{x}))} to reproduce \var{x}, but the
output may be more pleasant to look at:

\begin{verbatim}
>>> print str(0.1)
0.1
\end{verbatim}

It's important to realize that this is, in a real sense, an illusion:
the value in the machine is not exactly 1/10, you're simply rounding
the \emph{display} of the true machine value.

Other surprises follow from this one.  For example, after seeing

\begin{verbatim}
>>> 0.1
0.10000000000000001
\end{verbatim}

you may be tempted to use the \function{round()} function to chop it
back to the single digit you expect.  But that makes no difference:

\begin{verbatim}
>>> round(0.1, 1)
0.10000000000000001
\end{verbatim}

The problem is that the binary floating-point value stored for "0.1"
was already the best possible binary approximation to 1/10, so trying
to round it again can't make it better:  it was already as good as it
gets.

Another consequence is that since 0.1 is not exactly 1/10, adding 0.1
to itself 10 times may not yield exactly 1.0, either:

\begin{verbatim}
>>> sum = 0.0
>>> for i in range(10):
...     sum += 0.1
...
>>> sum
0.99999999999999989
\end{verbatim}

Binary floating-point arithmetic holds many surprises like this.  The
problem with "0.1" is explained in precise detail below, in the
"Representation Error" section.  See
\citetitle[http://www.lahey.com/float.htm]{The Perils of Floating
Point} for a more complete account of other common surprises.

As that says near the end, ``there are no easy answers.''  Still,
don't be unduly wary of floating-point!  The errors in Python float
operations are inherited from the floating-point hardware, and on most
machines are on the order of no more than 1 part in 2**53 per
operation.  That's more than adequate for most tasks, but you do need
to keep in mind that it's not decimal arithmetic, and that every float
operation can suffer a new rounding error.

While pathological cases do exist, for most casual use of
floating-point arithmetic you'll see the result you expect in the end
if you simply round the display of your final results to the number of
decimal digits you expect.  \function{str()} usually suffices, and for
finer control see the discussion of Python's \code{\%} format
operator: the \code{\%g}, \code{\%f} and \code{\%e} format codes
supply flexible and easy ways to round float results for display.


\section{Representation Error
         \label{fp-error}}

This section explains the ``0.1'' example in detail, and shows how
you can perform an exact analysis of cases like this yourself.  Basic
familiarity with binary floating-point representation is assumed.

\dfn{Representation error} refers to that some (most, actually)
decimal fractions cannot be represented exactly as binary (base 2)
fractions.  This is the chief reason why Python (or Perl, C, \Cpp,
Java, Fortran, and many others) often won't display the exact decimal
number you expect:

\begin{verbatim}
>>> 0.1
0.10000000000000001
\end{verbatim}

Why is that?  1/10 is not exactly representable as a binary fraction.
Almost all machines today (November 2000) use IEEE-754 floating point
arithmetic, and almost all platforms map Python floats to IEEE-754
"double precision".  754 doubles contain 53 bits of precision, so on
input the computer strives to convert 0.1 to the closest fraction it can
of the form \var{J}/2**\var{N} where \var{J} is an integer containing
exactly 53 bits.  Rewriting

\begin{verbatim}
 1 / 10 ~= J / (2**N)
\end{verbatim}

as

\begin{verbatim}
J ~= 2**N / 10
\end{verbatim}

and recalling that \var{J} has exactly 53 bits (is \code{>= 2**52} but
\code{< 2**53}), the best value for \var{N} is 56:

\begin{verbatim}
>>> 2L**52
4503599627370496L
>>> 2L**53
9007199254740992L
>>> 2L**56/10
7205759403792793L
\end{verbatim}

That is, 56 is the only value for \var{N} that leaves \var{J} with
exactly 53 bits.  The best possible value for \var{J} is then that
quotient rounded:

\begin{verbatim}
>>> q, r = divmod(2L**56, 10)
>>> r
6L
\end{verbatim}

Since the remainder is more than half of 10, the best approximation is
obtained by rounding up:

\begin{verbatim}
>>> q+1
7205759403792794L
\end{verbatim}

Therefore the best possible approximation to 1/10 in 754 double
precision is that over 2**56, or

\begin{verbatim}
7205759403792794 / 72057594037927936
\end{verbatim}

Note that since we rounded up, this is actually a little bit larger than
1/10; if we had not rounded up, the quotient would have been a little
bit smaller than 1/10.  But in no case can it be \emph{exactly} 1/10!

So the computer never ``sees'' 1/10:  what it sees is the exact
fraction given above, the best 754 double approximation it can get:

\begin{verbatim}
>>> .1 * 2L**56
7205759403792794.0
\end{verbatim}

If we multiply that fraction by 10**30, we can see the (truncated)
value of its 30 most significant decimal digits:

\begin{verbatim}
>>> 7205759403792794L * 10L**30 / 2L**56
100000000000000005551115123125L
\end{verbatim}

meaning that the exact number stored in the computer is approximately
equal to the decimal value 0.100000000000000005551115123125.  Rounding
that to 17 significant digits gives the 0.10000000000000001 that Python
displays (well, will display on any 754-conforming platform that does
best-possible input and output conversions in its C library --- yours may
not!).

\chapter{History and License}
\section{History of the software}

Python was created in the early 1990s by Guido van Rossum at Stichting
Mathematisch Centrum (CWI, see \url{http://www.cwi.nl/}) in the Netherlands
as a successor of a language called ABC.  Guido remains Python's
principal author, although it includes many contributions from others.

In 1995, Guido continued his work on Python at the Corporation for
National Research Initiatives (CNRI, see \url{http://www.cnri.reston.va.us/})
in Reston, Virginia where he released several versions of the
software.

In May 2000, Guido and the Python core development team moved to
BeOpen.com to form the BeOpen PythonLabs team.  In October of the same
year, the PythonLabs team moved to Digital Creations (now Zope
Corporation; see \url{http://www.zope.com/}).  In 2001, the Python
Software Foundation (PSF, see \url{http://www.python.org/psf/}) was
formed, a non-profit organization created specifically to own
Python-related Intellectual Property.  Zope Corporation is a
sponsoring member of the PSF.

All Python releases are Open Source (see
\url{http://www.opensource.org/} for the Open Source Definition).
Historically, most, but not all, Python releases have also been
GPL-compatible; the table below summarizes the various releases.

\begin{tablev}{c|c|c|c|c}{textrm}%
  {Release}{Derived from}{Year}{Owner}{GPL compatible?}
  \linev{0.9.0 thru 1.2}{n/a}{1991-1995}{CWI}{yes}
  \linev{1.3 thru 1.5.2}{1.2}{1995-1999}{CNRI}{yes}
  \linev{1.6}{1.5.2}{2000}{CNRI}{no}
  \linev{2.0}{1.6}{2000}{BeOpen.com}{no}
  \linev{1.6.1}{1.6}{2001}{CNRI}{no}
  \linev{2.1}{2.0+1.6.1}{2001}{PSF}{no}
  \linev{2.0.1}{2.0+1.6.1}{2001}{PSF}{yes}
  \linev{2.1.1}{2.1+2.0.1}{2001}{PSF}{yes}
  \linev{2.2}{2.1.1}{2001}{PSF}{yes}
  \linev{2.1.2}{2.1.1}{2002}{PSF}{yes}
  \linev{2.1.3}{2.1.2}{2002}{PSF}{yes}
  \linev{2.2.1}{2.2}{2002}{PSF}{yes}
  \linev{2.2.2}{2.2.1}{2002}{PSF}{yes}
  \linev{2.2.3}{2.2.2}{2002-2003}{PSF}{yes}
  \linev{2.3}{2.2.2}{2002-2003}{PSF}{yes}
  \linev{2.3.1}{2.3}{2002-2003}{PSF}{yes}
  \linev{2.3.2}{2.3.1}{2003}{PSF}{yes}
  \linev{2.3.3}{2.3.2}{2003}{PSF}{yes}
  \linev{2.3.4}{2.3.3}{2004}{PSF}{yes}
  \linev{2.3.5}{2.3.4}{2005}{PSF}{yes}
  \linev{2.4}{2.3}{2004}{PSF}{yes}
\end{tablev}

\note{GPL-compatible doesn't mean that we're distributing
Python under the GPL.  All Python licenses, unlike the GPL, let you
distribute a modified version without making your changes open source.
The GPL-compatible licenses make it possible to combine Python with
other software that is released under the GPL; the others don't.}

Thanks to the many outside volunteers who have worked under Guido's
direction to make these releases possible.


\section{Terms and conditions for accessing or otherwise using Python}

\centerline{\strong{PSF LICENSE AGREEMENT FOR PYTHON \version}}

\begin{enumerate}
\item
This LICENSE AGREEMENT is between the Python Software Foundation
(``PSF''), and the Individual or Organization (``Licensee'') accessing
and otherwise using Python \version{} software in source or binary
form and its associated documentation.

\item
Subject to the terms and conditions of this License Agreement, PSF
hereby grants Licensee a nonexclusive, royalty-free, world-wide
license to reproduce, analyze, test, perform and/or display publicly,
prepare derivative works, distribute, and otherwise use Python
\version{} alone or in any derivative version, provided, however, that
PSF's License Agreement and PSF's notice of copyright, i.e.,
``Copyright \copyright{} 2001-2004 Python Software Foundation; All
Rights Reserved'' are retained in Python \version{} alone or in any
derivative version prepared by Licensee.

\item
In the event Licensee prepares a derivative work that is based on
or incorporates Python \version{} or any part thereof, and wants to
make the derivative work available to others as provided herein, then
Licensee hereby agrees to include in any such work a brief summary of
the changes made to Python \version.

\item
PSF is making Python \version{} available to Licensee on an ``AS IS''
basis.  PSF MAKES NO REPRESENTATIONS OR WARRANTIES, EXPRESS OR
IMPLIED.  BY WAY OF EXAMPLE, BUT NOT LIMITATION, PSF MAKES NO AND
DISCLAIMS ANY REPRESENTATION OR WARRANTY OF MERCHANTABILITY OR FITNESS
FOR ANY PARTICULAR PURPOSE OR THAT THE USE OF PYTHON \version{} WILL
NOT INFRINGE ANY THIRD PARTY RIGHTS.

\item
PSF SHALL NOT BE LIABLE TO LICENSEE OR ANY OTHER USERS OF PYTHON
\version{} FOR ANY INCIDENTAL, SPECIAL, OR CONSEQUENTIAL DAMAGES OR
LOSS AS A RESULT OF MODIFYING, DISTRIBUTING, OR OTHERWISE USING PYTHON
\version, OR ANY DERIVATIVE THEREOF, EVEN IF ADVISED OF THE
POSSIBILITY THEREOF.

\item
This License Agreement will automatically terminate upon a material
breach of its terms and conditions.

\item
Nothing in this License Agreement shall be deemed to create any
relationship of agency, partnership, or joint venture between PSF and
Licensee.  This License Agreement does not grant permission to use PSF
trademarks or trade name in a trademark sense to endorse or promote
products or services of Licensee, or any third party.

\item
By copying, installing or otherwise using Python \version, Licensee
agrees to be bound by the terms and conditions of this License
Agreement.
\end{enumerate}


\centerline{\strong{BEOPEN.COM LICENSE AGREEMENT FOR PYTHON 2.0}}

\centerline{\strong{BEOPEN PYTHON OPEN SOURCE LICENSE AGREEMENT VERSION 1}}

\begin{enumerate}
\item
This LICENSE AGREEMENT is between BeOpen.com (``BeOpen''), having an
office at 160 Saratoga Avenue, Santa Clara, CA 95051, and the
Individual or Organization (``Licensee'') accessing and otherwise
using this software in source or binary form and its associated
documentation (``the Software'').

\item
Subject to the terms and conditions of this BeOpen Python License
Agreement, BeOpen hereby grants Licensee a non-exclusive,
royalty-free, world-wide license to reproduce, analyze, test, perform
and/or display publicly, prepare derivative works, distribute, and
otherwise use the Software alone or in any derivative version,
provided, however, that the BeOpen Python License is retained in the
Software, alone or in any derivative version prepared by Licensee.

\item
BeOpen is making the Software available to Licensee on an ``AS IS''
basis.  BEOPEN MAKES NO REPRESENTATIONS OR WARRANTIES, EXPRESS OR
IMPLIED.  BY WAY OF EXAMPLE, BUT NOT LIMITATION, BEOPEN MAKES NO AND
DISCLAIMS ANY REPRESENTATION OR WARRANTY OF MERCHANTABILITY OR FITNESS
FOR ANY PARTICULAR PURPOSE OR THAT THE USE OF THE SOFTWARE WILL NOT
INFRINGE ANY THIRD PARTY RIGHTS.

\item
BEOPEN SHALL NOT BE LIABLE TO LICENSEE OR ANY OTHER USERS OF THE
SOFTWARE FOR ANY INCIDENTAL, SPECIAL, OR CONSEQUENTIAL DAMAGES OR LOSS
AS A RESULT OF USING, MODIFYING OR DISTRIBUTING THE SOFTWARE, OR ANY
DERIVATIVE THEREOF, EVEN IF ADVISED OF THE POSSIBILITY THEREOF.

\item
This License Agreement will automatically terminate upon a material
breach of its terms and conditions.

\item
This License Agreement shall be governed by and interpreted in all
respects by the law of the State of California, excluding conflict of
law provisions.  Nothing in this License Agreement shall be deemed to
create any relationship of agency, partnership, or joint venture
between BeOpen and Licensee.  This License Agreement does not grant
permission to use BeOpen trademarks or trade names in a trademark
sense to endorse or promote products or services of Licensee, or any
third party.  As an exception, the ``BeOpen Python'' logos available
at http://www.pythonlabs.com/logos.html may be used according to the
permissions granted on that web page.

\item
By copying, installing or otherwise using the software, Licensee
agrees to be bound by the terms and conditions of this License
Agreement.
\end{enumerate}


\centerline{\strong{CNRI LICENSE AGREEMENT FOR PYTHON 1.6.1}}

\begin{enumerate}
\item
This LICENSE AGREEMENT is between the Corporation for National
Research Initiatives, having an office at 1895 Preston White Drive,
Reston, VA 20191 (``CNRI''), and the Individual or Organization
(``Licensee'') accessing and otherwise using Python 1.6.1 software in
source or binary form and its associated documentation.

\item
Subject to the terms and conditions of this License Agreement, CNRI
hereby grants Licensee a nonexclusive, royalty-free, world-wide
license to reproduce, analyze, test, perform and/or display publicly,
prepare derivative works, distribute, and otherwise use Python 1.6.1
alone or in any derivative version, provided, however, that CNRI's
License Agreement and CNRI's notice of copyright, i.e., ``Copyright
\copyright{} 1995-2001 Corporation for National Research Initiatives;
All Rights Reserved'' are retained in Python 1.6.1 alone or in any
derivative version prepared by Licensee.  Alternately, in lieu of
CNRI's License Agreement, Licensee may substitute the following text
(omitting the quotes): ``Python 1.6.1 is made available subject to the
terms and conditions in CNRI's License Agreement.  This Agreement
together with Python 1.6.1 may be located on the Internet using the
following unique, persistent identifier (known as a handle):
1895.22/1013.  This Agreement may also be obtained from a proxy server
on the Internet using the following URL:
\url{http://hdl.handle.net/1895.22/1013}.''

\item
In the event Licensee prepares a derivative work that is based on
or incorporates Python 1.6.1 or any part thereof, and wants to make
the derivative work available to others as provided herein, then
Licensee hereby agrees to include in any such work a brief summary of
the changes made to Python 1.6.1.

\item
CNRI is making Python 1.6.1 available to Licensee on an ``AS IS''
basis.  CNRI MAKES NO REPRESENTATIONS OR WARRANTIES, EXPRESS OR
IMPLIED.  BY WAY OF EXAMPLE, BUT NOT LIMITATION, CNRI MAKES NO AND
DISCLAIMS ANY REPRESENTATION OR WARRANTY OF MERCHANTABILITY OR FITNESS
FOR ANY PARTICULAR PURPOSE OR THAT THE USE OF PYTHON 1.6.1 WILL NOT
INFRINGE ANY THIRD PARTY RIGHTS.

\item
CNRI SHALL NOT BE LIABLE TO LICENSEE OR ANY OTHER USERS OF PYTHON
1.6.1 FOR ANY INCIDENTAL, SPECIAL, OR CONSEQUENTIAL DAMAGES OR LOSS AS
A RESULT OF MODIFYING, DISTRIBUTING, OR OTHERWISE USING PYTHON 1.6.1,
OR ANY DERIVATIVE THEREOF, EVEN IF ADVISED OF THE POSSIBILITY THEREOF.

\item
This License Agreement will automatically terminate upon a material
breach of its terms and conditions.

\item
This License Agreement shall be governed by the federal
intellectual property law of the United States, including without
limitation the federal copyright law, and, to the extent such
U.S. federal law does not apply, by the law of the Commonwealth of
Virginia, excluding Virginia's conflict of law provisions.
Notwithstanding the foregoing, with regard to derivative works based
on Python 1.6.1 that incorporate non-separable material that was
previously distributed under the GNU General Public License (GPL), the
law of the Commonwealth of Virginia shall govern this License
Agreement only as to issues arising under or with respect to
Paragraphs 4, 5, and 7 of this License Agreement.  Nothing in this
License Agreement shall be deemed to create any relationship of
agency, partnership, or joint venture between CNRI and Licensee.  This
License Agreement does not grant permission to use CNRI trademarks or
trade name in a trademark sense to endorse or promote products or
services of Licensee, or any third party.

\item
By clicking on the ``ACCEPT'' button where indicated, or by copying,
installing or otherwise using Python 1.6.1, Licensee agrees to be
bound by the terms and conditions of this License Agreement.
\end{enumerate}

\centerline{ACCEPT}



\centerline{\strong{CWI LICENSE AGREEMENT FOR PYTHON 0.9.0 THROUGH 1.2}}

Copyright \copyright{} 1991 - 1995, Stichting Mathematisch Centrum
Amsterdam, The Netherlands.  All rights reserved.

Permission to use, copy, modify, and distribute this software and its
documentation for any purpose and without fee is hereby granted,
provided that the above copyright notice appear in all copies and that
both that copyright notice and this permission notice appear in
supporting documentation, and that the name of Stichting Mathematisch
Centrum or CWI not be used in advertising or publicity pertaining to
distribution of the software without specific, written prior
permission.

STICHTING MATHEMATISCH CENTRUM DISCLAIMS ALL WARRANTIES WITH REGARD TO
THIS SOFTWARE, INCLUDING ALL IMPLIED WARRANTIES OF MERCHANTABILITY AND
FITNESS, IN NO EVENT SHALL STICHTING MATHEMATISCH CENTRUM BE LIABLE
FOR ANY SPECIAL, INDIRECT OR CONSEQUENTIAL DAMAGES OR ANY DAMAGES
WHATSOEVER RESULTING FROM LOSS OF USE, DATA OR PROFITS, WHETHER IN AN
ACTION OF CONTRACT, NEGLIGENCE OR OTHER TORTIOUS ACTION, ARISING OUT
OF OR IN CONNECTION WITH THE USE OR PERFORMANCE OF THIS SOFTWARE.


\section{Licenses and Acknowledgements for Incorporated Software}

This section is an incomplete, but growing list of licenses and
acknowledgements for third-party software incorporated in the
Python distribution.


\subsection{Mersenne Twister}

The \module{_random} module includes code based on a download from
\url{http://www.math.keio.ac.jp/~matumoto/MT2002/emt19937ar.html}.
The following are the verbatim comments from the original code:

\begin{verbatim}
A C-program for MT19937, with initialization improved 2002/1/26.
Coded by Takuji Nishimura and Makoto Matsumoto.

Before using, initialize the state by using init_genrand(seed)
or init_by_array(init_key, key_length).

Copyright (C) 1997 - 2002, Makoto Matsumoto and Takuji Nishimura,
All rights reserved.

Redistribution and use in source and binary forms, with or without
modification, are permitted provided that the following conditions
are met:

 1. Redistributions of source code must retain the above copyright
    notice, this list of conditions and the following disclaimer.

 2. Redistributions in binary form must reproduce the above copyright
    notice, this list of conditions and the following disclaimer in the
    documentation and/or other materials provided with the distribution.

 3. The names of its contributors may not be used to endorse or promote
    products derived from this software without specific prior written
    permission.

THIS SOFTWARE IS PROVIDED BY THE COPYRIGHT HOLDERS AND CONTRIBUTORS
"AS IS" AND ANY EXPRESS OR IMPLIED WARRANTIES, INCLUDING, BUT NOT
LIMITED TO, THE IMPLIED WARRANTIES OF MERCHANTABILITY AND FITNESS FOR
A PARTICULAR PURPOSE ARE DISCLAIMED.  IN NO EVENT SHALL THE COPYRIGHT OWNER OR
CONTRIBUTORS BE LIABLE FOR ANY DIRECT, INDIRECT, INCIDENTAL, SPECIAL,
EXEMPLARY, OR CONSEQUENTIAL DAMAGES (INCLUDING, BUT NOT LIMITED TO,
PROCUREMENT OF SUBSTITUTE GOODS OR SERVICES; LOSS OF USE, DATA, OR
PROFITS; OR BUSINESS INTERRUPTION) HOWEVER CAUSED AND ON ANY THEORY OF
LIABILITY, WHETHER IN CONTRACT, STRICT LIABILITY, OR TORT (INCLUDING
NEGLIGENCE OR OTHERWISE) ARISING IN ANY WAY OUT OF THE USE OF THIS
SOFTWARE, EVEN IF ADVISED OF THE POSSIBILITY OF SUCH DAMAGE.


Any feedback is very welcome.
http://www.math.keio.ac.jp/matumoto/emt.html
email: matumoto@math.keio.ac.jp
\end{verbatim}



\subsection{Sockets}

The \module{socket} module uses the functions, \function{getaddrinfo},
and \function{getnameinfo}, which are coded in separate source files
from the WIDE Project, \url{http://www.wide.ad.jp/about/index.html}.

\begin{verbatim}      
Copyright (C) 1995, 1996, 1997, and 1998 WIDE Project.
All rights reserved.
 
Redistribution and use in source and binary forms, with or without
modification, are permitted provided that the following conditions
are met:
1. Redistributions of source code must retain the above copyright
   notice, this list of conditions and the following disclaimer.
2. Redistributions in binary form must reproduce the above copyright
   notice, this list of conditions and the following disclaimer in the
   documentation and/or other materials provided with the distribution.
3. Neither the name of the project nor the names of its contributors
   may be used to endorse or promote products derived from this software
   without specific prior written permission.

THIS SOFTWARE IS PROVIDED BY THE PROJECT AND CONTRIBUTORS ``AS IS'' AND
GAI_ANY EXPRESS OR IMPLIED WARRANTIES, INCLUDING, BUT NOT LIMITED TO, THE
IMPLIED WARRANTIES OF MERCHANTABILITY AND FITNESS FOR A PARTICULAR PURPOSE
ARE DISCLAIMED.  IN NO EVENT SHALL THE PROJECT OR CONTRIBUTORS BE LIABLE
FOR GAI_ANY DIRECT, INDIRECT, INCIDENTAL, SPECIAL, EXEMPLARY, OR CONSEQUENTIAL
DAMAGES (INCLUDING, BUT NOT LIMITED TO, PROCUREMENT OF SUBSTITUTE GOODS
OR SERVICES; LOSS OF USE, DATA, OR PROFITS; OR BUSINESS INTERRUPTION)
HOWEVER CAUSED AND ON GAI_ANY THEORY OF LIABILITY, WHETHER IN CONTRACT, STRICT
LIABILITY, OR TORT (INCLUDING NEGLIGENCE OR OTHERWISE) ARISING IN GAI_ANY WAY
OUT OF THE USE OF THIS SOFTWARE, EVEN IF ADVISED OF THE POSSIBILITY OF
SUCH DAMAGE.
\end{verbatim}



\subsection{Floating point exception control}

The source for the \module{fpectl} module includes the following notice:

\begin{verbatim}
     ---------------------------------------------------------------------  
    /                       Copyright (c) 1996.                           \ 
   |          The Regents of the University of California.                 |
   |                        All rights reserved.                           |
   |                                                                       |
   |   Permission to use, copy, modify, and distribute this software for   |
   |   any purpose without fee is hereby granted, provided that this en-   |
   |   tire notice is included in all copies of any software which is or   |
   |   includes  a  copy  or  modification  of  this software and in all   |
   |   copies of the supporting documentation for such software.           |
   |                                                                       |
   |   This  work was produced at the University of California, Lawrence   |
   |   Livermore National Laboratory under  contract  no.  W-7405-ENG-48   |
   |   between  the  U.S.  Department  of  Energy and The Regents of the   |
   |   University of California for the operation of UC LLNL.              |
   |                                                                       |
   |                              DISCLAIMER                               |
   |                                                                       |
   |   This  software was prepared as an account of work sponsored by an   |
   |   agency of the United States Government. Neither the United States   |
   |   Government  nor the University of California nor any of their em-   |
   |   ployees, makes any warranty, express or implied, or  assumes  any   |
   |   liability  or  responsibility  for the accuracy, completeness, or   |
   |   usefulness of any information,  apparatus,  product,  or  process   |
   |   disclosed,   or  represents  that  its  use  would  not  infringe   |
   |   privately-owned rights. Reference herein to any specific  commer-   |
   |   cial  products,  process,  or  service  by trade name, trademark,   |
   |   manufacturer, or otherwise, does not  necessarily  constitute  or   |
   |   imply  its endorsement, recommendation, or favoring by the United   |
   |   States Government or the University of California. The views  and   |
   |   opinions  of authors expressed herein do not necessarily state or   |
   |   reflect those of the United States Government or  the  University   |
   |   of  California,  and shall not be used for advertising or product   |
    \  endorsement purposes.                                              / 
     ---------------------------------------------------------------------
\end{verbatim}



\subsection{MD5 message digest algorithm}

The source code for the \module{md5} module contains the following notice:

\begin{verbatim}
Copyright (C) 1991-2, RSA Data Security, Inc. Created 1991. All
rights reserved.

License to copy and use this software is granted provided that it
is identified as the "RSA Data Security, Inc. MD5 Message-Digest
Algorithm" in all material mentioning or referencing this software
or this function.

License is also granted to make and use derivative works provided
that such works are identified as "derived from the RSA Data
Security, Inc. MD5 Message-Digest Algorithm" in all material
mentioning or referencing the derived work.

RSA Data Security, Inc. makes no representations concerning either
the merchantability of this software or the suitability of this
software for any particular purpose. It is provided "as is"
without express or implied warranty of any kind.

These notices must be retained in any copies of any part of this
documentation and/or software.
\end{verbatim}



\subsection{Asynchronous socket services}

The \module{asynchat} and \module{asyncore} modules contain the
following notice:

\begin{verbatim}      
 Copyright 1996 by Sam Rushing

                         All Rights Reserved

 Permission to use, copy, modify, and distribute this software and
 its documentation for any purpose and without fee is hereby
 granted, provided that the above copyright notice appear in all
 copies and that both that copyright notice and this permission
 notice appear in supporting documentation, and that the name of Sam
 Rushing not be used in advertising or publicity pertaining to
 distribution of the software without specific, written prior
 permission.

 SAM RUSHING DISCLAIMS ALL WARRANTIES WITH REGARD TO THIS SOFTWARE,
 INCLUDING ALL IMPLIED WARRANTIES OF MERCHANTABILITY AND FITNESS, IN
 NO EVENT SHALL SAM RUSHING BE LIABLE FOR ANY SPECIAL, INDIRECT OR
 CONSEQUENTIAL DAMAGES OR ANY DAMAGES WHATSOEVER RESULTING FROM LOSS
 OF USE, DATA OR PROFITS, WHETHER IN AN ACTION OF CONTRACT,
 NEGLIGENCE OR OTHER TORTIOUS ACTION, ARISING OUT OF OR IN
 CONNECTION WITH THE USE OR PERFORMANCE OF THIS SOFTWARE.
\end{verbatim}


\subsection{Cookie management}

The \module{Cookie} module contains the following notice:

\begin{verbatim}
 Copyright 2000 by Timothy O'Malley <timo@alum.mit.edu>

                All Rights Reserved

 Permission to use, copy, modify, and distribute this software
 and its documentation for any purpose and without fee is hereby
 granted, provided that the above copyright notice appear in all
 copies and that both that copyright notice and this permission
 notice appear in supporting documentation, and that the name of
 Timothy O'Malley  not be used in advertising or publicity
 pertaining to distribution of the software without specific, written
 prior permission.

 Timothy O'Malley DISCLAIMS ALL WARRANTIES WITH REGARD TO THIS
 SOFTWARE, INCLUDING ALL IMPLIED WARRANTIES OF MERCHANTABILITY
 AND FITNESS, IN NO EVENT SHALL Timothy O'Malley BE LIABLE FOR
 ANY SPECIAL, INDIRECT OR CONSEQUENTIAL DAMAGES OR ANY DAMAGES
 WHATSOEVER RESULTING FROM LOSS OF USE, DATA OR PROFITS,
 WHETHER IN AN ACTION OF CONTRACT, NEGLIGENCE OR OTHER TORTIOUS
 ACTION, ARISING OUT OF OR IN CONNECTION WITH THE USE OR
 PERFORMANCE OF THIS SOFTWARE.
\end{verbatim}      



\subsection{Profiling}

The \module{profile} and \module{pstats} modules contain
the following notice:

\begin{verbatim}
 Copyright 1994, by InfoSeek Corporation, all rights reserved.
 Written by James Roskind

 Permission to use, copy, modify, and distribute this Python software
 and its associated documentation for any purpose (subject to the
 restriction in the following sentence) without fee is hereby granted,
 provided that the above copyright notice appears in all copies, and
 that both that copyright notice and this permission notice appear in
 supporting documentation, and that the name of InfoSeek not be used in
 advertising or publicity pertaining to distribution of the software
 without specific, written prior permission.  This permission is
 explicitly restricted to the copying and modification of the software
 to remain in Python, compiled Python, or other languages (such as C)
 wherein the modified or derived code is exclusively imported into a
 Python module.

 INFOSEEK CORPORATION DISCLAIMS ALL WARRANTIES WITH REGARD TO THIS
 SOFTWARE, INCLUDING ALL IMPLIED WARRANTIES OF MERCHANTABILITY AND
 FITNESS. IN NO EVENT SHALL INFOSEEK CORPORATION BE LIABLE FOR ANY
 SPECIAL, INDIRECT OR CONSEQUENTIAL DAMAGES OR ANY DAMAGES WHATSOEVER
 RESULTING FROM LOSS OF USE, DATA OR PROFITS, WHETHER IN AN ACTION OF
 CONTRACT, NEGLIGENCE OR OTHER TORTIOUS ACTION, ARISING OUT OF OR IN
 CONNECTION WITH THE USE OR PERFORMANCE OF THIS SOFTWARE.
\end{verbatim}



\subsection{Execution tracing}

The \module{trace} module contains the following notice:

\begin{verbatim}
 portions copyright 2001, Autonomous Zones Industries, Inc., all rights...
 err...  reserved and offered to the public under the terms of the
 Python 2.2 license.
 Author: Zooko O'Whielacronx
 http://zooko.com/
 mailto:zooko@zooko.com

 Copyright 2000, Mojam Media, Inc., all rights reserved.
 Author: Skip Montanaro

 Copyright 1999, Bioreason, Inc., all rights reserved.
 Author: Andrew Dalke

 Copyright 1995-1997, Automatrix, Inc., all rights reserved.
 Author: Skip Montanaro

 Copyright 1991-1995, Stichting Mathematisch Centrum, all rights reserved.


 Permission to use, copy, modify, and distribute this Python software and
 its associated documentation for any purpose without fee is hereby
 granted, provided that the above copyright notice appears in all copies,
 and that both that copyright notice and this permission notice appear in
 supporting documentation, and that the name of neither Automatrix,
 Bioreason or Mojam Media be used in advertising or publicity pertaining to
 distribution of the software without specific, written prior permission.
\end{verbatim} 



\subsection{UUencode and UUdecode functions}

The \module{uu} module contains the following notice:

\begin{verbatim}
 Copyright 1994 by Lance Ellinghouse
 Cathedral City, California Republic, United States of America.
                        All Rights Reserved
 Permission to use, copy, modify, and distribute this software and its
 documentation for any purpose and without fee is hereby granted,
 provided that the above copyright notice appear in all copies and that
 both that copyright notice and this permission notice appear in
 supporting documentation, and that the name of Lance Ellinghouse
 not be used in advertising or publicity pertaining to distribution
 of the software without specific, written prior permission.
 LANCE ELLINGHOUSE DISCLAIMS ALL WARRANTIES WITH REGARD TO
 THIS SOFTWARE, INCLUDING ALL IMPLIED WARRANTIES OF MERCHANTABILITY AND
 FITNESS, IN NO EVENT SHALL LANCE ELLINGHOUSE CENTRUM BE LIABLE
 FOR ANY SPECIAL, INDIRECT OR CONSEQUENTIAL DAMAGES OR ANY DAMAGES
 WHATSOEVER RESULTING FROM LOSS OF USE, DATA OR PROFITS, WHETHER IN AN
 ACTION OF CONTRACT, NEGLIGENCE OR OTHER TORTIOUS ACTION, ARISING OUT
 OF OR IN CONNECTION WITH THE USE OR PERFORMANCE OF THIS SOFTWARE.

 Modified by Jack Jansen, CWI, July 1995:
 - Use binascii module to do the actual line-by-line conversion
   between ascii and binary. This results in a 1000-fold speedup. The C
   version is still 5 times faster, though.
 - Arguments more compliant with python standard
\end{verbatim}



\subsection{XML Remote Procedure Calls}

The \module{xmlrpclib} module contains the following notice:

\begin{verbatim}
     The XML-RPC client interface is

 Copyright (c) 1999-2002 by Secret Labs AB
 Copyright (c) 1999-2002 by Fredrik Lundh

 By obtaining, using, and/or copying this software and/or its
 associated documentation, you agree that you have read, understood,
 and will comply with the following terms and conditions:

 Permission to use, copy, modify, and distribute this software and
 its associated documentation for any purpose and without fee is
 hereby granted, provided that the above copyright notice appears in
 all copies, and that both that copyright notice and this permission
 notice appear in supporting documentation, and that the name of
 Secret Labs AB or the author not be used in advertising or publicity
 pertaining to distribution of the software without specific, written
 prior permission.

 SECRET LABS AB AND THE AUTHOR DISCLAIMS ALL WARRANTIES WITH REGARD
 TO THIS SOFTWARE, INCLUDING ALL IMPLIED WARRANTIES OF MERCHANT-
 ABILITY AND FITNESS.  IN NO EVENT SHALL SECRET LABS AB OR THE AUTHOR
 BE LIABLE FOR ANY SPECIAL, INDIRECT OR CONSEQUENTIAL DAMAGES OR ANY
 DAMAGES WHATSOEVER RESULTING FROM LOSS OF USE, DATA OR PROFITS,
 WHETHER IN AN ACTION OF CONTRACT, NEGLIGENCE OR OTHER TORTIOUS
 ACTION, ARISING OUT OF OR IN CONNECTION WITH THE USE OR PERFORMANCE
 OF THIS SOFTWARE.
\end{verbatim}


\chapter{Glossary\label{glossary}}

%%% keep the entries sorted and include at least one \index{} item for each
%%% cross-references are marked with \emph{entry}

\begin{description}


\index{>>>}
\item[\code{>\code{>}>}]
The typical Python prompt of the interactive shell.  Often seen for
code examples that can be tried right away in the interpreter.

\index{...}
\item[\code{.\code{.}.}]
The typical Python prompt of the interactive shell when entering code
for an indented code block.

\index{BDFL}
\item[BDFL]
Benevolent Dictator For Life, a.k.a. \ulink{Guido van
Rossum}{http://www.python.org/\textasciitilde{}guido/}, Python's creator.

\index{byte code}
\item[byte code]
The internal representation of a Python program in the interpreter.
The byte code is also cached in the \code{.pyc} and \code{.pyo}
files so that executing the same file is faster the second time
(compilation from source to byte code can be saved).  This
``intermediate language'' is said to run on a ``virtual
machine'' that calls the subroutines corresponding to each bytecode.

\index{classic class}
\item[classic class]
Any class which does not inherit from \class{object}.  See
\emph{new-style class}.

\index{coercion}
\item[coercion]
Converting data from one type to another.  For example,
{}\code{int(3.15)} coerces the floating point number to the integer,
{}\code{3}.  Most mathematical operations have rules for coercing
their arguments to a common type.  For instance, adding \code{3+4.5},
causes the integer \code{3} to be coerced to be a float
{}\code{3.0} before adding to \code{4.5} resulting in the float
{}\code{7.5}.

\index{descriptor}
\item[descriptor]
Any \emph{new-style} object that defines the methods
{}\method{__get__()}, \method{__set__()}, or \method{__delete__()}.
When a class attribute is a descriptor, its special binding behavior
is triggered upon attribute lookup.  Normally, writing \var{a.b} looks
up the object \var{b} in the class dictionary for \var{a}, but if
{}\var{b} is a descriptor, the defined method gets called.
Understanding descriptors is a key to a deep understanding of Python
because they are the basis for many features including functions,
methods, properties, class methods, static methods, and reference to
super classes.

\index{dictionary}
\item[dictionary]
An associative array, where arbitrary keys are mapped to values.  The
use of \class{dict} much resembles that for \class{list}, but the keys
can be any object with a \method{__hash__()} function, not just
integers starting from zero.  Called a hash in Perl.

\index{EAFP}
\item[EAFP]
Easier to ask for forgiveness than permission.  This common Python
coding style assumes the existence of valid keys or attributes and
catches exceptions if the assumption proves false.  This clean and
fast style is characterized by the presence of many \keyword{try} and
{}\keyword{except} statements.  The technique contrasts with the
{}\emph{LBYL} style that is common in many other languages such as C.

\index{__future__}
\item[__future__]
A pseudo module which programmers can use to enable new language
features which are not compatible with the current interpreter.  For
example, the expression \code{11/4} currently evaluates to \code{2}.
If the module in which it is executed had enabled \emph{true division}
by executing:

\begin{verbatim}
from __future__ import division
\end{verbatim}

the expression \code{11/4} would evaluate to \code{2.75}.  By actually
importing the \ulink{\module{__future__}}{../lib/module-future.html}
module and evaluating its variables, you can see when a new feature
was first added to the language and when it will become the default:

\begin{verbatim}
>>> import __future__
>>> __future__.division
_Feature((2, 2, 0, 'alpha', 2), (3, 0, 0, 'alpha', 0), 8192)
\end{verbatim}

\index{generator}
\item[generator]
A function that returns an iterator.  It looks like a normal function
except that the \keyword{yield} keyword is used instead of
{}\keyword{return}.  Generator functions often contain one or more
{}\keyword{for} or \keyword{while} loops that \keyword{yield} elements
back to the caller.  The function execution is stopped at the
{}\keyword{yield} keyword (returning the result) and is resumed there
when the next element is requested by calling the \method{next()}
method of the returned iterator.

\index{GIL}
\item[GIL]
See \emph{global interpreter lock}.

\index{global interpreter lock}
\item[global interpreter lock]
The lock used by Python threads to assure that only one thread can be
run at a time.  This simplifies Python by assuring that no two
processes can access the same memory at the same time.  Locking the
entire interpreter makes it easier for the interpreter to be
multi-threaded, at the expense of some parallelism on multi-processor
machines.  Efforts have been made in the past to create a
``free-threaded'' interpreter (one which locks shared data at a much
finer granularity), but performance suffered in the common
single-processor case.

\index{IDLE}
\item[IDLE]
An Integrated Development Environment for Python.  IDLE is a
basic editor and interpreter environment that ships with the standard
distribution of Python.  Good for beginners, it also serves as clear
example code for those wanting to implement a moderately
sophisticated, multi-platform GUI application.

\index{immutable}
\item[immutable]
A object with fixed value.  Immutable objects are numbers, strings or
tuples (and more).  Such an object cannot be altered.  A new object
has to be created if a different value has to be stored.  They play an
important role in places where a constant hash value is needed.  For
example as a key in a dictionary.

\index{integer division}
\item[integer division]
Mathematical division discarding any remainder.  For example, the
expression \code{11/4} currently evaluates to \code{2} in contrast
to the \code{2.75} returned by float division.  Also called
{}\emph{floor division}.  When dividing two integers the outcome will
always be another integer (having the floor function applied to it).
However, if one of the operands is another numeric type (such as a
{}\class{float}), the result will be coerced (see \emph{coercion}) to
a common type.  For example, a integer divided by a float will result
in a float value, possibly with a decimal fraction.  Integer division
can be forced by using the \code{//} operator instead of the \code{/}
operator.  See also \emph{__future__}.

\index{interactive}
\item[interactive]
Python has an interactive interpreter which means that you can try out
things and directly see its result.  Just launch \code{python} with no
arguments (possibly by selecting it from your computer's main menu).
It is a very powerful way to test out new ideas or inspect modules and
packages (remember \code{help(x)}).

\index{interpreted}
\item[interpreted]
Python is an interpreted language, opposed to a compiled one.  This
means that the source files can be run right away without first making
an executable which is then run.  Interpreted languages typically have
a shorter development/debug cycle than compiled ones.  See also
{}\emph{interactive}.

\index{iterable}
\item[iterable]
A container object capable of returning its members one at a time.
Examples of iterables include all sequence types (such as \class{list},
{}\class{str}, and \class{tuple}) and some non-sequence types like
{}\class{dict} and \class{file} and objects of any classes you define
with an \method{__iter__()} or \method{__getitem__()} method.  Iterables
can be used in a \keyword{for} loop and in many other places where a
sequence is needed (\function{zip()}, \function{map()}, ...).  When an
iterable object is passed as an argument to the builtin function
{}\function{iter()}, it returns an iterator for the object.  This
iterator is good for one pass over the set of values.  When using
iterables, it is usually not necessary to call \function{iter()} or
deal with iterator objects yourself.  The \code{for} statement does
that automatically for you, creating a temporary unnamed variable to
hold the iterator for the duration of the loop.  See also
{}\emph{iterator}, \emph{sequence}, and \emph{generator}.

\index{iterator}
\item[iterator]
An object representing a stream of data.  Repeated calls to the
iterator's \method{next()} method return successive items in the
stream.  When no more data is available a \exception{StopIteration}
exception is raised instead.  At this point, the iterator object is
exhausted and any further calls to its \method{next()} method just
raise \exception{StopIteration} again.  Iterators are required to have
an \method{__iter__()} method that returns the iterator object
itself so every iterator is also iterable and may be used in most
places where other iterables are accepted.  One notable exception is
code that attempts multiple iteration passes.  A container object
(such as a \class{list}) produces a fresh new iterator each time you
pass it to the \function{iter()} function or use it in a
{}\keyword{for} loop.  Attempting this with an iterator will just
return the same exhausted iterator object from the second iteration
pass, making it appear like an empty container.

\index{list comprehension}
\item[list comprehension]
A compact way to process all or a subset of elements in a sequence and
return a list with the results.  \code{result = ["0x\%02x"
\% x for x in range(256) if x \% 2 == 0]} generates a list of strings
containing hex numbers (0x..) that are even and in the range from 0 to 255.
The \keyword{if} clause is optional.  If omitted, all elements in
{}\code{range(256)} are processed in that case.

\index{mapping}
\item[mapping]
A container object (such as \class{dict}) that supports arbitrary key
lookups using the special method \method{__getitem__()}.

\index{metaclass}
\item[metaclass]
The class of a class.  Class definitions create a class name, a class
dictionary, and a list of base classes.  The metaclass is responsible
for taking those three arguments and creating the class.  Most object
oriented programming languages provide a default implementation.  What
makes Python special is that it is possible to create custom
metaclasses.  Most users never need this tool, but when the need
arises, metaclasses can provide powerful, elegant solutions.  They
have been used for logging attribute access, adding thread-safety,
tracking object creation, implementing singletons, and many other
tasks.

\index{LBYL}
\item[LBYL]
Look before you leap.  This coding style explicitly tests for
pre-conditions before making calls or lookups.  This style contrasts
with the \emph{EAFP} approach and is characterized the presence of
many \keyword{if} statements.

\index{mutable}
\item[mutable]
Mutable objects can change their value but keep their \function{id()}.
See also \emph{immutable}.

\index{namespace}
\item[namespace]
The place where a variable is stored.  Namespaces are implemented as
dictionary.  There is the local, global and builtins namespace and the
nested namespaces in objects (in methods).  Namespaces support
modularity by preventing naming conflicts.  For instance, the
functions \function{__builtin__.open()} and \function{os.open()} are
distinguished by their namespaces.  Namespaces also aid readability
and maintainability by making it clear which modules implement a
function.  For instance, writing \function{random.seed()} or
{}\function{itertools.izip()} makes it clear that those functions are
implemented by the \ulink{\module{random}}{../lib/module-random.html}
and \ulink{\module{itertools}}{../lib/module-itertools.html} modules
respectively.

\index{nested scope}
\item[nested scope]
The ability to refer to a variable in an enclosing definition.  For
instance, a function defined inside another function can refer to
variables in the outer function.  Note that nested scopes work only
for reference and not for assignment which will always write to the
innermost scope.  In contrast, local variables both read and write in
the innermost scope.  Likewise, global variables read and write to the
global namespace.

\index{new-style class}
\item[new-style class]
Any class that inherits from \class{object}.  This includes all
built-in types like \class{list} and \class{dict}.  Only new-style
classes can use Python's newer, versatile features like
{}\method{__slots__}, descriptors, properties,
\method{__getattribute__()}, class methods, and static methods.

\index{Python3000}
\item[Python3000]
A mythical python release, allowed not to be backward compatible, with
telepathic interface.

\index{__slots__}
\item[__slots__]
A declaration inside a \emph{new-style class} that saves memory by
pre-declaring space for instance attributes and eliminating instance
dictionaries.  Though popular, the technique is somewhat tricky to get
right and is best reserved for rare cases where there are large
numbers of instances in a memory critical application.

\index{sequence}
\item[sequence]
An \emph{iterable} which supports efficient element access using
integer indices via the \method{__getitem__()} and
{}\method{__len__()} special methods.  Some built-in sequence types
are \class{list}, \class{str}, \class{tuple}, and \class{unicode}.
Note that \class{dict} also supports \method{__getitem__()} and
{}\method{__len__()}, but is considered a mapping rather than a
sequence because the lookups use arbitrary \emph{immutable} keys
rather than integers.

\index{Zen of Python}
\item[Zen of Python]
Listing of Python design principles and philosophies that are helpful
in understanding and using the language.  The listing can be found by
typing ``\code{import this}'' at the interactive prompt.

\end{description}


\documentclass{manual}
\usepackage[T1]{fontenc}

% Things to do:
% Add a section on file I/O
% Write a chapter entitled ``Some Useful Modules''
%  --re, math+cmath
% Should really move the Python startup file info to an appendix

\title{Python Tutorial}

\author{
	Guido van Rossum \\
	Dept. AA, CWI, P.O. Box 94079 \\
	1090 GB Amsterdam, The Netherlands \\
	E-mail: {\tt guido@cwi.nl}
}

\date{17 March 1995 \\ Release 1.2-proof-2} % XXX update before release!


\begin{document}

\maketitle

\ifhtml
\chapter*{Front Matter\label{front}}
\fi

\strong{BEOPEN.COM TERMS AND CONDITIONS FOR PYTHON 2.0}

\centerline{\strong{BEOPEN PYTHON OPEN SOURCE LICENSE AGREEMENT VERSION 1}}

\begin{enumerate}

\item
This LICENSE AGREEMENT is between BeOpen.com (``BeOpen''), having an
office at 160 Saratoga Avenue, Santa Clara, CA 95051, and the
Individual or Organization (``Licensee'') accessing and otherwise
using this software in source or binary form and its associated
documentation (``the Software'').

\item
Subject to the terms and conditions of this BeOpen Python License
Agreement, BeOpen hereby grants Licensee a non-exclusive,
royalty-free, world-wide license to reproduce, analyze, test, perform
and/or display publicly, prepare derivative works, distribute, and
otherwise use the Software alone or in any derivative version,
provided, however, that the BeOpen Python License is retained in the
Software, alone or in any derivative version prepared by Licensee.

\item
BeOpen is making the Software available to Licensee on an ``AS IS''
basis.  BEOPEN MAKES NO REPRESENTATIONS OR WARRANTIES, EXPRESS OR
IMPLIED.  BY WAY OF EXAMPLE, BUT NOT LIMITATION, BEOPEN MAKES NO AND
DISCLAIMS ANY REPRESENTATION OR WARRANTY OF MERCHANTABILITY OR FITNESS
FOR ANY PARTICULAR PURPOSE OR THAT THE USE OF THE SOFTWARE WILL NOT
INFRINGE ANY THIRD PARTY RIGHTS.

\item
BEOPEN SHALL NOT BE LIABLE TO LICENSEE OR ANY OTHER USERS OF THE
SOFTWARE FOR ANY INCIDENTAL, SPECIAL, OR CONSEQUENTIAL DAMAGES OR LOSS
AS A RESULT OF USING, MODIFYING OR DISTRIBUTING THE SOFTWARE, OR ANY
DERIVATIVE THEREOF, EVEN IF ADVISED OF THE POSSIBILITY THEREOF.

\item
This License Agreement will automatically terminate upon a material
breach of its terms and conditions.

\item
This License Agreement shall be governed by and interpreted in all
respects by the law of the State of California, excluding conflict of
law provisions.  Nothing in this License Agreement shall be deemed to
create any relationship of agency, partnership, or joint venture
between BeOpen and Licensee.  This License Agreement does not grant
permission to use BeOpen trademarks or trade names in a trademark
sense to endorse or promote products or services of Licensee, or any
third party.  As an exception, the ``BeOpen Python'' logos available
at http://www.pythonlabs.com/logos.html may be used according to the
permissions granted on that web page.

\item
By copying, installing or otherwise using the software, Licensee
agrees to be bound by the terms and conditions of this License
Agreement.
\end{enumerate}


\centerline{\strong{CNRI OPEN SOURCE LICENSE AGREEMENT}}

Python 1.6 is made available subject to the terms and conditions in
CNRI's License Agreement.  This Agreement together with Python 1.6 may
be located on the Internet using the following unique, persistent
identifier (known as a handle): 1895.22/1012.  This Agreement may also
be obtained from a proxy server on the Internet using the following
URL: \url{http://hdl.handle.net/1895.22/1012}.


\centerline{\strong{CWI PERMISSIONS STATEMENT AND DISCLAIMER}}

Copyright \copyright{} 1991 - 1995, Stichting Mathematisch Centrum
Amsterdam, The Netherlands.  All rights reserved.

Permission to use, copy, modify, and distribute this software and its
documentation for any purpose and without fee is hereby granted,
provided that the above copyright notice appear in all copies and that
both that copyright notice and this permission notice appear in
supporting documentation, and that the name of Stichting Mathematisch
Centrum or CWI not be used in advertising or publicity pertaining to
distribution of the software without specific, written prior
permission.

STICHTING MATHEMATISCH CENTRUM DISCLAIMS ALL WARRANTIES WITH REGARD TO
THIS SOFTWARE, INCLUDING ALL IMPLIED WARRANTIES OF MERCHANTABILITY AND
FITNESS, IN NO EVENT SHALL STICHTING MATHEMATISCH CENTRUM BE LIABLE
FOR ANY SPECIAL, INDIRECT OR CONSEQUENTIAL DAMAGES OR ANY DAMAGES
WHATSOEVER RESULTING FROM LOSS OF USE, DATA OR PROFITS, WHETHER IN AN
ACTION OF CONTRACT, NEGLIGENCE OR OTHER TORTIOUS ACTION, ARISING OUT
OF OR IN CONNECTION WITH THE USE OR PERFORMANCE OF THIS SOFTWARE.


\begin{abstract}

\noindent
Python is an easy to learn, powerful programming language.  It has
efficient high-level data structures and a simple but effective
approach to object-oriented programming.  Python's elegant syntax and
dynamic typing, together with its interpreted nature, make it an ideal 
language for scripting and rapid application development in many areas 
on most platforms.

The Python interpreter and the extensive standard library are freely
available in source or binary form for all major platforms from the
Python Web site, \url{http://www.python.org/}, and can be freely
distributed.  The same site also contains distributions of and
pointers to many free third party Python modules, programs and tools,
and additional documentation.

The Python interpreter is easily extended with new functions and data
types implemented in C or \Cpp{} (or other languages callable from C).
Python is also suitable as an extension language for customizable
applications.

This tutorial introduces the reader informally to the basic concepts
and features of the Python language and system.  It helps to have a
Python interpreter handy for hands-on experience, but all examples are
self-contained, so the tutorial can be read off-line as well.

For a description of standard objects and modules, see the
\citetitle[../lib/lib.html]{Python Library Reference} document.  The
\citetitle[../ref/ref.html]{Python Reference Manual} gives a more
formal definition of the language.  To write extensions in C or
\Cpp, read \citetitle[../ext/ext.html]{Extending and Embedding the
Python Interpreter} and \citetitle[../api/api.html]{Python/C API
Reference}.  There are also several books covering Python in depth.

This tutorial does not attempt to be comprehensive and cover every
single feature, or even every commonly used feature.  Instead, it
introduces many of Python's most noteworthy features, and will give
you a good idea of the language's flavor and style.  After reading it,
you will be able to read and write Python modules and programs, and
you will be ready to learn more about the various Python library
modules described in the \citetitle[../lib/lib.html]{Python Library
Reference}.

\end{abstract}

\tableofcontents


\chapter{Whetting Your Appetite \label{intro}}

If you ever wrote a large shell script, you probably know this
feeling: you'd love to add yet another feature, but it's already so
slow, and so big, and so complicated; or the feature involves a system
call or other function that is only accessible from C \ldots Usually
the problem at hand isn't serious enough to warrant rewriting the
script in C; perhaps the problem requires variable-length strings or
other data types (like sorted lists of file names) that are easy in
the shell but lots of work to implement in C, or perhaps you're not
sufficiently familiar with C.

Another situation: perhaps you have to work with several C libraries,
and the usual C write/compile/test/re-compile cycle is too slow.  You
need to develop software more quickly.  Possibly perhaps you've
written a program that could use an extension language, and you don't
want to design a language, write and debug an interpreter for it, then
tie it into your application.

In such cases, Python may be just the language for you.  Python is
simple to use, but it is a real programming language, offering much
more structure and support for large programs than the shell has.  On
the other hand, it also offers much more error checking than C, and,
being a \emph{very-high-level language}, it has high-level data types
built in, such as flexible arrays and dictionaries that would cost you
days to implement efficiently in C.  Because of its more general data
types Python is applicable to a much larger problem domain than
\emph{Awk} or even \emph{Perl}, yet many things are at least as easy
in Python as in those languages.

Python allows you to split up your program in modules that can be
reused in other Python programs.  It comes with a large collection of
standard modules that you can use as the basis of your programs --- or
as examples to start learning to program in Python.  There are also
built-in modules that provide things like file I/O, system calls,
sockets, and even interfaces to graphical user interface toolkits like Tk.  

Python is an interpreted language, which can save you considerable time
during program development because no compilation and linking is
necessary.  The interpreter can be used interactively, which makes it
easy to experiment with features of the language, to write throw-away
programs, or to test functions during bottom-up program development.
It is also a handy desk calculator.

Python allows writing very compact and readable programs.  Programs
written in Python are typically much shorter than equivalent C or
\Cpp{} programs, for several reasons:
\begin{itemize}
\item
the high-level data types allow you to express complex operations in a
single statement;
\item
statement grouping is done by indentation instead of begin/end
brackets;
\item
no variable or argument declarations are necessary.
\end{itemize}

Python is \emph{extensible}: if you know how to program in C it is easy
to add a new built-in function or module to the interpreter, either to
perform critical operations at maximum speed, or to link Python
programs to libraries that may only be available in binary form (such
as a vendor-specific graphics library).  Once you are really hooked,
you can link the Python interpreter into an application written in C
and use it as an extension or command language for that application.

By the way, the language is named after the BBC show ``Monty Python's
Flying Circus'' and has nothing to do with nasty reptiles.  Making
references to Monty Python skits in documentation is not only allowed,
it is encouraged!

%\section{Where From Here \label{where}}

Now that you are all excited about Python, you'll want to examine it
in some more detail.  Since the best way to learn a language is
using it, you are invited here to do so.

In the next chapter, the mechanics of using the interpreter are
explained.  This is rather mundane information, but essential for
trying out the examples shown later.

The rest of the tutorial introduces various features of the Python
language and system through examples, beginning with simple
expressions, statements and data types, through functions and modules,
and finally touching upon advanced concepts like exceptions
and user-defined classes.

\chapter{Using the Python Interpreter \label{using}}

\section{Invoking the Interpreter \label{invoking}}

The Python interpreter is usually installed as
\file{/usr/local/bin/python} on those machines where it is available;
putting \file{/usr/local/bin} in your \UNIX{} shell's search path
makes it possible to start it by typing the command

\begin{verbatim}
python
\end{verbatim}

to the shell.  Since the choice of the directory where the interpreter
lives is an installation option, other places are possible; check with
your local Python guru or system administrator.  (E.g.,
\file{/usr/local/python} is a popular alternative location.)

Typing an end-of-file character (\kbd{Control-D} on \UNIX,
\kbd{Control-Z} on Windows) at the primary prompt causes the
interpreter to exit with a zero exit status.  If that doesn't work,
you can exit the interpreter by typing the following commands:
\samp{import sys; sys.exit()}.

The interpreter's line-editing features usually aren't very
sophisticated.  On \UNIX, whoever installed the interpreter may have
enabled support for the GNU readline library, which adds more
elaborate interactive editing and history features. Perhaps the
quickest check to see whether command line editing is supported is
typing Control-P to the first Python prompt you get.  If it beeps, you
have command line editing; see Appendix \ref{interacting} for an
introduction to the keys.  If nothing appears to happen, or if
\code{\^P} is echoed, command line editing isn't available; you'll
only be able to use backspace to remove characters from the current
line.

The interpreter operates somewhat like the \UNIX{} shell: when called
with standard input connected to a tty device, it reads and executes
commands interactively; when called with a file name argument or with
a file as standard input, it reads and executes a \emph{script} from
that file. 

A third way of starting the interpreter is
\samp{\program{python} \programopt{-c} \var{command} [arg] ...}, which
executes the statement(s) in \var{command}, analogous to the shell's
\programopt{-c} option.  Since Python statements often contain spaces
or other characters that are special to the shell, it is best to quote 
\var{command} in its entirety with double quotes.

Note that there is a difference between \samp{python file} and
\samp{python <file}.  In the latter case, input requests from the
program, such as calls to \function{input()} and \function{raw_input()}, are
satisfied from \emph{file}.  Since this file has already been read
until the end by the parser before the program starts executing, the
program will encounter end-of-file immediately.  In the former case
(which is usually what you want) they are satisfied from whatever file
or device is connected to standard input of the Python interpreter.

When a script file is used, it is sometimes useful to be able to run
the script and enter interactive mode afterwards.  This can be done by
passing \programopt{-i} before the script.  (This does not work if the
script is read from standard input, for the same reason as explained
in the previous paragraph.)

\subsection{Argument Passing \label{argPassing}}

When known to the interpreter, the script name and additional
arguments thereafter are passed to the script in the variable
\code{sys.argv}, which is a list of strings.  Its length is at least
one; when no script and no arguments are given, \code{sys.argv[0]} is
an empty string.  When the script name is given as \code{'-'} (meaning 
standard input), \code{sys.argv[0]} is set to \code{'-'}.  When
\programopt{-c} \var{command} is used, \code{sys.argv[0]} is set to
\code{'-c'}.  Options found after \programopt{-c} \var{command} are
not consumed by the Python interpreter's option processing but left in
\code{sys.argv} for the command to handle.

\subsection{Interactive Mode \label{interactive}}

When commands are read from a tty, the interpreter is said to be in
\emph{interactive mode}.  In this mode it prompts for the next command
with the \emph{primary prompt}, usually three greater-than signs
(\samp{>\code{>}>~}); for continuation lines it prompts with the
\emph{secondary prompt}, by default three dots (\samp{...~}).
The interpreter prints a welcome message stating its version number
and a copyright notice before printing the first prompt:

\begin{verbatim}
python
Python 1.5.2b2 (#1, Feb 28 1999, 00:02:06)  [GCC 2.8.1] on sunos5
Copyright 1991-1995 Stichting Mathematisch Centrum, Amsterdam
>>>
\end{verbatim}

Continuation lines are needed when entering a multi-line construct.
As an example, take a look at this \keyword{if} statement:

\begin{verbatim}
>>> the_world_is_flat = 1
>>> if the_world_is_flat:
...     print "Be careful not to fall off!"
... 
Be careful not to fall off!
\end{verbatim}


\section{The Interpreter and Its Environment \label{interp}}

\subsection{Error Handling \label{error}}

When an error occurs, the interpreter prints an error
message and a stack trace.  In interactive mode, it then returns to
the primary prompt; when input came from a file, it exits with a
nonzero exit status after printing
the stack trace.  (Exceptions handled by an \keyword{except} clause in a
\keyword{try} statement are not errors in this context.)  Some errors are
unconditionally fatal and cause an exit with a nonzero exit; this
applies to internal inconsistencies and some cases of running out of
memory.  All error messages are written to the standard error stream;
normal output from the executed commands is written to standard
output.

Typing the interrupt character (usually Control-C or DEL) to the
primary or secondary prompt cancels the input and returns to the
primary prompt.\footnote{
        A problem with the GNU Readline package may prevent this.
}
Typing an interrupt while a command is executing raises the
\exception{KeyboardInterrupt} exception, which may be handled by a
\keyword{try} statement.

\subsection{Executable Python Scripts \label{scripts}}

On BSD'ish \UNIX{} systems, Python scripts can be made directly
executable, like shell scripts, by putting the line

\begin{verbatim}
#! /usr/bin/env python
\end{verbatim}

(assuming that the interpreter is on the user's \envvar{PATH}) at the
beginning of the script and giving the file an executable mode.  The
\samp{\#!} must be the first two characters of the file.  On some
platforms, this first line must end with a \UNIX-style line ending
(\character{\e n}), not a Mac OS (\character{\e r}) or Windows
(\character{\e r\e n}) line ending.  Note that
the hash, or pound, character, \character{\#}, is used to start a
comment in Python.

The script can be given a executable mode, or permission, using the
\program{chmod} command:

\begin{verbatim}
$ chmod +x myscript.py
\end{verbatim} % $ <-- bow to font-lock


\subsection{Source Code Encoding}

It is possible to use encodings different than \ASCII{} in Python source
files. The best way to do it is to put one more special comment line
right after the \code{\#!} line to define the source file encoding:

\begin{verbatim}
# -*- coding: iso-8859-1 -*- 
\end{verbatim}

With that declaration, all characters in the source file will be treated as
{}\code{iso-8859-1}, and it will be
possible to directly write Unicode string literals in the selected
encoding.  The list of possible encodings can be found in the
\citetitle[../lib/lib.html]{Python Library Reference}, in the section
on \module{codecs}.

If your editor supports saving files as \code{UTF-8} with an UTF-8
signature (aka BOM -- Byte Order Mark), you can use that instead of an
encoding declaration. IDLE supports this capability if
\code{Options/General/Default Source Encoding/UTF-8} is set. Notice
that this signature is not understood in older Python releases (2.2
and earlier), and also not understood by the operating system for
\code{\#!} files. 

By using UTF-8 (either through the signature or an encoding
declaration), characters of most languages in the world can be used
simultaneously in string literals and comments. Using non-ASCII
characters in identifiers is not supported. To display all these
characters properly, your editor must recognize that the file is
UTF-8, and it must use a font that supports all the characters in the
file.

\subsection{The Interactive Startup File \label{startup}}

% XXX This should probably be dumped in an appendix, since most people
% don't use Python interactively in non-trivial ways.

When you use Python interactively, it is frequently handy to have some
standard commands executed every time the interpreter is started.  You
can do this by setting an environment variable named
\envvar{PYTHONSTARTUP} to the name of a file containing your start-up
commands.  This is similar to the \file{.profile} feature of the
\UNIX{} shells.

This file is only read in interactive sessions, not when Python reads
commands from a script, and not when \file{/dev/tty} is given as the
explicit source of commands (which otherwise behaves like an
interactive session).  It is executed in the same namespace where
interactive commands are executed, so that objects that it defines or
imports can be used without qualification in the interactive session.
You can also change the prompts \code{sys.ps1} and \code{sys.ps2} in
this file.

If you want to read an additional start-up file from the current
directory, you can program this in the global start-up file using code
like \samp{if os.path.isfile('.pythonrc.py'):
execfile('.pythonrc.py')}.  If you want to use the startup file in a
script, you must do this explicitly in the script:

\begin{verbatim}
import os
filename = os.environ.get('PYTHONSTARTUP')
if filename and os.path.isfile(filename):
    execfile(filename)
\end{verbatim}


\chapter{An Informal Introduction to Python \label{informal}}

In the following examples, input and output are distinguished by the
presence or absence of prompts (\samp{>\code{>}>~} and \samp{...~}): to repeat
the example, you must type everything after the prompt, when the
prompt appears; lines that do not begin with a prompt are output from
the interpreter. %
%\footnote{
%        I'd prefer to use different fonts to distinguish input
%        from output, but the amount of LaTeX hacking that would require
%        is currently beyond my ability.
%}
Note that a secondary prompt on a line by itself in an example means
you must type a blank line; this is used to end a multi-line command.

Many of the examples in this manual, even those entered at the
interactive prompt, include comments.  Comments in Python start with
the hash character, \character{\#}, and extend to the end of the
physical line.  A comment may appear at the start of a line or
following whitespace or code, but not within a string literal.  A hash 
character within a string literal is just a hash character.

Some examples:

\begin{verbatim}
# this is the first comment
SPAM = 1                 # and this is the second comment
                         # ... and now a third!
STRING = "# This is not a comment."
\end{verbatim}


\section{Using Python as a Calculator \label{calculator}}

Let's try some simple Python commands.  Start the interpreter and wait
for the primary prompt, \samp{>\code{>}>~}.  (It shouldn't take long.)

\subsection{Numbers \label{numbers}}

The interpreter acts as a simple calculator: you can type an
expression at it and it will write the value.  Expression syntax is
straightforward: the operators \code{+}, \code{-}, \code{*} and
\code{/} work just like in most other languages (for example, Pascal
or C); parentheses can be used for grouping.  For example:

\begin{verbatim}
>>> 2+2
4
>>> # This is a comment
... 2+2
4
>>> 2+2  # and a comment on the same line as code
4
>>> (50-5*6)/4
5
>>> # Integer division returns the floor:
... 7/3
2
>>> 7/-3
-3
\end{verbatim}

Like in C, the equal sign (\character{=}) is used to assign a value to a
variable.  The value of an assignment is not written:

\begin{verbatim}
>>> width = 20
>>> height = 5*9
>>> width * height
900
\end{verbatim}

A value can be assigned to several variables simultaneously:

\begin{verbatim}
>>> x = y = z = 0  # Zero x, y and z
>>> x
0
>>> y
0
>>> z
0
\end{verbatim}

There is full support for floating point; operators with mixed type
operands convert the integer operand to floating point:

\begin{verbatim}
>>> 3 * 3.75 / 1.5
7.5
>>> 7.0 / 2
3.5
\end{verbatim}

Complex numbers are also supported; imaginary numbers are written with
a suffix of \samp{j} or \samp{J}.  Complex numbers with a nonzero
real component are written as \samp{(\var{real}+\var{imag}j)}, or can
be created with the \samp{complex(\var{real}, \var{imag})} function.

\begin{verbatim}
>>> 1j * 1J
(-1+0j)
>>> 1j * complex(0,1)
(-1+0j)
>>> 3+1j*3
(3+3j)
>>> (3+1j)*3
(9+3j)
>>> (1+2j)/(1+1j)
(1.5+0.5j)
\end{verbatim}

Complex numbers are always represented as two floating point numbers,
the real and imaginary part.  To extract these parts from a complex
number \var{z}, use \code{\var{z}.real} and \code{\var{z}.imag}.  

\begin{verbatim}
>>> a=1.5+0.5j
>>> a.real
1.5
>>> a.imag
0.5
\end{verbatim}

The conversion functions to floating point and integer
(\function{float()}, \function{int()} and \function{long()}) don't
work for complex numbers --- there is no one correct way to convert a
complex number to a real number.  Use \code{abs(\var{z})} to get its
magnitude (as a float) or \code{z.real} to get its real part.

\begin{verbatim}
>>> a=3.0+4.0j
>>> float(a)
Traceback (most recent call last):
  File "<stdin>", line 1, in ?
TypeError: can't convert complex to float; use e.g. abs(z)
>>> a.real
3.0
>>> a.imag
4.0
>>> abs(a)  # sqrt(a.real**2 + a.imag**2)
5.0
>>>
\end{verbatim}

In interactive mode, the last printed expression is assigned to the
variable \code{_}.  This means that when you are using Python as a
desk calculator, it is somewhat easier to continue calculations, for
example:

\begin{verbatim}
>>> tax = 12.5 / 100
>>> price = 100.50
>>> price * tax
12.5625
>>> price + _
113.0625
>>> round(_, 2)
113.06
>>>
\end{verbatim}

This variable should be treated as read-only by the user.  Don't
explicitly assign a value to it --- you would create an independent
local variable with the same name masking the built-in variable with
its magic behavior.

\subsection{Strings \label{strings}}

Besides numbers, Python can also manipulate strings, which can be
expressed in several ways.  They can be enclosed in single quotes or
double quotes:

\begin{verbatim}
>>> 'spam eggs'
'spam eggs'
>>> 'doesn\'t'
"doesn't"
>>> "doesn't"
"doesn't"
>>> '"Yes," he said.'
'"Yes," he said.'
>>> "\"Yes,\" he said."
'"Yes," he said.'
>>> '"Isn\'t," she said.'
'"Isn\'t," she said.'
\end{verbatim}

String literals can span multiple lines in several ways.  Continuation
lines can be used, with a backslash as the last character on the line
indicating that the next line is a logical continuation of the line:

\begin{verbatim}
hello = "This is a rather long string containing\n\
several lines of text just as you would do in C.\n\
    Note that whitespace at the beginning of the line is\
 significant."

print hello
\end{verbatim}

Note that newlines would still need to be embedded in the string using
\code{\e n}; the newline following the trailing backslash is
discarded.  This example would print the following:

\begin{verbatim}
This is a rather long string containing
several lines of text just as you would do in C.
    Note that whitespace at the beginning of the line is significant.
\end{verbatim}

If we make the string literal a ``raw'' string, however, the
\code{\e n} sequences are not converted to newlines, but the backslash
at the end of the line, and the newline character in the source, are
both included in the string as data.  Thus, the example:

\begin{verbatim}
hello = r"This is a rather long string containing\n\
several lines of text much as you would do in C."

print hello
\end{verbatim}

would print:

\begin{verbatim}
This is a rather long string containing\n\
several lines of text much as you would do in C.
\end{verbatim}

Or, strings can be surrounded in a pair of matching triple-quotes:
\code{"""} or \code{'\code{'}'}.  End of lines do not need to be escaped
when using triple-quotes, but they will be included in the string.

\begin{verbatim}
print """
Usage: thingy [OPTIONS] 
     -h                        Display this usage message
     -H hostname               Hostname to connect to
"""
\end{verbatim}

produces the following output:

\begin{verbatim}
Usage: thingy [OPTIONS] 
     -h                        Display this usage message
     -H hostname               Hostname to connect to
\end{verbatim}

The interpreter prints the result of string operations in the same way
as they are typed for input: inside quotes, and with quotes and other
funny characters escaped by backslashes, to show the precise
value.  The string is enclosed in double quotes if the string contains
a single quote and no double quotes, else it's enclosed in single
quotes.  (The \keyword{print} statement, described later, can be used
to write strings without quotes or escapes.)

Strings can be concatenated (glued together) with the
\code{+} operator, and repeated with \code{*}:

\begin{verbatim}
>>> word = 'Help' + 'A'
>>> word
'HelpA'
>>> '<' + word*5 + '>'
'<HelpAHelpAHelpAHelpAHelpA>'
\end{verbatim}

Two string literals next to each other are automatically concatenated;
the first line above could also have been written \samp{word = 'Help'
'A'}; this only works with two literals, not with arbitrary string
expressions:

\begin{verbatim}
>>> import string
>>> 'str' 'ing'                   #  <-  This is ok
'string'
>>> string.strip('str') + 'ing'   #  <-  This is ok
'string'
>>> string.strip('str') 'ing'     #  <-  This is invalid
  File "<stdin>", line 1, in ?
    string.strip('str') 'ing'
                            ^
SyntaxError: invalid syntax
\end{verbatim}

Strings can be subscripted (indexed); like in C, the first character
of a string has subscript (index) 0.  There is no separate character
type; a character is simply a string of size one.  Like in Icon,
substrings can be specified with the \emph{slice notation}: two indices
separated by a colon.

\begin{verbatim}
>>> word[4]
'A'
>>> word[0:2]
'He'
>>> word[2:4]
'lp'
\end{verbatim}

Slice indices have useful defaults; an omitted first index defaults to
zero, an omitted second index defaults to the size of the string being
sliced.

\begin{verbatim}
>>> word[:2]    # The first two characters
'He'
>>> word[2:]    # All but the first two characters
'lpA'
\end{verbatim}

Unlike a C string, Python strings cannot be changed.  Assigning to an 
indexed position in the string results in an error:

\begin{verbatim}
>>> word[0] = 'x'
Traceback (most recent call last):
  File "<stdin>", line 1, in ?
TypeError: object doesn't support item assignment
>>> word[:1] = 'Splat'
Traceback (most recent call last):
  File "<stdin>", line 1, in ?
TypeError: object doesn't support slice assignment
\end{verbatim}

However, creating a new string with the combined content is easy and
efficient:

\begin{verbatim}
>>> 'x' + word[1:]
'xelpA'
>>> 'Splat' + word[4]
'SplatA'
\end{verbatim}

Here's a useful invariant of slice operations:
\code{s[:i] + s[i:]} equals \code{s}.

\begin{verbatim}
>>> word[:2] + word[2:]
'HelpA'
>>> word[:3] + word[3:]
'HelpA'
\end{verbatim}

Degenerate slice indices are handled gracefully: an index that is too
large is replaced by the string size, an upper bound smaller than the
lower bound returns an empty string.

\begin{verbatim}
>>> word[1:100]
'elpA'
>>> word[10:]
''
>>> word[2:1]
''
\end{verbatim}

Indices may be negative numbers, to start counting from the right.
For example:

\begin{verbatim}
>>> word[-1]     # The last character
'A'
>>> word[-2]     # The last-but-one character
'p'
>>> word[-2:]    # The last two characters
'pA'
>>> word[:-2]    # All but the last two characters
'Hel'
\end{verbatim}

But note that -0 is really the same as 0, so it does not count from
the right!

\begin{verbatim}
>>> word[-0]     # (since -0 equals 0)
'H'
\end{verbatim}

Out-of-range negative slice indices are truncated, but don't try this
for single-element (non-slice) indices:

\begin{verbatim}
>>> word[-100:]
'HelpA'
>>> word[-10]    # error
Traceback (most recent call last):
  File "<stdin>", line 1, in ?
IndexError: string index out of range
\end{verbatim}

The best way to remember how slices work is to think of the indices as
pointing \emph{between} characters, with the left edge of the first
character numbered 0.  Then the right edge of the last character of a
string of \var{n} characters has index \var{n}, for example:

\begin{verbatim}
 +---+---+---+---+---+ 
 | H | e | l | p | A |
 +---+---+---+---+---+ 
 0   1   2   3   4   5 
-5  -4  -3  -2  -1
\end{verbatim}

The first row of numbers gives the position of the indices 0...5 in
the string; the second row gives the corresponding negative indices.
The slice from \var{i} to \var{j} consists of all characters between
the edges labeled \var{i} and \var{j}, respectively.

For non-negative indices, the length of a slice is the difference of
the indices, if both are within bounds.  For example, the length of
\code{word[1:3]} is 2.

The built-in function \function{len()} returns the length of a string:

\begin{verbatim}
>>> s = 'supercalifragilisticexpialidocious'
>>> len(s)
34
\end{verbatim}


\subsection{Unicode Strings \label{unicodeStrings}}
\sectionauthor{Marc-Andre Lemburg}{mal@lemburg.com}

Starting with Python 2.0 a new data type for storing text data is
available to the programmer: the Unicode object. It can be used to
store and manipulate Unicode data (see \url{http://www.unicode.org/})
and integrates well with the existing string objects providing
auto-conversions where necessary.

Unicode has the advantage of providing one ordinal for every character
in every script used in modern and ancient texts. Previously, there
were only 256 possible ordinals for script characters and texts were
typically bound to a code page which mapped the ordinals to script
characters. This lead to very much confusion especially with respect
to internationalization (usually written as \samp{i18n} ---
\character{i} + 18 characters + \character{n}) of software.  Unicode
solves these problems by defining one code page for all scripts.

Creating Unicode strings in Python is just as simple as creating
normal strings:

\begin{verbatim}
>>> u'Hello World !'
u'Hello World !'
\end{verbatim}

The small \character{u} in front of the quote indicates that an
Unicode string is supposed to be created. If you want to include
special characters in the string, you can do so by using the Python
\emph{Unicode-Escape} encoding. The following example shows how:

\begin{verbatim}
>>> u'Hello\u0020World !'
u'Hello World !'
\end{verbatim}

The escape sequence \code{\e u0020} indicates to insert the Unicode
character with the ordinal value 0x0020 (the space character) at the
given position.

Other characters are interpreted by using their respective ordinal
values directly as Unicode ordinals.  If you have literal strings
in the standard Latin-1 encoding that is used in many Western countries,
you will find it convenient that the lower 256 characters
of Unicode are the same as the 256 characters of Latin-1.

For experts, there is also a raw mode just like the one for normal
strings. You have to prefix the opening quote with 'ur' to have
Python use the \emph{Raw-Unicode-Escape} encoding. It will only apply
the above \code{\e uXXXX} conversion if there is an uneven number of
backslashes in front of the small 'u'.

\begin{verbatim}
>>> ur'Hello\u0020World !'
u'Hello World !'
>>> ur'Hello\\u0020World !'
u'Hello\\\\u0020World !'
\end{verbatim}

The raw mode is most useful when you have to enter lots of
backslashes, as can be necessary in regular expressions.

Apart from these standard encodings, Python provides a whole set of
other ways of creating Unicode strings on the basis of a known
encoding. 

The built-in function \function{unicode()}\bifuncindex{unicode} provides
access to all registered Unicode codecs (COders and DECoders). Some of
the more well known encodings which these codecs can convert are
\emph{Latin-1}, \emph{ASCII}, \emph{UTF-8}, and \emph{UTF-16}.
The latter two are variable-length encodings that store each Unicode
character in one or more bytes. The default encoding is
normally set to ASCII, which passes through characters in the range
0 to 127 and rejects any other characters with an error.
When a Unicode string is printed, written to a file, or converted
with \function{str()}, conversion takes place using this default encoding.

\begin{verbatim}
>>> u"abc"
u'abc'
>>> str(u"abc")
'abc'
>>> u"���"
u'\xe4\xf6\xfc'
>>> str(u"���")
Traceback (most recent call last):
  File "<stdin>", line 1, in ?
UnicodeEncodeError: 'ascii' codec can't encode characters in position 0-2: ordinal not in range(128)
\end{verbatim}

To convert a Unicode string into an 8-bit string using a specific
encoding, Unicode objects provide an \function{encode()} method
that takes one argument, the name of the encoding.  Lowercase names
for encodings are preferred.

\begin{verbatim}
>>> u"���".encode('utf-8')
'\xc3\xa4\xc3\xb6\xc3\xbc'
\end{verbatim}

If you have data in a specific encoding and want to produce a
corresponding Unicode string from it, you can use the
\function{unicode()} function with the encoding name as the second
argument.

\begin{verbatim}
>>> unicode('\xc3\xa4\xc3\xb6\xc3\xbc', 'utf-8')
u'\xe4\xf6\xfc'
\end{verbatim}

\subsection{Lists \label{lists}}

Python knows a number of \emph{compound} data types, used to group
together other values.  The most versatile is the \emph{list}, which
can be written as a list of comma-separated values (items) between
square brackets.  List items need not all have the same type.

\begin{verbatim}
>>> a = ['spam', 'eggs', 100, 1234]
>>> a
['spam', 'eggs', 100, 1234]
\end{verbatim}

Like string indices, list indices start at 0, and lists can be sliced,
concatenated and so on:

\begin{verbatim}
>>> a[0]
'spam'
>>> a[3]
1234
>>> a[-2]
100
>>> a[1:-1]
['eggs', 100]
>>> a[:2] + ['bacon', 2*2]
['spam', 'eggs', 'bacon', 4]
>>> 3*a[:3] + ['Boe!']
['spam', 'eggs', 100, 'spam', 'eggs', 100, 'spam', 'eggs', 100, 'Boe!']
\end{verbatim}

Unlike strings, which are \emph{immutable}, it is possible to change
individual elements of a list:

\begin{verbatim}
>>> a
['spam', 'eggs', 100, 1234]
>>> a[2] = a[2] + 23
>>> a
['spam', 'eggs', 123, 1234]
\end{verbatim}

Assignment to slices is also possible, and this can even change the size
of the list:

\begin{verbatim}
>>> # Replace some items:
... a[0:2] = [1, 12]
>>> a
[1, 12, 123, 1234]
>>> # Remove some:
... a[0:2] = []
>>> a
[123, 1234]
>>> # Insert some:
... a[1:1] = ['bletch', 'xyzzy']
>>> a
[123, 'bletch', 'xyzzy', 1234]
>>> a[:0] = a     # Insert (a copy of) itself at the beginning
>>> a
[123, 'bletch', 'xyzzy', 1234, 123, 'bletch', 'xyzzy', 1234]
\end{verbatim}

The built-in function \function{len()} also applies to lists:

\begin{verbatim}
>>> len(a)
8
\end{verbatim}

It is possible to nest lists (create lists containing other lists),
for example:

\begin{verbatim}
>>> q = [2, 3]
>>> p = [1, q, 4]
>>> len(p)
3
>>> p[1]
[2, 3]
>>> p[1][0]
2
>>> p[1].append('xtra')     # See section 5.1
>>> p
[1, [2, 3, 'xtra'], 4]
>>> q
[2, 3, 'xtra']
\end{verbatim}

Note that in the last example, \code{p[1]} and \code{q} really refer to
the same object!  We'll come back to \emph{object semantics} later.

\section{First Steps Towards Programming \label{firstSteps}}

Of course, we can use Python for more complicated tasks than adding
two and two together.  For instance, we can write an initial
sub-sequence of the \emph{Fibonacci} series as follows:

\begin{verbatim}
>>> # Fibonacci series:
... # the sum of two elements defines the next
... a, b = 0, 1
>>> while b < 10:
...       print b
...       a, b = b, a+b
... 
1
1
2
3
5
8
\end{verbatim}

This example introduces several new features.

\begin{itemize}

\item
The first line contains a \emph{multiple assignment}: the variables
\code{a} and \code{b} simultaneously get the new values 0 and 1.  On the
last line this is used again, demonstrating that the expressions on
the right-hand side are all evaluated first before any of the
assignments take place.  The right-hand side expressions are evaluated 
from the left to the right.

\item
The \keyword{while} loop executes as long as the condition (here:
\code{b < 10}) remains true.  In Python, like in C, any non-zero
integer value is true; zero is false.  The condition may also be a
string or list value, in fact any sequence; anything with a non-zero
length is true, empty sequences are false.  The test used in the
example is a simple comparison.  The standard comparison operators are
written the same as in C: \code{<} (less than), \code{>} (greater than),
\code{==} (equal to), \code{<=} (less than or equal to),
\code{>=} (greater than or equal to) and \code{!=} (not equal to).

\item
The \emph{body} of the loop is \emph{indented}: indentation is Python's
way of grouping statements.  Python does not (yet!) provide an
intelligent input line editing facility, so you have to type a tab or
space(s) for each indented line.  In practice you will prepare more
complicated input for Python with a text editor; most text editors have
an auto-indent facility.  When a compound statement is entered
interactively, it must be followed by a blank line to indicate
completion (since the parser cannot guess when you have typed the last
line).  Note that each line within a basic block must be indented by
the same amount.

\item
The \keyword{print} statement writes the value of the expression(s) it is
given.  It differs from just writing the expression you want to write
(as we did earlier in the calculator examples) in the way it handles
multiple expressions and strings.  Strings are printed without quotes,
and a space is inserted between items, so you can format things nicely,
like this:

\begin{verbatim}
>>> i = 256*256
>>> print 'The value of i is', i
The value of i is 65536
\end{verbatim}

A trailing comma avoids the newline after the output:

\begin{verbatim}
>>> a, b = 0, 1
>>> while b < 1000:
...     print b,
...     a, b = b, a+b
... 
1 1 2 3 5 8 13 21 34 55 89 144 233 377 610 987
\end{verbatim}

Note that the interpreter inserts a newline before it prints the next
prompt if the last line was not completed.

\end{itemize}


\chapter{More Control Flow Tools \label{moreControl}}

Besides the \keyword{while} statement just introduced, Python knows
the usual control flow statements known from other languages, with
some twists.

\section{\keyword{if} Statements \label{if}}

Perhaps the most well-known statement type is the
\keyword{if} statement.  For example:

\begin{verbatim}
>>> x = int(raw_input("Please enter an integer: "))
>>> if x < 0:
...      x = 0
...      print 'Negative changed to zero'
... elif x == 0:
...      print 'Zero'
... elif x == 1:
...      print 'Single'
... else:
...      print 'More'
... 
\end{verbatim}

There can be zero or more \keyword{elif} parts, and the
\keyword{else} part is optional.  The keyword `\keyword{elif}' is
short for `else if', and is useful to avoid excessive indentation.  An 
\keyword{if} \ldots\ \keyword{elif} \ldots\ \keyword{elif} \ldots\ sequence
%    Weird spacings happen here if the wrapping of the source text
%    gets changed in the wrong way.
is a substitute for the \keyword{switch} or
\keyword{case} statements found in other languages.


\section{\keyword{for} Statements \label{for}}

The \keyword{for}\stindex{for} statement in Python differs a bit from
what you may be used to in C or Pascal.  Rather than always
iterating over an arithmetic progression of numbers (like in Pascal),
or giving the user the ability to define both the iteration step and
halting condition (as C), Python's
\keyword{for}\stindex{for} statement iterates over the items of any
sequence (a list or a string), in the order that they appear in
the sequence.  For example (no pun intended):
% One suggestion was to give a real C example here, but that may only
% serve to confuse non-C programmers.

\begin{verbatim}
>>> # Measure some strings:
... a = ['cat', 'window', 'defenestrate']
>>> for x in a:
...     print x, len(x)
... 
cat 3
window 6
defenestrate 12
\end{verbatim}

It is not safe to modify the sequence being iterated over in the loop
(this can only happen for mutable sequence types, such as lists).  If
you need to modify the list you are iterating over (for example, to
duplicate selected items) you must iterate over a copy.  The slice
notation makes this particularly convenient:

\begin{verbatim}
>>> for x in a[:]: # make a slice copy of the entire list
...    if len(x) > 6: a.insert(0, x)
... 
>>> a
['defenestrate', 'cat', 'window', 'defenestrate']
\end{verbatim}


\section{The \function{range()} Function \label{range}}

If you do need to iterate over a sequence of numbers, the built-in
function \function{range()} comes in handy.  It generates lists
containing arithmetic progressions:

\begin{verbatim}
>>> range(10)
[0, 1, 2, 3, 4, 5, 6, 7, 8, 9]
\end{verbatim}

The given end point is never part of the generated list;
\code{range(10)} generates a list of 10 values, exactly the legal
indices for items of a sequence of length 10.  It is possible to let
the range start at another number, or to specify a different increment
(even negative; sometimes this is called the `step'):

\begin{verbatim}
>>> range(5, 10)
[5, 6, 7, 8, 9]
>>> range(0, 10, 3)
[0, 3, 6, 9]
>>> range(-10, -100, -30)
[-10, -40, -70]
\end{verbatim}

To iterate over the indices of a sequence, combine
\function{range()} and \function{len()} as follows:

\begin{verbatim}
>>> a = ['Mary', 'had', 'a', 'little', 'lamb']
>>> for i in range(len(a)):
...     print i, a[i]
... 
0 Mary
1 had
2 a
3 little
4 lamb
\end{verbatim}


\section{\keyword{break} and \keyword{continue} Statements, and
         \keyword{else} Clauses on Loops
         \label{break}}

The \keyword{break} statement, like in C, breaks out of the smallest
enclosing \keyword{for} or \keyword{while} loop.

The \keyword{continue} statement, also borrowed from C, continues
with the next iteration of the loop.

Loop statements may have an \code{else} clause; it is executed when
the loop terminates through exhaustion of the list (with
\keyword{for}) or when the condition becomes false (with
\keyword{while}), but not when the loop is terminated by a
\keyword{break} statement.  This is exemplified by the following loop,
which searches for prime numbers:

\begin{verbatim}
>>> for n in range(2, 10):
...     for x in range(2, n):
...         if n % x == 0:
...            print n, 'equals', x, '*', n/x
...            break
...     else:
...          # loop fell through without finding a factor
...          print n, 'is a prime number'
... 
2 is a prime number
3 is a prime number
4 equals 2 * 2
5 is a prime number
6 equals 2 * 3
7 is a prime number
8 equals 2 * 4
9 equals 3 * 3
\end{verbatim}


\section{\keyword{pass} Statements \label{pass}}

The \keyword{pass} statement does nothing.
It can be used when a statement is required syntactically but the
program requires no action.
For example:

\begin{verbatim}
>>> while True:
...       pass # Busy-wait for keyboard interrupt
... 
\end{verbatim}


\section{Defining Functions \label{functions}}

We can create a function that writes the Fibonacci series to an
arbitrary boundary:

\begin{verbatim}
>>> def fib(n):    # write Fibonacci series up to n
...     """Print a Fibonacci series up to n."""
...     a, b = 0, 1
...     while b < n:
...         print b,
...         a, b = b, a+b
... 
>>> # Now call the function we just defined:
... fib(2000)
1 1 2 3 5 8 13 21 34 55 89 144 233 377 610 987 1597
\end{verbatim}

The keyword \keyword{def} introduces a function \emph{definition}.  It
must be followed by the function name and the parenthesized list of
formal parameters.  The statements that form the body of the function
start at the next line, and must be indented.  The first statement of
the function body can optionally be a string literal; this string
literal is the function's \index{documentation strings}documentation
string, or \dfn{docstring}.\index{docstrings}\index{strings, documentation}

There are tools which use docstrings to automatically produce online
or printed documentation, or to let the user interactively browse
through code; it's good practice to include docstrings in code that
you write, so try to make a habit of it.

The \emph{execution} of a function introduces a new symbol table used
for the local variables of the function.  More precisely, all variable
assignments in a function store the value in the local symbol table;
whereas variable references first look in the local symbol table, then
in the global symbol table, and then in the table of built-in names.
Thus,  global variables cannot be directly assigned a value within a
function (unless named in a \keyword{global} statement), although
they may be referenced.

The actual parameters (arguments) to a function call are introduced in
the local symbol table of the called function when it is called; thus,
arguments are passed using \emph{call by value} (where the
\emph{value} is always an object \emph{reference}, not the value of
the object).\footnote{
         Actually, \emph{call by object reference} would be a better
         description, since if a mutable object is passed, the caller
         will see any changes the callee makes to it (items
         inserted into a list).
} When a function calls another function, a new local symbol table is
created for that call.

A function definition introduces the function name in the current
symbol table.  The value of the function name
has a type that is recognized by the interpreter as a user-defined
function.  This value can be assigned to another name which can then
also be used as a function.  This serves as a general renaming
mechanism:

\begin{verbatim}
>>> fib
<function object at 10042ed0>
>>> f = fib
>>> f(100)
1 1 2 3 5 8 13 21 34 55 89
\end{verbatim}

You might object that \code{fib} is not a function but a procedure.  In
Python, like in C, procedures are just functions that don't return a
value.  In fact, technically speaking, procedures do return a value,
albeit a rather boring one.  This value is called \code{None} (it's a
built-in name).  Writing the value \code{None} is normally suppressed by
the interpreter if it would be the only value written.  You can see it
if you really want to:

\begin{verbatim}
>>> print fib(0)
None
\end{verbatim}

It is simple to write a function that returns a list of the numbers of
the Fibonacci series, instead of printing it:

\begin{verbatim}
>>> def fib2(n): # return Fibonacci series up to n
...     """Return a list containing the Fibonacci series up to n."""
...     result = []
...     a, b = 0, 1
...     while b < n:
...         result.append(b)    # see below
...         a, b = b, a+b
...     return result
... 
>>> f100 = fib2(100)    # call it
>>> f100                # write the result
[1, 1, 2, 3, 5, 8, 13, 21, 34, 55, 89]
\end{verbatim}

This example, as usual, demonstrates some new Python features:

\begin{itemize}

\item
The \keyword{return} statement returns with a value from a function.
\keyword{return} without an expression argument returns \code{None}.
Falling off the end of a procedure also returns \code{None}.

\item
The statement \code{result.append(b)} calls a \emph{method} of the list
object \code{result}.  A method is a function that `belongs' to an
object and is named \code{obj.methodname}, where \code{obj} is some
object (this may be an expression), and \code{methodname} is the name
of a method that is defined by the object's type.  Different types
define different methods.  Methods of different types may have the
same name without causing ambiguity.  (It is possible to define your
own object types and methods, using \emph{classes}, as discussed later
in this tutorial.)
The method \method{append()} shown in the example, is defined for
list objects; it adds a new element at the end of the list.  In this
example it is equivalent to \samp{result = result + [b]}, but more
efficient.

\end{itemize}

\section{More on Defining Functions \label{defining}}

It is also possible to define functions with a variable number of
arguments.  There are three forms, which can be combined.

\subsection{Default Argument Values \label{defaultArgs}}

The most useful form is to specify a default value for one or more
arguments.  This creates a function that can be called with fewer
arguments than it is defined

\begin{verbatim}
def ask_ok(prompt, retries=4, complaint='Yes or no, please!'):
    while True:
        ok = raw_input(prompt)
        if ok in ('y', 'ye', 'yes'): return 1
        if ok in ('n', 'no', 'nop', 'nope'): return 0
        retries = retries - 1
        if retries < 0: raise IOError, 'refusenik user'
        print complaint
\end{verbatim}

This function can be called either like this:
\code{ask_ok('Do you really want to quit?')} or like this:
\code{ask_ok('OK to overwrite the file?', 2)}.

The default values are evaluated at the point of function definition
in the \emph{defining} scope, so that

\begin{verbatim}
i = 5

def f(arg=i):
    print arg

i = 6
f()
\end{verbatim}

will print \code{5}.

\strong{Important warning:}  The default value is evaluated only once.
This makes a difference when the default is a mutable object such as a
list, dictionary, or instances of most classes.  For example, the
following function accumulates the arguments passed to it on
subsequent calls:

\begin{verbatim}
def f(a, L=[]):
    L.append(a)
    return L

print f(1)
print f(2)
print f(3)
\end{verbatim}

This will print

\begin{verbatim}
[1]
[1, 2]
[1, 2, 3]
\end{verbatim}

If you don't want the default to be shared between subsequent calls,
you can write the function like this instead:

\begin{verbatim}
def f(a, L=None):
    if L is None:
        L = []
    L.append(a)
    return L
\end{verbatim}

\subsection{Keyword Arguments \label{keywordArgs}}

Functions can also be called using
keyword arguments of the form \samp{\var{keyword} = \var{value}}.  For
instance, the following function:

\begin{verbatim}
def parrot(voltage, state='a stiff', action='voom', type='Norwegian Blue'):
    print "-- This parrot wouldn't", action,
    print "if you put", voltage, "Volts through it."
    print "-- Lovely plumage, the", type
    print "-- It's", state, "!"
\end{verbatim}

could be called in any of the following ways:

\begin{verbatim}
parrot(1000)
parrot(action = 'VOOOOOM', voltage = 1000000)
parrot('a thousand', state = 'pushing up the daisies')
parrot('a million', 'bereft of life', 'jump')
\end{verbatim}

but the following calls would all be invalid:

\begin{verbatim}
parrot()                     # required argument missing
parrot(voltage=5.0, 'dead')  # non-keyword argument following keyword
parrot(110, voltage=220)     # duplicate value for argument
parrot(actor='John Cleese')  # unknown keyword
\end{verbatim}

In general, an argument list must have any positional arguments
followed by any keyword arguments, where the keywords must be chosen
from the formal parameter names.  It's not important whether a formal
parameter has a default value or not.  No argument may receive a
value more than once --- formal parameter names corresponding to
positional arguments cannot be used as keywords in the same calls.
Here's an example that fails due to this restriction:

\begin{verbatim}
>>> def function(a):
...     pass
... 
>>> function(0, a=0)
Traceback (most recent call last):
  File "<stdin>", line 1, in ?
TypeError: function() got multiple values for keyword argument 'a'
\end{verbatim}

When a final formal parameter of the form \code{**\var{name}} is
present, it receives a dictionary containing all keyword arguments
whose keyword doesn't correspond to a formal parameter.  This may be
combined with a formal parameter of the form
\code{*\var{name}} (described in the next subsection) which receives a
tuple containing the positional arguments beyond the formal parameter
list.  (\code{*\var{name}} must occur before \code{**\var{name}}.)
For example, if we define a function like this:

\begin{verbatim}
def cheeseshop(kind, *arguments, **keywords):
    print "-- Do you have any", kind, '?'
    print "-- I'm sorry, we're all out of", kind
    for arg in arguments: print arg
    print '-'*40
    keys = keywords.keys()
    keys.sort()
    for kw in keys: print kw, ':', keywords[kw]
\end{verbatim}

It could be called like this:

\begin{verbatim}
cheeseshop('Limburger', "It's very runny, sir.",
           "It's really very, VERY runny, sir.",
           client='John Cleese',
           shopkeeper='Michael Palin',
           sketch='Cheese Shop Sketch')
\end{verbatim}

and of course it would print:

\begin{verbatim}
-- Do you have any Limburger ?
-- I'm sorry, we're all out of Limburger
It's very runny, sir.
It's really very, VERY runny, sir.
----------------------------------------
client : John Cleese
shopkeeper : Michael Palin
sketch : Cheese Shop Sketch
\end{verbatim}

Note that the \method{sort()} method of the list of keyword argument
names is called before printing the contents of the \code{keywords}
dictionary; if this is not done, the order in which the arguments are
printed is undefined.


\subsection{Arbitrary Argument Lists \label{arbitraryArgs}}

Finally, the least frequently used option is to specify that a
function can be called with an arbitrary number of arguments.  These
arguments will be wrapped up in a tuple.  Before the variable number
of arguments, zero or more normal arguments may occur.

\begin{verbatim}
def fprintf(file, format, *args):
    file.write(format % args)
\end{verbatim}


\subsection{Lambda Forms \label{lambda}}

By popular demand, a few features commonly found in functional
programming languages and Lisp have been added to Python.  With the
\keyword{lambda} keyword, small anonymous functions can be created.
Here's a function that returns the sum of its two arguments:
\samp{lambda a, b: a+b}.  Lambda forms can be used wherever function
objects are required.  They are syntactically restricted to a single
expression.  Semantically, they are just syntactic sugar for a normal
function definition.  Like nested function definitions, lambda forms
can reference variables from the containing scope:

\begin{verbatim}
>>> def make_incrementor(n):
...     return lambda x: x + n
...
>>> f = make_incrementor(42)
>>> f(0)
42
>>> f(1)
43
\end{verbatim}


\subsection{Documentation Strings \label{docstrings}}

There are emerging conventions about the content and formatting of
documentation strings.
\index{docstrings}\index{documentation strings}
\index{strings, documentation}

The first line should always be a short, concise summary of the
object's purpose.  For brevity, it should not explicitly state the
object's name or type, since these are available by other means
(except if the name happens to be a verb describing a function's
operation).  This line should begin with a capital letter and end with
a period.

If there are more lines in the documentation string, the second line
should be blank, visually separating the summary from the rest of the
description.  The following lines should be one or more paragraphs
describing the object's calling conventions, its side effects, etc.

The Python parser does not strip indentation from multi-line string
literals in Python, so tools that process documentation have to strip
indentation if desired.  This is done using the following convention.
The first non-blank line \emph{after} the first line of the string
determines the amount of indentation for the entire documentation
string.  (We can't use the first line since it is generally adjacent
to the string's opening quotes so its indentation is not apparent in
the string literal.)  Whitespace ``equivalent'' to this indentation is
then stripped from the start of all lines of the string.  Lines that
are indented less should not occur, but if they occur all their
leading whitespace should be stripped.  Equivalence of whitespace
should be tested after expansion of tabs (to 8 spaces, normally).

Here is an example of a multi-line docstring:

\begin{verbatim}
>>> def my_function():
...     """Do nothing, but document it.
... 
...     No, really, it doesn't do anything.
...     """
...     pass
... 
>>> print my_function.__doc__
Do nothing, but document it.

    No, really, it doesn't do anything.
    
\end{verbatim}



\chapter{Data Structures \label{structures}}

This chapter describes some things you've learned about already in
more detail, and adds some new things as well.


\section{More on Lists \label{moreLists}}

The list data type has some more methods.  Here are all of the methods
of list objects:

\begin{methoddesc}[list]{append}{x}
Add an item to the end of the list;
equivalent to \code{a[len(a):] = [\var{x}]}.
\end{methoddesc}

\begin{methoddesc}[list]{extend}{L}
Extend the list by appending all the items in the given list;
equivalent to \code{a[len(a):] = \var{L}}.
\end{methoddesc}

\begin{methoddesc}[list]{insert}{i, x}
Insert an item at a given position.  The first argument is the index
of the element before which to insert, so \code{a.insert(0, \var{x})}
inserts at the front of the list, and \code{a.insert(len(a), \var{x})}
is equivalent to \code{a.append(\var{x})}.
\end{methoddesc}

\begin{methoddesc}[list]{remove}{x}
Remove the first item from the list whose value is \var{x}.
It is an error if there is no such item.
\end{methoddesc}

\begin{methoddesc}[list]{pop}{\optional{i}}
Remove the item at the given position in the list, and return it.  If
no index is specified, \code{a.pop()} returns the last item in the
list.  The item is also removed from the list.  (The square brackets
around the \var{i} in the method signature denote that the parameter
is optional, not that you should type square brackets at that
position.  You will see this notation frequently in the
\citetitle[../lib/lib.html]{Python Library Reference}.)
\end{methoddesc}

\begin{methoddesc}[list]{index}{x}
Return the index in the list of the first item whose value is \var{x}.
It is an error if there is no such item.
\end{methoddesc}

\begin{methoddesc}[list]{count}{x}
Return the number of times \var{x} appears in the list.
\end{methoddesc}

\begin{methoddesc}[list]{sort}{}
Sort the items of the list, in place.
\end{methoddesc}

\begin{methoddesc}[list]{reverse}{}
Reverse the elements of the list, in place.
\end{methoddesc}

An example that uses most of the list methods:

\begin{verbatim}
>>> a = [66.6, 333, 333, 1, 1234.5]
>>> print a.count(333), a.count(66.6), a.count('x')
2 1 0
>>> a.insert(2, -1)
>>> a.append(333)
>>> a
[66.6, 333, -1, 333, 1, 1234.5, 333]
>>> a.index(333)
1
>>> a.remove(333)
>>> a
[66.6, -1, 333, 1, 1234.5, 333]
>>> a.reverse()
>>> a
[333, 1234.5, 1, 333, -1, 66.6]
>>> a.sort()
>>> a
[-1, 1, 66.6, 333, 333, 1234.5]
\end{verbatim}


\subsection{Using Lists as Stacks \label{lists-as-stacks}}
\sectionauthor{Ka-Ping Yee}{ping@lfw.org}

The list methods make it very easy to use a list as a stack, where the
last element added is the first element retrieved (``last-in,
first-out'').  To add an item to the top of the stack, use
\method{append()}.  To retrieve an item from the top of the stack, use
\method{pop()} without an explicit index.  For example:

\begin{verbatim}
>>> stack = [3, 4, 5]
>>> stack.append(6)
>>> stack.append(7)
>>> stack
[3, 4, 5, 6, 7]
>>> stack.pop()
7
>>> stack
[3, 4, 5, 6]
>>> stack.pop()
6
>>> stack.pop()
5
>>> stack
[3, 4]
\end{verbatim}


\subsection{Using Lists as Queues \label{lists-as-queues}}
\sectionauthor{Ka-Ping Yee}{ping@lfw.org}

You can also use a list conveniently as a queue, where the first
element added is the first element retrieved (``first-in,
first-out'').  To add an item to the back of the queue, use
\method{append()}.  To retrieve an item from the front of the queue,
use \method{pop()} with \code{0} as the index.  For example:

\begin{verbatim}
>>> queue = ["Eric", "John", "Michael"]
>>> queue.append("Terry")           # Terry arrives
>>> queue.append("Graham")          # Graham arrives
>>> queue.pop(0)
'Eric'
>>> queue.pop(0)
'John'
>>> queue
['Michael', 'Terry', 'Graham']
\end{verbatim}


\subsection{Functional Programming Tools \label{functional}}

There are three built-in functions that are very useful when used with
lists: \function{filter()}, \function{map()}, and \function{reduce()}.

\samp{filter(\var{function}, \var{sequence})} returns a sequence (of
the same type, if possible) consisting of those items from the
sequence for which \code{\var{function}(\var{item})} is true.  For
example, to compute some primes:

\begin{verbatim}
>>> def f(x): return x % 2 != 0 and x % 3 != 0
...
>>> filter(f, range(2, 25))
[5, 7, 11, 13, 17, 19, 23]
\end{verbatim}

\samp{map(\var{function}, \var{sequence})} calls
\code{\var{function}(\var{item})} for each of the sequence's items and
returns a list of the return values.  For example, to compute some
cubes:

\begin{verbatim}
>>> def cube(x): return x*x*x
...
>>> map(cube, range(1, 11))
[1, 8, 27, 64, 125, 216, 343, 512, 729, 1000]
\end{verbatim}

More than one sequence may be passed; the function must then have as
many arguments as there are sequences and is called with the
corresponding item from each sequence (or \code{None} if some sequence
is shorter than another).  If \code{None} is passed for the function,
a function returning its argument(s) is substituted.

Combining these two special cases, we see that
\samp{map(None, \var{list1}, \var{list2})} is a convenient way of
turning a pair of lists into a list of pairs.  For example:

\begin{verbatim}
>>> seq = range(8)
>>> def square(x): return x*x
...
>>> map(None, seq, map(square, seq))
[(0, 0), (1, 1), (2, 4), (3, 9), (4, 16), (5, 25), (6, 36), (7, 49)]
\end{verbatim}

\samp{reduce(\var{func}, \var{sequence})} returns a single value
constructed by calling the binary function \var{func} on the first two
items of the sequence, then on the result and the next item, and so
on.  For example, to compute the sum of the numbers 1 through 10:

\begin{verbatim}
>>> def add(x,y): return x+y
...
>>> reduce(add, range(1, 11))
55
\end{verbatim}

If there's only one item in the sequence, its value is returned; if
the sequence is empty, an exception is raised.

A third argument can be passed to indicate the starting value.  In this
case the starting value is returned for an empty sequence, and the
function is first applied to the starting value and the first sequence
item, then to the result and the next item, and so on.  For example,

\begin{verbatim}
>>> def sum(seq):
...     def add(x,y): return x+y
...     return reduce(add, seq, 0)
... 
>>> sum(range(1, 11))
55
>>> sum([])
0
\end{verbatim}

Don't use this example's definition of \function{sum()}: since summing
numbers is such a common need, a built-in function
\code{sum(\var{sequence})} is already provided, and works exactly like
this.
\versionadded{2.3}

\subsection{List Comprehensions}

List comprehensions provide a concise way to create lists without resorting
to use of \function{map()}, \function{filter()} and/or \keyword{lambda}.
The resulting list definition tends often to be clearer than lists built
using those constructs.  Each list comprehension consists of an expression
followed by a \keyword{for} clause, then zero or more \keyword{for} or
\keyword{if} clauses.  The result will be a list resulting from evaluating
the expression in the context of the \keyword{for} and \keyword{if} clauses
which follow it.  If the expression would evaluate to a tuple, it must be
parenthesized.

\begin{verbatim}
>>> freshfruit = ['  banana', '  loganberry ', 'passion fruit  ']
>>> [weapon.strip() for weapon in freshfruit]
['banana', 'loganberry', 'passion fruit']
>>> vec = [2, 4, 6]
>>> [3*x for x in vec]
[6, 12, 18]
>>> [3*x for x in vec if x > 3]
[12, 18]
>>> [3*x for x in vec if x < 2]
[]
>>> [[x,x**2] for x in vec]
[[2, 4], [4, 16], [6, 36]]
>>> [x, x**2 for x in vec]	# error - parens required for tuples
  File "<stdin>", line 1, in ?
    [x, x**2 for x in vec]
               ^
SyntaxError: invalid syntax
>>> [(x, x**2) for x in vec]
[(2, 4), (4, 16), (6, 36)]
>>> vec1 = [2, 4, 6]
>>> vec2 = [4, 3, -9]
>>> [x*y for x in vec1 for y in vec2]
[8, 6, -18, 16, 12, -36, 24, 18, -54]
>>> [x+y for x in vec1 for y in vec2]
[6, 5, -7, 8, 7, -5, 10, 9, -3]
>>> [vec1[i]*vec2[i] for i in range(len(vec1))]
[8, 12, -54]
\end{verbatim}

To make list comprehensions match the behavior of \keyword{for}
loops, assignments to the loop variable remain visible outside
of the comprehension:

\begin{verbatim}
>>> x = 100                     # this gets overwritten
>>> [x**3 for x in range(5)]
[0, 1, 8, 27, 64]
>>> x                           # the final value for range(5)
4
\end{verbatim}


\section{The \keyword{del} statement \label{del}}

There is a way to remove an item from a list given its index instead
of its value: the \keyword{del} statement.  This can also be used to
remove slices from a list (which we did earlier by assignment of an
empty list to the slice).  For example:

\begin{verbatim}
>>> a = [-1, 1, 66.6, 333, 333, 1234.5]
>>> del a[0]
>>> a
[1, 66.6, 333, 333, 1234.5]
>>> del a[2:4]
>>> a
[1, 66.6, 1234.5]
\end{verbatim}

\keyword{del} can also be used to delete entire variables:

\begin{verbatim}
>>> del a
\end{verbatim}

Referencing the name \code{a} hereafter is an error (at least until
another value is assigned to it).  We'll find other uses for
\keyword{del} later.


\section{Tuples and Sequences \label{tuples}}

We saw that lists and strings have many common properties, such as
indexing and slicing operations.  They are two examples of
\emph{sequence} data types.  Since Python is an evolving language,
other sequence data types may be added.  There is also another
standard sequence data type: the \emph{tuple}.

A tuple consists of a number of values separated by commas, for
instance:

\begin{verbatim}
>>> t = 12345, 54321, 'hello!'
>>> t[0]
12345
>>> t
(12345, 54321, 'hello!')
>>> # Tuples may be nested:
... u = t, (1, 2, 3, 4, 5)
>>> u
((12345, 54321, 'hello!'), (1, 2, 3, 4, 5))
\end{verbatim}

As you see, on output tuples are alway enclosed in parentheses, so
that nested tuples are interpreted correctly; they may be input with
or without surrounding parentheses, although often parentheses are
necessary anyway (if the tuple is part of a larger expression).

Tuples have many uses.  For example: (x, y) coordinate pairs, employee
records from a database, etc.  Tuples, like strings, are immutable: it
is not possible to assign to the individual items of a tuple (you can
simulate much of the same effect with slicing and concatenation,
though).  It is also possible to create tuples which contain mutable
objects, such as lists.

A special problem is the construction of tuples containing 0 or 1
items: the syntax has some extra quirks to accommodate these.  Empty
tuples are constructed by an empty pair of parentheses; a tuple with
one item is constructed by following a value with a comma
(it is not sufficient to enclose a single value in parentheses).
Ugly, but effective.  For example:

\begin{verbatim}
>>> empty = ()
>>> singleton = 'hello',    # <-- note trailing comma
>>> len(empty)
0
>>> len(singleton)
1
>>> singleton
('hello',)
\end{verbatim}

The statement \code{t = 12345, 54321, 'hello!'} is an example of
\emph{tuple packing}: the values \code{12345}, \code{54321} and
\code{'hello!'} are packed together in a tuple.  The reverse operation
is also possible:

\begin{verbatim}
>>> x, y, z = t
\end{verbatim}

This is called, appropriately enough, \emph{sequence unpacking}.
Sequence unpacking requires that the list of variables on the left
have the same number of elements as the length of the sequence.  Note
that multiple assignment is really just a combination of tuple packing
and sequence unpacking!

There is a small bit of asymmetry here:  packing multiple values
always creates a tuple, and unpacking works for any sequence.

% XXX Add a bit on the difference between tuples and lists.


\section{Dictionaries \label{dictionaries}}

Another useful data type built into Python is the \emph{dictionary}.
Dictionaries are sometimes found in other languages as ``associative
memories'' or ``associative arrays''.  Unlike sequences, which are
indexed by a range of numbers, dictionaries are indexed by \emph{keys},
which can be any immutable type; strings and numbers can always be
keys.  Tuples can be used as keys if they contain only strings,
numbers, or tuples; if a tuple contains any mutable object either
directly or indirectly, it cannot be used as a key.  You can't use
lists as keys, since lists can be modified in place using their
\method{append()} and \method{extend()} methods, as well as slice and
indexed assignments.

It is best to think of a dictionary as an unordered set of
\emph{key: value} pairs, with the requirement that the keys are unique
(within one dictionary).
A pair of braces creates an empty dictionary: \code{\{\}}.
Placing a comma-separated list of key:value pairs within the
braces adds initial key:value pairs to the dictionary; this is also the
way dictionaries are written on output.

The main operations on a dictionary are storing a value with some key
and extracting the value given the key.  It is also possible to delete
a key:value pair
with \code{del}.
If you store using a key that is already in use, the old value
associated with that key is forgotten.  It is an error to extract a
value using a non-existent key.

The \code{keys()} method of a dictionary object returns a list of all
the keys used in the dictionary, in random order (if you want it
sorted, just apply the \code{sort()} method to the list of keys).  To
check whether a single key is in the dictionary, use the
\code{has_key()} method of the dictionary.

Here is a small example using a dictionary:

\begin{verbatim}
>>> tel = {'jack': 4098, 'sape': 4139}
>>> tel['guido'] = 4127
>>> tel
{'sape': 4139, 'guido': 4127, 'jack': 4098}
>>> tel['jack']
4098
>>> del tel['sape']
>>> tel['irv'] = 4127
>>> tel
{'guido': 4127, 'irv': 4127, 'jack': 4098}
>>> tel.keys()
['guido', 'irv', 'jack']
>>> tel.has_key('guido')
True
\end{verbatim}

The \function{dict()} contructor builds dictionaries directly from
lists of key-value pairs stored as tuples.  When the pairs form a
pattern, list comprehensions can compactly specify the key-value list.

\begin{verbatim}
>>> dict([('sape', 4139), ('guido', 4127), ('jack', 4098)])
{'sape': 4139, 'jack': 4098, 'guido': 4127}
>>> dict([(x, x**2) for x in vec])     # use a list comprehension
{2: 4, 4: 16, 6: 36}
\end{verbatim}


\section{Looping Techniques \label{loopidioms}}

When looping through dictionaries, the key and corresponding value can
be retrieved at the same time using the \method{items()} method.

\begin{verbatim}
>>> knights = {'gallahad': 'the pure', 'robin': 'the brave'}
>>> for k, v in knights.items():
...     print k, v
...
gallahad the pure
robin the brave
\end{verbatim}
 
When looping through a sequence, the position index and corresponding
value can be retrieved at the same time using the
\function{enumerate()} function.

\begin{verbatim} 
>>> for i, v in enumerate(['tic', 'tac', 'toe']):
...     print i, v
...
0 tic
1 tac
2 toe
\end{verbatim}

To loop over two or more sequences at the same time, the entries
can be paired with the \function{zip()} function.

\begin{verbatim}
>>> questions = ['name', 'quest', 'favorite color']
>>> answers = ['lancelot', 'the holy grail', 'blue']
>>> for q, a in zip(questions, answers):
...     print 'What is your %s?  It is %s.' % (q, a)
...	
What is your name?  It is lancelot.
What is your quest?  It is the holy grail.
What is your favorite color?  It is blue.
\end{verbatim}


\section{More on Conditions \label{conditions}}

The conditions used in \code{while} and \code{if} statements above can
contain other operators besides comparisons.

The comparison operators \code{in} and \code{not in} check whether a value
occurs (does not occur) in a sequence.  The operators \code{is} and
\code{is not} compare whether two objects are really the same object; this
only matters for mutable objects like lists.  All comparison operators
have the same priority, which is lower than that of all numerical
operators.

Comparisons can be chained.  For example, \code{a < b == c} tests
whether \code{a} is less than \code{b} and moreover \code{b} equals
\code{c}.

Comparisons may be combined by the Boolean operators \code{and} and
\code{or}, and the outcome of a comparison (or of any other Boolean
expression) may be negated with \code{not}.  These all have lower
priorities than comparison operators again; between them, \code{not} has
the highest priority, and \code{or} the lowest, so that
\code{A and not B or C} is equivalent to \code{(A and (not B)) or C}.  Of
course, parentheses can be used to express the desired composition.

The Boolean operators \code{and} and \code{or} are so-called
\emph{short-circuit} operators: their arguments are evaluated from
left to right, and evaluation stops as soon as the outcome is
determined.  For example, if \code{A} and \code{C} are true but
\code{B} is false, \code{A and B and C} does not evaluate the
expression \code{C}.  In general, the return value of a short-circuit
operator, when used as a general value and not as a Boolean, is the
last evaluated argument.

It is possible to assign the result of a comparison or other Boolean
expression to a variable.  For example,

\begin{verbatim}
>>> string1, string2, string3 = '', 'Trondheim', 'Hammer Dance'
>>> non_null = string1 or string2 or string3
>>> non_null
'Trondheim'
\end{verbatim}

Note that in Python, unlike C, assignment cannot occur inside expressions.
C programmers may grumble about this, but it avoids a common class of
problems encountered in C programs: typing \code{=} in an expression when
\code{==} was intended.


\section{Comparing Sequences and Other Types \label{comparing}}

Sequence objects may be compared to other objects with the same
sequence type.  The comparison uses \emph{lexicographical} ordering:
first the first two items are compared, and if they differ this
determines the outcome of the comparison; if they are equal, the next
two items are compared, and so on, until either sequence is exhausted.
If two items to be compared are themselves sequences of the same type,
the lexicographical comparison is carried out recursively.  If all
items of two sequences compare equal, the sequences are considered
equal.  If one sequence is an initial sub-sequence of the other, the
shorter sequence is the smaller (lesser) one.  Lexicographical
ordering for strings uses the \ASCII{} ordering for individual
characters.  Some examples of comparisons between sequences with the
same types:

\begin{verbatim}
(1, 2, 3)              < (1, 2, 4)
[1, 2, 3]              < [1, 2, 4]
'ABC' < 'C' < 'Pascal' < 'Python'
(1, 2, 3, 4)           < (1, 2, 4)
(1, 2)                 < (1, 2, -1)
(1, 2, 3)             == (1.0, 2.0, 3.0)
(1, 2, ('aa', 'ab'))   < (1, 2, ('abc', 'a'), 4)
\end{verbatim}

Note that comparing objects of different types is legal.  The outcome
is deterministic but arbitrary: the types are ordered by their name.
Thus, a list is always smaller than a string, a string is always
smaller than a tuple, etc.  Mixed numeric types are compared according
to their numeric value, so 0 equals 0.0, etc.\footnote{
        The rules for comparing objects of different types should
        not be relied upon; they may change in a future version of
        the language.
}


\chapter{Modules \label{modules}}

If you quit from the Python interpreter and enter it again, the
definitions you have made (functions and variables) are lost.
Therefore, if you want to write a somewhat longer program, you are
better off using a text editor to prepare the input for the interpreter
and running it with that file as input instead.  This is known as creating a
\emph{script}.  As your program gets longer, you may want to split it
into several files for easier maintenance.  You may also want to use a
handy function that you've written in several programs without copying
its definition into each program.

To support this, Python has a way to put definitions in a file and use
them in a script or in an interactive instance of the interpreter.
Such a file is called a \emph{module}; definitions from a module can be
\emph{imported} into other modules or into the \emph{main} module (the
collection of variables that you have access to in a script
executed at the top level
and in calculator mode).

A module is a file containing Python definitions and statements.  The
file name is the module name with the suffix \file{.py} appended.  Within
a module, the module's name (as a string) is available as the value of
the global variable \code{__name__}.  For instance, use your favorite text
editor to create a file called \file{fibo.py} in the current directory
with the following contents:

\begin{verbatim}
# Fibonacci numbers module

def fib(n):    # write Fibonacci series up to n
    a, b = 0, 1
    while b < n:
        print b,
        a, b = b, a+b

def fib2(n): # return Fibonacci series up to n
    result = []
    a, b = 0, 1
    while b < n:
        result.append(b)
        a, b = b, a+b
    return result
\end{verbatim}

Now enter the Python interpreter and import this module with the
following command:

\begin{verbatim}
>>> import fibo
\end{verbatim}

This does not enter the names of the functions defined in \code{fibo} 
directly in the current symbol table; it only enters the module name
\code{fibo} there.
Using the module name you can access the functions:

\begin{verbatim}
>>> fibo.fib(1000)
1 1 2 3 5 8 13 21 34 55 89 144 233 377 610 987
>>> fibo.fib2(100)
[1, 1, 2, 3, 5, 8, 13, 21, 34, 55, 89]
>>> fibo.__name__
'fibo'
\end{verbatim}

If you intend to use a function often you can assign it to a local name:

\begin{verbatim}
>>> fib = fibo.fib
>>> fib(500)
1 1 2 3 5 8 13 21 34 55 89 144 233 377
\end{verbatim}


\section{More on Modules \label{moreModules}}

A module can contain executable statements as well as function
definitions.
These statements are intended to initialize the module.
They are executed only the
\emph{first} time the module is imported somewhere.\footnote{
        In fact function definitions are also `statements' that are
        `executed'; the execution enters the function name in the
        module's global symbol table.
}

Each module has its own private symbol table, which is used as the
global symbol table by all functions defined in the module.
Thus, the author of a module can use global variables in the module
without worrying about accidental clashes with a user's global
variables.
On the other hand, if you know what you are doing you can touch a
module's global variables with the same notation used to refer to its
functions,
\code{modname.itemname}.

Modules can import other modules.  It is customary but not required to
place all \keyword{import} statements at the beginning of a module (or
script, for that matter).  The imported module names are placed in the
importing module's global symbol table.

There is a variant of the \keyword{import} statement that imports
names from a module directly into the importing module's symbol
table.  For example:

\begin{verbatim}
>>> from fibo import fib, fib2
>>> fib(500)
1 1 2 3 5 8 13 21 34 55 89 144 233 377
\end{verbatim}

This does not introduce the module name from which the imports are taken
in the local symbol table (so in the example, \code{fibo} is not
defined).

There is even a variant to import all names that a module defines:

\begin{verbatim}
>>> from fibo import *
>>> fib(500)
1 1 2 3 5 8 13 21 34 55 89 144 233 377
\end{verbatim}

This imports all names except those beginning with an underscore
(\code{_}).


\subsection{The Module Search Path \label{searchPath}}

\indexiii{module}{search}{path}
When a module named \module{spam} is imported, the interpreter searches
for a file named \file{spam.py} in the current directory,
and then in the list of directories specified by
the environment variable \envvar{PYTHONPATH}.  This has the same syntax as
the shell variable \envvar{PATH}, that is, a list of
directory names.  When \envvar{PYTHONPATH} is not set, or when the file
is not found there, the search continues in an installation-dependent
default path; on \UNIX, this is usually \file{.:/usr/local/lib/python}.

Actually, modules are searched in the list of directories given by the 
variable \code{sys.path} which is initialized from the directory 
containing the input script (or the current directory),
\envvar{PYTHONPATH} and the installation-dependent default.  This allows
Python programs that know what they're doing to modify or replace the 
module search path.  Note that because the directory containing the
script being run is on the search path, it is important that the
script not have the same name as a standard module, or Python will
attempt to load the script as a module when that module is imported.
This will generally be an error.  See section~\ref{standardModules},
``Standard Modules.'' for more information.


\subsection{``Compiled'' Python files}

As an important speed-up of the start-up time for short programs that
use a lot of standard modules, if a file called \file{spam.pyc} exists
in the directory where \file{spam.py} is found, this is assumed to
contain an already-``byte-compiled'' version of the module \module{spam}.
The modification time of the version of \file{spam.py} used to create
\file{spam.pyc} is recorded in \file{spam.pyc}, and the
\file{.pyc} file is ignored if these don't match.

Normally, you don't need to do anything to create the
\file{spam.pyc} file.  Whenever \file{spam.py} is successfully
compiled, an attempt is made to write the compiled version to
\file{spam.pyc}.  It is not an error if this attempt fails; if for any
reason the file is not written completely, the resulting
\file{spam.pyc} file will be recognized as invalid and thus ignored
later.  The contents of the \file{spam.pyc} file are platform
independent, so a Python module directory can be shared by machines of
different architectures.

Some tips for experts:

\begin{itemize}

\item
When the Python interpreter is invoked with the \programopt{-O} flag,
optimized code is generated and stored in \file{.pyo} files.  The
optimizer currently doesn't help much; it only removes
\keyword{assert} statements.  When \programopt{-O} is used, \emph{all}
bytecode is optimized; \code{.pyc} files are ignored and \code{.py}
files are compiled to optimized bytecode.

\item
Passing two \programopt{-O} flags to the Python interpreter
(\programopt{-OO}) will cause the bytecode compiler to perform
optimizations that could in some rare cases result in malfunctioning
programs.  Currently only \code{__doc__} strings are removed from the
bytecode, resulting in more compact \file{.pyo} files.  Since some
programs may rely on having these available, you should only use this
option if you know what you're doing.

\item
A program doesn't run any faster when it is read from a \file{.pyc} or
\file{.pyo} file than when it is read from a \file{.py} file; the only
thing that's faster about \file{.pyc} or \file{.pyo} files is the
speed with which they are loaded.

\item
When a script is run by giving its name on the command line, the
bytecode for the script is never written to a \file{.pyc} or
\file{.pyo} file.  Thus, the startup time of a script may be reduced
by moving most of its code to a module and having a small bootstrap
script that imports that module.  It is also possible to name a
\file{.pyc} or \file{.pyo} file directly on the command line.

\item
It is possible to have a file called \file{spam.pyc} (or
\file{spam.pyo} when \programopt{-O} is used) without a file
\file{spam.py} for the same module.  This can be used to distribute a
library of Python code in a form that is moderately hard to reverse
engineer.

\item
The module \module{compileall}\refstmodindex{compileall} can create
\file{.pyc} files (or \file{.pyo} files when \programopt{-O} is used) for
all modules in a directory.

\end{itemize}


\section{Standard Modules \label{standardModules}}

Python comes with a library of standard modules, described in a separate
document, the \citetitle[../lib/lib.html]{Python Library Reference}
(``Library Reference'' hereafter).  Some modules are built into the
interpreter; these provide access to operations that are not part of
the core of the language but are nevertheless built in, either for
efficiency or to provide access to operating system primitives such as
system calls.  The set of such modules is a configuration option which
also dependson the underlying platform  For example,
the \module{amoeba} module is only provided on systems that somehow
support Amoeba primitives.  One particular module deserves some
attention: \module{sys}\refstmodindex{sys}, which is built into every
Python interpreter.  The variables \code{sys.ps1} and
\code{sys.ps2} define the strings used as primary and secondary
prompts:

\begin{verbatim}
>>> import sys
>>> sys.ps1
'>>> '
>>> sys.ps2
'... '
>>> sys.ps1 = 'C> '
C> print 'Yuck!'
Yuck!
C>

\end{verbatim}

These two variables are only defined if the interpreter is in
interactive mode.

The variable \code{sys.path} is a list of strings that determine the
interpreter's search path for modules. It is initialized to a default
path taken from the environment variable \envvar{PYTHONPATH}, or from
a built-in default if \envvar{PYTHONPATH} is not set.  You can modify
it using standard list operations: 

\begin{verbatim}
>>> import sys
>>> sys.path.append('/ufs/guido/lib/python')
\end{verbatim}

\section{The \function{dir()} Function \label{dir}}

The built-in function \function{dir()} is used to find out which names
a module defines.  It returns a sorted list of strings:

\begin{verbatim}
>>> import fibo, sys
>>> dir(fibo)
['__name__', 'fib', 'fib2']
>>> dir(sys)
['__displayhook__', '__doc__', '__excepthook__', '__name__', '__stderr__',
 '__stdin__', '__stdout__', '_getframe', 'api_version', 'argv', 
 'builtin_module_names', 'byteorder', 'callstats', 'copyright',
 'displayhook', 'exc_clear', 'exc_info', 'exc_type', 'excepthook',
 'exec_prefix', 'executable', 'exit', 'getdefaultencoding', 'getdlopenflags',
 'getrecursionlimit', 'getrefcount', 'hexversion', 'maxint', 'maxunicode',
 'meta_path', 'modules', 'path', 'path_hooks', 'path_importer_cache',
 'platform', 'prefix', 'ps1', 'ps2', 'setcheckinterval', 'setdlopenflags',
 'setprofile', 'setrecursionlimit', 'settrace', 'stderr', 'stdin', 'stdout',
 'version', 'version_info', 'warnoptions']
\end{verbatim}

Without arguments, \function{dir()} lists the names you have defined
currently:

\begin{verbatim}
>>> a = [1, 2, 3, 4, 5]
>>> import fibo, sys
>>> fib = fibo.fib
>>> dir()
['__name__', 'a', 'fib', 'fibo', 'sys']
\end{verbatim}

Note that it lists all types of names: variables, modules, functions, etc.

\function{dir()} does not list the names of built-in functions and
variables.  If you want a list of those, they are defined in the
standard module \module{__builtin__}\refbimodindex{__builtin__}:

\begin{verbatim}
>>> import __builtin__
>>> dir(__builtin__)
['ArithmeticError', 'AssertionError', 'AttributeError',
 'DeprecationWarning', 'EOFError', 'Ellipsis', 'EnvironmentError',
 'Exception', 'False', 'FloatingPointError', 'IOError', 'ImportError',
 'IndentationError', 'IndexError', 'KeyError', 'KeyboardInterrupt',
 'LookupError', 'MemoryError', 'NameError', 'None', 'NotImplemented',
 'NotImplementedError', 'OSError', 'OverflowError', 'OverflowWarning',
 'PendingDeprecationWarning', 'ReferenceError',
 'RuntimeError', 'RuntimeWarning', 'StandardError', 'StopIteration',
 'SyntaxError', 'SyntaxWarning', 'SystemError', 'SystemExit', 'TabError',
 'True', 'TypeError', 'UnboundLocalError', 'UnicodeError', 'UserWarning',
 'ValueError', 'Warning', 'ZeroDivisionError', '__debug__', '__doc__',
 '__import__', '__name__', 'abs', 'apply', 'bool', 'buffer',
 'callable', 'chr', 'classmethod', 'cmp', 'coerce', 'compile', 'complex',
 'copyright', 'credits', 'delattr', 'dict', 'dir', 'divmod',
 'enumerate', 'eval', 'execfile', 'exit', 'file', 'filter', 'float',
 'getattr', 'globals', 'hasattr', 'hash', 'help', 'hex', 'id',
 'input', 'int', 'intern', 'isinstance', 'issubclass', 'iter',
 'len', 'license', 'list', 'locals', 'long', 'map', 'max', 'min',
 'object', 'oct', 'open', 'ord', 'pow', 'property', 'quit',
 'range', 'raw_input', 'reduce', 'reload', 'repr', 'round',
 'setattr', 'slice', 'staticmethod', 'str', 'string', 'sum', 'super',
 'tuple', 'type', 'unichr', 'unicode', 'vars', 'xrange', 'zip']
\end{verbatim}


\section{Packages \label{packages}}

Packages are a way of structuring Python's module namespace
by using ``dotted module names''.  For example, the module name
\module{A.B} designates a submodule named \samp{B} in a package named
\samp{A}.  Just like the use of modules saves the authors of different
modules from having to worry about each other's global variable names,
the use of dotted module names saves the authors of multi-module
packages like NumPy or the Python Imaging Library from having to worry
about each other's module names.

Suppose you want to design a collection of modules (a ``package'') for
the uniform handling of sound files and sound data.  There are many
different sound file formats (usually recognized by their extension,
for example: \file{.wav}, \file{.aiff}, \file{.au}), so you may need
to create and maintain a growing collection of modules for the
conversion between the various file formats.  There are also many
different operations you might want to perform on sound data (such as
mixing, adding echo, applying an equalizer function, creating an
artificial stereo effect), so in addition you will be writing a
never-ending stream of modules to perform these operations.  Here's a
possible structure for your package (expressed in terms of a
hierarchical filesystem):

\begin{verbatim}
Sound/                          Top-level package
      __init__.py               Initialize the sound package
      Formats/                  Subpackage for file format conversions
              __init__.py
              wavread.py
              wavwrite.py
              aiffread.py
              aiffwrite.py
              auread.py
              auwrite.py
              ...
      Effects/                  Subpackage for sound effects
              __init__.py
              echo.py
              surround.py
              reverse.py
              ...
      Filters/                  Subpackage for filters
              __init__.py
              equalizer.py
              vocoder.py
              karaoke.py
              ...
\end{verbatim}

When importing the package, Python searchs through the directories
on \code{sys.path} looking for the package subdirectory.

The \file{__init__.py} files are required to make Python treat the
directories as containing packages; this is done to prevent
directories with a common name, such as \samp{string}, from
unintentionally hiding valid modules that occur later on the module
search path. In the simplest case, \file{__init__.py} can just be an
empty file, but it can also execute initialization code for the
package or set the \code{__all__} variable, described later.

Users of the package can import individual modules from the
package, for example:

\begin{verbatim}
import Sound.Effects.echo
\end{verbatim}

This loads the submodule \module{Sound.Effects.echo}.  It must be referenced
with its full name.

\begin{verbatim}
Sound.Effects.echo.echofilter(input, output, delay=0.7, atten=4)
\end{verbatim}

An alternative way of importing the submodule is:

\begin{verbatim}
from Sound.Effects import echo
\end{verbatim}

This also loads the submodule \module{echo}, and makes it available without
its package prefix, so it can be used as follows:

\begin{verbatim}
echo.echofilter(input, output, delay=0.7, atten=4)
\end{verbatim}

Yet another variation is to import the desired function or variable directly:

\begin{verbatim}
from Sound.Effects.echo import echofilter
\end{verbatim}

Again, this loads the submodule \module{echo}, but this makes its function
\function{echofilter()} directly available:

\begin{verbatim}
echofilter(input, output, delay=0.7, atten=4)
\end{verbatim}

Note that when using \code{from \var{package} import \var{item}}, the
item can be either a submodule (or subpackage) of the package, or some 
other name defined in the package, like a function, class or
variable.  The \code{import} statement first tests whether the item is
defined in the package; if not, it assumes it is a module and attempts
to load it.  If it fails to find it, an
\exception{ImportError} exception is raised.

Contrarily, when using syntax like \code{import
\var{item.subitem.subsubitem}}, each item except for the last must be
a package; the last item can be a module or a package but can't be a
class or function or variable defined in the previous item.

\subsection{Importing * From a Package \label{pkg-import-star}}
%The \code{__all__} Attribute

Now what happens when the user writes \code{from Sound.Effects import
*}?  Ideally, one would hope that this somehow goes out to the
filesystem, finds which submodules are present in the package, and
imports them all.  Unfortunately, this operation does not work very
well on Mac and Windows platforms, where the filesystem does not
always have accurate information about the case of a filename!  On
these platforms, there is no guaranteed way to know whether a file
\file{ECHO.PY} should be imported as a module \module{echo},
\module{Echo} or \module{ECHO}.  (For example, Windows 95 has the
annoying practice of showing all file names with a capitalized first
letter.)  The DOS 8+3 filename restriction adds another interesting
problem for long module names.

The only solution is for the package author to provide an explicit
index of the package.  The import statement uses the following
convention: if a package's \file{__init__.py} code defines a list
named \code{__all__}, it is taken to be the list of module names that
should be imported when \code{from \var{package} import *} is
encountered.  It is up to the package author to keep this list
up-to-date when a new version of the package is released.  Package
authors may also decide not to support it, if they don't see a use for
importing * from their package.  For example, the file
\file{Sounds/Effects/__init__.py} could contain the following code:

\begin{verbatim}
__all__ = ["echo", "surround", "reverse"]
\end{verbatim}

This would mean that \code{from Sound.Effects import *} would
import the three named submodules of the \module{Sound} package.

If \code{__all__} is not defined, the statement \code{from Sound.Effects
import *} does \emph{not} import all submodules from the package
\module{Sound.Effects} into the current namespace; it only ensures that the
package \module{Sound.Effects} has been imported (possibly running its
initialization code, \file{__init__.py}) and then imports whatever names are
defined in the package.  This includes any names defined (and
submodules explicitly loaded) by \file{__init__.py}.  It also includes any
submodules of the package that were explicitly loaded by previous
import statements.  Consider this code:

\begin{verbatim}
import Sound.Effects.echo
import Sound.Effects.surround
from Sound.Effects import *
\end{verbatim}

In this example, the echo and surround modules are imported in the
current namespace because they are defined in the
\module{Sound.Effects} package when the \code{from...import} statement
is executed.  (This also works when \code{__all__} is defined.)

Note that in general the practice of importing \code{*} from a module or
package is frowned upon, since it often causes poorly readable code.
However, it is okay to use it to save typing in interactive sessions,
and certain modules are designed to export only names that follow
certain patterns.

Remember, there is nothing wrong with using \code{from Package
import specific_submodule}!  In fact, this is the
recommended notation unless the importing module needs to use
submodules with the same name from different packages.


\subsection{Intra-package References}

The submodules often need to refer to each other.  For example, the
\module{surround} module might use the \module{echo} module.  In fact, such references
are so common that the \code{import} statement first looks in the
containing package before looking in the standard module search path.
Thus, the surround module can simply use \code{import echo} or
\code{from echo import echofilter}.  If the imported module is not
found in the current package (the package of which the current module
is a submodule), the \code{import} statement looks for a top-level module
with the given name.

When packages are structured into subpackages (as with the
\module{Sound} package in the example), there's no shortcut to refer
to submodules of sibling packages - the full name of the subpackage
must be used.  For example, if the module
\module{Sound.Filters.vocoder} needs to use the \module{echo} module
in the \module{Sound.Effects} package, it can use \code{from
Sound.Effects import echo}.

%(One could design a notation to refer to parent packages, similar to
%the use of ".." to refer to the parent directory in \UNIX{} and Windows
%filesystems.  In fact, the \module{ni} module, which was the
%ancestor of this package system, supported this using \code{__} for
%the package containing the current module,
%\code{__.__} for the parent package, and so on.  This feature was dropped
%because of its awkwardness; since most packages will have a relative
%shallow substructure, this is no big loss.)

\subsection{Packages in Multiple Directories}

Packages support one more special attribute, \member{__path__}.  This
is initialized to be a list containing the name of the directory
holding the package's \file{__init__.py} before the code in that file
is executed.  This variable can be modified; doing so affects future
searches for modules and subpackages contained in the package.

While this feature is not often needed, it can be used to extend the
set of modules found in a package.



\chapter{Input and Output \label{io}}

There are several ways to present the output of a program; data can be
printed in a human-readable form, or written to a file for future use.
This chapter will discuss some of the possibilities.


\section{Fancier Output Formatting \label{formatting}}

So far we've encountered two ways of writing values: \emph{expression
statements} and the \keyword{print} statement.  (A third way is using
the \method{write()} method of file objects; the standard output file
can be referenced as \code{sys.stdout}.  See the Library Reference for
more information on this.)

Often you'll want more control over the formatting of your output than
simply printing space-separated values.  There are two ways to format
your output; the first way is to do all the string handling yourself;
using string slicing and concatenation operations you can create any
lay-out you can imagine.  The standard module
\module{string}\refstmodindex{string} contains some useful operations
for padding strings to a given column width; these will be discussed
shortly.  The second way is to use the \code{\%} operator with a
string as the left argument.  The \code{\%} operator interprets the
left argument much like a \cfunction{sprintf()}-style format
string to be applied to the right argument, and returns the string
resulting from this formatting operation.

One question remains, of course: how do you convert values to strings?
Luckily, Python has ways to convert any value to a string: pass it to
the \function{repr()}  or \function{str()} functions.  Reverse quotes
(\code{``}) are equivalent to \function{repr()}, but their use is
discouraged.

The \function{str()} function is meant to return representations of
values which are fairly human-readable, while \function{repr()} is
meant to generate representations which can be read by the interpreter
(or will force a \exception{SyntaxError} if there is not equivalent
syntax).  For objects which don't have a particular representation for
human consumption, \function{str()} will return the same value as
\function{repr()}.  Many values, such as numbers or structures like
lists and dictionaries, have the same representation using either
function.  Strings and floating point numbers, in particular, have two
distinct representations.

Some examples:

\begin{verbatim}
>>> s = 'Hello, world.'
>>> str(s)
'Hello, world.'
>>> repr(s)
"'Hello, world.'"
>>> str(0.1)
'0.1'
>>> repr(0.1)
'0.10000000000000001'
>>> x = 10 * 3.25
>>> y = 200 * 200
>>> s = 'The value of x is ' + repr(x) + ', and y is ' + repr(y) + '...'
>>> print s
The value of x is 32.5, and y is 40000...
>>> # The repr() of a string adds string quotes and backslashes:
... hello = 'hello, world\n'
>>> hellos = repr(hello)
>>> print hellos
'hello, world\n'
>>> # The argument to repr() may be any Python object:
... repr((x, y, ('spam', 'eggs')))
"(32.5, 40000, ('spam', 'eggs'))"
>>> # reverse quotes are convenient in interactive sessions:
... `x, y, ('spam', 'eggs')`
"(32.5, 40000, ('spam', 'eggs'))"
\end{verbatim}

Here are two ways to write a table of squares and cubes:

\begin{verbatim}
>>> import string
>>> for x in range(1, 11):
...     print string.rjust(repr(x), 2), string.rjust(repr(x*x), 3),
...     # Note trailing comma on previous line
...     print string.rjust(repr(x*x*x), 4)
...
 1   1    1
 2   4    8
 3   9   27
 4  16   64
 5  25  125
 6  36  216
 7  49  343
 8  64  512
 9  81  729
10 100 1000
>>> for x in range(1,11):
...     print '%2d %3d %4d' % (x, x*x, x*x*x)
... 
 1   1    1
 2   4    8
 3   9   27
 4  16   64
 5  25  125
 6  36  216
 7  49  343
 8  64  512
 9  81  729
10 100 1000
\end{verbatim}

(Note that one space between each column was added by the way
\keyword{print} works: it always adds spaces between its arguments.)

This example demonstrates the function \function{string.rjust()},
which right-justifies a string in a field of a given width by padding
it with spaces on the left.  There are similar functions
\function{string.ljust()} and \function{string.center()}.  These
functions do not write anything, they just return a new string.  If
the input string is too long, they don't truncate it, but return it
unchanged; this will mess up your column lay-out but that's usually
better than the alternative, which would be lying about a value.  (If
you really want truncation you can always add a slice operation, as in
\samp{string.ljust(x,~n)[0:n]}.)

There is another function, \function{string.zfill()}, which pads a
numeric string on the left with zeros.  It understands about plus and
minus signs:

\begin{verbatim}
>>> import string
>>> string.zfill('12', 5)
'00012'
>>> string.zfill('-3.14', 7)
'-003.14'
>>> string.zfill('3.14159265359', 5)
'3.14159265359'
\end{verbatim}

Using the \code{\%} operator looks like this:

\begin{verbatim}
>>> import math
>>> print 'The value of PI is approximately %5.3f.' % math.pi
The value of PI is approximately 3.142.
\end{verbatim}

If there is more than one format in the string, you need to pass a
tuple as right operand, as in this example:

\begin{verbatim}
>>> table = {'Sjoerd': 4127, 'Jack': 4098, 'Dcab': 7678}
>>> for name, phone in table.items():
...     print '%-10s ==> %10d' % (name, phone)
... 
Jack       ==>       4098
Dcab       ==>       7678
Sjoerd     ==>       4127
\end{verbatim}

Most formats work exactly as in C and require that you pass the proper
type; however, if you don't you get an exception, not a core dump.
The \code{\%s} format is more relaxed: if the corresponding argument is
not a string object, it is converted to string using the
\function{str()} built-in function.  Using \code{*} to pass the width
or precision in as a separate (integer) argument is supported.  The
C formats \code{\%n} and \code{\%p} are not supported.

If you have a really long format string that you don't want to split
up, it would be nice if you could reference the variables to be
formatted by name instead of by position.  This can be done by using
form \code{\%(name)format}, as shown here:

\begin{verbatim}
>>> table = {'Sjoerd': 4127, 'Jack': 4098, 'Dcab': 8637678}
>>> print 'Jack: %(Jack)d; Sjoerd: %(Sjoerd)d; Dcab: %(Dcab)d' % table
Jack: 4098; Sjoerd: 4127; Dcab: 8637678
\end{verbatim}

This is particularly useful in combination with the new built-in
\function{vars()} function, which returns a dictionary containing all
local variables.

\section{Reading and Writing Files \label{files}}

% Opening files 
\function{open()}\bifuncindex{open} returns a file
object\obindex{file}, and is most commonly used with two arguments:
\samp{open(\var{filename}, \var{mode})}.

\begin{verbatim}
>>> f=open('/tmp/workfile', 'w')
>>> print f
<open file '/tmp/workfile', mode 'w' at 80a0960>
\end{verbatim}

The first argument is a string containing the filename.  The second
argument is another string containing a few characters describing the
way in which the file will be used.  \var{mode} can be \code{'r'} when
the file will only be read, \code{'w'} for only writing (an existing
file with the same name will be erased), and \code{'a'} opens the file
for appending; any data written to the file is automatically added to
the end.  \code{'r+'} opens the file for both reading and writing.
The \var{mode} argument is optional; \code{'r'} will be assumed if
it's omitted.

On Windows and the Macintosh, \code{'b'} appended to the
mode opens the file in binary mode, so there are also modes like
\code{'rb'}, \code{'wb'}, and \code{'r+b'}.  Windows makes a
distinction between text and binary files; the end-of-line characters
in text files are automatically altered slightly when data is read or
written.  This behind-the-scenes modification to file data is fine for
\ASCII{} text files, but it'll corrupt binary data like that in JPEGs or
\file{.EXE} files.  Be very careful to use binary mode when reading and
writing such files.  (Note that the precise semantics of text mode on
the Macintosh depends on the underlying C library being used.)

\subsection{Methods of File Objects \label{fileMethods}}

The rest of the examples in this section will assume that a file
object called \code{f} has already been created.

To read a file's contents, call \code{f.read(\var{size})}, which reads
some quantity of data and returns it as a string.  \var{size} is an
optional numeric argument.  When \var{size} is omitted or negative,
the entire contents of the file will be read and returned; it's your
problem if the file is twice as large as your machine's memory.
Otherwise, at most \var{size} bytes are read and returned.  If the end
of the file has been reached, \code{f.read()} will return an empty
string (\code {""}).
\begin{verbatim}
>>> f.read()
'This is the entire file.\n'
>>> f.read()
''
\end{verbatim}

\code{f.readline()} reads a single line from the file; a newline
character (\code{\e n}) is left at the end of the string, and is only
omitted on the last line of the file if the file doesn't end in a
newline.  This makes the return value unambiguous; if
\code{f.readline()} returns an empty string, the end of the file has
been reached, while a blank line is represented by \code{'\e n'}, a
string containing only a single newline.  

\begin{verbatim}
>>> f.readline()
'This is the first line of the file.\n'
>>> f.readline()
'Second line of the file\n'
>>> f.readline()
''
\end{verbatim}

\code{f.readlines()} returns a list containing all the lines of data
in the file.  If given an optional parameter \var{sizehint}, it reads
that many bytes from the file and enough more to complete a line, and
returns the lines from that.  This is often used to allow efficient
reading of a large file by lines, but without having to load the
entire file in memory.  Only complete lines will be returned.

\begin{verbatim}
>>> f.readlines()
['This is the first line of the file.\n', 'Second line of the file\n']
\end{verbatim}

\code{f.write(\var{string})} writes the contents of \var{string} to
the file, returning \code{None}.  

\begin{verbatim}
>>> f.write('This is a test\n')
\end{verbatim}

\code{f.tell()} returns an integer giving the file object's current
position in the file, measured in bytes from the beginning of the
file.  To change the file object's position, use
\samp{f.seek(\var{offset}, \var{from_what})}.  The position is
computed from adding \var{offset} to a reference point; the reference
point is selected by the \var{from_what} argument.  A
\var{from_what} value of 0 measures from the beginning of the file, 1
uses the current file position, and 2 uses the end of the file as the
reference point.  \var{from_what} can be omitted and defaults to 0,
using the beginning of the file as the reference point.

\begin{verbatim}
>>> f=open('/tmp/workfile', 'r+')
>>> f.write('0123456789abcdef')
>>> f.seek(5)     # Go to the 6th byte in the file
>>> f.read(1)        
'5'
>>> f.seek(-3, 2) # Go to the 3rd byte before the end
>>> f.read(1)
'd'
\end{verbatim}

When you're done with a file, call \code{f.close()} to close it and
free up any system resources taken up by the open file.  After calling
\code{f.close()}, attempts to use the file object will automatically fail.

\begin{verbatim}
>>> f.close()
>>> f.read()
Traceback (most recent call last):
  File "<stdin>", line 1, in ?
ValueError: I/O operation on closed file
\end{verbatim}

File objects have some additional methods, such as
\method{isatty()} and \method{truncate()} which are less frequently
used; consult the Library Reference for a complete guide to file
objects.

\subsection{The \module{pickle} Module \label{pickle}}
\refstmodindex{pickle}

Strings can easily be written to and read from a file. Numbers take a
bit more effort, since the \method{read()} method only returns
strings, which will have to be passed to a function like
\function{string.atoi()}, which takes a string like \code{'123'} and
returns its numeric value 123.  However, when you want to save more
complex data types like lists, dictionaries, or class instances,
things get a lot more complicated.

Rather than have users be constantly writing and debugging code to
save complicated data types, Python provides a standard module called
\module{pickle}.  This is an amazing module that can take almost
any Python object (even some forms of Python code!), and convert it to
a string representation; this process is called \dfn{pickling}.  
Reconstructing the object from the string representation is called
\dfn{unpickling}.  Between pickling and unpickling, the string
representing the object may have been stored in a file or data, or
sent over a network connection to some distant machine.

If you have an object \code{x}, and a file object \code{f} that's been
opened for writing, the simplest way to pickle the object takes only
one line of code:

\begin{verbatim}
pickle.dump(x, f)
\end{verbatim}

To unpickle the object again, if \code{f} is a file object which has
been opened for reading:

\begin{verbatim}
x = pickle.load(f)
\end{verbatim}

(There are other variants of this, used when pickling many objects or
when you don't want to write the pickled data to a file; consult the
complete documentation for \module{pickle} in the Library Reference.)

\module{pickle} is the standard way to make Python objects which can
be stored and reused by other programs or by a future invocation of
the same program; the technical term for this is a
\dfn{persistent} object.  Because \module{pickle} is so widely used,
many authors who write Python extensions take care to ensure that new
data types such as matrices can be properly pickled and unpickled.



\chapter{Errors and Exceptions \label{errors}}

Until now error messages haven't been more than mentioned, but if you
have tried out the examples you have probably seen some.  There are
(at least) two distinguishable kinds of errors:
\emph{syntax errors} and \emph{exceptions}.

\section{Syntax Errors \label{syntaxErrors}}

Syntax errors, also known as parsing errors, are perhaps the most common
kind of complaint you get while you are still learning Python:

\begin{verbatim}
>>> while True print 'Hello world'
  File "<stdin>", line 1, in ?
    while True print 'Hello world'
                   ^
SyntaxError: invalid syntax
\end{verbatim}

The parser repeats the offending line and displays a little `arrow'
pointing at the earliest point in the line where the error was
detected.  The error is caused by (or at least detected at) the token
\emph{preceding} the arrow: in the example, the error is detected at
the keyword \keyword{print}, since a colon (\character{:}) is missing
before it.  File name and line number are printed so you know where to
look in case the input came from a script.

\section{Exceptions \label{exceptions}}

Even if a statement or expression is syntactically correct, it may
cause an error when an attempt is made to execute it.
Errors detected during execution are called \emph{exceptions} and are
not unconditionally fatal: you will soon learn how to handle them in
Python programs.  Most exceptions are not handled by programs,
however, and result in error messages as shown here:

\begin{verbatim}
>>> 10 * (1/0)
Traceback (most recent call last):
  File "<stdin>", line 1, in ?
ZeroDivisionError: integer division or modulo by zero
>>> 4 + spam*3
Traceback (most recent call last):
  File "<stdin>", line 1, in ?
NameError: name 'spam' is not defined
>>> '2' + 2
Traceback (most recent call last):
  File "<stdin>", line 1, in ?
TypeError: cannot concatenate 'str' and 'int' objects
\end{verbatim}

The last line of the error message indicates what happened.
Exceptions come in different types, and the type is printed as part of
the message: the types in the example are
\exception{ZeroDivisionError}, \exception{NameError} and
\exception{TypeError}.
The string printed as the exception type is the name of the built-in
name for the exception that occurred.  This is true for all built-in
exceptions, but need not be true for user-defined exceptions (although
it is a useful convention).
Standard exception names are built-in identifiers (not reserved
keywords).

The rest of the line is a detail whose interpretation depends on the
exception type; its meaning is dependent on the exception type.

The preceding part of the error message shows the context where the
exception happened, in the form of a stack backtrace.
In general it contains a stack backtrace listing source lines; however,
it will not display lines read from standard input.

The \citetitle[../lib/module-exceptions.html]{Python Library
Reference} lists the built-in exceptions and their meanings.


\section{Handling Exceptions \label{handling}}

It is possible to write programs that handle selected exceptions.
Look at the following example, which asks the user for input until a
valid integer has been entered, but allows the user to interrupt the
program (using \kbd{Control-C} or whatever the operating system
supports); note that a user-generated interruption is signalled by
raising the \exception{KeyboardInterrupt} exception.

\begin{verbatim}
>>> while True:
...     try:
...         x = int(raw_input("Please enter a number: "))
...         break
...     except ValueError:
...         print "Oops! That was no valid number.  Try again..."
...     
\end{verbatim}

The \keyword{try} statement works as follows.

\begin{itemize}
\item
First, the \emph{try clause} (the statement(s) between the
\keyword{try} and \keyword{except} keywords) is executed.

\item
If no exception occurs, the \emph{except\ clause} is skipped and
execution of the \keyword{try} statement is finished.

\item
If an exception occurs during execution of the try clause, the rest of
the clause is skipped.  Then if its type matches the exception named
after the \keyword{except} keyword, the rest of the try clause is
skipped, the except clause is executed, and then execution continues
after the \keyword{try} statement.

\item
If an exception occurs which does not match the exception named in the
except clause, it is passed on to outer \keyword{try} statements; if
no handler is found, it is an \emph{unhandled exception} and execution
stops with a message as shown above.

\end{itemize}

A \keyword{try} statement may have more than one except clause, to
specify handlers for different exceptions.  At most one handler will
be executed.  Handlers only handle exceptions that occur in the
corresponding try clause, not in other handlers of the same
\keyword{try} statement.  An except clause may name multiple exceptions
as a parenthesized list, for example:

\begin{verbatim}
... except (RuntimeError, TypeError, NameError):
...     pass
\end{verbatim}

The last except clause may omit the exception name(s), to serve as a
wildcard.  Use this with extreme caution, since it is easy to mask a
real programming error in this way!  It can also be used to print an
error message and then re-raise the exception (allowing a caller to
handle the exception as well):

\begin{verbatim}
import string, sys

try:
    f = open('myfile.txt')
    s = f.readline()
    i = int(string.strip(s))
except IOError, (errno, strerror):
    print "I/O error(%s): %s" % (errno, strerror)
except ValueError:
    print "Could not convert data to an integer."
except:
    print "Unexpected error:", sys.exc_info()[0]
    raise
\end{verbatim}

The \keyword{try} \ldots\ \keyword{except} statement has an optional
\emph{else clause}, which, when present, must follow all except
clauses.  It is useful for code that must be executed if the try
clause does not raise an exception.  For example:

\begin{verbatim}
for arg in sys.argv[1:]:
    try:
        f = open(arg, 'r')
    except IOError:
        print 'cannot open', arg
    else:
        print arg, 'has', len(f.readlines()), 'lines'
        f.close()
\end{verbatim}

The use of the \keyword{else} clause is better than adding additional
code to the \keyword{try} clause because it avoids accidentally
catching an exception that wasn't raised by the code being protected
by the \keyword{try} \ldots\ \keyword{except} statement.


When an exception occurs, it may have an associated value, also known as
the exception's \emph{argument}.
The presence and type of the argument depend on the exception type.
For exception types which have an argument, the except clause may
specify a variable after the exception name (or list) to receive the
argument's value, as follows:

\begin{verbatim}
>>> try:
...     spam()
... except NameError, x:
...     print 'name', x, 'undefined'
... 
name spam undefined
\end{verbatim}

If an exception has an argument, it is printed as the last part
(`detail') of the message for unhandled exceptions.

Exception handlers don't just handle exceptions if they occur
immediately in the try clause, but also if they occur inside functions
that are called (even indirectly) in the try clause.
For example:

\begin{verbatim}
>>> def this_fails():
...     x = 1/0
... 
>>> try:
...     this_fails()
... except ZeroDivisionError, detail:
...     print 'Handling run-time error:', detail
... 
Handling run-time error: integer division or modulo
\end{verbatim}


\section{Raising Exceptions \label{raising}}

The \keyword{raise} statement allows the programmer to force a
specified exception to occur.
For example:

\begin{verbatim}
>>> raise NameError, 'HiThere'
Traceback (most recent call last):
  File "<stdin>", line 1, in ?
NameError: HiThere
\end{verbatim}

The first argument to \keyword{raise} names the exception to be
raised.  The optional second argument specifies the exception's
argument.

If you need to determine whether an exception was raised but don't
intend to handle it, a simpler form of the \keyword{raise} statement
allows you to re-raise the exception:

\begin{verbatim}
>>> try:
...     raise NameError, 'HiThere'
... except NameError:
...     print 'An exception flew by!'
...     raise
...
An exception flew by!
Traceback (most recent call last):
  File "<stdin>", line 2, in ?
NameError: HiThere
\end{verbatim}


\section{User-defined Exceptions \label{userExceptions}}

Programs may name their own exceptions by creating a new exception
class.  Exceptions should typically be derived from the
\exception{Exception} class, either directly or indirectly.  For
example:

\begin{verbatim}
>>> class MyError(Exception):
...     def __init__(self, value):
...         self.value = value
...     def __str__(self):
...         return repr(self.value)
... 
>>> try:
...     raise MyError(2*2)
... except MyError, e:
...     print 'My exception occurred, value:', e.value
... 
My exception occurred, value: 4
>>> raise MyError, 'oops!'
Traceback (most recent call last):
  File "<stdin>", line 1, in ?
__main__.MyError: 'oops!'
\end{verbatim}

Exception classes can be defined which do anything any other class can
do, but are usually kept simple, often only offering a number of
attributes that allow information about the error to be extracted by
handlers for the exception.  When creating a module which can raise
several distinct errors, a common practice is to create a base class
for exceptions defined by that module, and subclass that to create
specific exception classes for different error conditions:

\begin{verbatim}
class Error(Exception):
    """Base class for exceptions in this module."""
    pass

class InputError(Error):
    """Exception raised for errors in the input.

    Attributes:
        expression -- input expression in which the error occurred
        message -- explanation of the error
    """

    def __init__(self, expression, message):
        self.expression = expression
        self.message = message

class TransitionError(Error):
    """Raised when an operation attempts a state transition that's not
    allowed.

    Attributes:
        previous -- state at beginning of transition
        next -- attempted new state
        message -- explanation of why the specific transition is not allowed
    """

    def __init__(self, previous, next, message):
        self.previous = previous
        self.next = next
        self.message = message
\end{verbatim}

Most exceptions are defined with names that end in ``Error,'' similar
to the naming of the standard exceptions.

Many standard modules define their own exceptions to report errors
that may occur in functions they define.  More information on classes
is presented in chapter \ref{classes}, ``Classes.''


\section{Defining Clean-up Actions \label{cleanup}}

The \keyword{try} statement has another optional clause which is
intended to define clean-up actions that must be executed under all
circumstances.  For example:

\begin{verbatim}
>>> try:
...     raise KeyboardInterrupt
... finally:
...     print 'Goodbye, world!'
... 
Goodbye, world!
Traceback (most recent call last):
  File "<stdin>", line 2, in ?
KeyboardInterrupt
\end{verbatim}

A \emph{finally clause} is executed whether or not an exception has
occurred in the try clause.  When an exception has occurred, it is
re-raised after the finally clause is executed.  The finally clause is
also executed ``on the way out'' when the \keyword{try} statement is
left via a \keyword{break} or \keyword{return} statement.

The code in the finally clause is useful for releasing external
resources (such as files or network connections), regardless of
whether or not the use of the resource was successful.

A \keyword{try} statement must either have one or more except clauses
or one finally clause, but not both.


\chapter{Classes \label{classes}}

Python's class mechanism adds classes to the language with a minimum
of new syntax and semantics.  It is a mixture of the class mechanisms
found in \Cpp{} and Modula-3.  As is true for modules, classes in Python
do not put an absolute barrier between definition and user, but rather
rely on the politeness of the user not to ``break into the
definition.''  The most important features of classes are retained
with full power, however: the class inheritance mechanism allows
multiple base classes, a derived class can override any methods of its
base class or classes, a method can call the method of a base class with the
same name.  Objects can contain an arbitrary amount of private data.

In \Cpp{} terminology, all class members (including the data members) are
\emph{public}, and all member functions are \emph{virtual}.  There are
no special constructors or destructors.  As in Modula-3, there are no
shorthands for referencing the object's members from its methods: the
method function is declared with an explicit first argument
representing the object, which is provided implicitly by the call.  As
in Smalltalk, classes themselves are objects, albeit in the wider
sense of the word: in Python, all data types are objects.  This
provides semantics for importing and renaming.  But, just like in
\Cpp{} or Modula-3, built-in types cannot be used as base classes for
extension by the user.  Also, like in \Cpp{} but unlike in Modula-3, most
built-in operators with special syntax (arithmetic operators,
subscripting etc.) can be redefined for class instances.

\section{A Word About Terminology \label{terminology}}

Lacking universally accepted terminology to talk about classes, I will
make occasional use of Smalltalk and \Cpp{} terms.  (I would use Modula-3
terms, since its object-oriented semantics are closer to those of
Python than \Cpp, but I expect that few readers have heard of it.)

I also have to warn you that there's a terminological pitfall for
object-oriented readers: the word ``object'' in Python does not
necessarily mean a class instance.  Like \Cpp{} and Modula-3, and
unlike Smalltalk, not all types in Python are classes: the basic
built-in types like integers and lists are not, and even somewhat more
exotic types like files aren't.  However, \emph{all} Python types
share a little bit of common semantics that is best described by using
the word object.

Objects have individuality, and multiple names (in multiple scopes)
can be bound to the same object.  This is known as aliasing in other
languages.  This is usually not appreciated on a first glance at
Python, and can be safely ignored when dealing with immutable basic
types (numbers, strings, tuples).  However, aliasing has an
(intended!) effect on the semantics of Python code involving mutable
objects such as lists, dictionaries, and most types representing
entities outside the program (files, windows, etc.).  This is usually
used to the benefit of the program, since aliases behave like pointers
in some respects.  For example, passing an object is cheap since only
a pointer is passed by the implementation; and if a function modifies
an object passed as an argument, the caller will see the change --- this
eliminates the need for two different argument passing mechanisms as in
Pascal.


\section{Python Scopes and Name Spaces \label{scopes}}

Before introducing classes, I first have to tell you something about
Python's scope rules.  Class definitions play some neat tricks with
namespaces, and you need to know how scopes and namespaces work to
fully understand what's going on.  Incidentally, knowledge about this
subject is useful for any advanced Python programmer.

Let's begin with some definitions.

A \emph{namespace} is a mapping from names to objects.  Most
namespaces are currently implemented as Python dictionaries, but
that's normally not noticeable in any way (except for performance),
and it may change in the future.  Examples of namespaces are: the set
of built-in names (functions such as \function{abs()}, and built-in
exception names); the global names in a module; and the local names in
a function invocation.  In a sense the set of attributes of an object
also form a namespace.  The important thing to know about namespaces
is that there is absolutely no relation between names in different
namespaces; for instance, two different modules may both define a
function ``maximize'' without confusion --- users of the modules must
prefix it with the module name.

By the way, I use the word \emph{attribute} for any name following a
dot --- for example, in the expression \code{z.real}, \code{real} is
an attribute of the object \code{z}.  Strictly speaking, references to
names in modules are attribute references: in the expression
\code{modname.funcname}, \code{modname} is a module object and
\code{funcname} is an attribute of it.  In this case there happens to
be a straightforward mapping between the module's attributes and the
global names defined in the module: they share the same namespace!
\footnote{
        Except for one thing.  Module objects have a secret read-only
        attribute called \member{__dict__} which returns the dictionary
        used to implement the module's namespace; the name
        \member{__dict__} is an attribute but not a global name.
        Obviously, using this violates the abstraction of namespace
        implementation, and should be restricted to things like
        post-mortem debuggers.
}

Attributes may be read-only or writable.  In the latter case,
assignment to attributes is possible.  Module attributes are writable:
you can write \samp{modname.the_answer = 42}.  Writable attributes may
also be deleted with the \keyword{del} statement.  For example,
\samp{del modname.the_answer} will remove the attribute
\member{the_answer} from the object named by \code{modname}.

Name spaces are created at different moments and have different
lifetimes.  The namespace containing the built-in names is created
when the Python interpreter starts up, and is never deleted.  The
global namespace for a module is created when the module definition
is read in; normally, module namespaces also last until the
interpreter quits.  The statements executed by the top-level
invocation of the interpreter, either read from a script file or
interactively, are considered part of a module called
\module{__main__}, so they have their own global namespace.  (The
built-in names actually also live in a module; this is called
\module{__builtin__}.)

The local namespace for a function is created when the function is
called, and deleted when the function returns or raises an exception
that is not handled within the function.  (Actually, forgetting would
be a better way to describe what actually happens.)  Of course,
recursive invocations each have their own local namespace.

A \emph{scope} is a textual region of a Python program where a
namespace is directly accessible.  ``Directly accessible'' here means
that an unqualified reference to a name attempts to find the name in
the namespace.

Although scopes are determined statically, they are used dynamically.
At any time during execution, there are at least three nested scopes whose
namespaces are directly accessible: the innermost scope, which is searched
first, contains the local names; the namespaces of any enclosing
functions, which are searched starting with the nearest enclosing scope;
the middle scope, searched next, contains the current module's global names;
and the outermost scope (searched last) is the namespace containing built-in
names.

If a name is declared global, then all references and assignments go
directly to the middle scope containing the module's global names.
Otherwise, all variables found outside of the innermost scope are read-only.

Usually, the local scope references the local names of the (textually)
current function.  Outside of functions, the local scope references
the same namespace as the global scope: the module's namespace.
Class definitions place yet another namespace in the local scope.

It is important to realize that scopes are determined textually: the
global scope of a function defined in a module is that module's
namespace, no matter from where or by what alias the function is
called.  On the other hand, the actual search for names is done
dynamically, at run time --- however, the language definition is
evolving towards static name resolution, at ``compile'' time, so don't
rely on dynamic name resolution!  (In fact, local variables are
already determined statically.)

A special quirk of Python is that assignments always go into the
innermost scope.  Assignments do not copy data --- they just
bind names to objects.  The same is true for deletions: the statement
\samp{del x} removes the binding of \code{x} from the namespace
referenced by the local scope.  In fact, all operations that introduce
new names use the local scope: in particular, import statements and
function definitions bind the module or function name in the local
scope.  (The \keyword{global} statement can be used to indicate that
particular variables live in the global scope.)


\section{A First Look at Classes \label{firstClasses}}

Classes introduce a little bit of new syntax, three new object types,
and some new semantics.


\subsection{Class Definition Syntax \label{classDefinition}}

The simplest form of class definition looks like this:

\begin{verbatim}
class ClassName:
    <statement-1>
    .
    .
    .
    <statement-N>
\end{verbatim}

Class definitions, like function definitions
(\keyword{def} statements) must be executed before they have any
effect.  (You could conceivably place a class definition in a branch
of an \keyword{if} statement, or inside a function.)

In practice, the statements inside a class definition will usually be
function definitions, but other statements are allowed, and sometimes
useful --- we'll come back to this later.  The function definitions
inside a class normally have a peculiar form of argument list,
dictated by the calling conventions for methods --- again, this is
explained later.

When a class definition is entered, a new namespace is created, and
used as the local scope --- thus, all assignments to local variables
go into this new namespace.  In particular, function definitions bind
the name of the new function here.

When a class definition is left normally (via the end), a \emph{class
object} is created.  This is basically a wrapper around the contents
of the namespace created by the class definition; we'll learn more
about class objects in the next section.  The original local scope
(the one in effect just before the class definitions was entered) is
reinstated, and the class object is bound here to the class name given
in the class definition header (\class{ClassName} in the example).


\subsection{Class Objects \label{classObjects}}

Class objects support two kinds of operations: attribute references
and instantiation.

\emph{Attribute references} use the standard syntax used for all
attribute references in Python: \code{obj.name}.  Valid attribute
names are all the names that were in the class's namespace when the
class object was created.  So, if the class definition looked like
this:

\begin{verbatim}
class MyClass:
    "A simple example class"
    i = 12345
    def f(self):
        return 'hello world'
\end{verbatim}

then \code{MyClass.i} and \code{MyClass.f} are valid attribute
references, returning an integer and a method object, respectively.
Class attributes can also be assigned to, so you can change the value
of \code{MyClass.i} by assignment.  \member{__doc__} is also a valid
attribute, returning the docstring belonging to the class: \code{"A
simple example class"}. 

Class \emph{instantiation} uses function notation.  Just pretend that
the class object is a parameterless function that returns a new
instance of the class.  For example (assuming the above class):

\begin{verbatim}
x = MyClass()
\end{verbatim}

creates a new \emph{instance} of the class and assigns this object to
the local variable \code{x}.

The instantiation operation (``calling'' a class object) creates an
empty object.  Many classes like to create objects in a known initial
state.  Therefore a class may define a special method named
\method{__init__()}, like this:

\begin{verbatim}
    def __init__(self):
        self.data = []
\end{verbatim}

When a class defines an \method{__init__()} method, class
instantiation automatically invokes \method{__init__()} for the
newly-created class instance.  So in this example, a new, initialized
instance can be obtained by:

\begin{verbatim}
x = MyClass()
\end{verbatim}

Of course, the \method{__init__()} method may have arguments for
greater flexibility.  In that case, arguments given to the class
instantiation operator are passed on to \method{__init__()}.  For
example,

\begin{verbatim}
>>> class Complex:
...     def __init__(self, realpart, imagpart):
...         self.r = realpart
...         self.i = imagpart
... 
>>> x = Complex(3.0, -4.5)
>>> x.r, x.i
(3.0, -4.5)
\end{verbatim}


\subsection{Instance Objects \label{instanceObjects}}

Now what can we do with instance objects?  The only operations
understood by instance objects are attribute references.  There are
two kinds of valid attribute names.

The first I'll call \emph{data attributes}.  These correspond to
``instance variables'' in Smalltalk, and to ``data members'' in
\Cpp.  Data attributes need not be declared; like local variables,
they spring into existence when they are first assigned to.  For
example, if \code{x} is the instance of \class{MyClass} created above,
the following piece of code will print the value \code{16}, without
leaving a trace:

\begin{verbatim}
x.counter = 1
while x.counter < 10:
    x.counter = x.counter * 2
print x.counter
del x.counter
\end{verbatim}

The second kind of attribute references understood by instance objects
are \emph{methods}.  A method is a function that ``belongs to'' an
object.  (In Python, the term method is not unique to class instances:
other object types can have methods as well.  For example, list objects have
methods called append, insert, remove, sort, and so on.  However,
below, we'll use the term method exclusively to mean methods of class
instance objects, unless explicitly stated otherwise.)

Valid method names of an instance object depend on its class.  By
definition, all attributes of a class that are (user-defined) function 
objects define corresponding methods of its instances.  So in our
example, \code{x.f} is a valid method reference, since
\code{MyClass.f} is a function, but \code{x.i} is not, since
\code{MyClass.i} is not.  But \code{x.f} is not the same thing as
\code{MyClass.f} --- it is a \obindex{method}\emph{method object}, not
a function object.


\subsection{Method Objects \label{methodObjects}}

Usually, a method is called immediately:

\begin{verbatim}
x.f()
\end{verbatim}

In our example, this will return the string \code{'hello world'}.
However, it is not necessary to call a method right away:
\code{x.f} is a method object, and can be stored away and called at a
later time.  For example:

\begin{verbatim}
xf = x.f
while True:
    print xf()
\end{verbatim}

will continue to print \samp{hello world} until the end of time.

What exactly happens when a method is called?  You may have noticed
that \code{x.f()} was called without an argument above, even though
the function definition for \method{f} specified an argument.  What
happened to the argument?  Surely Python raises an exception when a
function that requires an argument is called without any --- even if
the argument isn't actually used...

Actually, you may have guessed the answer: the special thing about
methods is that the object is passed as the first argument of the
function.  In our example, the call \code{x.f()} is exactly equivalent
to \code{MyClass.f(x)}.  In general, calling a method with a list of
\var{n} arguments is equivalent to calling the corresponding function
with an argument list that is created by inserting the method's object
before the first argument.

If you still don't understand how methods work, a look at the
implementation can perhaps clarify matters.  When an instance
attribute is referenced that isn't a data attribute, its class is
searched.  If the name denotes a valid class attribute that is a
function object, a method object is created by packing (pointers to)
the instance object and the function object just found together in an
abstract object: this is the method object.  When the method object is
called with an argument list, it is unpacked again, a new argument
list is constructed from the instance object and the original argument
list, and the function object is called with this new argument list.


\section{Random Remarks \label{remarks}}

[These should perhaps be placed more carefully...]


Data attributes override method attributes with the same name; to
avoid accidental name conflicts, which may cause hard-to-find bugs in
large programs, it is wise to use some kind of convention that
minimizes the chance of conflicts.  Possible conventions include
capitalizing method names, prefixing data attribute names with a small
unique string (perhaps just an underscore), or using verbs for methods
and nouns for data attributes.


Data attributes may be referenced by methods as well as by ordinary
users (``clients'') of an object.  In other words, classes are not
usable to implement pure abstract data types.  In fact, nothing in
Python makes it possible to enforce data hiding --- it is all based
upon convention.  (On the other hand, the Python implementation,
written in C, can completely hide implementation details and control
access to an object if necessary; this can be used by extensions to
Python written in C.)


Clients should use data attributes with care --- clients may mess up
invariants maintained by the methods by stamping on their data
attributes.  Note that clients may add data attributes of their own to
an instance object without affecting the validity of the methods, as
long as name conflicts are avoided --- again, a naming convention can
save a lot of headaches here.


There is no shorthand for referencing data attributes (or other
methods!) from within methods.  I find that this actually increases
the readability of methods: there is no chance of confusing local
variables and instance variables when glancing through a method.


Conventionally, the first argument of methods is often called
\code{self}.  This is nothing more than a convention: the name
\code{self} has absolutely no special meaning to Python.  (Note,
however, that by not following the convention your code may be less
readable by other Python programmers, and it is also conceivable that
a \emph{class browser} program be written which relies upon such a
convention.)


Any function object that is a class attribute defines a method for
instances of that class.  It is not necessary that the function
definition is textually enclosed in the class definition: assigning a
function object to a local variable in the class is also ok.  For
example:

\begin{verbatim}
# Function defined outside the class
def f1(self, x, y):
    return min(x, x+y)

class C:
    f = f1
    def g(self):
        return 'hello world'
    h = g
\end{verbatim}

Now \code{f}, \code{g} and \code{h} are all attributes of class
\class{C} that refer to function objects, and consequently they are all
methods of instances of \class{C} --- \code{h} being exactly equivalent
to \code{g}.  Note that this practice usually only serves to confuse
the reader of a program.


Methods may call other methods by using method attributes of the
\code{self} argument:

\begin{verbatim}
class Bag:
    def __init__(self):
        self.data = []
    def add(self, x):
        self.data.append(x)
    def addtwice(self, x):
        self.add(x)
        self.add(x)
\end{verbatim}

Methods may reference global names in the same way as ordinary
functions.  The global scope associated with a method is the module
containing the class definition.  (The class itself is never used as a
global scope!)  While one rarely encounters a good reason for using
global data in a method, there are many legitimate uses of the global
scope: for one thing, functions and modules imported into the global
scope can be used by methods, as well as functions and classes defined
in it.  Usually, the class containing the method is itself defined in
this global scope, and in the next section we'll find some good
reasons why a method would want to reference its own class!


\section{Inheritance \label{inheritance}}

Of course, a language feature would not be worthy of the name ``class''
without supporting inheritance.  The syntax for a derived class
definition looks as follows:

\begin{verbatim}
class DerivedClassName(BaseClassName):
    <statement-1>
    .
    .
    .
    <statement-N>
\end{verbatim}

The name \class{BaseClassName} must be defined in a scope containing
the derived class definition.  Instead of a base class name, an
expression is also allowed.  This is useful when the base class is
defined in another module,

\begin{verbatim}
class DerivedClassName(modname.BaseClassName):
\end{verbatim}

Execution of a derived class definition proceeds the same as for a
base class.  When the class object is constructed, the base class is
remembered.  This is used for resolving attribute references: if a
requested attribute is not found in the class, it is searched in the
base class.  This rule is applied recursively if the base class itself
is derived from some other class.

There's nothing special about instantiation of derived classes:
\code{DerivedClassName()} creates a new instance of the class.  Method
references are resolved as follows: the corresponding class attribute
is searched, descending down the chain of base classes if necessary,
and the method reference is valid if this yields a function object.

Derived classes may override methods of their base classes.  Because
methods have no special privileges when calling other methods of the
same object, a method of a base class that calls another method
defined in the same base class, may in fact end up calling a method of
a derived class that overrides it.  (For \Cpp{} programmers: all methods
in Python are effectively \keyword{virtual}.)

An overriding method in a derived class may in fact want to extend
rather than simply replace the base class method of the same name.
There is a simple way to call the base class method directly: just
call \samp{BaseClassName.methodname(self, arguments)}.  This is
occasionally useful to clients as well.  (Note that this only works if
the base class is defined or imported directly in the global scope.)


\subsection{Multiple Inheritance \label{multiple}}

Python supports a limited form of multiple inheritance as well.  A
class definition with multiple base classes looks as follows:

\begin{verbatim}
class DerivedClassName(Base1, Base2, Base3):
    <statement-1>
    .
    .
    .
    <statement-N>
\end{verbatim}

The only rule necessary to explain the semantics is the resolution
rule used for class attribute references.  This is depth-first,
left-to-right.  Thus, if an attribute is not found in
\class{DerivedClassName}, it is searched in \class{Base1}, then
(recursively) in the base classes of \class{Base1}, and only if it is
not found there, it is searched in \class{Base2}, and so on.

(To some people breadth first --- searching \class{Base2} and
\class{Base3} before the base classes of \class{Base1} --- looks more
natural.  However, this would require you to know whether a particular
attribute of \class{Base1} is actually defined in \class{Base1} or in
one of its base classes before you can figure out the consequences of
a name conflict with an attribute of \class{Base2}.  The depth-first
rule makes no differences between direct and inherited attributes of
\class{Base1}.)

It is clear that indiscriminate use of multiple inheritance is a
maintenance nightmare, given the reliance in Python on conventions to
avoid accidental name conflicts.  A well-known problem with multiple
inheritance is a class derived from two classes that happen to have a
common base class.  While it is easy enough to figure out what happens
in this case (the instance will have a single copy of ``instance
variables'' or data attributes used by the common base class), it is
not clear that these semantics are in any way useful.


\section{Private Variables \label{private}}

There is limited support for class-private
identifiers.  Any identifier of the form \code{__spam} (at least two
leading underscores, at most one trailing underscore) is now textually
replaced with \code{_classname__spam}, where \code{classname} is the
current class name with leading underscore(s) stripped.  This mangling
is done without regard of the syntactic position of the identifier, so
it can be used to define class-private instance and class variables,
methods, as well as globals, and even to store instance variables
private to this class on instances of \emph{other} classes.  Truncation
may occur when the mangled name would be longer than 255 characters.
Outside classes, or when the class name consists of only underscores,
no mangling occurs.

Name mangling is intended to give classes an easy way to define
``private'' instance variables and methods, without having to worry
about instance variables defined by derived classes, or mucking with
instance variables by code outside the class.  Note that the mangling
rules are designed mostly to avoid accidents; it still is possible for
a determined soul to access or modify a variable that is considered
private.  This can even be useful in special circumstances, such as in
the debugger, and that's one reason why this loophole is not closed.
(Buglet: derivation of a class with the same name as the base class
makes use of private variables of the base class possible.)

Notice that code passed to \code{exec}, \code{eval()} or
\code{evalfile()} does not consider the classname of the invoking 
class to be the current class; this is similar to the effect of the 
\code{global} statement, the effect of which is likewise restricted to 
code that is byte-compiled together.  The same restriction applies to
\code{getattr()}, \code{setattr()} and \code{delattr()}, as well as
when referencing \code{__dict__} directly.


\section{Odds and Ends \label{odds}}

Sometimes it is useful to have a data type similar to the Pascal
``record'' or C ``struct'', bundling together a couple of named data
items.  An empty class definition will do nicely:

\begin{verbatim}
class Employee:
    pass

john = Employee() # Create an empty employee record

# Fill the fields of the record
john.name = 'John Doe'
john.dept = 'computer lab'
john.salary = 1000
\end{verbatim}

A piece of Python code that expects a particular abstract data type
can often be passed a class that emulates the methods of that data
type instead.  For instance, if you have a function that formats some
data from a file object, you can define a class with methods
\method{read()} and \method{readline()} that gets the data from a string
buffer instead, and pass it as an argument.%  (Unfortunately, this
%technique has its limitations: a class can't define operations that
%are accessed by special syntax such as sequence subscripting or
%arithmetic operators, and assigning such a ``pseudo-file'' to
%\code{sys.stdin} will not cause the interpreter to read further input
%from it.)


Instance method objects have attributes, too: \code{m.im_self} is the
object of which the method is an instance, and \code{m.im_func} is the
function object corresponding to the method.


\section{Exceptions Are Classes Too\label{exceptionClasses}}

User-defined exceptions are identified by classes as well.  Using this
mechanism it is possible to create extensible hierarchies of exceptions.

There are two new valid (semantic) forms for the raise statement:

\begin{verbatim}
raise Class, instance

raise instance
\end{verbatim}

In the first form, \code{instance} must be an instance of
\class{Class} or of a class derived from it.  The second form is a
shorthand for:

\begin{verbatim}
raise instance.__class__, instance
\end{verbatim}

A class in an except clause is compatible with an exception if it is the same
class or a base class thereof (but not the other way around --- an
except clause listing a derived class is not compatible with a base
class).  For example, the following code will print B, C, D in that
order:

\begin{verbatim}
class B:
    pass
class C(B):
    pass
class D(C):
    pass

for c in [B, C, D]:
    try:
        raise c()
    except D:
        print "D"
    except C:
        print "C"
    except B:
        print "B"
\end{verbatim}

Note that if the except clauses were reversed (with
\samp{except B} first), it would have printed B, B, B --- the first
matching except clause is triggered.

When an error message is printed for an unhandled exception which is a
class, the class name is printed, then a colon and a space, and
finally the instance converted to a string using the built-in function
\function{str()}.


\section{Iterators\label{iterators}}

By now, you've probably noticed that most container objects can looped over
using a \code{for} statement:

\begin{verbatim}
for element in [1, 2, 3]:
    print element
for element in (1, 2, 3):
    print element
for key in {'one':1, 'two':2}:
    print key
for char in "123":
    print char
for line in open("myfile.txt"):
    print line
\end{verbatim}

This style of access is clear, concise, and convenient.  The use of iterators
pervades and unifies Python.  Behind the scenes, the \code{for} statement calls
\function{iter()} on the container object.  The function returns an iterator
object that defines the method \method{next()} which accesses elements in the
container one at a time.  When there are no more elements, \method{next()}
raises a \exception{StopIteration} exception which tells the \code{for} loop
to terminate.  This example shows how it all works:

\begin{verbatim}
>>> s = 'abc'
>>> it = iter(s)
>>> it
<iterator object at 0x00A1DB50>
>>> it.next()
'a'
>>> it.next()
'b'
>>> it.next()
'c'
>>> it.next()

Traceback (most recent call last):
  File "<pyshell#6>", line 1, in -toplevel-
    it.next()
StopIteration
\end{verbatim}

Having seen the mechanics behind the iterator protocol, it is easy to add
iterator behavior to your classes.  Define a \method{__iter__()} method
which returns an object with a \method{next()} method.  If the class defines
\method{next()}, then \method{__iter__()} can just return \code{self}:

\begin{verbatim}
>>> class Reverse:
    "Iterator for looping over a sequence backwards"
    def __init__(self, data):
        self.data = data
        self.index = len(data)
    def __iter__(self):
        return self
    def next(self):
        if self.index == 0:
            raise StopIteration
        self.index = self.index - 1
        return self.data[self.index]

>>> for char in Reverse('spam'):
	print char

m
a
p
s
\end{verbatim}


\section{Generators\label{generators}}

Generators are a simple and powerful tool for creating iterators.  They are
written like regular functions but use the \keyword{yield} statement whenever
they want to return data.  Each time the \method{next()} is called, the
generator resumes where it left-off (it remembers all the data values and
which statement was last executed).  An example shows that generators can
be trivially easy to create:

\begin{verbatim}
>>> def reverse(data):
	for index in range(len(data)-1, -1, -1):
		yield data[index]
		
>>> for char in reverse('golf'):
	print char

f
l
o
g	
\end{verbatim}

Anything that can be done with generators can also be done with class based
iterators as described in the previous section.  What makes generators so
compact is that the \method{__iter__()} and \method{next()} methods are
created automatically.

Another other key feature is that the local variables and execution state
are automatically saved between calls.  This made the function easier to write
and much more clear than an approach using class variables like
\code{self.index} and \code{self.data}.

In addition to automatic method creation and saving program state, when
generators terminate, they automatically raise \exception{StopIteration}.
In combination, these features make it easy to create iterators with no
more effort than writing a regular function.


\chapter{What Now? \label{whatNow}}

Reading this tutorial has probably reinforced your interest in using
Python --- you should be eager to apply Python to solve your
real-world problems.  Now what should you do?

You should read, or at least page through, the
\citetitle[../lib/lib.html]{Python Library Reference},
which gives complete (though terse) reference material about types,
functions, and modules that can save you a lot of time when writing
Python programs.  The standard Python distribution includes a
\emph{lot} of code in both C and Python; there are modules to read
\UNIX{} mailboxes, retrieve documents via HTTP, generate random
numbers, parse command-line options, write CGI programs, compress
data, and a lot more; skimming through the Library Reference will give
you an idea of what's available.

The major Python Web site is \url{http://www.python.org/}; it contains
code, documentation, and pointers to Python-related pages around the
Web.  This Web site is mirrored in various places around the
world, such as Europe, Japan, and Australia; a mirror may be faster
than the main site, depending on your geographical location.  A more
informal site is \url{http://starship.python.net/}, which contains a
bunch of Python-related personal home pages; many people have
downloadable software there. Many more user-created Python modules
can be found in a third-party repository at
\url{http://www.vex.net/parnassus}.

For Python-related questions and problem reports, you can post to the
newsgroup \newsgroup{comp.lang.python}, or send them to the mailing
list at \email{python-list@python.org}.  The newsgroup and mailing list
are gatewayed, so messages posted to one will automatically be
forwarded to the other.  There are around 120 postings a day (with peaks
up to several hundred),
% Postings figure based on average of last six months activity as
% reported by www.egroups.com; Jan. 2000 - June 2000: 21272 msgs / 182
% days = 116.9 msgs / day and steadily increasing.
asking (and answering) questions, suggesting new features, and
announcing new modules.  Before posting, be sure to check the list of
Frequently Asked Questions (also called the FAQ), at
\url{http://www.python.org/doc/FAQ.html}, or look for it in the
\file{Misc/} directory of the Python source distribution.  Mailing
list archives are available at \url{http://www.python.org/pipermail/}.
The FAQ answers many of the questions that come up again and again,
and may already contain the solution for your problem.


\appendix

\chapter{Interactive Input Editing and History Substitution\label{interacting}}

Some versions of the Python interpreter support editing of the current
input line and history substitution, similar to facilities found in
the Korn shell and the GNU Bash shell.  This is implemented using the
\emph{GNU Readline} library, which supports Emacs-style and vi-style
editing.  This library has its own documentation which I won't
duplicate here; however, the basics are easily explained.  The
interactive editing and history described here are optionally
available in the \UNIX{} and CygWin versions of the interpreter.

This chapter does \emph{not} document the editing facilities of Mark
Hammond's PythonWin package or the Tk-based environment, IDLE,
distributed with Python.  The command line history recall which
operates within DOS boxes on NT and some other DOS and Windows flavors 
is yet another beast.

\section{Line Editing \label{lineEditing}}

If supported, input line editing is active whenever the interpreter
prints a primary or secondary prompt.  The current line can be edited
using the conventional Emacs control characters.  The most important
of these are: \kbd{C-A} (Control-A) moves the cursor to the beginning
of the line, \kbd{C-E} to the end, \kbd{C-B} moves it one position to
the left, \kbd{C-F} to the right.  Backspace erases the character to
the left of the cursor, \kbd{C-D} the character to its right.
\kbd{C-K} kills (erases) the rest of the line to the right of the
cursor, \kbd{C-Y} yanks back the last killed string.
\kbd{C-underscore} undoes the last change you made; it can be repeated
for cumulative effect.

\section{History Substitution \label{history}}

History substitution works as follows.  All non-empty input lines
issued are saved in a history buffer, and when a new prompt is given
you are positioned on a new line at the bottom of this buffer.
\kbd{C-P} moves one line up (back) in the history buffer,
\kbd{C-N} moves one down.  Any line in the history buffer can be
edited; an asterisk appears in front of the prompt to mark a line as
modified.  Pressing the \kbd{Return} key passes the current line to
the interpreter.  \kbd{C-R} starts an incremental reverse search;
\kbd{C-S} starts a forward search.

\section{Key Bindings \label{keyBindings}}

The key bindings and some other parameters of the Readline library can
be customized by placing commands in an initialization file called
\file{\~{}/.inputrc}.  Key bindings have the form

\begin{verbatim}
key-name: function-name
\end{verbatim}

or

\begin{verbatim}
"string": function-name
\end{verbatim}

and options can be set with

\begin{verbatim}
set option-name value
\end{verbatim}

For example:

\begin{verbatim}
# I prefer vi-style editing:
set editing-mode vi

# Edit using a single line:
set horizontal-scroll-mode On

# Rebind some keys:
Meta-h: backward-kill-word
"\C-u": universal-argument
"\C-x\C-r": re-read-init-file
\end{verbatim}

Note that the default binding for \kbd{Tab} in Python is to insert a
\kbd{Tab} character instead of Readline's default filename completion
function.  If you insist, you can override this by putting

\begin{verbatim}
Tab: complete
\end{verbatim}

in your \file{\~{}/.inputrc}.  (Of course, this makes it harder to
type indented continuation lines.)

Automatic completion of variable and module names is optionally
available.  To enable it in the interpreter's interactive mode, add
the following to your startup file:\footnote{
  Python will execute the contents of a file identified by the
  \envvar{PYTHONSTARTUP} environment variable when you start an
  interactive interpreter.}
\refstmodindex{rlcompleter}\refbimodindex{readline}

\begin{verbatim}
import rlcompleter, readline
readline.parse_and_bind('tab: complete')
\end{verbatim}

This binds the \kbd{Tab} key to the completion function, so hitting
the \kbd{Tab} key twice suggests completions; it looks at Python
statement names, the current local variables, and the available module
names.  For dotted expressions such as \code{string.a}, it will
evaluate the the expression up to the final \character{.} and then
suggest completions from the attributes of the resulting object.  Note
that this may execute application-defined code if an object with a
\method{__getattr__()} method is part of the expression.

A more capable startup file might look like this example.  Note that
this deletes the names it creates once they are no longer needed; this
is done since the startup file is executed in the same namespace as
the interactive commands, and removing the names avoids creating side
effects in the interactive environments.  You may find it convenient
to keep some of the imported modules, such as \module{os}, which turn
out to be needed in most sessions with the interpreter.

\begin{verbatim}
# Add auto-completion and a stored history file of commands to your Python
# interactive interpreter. Requires Python 2.0+, readline. Autocomplete is
# bound to the Esc key by default (you can change it - see readline docs).
#
# Store the file in ~/.pystartup, and set an environment variable to point
# to it:  "export PYTHONSTARTUP=/max/home/itamar/.pystartup" in bash.
#
# Note that PYTHONSTARTUP does *not* expand "~", so you have to put in the
# full path to your home directory.

import atexit
import os
import readline
import rlcompleter

historyPath = os.path.expanduser("~/.pyhistory")

def save_history(historyPath=historyPath):
    import readline
    readline.write_history_file(historyPath)

if os.path.exists(historyPath):
    readline.read_history_file(historyPath)

atexit.register(save_history)
del os, atexit, readline, rlcompleter, save_history, historyPath
\end{verbatim}


\section{Commentary \label{commentary}}

This facility is an enormous step forward compared to earlier versions
of the interpreter; however, some wishes are left: It would be nice if
the proper indentation were suggested on continuation lines (the
parser knows if an indent token is required next).  The completion
mechanism might use the interpreter's symbol table.  A command to
check (or even suggest) matching parentheses, quotes, etc., would also
be useful.


\chapter{Floating Point Arithmetic:  Issues and Limitations\label{fp-issues}}
\sectionauthor{Tim Peters}{tim_one@email.msn.com}

Floating-point numbers are represented in computer hardware as
base 2 (binary) fractions.  For example, the decimal fraction

\begin{verbatim}
0.125
\end{verbatim}

has value 1/10 + 2/100 + 5/1000, and in the same way the binary fraction

\begin{verbatim}
0.001
\end{verbatim}

has value 0/2 + 0/4 + 1/8.  These two fractions have identical values,
the only real difference being that the first is written in base 10
fractional notation, and the second in base 2.

Unfortunately, most decimal fractions cannot be represented exactly as
binary fractions.  A consequence is that, in general, the decimal
floating-point numbers you enter are only approximated by the binary
floating-point numbers actually stored in the machine.

The problem is easier to understand at first in base 10.  Consider the
fraction 1/3.  You can approximate that as a base 10 fraction:

\begin{verbatim}
0.3
\end{verbatim}

or, better,

\begin{verbatim}
0.33
\end{verbatim}

or, better,

\begin{verbatim}
0.333
\end{verbatim}

and so on.  No matter how many digits you're willing to write down, the
result will never be exactly 1/3, but will be an increasingly better
approximation to 1/3.

In the same way, no matter how many base 2 digits you're willing to
use, the decimal value 0.1 cannot be represented exactly as a base 2
fraction.  In base 2, 1/10 is the infinitely repeating fraction

\begin{verbatim}
0.0001100110011001100110011001100110011001100110011...
\end{verbatim}

Stop at any finite number of bits, and you get an approximation.  This
is why you see things like:

\begin{verbatim}
>>> 0.1
0.10000000000000001
\end{verbatim}

On most machines today, that is what you'll see if you enter 0.1 at
a Python prompt.  You may not, though, because the number of bits
used by the hardware to store floating-point values can vary across
machines, and Python only prints a decimal approximation to the true
decimal value of the binary approximation stored by the machine.  On
most machines, if Python were to print the true decimal value of
the binary approximation stored for 0.1, it would have to display

\begin{verbatim}
>>> 0.1
0.1000000000000000055511151231257827021181583404541015625
\end{verbatim}

instead!  The Python prompt (implicitly) uses the builtin
\function{repr()} function to obtain a string version of everything it
displays.  For floats, \code{repr(\var{float})} rounds the true
decimal value to 17 significant digits, giving

\begin{verbatim}
0.10000000000000001
\end{verbatim}

\code{repr(\var{float})} produces 17 significant digits because it
turns out that's enough (on most machines) so that
\code{eval(repr(\var{x})) == \var{x}} exactly for all finite floats
\var{x}, but rounding to 16 digits is not enough to make that true.

Note that this is in the very nature of binary floating-point: this is
not a bug in Python, it is not a bug in your code either, and you'll
see the same kind of thing in all languages that support your
hardware's floating-point arithmetic (although some languages may
not \emph{display} the difference by default, or in all output modes).

Python's builtin \function{str()} function produces only 12
significant digits, and you may wish to use that instead.  It's
unusual for \code{eval(str(\var{x}))} to reproduce \var{x}, but the
output may be more pleasant to look at:

\begin{verbatim}
>>> print str(0.1)
0.1
\end{verbatim}

It's important to realize that this is, in a real sense, an illusion:
the value in the machine is not exactly 1/10, you're simply rounding
the \emph{display} of the true machine value.

Other surprises follow from this one.  For example, after seeing

\begin{verbatim}
>>> 0.1
0.10000000000000001
\end{verbatim}

you may be tempted to use the \function{round()} function to chop it
back to the single digit you expect.  But that makes no difference:

\begin{verbatim}
>>> round(0.1, 1)
0.10000000000000001
\end{verbatim}

The problem is that the binary floating-point value stored for "0.1"
was already the best possible binary approximation to 1/10, so trying
to round it again can't make it better:  it was already as good as it
gets.

Another consequence is that since 0.1 is not exactly 1/10, adding 0.1
to itself 10 times may not yield exactly 1.0, either:

\begin{verbatim}
>>> sum = 0.0
>>> for i in range(10):
...     sum += 0.1
...
>>> sum
0.99999999999999989
\end{verbatim}

Binary floating-point arithmetic holds many surprises like this.  The
problem with "0.1" is explained in precise detail below, in the
"Representation Error" section.  See
\citetitle[http://www.lahey.com/float.htm]{The Perils of Floating
Point} for a more complete account of other common surprises.

As that says near the end, ``there are no easy answers.''  Still,
don't be unduly wary of floating-point!  The errors in Python float
operations are inherited from the floating-point hardware, and on most
machines are on the order of no more than 1 part in 2**53 per
operation.  That's more than adequate for most tasks, but you do need
to keep in mind that it's not decimal arithmetic, and that every float
operation can suffer a new rounding error.

While pathological cases do exist, for most casual use of
floating-point arithmetic you'll see the result you expect in the end
if you simply round the display of your final results to the number of
decimal digits you expect.  \function{str()} usually suffices, and for
finer control see the discussion of Pythons's \code{\%} format
operator: the \code{\%g}, \code{\%f} and \code{\%e} format codes
supply flexible and easy ways to round float results for display.


\section{Representation Error
         \label{fp-error}}

This section explains the ``0.1'' example in detail, and shows how
you can perform an exact analysis of cases like this yourself.  Basic
familiarity with binary floating-point representation is assumed.

\dfn{Representation error} refers to that some (most, actually)
decimal fractions cannot be represented exactly as binary (base 2)
fractions.  This is the chief reason why Python (or Perl, C, \Cpp,
Java, Fortran, and many others) often won't display the exact decimal
number you expect:

\begin{verbatim}
>>> 0.1
0.10000000000000001
\end{verbatim}

Why is that?  1/10 is not exactly representable as a binary fraction.
Almost all machines today (November 2000) use IEEE-754 floating point
arithmetic, and almost all platforms map Python floats to IEEE-754
"double precision".  754 doubles contain 53 bits of precision, so on
input the computer strives to convert 0.1 to the closest fraction it can
of the form \var{J}/2**\var{N} where \var{J} is an integer containing
exactly 53 bits.  Rewriting

\begin{verbatim}
 1 / 10 ~= J / (2**N)
\end{verbatim}

as

\begin{verbatim}
J ~= 2**N / 10
\end{verbatim}

and recalling that \var{J} has exactly 53 bits (is \code{>= 2**52} but
\code{< 2**53}), the best value for \var{N} is 56:

\begin{verbatim}
>>> 2L**52
4503599627370496L
>>> 2L**53
9007199254740992L
>>> 2L**56/10
7205759403792793L
\end{verbatim}

That is, 56 is the only value for \var{N} that leaves \var{J} with
exactly 53 bits.  The best possible value for \var{J} is then that
quotient rounded:

\begin{verbatim}
>>> q, r = divmod(2L**56, 10)
>>> r
6L
\end{verbatim}

Since the remainder is more than half of 10, the best approximation is
obtained by rounding up:

\begin{verbatim}
>>> q+1
7205759403792794L
\end{verbatim}

Therefore the best possible approximation to 1/10 in 754 double
precision is that over 2**56, or

\begin{verbatim}
7205759403792794 / 72057594037927936
\end{verbatim}

Note that since we rounded up, this is actually a little bit larger than
1/10; if we had not rounded up, the quotient would have been a little
bit smaller than 1/10.  But in no case can it be \emph{exactly} 1/10!

So the computer never ``sees'' 1/10:  what it sees is the exact
fraction given above, the best 754 double approximation it can get:

\begin{verbatim}
>>> .1 * 2L**56
7205759403792794.0
\end{verbatim}

If we multiply that fraction by 10**30, we can see the (truncated)
value of its 30 most significant decimal digits:

\begin{verbatim}
>>> 7205759403792794L * 10L**30 / 2L**56
100000000000000005551115123125L
\end{verbatim}

meaning that the exact number stored in the computer is approximately
equal to the decimal value 0.100000000000000005551115123125.  Rounding
that to 17 significant digits gives the 0.10000000000000001 that Python
displays (well, will display on any 754-conforming platform that does
best-possible input and output conversions in its C library --- yours may
not!).

\chapter{History and License}
\section{History of the software}

Python was created in the early 1990s by Guido van Rossum at Stichting
Mathematisch Centrum (CWI, see \url{http://www.cwi.nl/}) in the Netherlands
as a successor of a language called ABC.  Guido remains Python's
principal author, although it includes many contributions from others.

In 1995, Guido continued his work on Python at the Corporation for
National Research Initiatives (CNRI, see \url{http://www.cnri.reston.va.us/})
in Reston, Virginia where he released several versions of the
software.

In May 2000, Guido and the Python core development team moved to
BeOpen.com to form the BeOpen PythonLabs team.  In October of the same
year, the PythonLabs team moved to Digital Creations (now Zope
Corporation; see \url{http://www.zope.com/}).  In 2001, the Python
Software Foundation (PSF, see \url{http://www.python.org/psf/}) was
formed, a non-profit organization created specifically to own
Python-related Intellectual Property.  Zope Corporation is a
sponsoring member of the PSF.

All Python releases are Open Source (see
\url{http://www.opensource.org/} for the Open Source Definition).
Historically, most, but not all, Python releases have also been
GPL-compatible; the table below summarizes the various releases.

\begin{tablev}{c|c|c|c|c}{textrm}%
  {Release}{Derived from}{Year}{Owner}{GPL compatible?}
  \linev{0.9.0 thru 1.2}{n/a}{1991-1995}{CWI}{yes}
  \linev{1.3 thru 1.5.2}{1.2}{1995-1999}{CNRI}{yes}
  \linev{1.6}{1.5.2}{2000}{CNRI}{no}
  \linev{2.0}{1.6}{2000}{BeOpen.com}{no}
  \linev{1.6.1}{1.6}{2001}{CNRI}{no}
  \linev{2.1}{2.0+1.6.1}{2001}{PSF}{no}
  \linev{2.0.1}{2.0+1.6.1}{2001}{PSF}{yes}
  \linev{2.1.1}{2.1+2.0.1}{2001}{PSF}{yes}
  \linev{2.2}{2.1.1}{2001}{PSF}{yes}
  \linev{2.1.2}{2.1.1}{2002}{PSF}{yes}
  \linev{2.1.3}{2.1.2}{2002}{PSF}{yes}
  \linev{2.2.1}{2.2}{2002}{PSF}{yes}
  \linev{2.2.2}{2.2.1}{2002}{PSF}{yes}
  \linev{2.2.3}{2.2.2}{2002-2003}{PSF}{yes}
  \linev{2.3}{2.2.2}{2002-2003}{PSF}{yes}
  \linev{2.3.1}{2.3}{2002-2003}{PSF}{yes}
  \linev{2.3.2}{2.3.1}{2003}{PSF}{yes}
  \linev{2.3.3}{2.3.2}{2003}{PSF}{yes}
  \linev{2.3.4}{2.3.3}{2004}{PSF}{yes}
  \linev{2.3.5}{2.3.4}{2005}{PSF}{yes}
  \linev{2.4}{2.3}{2004}{PSF}{yes}
\end{tablev}

\note{GPL-compatible doesn't mean that we're distributing
Python under the GPL.  All Python licenses, unlike the GPL, let you
distribute a modified version without making your changes open source.
The GPL-compatible licenses make it possible to combine Python with
other software that is released under the GPL; the others don't.}

Thanks to the many outside volunteers who have worked under Guido's
direction to make these releases possible.


\section{Terms and conditions for accessing or otherwise using Python}

\centerline{\strong{PSF LICENSE AGREEMENT FOR PYTHON \version}}

\begin{enumerate}
\item
This LICENSE AGREEMENT is between the Python Software Foundation
(``PSF''), and the Individual or Organization (``Licensee'') accessing
and otherwise using Python \version{} software in source or binary
form and its associated documentation.

\item
Subject to the terms and conditions of this License Agreement, PSF
hereby grants Licensee a nonexclusive, royalty-free, world-wide
license to reproduce, analyze, test, perform and/or display publicly,
prepare derivative works, distribute, and otherwise use Python
\version{} alone or in any derivative version, provided, however, that
PSF's License Agreement and PSF's notice of copyright, i.e.,
``Copyright \copyright{} 2001-2004 Python Software Foundation; All
Rights Reserved'' are retained in Python \version{} alone or in any
derivative version prepared by Licensee.

\item
In the event Licensee prepares a derivative work that is based on
or incorporates Python \version{} or any part thereof, and wants to
make the derivative work available to others as provided herein, then
Licensee hereby agrees to include in any such work a brief summary of
the changes made to Python \version.

\item
PSF is making Python \version{} available to Licensee on an ``AS IS''
basis.  PSF MAKES NO REPRESENTATIONS OR WARRANTIES, EXPRESS OR
IMPLIED.  BY WAY OF EXAMPLE, BUT NOT LIMITATION, PSF MAKES NO AND
DISCLAIMS ANY REPRESENTATION OR WARRANTY OF MERCHANTABILITY OR FITNESS
FOR ANY PARTICULAR PURPOSE OR THAT THE USE OF PYTHON \version{} WILL
NOT INFRINGE ANY THIRD PARTY RIGHTS.

\item
PSF SHALL NOT BE LIABLE TO LICENSEE OR ANY OTHER USERS OF PYTHON
\version{} FOR ANY INCIDENTAL, SPECIAL, OR CONSEQUENTIAL DAMAGES OR
LOSS AS A RESULT OF MODIFYING, DISTRIBUTING, OR OTHERWISE USING PYTHON
\version, OR ANY DERIVATIVE THEREOF, EVEN IF ADVISED OF THE
POSSIBILITY THEREOF.

\item
This License Agreement will automatically terminate upon a material
breach of its terms and conditions.

\item
Nothing in this License Agreement shall be deemed to create any
relationship of agency, partnership, or joint venture between PSF and
Licensee.  This License Agreement does not grant permission to use PSF
trademarks or trade name in a trademark sense to endorse or promote
products or services of Licensee, or any third party.

\item
By copying, installing or otherwise using Python \version, Licensee
agrees to be bound by the terms and conditions of this License
Agreement.
\end{enumerate}


\centerline{\strong{BEOPEN.COM LICENSE AGREEMENT FOR PYTHON 2.0}}

\centerline{\strong{BEOPEN PYTHON OPEN SOURCE LICENSE AGREEMENT VERSION 1}}

\begin{enumerate}
\item
This LICENSE AGREEMENT is between BeOpen.com (``BeOpen''), having an
office at 160 Saratoga Avenue, Santa Clara, CA 95051, and the
Individual or Organization (``Licensee'') accessing and otherwise
using this software in source or binary form and its associated
documentation (``the Software'').

\item
Subject to the terms and conditions of this BeOpen Python License
Agreement, BeOpen hereby grants Licensee a non-exclusive,
royalty-free, world-wide license to reproduce, analyze, test, perform
and/or display publicly, prepare derivative works, distribute, and
otherwise use the Software alone or in any derivative version,
provided, however, that the BeOpen Python License is retained in the
Software, alone or in any derivative version prepared by Licensee.

\item
BeOpen is making the Software available to Licensee on an ``AS IS''
basis.  BEOPEN MAKES NO REPRESENTATIONS OR WARRANTIES, EXPRESS OR
IMPLIED.  BY WAY OF EXAMPLE, BUT NOT LIMITATION, BEOPEN MAKES NO AND
DISCLAIMS ANY REPRESENTATION OR WARRANTY OF MERCHANTABILITY OR FITNESS
FOR ANY PARTICULAR PURPOSE OR THAT THE USE OF THE SOFTWARE WILL NOT
INFRINGE ANY THIRD PARTY RIGHTS.

\item
BEOPEN SHALL NOT BE LIABLE TO LICENSEE OR ANY OTHER USERS OF THE
SOFTWARE FOR ANY INCIDENTAL, SPECIAL, OR CONSEQUENTIAL DAMAGES OR LOSS
AS A RESULT OF USING, MODIFYING OR DISTRIBUTING THE SOFTWARE, OR ANY
DERIVATIVE THEREOF, EVEN IF ADVISED OF THE POSSIBILITY THEREOF.

\item
This License Agreement will automatically terminate upon a material
breach of its terms and conditions.

\item
This License Agreement shall be governed by and interpreted in all
respects by the law of the State of California, excluding conflict of
law provisions.  Nothing in this License Agreement shall be deemed to
create any relationship of agency, partnership, or joint venture
between BeOpen and Licensee.  This License Agreement does not grant
permission to use BeOpen trademarks or trade names in a trademark
sense to endorse or promote products or services of Licensee, or any
third party.  As an exception, the ``BeOpen Python'' logos available
at http://www.pythonlabs.com/logos.html may be used according to the
permissions granted on that web page.

\item
By copying, installing or otherwise using the software, Licensee
agrees to be bound by the terms and conditions of this License
Agreement.
\end{enumerate}


\centerline{\strong{CNRI LICENSE AGREEMENT FOR PYTHON 1.6.1}}

\begin{enumerate}
\item
This LICENSE AGREEMENT is between the Corporation for National
Research Initiatives, having an office at 1895 Preston White Drive,
Reston, VA 20191 (``CNRI''), and the Individual or Organization
(``Licensee'') accessing and otherwise using Python 1.6.1 software in
source or binary form and its associated documentation.

\item
Subject to the terms and conditions of this License Agreement, CNRI
hereby grants Licensee a nonexclusive, royalty-free, world-wide
license to reproduce, analyze, test, perform and/or display publicly,
prepare derivative works, distribute, and otherwise use Python 1.6.1
alone or in any derivative version, provided, however, that CNRI's
License Agreement and CNRI's notice of copyright, i.e., ``Copyright
\copyright{} 1995-2001 Corporation for National Research Initiatives;
All Rights Reserved'' are retained in Python 1.6.1 alone or in any
derivative version prepared by Licensee.  Alternately, in lieu of
CNRI's License Agreement, Licensee may substitute the following text
(omitting the quotes): ``Python 1.6.1 is made available subject to the
terms and conditions in CNRI's License Agreement.  This Agreement
together with Python 1.6.1 may be located on the Internet using the
following unique, persistent identifier (known as a handle):
1895.22/1013.  This Agreement may also be obtained from a proxy server
on the Internet using the following URL:
\url{http://hdl.handle.net/1895.22/1013}.''

\item
In the event Licensee prepares a derivative work that is based on
or incorporates Python 1.6.1 or any part thereof, and wants to make
the derivative work available to others as provided herein, then
Licensee hereby agrees to include in any such work a brief summary of
the changes made to Python 1.6.1.

\item
CNRI is making Python 1.6.1 available to Licensee on an ``AS IS''
basis.  CNRI MAKES NO REPRESENTATIONS OR WARRANTIES, EXPRESS OR
IMPLIED.  BY WAY OF EXAMPLE, BUT NOT LIMITATION, CNRI MAKES NO AND
DISCLAIMS ANY REPRESENTATION OR WARRANTY OF MERCHANTABILITY OR FITNESS
FOR ANY PARTICULAR PURPOSE OR THAT THE USE OF PYTHON 1.6.1 WILL NOT
INFRINGE ANY THIRD PARTY RIGHTS.

\item
CNRI SHALL NOT BE LIABLE TO LICENSEE OR ANY OTHER USERS OF PYTHON
1.6.1 FOR ANY INCIDENTAL, SPECIAL, OR CONSEQUENTIAL DAMAGES OR LOSS AS
A RESULT OF MODIFYING, DISTRIBUTING, OR OTHERWISE USING PYTHON 1.6.1,
OR ANY DERIVATIVE THEREOF, EVEN IF ADVISED OF THE POSSIBILITY THEREOF.

\item
This License Agreement will automatically terminate upon a material
breach of its terms and conditions.

\item
This License Agreement shall be governed by the federal
intellectual property law of the United States, including without
limitation the federal copyright law, and, to the extent such
U.S. federal law does not apply, by the law of the Commonwealth of
Virginia, excluding Virginia's conflict of law provisions.
Notwithstanding the foregoing, with regard to derivative works based
on Python 1.6.1 that incorporate non-separable material that was
previously distributed under the GNU General Public License (GPL), the
law of the Commonwealth of Virginia shall govern this License
Agreement only as to issues arising under or with respect to
Paragraphs 4, 5, and 7 of this License Agreement.  Nothing in this
License Agreement shall be deemed to create any relationship of
agency, partnership, or joint venture between CNRI and Licensee.  This
License Agreement does not grant permission to use CNRI trademarks or
trade name in a trademark sense to endorse or promote products or
services of Licensee, or any third party.

\item
By clicking on the ``ACCEPT'' button where indicated, or by copying,
installing or otherwise using Python 1.6.1, Licensee agrees to be
bound by the terms and conditions of this License Agreement.
\end{enumerate}

\centerline{ACCEPT}



\centerline{\strong{CWI LICENSE AGREEMENT FOR PYTHON 0.9.0 THROUGH 1.2}}

Copyright \copyright{} 1991 - 1995, Stichting Mathematisch Centrum
Amsterdam, The Netherlands.  All rights reserved.

Permission to use, copy, modify, and distribute this software and its
documentation for any purpose and without fee is hereby granted,
provided that the above copyright notice appear in all copies and that
both that copyright notice and this permission notice appear in
supporting documentation, and that the name of Stichting Mathematisch
Centrum or CWI not be used in advertising or publicity pertaining to
distribution of the software without specific, written prior
permission.

STICHTING MATHEMATISCH CENTRUM DISCLAIMS ALL WARRANTIES WITH REGARD TO
THIS SOFTWARE, INCLUDING ALL IMPLIED WARRANTIES OF MERCHANTABILITY AND
FITNESS, IN NO EVENT SHALL STICHTING MATHEMATISCH CENTRUM BE LIABLE
FOR ANY SPECIAL, INDIRECT OR CONSEQUENTIAL DAMAGES OR ANY DAMAGES
WHATSOEVER RESULTING FROM LOSS OF USE, DATA OR PROFITS, WHETHER IN AN
ACTION OF CONTRACT, NEGLIGENCE OR OTHER TORTIOUS ACTION, ARISING OUT
OF OR IN CONNECTION WITH THE USE OR PERFORMANCE OF THIS SOFTWARE.


\section{Licenses and Acknowledgements for Incorporated Software}

This section is an incomplete, but growing list of licenses and
acknowledgements for third-party software incorporated in the
Python distribution.


\subsection{Mersenne Twister}

The \module{_random} module includes code based on a download from
\url{http://www.math.keio.ac.jp/~matumoto/MT2002/emt19937ar.html}.
The following are the verbatim comments from the original code:

\begin{verbatim}
A C-program for MT19937, with initialization improved 2002/1/26.
Coded by Takuji Nishimura and Makoto Matsumoto.

Before using, initialize the state by using init_genrand(seed)
or init_by_array(init_key, key_length).

Copyright (C) 1997 - 2002, Makoto Matsumoto and Takuji Nishimura,
All rights reserved.

Redistribution and use in source and binary forms, with or without
modification, are permitted provided that the following conditions
are met:

 1. Redistributions of source code must retain the above copyright
    notice, this list of conditions and the following disclaimer.

 2. Redistributions in binary form must reproduce the above copyright
    notice, this list of conditions and the following disclaimer in the
    documentation and/or other materials provided with the distribution.

 3. The names of its contributors may not be used to endorse or promote
    products derived from this software without specific prior written
    permission.

THIS SOFTWARE IS PROVIDED BY THE COPYRIGHT HOLDERS AND CONTRIBUTORS
"AS IS" AND ANY EXPRESS OR IMPLIED WARRANTIES, INCLUDING, BUT NOT
LIMITED TO, THE IMPLIED WARRANTIES OF MERCHANTABILITY AND FITNESS FOR
A PARTICULAR PURPOSE ARE DISCLAIMED.  IN NO EVENT SHALL THE COPYRIGHT OWNER OR
CONTRIBUTORS BE LIABLE FOR ANY DIRECT, INDIRECT, INCIDENTAL, SPECIAL,
EXEMPLARY, OR CONSEQUENTIAL DAMAGES (INCLUDING, BUT NOT LIMITED TO,
PROCUREMENT OF SUBSTITUTE GOODS OR SERVICES; LOSS OF USE, DATA, OR
PROFITS; OR BUSINESS INTERRUPTION) HOWEVER CAUSED AND ON ANY THEORY OF
LIABILITY, WHETHER IN CONTRACT, STRICT LIABILITY, OR TORT (INCLUDING
NEGLIGENCE OR OTHERWISE) ARISING IN ANY WAY OUT OF THE USE OF THIS
SOFTWARE, EVEN IF ADVISED OF THE POSSIBILITY OF SUCH DAMAGE.


Any feedback is very welcome.
http://www.math.keio.ac.jp/matumoto/emt.html
email: matumoto@math.keio.ac.jp
\end{verbatim}



\subsection{Sockets}

The \module{socket} module uses the functions, \function{getaddrinfo},
and \function{getnameinfo}, which are coded in separate source files
from the WIDE Project, \url{http://www.wide.ad.jp/about/index.html}.

\begin{verbatim}      
Copyright (C) 1995, 1996, 1997, and 1998 WIDE Project.
All rights reserved.
 
Redistribution and use in source and binary forms, with or without
modification, are permitted provided that the following conditions
are met:
1. Redistributions of source code must retain the above copyright
   notice, this list of conditions and the following disclaimer.
2. Redistributions in binary form must reproduce the above copyright
   notice, this list of conditions and the following disclaimer in the
   documentation and/or other materials provided with the distribution.
3. Neither the name of the project nor the names of its contributors
   may be used to endorse or promote products derived from this software
   without specific prior written permission.

THIS SOFTWARE IS PROVIDED BY THE PROJECT AND CONTRIBUTORS ``AS IS'' AND
GAI_ANY EXPRESS OR IMPLIED WARRANTIES, INCLUDING, BUT NOT LIMITED TO, THE
IMPLIED WARRANTIES OF MERCHANTABILITY AND FITNESS FOR A PARTICULAR PURPOSE
ARE DISCLAIMED.  IN NO EVENT SHALL THE PROJECT OR CONTRIBUTORS BE LIABLE
FOR GAI_ANY DIRECT, INDIRECT, INCIDENTAL, SPECIAL, EXEMPLARY, OR CONSEQUENTIAL
DAMAGES (INCLUDING, BUT NOT LIMITED TO, PROCUREMENT OF SUBSTITUTE GOODS
OR SERVICES; LOSS OF USE, DATA, OR PROFITS; OR BUSINESS INTERRUPTION)
HOWEVER CAUSED AND ON GAI_ANY THEORY OF LIABILITY, WHETHER IN CONTRACT, STRICT
LIABILITY, OR TORT (INCLUDING NEGLIGENCE OR OTHERWISE) ARISING IN GAI_ANY WAY
OUT OF THE USE OF THIS SOFTWARE, EVEN IF ADVISED OF THE POSSIBILITY OF
SUCH DAMAGE.
\end{verbatim}



\subsection{Floating point exception control}

The source for the \module{fpectl} module includes the following notice:

\begin{verbatim}
     ---------------------------------------------------------------------  
    /                       Copyright (c) 1996.                           \ 
   |          The Regents of the University of California.                 |
   |                        All rights reserved.                           |
   |                                                                       |
   |   Permission to use, copy, modify, and distribute this software for   |
   |   any purpose without fee is hereby granted, provided that this en-   |
   |   tire notice is included in all copies of any software which is or   |
   |   includes  a  copy  or  modification  of  this software and in all   |
   |   copies of the supporting documentation for such software.           |
   |                                                                       |
   |   This  work was produced at the University of California, Lawrence   |
   |   Livermore National Laboratory under  contract  no.  W-7405-ENG-48   |
   |   between  the  U.S.  Department  of  Energy and The Regents of the   |
   |   University of California for the operation of UC LLNL.              |
   |                                                                       |
   |                              DISCLAIMER                               |
   |                                                                       |
   |   This  software was prepared as an account of work sponsored by an   |
   |   agency of the United States Government. Neither the United States   |
   |   Government  nor the University of California nor any of their em-   |
   |   ployees, makes any warranty, express or implied, or  assumes  any   |
   |   liability  or  responsibility  for the accuracy, completeness, or   |
   |   usefulness of any information,  apparatus,  product,  or  process   |
   |   disclosed,   or  represents  that  its  use  would  not  infringe   |
   |   privately-owned rights. Reference herein to any specific  commer-   |
   |   cial  products,  process,  or  service  by trade name, trademark,   |
   |   manufacturer, or otherwise, does not  necessarily  constitute  or   |
   |   imply  its endorsement, recommendation, or favoring by the United   |
   |   States Government or the University of California. The views  and   |
   |   opinions  of authors expressed herein do not necessarily state or   |
   |   reflect those of the United States Government or  the  University   |
   |   of  California,  and shall not be used for advertising or product   |
    \  endorsement purposes.                                              / 
     ---------------------------------------------------------------------
\end{verbatim}



\subsection{MD5 message digest algorithm}

The source code for the \module{md5} module contains the following notice:

\begin{verbatim}
Copyright (C) 1991-2, RSA Data Security, Inc. Created 1991. All
rights reserved.

License to copy and use this software is granted provided that it
is identified as the "RSA Data Security, Inc. MD5 Message-Digest
Algorithm" in all material mentioning or referencing this software
or this function.

License is also granted to make and use derivative works provided
that such works are identified as "derived from the RSA Data
Security, Inc. MD5 Message-Digest Algorithm" in all material
mentioning or referencing the derived work.

RSA Data Security, Inc. makes no representations concerning either
the merchantability of this software or the suitability of this
software for any particular purpose. It is provided "as is"
without express or implied warranty of any kind.

These notices must be retained in any copies of any part of this
documentation and/or software.
\end{verbatim}



\subsection{Asynchronous socket services}

The \module{asynchat} and \module{asyncore} modules contain the
following notice:

\begin{verbatim}      
 Copyright 1996 by Sam Rushing

                         All Rights Reserved

 Permission to use, copy, modify, and distribute this software and
 its documentation for any purpose and without fee is hereby
 granted, provided that the above copyright notice appear in all
 copies and that both that copyright notice and this permission
 notice appear in supporting documentation, and that the name of Sam
 Rushing not be used in advertising or publicity pertaining to
 distribution of the software without specific, written prior
 permission.

 SAM RUSHING DISCLAIMS ALL WARRANTIES WITH REGARD TO THIS SOFTWARE,
 INCLUDING ALL IMPLIED WARRANTIES OF MERCHANTABILITY AND FITNESS, IN
 NO EVENT SHALL SAM RUSHING BE LIABLE FOR ANY SPECIAL, INDIRECT OR
 CONSEQUENTIAL DAMAGES OR ANY DAMAGES WHATSOEVER RESULTING FROM LOSS
 OF USE, DATA OR PROFITS, WHETHER IN AN ACTION OF CONTRACT,
 NEGLIGENCE OR OTHER TORTIOUS ACTION, ARISING OUT OF OR IN
 CONNECTION WITH THE USE OR PERFORMANCE OF THIS SOFTWARE.
\end{verbatim}


\subsection{Cookie management}

The \module{Cookie} module contains the following notice:

\begin{verbatim}
 Copyright 2000 by Timothy O'Malley <timo@alum.mit.edu>

                All Rights Reserved

 Permission to use, copy, modify, and distribute this software
 and its documentation for any purpose and without fee is hereby
 granted, provided that the above copyright notice appear in all
 copies and that both that copyright notice and this permission
 notice appear in supporting documentation, and that the name of
 Timothy O'Malley  not be used in advertising or publicity
 pertaining to distribution of the software without specific, written
 prior permission.

 Timothy O'Malley DISCLAIMS ALL WARRANTIES WITH REGARD TO THIS
 SOFTWARE, INCLUDING ALL IMPLIED WARRANTIES OF MERCHANTABILITY
 AND FITNESS, IN NO EVENT SHALL Timothy O'Malley BE LIABLE FOR
 ANY SPECIAL, INDIRECT OR CONSEQUENTIAL DAMAGES OR ANY DAMAGES
 WHATSOEVER RESULTING FROM LOSS OF USE, DATA OR PROFITS,
 WHETHER IN AN ACTION OF CONTRACT, NEGLIGENCE OR OTHER TORTIOUS
 ACTION, ARISING OUT OF OR IN CONNECTION WITH THE USE OR
 PERFORMANCE OF THIS SOFTWARE.
\end{verbatim}      



\subsection{Profiling}

The \module{profile} and \module{pstats} modules contain
the following notice:

\begin{verbatim}
 Copyright 1994, by InfoSeek Corporation, all rights reserved.
 Written by James Roskind

 Permission to use, copy, modify, and distribute this Python software
 and its associated documentation for any purpose (subject to the
 restriction in the following sentence) without fee is hereby granted,
 provided that the above copyright notice appears in all copies, and
 that both that copyright notice and this permission notice appear in
 supporting documentation, and that the name of InfoSeek not be used in
 advertising or publicity pertaining to distribution of the software
 without specific, written prior permission.  This permission is
 explicitly restricted to the copying and modification of the software
 to remain in Python, compiled Python, or other languages (such as C)
 wherein the modified or derived code is exclusively imported into a
 Python module.

 INFOSEEK CORPORATION DISCLAIMS ALL WARRANTIES WITH REGARD TO THIS
 SOFTWARE, INCLUDING ALL IMPLIED WARRANTIES OF MERCHANTABILITY AND
 FITNESS. IN NO EVENT SHALL INFOSEEK CORPORATION BE LIABLE FOR ANY
 SPECIAL, INDIRECT OR CONSEQUENTIAL DAMAGES OR ANY DAMAGES WHATSOEVER
 RESULTING FROM LOSS OF USE, DATA OR PROFITS, WHETHER IN AN ACTION OF
 CONTRACT, NEGLIGENCE OR OTHER TORTIOUS ACTION, ARISING OUT OF OR IN
 CONNECTION WITH THE USE OR PERFORMANCE OF THIS SOFTWARE.
\end{verbatim}



\subsection{Execution tracing}

The \module{trace} module contains the following notice:

\begin{verbatim}
 portions copyright 2001, Autonomous Zones Industries, Inc., all rights...
 err...  reserved and offered to the public under the terms of the
 Python 2.2 license.
 Author: Zooko O'Whielacronx
 http://zooko.com/
 mailto:zooko@zooko.com

 Copyright 2000, Mojam Media, Inc., all rights reserved.
 Author: Skip Montanaro

 Copyright 1999, Bioreason, Inc., all rights reserved.
 Author: Andrew Dalke

 Copyright 1995-1997, Automatrix, Inc., all rights reserved.
 Author: Skip Montanaro

 Copyright 1991-1995, Stichting Mathematisch Centrum, all rights reserved.


 Permission to use, copy, modify, and distribute this Python software and
 its associated documentation for any purpose without fee is hereby
 granted, provided that the above copyright notice appears in all copies,
 and that both that copyright notice and this permission notice appear in
 supporting documentation, and that the name of neither Automatrix,
 Bioreason or Mojam Media be used in advertising or publicity pertaining to
 distribution of the software without specific, written prior permission.
\end{verbatim} 



\subsection{UUencode and UUdecode functions}

The \module{uu} module contains the following notice:

\begin{verbatim}
 Copyright 1994 by Lance Ellinghouse
 Cathedral City, California Republic, United States of America.
                        All Rights Reserved
 Permission to use, copy, modify, and distribute this software and its
 documentation for any purpose and without fee is hereby granted,
 provided that the above copyright notice appear in all copies and that
 both that copyright notice and this permission notice appear in
 supporting documentation, and that the name of Lance Ellinghouse
 not be used in advertising or publicity pertaining to distribution
 of the software without specific, written prior permission.
 LANCE ELLINGHOUSE DISCLAIMS ALL WARRANTIES WITH REGARD TO
 THIS SOFTWARE, INCLUDING ALL IMPLIED WARRANTIES OF MERCHANTABILITY AND
 FITNESS, IN NO EVENT SHALL LANCE ELLINGHOUSE CENTRUM BE LIABLE
 FOR ANY SPECIAL, INDIRECT OR CONSEQUENTIAL DAMAGES OR ANY DAMAGES
 WHATSOEVER RESULTING FROM LOSS OF USE, DATA OR PROFITS, WHETHER IN AN
 ACTION OF CONTRACT, NEGLIGENCE OR OTHER TORTIOUS ACTION, ARISING OUT
 OF OR IN CONNECTION WITH THE USE OR PERFORMANCE OF THIS SOFTWARE.

 Modified by Jack Jansen, CWI, July 1995:
 - Use binascii module to do the actual line-by-line conversion
   between ascii and binary. This results in a 1000-fold speedup. The C
   version is still 5 times faster, though.
 - Arguments more compliant with python standard
\end{verbatim}



\subsection{XML Remote Procedure Calls}

The \module{xmlrpclib} module contains the following notice:

\begin{verbatim}
     The XML-RPC client interface is

 Copyright (c) 1999-2002 by Secret Labs AB
 Copyright (c) 1999-2002 by Fredrik Lundh

 By obtaining, using, and/or copying this software and/or its
 associated documentation, you agree that you have read, understood,
 and will comply with the following terms and conditions:

 Permission to use, copy, modify, and distribute this software and
 its associated documentation for any purpose and without fee is
 hereby granted, provided that the above copyright notice appears in
 all copies, and that both that copyright notice and this permission
 notice appear in supporting documentation, and that the name of
 Secret Labs AB or the author not be used in advertising or publicity
 pertaining to distribution of the software without specific, written
 prior permission.

 SECRET LABS AB AND THE AUTHOR DISCLAIMS ALL WARRANTIES WITH REGARD
 TO THIS SOFTWARE, INCLUDING ALL IMPLIED WARRANTIES OF MERCHANT-
 ABILITY AND FITNESS.  IN NO EVENT SHALL SECRET LABS AB OR THE AUTHOR
 BE LIABLE FOR ANY SPECIAL, INDIRECT OR CONSEQUENTIAL DAMAGES OR ANY
 DAMAGES WHATSOEVER RESULTING FROM LOSS OF USE, DATA OR PROFITS,
 WHETHER IN AN ACTION OF CONTRACT, NEGLIGENCE OR OTHER TORTIOUS
 ACTION, ARISING OUT OF OR IN CONNECTION WITH THE USE OR PERFORMANCE
 OF THIS SOFTWARE.
\end{verbatim}


\end{document}


\end{document}

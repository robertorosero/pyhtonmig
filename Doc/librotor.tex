\section{Built-in Module \sectcode{rotor}}
\label{module-rotor}
\bimodindex{rotor}

This module implements a rotor-based encryption algorithm, contributed by
Lance Ellinghouse.  The design is derived from the Enigma device, a machine
used during World War II to encipher messages.  A rotor is simply a
permutation.  For example, if the character `A' is the origin of the rotor,
then a given rotor might map `A' to `L', `B' to `Z', `C' to `G', and so on.
To encrypt, we choose several different rotors, and set the origins of the
rotors to known positions; their initial position is the ciphering key.  To
encipher a character, we permute the original character by the first rotor,
and then apply the second rotor's permutation to the result. We continue
until we've applied all the rotors; the resulting character is our
ciphertext.  We then change the origin of the final rotor by one position,
from `A' to `B'; if the final rotor has made a complete revolution, then we
rotate the next-to-last rotor by one position, and apply the same procedure
recursively.  In other words, after enciphering one character, we advance
the rotors in the same fashion as a car's odometer. Decoding works in the
same way, except we reverse the permutations and apply them in the opposite
order.
\index{Ellinghouse, Lance}
\indexii{Enigma}{cipher}

The available functions in this module are:

\setindexsubitem{(in module rotor)}
\begin{funcdesc}{newrotor}{key\optional{\, numrotors}}
Return a rotor object. \var{key} is a string containing the encryption key
for the object; it can contain arbitrary binary data. The key will be used
to randomly generate the rotor permutations and their initial positions.
\var{numrotors} is the number of rotor permutations in the returned object;
if it is omitted, a default value of 6 will be used.
\end{funcdesc}

Rotor objects have the following methods:

\setindexsubitem{(rotor method)}
\begin{funcdesc}{setkey}{key}
Sets the rotor's key to \var{key}.
\end{funcdesc}

\begin{funcdesc}{encrypt}{plaintext}
Reset the rotor object to its initial state and encrypt \var{plaintext},
returning a string containing the ciphertext.  The ciphertext is always the
same length as the original plaintext.
\end{funcdesc}

\begin{funcdesc}{encryptmore}{plaintext}
Encrypt \var{plaintext} without resetting the rotor object, and return a
string containing the ciphertext.
\end{funcdesc}

\begin{funcdesc}{decrypt}{ciphertext}
Reset the rotor object to its initial state and decrypt \var{ciphertext},
returning a string containing the ciphertext.  The plaintext string will
always be the same length as the ciphertext.
\end{funcdesc}

\begin{funcdesc}{decryptmore}{ciphertext}
Decrypt \var{ciphertext} without resetting the rotor object, and return a
string containing the ciphertext.
\end{funcdesc}

An example usage:
\begin{verbatim}
>>> import rotor
>>> rt = rotor.newrotor('key', 12)
>>> rt.encrypt('bar')
'\2534\363'
>>> rt.encryptmore('bar')
'\357\375$'
>>> rt.encrypt('bar')
'\2534\363'
>>> rt.decrypt('\2534\363')
'bar'
>>> rt.decryptmore('\357\375$')
'bar'
>>> rt.decrypt('\357\375$')
'l(\315'
>>> del rt
\end{verbatim}
%
The module's code is not an exact simulation of the original Enigma device;
it implements the rotor encryption scheme differently from the original. The
most important difference is that in the original Enigma, there were only 5
or 6 different rotors in existence, and they were applied twice to each
character; the cipher key was the order in which they were placed in the
machine.  The Python rotor module uses the supplied key to initialize a
random number generator; the rotor permutations and their initial positions
are then randomly generated.  The original device only enciphered the
letters of the alphabet, while this module can handle any 8-bit binary data;
it also produces binary output.  This module can also operate with an
arbitrary number of rotors.

The original Enigma cipher was broken in 1944. % XXX: Is this right?
The version implemented here is probably a good deal more difficult to crack
(especially if you use many rotors), but it won't be impossible for
a truly skilful and determined attacker to break the cipher.  So if you want
to keep the NSA out of your files, this rotor cipher may well be unsafe, but
for discouraging casual snooping through your files, it will probably be
just fine, and may be somewhat safer than using the \UNIX{} \program{crypt}
command.
\index{NSA}
\index{National Security Agency}\index{crypt(1)}
% XXX How were Unix commands represented in the docs?

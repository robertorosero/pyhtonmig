\chapter{Compound statements\label{compound}}
\indexii{compound}{statement}

Compound statements contain (groups of) other statements; they affect
or control the execution of those other statements in some way.  In
general, compound statements span multiple lines, although in simple
incarnations a whole compound statement may be contained in one line.

The \keyword{if}, \keyword{while} and \keyword{for} statements implement
traditional control flow constructs.  \keyword{try} specifies exception
handlers and/or cleanup code for a group of statements.  Function and
class definitions are also syntactically compound statements.

Compound statements consist of one or more `clauses.'  A clause
consists of a header and a `suite.'  The clause headers of a
particular compound statement are all at the same indentation level.
Each clause header begins with a uniquely identifying keyword and ends
with a colon.  A suite is a group of statements controlled by a
clause.  A suite can be one or more semicolon-separated simple
statements on the same line as the header, following the header's
colon, or it can be one or more indented statements on subsequent
lines.  Only the latter form of suite can contain nested compound
statements; the following is illegal, mostly because it wouldn't be
clear to which \keyword{if} clause a following \keyword{else} clause would
belong:
\index{clause}
\index{suite}

\begin{verbatim}
if test1: if test2: print x
\end{verbatim}

Also note that the semicolon binds tighter than the colon in this
context, so that in the following example, either all or none of the
\keyword{print} statements are executed:

\begin{verbatim}
if x < y < z: print x; print y; print z
\end{verbatim}

Summarizing:

\begin{productionlist}
  \production{compound_stmt}
             {\token{if_stmt}}
  \productioncont{| \token{while_stmt}}
  \productioncont{| \token{for_stmt}}
  \productioncont{| \token{try_stmt}}
  \productioncont{| \token{funcdef}}
  \productioncont{| \token{classdef}}
  \production{suite}
             {\token{stmt_list} NEWLINE
              | NEWLINE INDENT \token{statement}+ DEDENT}
  \production{statement}
             {\token{stmt_list} NEWLINE | \token{compound_stmt}}
  \production{stmt_list}
             {\token{simple_stmt} (";" \token{simple_stmt})* [";"]}
\end{productionlist}

Note that statements always end in a
\code{NEWLINE}\index{NEWLINE token} possibly followed by a
\code{DEDENT}.\index{DEDENT token} Also note that optional
continuation clauses always begin with a keyword that cannot start a
statement, thus there are no ambiguities (the `dangling
\keyword{else}' problem is solved in Python by requiring nested
\keyword{if} statements to be indented).
\indexii{dangling}{else}

The formatting of the grammar rules in the following sections places
each clause on a separate line for clarity.


\section{The \keyword{if} statement\label{if}}
\stindex{if}

The \keyword{if} statement is used for conditional execution:

\begin{productionlist}
  \production{if_stmt}
             {"if" \token{expression} ":" \token{suite}}
  \productioncont{( "elif" \token{expression} ":" \token{suite} )*}
  \productioncont{["else" ":" \token{suite}]}
\end{productionlist}

It selects exactly one of the suites by evaluating the expressions one
by one until one is found to be true (see section~\ref{Booleans} for
the definition of true and false); then that suite is executed (and no
other part of the \keyword{if} statement is executed or evaluated).  If
all expressions are false, the suite of the \keyword{else} clause, if
present, is executed.
\kwindex{elif}
\kwindex{else}


\section{The \keyword{while} statement\label{while}}
\stindex{while}
\indexii{loop}{statement}

The \keyword{while} statement is used for repeated execution as long
as an expression is true:

\begin{productionlist}
  \production{while_stmt}
             {"while" \token{expression} ":" \token{suite}}
  \productioncont{["else" ":" \token{suite}]}
\end{productionlist}

This repeatedly tests the expression and, if it is true, executes the
first suite; if the expression is false (which may be the first time it
is tested) the suite of the \keyword{else} clause, if present, is
executed and the loop terminates.
\kwindex{else}

A \keyword{break} statement executed in the first suite terminates the
loop without executing the \keyword{else} clause's suite.  A
\keyword{continue} statement executed in the first suite skips the rest
of the suite and goes back to testing the expression.
\stindex{break}
\stindex{continue}


\section{The \keyword{for} statement\label{for}}
\stindex{for}
\indexii{loop}{statement}

The \keyword{for} statement is used to iterate over the elements of a
sequence (such as a string, tuple or list) or other iterable object:
\obindex{sequence}

\begin{productionlist}
  \production{for_stmt}
             {"for" \token{target_list} "in" \token{expression_list}
              ":" \token{suite}}
  \productioncont{["else" ":" \token{suite}]}
\end{productionlist}

The expression list is evaluated once; it should yield a sequence.  The
suite is then executed once for each item in the sequence, in the
order of ascending indices.  Each item in turn is assigned to the
target list using the standard rules for assignments, and then the
suite is executed.  When the items are exhausted (which is immediately
when the sequence is empty), the suite in the \keyword{else} clause, if
present, is executed, and the loop terminates.
\kwindex{in}
\kwindex{else}
\indexii{target}{list}

A \keyword{break} statement executed in the first suite terminates the
loop without executing the \keyword{else} clause's suite.  A
\keyword{continue} statement executed in the first suite skips the rest
of the suite and continues with the next item, or with the \keyword{else}
clause if there was no next item.
\stindex{break}
\stindex{continue}

The suite may assign to the variable(s) in the target list; this does
not affect the next item assigned to it.

The target list is not deleted when the loop is finished, but if the
sequence is empty, it will not have been assigned to at all by the
loop.  Hint: the built-in function \function{range()} returns a
sequence of integers suitable to emulate the effect of Pascal's
\code{for i := a to b do};
e.g., \code{range(3)} returns the list \code{[0, 1, 2]}.
\bifuncindex{range}
\indexii{Pascal}{language}

\warning{There is a subtlety when the sequence is being modified
by the loop (this can only occur for mutable sequences, i.e. lists).
An internal counter is used to keep track of which item is used next,
and this is incremented on each iteration.  When this counter has
reached the length of the sequence the loop terminates.  This means that
if the suite deletes the current (or a previous) item from the
sequence, the next item will be skipped (since it gets the index of
the current item which has already been treated).  Likewise, if the
suite inserts an item in the sequence before the current item, the
current item will be treated again the next time through the loop.
This can lead to nasty bugs that can be avoided by making a temporary
copy using a slice of the whole sequence, e.g.,
\index{loop!over mutable sequence}
\index{mutable sequence!loop over}}

\begin{verbatim}
for x in a[:]:
    if x < 0: a.remove(x)
\end{verbatim}


\section{The \keyword{try} statement\label{try}}
\stindex{try}

The \keyword{try} statement specifies exception handlers and/or cleanup
code for a group of statements:

\begin{productionlist}
  \production{try_stmt}
             {\token{try_exc_stmt} | \token{try_fin_stmt}}
  \production{try_exc_stmt}
             {"try" ":" \token{suite}}
  \productioncont{("except" [\token{expression}
                             ["," \token{target}]] ":" \token{suite})+}
  \productioncont{["else" ":" \token{suite}]}
  \production{try_fin_stmt}
             {"try" ":" \token{suite}
              "finally" ":" \token{suite}}
\end{productionlist}

There are two forms of \keyword{try} statement:
\keyword{try}...\keyword{except} and
\keyword{try}...\keyword{finally}.  These forms cannot be mixed (but
they can be nested in each other).

The \keyword{try}...\keyword{except} form specifies one or more
exception handlers
(the \keyword{except} clauses).  When no exception occurs in the
\keyword{try} clause, no exception handler is executed.  When an
exception occurs in the \keyword{try} suite, a search for an exception
handler is started.  This search inspects the except clauses in turn until
one is found that matches the exception.  An expression-less except
clause, if present, must be last; it matches any exception.  For an
except clause with an expression, that expression is evaluated, and the
clause matches the exception if the resulting object is ``compatible''
with the exception.  An object is compatible with an exception if it
is either the object that identifies the exception, or (for exceptions
that are classes) it is a base class of the exception, or it is a
tuple containing an item that is compatible with the exception.  Note
that the object identities must match, i.e. it must be the same
object, not just an object with the same value.
\kwindex{except}

If no except clause matches the exception, the search for an exception
handler continues in the surrounding code and on the invocation stack.

If the evaluation of an expression in the header of an except clause
raises an exception, the original search for a handler is canceled
and a search starts for the new exception in the surrounding code and
on the call stack (it is treated as if the entire \keyword{try} statement
raised the exception).

When a matching except clause is found, the exception's parameter is
assigned to the target specified in that except clause, if present,
and the except clause's suite is executed.  All except clauses must
have an executable block.  When the end of this block
is reached, execution continues normally after the entire try
statement.  (This means that if two nested handlers exist for the same
exception, and the exception occurs in the try clause of the inner
handler, the outer handler will not handle the exception.)

Before an except clause's suite is executed, details about the
exception are assigned to three variables in the
\module{sys}\refbimodindex{sys} module: \code{sys.exc_type} receives
the object identifying the exception; \code{sys.exc_value} receives
the exception's parameter; \code{sys.exc_traceback} receives a
traceback object\obindex{traceback} (see section~\ref{traceback})
identifying the point in the program where the exception occurred.
These details are also available through the \function{sys.exc_info()}
function, which returns a tuple \code{(\var{exc_type}, \var{exc_value},
\var{exc_traceback})}.  Use of the corresponding variables is
deprecated in favor of this function, since their use is unsafe in a
threaded program.  As of Python 1.5, the variables are restored to
their previous values (before the call) when returning from a function
that handled an exception.
\withsubitem{(in module sys)}{\ttindex{exc_type}
  \ttindex{exc_value}\ttindex{exc_traceback}}

The optional \keyword{else} clause is executed if and when control
flows off the end of the \keyword{try} clause.\footnote{
  Currently, control ``flows off the end'' except in the case of an
  exception or the execution of a \keyword{return},
  \keyword{continue}, or \keyword{break} statement.
} Exceptions in the \keyword{else} clause are not handled by the
preceding \keyword{except} clauses.
\kwindex{else}
\stindex{return}
\stindex{break}
\stindex{continue}

The \keyword{try}...\keyword{finally} form specifies a `cleanup' handler.  The
\keyword{try} clause is executed.  When no exception occurs, the
\keyword{finally} clause is executed.  When an exception occurs in the
\keyword{try} clause, the exception is temporarily saved, the
\keyword{finally} clause is executed, and then the saved exception is
re-raised.  If the \keyword{finally} clause raises another exception or
executes a \keyword{return} or \keyword{break} statement, the saved
exception is lost.  A \keyword{continue} statement is illegal in the
\keyword{finally} clause.  (The reason is a problem with the current
implementation -- this restriction may be lifted in the future).  The
exception information is not available to the program during execution of
the \keyword{finally} clause.
\kwindex{finally}

When a \keyword{return}, \keyword{break} or \keyword{continue} statement is
executed in the \keyword{try} suite of a \keyword{try}...\keyword{finally}
statement, the \keyword{finally} clause is also executed `on the way out.' A
\keyword{continue} statement is illegal in the \keyword{finally} clause.
(The reason is a problem with the current implementation --- this
restriction may be lifted in the future).
\stindex{return}
\stindex{break}
\stindex{continue}

Additional information on exceptions can be found in
section~\ref{exceptions}, and information on using the \keyword{raise}
statement to generate exceptions may be found in section~\ref{raise}.


\section{Function definitions\label{function}}
\indexii{function}{definition}
\stindex{def}

A function definition defines a user-defined function object (see
section~\ref{types}):
\obindex{user-defined function}
\obindex{function}

\begin{productionlist}
  \production{funcdef}
             {"def" \token{funcname} "(" [\token{parameter_list}] ")"
              ":" \token{suite}}
  \production{parameter_list}
             {(\token{defparameter} ",")*}
  \productioncont{("*" \token{identifier} [, "**" \token{identifier}]}
  \productioncont{| "**" \token{identifier}
                  | \token{defparameter} [","])}
  \production{defparameter}
             {\token{parameter} ["=" \token{expression}]}
  \production{sublist}
             {\token{parameter} ("," \token{parameter})* [","]}
  \production{parameter}
             {\token{identifier} | "(" \token{sublist} ")"}
  \production{funcname}
             {\token{identifier}}
\end{productionlist}

A function definition is an executable statement.  Its execution binds
the function name in the current local namespace to a function object
(a wrapper around the executable code for the function).  This
function object contains a reference to the current global namespace
as the global namespace to be used when the function is called.
\indexii{function}{name}
\indexii{name}{binding}

The function definition does not execute the function body; this gets
executed only when the function is called.

When one or more top-level parameters have the form \var{parameter}
\code{=} \var{expression}, the function is said to have ``default
parameter values.''  For a parameter with a
default value, the corresponding argument may be omitted from a call,
in which case the parameter's default value is substituted.  If a
parameter has a default value, all following parameters must also have
a default value --- this is a syntactic restriction that is not
expressed by the grammar.
\indexiii{default}{parameter}{value}

\strong{Default parameter values are evaluated when the function
definition is executed.}  This means that the expression is evaluated
once, when the function is defined, and that that same
``pre-computed'' value is used for each call.  This is especially
important to understand when a default parameter is a mutable object,
such as a list or a dictionary: if the function modifies the object
(e.g. by appending an item to a list), the default value is in effect
modified.  This is generally not what was intended.  A way around this 
is to use \code{None} as the default, and explicitly test for it in
the body of the function, e.g.:

\begin{verbatim}
def whats_on_the_telly(penguin=None):
    if penguin is None:
        penguin = []
    penguin.append("property of the zoo")
    return penguin
\end{verbatim}

Function call semantics are described in more detail in
section~\ref{calls}.
A function call always assigns values to all parameters mentioned in
the parameter list, either from position arguments, from keyword
arguments, or from default values.  If the form ``\code{*identifier}''
is present, it is initialized to a tuple receiving any excess
positional parameters, defaulting to the empty tuple.  If the form
``\code{**identifier}'' is present, it is initialized to a new
dictionary receiving any excess keyword arguments, defaulting to a
new empty dictionary.

It is also possible to create anonymous functions (functions not bound
to a name), for immediate use in expressions.  This uses lambda forms,
described in section~\ref{lambda}.  Note that the lambda form is
merely a shorthand for a simplified function definition; a function
defined in a ``\keyword{def}'' statement can be passed around or
assigned to another name just like a function defined by a lambda
form.  The ``\keyword{def}'' form is actually more powerful since it
allows the execution of multiple statements.
\indexii{lambda}{form}

\strong{Programmer's note:} Functions are first-class objects.  A
``\code{def}'' form executed inside a function definition defines a
local function that can be returned or passed around.  Free variables
used in the nested function can access the local variables of the
function containing the def.  See section~\ref{naming} for details.


\section{Class definitions\label{class}}
\indexii{class}{definition}
\stindex{class}

A class definition defines a class object (see section~\ref{types}):
\obindex{class}

\begin{productionlist}
  \production{classdef}
             {"class" \token{classname} [\token{inheritance}] ":"
              \token{suite}}
  \production{inheritance}
             {"(" [\token{expression_list}] ")"}
  \production{classname}
             {\token{identifier}}
\end{productionlist}

A class definition is an executable statement.  It first evaluates the
inheritance list, if present.  Each item in the inheritance list
should evaluate to a class object or class type which allows
subclassing.  The class's suite is then executed
in a new execution frame (see section~\ref{naming}), using a newly
created local namespace and the original global namespace.
(Usually, the suite contains only function definitions.)  When the
class's suite finishes execution, its execution frame is discarded but
its local namespace is saved.  A class object is then created using
the inheritance list for the base classes and the saved local
namespace for the attribute dictionary.  The class name is bound to this
class object in the original local namespace.
\index{inheritance}
\indexii{class}{name}
\indexii{name}{binding}
\indexii{execution}{frame}

\strong{Programmer's note:} Variables defined in the class definition
are class variables; they are shared by all instances.  To define
instance variables, they must be given a value in the
\method{__init__()} method or in another method.  Both class and
instance variables are accessible through the notation
``\code{self.name}'', and an instance variable hides a class variable
with the same name when accessed in this way.  Class variables with
immutable values can be used as defaults for instance variables.
For new-style classes, descriptors can be used to create instance
variables with different implementation details.

\chapter{Future statements and nested scopes \label{futures}}
\sectionauthor{Jeremy Hylton}{jeremy@alum.mit.edu}


The semantics of Python's static scoping will change in version 2.2 to
support resolution of unbound local names in enclosing functions'
namespaces.  The new semantics will be available in Python 2.1 through
the use of a ``future'' statement.  This appendix documents these two
features for Python 2.1; it will be removed in Python 2.2 and the
features will be documented in the main sections of this manual.


\section{Future statements \label{future-statements}}

A \dfn{future statement}\indexii{future}{statement} is a directive to
the compiler that a particular module should be compiled using syntax
or semantics that will be available in a specified future release of
Python.  The future statement is intended to ease migration to future
versions of Python that introduce incompatible changes to the
language.  It allows use of the new features on a per-module basis
before the release in which the feature becomes standard.

\begin{verbatim}
future_statement: "from" "__future__" "import" feature ["as" name]
                 ("," feature ["as" name])*

feature: identifier
name: identifier
\end{verbatim}

A future statement must appear near the top of the module.  The only
lines that can appear before a future statement are:

\begin{itemize}

\item the module docstring (if any),
\item comments,
\item blank lines, and
\item other future statements.

\end{itemize}

The only feature recognized by Python 2.1 is \samp{nested_scopes}.

A future statement is recognized and treated specially at compile time:
Changes to the semantics of core constructs are often implemented by
generating different code.  It may even be the case that a new feature
introduces new incompatible syntax (such as a new reserved word), in
which case the compiler may need to parse the module differently.  Such
decisions cannot be pushed off until runtime.

For any given release, the compiler knows which feature names have been
defined, and raises a compile-time error if a future statement contains
a feature not known to it.

The direct runtime semantics are the same as for any import statement:
there is a standard module \module{__future__}, described later, and
it will be imported in the usual way at the time the future statement
is executed.

The interesting runtime semantics depend on the specific feature
enabled by the future statement.

Note that there is nothing special about the statement:

\begin{verbatim}
import __future__ [as name]
\end{verbatim}

That is not a future statement; it's an ordinary import statement with
no special semantics or syntax restrictions.

Code compiled by an exec statement or calls to the builtin functions
\function{compile()} and \function{execfile()} that occur in a module
\module{M} containing a future statement will use the new syntax or
semantics associated with the future statement.

A future statement typed at an interactive interpreter prompt will
take effect for the rest of the interpreter session.  If an
interpreter is started with the \programopt{-i} option, is passed a
script name to execute, and the script includes a future statement, it
will be in effect in the interactive session started after the script
is executed.

\section{\module{__future__} ---
	 Future statement definitions}

\declaremodule[future]{standard}{__future__}
\modulesynopsis{Future statement definitions}

\module{__future__} is a real module, and serves three purposes:

\begin{itemize}

\item To avoid confusing existing tools that analyze import statements
      and expect to find the modules they're importing.

\item To ensure that future_statements run under releases prior to 2.1
      at least yield runtime exceptions (the import of
      \module{__future__} will fail, because there was no module of
      that name prior to 2.1). 

\item To document when incompatible changes were introduced, and when they
      will be --- or were --- made mandatory.  This is a form of executable
      documentation, and can be inspected programatically via importing
      \module{__future__} and examining its contents.

\end{itemize}

Each statment in \file{__future__.py} is of the form:

\begin{verbatim}
FeatureName = "_Feature(" OptionalRelease "," MandatoryRelease ")"
\end{verbatim}

where, normally, OptionalRelease is less then MandatoryRelease, and
both are 5-tuples of the same form as \code{sys.version_info}:

\begin{verbatim}
    (PY_MAJOR_VERSION, # the 2 in 2.1.0a3; an int
     PY_MINOR_VERSION, # the 1; an int
     PY_MICRO_VERSION, # the 0; an int
     PY_RELEASE_LEVEL, # "alpha", "beta", "candidate" or "final"; string
     PY_RELEASE_SERIAL # the 3; an int
    )
\end{verbatim}

OptionalRelease records the first release in which the feature was
accepted. 

In the case of MandatoryReleases that have not yet occurred,
MandatoryRelease predicts the release in which the feature will become
part of the language.

Else MandatoryRelease records when the feature became part of the
language; in releases at or after that, modules no longer need a
future statement to use the feature in question, but may continue to
use such imports. 

MandatoryRelease may also be \code{None}, meaning that a planned
feature got dropped.

Instances of class \class{_Feature} have two corresponding methods,
\method{getOptionalRelease()} and \method{getMandatoryRelease()}.

No feature description will ever be deleted from \module{__future__}.


\section{Nested scopes \label{nested-scopes}}

\indexii{nested}{scopes}

Nested scopes are left as an exercise for the reader.

% This document is largely a stub used to allow the email package docs
% to be formatted separately from the rest of the Python
% documentation.  This allows the documentation to be released
% independently of the rest of Python since the email package is being
% maintained for multiple Python versions, and on an accelerated
% schedule.

\documentclass{howto}

\title{email Package Reference}
\author{Barry Warsaw}
\authoraddress{\email{barry@python.org}}

\date{\today}
\release{3.0}			% software release, not documentation
\setreleaseinfo{}		% empty for final release
\setshortversion{3.0}		% major.minor only for software

\begin{document}

\maketitle

\begin{abstract}
  The \module{email} package provides classes and utilities to create,
  parse, generate, and modify email messages, conforming to all the
  relevant email and MIME related RFCs.
\end{abstract}

% The ugly "%begin{latexonly}" pseudo-environment suppresses the table
% of contents for HTML generation.
%
%begin{latexonly}
\tableofcontents
%end{latexonly}

\section{Introduction}
The \module{email} package provides classes and utilities to create,
parse, generate, and modify email messages, conforming to all the
relevant email and MIME related RFCs.

This document describes version 3.0 of the \module{email} package, which is
distributed with Python 2.4 and is available as a standalone distutils-based
package for use with Python 2.3.  \module{email} 3.0 is not compatible with
Python versions earlier than 2.3.  For more information about the
\module{email} package, including download links and mailing lists, see
\ulink{Python's email SIG}{http://www.python.org/sigs/email-sig}.

The documentation that follows was written for the Python project, so
if you're reading this as part of the standalone \module{email}
package documentation, there are a few notes to be aware of:

\begin{itemize}
\item Deprecation and ``version added'' notes are relative to the
      Python version a feature was added or deprecated.

\item If you're reading this documentation as part of the
      standalone \module{email} package, some of the internal links to
      other sections of the Python standard library may not resolve.

\end{itemize}

% Copyright (C) 2001 Python Software Foundation
% Author: barry@zope.com (Barry Warsaw)

\section{\module{email} ---
	 An email and MIME handling package}

\declaremodule{standard}{email}
\modulesynopsis{Package supporting the parsing, manipulating, and
    generating email messages, including MIME documents.}
\moduleauthor{Barry A. Warsaw}{barry@zope.com}

\versionadded{2.2}

The \module{email} package is a library for managing email messages,
including MIME and other \rfc{2822}-based message documents.  It
subsumes most of the functionality in several older standard modules
such as \refmodule{rfc822}, \refmodule{mimetools},
\refmodule{multifile}, and other non-standard packages such as
\module{mimecntl}.

The primary distinguishing feature of the \module{email} package is
that it splits the parsing and generating of email messages from the
internal \emph{object model} representation of email.  Applications
using the \module{email} package deal primarily with objects; you can
add sub-objects to messages, remove sub-objects from messages,
completely re-arrange the contents, etc.  There is a separate parser
and a separate generator which handles the transformation from flat
text to the object module, and then back to flat text again.  There
are also handy subclasses for some common MIME object types, and a few
miscellaneous utilities that help with such common tasks as extracting
and parsing message field values, creating RFC-compliant dates, etc.

The following sections describe the functionality of the
\module{email} package.  The ordering follows a progression that
should be common in applications: an email message is read as flat
text from a file or other source, the text is parsed to produce an
object model representation of the email message, this model is
manipulated, and finally the model is rendered back into
flat text.

It is perfectly feasible to create the object model out of whole cloth
--- i.e. completely from scratch.  From there, a similar progression
can be taken as above.  

Also included are detailed specifications of all the classes and
modules that the \module{email} package provides, the exception
classes you might encounter while using the \module{email} package,
some auxiliary utilities, and a few examples.  For users of the older
\module{mimelib} package, from which the \module{email} package is
descendent, a section on differences and porting is provided.

\subsection{Representing an email message}
\declaremodule{standard}{email.Message}
\modulesynopsis{The base class representing email messages.}

The central class in the \module{email} package is the
\class{Message} class; it is the base class for the \module{email}
object model.  \class{Message} provides the core functionality for
setting and querying header fields, and for accessing message bodies.

Conceptually, a \class{Message} object consists of \emph{headers} and
\emph{payloads}.  Headers are \rfc{2822} style field names and
values where the field name and value are separated by a colon.  The
colon is not part of either the field name or the field value.

Headers are stored and returned in case-preserving form but are
matched case-insensitively.  There may also be a single envelope
header, also known as the \emph{Unix-From} header or the
\code{From_} header.  The payload is either a string in the case of
simple message objects or a list of \class{Message} objects for
MIME container documents (e.g. \mimetype{multipart/*} and
\mimetype{message/rfc822}).

\class{Message} objects provide a mapping style interface for
accessing the message headers, and an explicit interface for accessing
both the headers and the payload.  It provides convenience methods for
generating a flat text representation of the message object tree, for
accessing commonly used header parameters, and for recursively walking
over the object tree.

Here are the methods of the \class{Message} class:

\begin{classdesc}{Message}{}
The constructor takes no arguments.
\end{classdesc}

\begin{methoddesc}[Message]{as_string}{\optional{unixfrom}}
Return the entire message flatten as a string.  When optional
\var{unixfrom} is \code{True}, the envelope header is included in the
returned string.  \var{unixfrom} defaults to \code{False}.
\end{methoddesc}

\begin{methoddesc}[Message]{__str__}{}
Equivalent to \method{as_string(unixfrom=True)}.
\end{methoddesc}

\begin{methoddesc}[Message]{is_multipart}{}
Return \code{True} if the message's payload is a list of
sub-\class{Message} objects, otherwise return \code{False}.  When
\method{is_multipart()} returns False, the payload should be a string
object.
\end{methoddesc}

\begin{methoddesc}[Message]{set_unixfrom}{unixfrom}
Set the message's envelope header to \var{unixfrom}, which should be a string.
\end{methoddesc}

\begin{methoddesc}[Message]{get_unixfrom}{}
Return the message's envelope header.  Defaults to \code{None} if the
envelope header was never set.
\end{methoddesc}

\begin{methoddesc}[Message]{attach}{payload}
Add the given \var{payload} to the current payload, which must be
\code{None} or a list of \class{Message} objects before the call.
After the call, the payload will always be a list of \class{Message}
objects.  If you want to set the payload to a scalar object (e.g. a
string), use \method{set_payload()} instead.
\end{methoddesc}

\begin{methoddesc}[Message]{get_payload}{\optional{i\optional{, decode}}}
Return a reference the current payload, which will be a list of
\class{Message} objects when \method{is_multipart()} is \code{True}, or a
string when \method{is_multipart()} is \code{False}.  If the
payload is a list and you mutate the list object, you modify the
message's payload in place.

With optional argument \var{i}, \method{get_payload()} will return the
\var{i}-th element of the payload, counting from zero, if
\method{is_multipart()} is \code{True}.  An \exception{IndexError}
will be raised if \var{i} is less than 0 or greater than or equal to
the number of items in the payload.  If the payload is a string
(i.e. \method{is_multipart()} is \code{False}) and \var{i} is given, a
\exception{TypeError} is raised.

Optional \var{decode} is a flag indicating whether the payload should be
decoded or not, according to the \mailheader{Content-Transfer-Encoding} header.
When \code{True} and the message is not a multipart, the payload will be
decoded if this header's value is \samp{quoted-printable} or
\samp{base64}.  If some other encoding is used, or
\mailheader{Content-Transfer-Encoding} header is
missing, the payload is returned as-is (undecoded).  If the message is
a multipart and the \var{decode} flag is \code{True}, then \code{None} is
returned.  The default for \var{decode} is \code{False}.
\end{methoddesc}

\begin{methoddesc}[Message]{set_payload}{payload\optional{, charset}}
Set the entire message object's payload to \var{payload}.  It is the
client's responsibility to ensure the payload invariants.  Optional
\var{charset} sets the message's default character set; see
\method{set_charset()} for details.

\versionchanged[\var{charset} argument added]{2.2.2}
\end{methoddesc}

\begin{methoddesc}[Message]{set_charset}{charset}
Set the character set of the payload to \var{charset}, which can
either be a \class{Charset} instance (see \refmodule{email.Charset}), a
string naming a character set,
or \code{None}.  If it is a string, it will be converted to a
\class{Charset} instance.  If \var{charset} is \code{None}, the
\code{charset} parameter will be removed from the
\mailheader{Content-Type} header. Anything else will generate a
\exception{TypeError}.

The message will be assumed to be of type \mimetype{text/*} encoded with
\code{charset.input_charset}.  It will be converted to
\code{charset.output_charset}
and encoded properly, if needed, when generating the plain text
representation of the message.  MIME headers
(\mailheader{MIME-Version}, \mailheader{Content-Type},
\mailheader{Content-Transfer-Encoding}) will be added as needed.

\versionadded{2.2.2}
\end{methoddesc}

\begin{methoddesc}[Message]{get_charset}{}
Return the \class{Charset} instance associated with the message's payload.
\versionadded{2.2.2}
\end{methoddesc}

The following methods implement a mapping-like interface for accessing
the message's \rfc{2822} headers.  Note that there are some
semantic differences between these methods and a normal mapping
(i.e. dictionary) interface.  For example, in a dictionary there are
no duplicate keys, but here there may be duplicate message headers.  Also,
in dictionaries there is no guaranteed order to the keys returned by
\method{keys()}, but in a \class{Message} object, headers are always
returned in the order they appeared in the original message, or were
added to the message later.  Any header deleted and then re-added are
always appended to the end of the header list.

These semantic differences are intentional and are biased toward
maximal convenience.

Note that in all cases, any envelope header present in the message is
not included in the mapping interface.

\begin{methoddesc}[Message]{__len__}{}
Return the total number of headers, including duplicates.
\end{methoddesc}

\begin{methoddesc}[Message]{__contains__}{name}
Return true if the message object has a field named \var{name}.
Matching is done case-insensitively and \var{name} should not include the
trailing colon.  Used for the \code{in} operator,
e.g.:

\begin{verbatim}
if 'message-id' in myMessage:
    print 'Message-ID:', myMessage['message-id']
\end{verbatim}
\end{methoddesc}

\begin{methoddesc}[Message]{__getitem__}{name}
Return the value of the named header field.  \var{name} should not
include the colon field separator.  If the header is missing,
\code{None} is returned; a \exception{KeyError} is never raised.

Note that if the named field appears more than once in the message's
headers, exactly which of those field values will be returned is
undefined.  Use the \method{get_all()} method to get the values of all
the extant named headers.
\end{methoddesc}

\begin{methoddesc}[Message]{__setitem__}{name, val}
Add a header to the message with field name \var{name} and value
\var{val}.  The field is appended to the end of the message's existing
fields.

Note that this does \emph{not} overwrite or delete any existing header
with the same name.  If you want to ensure that the new header is the
only one present in the message with field name
\var{name}, delete the field first, e.g.:

\begin{verbatim}
del msg['subject']
msg['subject'] = 'Python roolz!'
\end{verbatim}
\end{methoddesc}

\begin{methoddesc}[Message]{__delitem__}{name}
Delete all occurrences of the field with name \var{name} from the
message's headers.  No exception is raised if the named field isn't
present in the headers.
\end{methoddesc}

\begin{methoddesc}[Message]{has_key}{name}
Return true if the message contains a header field named \var{name},
otherwise return false.
\end{methoddesc}

\begin{methoddesc}[Message]{keys}{}
Return a list of all the message's header field names.
\end{methoddesc}

\begin{methoddesc}[Message]{values}{}
Return a list of all the message's field values.
\end{methoddesc}

\begin{methoddesc}[Message]{items}{}
Return a list of 2-tuples containing all the message's field headers
and values.
\end{methoddesc}

\begin{methoddesc}[Message]{get}{name\optional{, failobj}}
Return the value of the named header field.  This is identical to
\method{__getitem__()} except that optional \var{failobj} is returned
if the named header is missing (defaults to \code{None}).
\end{methoddesc}

Here are some additional useful methods:

\begin{methoddesc}[Message]{get_all}{name\optional{, failobj}}
Return a list of all the values for the field named \var{name}.
If there are no such named headers in the message, \var{failobj} is
returned (defaults to \code{None}).
\end{methoddesc}

\begin{methoddesc}[Message]{add_header}{_name, _value, **_params}
Extended header setting.  This method is similar to
\method{__setitem__()} except that additional header parameters can be
provided as keyword arguments.  \var{_name} is the header field to add
and \var{_value} is the \emph{primary} value for the header.

For each item in the keyword argument dictionary \var{_params}, the
key is taken as the parameter name, with underscores converted to
dashes (since dashes are illegal in Python identifiers).  Normally,
the parameter will be added as \code{key="value"} unless the value is
\code{None}, in which case only the key will be added.

Here's an example:

\begin{verbatim}
msg.add_header('Content-Disposition', 'attachment', filename='bud.gif')
\end{verbatim}

This will add a header that looks like

\begin{verbatim}
Content-Disposition: attachment; filename="bud.gif"
\end{verbatim}
\end{methoddesc}

\begin{methoddesc}[Message]{replace_header}{_name, _value}
Replace a header.  Replace the first header found in the message that
matches \var{_name}, retaining header order and field name case.  If
no matching header was found, a \exception{KeyError} is raised.

\versionadded{2.2.2}
\end{methoddesc}

\begin{methoddesc}[Message]{get_content_type}{}
Return the message's content type.  The returned string is coerced to
lower case of the form \mimetype{maintype/subtype}.  If there was no
\mailheader{Content-Type} header in the message the default type as
given by \method{get_default_type()} will be returned.  Since
according to \rfc{2045}, messages always have a default type,
\method{get_content_type()} will always return a value.

\rfc{2045} defines a message's default type to be
\mimetype{text/plain} unless it appears inside a
\mimetype{multipart/digest} container, in which case it would be
\mimetype{message/rfc822}.  If the \mailheader{Content-Type} header
has an invalid type specification, \rfc{2045} mandates that the
default type be \mimetype{text/plain}.

\versionadded{2.2.2}
\end{methoddesc}

\begin{methoddesc}[Message]{get_content_maintype}{}
Return the message's main content type.  This is the
\mimetype{maintype} part of the string returned by
\method{get_content_type()}.

\versionadded{2.2.2}
\end{methoddesc}

\begin{methoddesc}[Message]{get_content_subtype}{}
Return the message's sub-content type.  This is the \mimetype{subtype}
part of the string returned by \method{get_content_type()}.

\versionadded{2.2.2}
\end{methoddesc}

\begin{methoddesc}[Message]{get_default_type}{}
Return the default content type.  Most messages have a default content
type of \mimetype{text/plain}, except for messages that are subparts
of \mimetype{multipart/digest} containers.  Such subparts have a
default content type of \mimetype{message/rfc822}.

\versionadded{2.2.2}
\end{methoddesc}

\begin{methoddesc}[Message]{set_default_type}{ctype}
Set the default content type.  \var{ctype} should either be
\mimetype{text/plain} or \mimetype{message/rfc822}, although this is
not enforced.  The default content type is not stored in the
\mailheader{Content-Type} header.

\versionadded{2.2.2}
\end{methoddesc}

\begin{methoddesc}[Message]{get_params}{\optional{failobj\optional{,
    header\optional{, unquote}}}}
Return the message's \mailheader{Content-Type} parameters, as a list.  The
elements of the returned list are 2-tuples of key/value pairs, as
split on the \character{=} sign.  The left hand side of the
\character{=} is the key, while the right hand side is the value.  If
there is no \character{=} sign in the parameter the value is the empty
string, otherwise the value is as described in \method{get_param()} and is
unquoted if optional \var{unquote} is \code{True} (the default).

Optional \var{failobj} is the object to return if there is no
\mailheader{Content-Type} header.  Optional \var{header} is the header to
search instead of \mailheader{Content-Type}.

\versionchanged[\var{unquote} argument added]{2.2.2}
\end{methoddesc}

\begin{methoddesc}[Message]{get_param}{param\optional{,
    failobj\optional{, header\optional{, unquote}}}}
Return the value of the \mailheader{Content-Type} header's parameter
\var{param} as a string.  If the message has no \mailheader{Content-Type}
header or if there is no such parameter, then \var{failobj} is
returned (defaults to \code{None}).

Optional \var{header} if given, specifies the message header to use
instead of \mailheader{Content-Type}.

Parameter keys are always compared case insensitively.  The return
value can either be a string, or a 3-tuple if the parameter was
\rfc{2231} encoded.  When it's a 3-tuple, the elements of the value are of
the form \code{(CHARSET, LANGUAGE, VALUE)}, where \code{LANGUAGE} may
be the empty string.  Your application should be prepared to deal with
3-tuple return values, which it can convert to a Unicode string like
so:

\begin{verbatim}
param = msg.get_param('foo')
if isinstance(param, tuple):
    param = unicode(param[2], param[0])
\end{verbatim}

In any case, the parameter value (either the returned string, or the
\code{VALUE} item in the 3-tuple) is always unquoted, unless
\var{unquote} is set to \code{False}.

\versionchanged[\var{unquote} argument added, and 3-tuple return value
possible]{2.2.2}
\end{methoddesc}

\begin{methoddesc}[Message]{set_param}{param, value\optional{,
    header\optional{, requote\optional{, charset\optional{, language}}}}}

Set a parameter in the \mailheader{Content-Type} header.  If the
parameter already exists in the header, its value will be replaced
with \var{value}.  If the \mailheader{Content-Type} header as not yet
been defined for this message, it will be set to \mimetype{text/plain}
and the new parameter value will be appended as per \rfc{2045}.

Optional \var{header} specifies an alternative header to
\mailheader{Content-Type}, and all parameters will be quoted as
necessary unless optional \var{requote} is \code{False} (the default
is \code{True}).

If optional \var{charset} is specified, the parameter will be encoded
according to \rfc{2231}. Optional \var{language} specifies the RFC
2231 language, defaulting to the empty string.  Both \var{charset} and
\var{language} should be strings.

\versionadded{2.2.2}
\end{methoddesc}

\begin{methoddesc}[Message]{del_param}{param\optional{, header\optional{,
    requote}}}
Remove the given parameter completely from the
\mailheader{Content-Type} header.  The header will be re-written in
place without the parameter or its value.  All values will be quoted
as necessary unless \var{requote} is \code{False} (the default is
\code{True}).  Optional \var{header} specifies an alternative to
\mailheader{Content-Type}.

\versionadded{2.2.2}
\end{methoddesc}

\begin{methoddesc}[Message]{set_type}{type\optional{, header}\optional{,
    requote}}
Set the main type and subtype for the \mailheader{Content-Type}
header. \var{type} must be a string in the form
\mimetype{maintype/subtype}, otherwise a \exception{ValueError} is
raised.

This method replaces the \mailheader{Content-Type} header, keeping all
the parameters in place.  If \var{requote} is \code{False}, this
leaves the existing header's quoting as is, otherwise the parameters
will be quoted (the default).

An alternative header can be specified in the \var{header} argument.
When the \mailheader{Content-Type} header is set a
\mailheader{MIME-Version} header is also added.

\versionadded{2.2.2}
\end{methoddesc}

\begin{methoddesc}[Message]{get_filename}{\optional{failobj}}
Return the value of the \code{filename} parameter of the
\mailheader{Content-Disposition} header of the message, or \var{failobj} if
either the header is missing, or has no \code{filename} parameter.
The returned string will always be unquoted as per
\method{Utils.unquote()}.
\end{methoddesc}

\begin{methoddesc}[Message]{get_boundary}{\optional{failobj}}
Return the value of the \code{boundary} parameter of the
\mailheader{Content-Type} header of the message, or \var{failobj} if either
the header is missing, or has no \code{boundary} parameter.  The
returned string will always be unquoted as per
\method{Utils.unquote()}.
\end{methoddesc}

\begin{methoddesc}[Message]{set_boundary}{boundary}
Set the \code{boundary} parameter of the \mailheader{Content-Type}
header to \var{boundary}.  \method{set_boundary()} will always quote
\var{boundary} if necessary.  A \exception{HeaderParseError} is raised
if the message object has no \mailheader{Content-Type} header.

Note that using this method is subtly different than deleting the old
\mailheader{Content-Type} header and adding a new one with the new boundary
via \method{add_header()}, because \method{set_boundary()} preserves the
order of the \mailheader{Content-Type} header in the list of headers.
However, it does \emph{not} preserve any continuation lines which may
have been present in the original \mailheader{Content-Type} header.
\end{methoddesc}

\begin{methoddesc}[Message]{get_content_charset}{\optional{failobj}}
Return the \code{charset} parameter of the \mailheader{Content-Type}
header, coerced to lower case.  If there is no
\mailheader{Content-Type} header, or if that header has no
\code{charset} parameter, \var{failobj} is returned.

Note that this method differs from \method{get_charset()} which
returns the \class{Charset} instance for the default encoding of the
message body.

\versionadded{2.2.2}
\end{methoddesc}

\begin{methoddesc}[Message]{get_charsets}{\optional{failobj}}
Return a list containing the character set names in the message.  If
the message is a \mimetype{multipart}, then the list will contain one
element for each subpart in the payload, otherwise, it will be a list
of length 1.

Each item in the list will be a string which is the value of the
\code{charset} parameter in the \mailheader{Content-Type} header for the
represented subpart.  However, if the subpart has no
\mailheader{Content-Type} header, no \code{charset} parameter, or is not of
the \mimetype{text} main MIME type, then that item in the returned list
will be \var{failobj}.
\end{methoddesc}

\begin{methoddesc}[Message]{walk}{}
The \method{walk()} method is an all-purpose generator which can be
used to iterate over all the parts and subparts of a message object
tree, in depth-first traversal order.  You will typically use
\method{walk()} as the iterator in a \code{for} loop; each
iteration returns the next subpart.

Here's an example that prints the MIME type of every part of a
multipart message structure:

\begin{verbatim}
>>> for part in msg.walk():
>>>     print part.get_content_type()
multipart/report
text/plain
message/delivery-status
text/plain
text/plain
message/rfc822
\end{verbatim}
\end{methoddesc}

\class{Message} objects can also optionally contain two instance
attributes, which can be used when generating the plain text of a MIME
message.

\begin{datadesc}{preamble}
The format of a MIME document allows for some text between the blank
line following the headers, and the first multipart boundary string.
Normally, this text is never visible in a MIME-aware mail reader
because it falls outside the standard MIME armor.  However, when
viewing the raw text of the message, or when viewing the message in a
non-MIME aware reader, this text can become visible.

The \var{preamble} attribute contains this leading extra-armor text
for MIME documents.  When the \class{Parser} discovers some text after
the headers but before the first boundary string, it assigns this text
to the message's \var{preamble} attribute.  When the \class{Generator}
is writing out the plain text representation of a MIME message, and it
finds the message has a \var{preamble} attribute, it will write this
text in the area between the headers and the first boundary.  See
\refmodule{email.Parser} and \refmodule{email.Generator} for details.

Note that if the message object has no preamble, the
\var{preamble} attribute will be \code{None}.
\end{datadesc}

\begin{datadesc}{epilogue}
The \var{epilogue} attribute acts the same way as the \var{preamble}
attribute, except that it contains text that appears between the last
boundary and the end of the message.

One note: when generating the flat text for a \mimetype{multipart}
message that has no \var{epilogue} (using the standard
\class{Generator} class), no newline is added after the closing
boundary line.  If the message object has an \var{epilogue} and its
value does not start with a newline, a newline is printed after the
closing boundary.  This seems a little clumsy, but it makes the most
practical sense.  The upshot is that if you want to ensure that a
newline get printed after your closing \mimetype{multipart} boundary,
set the \var{epilogue} to the empty string.
\end{datadesc}

\subsubsection{Deprecated methods}

The following methods are deprecated in \module{email} version 2.
They are documented here for completeness.

\begin{methoddesc}[Message]{add_payload}{payload}
Add \var{payload} to the message object's existing payload.  If, prior
to calling this method, the object's payload was \code{None}
(i.e. never before set), then after this method is called, the payload
will be the argument \var{payload}.

If the object's payload was already a list
(i.e. \method{is_multipart()} returns 1), then \var{payload} is
appended to the end of the existing payload list.

For any other type of existing payload, \method{add_payload()} will
transform the new payload into a list consisting of the old payload
and \var{payload}, but only if the document is already a MIME
multipart document.  This condition is satisfied if the message's
\mailheader{Content-Type} header's main type is either
\mimetype{multipart}, or there is no \mailheader{Content-Type}
header.  In any other situation,
\exception{MultipartConversionError} is raised.

\deprecated{2.2.2}{Use the \method{attach()} method instead.}
\end{methoddesc}

\begin{methoddesc}[Message]{get_type}{\optional{failobj}}
Return the message's content type, as a string of the form
\mimetype{maintype/subtype} as taken from the
\mailheader{Content-Type} header.
The returned string is coerced to lowercase.

If there is no \mailheader{Content-Type} header in the message,
\var{failobj} is returned (defaults to \code{None}).

\deprecated{2.2.2}{Use the \method{get_content_type()} method instead.}
\end{methoddesc}

\begin{methoddesc}[Message]{get_main_type}{\optional{failobj}}
Return the message's \emph{main} content type.  This essentially returns the
\var{maintype} part of the string returned by \method{get_type()}, with the
same semantics for \var{failobj}.

\deprecated{2.2.2}{Use the \method{get_content_maintype()} method instead.}
\end{methoddesc}

\begin{methoddesc}[Message]{get_subtype}{\optional{failobj}}
Return the message's sub-content type.  This essentially returns the
\var{subtype} part of the string returned by \method{get_type()}, with the
same semantics for \var{failobj}.

\deprecated{2.2.2}{Use the \method{get_content_subtype()} method instead.}
\end{methoddesc}



\subsection{Parsing email messages}
\declaremodule{standard}{email.Parser}
\modulesynopsis{Parse flat text email messages to produce a message
	        object structure.}

Message object structures can be created in one of two ways: they can be
created from whole cloth by instantiating \class{Message} objects and
stringing them together via \method{attach()} and
\method{set_payload()} calls, or they can be created by parsing a flat text
representation of the email message.

The \module{email} package provides a standard parser that understands
most email document structures, including MIME documents.  You can
pass the parser a string or a file object, and the parser will return
to you the root \class{Message} instance of the object structure.  For
simple, non-MIME messages the payload of this root object will likely
be a string containing the text of the message.  For MIME
messages, the root object will return \code{True} from its
\method{is_multipart()} method, and the subparts can be accessed via
the \method{get_payload()} and \method{walk()} methods.

Note that the parser can be extended in limited ways, and of course
you can implement your own parser completely from scratch.  There is
no magical connection between the \module{email} package's bundled
parser and the \class{Message} class, so your custom parser can create
message object trees any way it finds necessary.

The primary parser class is \class{Parser} which parses both the
headers and the payload of the message.  In the case of
\mimetype{multipart} messages, it will recursively parse the body of
the container message.  Two modes of parsing are supported,
\emph{strict} parsing, which will usually reject any non-RFC compliant
message, and \emph{lax} parsing, which attempts to adjust for common
MIME formatting problems.

The \module{email.Parser} module also provides a second class, called
\class{HeaderParser} which can be used if you're only interested in
the headers of the message. \class{HeaderParser} can be much faster in
these situations, since it does not attempt to parse the message body,
instead setting the payload to the raw body as a string.
\class{HeaderParser} has the same API as the \class{Parser} class.

\subsubsection{Parser class API}

\begin{classdesc}{Parser}{\optional{_class\optional{, strict}}}
The constructor for the \class{Parser} class takes an optional
argument \var{_class}.  This must be a callable factory (such as a
function or a class), and it is used whenever a sub-message object
needs to be created.  It defaults to \class{Message} (see
\refmodule{email.Message}).  The factory will be called without
arguments.

The optional \var{strict} flag specifies whether strict or lax parsing
should be performed.  Normally, when things like MIME terminating
boundaries are missing, or when messages contain other formatting
problems, the \class{Parser} will raise a
\exception{MessageParseError}.  However, when lax parsing is enabled,
the \class{Parser} will attempt to work around such broken formatting
to produce a usable message structure (this doesn't mean
\exception{MessageParseError}s are never raised; some ill-formatted
messages just can't be parsed).  The \var{strict} flag defaults to
\code{False} since lax parsing usually provides the most convenient
behavior.

\versionchanged[The \var{strict} flag was added]{2.2.2}
\end{classdesc}

The other public \class{Parser} methods are:

\begin{methoddesc}[Parser]{parse}{fp\optional{, headersonly}}
Read all the data from the file-like object \var{fp}, parse the
resulting text, and return the root message object.  \var{fp} must
support both the \method{readline()} and the \method{read()} methods
on file-like objects.

The text contained in \var{fp} must be formatted as a block of \rfc{2822}
style headers and header continuation lines, optionally preceded by a
envelope header.  The header block is terminated either by the
end of the data or by a blank line.  Following the header block is the
body of the message (which may contain MIME-encoded subparts).

Optional \var{headersonly} is as with the \method{parse()} method.

\versionchanged[The \var{headersonly} flag was added]{2.2.2}
\end{methoddesc}

\begin{methoddesc}[Parser]{parsestr}{text\optional{, headersonly}}
Similar to the \method{parse()} method, except it takes a string
object instead of a file-like object.  Calling this method on a string
is exactly equivalent to wrapping \var{text} in a \class{StringIO}
instance first and calling \method{parse()}.

Optional \var{headersonly} is a flag specifying whether to stop
parsing after reading the headers or not.  The default is \code{False},
meaning it parses the entire contents of the file.

\versionchanged[The \var{headersonly} flag was added]{2.2.2}
\end{methoddesc}

Since creating a message object structure from a string or a file
object is such a common task, two functions are provided as a
convenience.  They are available in the top-level \module{email}
package namespace.

\begin{funcdesc}{message_from_string}{s\optional{, _class\optional{, strict}}}
Return a message object structure from a string.  This is exactly
equivalent to \code{Parser().parsestr(s)}.  Optional \var{_class} and
\var{strict} are interpreted as with the \class{Parser} class constructor.

\versionchanged[The \var{strict} flag was added]{2.2.2}
\end{funcdesc}

\begin{funcdesc}{message_from_file}{fp\optional{, _class\optional{, strict}}}
Return a message object structure tree from an open file object.  This
is exactly equivalent to \code{Parser().parse(fp)}.  Optional
\var{_class} and \var{strict} are interpreted as with the
\class{Parser} class constructor.

\versionchanged[The \var{strict} flag was added]{2.2.2}
\end{funcdesc}

Here's an example of how you might use this at an interactive Python
prompt:

\begin{verbatim}
>>> import email
>>> msg = email.message_from_string(myString)
\end{verbatim}

\subsubsection{Additional notes}

Here are some notes on the parsing semantics:

\begin{itemize}
\item Most non-\mimetype{multipart} type messages are parsed as a single
      message object with a string payload.  These objects will return
      \code{False} for \method{is_multipart()}.  Their
      \method{get_payload()} method will return a string object.
\item All \mimetype{multipart} type messages will be parsed as a
      container message object with a list of sub-message objects for
      their payload.  The outer container message will return
      \code{True} for \method{is_multipart()} and their
      \method{get_payload()} method will return the list of
      \class{Message} subparts.
\item Most messages with a content type of \mimetype{message/*}
      (e.g. \mimetype{message/deliver-status} and
      \mimetype{message/rfc822}) will also be parsed as container
      object containing a list payload of length 1.  Their
      \method{is_multipart()} method will return \code{True}.  The
      single element in the list payload will be a sub-message object.
\end{itemize}


\subsection{Generating MIME documents}
\declaremodule{standard}{email.generator}
\modulesynopsis{Generate flat text email messages from a message structure.}

One of the most common tasks is to generate the flat text of the email
message represented by a message object structure.  You will need to do
this if you want to send your message via the \refmodule{smtplib}
module or the \refmodule{nntplib} module, or print the message on the
console.  Taking a message object structure and producing a flat text
document is the job of the \class{Generator} class.

Again, as with the \refmodule{email.parser} module, you aren't limited
to the functionality of the bundled generator; you could write one
from scratch yourself.  However the bundled generator knows how to
generate most email in a standards-compliant way, should handle MIME
and non-MIME email messages just fine, and is designed so that the
transformation from flat text, to a message structure via the
\class{Parser} class, and back to flat text, is idempotent (the input
is identical to the output).

Here are the public methods of the \class{Generator} class, imported from the
\module{email.generator} module:

\begin{classdesc}{Generator}{outfp\optional{, mangle_from_\optional{,
    maxheaderlen}}}
The constructor for the \class{Generator} class takes a file-like
object called \var{outfp} for an argument.  \var{outfp} must support
the \method{write()} method and be usable as the output file in a
Python extended print statement.

Optional \var{mangle_from_} is a flag that, when \code{True}, puts a
\samp{>} character in front of any line in the body that starts exactly as
\samp{From }, i.e. \code{From} followed by a space at the beginning of the
line.  This is the only guaranteed portable way to avoid having such
lines be mistaken for a Unix mailbox format envelope header separator (see
\ulink{WHY THE CONTENT-LENGTH FORMAT IS BAD}
{http://home.netscape.com/eng/mozilla/2.0/relnotes/demo/content-length.html}
for details).  \var{mangle_from_} defaults to \code{True}, but you
might want to set this to \code{False} if you are not writing Unix
mailbox format files.

Optional \var{maxheaderlen} specifies the longest length for a
non-continued header.  When a header line is longer than
\var{maxheaderlen} (in characters, with tabs expanded to 8 spaces),
the header will be split as defined in the \module{email.header.Header}
class.  Set to zero to disable header wrapping.  The default is 78, as
recommended (but not required) by \rfc{2822}.
\end{classdesc}

The other public \class{Generator} methods are:

\begin{methoddesc}[Generator]{flatten}{msg\optional{, unixfrom}}
Print the textual representation of the message object structure rooted at
\var{msg} to the output file specified when the \class{Generator}
instance was created.  Subparts are visited depth-first and the
resulting text will be properly MIME encoded.

Optional \var{unixfrom} is a flag that forces the printing of the
envelope header delimiter before the first \rfc{2822} header of the
root message object.  If the root object has no envelope header, a
standard one is crafted.  By default, this is set to \code{False} to
inhibit the printing of the envelope delimiter.

Note that for subparts, no envelope header is ever printed.

\versionadded{2.2.2}
\end{methoddesc}

\begin{methoddesc}[Generator]{clone}{fp}
Return an independent clone of this \class{Generator} instance with
the exact same options.

\versionadded{2.2.2}
\end{methoddesc}

\begin{methoddesc}[Generator]{write}{s}
Write the string \var{s} to the underlying file object,
i.e. \var{outfp} passed to \class{Generator}'s constructor.  This
provides just enough file-like API for \class{Generator} instances to
be used in extended print statements.
\end{methoddesc}

As a convenience, see the methods \method{Message.as_string()} and
\code{str(aMessage)}, a.k.a. \method{Message.__str__()}, which
simplify the generation of a formatted string representation of a
message object.  For more detail, see \refmodule{email.message}.

The \module{email.generator} module also provides a derived class,
called \class{DecodedGenerator} which is like the \class{Generator}
base class, except that non-\mimetype{text} parts are substituted with
a format string representing the part.

\begin{classdesc}{DecodedGenerator}{outfp\optional{, mangle_from_\optional{,
    maxheaderlen\optional{, fmt}}}}

This class, derived from \class{Generator} walks through all the
subparts of a message.  If the subpart is of main type
\mimetype{text}, then it prints the decoded payload of the subpart.
Optional \var{_mangle_from_} and \var{maxheaderlen} are as with the
\class{Generator} base class.

If the subpart is not of main type \mimetype{text}, optional \var{fmt}
is a format string that is used instead of the message payload.
\var{fmt} is expanded with the following keywords, \samp{\%(keyword)s}
format:

\begin{itemize}
\item \code{type} -- Full MIME type of the non-\mimetype{text} part

\item \code{maintype} -- Main MIME type of the non-\mimetype{text} part

\item \code{subtype} -- Sub-MIME type of the non-\mimetype{text} part

\item \code{filename} -- Filename of the non-\mimetype{text} part

\item \code{description} -- Description associated with the
      non-\mimetype{text} part

\item \code{encoding} -- Content transfer encoding of the
      non-\mimetype{text} part

\end{itemize}

The default value for \var{fmt} is \code{None}, meaning

\begin{verbatim}
[Non-text (%(type)s) part of message omitted, filename %(filename)s]
\end{verbatim}

\versionadded{2.2.2}
\end{classdesc}

\versionchanged[The previously deprecated method \method{__call__()} was
removed]{2.5}


\subsection{Creating email and MIME objects from scratch}

Ordinarily, you get a message object tree by passing some text to a
parser, which parses the text and returns the root of the message
object tree.  However you can also build a complete object tree from
scratch, or even individual \class{Message} objects by hand.  In fact,
you can also take an existing tree and add new \class{Message}
objects, move them around, etc.  This makes a very convenient
interface for slicing-and-dicing MIME messages.

You can create a new object tree by creating \class{Message}
instances, adding payloads and all the appropriate headers manually.
For MIME messages though, the \module{email} package provides some
convenient classes to make things easier.  Each of these classes
should be imported from a module with the same name as the class, from
within the \module{email} package.  E.g.:

\begin{verbatim}
import email.MIMEImage.MIMEImage
\end{verbatim}

or

\begin{verbatim}
from email.MIMEText import MIMEText
\end{verbatim}

Here are the classes:

\begin{classdesc}{MIMEBase}{_maintype, _subtype, **_params}
This is the base class for all the MIME-specific subclasses of
\class{Message}.  Ordinarily you won't create instances specifically
of \class{MIMEBase}, although you could.  \class{MIMEBase} is provided
primarily as a convenient base class for more specific MIME-aware
subclasses.

\var{_maintype} is the \mailheader{Content-Type} major type
(e.g. \mimetype{text} or \mimetype{image}), and \var{_subtype} is the
\mailheader{Content-Type} minor type 
(e.g. \mimetype{plain} or \mimetype{gif}).  \var{_params} is a parameter
key/value dictionary and is passed directly to
\method{Message.add_header()}.

The \class{MIMEBase} class always adds a \mailheader{Content-Type} header
(based on \var{_maintype}, \var{_subtype}, and \var{_params}), and a
\mailheader{MIME-Version} header (always set to \code{1.0}).
\end{classdesc}

\begin{classdesc}{MIMEImage}{_imagedata\optional{, _subtype\optional{,
    _encoder\optional{, **_params}}}}

A subclass of \class{MIMEBase}, the \class{MIMEImage} class is used to
create MIME message objects of major type \mimetype{image}.
\var{_imagedata} is a string containing the raw image data.  If this
data can be decoded by the standard Python module \refmodule{imghdr},
then the subtype will be automatically included in the
\mailheader{Content-Type} header.  Otherwise you can explicitly specify the
image subtype via the \var{_subtype} parameter.  If the minor type could
not be guessed and \var{_subtype} was not given, then \exception{TypeError}
is raised.

Optional \var{_encoder} is a callable (i.e. function) which will
perform the actual encoding of the image data for transport.  This
callable takes one argument, which is the \class{MIMEImage} instance.
It should use \method{get_payload()} and \method{set_payload()} to
change the payload to encoded form.  It should also add any
\mailheader{Content-Transfer-Encoding} or other headers to the message
object as necessary.  The default encoding is \emph{Base64}.  See the
\refmodule{email.Encoders} module for a list of the built-in encoders.

\var{_params} are passed straight through to the \class{MIMEBase}
constructor.
\end{classdesc}

\begin{classdesc}{MIMEText}{_text\optional{, _subtype\optional{,
    _charset\optional{, _encoder}}}}

A subclass of \class{MIMEBase}, the \class{MIMEText} class is used to
create MIME objects of major type \mimetype{text}.  \var{_text} is the
string for the payload.  \var{_subtype} is the minor type and defaults
to \mimetype{plain}.  \var{_charset} is the character set of the text and is
passed as a parameter to the \class{MIMEBase} constructor; it defaults
to \code{us-ascii}.  No guessing or encoding is performed on the text
data, but a newline is appended to \var{_text} if it doesn't already
end with a newline.

The \var{_encoding} argument is as with the \class{MIMEImage} class
constructor, except that the default encoding for \class{MIMEText}
objects is one that doesn't actually modify the payload, but does set
the \mailheader{Content-Transfer-Encoding} header to \code{7bit} or
\code{8bit} as appropriate.
\end{classdesc}

\begin{classdesc}{MIMEMessage}{_msg\optional{, _subtype}}
A subclass of \class{MIMEBase}, the \class{MIMEMessage} class is used to
create MIME objects of main type \mimetype{message}.  \var{_msg} is used as
the payload, and must be an instance of class \class{Message} (or a
subclass thereof), otherwise a \exception{TypeError} is raised.

Optional \var{_subtype} sets the subtype of the message; it defaults
to \mimetype{rfc822}.
\end{classdesc}

\subsection{Encoders}
\declaremodule{standard}{email.Encoders}
\modulesynopsis{Encoders for email message payloads.}

When creating \class{Message} objects from scratch, you often need to
encode the payloads for transport through compliant mail servers.
This is especially true for \mimetype{image/*} and \mimetype{text/*}
type messages containing binary data.

The \module{email} package provides some convenient encodings in its
\module{Encoders} module.  These encoders are actually used by the
\class{MIMEAudio} and \class{MIMEImage} class constructors to provide default
encodings.  All encoder functions take exactly one argument, the message
object to encode.  They usually extract the payload, encode it, and reset the
payload to this newly encoded value.  They should also set the
\mailheader{Content-Transfer-Encoding} header as appropriate.

Here are the encoding functions provided:

\begin{funcdesc}{encode_quopri}{msg}
Encodes the payload into quoted-printable form and sets the
\mailheader{Content-Transfer-Encoding} header to
\code{quoted-printable}\footnote{Note that encoding with
\method{encode_quopri()} also encodes all tabs and space characters in
the data.}.
This is a good encoding to use when most of your payload is normal
printable data, but contains a few unprintable characters.
\end{funcdesc}

\begin{funcdesc}{encode_base64}{msg}
Encodes the payload into base64 form and sets the
\mailheader{Content-Transfer-Encoding} header to
\code{base64}.  This is a good encoding to use when most of your payload
is unprintable data since it is a more compact form than
quoted-printable.  The drawback of base64 encoding is that it
renders the text non-human readable.
\end{funcdesc}

\begin{funcdesc}{encode_7or8bit}{msg}
This doesn't actually modify the message's payload, but it does set
the \mailheader{Content-Transfer-Encoding} header to either \code{7bit} or
\code{8bit} as appropriate, based on the payload data.
\end{funcdesc}

\begin{funcdesc}{encode_noop}{msg}
This does nothing; it doesn't even set the
\mailheader{Content-Transfer-Encoding} header.
\end{funcdesc}


\subsection{Exception classes}
\declaremodule{standard}{email.Errors}
\modulesynopsis{The exception classes used by the email package.}

The following exception classes are defined in the
\module{email.Errors} module:

\begin{excclassdesc}{MessageError}{}
This is the base class for all exceptions that the \module{email}
package can raise.  It is derived from the standard
\exception{Exception} class and defines no additional methods.
\end{excclassdesc}

\begin{excclassdesc}{MessageParseError}{}
This is the base class for exceptions thrown by the \class{Parser}
class.  It is derived from \exception{MessageError}.
\end{excclassdesc}

\begin{excclassdesc}{HeaderParseError}{}
Raised under some error conditions when parsing the \rfc{2822} headers of
a message, this class is derived from \exception{MessageParseError}.
It can be raised from the \method{Parser.parse()} or
\method{Parser.parsestr()} methods.

Situations where it can be raised include finding an envelope
header after the first \rfc{2822} header of the message, finding a
continuation line before the first \rfc{2822} header is found, or finding
a line in the headers which is neither a header or a continuation
line.
\end{excclassdesc}

\begin{excclassdesc}{BoundaryError}{}
Raised under some error conditions when parsing the \rfc{2822} headers of
a message, this class is derived from \exception{MessageParseError}.
It can be raised from the \method{Parser.parse()} or
\method{Parser.parsestr()} methods.

Situations where it can be raised include not being able to find the
starting or terminating boundary in a \mimetype{multipart/*} message
when strict parsing is used.
\end{excclassdesc}

\begin{excclassdesc}{MultipartConversionError}{}
Raised when a payload is added to a \class{Message} object using
\method{add_payload()}, but the payload is already a scalar and the
message's \mailheader{Content-Type} main type is not either
\mimetype{multipart} or missing.  \exception{MultipartConversionError}
multiply inherits from \exception{MessageError} and the built-in
\exception{TypeError}.

Since \method{Message.add_payload()} is deprecated, this exception is
rarely raised in practice.  However the exception may also be raised
if the \method{attach()} method is called on an instance of a class
derived from \class{MIMENonMultipart} (e.g. \class{MIMEImage}).
\end{excclassdesc}


\subsection{Miscellaneous utilities}
\declaremodule{standard}{email.Utils}
\modulesynopsis{Miscellaneous email package utilities.}

There are several useful utilities provided with the \module{email}
package.

\begin{funcdesc}{quote}{str}
Return a new string with backslashes in \var{str} replaced by two
backslashes, and double quotes replaced by backslash-double quote.
\end{funcdesc}

\begin{funcdesc}{unquote}{str}
Return a new string which is an \emph{unquoted} version of \var{str}.
If \var{str} ends and begins with double quotes, they are stripped
off.  Likewise if \var{str} ends and begins with angle brackets, they
are stripped off.
\end{funcdesc}

\begin{funcdesc}{parseaddr}{address}
Parse address -- which should be the value of some address-containing
field such as \mailheader{To} or \mailheader{Cc} -- into its constituent
\emph{realname} and \emph{email address} parts.  Returns a tuple of that
information, unless the parse fails, in which case a 2-tuple of
\code{('', '')} is returned.
\end{funcdesc}

\begin{funcdesc}{formataddr}{pair}
The inverse of \method{parseaddr()}, this takes a 2-tuple of the form
\code{(realname, email_address)} and returns the string value suitable
for a \mailheader{To} or \mailheader{Cc} header.  If the first element of
\var{pair} is false, then the second element is returned unmodified.
\end{funcdesc}

\begin{funcdesc}{getaddresses}{fieldvalues}
This method returns a list of 2-tuples of the form returned by
\code{parseaddr()}.  \var{fieldvalues} is a sequence of header field
values as might be returned by \method{Message.get_all()}.  Here's a
simple example that gets all the recipients of a message:

\begin{verbatim}
from email.Utils import getaddresses

tos = msg.get_all('to', [])
ccs = msg.get_all('cc', [])
resent_tos = msg.get_all('resent-to', [])
resent_ccs = msg.get_all('resent-cc', [])
all_recipients = getaddresses(tos + ccs + resent_tos + resent_ccs)
\end{verbatim}
\end{funcdesc}

\begin{funcdesc}{parsedate}{date}
Attempts to parse a date according to the rules in \rfc{2822}.
however, some mailers don't follow that format as specified, so
\function{parsedate()} tries to guess correctly in such cases. 
\var{date} is a string containing an \rfc{2822} date, such as 
\code{"Mon, 20 Nov 1995 19:12:08 -0500"}.  If it succeeds in parsing
the date, \function{parsedate()} returns a 9-tuple that can be passed
directly to \function{time.mktime()}; otherwise \code{None} will be
returned.  Note that fields 6, 7, and 8 of the result tuple are not
usable.
\end{funcdesc}

\begin{funcdesc}{parsedate_tz}{date}
Performs the same function as \function{parsedate()}, but returns
either \code{None} or a 10-tuple; the first 9 elements make up a tuple
that can be passed directly to \function{time.mktime()}, and the tenth
is the offset of the date's timezone from UTC (which is the official
term for Greenwich Mean Time)\footnote{Note that the sign of the timezone
offset is the opposite of the sign of the \code{time.timezone}
variable for the same timezone; the latter variable follows the
\POSIX{} standard while this module follows \rfc{2822}.}.  If the input
string has no timezone, the last element of the tuple returned is
\code{None}.  Note that fields 6, 7, and 8 of the result tuple are not
usable.
\end{funcdesc}

\begin{funcdesc}{mktime_tz}{tuple}
Turn a 10-tuple as returned by \function{parsedate_tz()} into a UTC
timestamp.  It the timezone item in the tuple is \code{None}, assume
local time.  Minor deficiency: \function{mktime_tz()} interprets the
first 8 elements of \var{tuple} as a local time and then compensates
for the timezone difference.  This may yield a slight error around
changes in daylight savings time, though not worth worrying about for
common use.
\end{funcdesc}

\begin{funcdesc}{formatdate}{\optional{timeval\optional{, localtime}}}
Returns a date string as per \rfc{2822}, e.g.:

\begin{verbatim}
Fri, 09 Nov 2001 01:08:47 -0000
\end{verbatim}

Optional \var{timeval} if given is a floating point time value as
accepted by \function{time.gmtime()} and \function{time.localtime()},
otherwise the current time is used.

Optional \var{localtime} is a flag that when \code{True}, interprets
\var{timeval}, and returns a date relative to the local timezone
instead of UTC, properly taking daylight savings time into account.
The default is \code{False} meaning UTC is used.
\end{funcdesc}

\begin{funcdesc}{make_msgid}{\optional{idstring}}
Returns a string suitable for an \rfc{2822}-compliant
\mailheader{Message-ID} header.  Optional \var{idstring} if given, is
a string used to strengthen the uniqueness of the message id.
\end{funcdesc}

\begin{funcdesc}{decode_rfc2231}{s}
Decode the string \var{s} according to \rfc{2231}.
\end{funcdesc}

\begin{funcdesc}{encode_rfc2231}{s\optional{, charset\optional{, language}}}
Encode the string \var{s} according to \rfc{2231}.  Optional
\var{charset} and \var{language}, if given is the character set name
and language name to use.  If neither is given, \var{s} is returned
as-is.  If \var{charset} is given but \var{language} is not, the
string is encoded using the empty string for \var{language}.
\end{funcdesc}

\begin{funcdesc}{decode_params}{params}
Decode parameters list according to \rfc{2231}.  \var{params} is a
sequence of 2-tuples containing elements of the form
\code{(content-type, string-value)}.
\end{funcdesc}

The following functions have been deprecated:

\begin{funcdesc}{dump_address_pair}{pair}
\deprecated{2.2.2}{Use \function{formataddr()} instead.}
\end{funcdesc}

\begin{funcdesc}{decode}{s}
\deprecated{2.2.2}{Use \method{Header.decode_header()} instead.}
\end{funcdesc}


\begin{funcdesc}{encode}{s\optional{, charset\optional{, encoding}}}
\deprecated{2.2.2}{Use \method{Header.encode()} instead.}
\end{funcdesc}



\subsection{Iterators}
\declaremodule{standard}{email.Iterators}
\modulesynopsis{Iterate over a  message object tree.}

Iterating over a message object tree is fairly easy with the
\method{Message.walk()} method.  The \module{email.Iterators} module
provides some useful higher level iterations over message object
trees.

\begin{funcdesc}{body_line_iterator}{msg}
This iterates over all the payloads in all the subparts of \var{msg},
returning the string payloads line-by-line.  It skips over all the
subpart headers, and it skips over any subpart with a payload that
isn't a Python string.  This is somewhat equivalent to reading the
flat text representation of the message from a file using
\method{readline()}, skipping over all the intervening headers.
\end{funcdesc}

\begin{funcdesc}{typed_subpart_iterator}{msg\optional{,
    maintype\optional{, subtype}}}
This iterates over all the subparts of \var{msg}, returning only those
subparts that match the MIME type specified by \var{maintype} and
\var{subtype}.

Note that \var{subtype} is optional; if omitted, then subpart MIME
type matching is done only with the main type.  \var{maintype} is
optional too; it defaults to \mimetype{text}.

Thus, by default \function{typed_subpart_iterator()} returns each
subpart that has a MIME type of \mimetype{text/*}.
\end{funcdesc}

The following function has been added as a useful debugging tool.  It
should \emph{not} be considered part of the supported public interface
for the package.

\begin{funcdesc}{_structure}{msg\optional{, fp\optional{, level}}}
Prints an indented representation of the content types of the
message object structure.  For example:

\begin{verbatim}
>>> msg = email.message_from_file(somefile)
>>> _structure(msg)
multipart/mixed
    text/plain
    text/plain
    multipart/digest
        message/rfc822
            text/plain
        message/rfc822
            text/plain
        message/rfc822
            text/plain
        message/rfc822
            text/plain
        message/rfc822
            text/plain
    text/plain
\end{verbatim}

Optional \var{fp} is a file-like object to print the output to.  It
must be suitable for Python's extended print statement.  \var{level}
is used internally.
\end{funcdesc}


\subsection{Differences from \module{mimelib}}

The \module{email} package was originally prototyped as a separate
library called
\ulink{\module{mimelib}}{http://mimelib.sf.net/}.
Changes have been made so that
method names are more consistent, and some methods or modules have
either been added or removed.  The semantics of some of the methods
have also changed.  For the most part, any functionality available in
\module{mimelib} is still available in the \module{email} package,
albeit often in a different way.

Here is a brief description of the differences between the
\module{mimelib} and the \module{email} packages, along with hints on
how to port your applications.

Of course, the most visible difference between the two packages is
that the package name has been changed to \module{email}.  In
addition, the top-level package has the following differences:

\begin{itemize}
\item \function{messageFromString()} has been renamed to
      \function{message_from_string()}.
\item \function{messageFromFile()} has been renamed to
      \function{message_from_file()}.
\end{itemize}

The \class{Message} class has the following differences:

\begin{itemize}
\item The method \method{asString()} was renamed to \method{as_string()}.
\item The method \method{ismultipart()} was renamed to
      \method{is_multipart()}.
\item The \method{get_payload()} method has grown a \var{decode}
      optional argument.
\item The method \method{getall()} was renamed to \method{get_all()}.
\item The method \method{addheader()} was renamed to \method{add_header()}.
\item The method \method{gettype()} was renamed to \method{get_type()}.
\item The method\method{getmaintype()} was renamed to
      \method{get_main_type()}.
\item The method \method{getsubtype()} was renamed to
      \method{get_subtype()}.
\item The method \method{getparams()} was renamed to
      \method{get_params()}.
      Also, whereas \method{getparams()} returned a list of strings,
      \method{get_params()} returns a list of 2-tuples, effectively
      the key/value pairs of the parameters, split on the \character{=}
      sign.
\item The method \method{getparam()} was renamed to \method{get_param()}.
\item The method \method{getcharsets()} was renamed to
      \method{get_charsets()}.
\item The method \method{getfilename()} was renamed to
      \method{get_filename()}.
\item The method \method{getboundary()} was renamed to
      \method{get_boundary()}.
\item The method \method{setboundary()} was renamed to
      \method{set_boundary()}.
\item The method \method{getdecodedpayload()} was removed.  To get
      similar functionality, pass the value 1 to the \var{decode} flag
      of the {get_payload()} method.
\item The method \method{getpayloadastext()} was removed.  Similar
      functionality
      is supported by the \class{DecodedGenerator} class in the
      \refmodule{email.Generator} module.
\item The method \method{getbodyastext()} was removed.  You can get
      similar functionality by creating an iterator with
      \function{typed_subpart_iterator()} in the
      \refmodule{email.Iterators} module.
\end{itemize}

The \class{Parser} class has no differences in its public interface.
It does have some additional smarts to recognize
\mimetype{message/delivery-status} type messages, which it represents as
a \class{Message} instance containing separate \class{Message}
subparts for each header block in the delivery status
notification\footnote{Delivery Status Notifications (DSN) are defined
in \rfc{1894}}.

The \class{Generator} class has no differences in its public
interface.  There is a new class in the \refmodule{email.Generator}
module though, called \class{DecodedGenerator} which provides most of
the functionality previously available in the
\method{Message.getpayloadastext()} method.

The following modules and classes have been changed:

\begin{itemize}
\item The \class{MIMEBase} class constructor arguments \var{_major}
      and \var{_minor} have changed to \var{_maintype} and
      \var{_subtype} respectively.
\item The \code{Image} class/module has been renamed to
      \code{MIMEImage}.  The \var{_minor} argument has been renamed to
      \var{_subtype}.
\item The \code{Text} class/module has been renamed to
      \code{MIMEText}.  The \var{_minor} argument has been renamed to
      \var{_subtype}.
\item The \code{MessageRFC822} class/module has been renamed to
      \code{MIMEMessage}.  Note that an earlier version of
      \module{mimelib} called this class/module \code{RFC822}, but
      that clashed with the Python standard library module
      \refmodule{rfc822} on some case-insensitive file systems.

      Also, the \class{MIMEMessage} class now represents any kind of
      MIME message with main type \mimetype{message}.  It takes an
      optional argument \var{_subtype} which is used to set the MIME
      subtype.  \var{_subtype} defaults to \mimetype{rfc822}.
\end{itemize}

\module{mimelib} provided some utility functions in its
\module{address} and \module{date} modules.  All of these functions
have been moved to the \refmodule{email.Utils} module.

The \code{MsgReader} class/module has been removed.  Its functionality
is most closely supported in the \function{body_line_iterator()}
function in the \refmodule{email.Iterators} module.

\subsection{Examples}

Coming soon...


\end{document}

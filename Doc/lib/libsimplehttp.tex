\section{\module{SimpleHTTPServer} ---
         Simple HTTP request handler}

\declaremodule{standard}{SimpleHTTPServer}
\sectionauthor{Moshe Zadka}{mzadka@geocities.com}
\modulesynopsis{This module provides a basic request handler for HTTP
                servers.}


The \module{SimpleHTTPServer} module defines a request-handler class,
interface compatible with \class{BaseHTTPServer.BaseHTTPRequestHandler}
which serves files only from a base directory.

The \module{SimpleHTTPServer} module defines the following class:

\begin{classdesc}{SimpleHTTPRequestHandler}{request, client_address, server}
This class is used, to serve files from current directory and below,
directly mapping the directory structure to HTTP requests.

A lot of the work is done by the base class
\class{BaseHTTPServer.BaseHTTPRequestHandler}, such as parsing the
request.  This class implements the \function{do_GET()} and
\function{do_HEAD()} functions.
\end{classdesc}

The \class{SimpleHTTPRequestHandler} defines the following member
variables:

\begin{memberdesc}{server_version}
This will be \code{"SimpleHTTP/" + __version__}, where \code{__version__}
is defined in the module.
\end{memberdesc}

\begin{memberdesc}{extensions_map}
A dictionary mapping suffixes into MIME types. Default is signified
by an empty string, and is considered to be \code{text/plain}.
The mapping is used case-insensitively, and so should contain only
lower-cased keys.
\end{memberdesc}

The \class{SimpleHTTPRequestHandler} defines the following methods:

\begin{methoddesc}{do_HEAD}{}
This method serves the \code{'HEAD'} request type: it sends the
headers it would send for the equivalent \code{GET} request. See the
\method{do_GET()} method for more complete explanation of the possible
headers.
\end{methoddesc}

\begin{methoddesc}{do_GET}{}
The request is mapped to a local file by interpreting the request as
a path relative to the current working directory.

If the request was mapped to a directory, a \code{403} respond is output,
followed by the explanation \code{'Directory listing not supported'}.
Any \exception{IOError} exception in opening the requested file, is mapped
to a \code{404}, \code{'File not found'} error. Otherwise, the content
type is guessed using the \var{extensions_map} variable.

A \code{'Content-type:'} with the guessed content type is output, and
then a blank line, signifying end of headers, and then the contents of
the file. The file is always opened in binary mode.

For example usage, see the implementation of the \function{test()}
function.
\end{methoddesc}


\begin{seealso}
  \seemodule{BaseHTTPServer}{Base class implementation for Web server
                             and request handler.}
\end{seealso}

\section{\module{urllib2} ---
         extensible library for opening URLs}

\declaremodule{standard}{urllib2}
\moduleauthor{Jeremy Hylton}{jhylton@users.sourceforge.net}
\sectionauthor{Moshe Zadka}{moshez@users.sourceforge.net}

\modulesynopsis{An extensible library for opening URLs using a variety of 
                protocols}

The \module{urllib2} module defines functions and classes which help
in opening URLs (mostly HTTP) in a complex world --- basic and digest
authentication, redirections and more.

The \module{urllib2} module defines the following functions:

\begin{funcdesc}{urlopen}{url\optional{, data}}
Open the URL \var{url}, which can be either a string or a \class{Request}
object (currently the code checks that it really is a \class{Request}
instance, or an instance of a subclass of \class{Request}).

\var{data} should be a string, which specifies additional data to
send to the server. In HTTP requests, which are the only ones that
support \var{data}, it should be a buffer in the format of
\mimetype{application/x-www-form-urlencoded}, for example one returned
from \function{urllib.urlencode()}.

This function returns a file-like object with two additional methods:

\begin{itemize}
  \item \method{geturl()} --- return the URL of the resource retrieved
  \item \method{info()} --- return the meta-information of the page, as
                            a dictionary-like object
\end{itemize}

Raises \exception{URLError} on errors.
\end{funcdesc}

\begin{funcdesc}{install_opener}{opener}
Install an \class{OpenerDirector} instance as the default opener.
The code does not check for a real \class{OpenerDirector}, and any
class with the appropriate interface will work.
\end{funcdesc}

\begin{funcdesc}{build_opener}{\optional{handler, \moreargs}}
Return an \class{OpenerDirector} instance, which chains the
handlers in the order given. \var{handler}s can be either instances
of \class{BaseHandler}, or subclasses of \class{BaseHandler} (in
which case it must be possible to call the constructor without
any parameters).  Instances of the following classes will be in
front of the \var{handler}s, unless the \var{handler}s contain
them, instances of them or subclasses of them:
\class{ProxyHandler}, \class{UnknownHandler}, \class{HTTPHandler},
\class{HTTPDefaultErrorHandler}, \class{HTTPRedirectHandler},
\class{FTPHandler}, \class{FileHandler}

If the Python installation has SSL support (\function{socket.ssl()}
exists), \class{HTTPSHandler} will also be added.

Beginning in Python 2.3, a \class{BaseHandler} subclass may also
change its \member{handler_order} member variable to modify its
position in the handlers list. Besides \class{ProxyHandler}, which has
\member{handler_order} of \code{100}, all handlers currently have it
set to \code{500}.
\end{funcdesc}


The following exceptions are raised as appropriate:

\begin{excdesc}{URLError}
The handlers raise this exception (or derived exceptions) when they
run into a problem.  It is a subclass of \exception{IOError}.
\end{excdesc}

\begin{excdesc}{HTTPError}
A subclass of \exception{URLError}, it can also function as a 
non-exceptional file-like return value (the same thing that
\function{urlopen()} returns).  This is useful when handling exotic
HTTP errors, such as requests for authentication.
\end{excdesc}

\begin{excdesc}{GopherError}
A subclass of \exception{URLError}, this is the error raised by the
Gopher handler.
\end{excdesc}


The following classes are provided:

\begin{classdesc}{Request}{url\optional{, data\optional{, headers}}}
This class is an abstraction of a URL request.

\var{url} should be a string which is a valid URL.  For a description
of \var{data} see the \method{add_data()} description.
\var{headers} should be a dictionary, and will be treated as if
\method{add_header()} was called with each key and value as arguments.
\end{classdesc}

\begin{classdesc}{OpenerDirector}{}
The \class{OpenerDirector} class opens URLs via \class{BaseHandler}s
chained together. It manages the chaining of handlers, and recovery
from errors.
\end{classdesc}

\begin{classdesc}{BaseHandler}{}
This is the base class for all registered handlers --- and handles only
the simple mechanics of registration.
\end{classdesc}

\begin{classdesc}{HTTPDefaultErrorHandler}{}
A class which defines a default handler for HTTP error responses; all
responses are turned into \exception{HTTPError} exceptions.
\end{classdesc}

\begin{classdesc}{HTTPRedirectHandler}{}
A class to handle redirections.
\end{classdesc}

\begin{classdesc}{ProxyHandler}{\optional{proxies}}
Cause requests to go through a proxy.
If \var{proxies} is given, it must be a dictionary mapping
protocol names to URLs of proxies.
The default is to read the list of proxies from the environment
variables \var{protocol}_proxy.
\end{classdesc}

\begin{classdesc}{HTTPPasswordMgr}{}
Keep a database of 
\code{(\var{realm}, \var{uri}) -> (\var{user}, \var{password})}
mappings.
\end{classdesc}

\begin{classdesc}{HTTPPasswordMgrWithDefaultRealm}{}
Keep a database of 
\code{(\var{realm}, \var{uri}) -> (\var{user}, \var{password})} mappings.
A realm of \code{None} is considered a catch-all realm, which is searched
if no other realm fits.
\end{classdesc}

\begin{classdesc}{AbstractBasicAuthHandler}{\optional{password_mgr}}
This is a mixin class that helps with HTTP authentication, both
to the remote host and to a proxy.
\var{password_mgr}, if given, should be something that is compatible
with \class{HTTPPasswordMgr}; refer to section~\ref{http-password-mgr}
for information on the interface that must be supported.
\end{classdesc}

\begin{classdesc}{HTTPBasicAuthHandler}{\optional{password_mgr}}
Handle authentication with the remote host.
\var{password_mgr}, if given, should be something that is compatible
with \class{HTTPPasswordMgr}; refer to section~\ref{http-password-mgr}
for information on the interface that must be supported.
\end{classdesc}

\begin{classdesc}{ProxyBasicAuthHandler}{\optional{password_mgr}}
Handle authentication with the proxy.
\var{password_mgr}, if given, should be something that is compatible
with \class{HTTPPasswordMgr}; refer to section~\ref{http-password-mgr}
for information on the interface that must be supported.
\end{classdesc}

\begin{classdesc}{AbstractDigestAuthHandler}{\optional{password_mgr}}
This is a mixin class that helps with HTTP authentication, both
to the remote host and to a proxy.
\var{password_mgr}, if given, should be something that is compatible
with \class{HTTPPasswordMgr}; refer to section~\ref{http-password-mgr}
for information on the interface that must be supported.
\end{classdesc}

\begin{classdesc}{HTTPDigestAuthHandler}{\optional{password_mgr}}
Handle authentication with the remote host.
\var{password_mgr}, if given, should be something that is compatible
with \class{HTTPPasswordMgr}; refer to section~\ref{http-password-mgr}
for information on the interface that must be supported.
\end{classdesc}

\begin{classdesc}{ProxyDigestAuthHandler}{\optional{password_mgr}}
Handle authentication with the proxy.
\var{password_mgr}, if given, should be something that is compatible
with \class{HTTPPasswordMgr}; refer to section~\ref{http-password-mgr}
for information on the interface that must be supported.
\end{classdesc}

\begin{classdesc}{HTTPHandler}{}
A class to handle opening of HTTP URLs.
\end{classdesc}

\begin{classdesc}{HTTPSHandler}{}
A class to handle opening of HTTPS URLs.
\end{classdesc}

\begin{classdesc}{FileHandler}{}
Open local files.
\end{classdesc}

\begin{classdesc}{FTPHandler}{}
Open FTP URLs.
\end{classdesc}

\begin{classdesc}{CacheFTPHandler}{}
Open FTP URLs, keeping a cache of open FTP connections to minimize
delays.
\end{classdesc}

\begin{classdesc}{GopherHandler}{}
Open gopher URLs.
\end{classdesc}

\begin{classdesc}{UnknownHandler}{}
A catch-all class to handle unknown URLs.
\end{classdesc}


\subsection{Request Objects \label{request-objects}}

The following methods describe all of \class{Request}'s public interface,
and so all must be overridden in subclasses.

\begin{methoddesc}[Request]{add_data}{data}
Set the \class{Request} data to \var{data}.  This is ignored
by all handlers except HTTP handlers --- and there it should be an
\mimetype{application/x-www-form-encoded} buffer, and will change the
request to be \code{POST} rather than \code{GET}. 
\end{methoddesc}

\begin{methoddesc}[Request]{get_method}{}
Return a string indicating the HTTP request method.  This is only
meaningful for HTTP requests, and currently always takes one of the
values ("GET", "POST").
\end{methoddesc}

\begin{methoddesc}[Request]{has_data}{}
Return whether the instance has a non-\code{None} data.
\end{methoddesc}

\begin{methoddesc}[Request]{get_data}{}
Return the instance's data.
\end{methoddesc}

\begin{methoddesc}[Request]{add_header}{key, val}
Add another header to the request.  Headers are currently ignored by
all handlers except HTTP handlers, where they are added to the list
of headers sent to the server.  Note that there cannot be more than
one header with the same name, and later calls will overwrite
previous calls in case the \var{key} collides.  Currently, this is
no loss of HTTP functionality, since all headers which have meaning
when used more than once have a (header-specific) way of gaining the
same functionality using only one header.
\end{methoddesc}

\begin{methoddesc}[Request]{get_full_url}{}
Return the URL given in the constructor.
\end{methoddesc}

\begin{methoddesc}[Request]{get_type}{}
Return the type of the URL --- also known as the scheme.
\end{methoddesc}

\begin{methoddesc}[Request]{get_host}{}
Return the host to which a connection will be made.
\end{methoddesc}

\begin{methoddesc}[Request]{get_selector}{}
Return the selector --- the part of the URL that is sent to
the server.
\end{methoddesc}

\begin{methoddesc}[Request]{set_proxy}{host, type}
Prepare the request by connecting to a proxy server. The \var{host}
and \var{type} will replace those of the instance, and the instance's
selector will be the original URL given in the constructor.
\end{methoddesc}


\subsection{OpenerDirector Objects \label{opener-director-objects}}

\class{OpenerDirector} instances have the following methods:

\begin{methoddesc}[OpenerDirector]{add_handler}{handler}
\var{handler} should be an instance of \class{BaseHandler}.  The
following methods are searched, and added to the possible chains.

\begin{itemize}
  \item \method{\var{protocol}_open()} ---
    signal that the handler knows how to open \var{protocol} URLs.
  \item \method{\var{protocol}_error_\var{type}()} ---
    signal that the handler knows how to handle \var{type} errors from
    \var{protocol}.
\end{itemize}
\end{methoddesc}

\begin{methoddesc}[OpenerDirector]{close}{}
Explicitly break cycles, and delete all the handlers.
Because the \class{OpenerDirector} needs to know the registered handlers,
and a handler needs to know who the \class{OpenerDirector} who called
it is, there is a reference cycle.  Even though recent versions of Python
have cycle-collection, it is sometimes preferable to explicitly break
the cycles.
\end{methoddesc}

\begin{methoddesc}[OpenerDirector]{open}{url\optional{, data}}
Open the given \var{url} (which can be a request object or a string),
optionally passing the given \var{data}.
Arguments, return values and exceptions raised are the same as those
of \function{urlopen()} (which simply calls the \method{open()} method
on the default installed \class{OpenerDirector}).
\end{methoddesc}

\begin{methoddesc}[OpenerDirector]{error}{proto\optional{,
                                          arg\optional{, \moreargs}}}
Handle an error in a given protocol.  This will call the registered
error handlers for the given protocol with the given arguments (which
are protocol specific).  The HTTP protocol is a special case which
uses the HTTP response code to determine the specific error handler;
refer to the \method{http_error_*()} methods of the handler classes.

Return values and exceptions raised are the same as those
of \function{urlopen()}.
\end{methoddesc}


\subsection{BaseHandler Objects \label{base-handler-objects}}

\class{BaseHandler} objects provide a couple of methods that are
directly useful, and others that are meant to be used by derived
classes.  These are intended for direct use:

\begin{methoddesc}[BaseHandler]{add_parent}{director}
Add a director as parent.
\end{methoddesc}

\begin{methoddesc}[BaseHandler]{close}{}
Remove any parents.
\end{methoddesc}

The following members and methods should only be used by classes
derived from \class{BaseHandler}:

\begin{memberdesc}[BaseHandler]{parent}
A valid \class{OpenerDirector}, which can be used to open using a
different protocol, or handle errors.
\end{memberdesc}

\begin{methoddesc}[BaseHandler]{default_open}{req}
This method is \emph{not} defined in \class{BaseHandler}, but
subclasses should define it if they want to catch all URLs.

This method, if implemented, will be called by the parent
\class{OpenerDirector}.  It should return a file-like object as
described in the return value of the \method{open()} of
\class{OpenerDirector}, or \code{None}.  It should raise
\exception{URLError}, unless a truly exceptional thing happens (for
example, \exception{MemoryError} should not be mapped to
\exception{URLError}).

This method will be called before any protocol-specific open method.
\end{methoddesc}

\begin{methoddescni}[BaseHandler]{\var{protocol}_open}{req}
This method is \emph{not} defined in \class{BaseHandler}, but
subclasses should define it if they want to handle URLs with the given
protocol.

This method, if defined, will be called by the parent
\class{OpenerDirector}.  Return values should be the same as for 
\method{default_open()}.
\end{methoddescni}

\begin{methoddesc}[BaseHandler]{unknown_open}{req}
This method is \var{not} defined in \class{BaseHandler}, but
subclasses should define it if they want to catch all URLs with no
specific registered handler to open it.

This method, if implemented, will be called by the \member{parent} 
\class{OpenerDirector}.  Return values should be the same as for 
\method{default_open()}.
\end{methoddesc}

\begin{methoddesc}[BaseHandler]{http_error_default}{req, fp, code, msg, hdrs}
This method is \emph{not} defined in \class{BaseHandler}, but
subclasses should override it if they intend to provide a catch-all
for otherwise unhandled HTTP errors.  It will be called automatically
by the  \class{OpenerDirector} getting the error, and should not
normally be called in other circumstances.

\var{req} will be a \class{Request} object, \var{fp} will be a
file-like object with the HTTP error body, \var{code} will be the
three-digit code of the error, \var{msg} will be the user-visible
explanation of the code and \var{hdrs} will be a mapping object with
the headers of the error.

Return values and exceptions raised should be the same as those
of \function{urlopen()}.
\end{methoddesc}

\begin{methoddesc}[BaseHandler]{http_error_\var{nnn}}{req, fp, code, msg, hdrs}
\var{nnn} should be a three-digit HTTP error code.  This method is
also not defined in \class{BaseHandler}, but will be called, if it
exists, on an instance of a subclass, when an HTTP error with code
\var{nnn} occurs.

Subclasses should override this method to handle specific HTTP
errors.

Arguments, return values and exceptions raised should be the same as
for \method{http_error_default()}.
\end{methoddesc}

\subsection{HTTPRedirectHandler Objects \label{http-redirect-handler}}

\note{Some HTTP redirections require action from this module's client
  code.  If this is the case, \exception{HTTPError} is raised.  See
  \rfc{2616} for details of the precise meanings of the various
  redirection codes.}

\begin{methoddesc}[HTTPRedirectHandler]{redirect_request}{req,
                                                  fp, code, msg, hdrs}
Return a \class{Request} or \code{None} in response to a redirect.
This is called by the default implementations of the
\method{http_error_30*()} methods when a redirection is received
from the server.  If a redirection should take place, return a new
\class{Request} to allow \method{http_error_30*()} to perform the
redirect.  Otherwise, raise \exception{HTTPError} if no other
\class{Handler} should try to handle this URL, or return \code{None}
if you can't but another \class{Handler} might.

\begin{notice}
 The default implementation of this method does not strictly
 follow \rfc{2616}, which says that 301 and 302 responses to \code{POST}
 requests must not be automatically redirected without confirmation by
 the user.  In reality, browsers do allow automatic redirection of
 these responses, changing the POST to a \code{GET}, and the default
 implementation reproduces this behavior.
\end{notice}
\end{methoddesc}


\begin{methoddesc}[HTTPRedirectHandler]{http_error_301}{req,
                                                  fp, code, msg, hdrs}
Redirect to the \code{Location:} URL.  This method is called by
the parent \class{OpenerDirector} when getting an HTTP
`moved permanently' response.
\end{methoddesc}

\begin{methoddesc}[HTTPRedirectHandler]{http_error_302}{req,
                                                  fp, code, msg, hdrs}
The same as \method{http_error_301()}, but called for the
`found' response.
\end{methoddesc}

\begin{methoddesc}[HTTPRedirectHandler]{http_error_303}{req,
                                                  fp, code, msg, hdrs}
The same as \method{http_error_301()}, but called for the
`see other' response.
\end{methoddesc}

\begin{methoddesc}[HTTPRedirectHandler]{http_error_307}{req,
                                                  fp, code, msg, hdrs}
The same as \method{http_error_301()}, but called for the
`temporary redirect' response.
\end{methoddesc}


\subsection{ProxyHandler Objects \label{proxy-handler}}

\begin{methoddescni}[ProxyHandler]{\var{protocol}_open}{request}
The \class{ProxyHandler} will have a method
\method{\var{protocol}_open()} for every \var{protocol} which has a
proxy in the \var{proxies} dictionary given in the constructor.  The
method will modify requests to go through the proxy, by calling
\code{request.set_proxy()}, and call the next handler in the chain to
actually execute the protocol.
\end{methoddescni}


\subsection{HTTPPasswordMgr Objects \label{http-password-mgr}}

These methods are available on \class{HTTPPasswordMgr} and
\class{HTTPPasswordMgrWithDefaultRealm} objects.

\begin{methoddesc}[HTTPPasswordMgr]{add_password}{realm, uri, user, passwd}
\var{uri} can be either a single URI, or a sequene of URIs. \var{realm},
\var{user} and \var{passwd} must be strings. This causes
\code{(\var{user}, \var{passwd})} to be used as authentication tokens
when authentication for \var{realm} and a super-URI of any of the
given URIs is given.
\end{methoddesc}  

\begin{methoddesc}[HTTPPasswordMgr]{find_user_password}{realm, authuri}
Get user/password for given realm and URI, if any.  This method will
return \code{(None, None)} if there is no matching user/password.

For \class{HTTPPasswordMgrWithDefaultRealm} objects, the realm
\code{None} will be searched if the given \var{realm} has no matching
user/password.
\end{methoddesc}


\subsection{AbstractBasicAuthHandler Objects
            \label{abstract-basic-auth-handler}}

\begin{methoddesc}[AbstractBasicAuthHandler]{handle_authentication_request}
                                            {authreq, host, req, headers}
Handle an authentication request by getting a user/password pair, and
re-trying the request.  \var{authreq} should be the name of the header
where the information about the realm is included in the request,
\var{host} is the host to authenticate to, \var{req} should be the
(failed) \class{Request} object, and \var{headers} should be the error
headers.
\end{methoddesc}


\subsection{HTTPBasicAuthHandler Objects
            \label{http-basic-auth-handler}}

\begin{methoddesc}[HTTPBasicAuthHandler]{http_error_401}{req, fp, code, 
                                                        msg, hdrs}
Retry the request with authentication information, if available.
\end{methoddesc}


\subsection{ProxyBasicAuthHandler Objects
            \label{proxy-basic-auth-handler}}

\begin{methoddesc}[ProxyBasicAuthHandler]{http_error_407}{req, fp, code, 
                                                        msg, hdrs}
Retry the request with authentication information, if available.
\end{methoddesc}


\subsection{AbstractDigestAuthHandler Objects
            \label{abstract-digest-auth-handler}}

\begin{methoddesc}[AbstractDigestAuthHandler]{handle_authentication_request}
                                            {authreq, host, req, headers}
\var{authreq} should be the name of the header where the information about
the realm is included in the request, \var{host} should be the host to
authenticate to, \var{req} should be the (failed) \class{Request}
object, and \var{headers} should be the error headers.
\end{methoddesc}


\subsection{HTTPDigestAuthHandler Objects
            \label{http-digest-auth-handler}}

\begin{methoddesc}[HTTPDigestAuthHandler]{http_error_401}{req, fp, code, 
                                                        msg, hdrs}
Retry the request with authentication information, if available.
\end{methoddesc}


\subsection{ProxyDigestAuthHandler Objects
            \label{proxy-digest-auth-handler}}

\begin{methoddesc}[ProxyDigestAuthHandler]{http_error_407}{req, fp, code, 
                                                        msg, hdrs}
Retry the request with authentication information, if available.
\end{methoddesc}


\subsection{HTTPHandler Objects \label{http-handler-objects}}

\begin{methoddesc}[HTTPHandler]{http_open}{req}
Send an HTTP request, which can be either GET or POST, depending on
\code{\var{req}.has_data()}.
\end{methoddesc}


\subsection{HTTPSHandler Objects \label{https-handler-objects}}

\begin{methoddesc}[HTTPSHandler]{https_open}{req}
Send an HTTPS request, which can be either GET or POST, depending on
\code{\var{req}.has_data()}.
\end{methoddesc}


\subsection{FileHandler Objects \label{file-handler-objects}}

\begin{methoddesc}[FileHandler]{file_open}{req}
Open the file locally, if there is no host name, or
the host name is \code{'localhost'}. Change the
protocol to \code{ftp} otherwise, and retry opening
it using \member{parent}.
\end{methoddesc}


\subsection{FTPHandler Objects \label{ftp-handler-objects}}

\begin{methoddesc}[FTPHandler]{ftp_open}{req}
Open the FTP file indicated by \var{req}.
The login is always done with empty username and password.
\end{methoddesc}


\subsection{CacheFTPHandler Objects \label{cacheftp-handler-objects}}

\class{CacheFTPHandler} objects are \class{FTPHandler} objects with
the following additional methods:

\begin{methoddesc}[CacheFTPHandler]{setTimeout}{t}
Set timeout of connections to \var{t} seconds.
\end{methoddesc}

\begin{methoddesc}[CacheFTPHandler]{setMaxConns}{m}
Set maximum number of cached connections to \var{m}.
\end{methoddesc}


\subsection{GopherHandler Objects \label{gopher-handler}}

\begin{methoddesc}[GopherHandler]{gopher_open}{req}
Open the gopher resource indicated by \var{req}.
\end{methoddesc}


\subsection{UnknownHandler Objects \label{unknown-handler-objects}}

\begin{methoddesc}[UnknownHandler]{unknown_open}{}
Raise a \exception{URLError} exception.
\end{methoddesc}


\subsection{Examples \label{urllib2-examples}}

This example gets the python.org main page and displays the first 100
bytes of it:

\begin{verbatim}
>>> import urllib2
>>> f = urllib2.urlopen('http://www.python.org/')
>>> print f.read(100)
<!DOCTYPE html PUBLIC "-//W3C//DTD HTML 4.01 Transitional//EN">
<?xml-stylesheet href="./css/ht2html
\end{verbatim}

Here we are sending a data-stream to the stdin of a CGI and reading
the data it returns to us:

\begin{verbatim}
>>> import urllib2
>>> req = urllib2.Request(url='https://localhost/cgi-bin/test.cgi',
...                       data='This data is passed to stdin of the CGI')
>>> f = urllib2.urlopen(req)
>>> print f.read()
Got Data: "This data is passed to stdin of the CGI"
\end{verbatim}

The code for the sample CGI used in the above example is:

\begin{verbatim}
#!/usr/bin/env python
import sys
data = sys.stdin.read()
print 'Content-type: text-plain\n\nGot Data: "%s"' % data
\end{verbatim}

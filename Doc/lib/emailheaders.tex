\declaremodule{standard}{email.Header}
\modulesynopsis{Representing non-ASCII headers}

\rfc{2822} is the base standard that describes the format of email
messages.  It derives from the older \rfc{822} standard which came
into widespread use at a time when most email was composed of \ASCII{}
characters only.  \rfc{2822} is a specification written assuming email
contains only 7-bit \ASCII{} characters.

Of course, as email has been deployed worldwide, it has become
internationalized, such that language specific character sets can now
be used in email messages.  The base standard still requires email
messages to be transfered using only 7-bit \ASCII{} characters, so a
slew of RFCs have been written describing how to encode email
containing non-\ASCII{} characters into \rfc{2822}-compliant format.
These RFCs include \rfc{2045}, \rfc{2046}, \rfc{2047}, and \rfc{2231}.
The \module{email} package supports these standards in its
\module{email.Header} and \module{email.Charset} modules.

If you want to include non-\ASCII{} characters in your email headers,
say in the \mailheader{Subject} or \mailheader{To} fields, you should
use the \class{Header} class and assign the field in the
\class{Message} object to an instance of \class{Header} instead of
using a string for the header value.  For example:

\begin{verbatim}
>>> from email.Message import Message
>>> from email.Header import Header
>>> msg = Message()
>>> h = Header('p\xf6stal', 'iso-8859-1')
>>> msg['Subject'] = h
>>> print msg.as_string()
Subject: =?iso-8859-1?q?p=F6stal?=


\end{verbatim}

Notice here how we wanted the \mailheader{Subject} field to contain a
non-\ASCII{} character?  We did this by creating a \class{Header}
instance and passing in the character set that the byte string was
encoded in.  When the subsequent \class{Message} instance was
flattened, the \mailheader{Subject} field was properly \rfc{2047}
encoded.  MIME-aware mail readers would show this header using the
embedded ISO-8859-1 character.

\versionadded{2.2.2}

Here is the \class{Header} class description:

\begin{classdesc}{Header}{\optional{s\optional{, charset\optional{,
    maxlinelen\optional{, header_name\optional{, continuation_ws\optional{,
    errors}}}}}}}
Create a MIME-compliant header that can contain strings in different
character sets.

Optional \var{s} is the initial header value.  If \code{None} (the
default), the initial header value is not set.  You can later append
to the header with \method{append()} method calls.  \var{s} may be a
byte string or a Unicode string, but see the \method{append()}
documentation for semantics.

Optional \var{charset} serves two purposes: it has the same meaning as
the \var{charset} argument to the \method{append()} method.  It also
sets the default character set for all subsequent \method{append()}
calls that omit the \var{charset} argument.  If \var{charset} is not
provided in the constructor (the default), the \code{us-ascii}
character set is used both as \var{s}'s initial charset and as the
default for subsequent \method{append()} calls.

The maximum line length can be specified explicit via
\var{maxlinelen}.  For splitting the first line to a shorter value (to
account for the field header which isn't included in \var{s},
e.g. \mailheader{Subject}) pass in the name of the field in
\var{header_name}.  The default \var{maxlinelen} is 76, and the
default value for \var{header_name} is \code{None}, meaning it is not
taken into account for the first line of a long, split header.

Optional \var{continuation_ws} must be \rfc{2822}-compliant folding
whitespace, and is usually either a space or a hard tab character.
This character will be prepended to continuation lines.
\end{classdesc}

Optional \var{errors} is passed straight through to the
\method{append()} method.

\begin{methoddesc}[Header]{append}{s\optional{, charset\optional{, errors}}}
Append the string \var{s} to the MIME header.

Optional \var{charset}, if given, should be a \class{Charset} instance
(see \refmodule{email.Charset}) or the name of a character set, which
will be converted to a \class{Charset} instance.  A value of
\code{None} (the default) means that the \var{charset} given in the
constructor is used.

\var{s} may be a byte string or a Unicode string.  If it is a byte
string (i.e. \code{isinstance(s, str)} is true), then
\var{charset} is the encoding of that byte string, and a
\exception{UnicodeError} will be raised if the string cannot be
decoded with that character set.

If \var{s} is a Unicode string, then \var{charset} is a hint
specifying the character set of the characters in the string.  In this
case, when producing an \rfc{2822}-compliant header using \rfc{2047}
rules, the Unicode string will be encoded using the following charsets
in order: \code{us-ascii}, the \var{charset} hint, \code{utf-8}.  The
first character set to not provoke a \exception{UnicodeError} is used.

Optional \var{errors} is passed through to any \function{unicode()} or
\function{ustr.encode()} call, and defaults to ``strict''.
\end{methoddesc}

\begin{methoddesc}[Header]{encode}{\optional{splitchars}}
Encode a message header into an RFC-compliant format, possibly
wrapping long lines and encapsulating non-\ASCII{} parts in base64 or
quoted-printable encodings.  Optional \var{splitchars} is a string
containing characters to split long ASCII lines on, in rough support
of \rfc{2822}'s \emph{highest level syntactic breaks}.  This doesn't
affect \rfc{2047} encoded lines.
\end{methoddesc}

The \class{Header} class also provides a number of methods to support
standard operators and built-in functions.

\begin{methoddesc}[Header]{__str__}{}
A synonym for \method{Header.encode()}.  Useful for
\code{str(aHeader)}.
\end{methoddesc}

\begin{methoddesc}[Header]{__unicode__}{}
A helper for the built-in \function{unicode()} function.  Returns the
header as a Unicode string.
\end{methoddesc}

\begin{methoddesc}[Header]{__eq__}{other}
This method allows you to compare two \class{Header} instances for equality.
\end{methoddesc}

\begin{methoddesc}[Header]{__ne__}{other}
This method allows you to compare two \class{Header} instances for inequality.
\end{methoddesc}

The \module{email.Header} module also provides the following
convenient functions.

\begin{funcdesc}{decode_header}{header}
Decode a message header value without converting the character set.
The header value is in \var{header}.

This function returns a list of \code{(decoded_string, charset)} pairs
containing each of the decoded parts of the header.  \var{charset} is
\code{None} for non-encoded parts of the header, otherwise a lower
case string containing the name of the character set specified in the
encoded string.

Here's an example:

\begin{verbatim}
>>> from email.Header import decode_header
>>> decode_header('=?iso-8859-1?q?p=F6stal?=')
[('p\\xf6stal', 'iso-8859-1')]
\end{verbatim}
\end{funcdesc}

\begin{funcdesc}{make_header}{decoded_seq\optional{, maxlinelen\optional{,
    header_name\optional{, continuation_ws}}}}
Create a \class{Header} instance from a sequence of pairs as returned
by \function{decode_header()}.

\function{decode_header()} takes a header value string and returns a
sequence of pairs of the format \code{(decoded_string, charset)} where
\var{charset} is the name of the character set.

This function takes one of those sequence of pairs and returns a
\class{Header} instance.  Optional \var{maxlinelen},
\var{header_name}, and \var{continuation_ws} are as in the
\class{Header} constructor.
\end{funcdesc}

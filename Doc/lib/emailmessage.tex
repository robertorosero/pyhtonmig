\declaremodule{standard}{email.Message}
\modulesynopsis{The base class representing email messages.}

The central class in the \module{email} package is the
\class{Message} class; it is the base class for the \module{email}
object model.  \class{Message} provides the core functionality for
setting and querying header fields, and for accessing message bodies.

Conceptually, a \class{Message} object consists of \emph{headers} and
\emph{payloads}.  Headers are \rfc{2822} style field names and
values where the field name and value are separated by a colon.  The
colon is not part of either the field name or the field value.

Headers are stored and returned in case-preserving form but are
matched case-insensitively.  There may also be a single envelope
header, also known as the \emph{Unix-From} header or the
\code{From_} header.  The payload is either a string in the case of
simple message objects or a list of \class{Message} objects for
MIME container documents (e.g. \mimetype{multipart/*} and
\mimetype{message/rfc822}).

\class{Message} objects provide a mapping style interface for
accessing the message headers, and an explicit interface for accessing
both the headers and the payload.  It provides convenience methods for
generating a flat text representation of the message object tree, for
accessing commonly used header parameters, and for recursively walking
over the object tree.

Here are the methods of the \class{Message} class:

\begin{classdesc}{Message}{}
The constructor takes no arguments.
\end{classdesc}

\begin{methoddesc}[Message]{as_string}{\optional{unixfrom}}
Return the entire message flatten as a string.  When optional
\var{unixfrom} is \code{True}, the envelope header is included in the
returned string.  \var{unixfrom} defaults to \code{False}.

Note that this method is provided as a convenience and may not always
format the message the way you want.  For more flexibility,
instantiate a \class{Generator} instance and use its
\method{flatten()} method directly.  For example:

\begin{verbatim}
from cStringIO import StringIO
from email.Generator import Generator
fp = StringIO()
g = Generator(mangle_from_=False, maxheaderlen=60)
g.flatten(msg)
text = fp.getvalue()
\end{verbatim}
\end{methoddesc}

\begin{methoddesc}[Message]{__str__}{}
Equivalent to \method{as_string(unixfrom=True)}.
\end{methoddesc}

\begin{methoddesc}[Message]{is_multipart}{}
Return \code{True} if the message's payload is a list of
sub-\class{Message} objects, otherwise return \code{False}.  When
\method{is_multipart()} returns False, the payload should be a string
object.
\end{methoddesc}

\begin{methoddesc}[Message]{set_unixfrom}{unixfrom}
Set the message's envelope header to \var{unixfrom}, which should be a string.
\end{methoddesc}

\begin{methoddesc}[Message]{get_unixfrom}{}
Return the message's envelope header.  Defaults to \code{None} if the
envelope header was never set.
\end{methoddesc}

\begin{methoddesc}[Message]{attach}{payload}
Add the given \var{payload} to the current payload, which must be
\code{None} or a list of \class{Message} objects before the call.
After the call, the payload will always be a list of \class{Message}
objects.  If you want to set the payload to a scalar object (e.g. a
string), use \method{set_payload()} instead.
\end{methoddesc}

\begin{methoddesc}[Message]{get_payload}{\optional{i\optional{, decode}}}
Return a reference the current payload, which will be a list of
\class{Message} objects when \method{is_multipart()} is \code{True}, or a
string when \method{is_multipart()} is \code{False}.  If the
payload is a list and you mutate the list object, you modify the
message's payload in place.

With optional argument \var{i}, \method{get_payload()} will return the
\var{i}-th element of the payload, counting from zero, if
\method{is_multipart()} is \code{True}.  An \exception{IndexError}
will be raised if \var{i} is less than 0 or greater than or equal to
the number of items in the payload.  If the payload is a string
(i.e. \method{is_multipart()} is \code{False}) and \var{i} is given, a
\exception{TypeError} is raised.

Optional \var{decode} is a flag indicating whether the payload should be
decoded or not, according to the \mailheader{Content-Transfer-Encoding} header.
When \code{True} and the message is not a multipart, the payload will be
decoded if this header's value is \samp{quoted-printable} or
\samp{base64}.  If some other encoding is used, or
\mailheader{Content-Transfer-Encoding} header is
missing, or if the payload has bogus base64 data, the payload is
returned as-is (undecoded).  If the message is a multipart and the
\var{decode} flag is \code{True}, then \code{None} is returned.  The
default for \var{decode} is \code{False}.
\end{methoddesc}

\begin{methoddesc}[Message]{set_payload}{payload\optional{, charset}}
Set the entire message object's payload to \var{payload}.  It is the
client's responsibility to ensure the payload invariants.  Optional
\var{charset} sets the message's default character set; see
\method{set_charset()} for details.

\versionchanged[\var{charset} argument added]{2.2.2}
\end{methoddesc}

\begin{methoddesc}[Message]{set_charset}{charset}
Set the character set of the payload to \var{charset}, which can
either be a \class{Charset} instance (see \refmodule{email.Charset}), a
string naming a character set,
or \code{None}.  If it is a string, it will be converted to a
\class{Charset} instance.  If \var{charset} is \code{None}, the
\code{charset} parameter will be removed from the
\mailheader{Content-Type} header. Anything else will generate a
\exception{TypeError}.

The message will be assumed to be of type \mimetype{text/*} encoded with
\code{charset.input_charset}.  It will be converted to
\code{charset.output_charset}
and encoded properly, if needed, when generating the plain text
representation of the message.  MIME headers
(\mailheader{MIME-Version}, \mailheader{Content-Type},
\mailheader{Content-Transfer-Encoding}) will be added as needed.

\versionadded{2.2.2}
\end{methoddesc}

\begin{methoddesc}[Message]{get_charset}{}
Return the \class{Charset} instance associated with the message's payload.
\versionadded{2.2.2}
\end{methoddesc}

The following methods implement a mapping-like interface for accessing
the message's \rfc{2822} headers.  Note that there are some
semantic differences between these methods and a normal mapping
(i.e. dictionary) interface.  For example, in a dictionary there are
no duplicate keys, but here there may be duplicate message headers.  Also,
in dictionaries there is no guaranteed order to the keys returned by
\method{keys()}, but in a \class{Message} object, headers are always
returned in the order they appeared in the original message, or were
added to the message later.  Any header deleted and then re-added are
always appended to the end of the header list.

These semantic differences are intentional and are biased toward
maximal convenience.

Note that in all cases, any envelope header present in the message is
not included in the mapping interface.

\begin{methoddesc}[Message]{__len__}{}
Return the total number of headers, including duplicates.
\end{methoddesc}

\begin{methoddesc}[Message]{__contains__}{name}
Return true if the message object has a field named \var{name}.
Matching is done case-insensitively and \var{name} should not include the
trailing colon.  Used for the \code{in} operator,
e.g.:

\begin{verbatim}
if 'message-id' in myMessage:
    print 'Message-ID:', myMessage['message-id']
\end{verbatim}
\end{methoddesc}

\begin{methoddesc}[Message]{__getitem__}{name}
Return the value of the named header field.  \var{name} should not
include the colon field separator.  If the header is missing,
\code{None} is returned; a \exception{KeyError} is never raised.

Note that if the named field appears more than once in the message's
headers, exactly which of those field values will be returned is
undefined.  Use the \method{get_all()} method to get the values of all
the extant named headers.
\end{methoddesc}

\begin{methoddesc}[Message]{__setitem__}{name, val}
Add a header to the message with field name \var{name} and value
\var{val}.  The field is appended to the end of the message's existing
fields.

Note that this does \emph{not} overwrite or delete any existing header
with the same name.  If you want to ensure that the new header is the
only one present in the message with field name
\var{name}, delete the field first, e.g.:

\begin{verbatim}
del msg['subject']
msg['subject'] = 'Python roolz!'
\end{verbatim}
\end{methoddesc}

\begin{methoddesc}[Message]{__delitem__}{name}
Delete all occurrences of the field with name \var{name} from the
message's headers.  No exception is raised if the named field isn't
present in the headers.
\end{methoddesc}

\begin{methoddesc}[Message]{has_key}{name}
Return true if the message contains a header field named \var{name},
otherwise return false.
\end{methoddesc}

\begin{methoddesc}[Message]{keys}{}
Return a list of all the message's header field names.
\end{methoddesc}

\begin{methoddesc}[Message]{values}{}
Return a list of all the message's field values.
\end{methoddesc}

\begin{methoddesc}[Message]{items}{}
Return a list of 2-tuples containing all the message's field headers
and values.
\end{methoddesc}

\begin{methoddesc}[Message]{get}{name\optional{, failobj}}
Return the value of the named header field.  This is identical to
\method{__getitem__()} except that optional \var{failobj} is returned
if the named header is missing (defaults to \code{None}).
\end{methoddesc}

Here are some additional useful methods:

\begin{methoddesc}[Message]{get_all}{name\optional{, failobj}}
Return a list of all the values for the field named \var{name}.
If there are no such named headers in the message, \var{failobj} is
returned (defaults to \code{None}).
\end{methoddesc}

\begin{methoddesc}[Message]{add_header}{_name, _value, **_params}
Extended header setting.  This method is similar to
\method{__setitem__()} except that additional header parameters can be
provided as keyword arguments.  \var{_name} is the header field to add
and \var{_value} is the \emph{primary} value for the header.

For each item in the keyword argument dictionary \var{_params}, the
key is taken as the parameter name, with underscores converted to
dashes (since dashes are illegal in Python identifiers).  Normally,
the parameter will be added as \code{key="value"} unless the value is
\code{None}, in which case only the key will be added.

Here's an example:

\begin{verbatim}
msg.add_header('Content-Disposition', 'attachment', filename='bud.gif')
\end{verbatim}

This will add a header that looks like

\begin{verbatim}
Content-Disposition: attachment; filename="bud.gif"
\end{verbatim}
\end{methoddesc}

\begin{methoddesc}[Message]{replace_header}{_name, _value}
Replace a header.  Replace the first header found in the message that
matches \var{_name}, retaining header order and field name case.  If
no matching header was found, a \exception{KeyError} is raised.

\versionadded{2.2.2}
\end{methoddesc}

\begin{methoddesc}[Message]{get_content_type}{}
Return the message's content type.  The returned string is coerced to
lower case of the form \mimetype{maintype/subtype}.  If there was no
\mailheader{Content-Type} header in the message the default type as
given by \method{get_default_type()} will be returned.  Since
according to \rfc{2045}, messages always have a default type,
\method{get_content_type()} will always return a value.

\rfc{2045} defines a message's default type to be
\mimetype{text/plain} unless it appears inside a
\mimetype{multipart/digest} container, in which case it would be
\mimetype{message/rfc822}.  If the \mailheader{Content-Type} header
has an invalid type specification, \rfc{2045} mandates that the
default type be \mimetype{text/plain}.

\versionadded{2.2.2}
\end{methoddesc}

\begin{methoddesc}[Message]{get_content_maintype}{}
Return the message's main content type.  This is the
\mimetype{maintype} part of the string returned by
\method{get_content_type()}.

\versionadded{2.2.2}
\end{methoddesc}

\begin{methoddesc}[Message]{get_content_subtype}{}
Return the message's sub-content type.  This is the \mimetype{subtype}
part of the string returned by \method{get_content_type()}.

\versionadded{2.2.2}
\end{methoddesc}

\begin{methoddesc}[Message]{get_default_type}{}
Return the default content type.  Most messages have a default content
type of \mimetype{text/plain}, except for messages that are subparts
of \mimetype{multipart/digest} containers.  Such subparts have a
default content type of \mimetype{message/rfc822}.

\versionadded{2.2.2}
\end{methoddesc}

\begin{methoddesc}[Message]{set_default_type}{ctype}
Set the default content type.  \var{ctype} should either be
\mimetype{text/plain} or \mimetype{message/rfc822}, although this is
not enforced.  The default content type is not stored in the
\mailheader{Content-Type} header.

\versionadded{2.2.2}
\end{methoddesc}

\begin{methoddesc}[Message]{get_params}{\optional{failobj\optional{,
    header\optional{, unquote}}}}
Return the message's \mailheader{Content-Type} parameters, as a list.  The
elements of the returned list are 2-tuples of key/value pairs, as
split on the \character{=} sign.  The left hand side of the
\character{=} is the key, while the right hand side is the value.  If
there is no \character{=} sign in the parameter the value is the empty
string, otherwise the value is as described in \method{get_param()} and is
unquoted if optional \var{unquote} is \code{True} (the default).

Optional \var{failobj} is the object to return if there is no
\mailheader{Content-Type} header.  Optional \var{header} is the header to
search instead of \mailheader{Content-Type}.

\versionchanged[\var{unquote} argument added]{2.2.2}
\end{methoddesc}

\begin{methoddesc}[Message]{get_param}{param\optional{,
    failobj\optional{, header\optional{, unquote}}}}
Return the value of the \mailheader{Content-Type} header's parameter
\var{param} as a string.  If the message has no \mailheader{Content-Type}
header or if there is no such parameter, then \var{failobj} is
returned (defaults to \code{None}).

Optional \var{header} if given, specifies the message header to use
instead of \mailheader{Content-Type}.

Parameter keys are always compared case insensitively.  The return
value can either be a string, or a 3-tuple if the parameter was
\rfc{2231} encoded.  When it's a 3-tuple, the elements of the value are of
the form \code{(CHARSET, LANGUAGE, VALUE)}, where \code{LANGUAGE} may
be the empty string.  Your application should be prepared to deal with
3-tuple return values, which it can convert to a Unicode string like
so:

\begin{verbatim}
param = msg.get_param('foo')
if isinstance(param, tuple):
    param = unicode(param[2], param[0])
\end{verbatim}

In any case, the parameter value (either the returned string, or the
\code{VALUE} item in the 3-tuple) is always unquoted, unless
\var{unquote} is set to \code{False}.

\versionchanged[\var{unquote} argument added, and 3-tuple return value
possible]{2.2.2}
\end{methoddesc}

\begin{methoddesc}[Message]{set_param}{param, value\optional{,
    header\optional{, requote\optional{, charset\optional{, language}}}}}

Set a parameter in the \mailheader{Content-Type} header.  If the
parameter already exists in the header, its value will be replaced
with \var{value}.  If the \mailheader{Content-Type} header as not yet
been defined for this message, it will be set to \mimetype{text/plain}
and the new parameter value will be appended as per \rfc{2045}.

Optional \var{header} specifies an alternative header to
\mailheader{Content-Type}, and all parameters will be quoted as
necessary unless optional \var{requote} is \code{False} (the default
is \code{True}).

If optional \var{charset} is specified, the parameter will be encoded
according to \rfc{2231}. Optional \var{language} specifies the RFC
2231 language, defaulting to the empty string.  Both \var{charset} and
\var{language} should be strings.

\versionadded{2.2.2}
\end{methoddesc}

\begin{methoddesc}[Message]{del_param}{param\optional{, header\optional{,
    requote}}}
Remove the given parameter completely from the
\mailheader{Content-Type} header.  The header will be re-written in
place without the parameter or its value.  All values will be quoted
as necessary unless \var{requote} is \code{False} (the default is
\code{True}).  Optional \var{header} specifies an alternative to
\mailheader{Content-Type}.

\versionadded{2.2.2}
\end{methoddesc}

\begin{methoddesc}[Message]{set_type}{type\optional{, header}\optional{,
    requote}}
Set the main type and subtype for the \mailheader{Content-Type}
header. \var{type} must be a string in the form
\mimetype{maintype/subtype}, otherwise a \exception{ValueError} is
raised.

This method replaces the \mailheader{Content-Type} header, keeping all
the parameters in place.  If \var{requote} is \code{False}, this
leaves the existing header's quoting as is, otherwise the parameters
will be quoted (the default).

An alternative header can be specified in the \var{header} argument.
When the \mailheader{Content-Type} header is set a
\mailheader{MIME-Version} header is also added.

\versionadded{2.2.2}
\end{methoddesc}

\begin{methoddesc}[Message]{get_filename}{\optional{failobj}}
Return the value of the \code{filename} parameter of the
\mailheader{Content-Disposition} header of the message, or \var{failobj} if
either the header is missing, or has no \code{filename} parameter.
The returned string will always be unquoted as per
\method{Utils.unquote()}.
\end{methoddesc}

\begin{methoddesc}[Message]{get_boundary}{\optional{failobj}}
Return the value of the \code{boundary} parameter of the
\mailheader{Content-Type} header of the message, or \var{failobj} if either
the header is missing, or has no \code{boundary} parameter.  The
returned string will always be unquoted as per
\method{Utils.unquote()}.
\end{methoddesc}

\begin{methoddesc}[Message]{set_boundary}{boundary}
Set the \code{boundary} parameter of the \mailheader{Content-Type}
header to \var{boundary}.  \method{set_boundary()} will always quote
\var{boundary} if necessary.  A \exception{HeaderParseError} is raised
if the message object has no \mailheader{Content-Type} header.

Note that using this method is subtly different than deleting the old
\mailheader{Content-Type} header and adding a new one with the new boundary
via \method{add_header()}, because \method{set_boundary()} preserves the
order of the \mailheader{Content-Type} header in the list of headers.
However, it does \emph{not} preserve any continuation lines which may
have been present in the original \mailheader{Content-Type} header.
\end{methoddesc}

\begin{methoddesc}[Message]{get_content_charset}{\optional{failobj}}
Return the \code{charset} parameter of the \mailheader{Content-Type}
header, coerced to lower case.  If there is no
\mailheader{Content-Type} header, or if that header has no
\code{charset} parameter, \var{failobj} is returned.

Note that this method differs from \method{get_charset()} which
returns the \class{Charset} instance for the default encoding of the
message body.

\versionadded{2.2.2}
\end{methoddesc}

\begin{methoddesc}[Message]{get_charsets}{\optional{failobj}}
Return a list containing the character set names in the message.  If
the message is a \mimetype{multipart}, then the list will contain one
element for each subpart in the payload, otherwise, it will be a list
of length 1.

Each item in the list will be a string which is the value of the
\code{charset} parameter in the \mailheader{Content-Type} header for the
represented subpart.  However, if the subpart has no
\mailheader{Content-Type} header, no \code{charset} parameter, or is not of
the \mimetype{text} main MIME type, then that item in the returned list
will be \var{failobj}.
\end{methoddesc}

\begin{methoddesc}[Message]{walk}{}
The \method{walk()} method is an all-purpose generator which can be
used to iterate over all the parts and subparts of a message object
tree, in depth-first traversal order.  You will typically use
\method{walk()} as the iterator in a \code{for} loop; each
iteration returns the next subpart.

Here's an example that prints the MIME type of every part of a
multipart message structure:

\begin{verbatim}
>>> for part in msg.walk():
>>>     print part.get_content_type()
multipart/report
text/plain
message/delivery-status
text/plain
text/plain
message/rfc822
\end{verbatim}
\end{methoddesc}

\class{Message} objects can also optionally contain two instance
attributes, which can be used when generating the plain text of a MIME
message.

\begin{datadesc}{preamble}
The format of a MIME document allows for some text between the blank
line following the headers, and the first multipart boundary string.
Normally, this text is never visible in a MIME-aware mail reader
because it falls outside the standard MIME armor.  However, when
viewing the raw text of the message, or when viewing the message in a
non-MIME aware reader, this text can become visible.

The \var{preamble} attribute contains this leading extra-armor text
for MIME documents.  When the \class{Parser} discovers some text after
the headers but before the first boundary string, it assigns this text
to the message's \var{preamble} attribute.  When the \class{Generator}
is writing out the plain text representation of a MIME message, and it
finds the message has a \var{preamble} attribute, it will write this
text in the area between the headers and the first boundary.  See
\refmodule{email.Parser} and \refmodule{email.Generator} for details.

Note that if the message object has no preamble, the
\var{preamble} attribute will be \code{None}.
\end{datadesc}

\begin{datadesc}{epilogue}
The \var{epilogue} attribute acts the same way as the \var{preamble}
attribute, except that it contains text that appears between the last
boundary and the end of the message.

One note: when generating the flat text for a \mimetype{multipart}
message that has no \var{epilogue} (using the standard
\class{Generator} class), no newline is added after the closing
boundary line.  If the message object has an \var{epilogue} and its
value does not start with a newline, a newline is printed after the
closing boundary.  This seems a little clumsy, but it makes the most
practical sense.  The upshot is that if you want to ensure that a
newline get printed after your closing \mimetype{multipart} boundary,
set the \var{epilogue} to the empty string.
\end{datadesc}

\subsubsection{Deprecated methods}

The following methods are deprecated in \module{email} version 2.
They are documented here for completeness.

\begin{methoddesc}[Message]{add_payload}{payload}
Add \var{payload} to the message object's existing payload.  If, prior
to calling this method, the object's payload was \code{None}
(i.e. never before set), then after this method is called, the payload
will be the argument \var{payload}.

If the object's payload was already a list
(i.e. \method{is_multipart()} returns 1), then \var{payload} is
appended to the end of the existing payload list.

For any other type of existing payload, \method{add_payload()} will
transform the new payload into a list consisting of the old payload
and \var{payload}, but only if the document is already a MIME
multipart document.  This condition is satisfied if the message's
\mailheader{Content-Type} header's main type is either
\mimetype{multipart}, or there is no \mailheader{Content-Type}
header.  In any other situation,
\exception{MultipartConversionError} is raised.

\deprecated{2.2.2}{Use the \method{attach()} method instead.}
\end{methoddesc}

\begin{methoddesc}[Message]{get_type}{\optional{failobj}}
Return the message's content type, as a string of the form
\mimetype{maintype/subtype} as taken from the
\mailheader{Content-Type} header.
The returned string is coerced to lowercase.

If there is no \mailheader{Content-Type} header in the message,
\var{failobj} is returned (defaults to \code{None}).

\deprecated{2.2.2}{Use the \method{get_content_type()} method instead.}
\end{methoddesc}

\begin{methoddesc}[Message]{get_main_type}{\optional{failobj}}
Return the message's \emph{main} content type.  This essentially returns the
\var{maintype} part of the string returned by \method{get_type()}, with the
same semantics for \var{failobj}.

\deprecated{2.2.2}{Use the \method{get_content_maintype()} method instead.}
\end{methoddesc}

\begin{methoddesc}[Message]{get_subtype}{\optional{failobj}}
Return the message's sub-content type.  This essentially returns the
\var{subtype} part of the string returned by \method{get_type()}, with the
same semantics for \var{failobj}.

\deprecated{2.2.2}{Use the \method{get_content_subtype()} method instead.}
\end{methoddesc}


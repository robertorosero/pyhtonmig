\section{\module{mimetypes} ---
         Map filenames to MIME types}

\declaremodule{standard}{mimetypes}
\modulesynopsis{Mapping of filename extensions to MIME types.}
\sectionauthor{Fred L. Drake, Jr.}{fdrake@acm.org}


\indexii{MIME}{content type}

The \module{mimetypes} module converts between a filename or URL and
the MIME type associated with the filename extension.  Conversions are
provided from filename to MIME type and from MIME type to filename
extension; encodings are not supported for the latter conversion.

The module provides one class and a number of convenience functions.
The functions are the normal interface to this module, but some
applications may be interested in the class as well.

The functions described below provide the primary interface for this
module.  If the module has not been initialized, they will call
\function{init()} if they rely on the information \function{init()}
sets up.


\begin{funcdesc}{guess_type}{filename\optional{, strict}}
Guess the type of a file based on its filename or URL, given by
\var{filename}.  The return value is a tuple \code{(\var{type},
\var{encoding})} where \var{type} is \code{None} if the type can't be
guessed (missing or unknown suffix) or a string of the form
\code{'\var{type}/\var{subtype}'}, usable for a MIME
\mailheader{content-type} header\indexii{MIME}{headers}.

\var{encoding} is \code{None} for no encoding or the name of the
program used to encode (e.g. \program{compress} or \program{gzip}).
The encoding is suitable for use as a \mailheader{Content-Encoding}
header, \emph{not} as a \mailheader{Content-Transfer-Encoding} header.
The mappings are table driven.  Encoding suffixes are case sensitive;
type suffixes are first tried case sensitively, then case
insensitively.

Optional \var{strict} is a flag specifying whether the list of known
MIME types is limited to only the official types \ulink{registered
with IANA}{http://www.isi.edu/in-notes/iana/assignments/media-types}
are recognized.  When \var{strict} is true (the default), only the
IANA types are supported; when \var{strict} is false, some additional
non-standard but commonly used MIME types are also recognized.
\end{funcdesc}

\begin{funcdesc}{guess_all_extensions}{type\optional{, strict}}
Guess the extensions for a file based on its MIME type, given by
\var{type}.
The return value is a list of strings giving all possible filename extensions,
including the leading dot (\character{.}).  The extensions are not guaranteed
to have been associated with any particular data stream, but would be mapped
to the MIME type \var{type} by \function{guess_type()}.  If no extension can
be guessed for \var{type}, \code{None} is returned.

Optional \var{strict} has the same meaning as with the
\function{guess_type()} function.
\end{funcdesc}


\begin{funcdesc}{guess_extension}{type\optional{, strict}}
Guess the extension for a file based on its MIME type, given by
\var{type}.
The return value is a string giving a filename extension, including the
leading dot (\character{.}).  The extension is not guaranteed to have been
associated with any particular data stream, but would be mapped to the 
MIME type \var{type} by \function{guess_type()}.  If no extension can
be guessed for \var{type}, \code{None} is returned.

Optional \var{strict} has the same meaning as with the
\function{guess_type()} function.
\end{funcdesc}


Some additional functions and data items are available for controlling
the behavior of the module.


\begin{funcdesc}{init}{\optional{files}}
Initialize the internal data structures.  If given, \var{files} must
be a sequence of file names which should be used to augment the
default type map.  If omitted, the file names to use are taken from
\constant{knownfiles}.  Each file named in \var{files} or
\constant{knownfiles} takes precedence over those named before it.
Calling \function{init()} repeatedly is allowed.
\end{funcdesc}

\begin{funcdesc}{read_mime_types}{filename}
Load the type map given in the file \var{filename}, if it exists.  The 
type map is returned as a dictionary mapping filename extensions,
including the leading dot (\character{.}), to strings of the form
\code{'\var{type}/\var{subtype}'}.  If the file \var{filename} does
not exist or cannot be read, \code{None} is returned.
\end{funcdesc}


\begin{funcdesc}{add_type}{type, ext\optional{, strict}}
Add a mapping from the mimetype \var{type} to the extension \var{ext}.
When the extension is already known, the new type will replace the old
one. When the type is already known the extension will be added
to the list of known extensions.

When \var{strict} is the mapping will added to the official
MIME types, otherwise to the non-standard ones.
\end{funcdesc}


\begin{datadesc}{inited}
Flag indicating whether or not the global data structures have been
initialized.  This is set to true by \function{init()}.
\end{datadesc}

\begin{datadesc}{knownfiles}
List of type map file names commonly installed.  These files are
typically named \file{mime.types} and are installed in different
locations by different packages.\index{file!mime.types}
\end{datadesc}

\begin{datadesc}{suffix_map}
Dictionary mapping suffixes to suffixes.  This is used to allow
recognition of encoded files for which the encoding and the type are
indicated by the same extension.  For example, the \file{.tgz}
extension is mapped to \file{.tar.gz} to allow the encoding and type
to be recognized separately.
\end{datadesc}

\begin{datadesc}{encodings_map}
Dictionary mapping filename extensions to encoding types.
\end{datadesc}

\begin{datadesc}{types_map}
Dictionary mapping filename extensions to MIME types.
\end{datadesc}

\begin{datadesc}{common_types}
Dictionary mapping filename extensions to non-standard, but commonly
found MIME types.
\end{datadesc}


The \class{MimeTypes} class may be useful for applications which may
want more than one MIME-type database:

\begin{classdesc}{MimeTypes}{\optional{filenames}}
  This class represents a MIME-types database.  By default, it
  provides access to the same database as the rest of this module.
  The initial database is a copy of that provided by the module, and
  may be extended by loading additional \file{mime.types}-style files
  into the database using the \method{read()} or \method{readfp()}
  methods.  The mapping dictionaries may also be cleared before
  loading additional data if the default data is not desired.

  The optional \var{filenames} parameter can be used to cause
  additional files to be loaded ``on top'' of the default database.

  \versionadded{2.2}
\end{classdesc}


\subsection{MimeTypes Objects \label{mimetypes-objects}}

\class{MimeTypes} instances provide an interface which is very like
that of the \refmodule{mimetypes} module.

\begin{datadesc}{suffix_map}
  Dictionary mapping suffixes to suffixes.  This is used to allow
  recognition of encoded files for which the encoding and the type are
  indicated by the same extension.  For example, the \file{.tgz}
  extension is mapped to \file{.tar.gz} to allow the encoding and type
  to be recognized separately.  This is initially a copy of the global
  \code{suffix_map} defined in the module.
\end{datadesc}

\begin{datadesc}{encodings_map}
  Dictionary mapping filename extensions to encoding types.  This is
  initially a copy of the global \code{encodings_map} defined in the
  module.
\end{datadesc}

\begin{datadesc}{types_map}
  Dictionary mapping filename extensions to MIME types.  This is
  initially a copy of the global \code{types_map} defined in the
  module.
\end{datadesc}

\begin{datadesc}{common_types}
  Dictionary mapping filename extensions to non-standard, but commonly
  found MIME types.  This is initially a copy of the global
  \code{common_types} defined in the module.
\end{datadesc}

\begin{methoddesc}{guess_extension}{type\optional{, strict}}
  Similar to the \function{guess_extension()} function, using the
  tables stored as part of the object.
\end{methoddesc}

\begin{methoddesc}{guess_type}{url\optional{, strict}}
  Similar to the \function{guess_type()} function, using the tables
  stored as part of the object.
\end{methoddesc}

\begin{methoddesc}{read}{path}
  Load MIME information from a file named \var{path}.  This uses
  \method{readfp()} to parse the file.
\end{methoddesc}

\begin{methoddesc}{readfp}{file}
  Load MIME type information from an open file.  The file must have
  the format of the standard \file{mime.types} files.
\end{methoddesc}

\section{\module{inspect} ---
         Inspect live objects}

\declaremodule{standard}{inspect}
\modulesynopsis{Extract information and source code from live objects.}
\moduleauthor{Ka-Ping Yee}{ping@lfw.org}
\sectionauthor{Ka-Ping Yee}{ping@lfw.org}

\versionadded{2.1}

The \module{inspect} module provides several useful functions
to help get information about live objects such as modules,
classes, methods, functions, tracebacks, frame objects, and
code objects.  For example, it can help you examine the
contents of a class, retrieve the source code of a method,
extract and format the argument list for a function, or
get all the information you need to display a detailed traceback.

There are four main kinds of services provided by this module:
type checking, getting source code, inspecting classes
and functions, and examining the interpreter stack.

\subsection{Types and members
            \label{inspect-types}}

The \function{getmembers()} function retrieves the members
of an object such as a class or module.
The nine functions whose names begin with ``is'' are mainly
provided as convenient choices for the second argument to
\function{getmembers()}.  They also help you determine when
you can expect to find the following special attributes:

\begin{tableiii}{c|l|l}{}{Type}{Attribute}{Description}
  \lineiii{module}{__doc__}{documentation string}
  \lineiii{}{__file__}{filename (missing for built-in modules)}
  \hline
  \lineiii{class}{__doc__}{documentation string}
  \lineiii{}{__module__}{name of module in which this class was defined}
  \hline
  \lineiii{method}{__doc__}{documentation string}
  \lineiii{}{__name__}{name with which this method was defined}
  \lineiii{}{im_class}{class object in which this method belongs}
  \lineiii{}{im_func}{function object containing implementation of method}
  \lineiii{}{im_self}{instance to which this method is bound, or \code{None}}
  \hline
  \lineiii{function}{__doc__}{documentation string}
  \lineiii{}{__name__}{name with which this function was defined}
  \lineiii{}{func_code}{code object containing compiled function bytecode}
  \lineiii{}{func_defaults}{tuple of any default values for arguments}
  \lineiii{}{func_doc}{(same as __doc__)}
  \lineiii{}{func_globals}{global namespace in which this function was defined}
  \lineiii{}{func_name}{(same as __name__)}
  \hline
  \lineiii{traceback}{tb_frame}{frame object at this level}
  \lineiii{}{tb_lasti}{index of last attempted instruction in bytecode}
  \lineiii{}{tb_lineno}{current line number in Python source code}
  \lineiii{}{tb_next}{next inner traceback object (called by this level)}
  \hline
  \lineiii{frame}{f_back}{next outer frame object (this frame's caller)}
  \lineiii{}{f_builtins}{built-in namespace seen by this frame}
  \lineiii{}{f_code}{code object being executed in this frame}
  \lineiii{}{f_exc_traceback}{traceback if raised in this frame, or \code{None}}
  \lineiii{}{f_exc_type}{exception type if raised in this frame, or \code{None}}
  \lineiii{}{f_exc_value}{exception value if raised in this frame, or \code{None}}
  \lineiii{}{f_globals}{global namespace seen by this frame}
  \lineiii{}{f_lasti}{index of last attempted instruction in bytecode}
  \lineiii{}{f_lineno}{current line number in Python source code}
  \lineiii{}{f_locals}{local namespace seen by this frame}
  \lineiii{}{f_restricted}{0 or 1 if frame is in restricted execution mode}
  \lineiii{}{f_trace}{tracing function for this frame, or \code{None}}
  \hline
  \lineiii{code}{co_argcount}{number of arguments (not including * or ** args)}
  \lineiii{}{co_code}{string of raw compiled bytecode}
  \lineiii{}{co_consts}{tuple of constants used in the bytecode}
  \lineiii{}{co_filename}{name of file in which this code object was created}
  \lineiii{}{co_firstlineno}{number of first line in Python source code}
  \lineiii{}{co_flags}{bitmap: 1=optimized \code{|} 2=newlocals \code{|} 4=*arg \code{|} 8=**arg}
  \lineiii{}{co_lnotab}{encoded mapping of line numbers to bytecode indices}
  \lineiii{}{co_name}{name with which this code object was defined}
  \lineiii{}{co_names}{tuple of names of local variables}
  \lineiii{}{co_nlocals}{number of local variables}
  \lineiii{}{co_stacksize}{virtual machine stack space required}
  \lineiii{}{co_varnames}{tuple of names of arguments and local variables}
  \hline
  \lineiii{builtin}{__doc__}{documentation string}
  \lineiii{}{__name__}{original name of this function or method}
  \lineiii{}{__self__}{instance to which a method is bound, or \code{None}}
\end{tableiii}

\begin{funcdesc}{getmembers}{object\optional{, predicate}}
  Return all the members of an object in a list of (name, value) pairs
  sorted by name.  If the optional \var{predicate} argument is supplied,
  only members for which the predicate returns a true value are included.
\end{funcdesc}

\begin{funcdesc}{ismodule}{object}
  Return true if the object is a module.
\end{funcdesc}

\begin{funcdesc}{isclass}{object}
  Return true if the object is a class.
\end{funcdesc}

\begin{funcdesc}{ismethod}{object}
  Return true if the object is a method.
\end{funcdesc}

\begin{funcdesc}{isfunction}{object}
  Return true if the object is a Python function or unnamed (lambda) function.
\end{funcdesc}

\begin{funcdesc}{istraceback}{object}
  Return true if the object is a traceback.
\end{funcdesc}

\begin{funcdesc}{isframe}{object}
  Return true if the object is a frame.
\end{funcdesc}

\begin{funcdesc}{iscode}{object}
  Return true if the object is a code.
\end{funcdesc}

\begin{funcdesc}{isbuiltin}{object}
  Return true if the object is a built-in function.
\end{funcdesc}

\begin{funcdesc}{isroutine}{object}
  Return true if the object is a user-defined or built-in function or method.
\end{funcdesc}

\subsection{Retrieving source code
            \label{inspect-source}}

\begin{funcdesc}{getdoc}{object}
  Get the documentation string for an object.
  All tabs are expanded to spaces.  To clean up docstrings that are
  indented to line up with blocks of code, any whitespace than can be
  uniformly removed from the second line onwards is removed.
\end{funcdesc}

\begin{funcdesc}{getcomments}{object}
  Return in a single string any lines of comments immediately preceding
  the object's source code (for a class, function, or method), or at the
  top of the Python source file (if the object is a module).
\end{funcdesc}

\begin{funcdesc}{getfile}{object}
  Return the name of the (text or binary) file in which an object was
  defined.  This will fail with a \exception{TypeError} if the object
  is a built-in module, class, or function.
\end{funcdesc}

\begin{funcdesc}{getmodule}{object}
  Try to guess which module an object was defined in.
\end{funcdesc}

\begin{funcdesc}{getsourcefile}{object}
  Return the name of the Python source file in which an object was
  defined.  This will fail with a \exception{TypeError} if the object
  is a built-in module, class, or function.
\end{funcdesc}

\begin{funcdesc}{getsourcelines}{object}
  Return a list of source lines and starting line number for an object.
  The argument may be a module, class, method, function, traceback, frame,
  or code object.  The source code is returned as a list of the lines
  corresponding to the object and the line number indicates where in the
  original source file the first line of code was found.  An
  \exception{IOError} is raised if the source code cannot be retrieved.
\end{funcdesc}

\begin{funcdesc}{getsource}{object}
  Return the text of the source code for an object.
  The argument may be a module, class, method, function, traceback, frame,
  or code object.  The source code is returned as a single string.  An
  \exception{IOError} is raised if the source code cannot be retrieved.
\end{funcdesc}

\subsection{Classes and functions
            \label{inspect-classes-functions}}

\begin{funcdesc}{getclasstree}{classes\optional{, unique}}
  Arrange the given list of classes into a hierarchy of nested lists.
  Where a nested list appears, it contains classes derived from the class
  whose entry immediately precedes the list.  Each entry is a 2-tuple
  containing a class and a tuple of its base classes.  If the \var{unique}
  argument is true, exactly one entry appears in the returned structure
  for each class in the given list.  Otherwise, classes using multiple
  inheritance and their descendants will appear multiple times.
\end{funcdesc}

\begin{funcdesc}{getargspec}{func}
  Get the names and default values of a function's arguments.
  A tuple of four things is returned: \code{(\var{args},
    \var{varargs}, \var{varkw}, \var{defaults})}.
  \var{args} is a list of the argument names (it may contain nested lists).
  \var{varargs} and \var{varkw} are the names of the \code{*} and
  \code{**} arguments or \code{None}.
  \var{defaults} is a tuple of default argument values; if this tuple
  has \var{n} elements, they correspond to the last \var{n} elements
  listed in \var{args}.
\end{funcdesc}

\begin{funcdesc}{getargvalues}{frame}
  Get information about arguments passed into a particular frame.
  A tuple of four things is returned: \code{(\var{args},
    \var{varargs}, \var{varkw}, \var{locals})}.
  \var{args} is a list of the argument names (it may contain nested
  lists).
  \var{varargs} and \var{varkw} are the names of the \code{*} and
  \code{**} arguments or \code{None}.
  \var{locals} is the locals dictionary of the given frame.
\end{funcdesc}

\begin{funcdesc}{formatargspec}{args\optional{, varargs, varkw, defaults,
      argformat, varargsformat, varkwformat, defaultformat}}
  
  Format a pretty argument spec from the four values returned by
  \function{getargspec()}.  The other four arguments are the
  corresponding optional formatting functions that are called to turn
  names and values into strings.
\end{funcdesc}

\begin{funcdesc}{formatargvalues}{args\optional{, varargs, varkw, locals,
      argformat, varargsformat, varkwformat, valueformat}}
  Format a pretty argument spec from the four values returned by
  \function{getargvalues()}.  The other four arguments are the
  corresponding optional formatting functions that are called to turn
  names and values into strings.
\end{funcdesc}

\subsection{The interpreter stack
            \label{inspect-stack}}

When the following functions return ``frame records,'' each record
is a tuple of six items: the frame object, the filename,
the line number of the current line, the function name, a list of
lines of context from the source code, and the index of the current
line within that list.
The optional \var{context} argument specifies the number of lines of
context to return, which are centered around the current line.

\begin{funcdesc}{getouterframes}{frame\optional{, context}}
  Get a list of frame records for a frame and all higher (calling)
  frames.
\end{funcdesc}

\begin{funcdesc}{getinnerframes}{traceback\optional{, context}}
  Get a list of frame records for a traceback's frame and all lower
  frames.
\end{funcdesc}

\begin{funcdesc}{currentframe}{}
  Return the frame object for the caller's stack frame.
\end{funcdesc}

\begin{funcdesc}{stack}{\optional{context}}
  Return a list of frame records for the stack above the caller's
  frame.
\end{funcdesc}

\begin{funcdesc}{trace}{\optional{context}}
  Return a list of frame records for the stack below the current
  exception.
\end{funcdesc}

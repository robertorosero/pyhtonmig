% THIS FILE IS AUTO-GENERATED!  DO NOT EDIT!
% (Your changes will be lost the next time it is generated.)
\section{\module{optparse} --- More powerful command line option parser}
\declaremodule{standard}{optparse}
\moduleauthor{Greg Ward}{gward@python.net}
\modulesynopsis{More convenient, flexible, and powerful command-line parsing library.}
\versionadded{2.3}
\sectionauthor{Greg Ward}{gward@python.net}
% An intro blurb used only when generating LaTeX docs for the Python
% manual (based on README.txt). 

\code{optparse} is a more convenient, flexible, and powerful library for
parsing command-line options than \code{getopt}.  \code{optparse} uses a more
declarative style of command-line parsing: you create an instance of
\class{OptionParser}, populate it with options, and parse the command line.
\code{optparse} allows users to specify options in the conventional GNU/POSIX
syntax, and additionally generates usage and help messages for you.

Here's an example of using \code{optparse} in a simple script:
\begin{verbatim}
from optparse import OptionParser
[...]
parser = OptionParser()
parser.add_option("-f", "--file", dest="filename",
                  help="write report to FILE", metavar="FILE")
parser.add_option("-q", "--quiet",
                  action="store_false", dest="verbose", default=True,
                  help="don't print status messages to stdout")

(options, args) = parser.parse_args()
\end{verbatim}

With these few lines of code, users of your script can now do the
``usual thing'' on the command-line, for example:
\begin{verbatim}
<yourscript> --file=outfile -q
\end{verbatim}

As it parses the command line, \code{optparse} sets attributes of the
\code{options} object returned by \method{parse{\_}args()} based on user-supplied
command-line values.  When \method{parse{\_}args()} returns from parsing this
command line, \code{options.filename} will be \code{"outfile"} and
\code{options.verbose} will be \code{False}.  \code{optparse} supports both long
and short options, allows short options to be merged together, and
allows options to be associated with their arguments in a variety of
ways.  Thus, the following command lines are all equivalent to the above
example:
\begin{verbatim}
<yourscript> -f outfile --quiet
<yourscript> --quiet --file outfile
<yourscript> -q -foutfile
<yourscript> -qfoutfile
\end{verbatim}

Additionally, users can run one of
\begin{verbatim}
<yourscript> -h
<yourscript> --help
\end{verbatim}

and \code{optparse} will print out a brief summary of your script's
options:
\begin{verbatim}
usage: <yourscript> [options]

options:
  -h, --help            show this help message and exit
  -f FILE, --file=FILE  write report to FILE
  -q, --quiet           don't print status messages to stdout
\end{verbatim}

where the value of \emph{yourscript} is determined at runtime (normally
from \code{sys.argv{[}0]}).
% $Id: intro.txt 413 2004-09-28 00:59:13Z greg $ 


\subsection{Background\label{optparse-background}}

\module{optparse} was explicitly designed to encourage the creation of programs with
straightforward, conventional command-line interfaces.  To that end, it
supports only the most common command-line syntax and semantics
conventionally used under \UNIX{}.  If you are unfamiliar with these
conventions, read this section to acquaint yourself with them.


\subsubsection{Terminology\label{optparse-terminology}}
\begin{description}
\item[argument]
a string entered on the command-line, and passed by the shell to
\code{execl()} or \code{execv()}.  In Python, arguments are elements of
\code{sys.argv{[}1:]} (\code{sys.argv{[}0]} is the name of the program being
executed).  \UNIX{} shells also use the term ``word''.

It is occasionally desirable to substitute an argument list other
than \code{sys.argv{[}1:]}, so you should read ``argument'' as ``an element of
\code{sys.argv{[}1:]}, or of some other list provided as a substitute for
\code{sys.argv{[}1:]}''.
\item[option   ]
an argument used to supply extra information to guide or customize the
execution of a program.  There are many different syntaxes for
options; the traditional \UNIX{} syntax is a hyphen (``-'') followed by a
single letter, e.g. \code{"-x"} or \code{"-F"}.  Also, traditional \UNIX{}
syntax allows multiple options to be merged into a single argument,
e.g.  \code{"-x -F"} is equivalent to \code{"-xF"}.  The GNU project
introduced \code{"-{}-"} followed by a series of hyphen-separated words,
e.g. \code{"-{}-file"} or \code{"-{}-dry-run"}.  These are the only two option
syntaxes provided by \module{optparse}.

Some other option syntaxes that the world has seen include:
\begin{itemize}
\item {} 
a hyphen followed by a few letters, e.g. \code{"-pf"} (this is
\emph{not} the same as multiple options merged into a single argument)

\item {} 
a hyphen followed by a whole word, e.g. \code{"-file"} (this is
technically equivalent to the previous syntax, but they aren't
usually seen in the same program)

\item {} 
a plus sign followed by a single letter, or a few letters,
or a word, e.g. \code{"+f"}, \code{"+rgb"}

\item {} 
a slash followed by a letter, or a few letters, or a word, e.g.
\code{"/f"}, \code{"/file"}

\end{itemize}

These option syntaxes are not supported by \module{optparse}, and they never will
be.  This is deliberate: the first three are non-standard on any
environment, and the last only makes sense if you're exclusively
targeting VMS, MS-DOS, and/or Windows.
\item[option argument]
an argument that follows an option, is closely associated with that
option, and is consumed from the argument list when that option is.
With \module{optparse}, option arguments may either be in a separate argument
from their option:
\begin{verbatim}
-f foo
--file foo
\end{verbatim}

or included in the same argument:
\begin{verbatim}
-ffoo
--file=foo
\end{verbatim}

Typically, a given option either takes an argument or it doesn't.
Lots of people want an ``optional option arguments'' feature, meaning
that some options will take an argument if they see it, and won't if
they don't.  This is somewhat controversial, because it makes parsing
ambiguous: if \code{"-a"} takes an optional argument and \code{"-b"} is
another option entirely, how do we interpret \code{"-ab"}?  Because of
this ambiguity, \module{optparse} does not support this feature.
\item[positional argument]
something leftover in the argument list after options have been
parsed, i.e. after options and their arguments have been parsed and
removed from the argument list.
\item[required option]
an option that must be supplied on the command-line; note that the
phrase ``required option'' is self-contradictory in English.  \module{optparse}
doesn't prevent you from implementing required options, but doesn't
give you much help at it either.  See \code{examples/required{\_}1.py} and
\code{examples/required{\_}2.py} in the \module{optparse} source distribution for two
ways to implement required options with \module{optparse}.
\end{description}

For example, consider this hypothetical command-line:
\begin{verbatim}
prog -v --report /tmp/report.txt foo bar
\end{verbatim}

\code{"-v"} and \code{"-{}-report"} are both options.  Assuming that
\longprogramopt{report} takes one argument, \code{"/tmp/report.txt"} is an option
argument.  \code{"foo"} and \code{"bar"} are positional arguments.


\subsubsection{What are options for?\label{optparse-what-options-for}}

Options are used to provide extra information to tune or customize the
execution of a program.  In case it wasn't clear, options are usually
\emph{optional}.  A program should be able to run just fine with no options
whatsoever.  (Pick a random program from the \UNIX{} or GNU toolsets.  Can
it run without any options at all and still make sense?  The main
exceptions are \code{find}, \code{tar}, and \code{dd}{---}all of which are mutant
oddballs that have been rightly criticized for their non-standard syntax
and confusing interfaces.)

Lots of people want their programs to have ``required options''.  Think
about it.  If it's required, then it's \emph{not optional}!  If there is a
piece of information that your program absolutely requires in order to
run successfully, that's what positional arguments are for.

As an example of good command-line interface design, consider the humble
\code{cp} utility, for copying files.  It doesn't make much sense to try to
copy files without supplying a destination and at least one source.
Hence, \code{cp} fails if you run it with no arguments.  However, it has a
flexible, useful syntax that does not require any options at all:
\begin{verbatim}
cp SOURCE DEST
cp SOURCE ... DEST-DIR
\end{verbatim}

You can get pretty far with just that.  Most \code{cp} implementations
provide a bunch of options to tweak exactly how the files are copied:
you can preserve mode and modification time, avoid following symlinks,
ask before clobbering existing files, etc.  But none of this distracts
from the core mission of \code{cp}, which is to copy either one file to
another, or several files to another directory.


\subsubsection{What are positional arguments for?\label{optparse-what-positional-arguments-for}}

Positional arguments are for those pieces of information that your
program absolutely, positively requires to run.

A good user interface should have as few absolute requirements as
possible.  If your program requires 17 distinct pieces of information in
order to run successfully, it doesn't much matter \emph{how} you get that
information from the user{---}most people will give up and walk away
before they successfully run the program.  This applies whether the user
interface is a command-line, a configuration file, or a GUI: if you make
that many demands on your users, most of them will simply give up.

In short, try to minimize the amount of information that users are
absolutely required to supply{---}use sensible defaults whenever
possible.  Of course, you also want to make your programs reasonably
flexible.  That's what options are for.  Again, it doesn't matter if
they are entries in a config file, widgets in the ``Preferences'' dialog
of a GUI, or command-line options{---}the more options you implement, the
more flexible your program is, and the more complicated its
implementation becomes.  Too much flexibility has drawbacks as well, of
course; too many options can overwhelm users and make your code much
harder to maintain.
% $Id: tao.txt 413 2004-09-28 00:59:13Z greg $ 


\subsection{Tutorial\label{optparse-tutorial}}

While \module{optparse} is quite flexible and powerful, it's also straightforward to
use in most cases.  This section covers the code patterns that are
common to any \module{optparse}-based program.

First, you need to import the OptionParser class; then, early in the
main program, create an OptionParser instance:
\begin{verbatim}
from optparse import OptionParser
[...]
parser = OptionParser()
\end{verbatim}

Then you can start defining options.  The basic syntax is:
\begin{verbatim}
parser.add_option(opt_str, ...,
                  attr=value, ...)
\end{verbatim}

Each option has one or more option strings, such as \code{"-f"} or
\code{"-{}-file"}, and several option attributes that tell \module{optparse} what to
expect and what to do when it encounters that option on the command
line.

Typically, each option will have one short option string and one long
option string, e.g.:
\begin{verbatim}
parser.add_option("-f", "--file", ...)
\end{verbatim}

You're free to define as many short option strings and as many long
option strings as you like (including zero), as long as there is at
least one option string overall.

The option strings passed to \method{add{\_}option()} are effectively labels for
the option defined by that call.  For brevity, we will frequently refer
to \emph{encountering an option} on the command line; in reality, \module{optparse}
encounters \emph{option strings} and looks up options from them.

Once all of your options are defined, instruct \module{optparse} to parse your
program's command line:
\begin{verbatim}
(options, args) = parser.parse_args()
\end{verbatim}

(If you like, you can pass a custom argument list to \method{parse{\_}args()},
but that's rarely necessary: by default it uses \code{sys.argv{[}1:]}.)

\method{parse{\_}args()} returns two values:
\begin{itemize}
\item {} 
\code{options}, an object containing values for all of your options{---}e.g. if \code{"-{}-file"} takes a single string argument, then
\code{options.file} will be the filename supplied by the user, or
\code{None} if the user did not supply that option

\item {} 
\code{args}, the list of positional arguments leftover after parsing
options

\end{itemize}

This tutorial section only covers the four most important option
attributes: \member{action}, \member{type}, \member{dest} (destination), and \member{help}.
Of these, \member{action} is the most fundamental.


\subsubsection{Understanding option actions\label{optparse-understanding-option-actions}}

Actions tell \module{optparse} what to do when it encounters an option on the
command line.  There is a fixed set of actions hard-coded into \module{optparse};
adding new actions is an advanced topic covered in section~\ref{optparse-extending-optparse}, Extending \module{optparse}.
Most actions tell \module{optparse} to store a value in some variable{---}for
example, take a string from the command line and store it in an
attribute of \code{options}.

If you don't specify an option action, \module{optparse} defaults to \code{store}.


\subsubsection{The store action\label{optparse-store-action}}

The most common option action is \code{store}, which tells \module{optparse} to take
the next argument (or the remainder of the current argument), ensure
that it is of the correct type, and store it to your chosen destination.

For example:
\begin{verbatim}
parser.add_option("-f", "--file",
                  action="store", type="string", dest="filename")
\end{verbatim}

Now let's make up a fake command line and ask \module{optparse} to parse it:
\begin{verbatim}
args = ["-f", "foo.txt"]
(options, args) = parser.parse_args(args)
\end{verbatim}

When \module{optparse} sees the option string \code{"-f"}, it consumes the next
argument, \code{"foo.txt"}, and stores it in \code{options.filename}.  So,
after this call to \method{parse{\_}args()}, \code{options.filename} is
\code{"foo.txt"}.

Some other option types supported by \module{optparse} are \code{int} and \code{float}.
Here's an option that expects an integer argument:
\begin{verbatim}
parser.add_option("-n", type="int", dest="num")
\end{verbatim}

Note that this option has no long option string, which is perfectly
acceptable.  Also, there's no explicit action, since the default is
\code{store}.

Let's parse another fake command-line.  This time, we'll jam the option
argument right up against the option: since \code{"-n42"} (one argument) is
equivalent to \code{"-n 42"} (two arguments), the code
\begin{verbatim}
(options, args) = parser.parse_args(["-n42"])
print options.num
\end{verbatim}

will print \code{"42"}.

If you don't specify a type, \module{optparse} assumes \code{string}.  Combined with the
fact that the default action is \code{store}, that means our first example
can be a lot shorter:
\begin{verbatim}
parser.add_option("-f", "--file", dest="filename")
\end{verbatim}

If you don't supply a destination, \module{optparse} figures out a sensible default
from the option strings: if the first long option string is
\code{"-{}-foo-bar"}, then the default destination is \code{foo{\_}bar}.  If there
are no long option strings, \module{optparse} looks at the first short option
string: the default destination for \code{"-f"} is \code{f}.

\module{optparse} also includes built-in \code{long} and \code{complex} types.  Adding
types is covered in section~\ref{optparse-extending-optparse}, Extending \module{optparse}.


\subsubsection{Handling boolean (flag) options\label{optparse-handling-boolean-options}}

Flag options{---}set a variable to true or false when a particular option
is seen{---}are quite common.  \module{optparse} supports them with two separate
actions, \code{store{\_}true} and \code{store{\_}false}.  For example, you might have a
\code{verbose} flag that is turned on with \code{"-v"} and off with \code{"-q"}:
\begin{verbatim}
parser.add_option("-v", action="store_true", dest="verbose")
parser.add_option("-q", action="store_false", dest="verbose")
\end{verbatim}

Here we have two different options with the same destination, which is
perfectly OK.  (It just means you have to be a bit careful when setting
default values{---}see below.)

When \module{optparse} encounters \code{"-v"} on the command line, it sets
\code{options.verbose} to \code{True}; when it encounters \code{"-q"},
\code{options.verbose} is set to \code{False}.


\subsubsection{Other actions\label{optparse-other-actions}}

Some other actions supported by \module{optparse} are:
\begin{description}
\item[\code{store{\_}const}]
store a constant value
\item[\code{append}]
append this option's argument to a list
\item[\code{count}]
increment a counter by one
\item[\code{callback}]
call a specified function
\end{description}

These are covered in section~\ref{optparse-reference-guide}, Reference Guide and section~\ref{optparse-option-callbacks}, Option Callbacks.


\subsubsection{Default values\label{optparse-default-values}}

All of the above examples involve setting some variable (the
``destination'') when certain command-line options are seen.  What happens
if those options are never seen?  Since we didn't supply any defaults,
they are all set to \code{None}.  This is usually fine, but sometimes you
want more control.  \module{optparse} lets you supply a default value for each
destination, which is assigned before the command line is parsed.

First, consider the verbose/quiet example.  If we want \module{optparse} to set
\code{verbose} to \code{True} unless \code{"-q"} is seen, then we can do this:
\begin{verbatim}
parser.add_option("-v", action="store_true", dest="verbose", default=True)
parser.add_option("-q", action="store_false", dest="verbose")
\end{verbatim}

Since default values apply to the \emph{destination} rather than to any
particular option, and these two options happen to have the same
destination, this is exactly equivalent:
\begin{verbatim}
parser.add_option("-v", action="store_true", dest="verbose")
parser.add_option("-q", action="store_false", dest="verbose", default=True)
\end{verbatim}

Consider this:
\begin{verbatim}
parser.add_option("-v", action="store_true", dest="verbose", default=False)
parser.add_option("-q", action="store_false", dest="verbose", default=True)
\end{verbatim}

Again, the default value for \code{verbose} will be \code{True}: the last
default value supplied for any particular destination is the one that
counts.

A clearer way to specify default values is the \method{set{\_}defaults()}
method of OptionParser, which you can call at any time before calling
\method{parse{\_}args()}:
\begin{verbatim}
parser.set_defaults(verbose=True)
parser.add_option(...)
(options, args) = parser.parse_args()
\end{verbatim}

As before, the last value specified for a given option destination is
the one that counts.  For clarity, try to use one method or the other of
setting default values, not both.


\subsubsection{Generating help\label{optparse-generating-help}}

\module{optparse}'s ability to generate help and usage text automatically is useful
for creating user-friendly command-line interfaces.  All you have to do
is supply a \member{help} value for each option, and optionally a short usage
message for your whole program.  Here's an OptionParser populated with
user-friendly (documented) options:
\begin{verbatim}
usage = "usage: %prog [options] arg1 arg2"
parser = OptionParser(usage=usage)
parser.add_option("-v", "--verbose",
                  action="store_true", dest="verbose", default=True,
                  help="make lots of noise [default]")
parser.add_option("-q", "--quiet",
                  action="store_false", dest="verbose", 
                  help="be vewwy quiet (I'm hunting wabbits)")
parser.add_option("-f", "--filename",
                  metavar="FILE", help="write output to FILE"),
parser.add_option("-m", "--mode",
                  default="intermediate",
                  help="interaction mode: novice, intermediate, "
                       "or expert [default: %default]")
\end{verbatim}

If \module{optparse} encounters either \code{"-h"} or \code{"-{}-help"} on the command-line,
or if you just call \method{parser.print{\_}help()}, it prints the following to
standard output:
\begin{verbatim}
usage: <yourscript> [options] arg1 arg2

options:
  -h, --help            show this help message and exit
  -v, --verbose         make lots of noise [default]
  -q, --quiet           be vewwy quiet (I'm hunting wabbits)
  -f FILE, --filename=FILE
                        write output to FILE
  -m MODE, --mode=MODE  interaction mode: novice, intermediate, or
                        expert [default: intermediate]
\end{verbatim}

(If the help output is triggered by a help option, \module{optparse} exits after
printing the help text.)

There's a lot going on here to help \module{optparse} generate the best possible
help message:
\begin{itemize}
\item {} 
the script defines its own usage message:
\begin{verbatim}
usage = "usage: %prog [options] arg1 arg2"
\end{verbatim}

\module{optparse} expands \code{"{\%}prog"} in the usage string to the name of the current
program, i.e. \code{os.path.basename(sys.argv{[}0])}.  The expanded string
is then printed before the detailed option help.

If you don't supply a usage string, \module{optparse} uses a bland but sensible
default: \code{"usage: {\%}prog {[}options]"}, which is fine if your script
doesn't take any positional arguments.

\item {} 
every option defines a help string, and doesn't worry about line-
wrapping{---}\module{optparse} takes care of wrapping lines and making the
help output look good.

\item {} 
options that take a value indicate this fact in their
automatically-generated help message, e.g. for the ``mode'' option:
\begin{verbatim}
-m MODE, --mode=MODE
\end{verbatim}

Here, ``MODE'' is called the meta-variable: it stands for the argument
that the user is expected to supply to \programopt{-m}/\longprogramopt{mode}.  By default,
\module{optparse} converts the destination variable name to uppercase and uses
that for the meta-variable.  Sometimes, that's not what you want{---}for example, the \longprogramopt{filename} option explicitly sets
\code{metavar="FILE"}, resulting in this automatically-generated option
description:
\begin{verbatim}
-f FILE, --filename=FILE
\end{verbatim}

This is important for more than just saving space, though: the
manually written help text uses the meta-variable ``FILE'' to clue the
user in that there's a connection between the semi-formal syntax ``-f
FILE'' and the informal semantic description ``write output to FILE''.
This is a simple but effective way to make your help text a lot
clearer and more useful for end users.

\item {} 
options that have a default value can include \code{{\%}default} in
the help string{---}\module{optparse} will replace it with \function{str()} of the
option's default value.  If an option has no default value (or the
default value is \code{None}), \code{{\%}default} expands to \code{none}.

\end{itemize}


\subsubsection{Printing a version string\label{optparse-printing-version-string}}

Similar to the brief usage string, \module{optparse} can also print a version string
for your program.  You have to supply the string as the \code{version}
argument to OptionParser:
\begin{verbatim}
parser = OptionParser(usage="%prog [-f] [-q]", version="%prog 1.0")
\end{verbatim}

\code{"{\%}prog"} is expanded just like it is in \code{usage}.  Apart
from that, \code{version} can contain anything you like.  When you supply
it, \module{optparse} automatically adds a \code{"-{}-version"} option to your parser.
If it encounters this option on the command line, it expands your
\code{version} string (by replacing \code{"{\%}prog"}), prints it to stdout, and
exits.

For example, if your script is called \code{/usr/bin/foo}:
\begin{verbatim}
$ /usr/bin/foo --version
foo 1.0
\end{verbatim}


\subsubsection{How \module{optparse} handles errors\label{optparse-how-optparse-handles-errors}}

There are two broad classes of errors that \module{optparse} has to worry about:
programmer errors and user errors.  Programmer errors are usually
erroneous calls to \code{parser.add{\_}option()}, e.g. invalid option strings,
unknown option attributes, missing option attributes, etc.  These are
dealt with in the usual way: raise an exception (either
\code{optparse.OptionError} or \code{TypeError}) and let the program crash.

Handling user errors is much more important, since they are guaranteed
to happen no matter how stable your code is.  \module{optparse} can automatically
detect some user errors, such as bad option arguments (passing \code{"-n
4x"} where \programopt{-n} takes an integer argument), missing arguments
(\code{"-n"} at the end of the command line, where \programopt{-n} takes an argument
of any type).  Also, you can call \code{parser.error()} to signal an
application-defined error condition:
\begin{verbatim}
(options, args) = parser.parse_args()
[...]
if options.a and options.b:
    parser.error("options -a and -b are mutually exclusive")
\end{verbatim}

In either case, \module{optparse} handles the error the same way: it prints the
program's usage message and an error message to standard error and
exits with error status 2.

Consider the first example above, where the user passes \code{"4x"} to an
option that takes an integer:
\begin{verbatim}
$ /usr/bin/foo -n 4x
usage: foo [options]

foo: error: option -n: invalid integer value: '4x'
\end{verbatim}

Or, where the user fails to pass a value at all:
\begin{verbatim}
$ /usr/bin/foo -n
usage: foo [options]

foo: error: -n option requires an argument
\end{verbatim}

\module{optparse}-generated error messages take care always to mention the option
involved in the error; be sure to do the same when calling
\code{parser.error()} from your application code.

If \module{optparse}'s default error-handling behaviour does not suite your needs,
you'll need to subclass OptionParser and override \code{exit()} and/or
\method{error()}.


\subsubsection{Putting it all together\label{optparse-putting-it-all-together}}

Here's what \module{optparse}-based scripts usually look like:
\begin{verbatim}
from optparse import OptionParser
[...]
def main():
    usage = "usage: %prog [options] arg"
    parser = OptionParser(usage)
    parser.add_option("-f", "--file", dest="filename",
                      help="read data from FILENAME")
    parser.add_option("-v", "--verbose",
                      action="store_true", dest="verbose")
    parser.add_option("-q", "--quiet",
                      action="store_false", dest="verbose")
    [...]
    (options, args) = parser.parse_args()
    if len(args) != 1:
        parser.error("incorrect number of arguments")
    if options.verbose:
        print "reading %s..." % options.filename
    [...]

if __name__ == "__main__":
    main()
\end{verbatim}
% $Id: tutorial.txt 515 2006-06-10 15:37:45Z gward $ 


\subsection{Reference Guide\label{optparse-reference-guide}}


\subsubsection{Creating the parser\label{optparse-creating-parser}}

The first step in using \module{optparse} is to create an OptionParser instance:
\begin{verbatim}
parser = OptionParser(...)
\end{verbatim}

The OptionParser constructor has no required arguments, but a number of
optional keyword arguments.  You should always pass them as keyword
arguments, i.e. do not rely on the order in which the arguments are
declared.
\begin{quote}
\begin{description}
\item[\code{usage} (default: \code{"{\%}prog {[}options]"})]
The usage summary to print when your program is run incorrectly or
with a help option.  When \module{optparse} prints the usage string, it expands
\code{{\%}prog} to \code{os.path.basename(sys.argv{[}0])} (or to \code{prog} if
you passed that keyword argument).  To suppress a usage message,
pass the special value \code{optparse.SUPPRESS{\_}USAGE}.
\item[\code{option{\_}list} (default: \code{{[}]})]
A list of Option objects to populate the parser with.  The options
in \code{option{\_}list} are added after any options in
\code{standard{\_}option{\_}list} (a class attribute that may be set by
OptionParser subclasses), but before any version or help options.
Deprecated; use \method{add{\_}option()} after creating the parser instead.
\item[\code{option{\_}class} (default: optparse.Option)]
Class to use when adding options to the parser in \method{add{\_}option()}.
\item[\code{version} (default: \code{None})]
A version string to print when the user supplies a version option.
If you supply a true value for \code{version}, \module{optparse} automatically adds
a version option with the single option string \code{"-{}-version"}.  The
substring \code{"{\%}prog"} is expanded the same as for \code{usage}.
\item[\code{conflict{\_}handler} (default: \code{"error"})]
Specifies what to do when options with conflicting option strings
are added to the parser; see section~\ref{optparse-conflicts-between-options}, Conflicts between options.
\item[\code{description} (default: \code{None})]
A paragraph of text giving a brief overview of your program.  \module{optparse}
reformats this paragraph to fit the current terminal width and
prints it when the user requests help (after \code{usage}, but before
the list of options).
\item[\code{formatter} (default: a new IndentedHelpFormatter)]
An instance of optparse.HelpFormatter that will be used for
printing help text.  \module{optparse} provides two concrete classes for this
purpose: IndentedHelpFormatter and TitledHelpFormatter.
\item[\code{add{\_}help{\_}option} (default: \code{True})]
If true, \module{optparse} will add a help option (with option strings \code{"-h"}
and \code{"-{}-help"}) to the parser.
\item[\code{prog}]
The string to use when expanding \code{"{\%}prog"} in \code{usage} and
\code{version} instead of \code{os.path.basename(sys.argv{[}0])}.
\end{description}
\end{quote}


\subsubsection{Populating the parser\label{optparse-populating-parser}}

There are several ways to populate the parser with options.  The
preferred way is by using \code{OptionParser.add{\_}option()}, as shown in
section~\ref{optparse-tutorial}, the tutorial.  \method{add{\_}option()} can be called in one of two
ways:
\begin{itemize}
\item {} 
pass it an Option instance (as returned by \function{make{\_}option()})

\item {} 
pass it any combination of positional and keyword arguments that are
acceptable to \function{make{\_}option()} (i.e., to the Option constructor),
and it will create the Option instance for you

\end{itemize}

The other alternative is to pass a list of pre-constructed Option
instances to the OptionParser constructor, as in:
\begin{verbatim}
option_list = [
    make_option("-f", "--filename",
                action="store", type="string", dest="filename"),
    make_option("-q", "--quiet",
                action="store_false", dest="verbose"),
    ]
parser = OptionParser(option_list=option_list)
\end{verbatim}

(\function{make{\_}option()} is a factory function for creating Option instances;
currently it is an alias for the Option constructor.  A future version
of \module{optparse} may split Option into several classes, and \function{make{\_}option()}
will pick the right class to instantiate.  Do not instantiate Option
directly.)


\subsubsection{Defining options\label{optparse-defining-options}}

Each Option instance represents a set of synonymous command-line option
strings, e.g. \programopt{-f} and \longprogramopt{file}.  You can
specify any number of short or long option strings, but you must specify
at least one overall option string.

The canonical way to create an Option instance is with the
\method{add{\_}option()} method of \class{OptionParser}:
\begin{verbatim}
parser.add_option(opt_str[, ...], attr=value, ...)
\end{verbatim}

To define an option with only a short option string:
\begin{verbatim}
parser.add_option("-f", attr=value, ...)
\end{verbatim}

And to define an option with only a long option string:
\begin{verbatim}
parser.add_option("--foo", attr=value, ...)
\end{verbatim}

The keyword arguments define attributes of the new Option object.  The
most important option attribute is \member{action}, and it largely determines
which other attributes are relevant or required.  If you pass irrelevant
option attributes, or fail to pass required ones, \module{optparse} raises an
OptionError exception explaining your mistake.

An options's \emph{action} determines what \module{optparse} does when it encounters this
option on the command-line.  The standard option actions hard-coded into
\module{optparse} are:
\begin{description}
\item[\code{store}]
store this option's argument (default)
\item[\code{store{\_}const}]
store a constant value
\item[\code{store{\_}true}]
store a true value
\item[\code{store{\_}false}]
store a false value
\item[\code{append}]
append this option's argument to a list
\item[\code{append{\_}const}]
append a constant value to a list
\item[\code{count}]
increment a counter by one
\item[\code{callback}]
call a specified function
\item[\member{help}]
print a usage message including all options and the
documentation for them
\end{description}

(If you don't supply an action, the default is \code{store}.  For this
action, you may also supply \member{type} and \member{dest} option attributes; see
below.)

As you can see, most actions involve storing or updating a value
somewhere.  \module{optparse} always creates a special object for this,
conventionally called \code{options} (it happens to be an instance of
\code{optparse.Values}).  Option arguments (and various other values) are
stored as attributes of this object, according to the \member{dest}
(destination) option attribute.

For example, when you call
\begin{verbatim}
parser.parse_args()
\end{verbatim}

one of the first things \module{optparse} does is create the \code{options} object:
\begin{verbatim}
options = Values()
\end{verbatim}

If one of the options in this parser is defined with
\begin{verbatim}
parser.add_option("-f", "--file", action="store", type="string", dest="filename")
\end{verbatim}

and the command-line being parsed includes any of the following:
\begin{verbatim}
-ffoo
-f foo
--file=foo
--file foo
\end{verbatim}

then \module{optparse}, on seeing this option, will do the equivalent of
\begin{verbatim}
options.filename = "foo"
\end{verbatim}

The \member{type} and \member{dest} option attributes are almost as important as
\member{action}, but \member{action} is the only one that makes sense for \emph{all}
options.


\subsubsection{Standard option actions\label{optparse-standard-option-actions}}

The various option actions all have slightly different requirements and
effects.  Most actions have several relevant option attributes which you
may specify to guide \module{optparse}'s behaviour; a few have required attributes,
which you must specify for any option using that action.
\begin{itemize}
\item {} 
\code{store} {[}relevant: \member{type}, \member{dest}, \code{nargs}, \code{choices}]

The option must be followed by an argument, which is
converted to a value according to \member{type} and stored in
\member{dest}.  If \code{nargs} {\textgreater} 1, multiple arguments will be consumed
from the command line; all will be converted according to
\member{type} and stored to \member{dest} as a tuple.  See the ``Option
types'' section below.

If \code{choices} is supplied (a list or tuple of strings), the type
defaults to \code{choice}.

If \member{type} is not supplied, it defaults to \code{string}.

If \member{dest} is not supplied, \module{optparse} derives a destination from the
first long option string (e.g., \code{"-{}-foo-bar"} implies \code{foo{\_}bar}).
If there are no long option strings, \module{optparse} derives a destination from
the first short option string (e.g., \code{"-f"} implies \code{f}).

Example:
\begin{verbatim}
parser.add_option("-f")
parser.add_option("-p", type="float", nargs=3, dest="point")
\end{verbatim}

As it parses the command line
\begin{verbatim}
-f foo.txt -p 1 -3.5 4 -fbar.txt
\end{verbatim}

\module{optparse} will set
\begin{verbatim}
options.f = "foo.txt"
options.point = (1.0, -3.5, 4.0)
options.f = "bar.txt"
\end{verbatim}

\item {} 
\code{store{\_}const} {[}required: \code{const}; relevant: \member{dest}]

The value \code{const} is stored in \member{dest}.

Example:
\begin{verbatim}
parser.add_option("-q", "--quiet",
                  action="store_const", const=0, dest="verbose")
parser.add_option("-v", "--verbose",
                  action="store_const", const=1, dest="verbose")
parser.add_option("--noisy",
                  action="store_const", const=2, dest="verbose")
\end{verbatim}

If \code{"-{}-noisy"} is seen, \module{optparse} will set
\begin{verbatim}
options.verbose = 2
\end{verbatim}

\item {} 
\code{store{\_}true} {[}relevant: \member{dest}]

A special case of \code{store{\_}const} that stores a true value
to \member{dest}.

\item {} 
\code{store{\_}false} {[}relevant: \member{dest}]

Like \code{store{\_}true}, but stores a false value.

Example:
\begin{verbatim}
parser.add_option("--clobber", action="store_true", dest="clobber")
parser.add_option("--no-clobber", action="store_false", dest="clobber")
\end{verbatim}

\item {} 
\code{append} {[}relevant: \member{type}, \member{dest}, \code{nargs}, \code{choices}]

The option must be followed by an argument, which is appended to the
list in \member{dest}.  If no default value for \member{dest} is supplied, an
empty list is automatically created when \module{optparse} first encounters this
option on the command-line.  If \code{nargs} {\textgreater} 1, multiple arguments are
consumed, and a tuple of length \code{nargs} is appended to \member{dest}.

The defaults for \member{type} and \member{dest} are the same as for the
\code{store} action.

Example:
\begin{verbatim}
parser.add_option("-t", "--tracks", action="append", type="int")
\end{verbatim}

If \code{"-t3"} is seen on the command-line, \module{optparse} does the equivalent of:
\begin{verbatim}
options.tracks = []
options.tracks.append(int("3"))
\end{verbatim}

If, a little later on, \code{"-{}-tracks=4"} is seen, it does:
\begin{verbatim}
options.tracks.append(int("4"))
\end{verbatim}

\item {} 
\code{append{\_}const} {[}required: \code{const}; relevant: \member{dest}]

Like \code{store{\_}const}, but the value \code{const} is appended to \member{dest};
as with \code{append}, \member{dest} defaults to \code{None}, and an an empty list is
automatically created the first time the option is encountered.

\item {} 
\code{count} {[}relevant: \member{dest}]

Increment the integer stored at \member{dest}.  If no default value is
supplied, \member{dest} is set to zero before being incremented the first
time.

Example:
\begin{verbatim}
parser.add_option("-v", action="count", dest="verbosity")
\end{verbatim}

The first time \code{"-v"} is seen on the command line, \module{optparse} does the
equivalent of:
\begin{verbatim}
options.verbosity = 0
options.verbosity += 1
\end{verbatim}

Every subsequent occurrence of \code{"-v"} results in
\begin{verbatim}
options.verbosity += 1
\end{verbatim}

\item {} 
\code{callback} {[}required: \code{callback};
relevant: \member{type}, \code{nargs}, \code{callback{\_}args}, \code{callback{\_}kwargs}]

Call the function specified by \code{callback}, which is called as
\begin{verbatim}
func(option, opt_str, value, parser, *args, **kwargs)
\end{verbatim}

See section~\ref{optparse-option-callbacks}, Option Callbacks for more detail.

\item {} 
\member{help}

Prints a complete help message for all the options in the
current option parser.  The help message is constructed from
the \code{usage} string passed to OptionParser's constructor and
the \member{help} string passed to every option.

If no \member{help} string is supplied for an option, it will still be
listed in the help message.  To omit an option entirely, use
the special value \code{optparse.SUPPRESS{\_}HELP}.

\module{optparse} automatically adds a \member{help} option to all OptionParsers, so
you do not normally need to create one.

Example:
\begin{verbatim}
from optparse import OptionParser, SUPPRESS_HELP

parser = OptionParser()
parser.add_option("-h", "--help", action="help"),
parser.add_option("-v", action="store_true", dest="verbose",
                  help="Be moderately verbose")
parser.add_option("--file", dest="filename",
                  help="Input file to read data from"),
parser.add_option("--secret", help=SUPPRESS_HELP)
\end{verbatim}

If \module{optparse} sees either \code{"-h"} or \code{"-{}-help"} on the command line, it
will print something like the following help message to stdout
(assuming \code{sys.argv{[}0]} is \code{"foo.py"}):
\begin{verbatim}
usage: foo.py [options]

options:
  -h, --help        Show this help message and exit
  -v                Be moderately verbose
  --file=FILENAME   Input file to read data from
\end{verbatim}

After printing the help message, \module{optparse} terminates your process
with \code{sys.exit(0)}.

\item {} 
\code{version}

Prints the version number supplied to the OptionParser to stdout and
exits.  The version number is actually formatted and printed by the
\code{print{\_}version()} method of OptionParser.  Generally only relevant
if the \code{version} argument is supplied to the OptionParser
constructor.  As with \member{help} options, you will rarely create
\code{version} options, since \module{optparse} automatically adds them when needed.

\end{itemize}


\subsubsection{Option attributes\label{optparse-option-attributes}}

The following option attributes may be passed as keyword arguments
to \code{parser.add{\_}option()}.  If you pass an option attribute
that is not relevant to a particular option, or fail to pass a required
option attribute, \module{optparse} raises OptionError.
\begin{itemize}
\item {} 
\member{action} (default: \code{"store"})

Determines \module{optparse}'s behaviour when this option is seen on the command
line; the available options are documented above.

\item {} 
\member{type} (default: \code{"string"})

The argument type expected by this option (e.g., \code{"string"} or
\code{"int"}); the available option types are documented below.

\item {} 
\member{dest} (default: derived from option strings)

If the option's action implies writing or modifying a value somewhere,
this tells \module{optparse} where to write it: \member{dest} names an attribute of the
\code{options} object that \module{optparse} builds as it parses the command line.

\item {} 
\code{default} (deprecated)

The value to use for this option's destination if the option is not
seen on the command line.  Deprecated; use \code{parser.set{\_}defaults()}
instead.

\item {} 
\code{nargs} (default: 1)

How many arguments of type \member{type} should be consumed when this
option is seen.  If {\textgreater} 1, \module{optparse} will store a tuple of values to
\member{dest}.

\item {} 
\code{const}

For actions that store a constant value, the constant value to store.

\item {} 
\code{choices}

For options of type \code{"choice"}, the list of strings the user
may choose from.

\item {} 
\code{callback}

For options with action \code{"callback"}, the callable to call when this
option is seen.  See section~\ref{optparse-option-callbacks}, Option Callbacks for detail on the arguments
passed to \code{callable}.

\item {} 
\code{callback{\_}args}, \code{callback{\_}kwargs}

Additional positional and keyword arguments to pass to \code{callback}
after the four standard callback arguments.

\item {} 
\member{help}

Help text to print for this option when listing all available options
after the user supplies a \member{help} option (such as \code{"-{}-help"}).
If no help text is supplied, the option will be listed without help
text.  To hide this option, use the special value \code{SUPPRESS{\_}HELP}.

\item {} 
\code{metavar} (default: derived from option strings)

Stand-in for the option argument(s) to use when printing help text.
See section~\ref{optparse-tutorial}, the tutorial for an example.

\end{itemize}


\subsubsection{Standard option types\label{optparse-standard-option-types}}

\module{optparse} has six built-in option types: \code{string}, \code{int}, \code{long},
\code{choice}, \code{float} and \code{complex}.  If you need to add new option
types, see section~\ref{optparse-extending-optparse}, Extending \module{optparse}.

Arguments to string options are not checked or converted in any way: the
text on the command line is stored in the destination (or passed to the
callback) as-is.

Integer arguments (type \code{int} or \code{long}) are parsed as follows:
\begin{quote}
\begin{itemize}
\item {} 
if the number starts with \code{0x}, it is parsed as a hexadecimal number

\item {} 
if the number starts with \code{0}, it is parsed as an octal number

\item {} 
if the number starts with \code{0b}, is is parsed as a binary number

\item {} 
otherwise, the number is parsed as a decimal number

\end{itemize}
\end{quote}

The conversion is done by calling either \code{int()} or \code{long()} with
the appropriate base (2, 8, 10, or 16).  If this fails, so will \module{optparse},
although with a more useful error message.

\code{float} and \code{complex} option arguments are converted directly with
\code{float()} and \code{complex()}, with similar error-handling.

\code{choice} options are a subtype of \code{string} options.  The \code{choices}
option attribute (a sequence of strings) defines the set of allowed
option arguments.  \code{optparse.check{\_}choice()} compares
user-supplied option arguments against this master list and raises
OptionValueError if an invalid string is given.


\subsubsection{Parsing arguments\label{optparse-parsing-arguments}}

The whole point of creating and populating an OptionParser is to call
its \method{parse{\_}args()} method:
\begin{verbatim}
(options, args) = parser.parse_args(args=None, options=None)
\end{verbatim}

where the input parameters are
\begin{description}
\item[\code{args}]
the list of arguments to process (default: \code{sys.argv{[}1:]})
\item[\code{options}]
object to store option arguments in (default: a new instance of
optparse.Values)
\end{description}

and the return values are
\begin{description}
\item[\code{options}]
the same object that was passed in as \code{options}, or the
optparse.Values instance created by \module{optparse}
\item[\code{args}]
the leftover positional arguments after all options have been
processed
\end{description}

The most common usage is to supply neither keyword argument.  If you
supply \code{options}, it will be modified with repeated \code{setattr()}
calls (roughly one for every option argument stored to an option
destination) and returned by \method{parse{\_}args()}.

If \method{parse{\_}args()} encounters any errors in the argument list, it calls
the OptionParser's \method{error()} method with an appropriate end-user error
message.  This ultimately terminates your process with an exit status of
2 (the traditional \UNIX{} exit status for command-line errors).


\subsubsection{Querying and manipulating your option parser\label{optparse-querying-manipulating-option-parser}}

Sometimes, it's useful to poke around your option parser and see what's
there.  OptionParser provides a couple of methods to help you out:
\begin{description}
\item[\code{has{\_}option(opt{\_}str)}]
Return true if the OptionParser has an option with 
option string \code{opt{\_}str} (e.g., \code{"-q"} or \code{"-{}-verbose"}).
\item[\code{get{\_}option(opt{\_}str)}]
Returns the Option instance with the option string \code{opt{\_}str}, or
\code{None} if no options have that option string.
\item[\code{remove{\_}option(opt{\_}str)}]
If the OptionParser has an option corresponding to \code{opt{\_}str},
that option is removed.  If that option provided any other
option strings, all of those option strings become invalid.
If \code{opt{\_}str} does not occur in any option belonging to this
OptionParser, raises ValueError.
\end{description}


\subsubsection{Conflicts between options\label{optparse-conflicts-between-options}}

If you're not careful, it's easy to define options with conflicting
option strings:
\begin{verbatim}
parser.add_option("-n", "--dry-run", ...)
[...]
parser.add_option("-n", "--noisy", ...)
\end{verbatim}

(This is particularly true if you've defined your own OptionParser
subclass with some standard options.)

Every time you add an option, \module{optparse} checks for conflicts with existing
options.  If it finds any, it invokes the current conflict-handling
mechanism.  You can set the conflict-handling mechanism either in the
constructor:
\begin{verbatim}
parser = OptionParser(..., conflict_handler=handler)
\end{verbatim}

or with a separate call:
\begin{verbatim}
parser.set_conflict_handler(handler)
\end{verbatim}

The available conflict handlers are:
\begin{quote}
\begin{description}
\item[\code{error} (default)]
assume option conflicts are a programming error and raise 
OptionConflictError
\item[\code{resolve}]
resolve option conflicts intelligently (see below)
\end{description}
\end{quote}

As an example, let's define an OptionParser that resolves conflicts
intelligently and add conflicting options to it:
\begin{verbatim}
parser = OptionParser(conflict_handler="resolve")
parser.add_option("-n", "--dry-run", ..., help="do no harm")
parser.add_option("-n", "--noisy", ..., help="be noisy")
\end{verbatim}

At this point, \module{optparse} detects that a previously-added option is already
using the \code{"-n"} option string.  Since \code{conflict{\_}handler} is
\code{"resolve"}, it resolves the situation by removing \code{"-n"} from the
earlier option's list of option strings.  Now \code{"-{}-dry-run"} is the
only way for the user to activate that option.  If the user asks for
help, the help message will reflect that:
\begin{verbatim}
options:
  --dry-run     do no harm
  [...]
  -n, --noisy   be noisy
\end{verbatim}

It's possible to whittle away the option strings for a previously-added
option until there are none left, and the user has no way of invoking
that option from the command-line.  In that case, \module{optparse} removes that
option completely, so it doesn't show up in help text or anywhere else.
Carrying on with our existing OptionParser:
\begin{verbatim}
parser.add_option("--dry-run", ..., help="new dry-run option")
\end{verbatim}

At this point, the original \programopt{-n/-{}-dry-run} option is no longer
accessible, so \module{optparse} removes it, leaving this help text:
\begin{verbatim}
options:
  [...]
  -n, --noisy   be noisy
  --dry-run     new dry-run option
\end{verbatim}


\subsubsection{Cleanup\label{optparse-cleanup}}

OptionParser instances have several cyclic references.  This should not
be a problem for Python's garbage collector, but you may wish to break
the cyclic references explicitly by calling \code{destroy()} on your
OptionParser once you are done with it.  This is particularly useful in
long-running applications where large object graphs are reachable from
your OptionParser.


\subsubsection{Other methods\label{optparse-other-methods}}

OptionParser supports several other public methods:
\begin{itemize}
\item {} 
\code{set{\_}usage(usage)}

Set the usage string according to the rules described above for the
\code{usage} constructor keyword argument.  Passing \code{None} sets the
default usage string; use \code{SUPPRESS{\_}USAGE} to suppress a usage
message.

\item {} 
\code{enable{\_}interspersed{\_}args()}, \code{disable{\_}interspersed{\_}args()}

Enable/disable positional arguments interspersed with options, similar
to GNU getopt (enabled by default).  For example, if \code{"-a"} and
\code{"-b"} are both simple options that take no arguments, \module{optparse}
normally accepts this syntax:
\begin{verbatim}
prog -a arg1 -b arg2
\end{verbatim}

and treats it as equivalent to
\begin{verbatim}
prog -a -b arg1 arg2
\end{verbatim}

To disable this feature, call \code{disable{\_}interspersed{\_}args()}.  This
restores traditional \UNIX{} syntax, where option parsing stops with the
first non-option argument.

\item {} 
\code{set{\_}defaults(dest=value, ...)}

Set default values for several option destinations at once.  Using
\method{set{\_}defaults()} is the preferred way to set default values for
options, since multiple options can share the same destination.  For
example, if several ``mode'' options all set the same destination, any
one of them can set the default, and the last one wins:
\begin{verbatim}
parser.add_option("--advanced", action="store_const",
                  dest="mode", const="advanced",
                  default="novice")    # overridden below
parser.add_option("--novice", action="store_const",
                  dest="mode", const="novice",
                  default="advanced")  # overrides above setting
\end{verbatim}

To avoid this confusion, use \method{set{\_}defaults()}:
\begin{verbatim}
parser.set_defaults(mode="advanced")
parser.add_option("--advanced", action="store_const",
                  dest="mode", const="advanced")
parser.add_option("--novice", action="store_const",
                  dest="mode", const="novice")
\end{verbatim}

\end{itemize}
% $Id: reference.txt 519 2006-06-11 14:39:11Z gward $ 


\subsection{Option Callbacks\label{optparse-option-callbacks}}

When \module{optparse}'s built-in actions and types aren't quite enough for your
needs, you have two choices: extend \module{optparse} or define a callback option.
Extending \module{optparse} is more general, but overkill for a lot of simple
cases.  Quite often a simple callback is all you need.

There are two steps to defining a callback option:
\begin{itemize}
\item {} 
define the option itself using the \code{callback} action

\item {} 
write the callback; this is a function (or method) that
takes at least four arguments, as described below

\end{itemize}


\subsubsection{Defining a callback option\label{optparse-defining-callback-option}}

As always, the easiest way to define a callback option is by using the
\code{parser.add{\_}option()} method.  Apart from \member{action}, the only option
attribute you must specify is \code{callback}, the function to call:
\begin{verbatim}
parser.add_option("-c", action="callback", callback=my_callback)
\end{verbatim}

\code{callback} is a function (or other callable object), so you must have
already defined \code{my{\_}callback()} when you create this callback option.
In this simple case, \module{optparse} doesn't even know if \programopt{-c} takes any
arguments, which usually means that the option takes no arguments{---}the
mere presence of \programopt{-c} on the command-line is all it needs to know.  In
some circumstances, though, you might want your callback to consume an
arbitrary number of command-line arguments.  This is where writing
callbacks gets tricky; it's covered later in this section.

\module{optparse} always passes four particular arguments to your callback, and it
will only pass additional arguments if you specify them via
\code{callback{\_}args} and \code{callback{\_}kwargs}.  Thus, the minimal callback
function signature is:
\begin{verbatim}
def my_callback(option, opt, value, parser):
\end{verbatim}

The four arguments to a callback are described below.

There are several other option attributes that you can supply when you
define a callback option:
\begin{description}
\item[\member{type}]
has its usual meaning: as with the \code{store} or \code{append} actions,
it instructs \module{optparse} to consume one argument and convert it to
\member{type}.  Rather than storing the converted value(s) anywhere,
though, \module{optparse} passes it to your callback function.
\item[\code{nargs}]
also has its usual meaning: if it is supplied and {\textgreater} 1, \module{optparse} will
consume \code{nargs} arguments, each of which must be convertible to
\member{type}.  It then passes a tuple of converted values to your
callback.
\item[\code{callback{\_}args}]
a tuple of extra positional arguments to pass to the callback
\item[\code{callback{\_}kwargs}]
a dictionary of extra keyword arguments to pass to the callback
\end{description}


\subsubsection{How callbacks are called\label{optparse-how-callbacks-called}}

All callbacks are called as follows:
\begin{verbatim}
func(option, opt_str, value, parser, *args, **kwargs)
\end{verbatim}

where
\begin{description}
\item[\code{option}]
is the Option instance that's calling the callback
\item[\code{opt{\_}str}]
is the option string seen on the command-line that's triggering the
callback.  (If an abbreviated long option was used, \code{opt{\_}str} will
be the full, canonical option string{---}e.g. if the user puts
\code{"-{}-foo"} on the command-line as an abbreviation for
\code{"-{}-foobar"}, then \code{opt{\_}str} will be \code{"-{}-foobar"}.)
\item[\code{value}]
is the argument to this option seen on the command-line.  \module{optparse} will
only expect an argument if \member{type} is set; the type of \code{value}
will be the type implied by the option's type.  If \member{type} for this
option is \code{None} (no argument expected), then \code{value} will be
\code{None}.  If \code{nargs} {\textgreater} 1, \code{value} will be a tuple of values of
the appropriate type.
\item[\code{parser}]
is the OptionParser instance driving the whole thing, mainly
useful because you can access some other interesting data through
its instance attributes:
\begin{description}
\item[\code{parser.largs}]
the current list of leftover arguments, ie. arguments that have
been consumed but are neither options nor option arguments.
Feel free to modify \code{parser.largs}, e.g. by adding more
arguments to it.  (This list will become \code{args}, the second
return value of \method{parse{\_}args()}.)
\item[\code{parser.rargs}]
the current list of remaining arguments, ie. with \code{opt{\_}str} and
\code{value} (if applicable) removed, and only the arguments
following them still there.  Feel free to modify
\code{parser.rargs}, e.g. by consuming more arguments.
\item[\code{parser.values}]
the object where option values are by default stored (an
instance of optparse.OptionValues).  This lets callbacks use the
same mechanism as the rest of \module{optparse} for storing option values;
you don't need to mess around with globals or closures.  You can
also access or modify the value(s) of any options already
encountered on the command-line.
\end{description}
\item[\code{args}]
is a tuple of arbitrary positional arguments supplied via the
\code{callback{\_}args} option attribute.
\item[\code{kwargs}]
is a dictionary of arbitrary keyword arguments supplied via
\code{callback{\_}kwargs}.
\end{description}


\subsubsection{Raising errors in a callback\label{optparse-raising-errors-in-callback}}

The callback function should raise OptionValueError if there are any
problems with the option or its argument(s).  \module{optparse} catches this and
terminates the program, printing the error message you supply to
stderr.  Your message should be clear, concise, accurate, and mention
the option at fault.  Otherwise, the user will have a hard time
figuring out what he did wrong.


\subsubsection{Callback example 1: trivial callback\label{optparse-callback-example-1}}

Here's an example of a callback option that takes no arguments, and
simply records that the option was seen:
\begin{verbatim}
def record_foo_seen(option, opt_str, value, parser):
    parser.saw_foo = True

parser.add_option("--foo", action="callback", callback=record_foo_seen)
\end{verbatim}

Of course, you could do that with the \code{store{\_}true} action.


\subsubsection{Callback example 2: check option order\label{optparse-callback-example-2}}

Here's a slightly more interesting example: record the fact that
\code{"-a"} is seen, but blow up if it comes after \code{"-b"} in the
command-line.
\begin{verbatim}
def check_order(option, opt_str, value, parser):
    if parser.values.b:
        raise OptionValueError("can't use -a after -b")
    parser.values.a = 1
[...]
parser.add_option("-a", action="callback", callback=check_order)
parser.add_option("-b", action="store_true", dest="b")
\end{verbatim}


\subsubsection{Callback example 3: check option order (generalized)\label{optparse-callback-example-3}}

If you want to re-use this callback for several similar options (set a
flag, but blow up if \code{"-b"} has already been seen), it needs a bit of
work: the error message and the flag that it sets must be
generalized.
\begin{verbatim}
def check_order(option, opt_str, value, parser):
    if parser.values.b:
        raise OptionValueError("can't use %s after -b" % opt_str)
    setattr(parser.values, option.dest, 1)
[...]
parser.add_option("-a", action="callback", callback=check_order, dest='a')
parser.add_option("-b", action="store_true", dest="b")
parser.add_option("-c", action="callback", callback=check_order, dest='c')
\end{verbatim}


\subsubsection{Callback example 4: check arbitrary condition\label{optparse-callback-example-4}}

Of course, you could put any condition in there{---}you're not limited
to checking the values of already-defined options.  For example, if
you have options that should not be called when the moon is full, all
you have to do is this:
\begin{verbatim}
def check_moon(option, opt_str, value, parser):
    if is_moon_full():
        raise OptionValueError("%s option invalid when moon is full"
                               % opt_str)
    setattr(parser.values, option.dest, 1)
[...]
parser.add_option("--foo",
                  action="callback", callback=check_moon, dest="foo")
\end{verbatim}

(The definition of \code{is{\_}moon{\_}full()} is left as an exercise for the
reader.)


\subsubsection{Callback example 5: fixed arguments\label{optparse-callback-example-5}}

Things get slightly more interesting when you define callback options
that take a fixed number of arguments.  Specifying that a callback
option takes arguments is similar to defining a \code{store} or \code{append}
option: if you define \member{type}, then the option takes one argument that
must be convertible to that type; if you further define \code{nargs}, then
the option takes \code{nargs} arguments.

Here's an example that just emulates the standard \code{store} action:
\begin{verbatim}
def store_value(option, opt_str, value, parser):
    setattr(parser.values, option.dest, value)
[...]
parser.add_option("--foo",
                  action="callback", callback=store_value,
                  type="int", nargs=3, dest="foo")
\end{verbatim}

Note that \module{optparse} takes care of consuming 3 arguments and converting them
to integers for you; all you have to do is store them.  (Or whatever;
obviously you don't need a callback for this example.)


\subsubsection{Callback example 6: variable arguments\label{optparse-callback-example-6}}

Things get hairy when you want an option to take a variable number of
arguments.  For this case, you must write a callback, as \module{optparse} doesn't
provide any built-in capabilities for it.  And you have to deal with
certain intricacies of conventional \UNIX{} command-line parsing that \module{optparse}
normally handles for you.  In particular, callbacks should implement
the conventional rules for bare \code{"-{}-"} and \code{"-"} arguments:
\begin{itemize}
\item {} 
either \code{"-{}-"} or \code{"-"} can be option arguments

\item {} 
bare \code{"-{}-"} (if not the argument to some option): halt command-line
processing and discard the \code{"-{}-"}

\item {} 
bare \code{"-"} (if not the argument to some option): halt command-line
processing but keep the \code{"-"} (append it to \code{parser.largs})

\end{itemize}

If you want an option that takes a variable number of arguments, there
are several subtle, tricky issues to worry about.  The exact
implementation you choose will be based on which trade-offs you're
willing to make for your application (which is why \module{optparse} doesn't support
this sort of thing directly).

Nevertheless, here's a stab at a callback for an option with variable
arguments:
\begin{verbatim}
def vararg_callback(option, opt_str, value, parser):
    assert value is None
    done = 0
    value = []
    rargs = parser.rargs
    while rargs:
        arg = rargs[0]

        # Stop if we hit an arg like "--foo", "-a", "-fx", "--file=f",
        # etc.  Note that this also stops on "-3" or "-3.0", so if
        # your option takes numeric values, you will need to handle
        # this.
        if ((arg[:2] == "--" and len(arg) > 2) or
            (arg[:1] == "-" and len(arg) > 1 and arg[1] != "-")):
            break
        else:
            value.append(arg)
            del rargs[0]

     setattr(parser.values, option.dest, value)

[...]
parser.add_option("-c", "--callback",
                  action="callback", callback=varargs)
\end{verbatim}

The main weakness with this particular implementation is that negative
numbers in the arguments following \code{"-c"} will be interpreted as
further options (probably causing an error), rather than as arguments to
\code{"-c"}.  Fixing this is left as an exercise for the reader.
% $Id: callbacks.txt 415 2004-09-30 02:26:17Z greg $ 


\subsection{Extending \module{optparse}\label{optparse-extending-optparse}}

Since the two major controlling factors in how \module{optparse} interprets
command-line options are the action and type of each option, the most
likely direction of extension is to add new actions and new types.


\subsubsection{Adding new types\label{optparse-adding-new-types}}

To add new types, you need to define your own subclass of \module{optparse}'s Option
class.  This class has a couple of attributes that define \module{optparse}'s types:
\member{TYPES} and \member{TYPE{\_}CHECKER}.

\member{TYPES} is a tuple of type names; in your subclass, simply define a new
tuple \member{TYPES} that builds on the standard one.

\member{TYPE{\_}CHECKER} is a dictionary mapping type names to type-checking
functions.  A type-checking function has the following signature:
\begin{verbatim}
def check_mytype(option, opt, value)
\end{verbatim}

where \code{option} is an \class{Option} instance, \code{opt} is an option string
(e.g., \code{"-f"}), and \code{value} is the string from the command line that
must be checked and converted to your desired type.  \code{check{\_}mytype()}
should return an object of the hypothetical type \code{mytype}.  The value
returned by a type-checking function will wind up in the OptionValues
instance returned by \method{OptionParser.parse{\_}args()}, or be passed to a
callback as the \code{value} parameter.

Your type-checking function should raise OptionValueError if it
encounters any problems.  OptionValueError takes a single string
argument, which is passed as-is to OptionParser's \method{error()} method,
which in turn prepends the program name and the string \code{"error:"} and
prints everything to stderr before terminating the process.

Here's a silly example that demonstrates adding a \code{complex} option
type to parse Python-style complex numbers on the command line.  (This
is even sillier than it used to be, because \module{optparse} 1.3 added built-in
support for complex numbers, but never mind.)

First, the necessary imports:
\begin{verbatim}
from copy import copy
from optparse import Option, OptionValueError
\end{verbatim}

You need to define your type-checker first, since it's referred to later
(in the \member{TYPE{\_}CHECKER} class attribute of your Option subclass):
\begin{verbatim}
def check_complex(option, opt, value):
    try:
        return complex(value)
    except ValueError:
        raise OptionValueError(
            "option %s: invalid complex value: %r" % (opt, value))
\end{verbatim}

Finally, the Option subclass:
\begin{verbatim}
class MyOption (Option):
    TYPES = Option.TYPES + ("complex",)
    TYPE_CHECKER = copy(Option.TYPE_CHECKER)
    TYPE_CHECKER["complex"] = check_complex
\end{verbatim}

(If we didn't make a \function{copy()} of \member{Option.TYPE{\_}CHECKER}, we would end
up modifying the \member{TYPE{\_}CHECKER} attribute of \module{optparse}'s Option class.
This being Python, nothing stops you from doing that except good manners
and common sense.)

That's it!  Now you can write a script that uses the new option type
just like any other \module{optparse}-based script, except you have to instruct your
OptionParser to use MyOption instead of Option:
\begin{verbatim}
parser = OptionParser(option_class=MyOption)
parser.add_option("-c", type="complex")
\end{verbatim}

Alternately, you can build your own option list and pass it to
OptionParser; if you don't use \method{add{\_}option()} in the above way, you
don't need to tell OptionParser which option class to use:
\begin{verbatim}
option_list = [MyOption("-c", action="store", type="complex", dest="c")]
parser = OptionParser(option_list=option_list)
\end{verbatim}


\subsubsection{Adding new actions\label{optparse-adding-new-actions}}

Adding new actions is a bit trickier, because you have to understand
that \module{optparse} has a couple of classifications for actions:
\begin{description}
\item[``store'' actions]
actions that result in \module{optparse} storing a value to an attribute of the
current OptionValues instance; these options require a \member{dest}
attribute to be supplied to the Option constructor
\item[``typed'' actions]
actions that take a value from the command line and expect it to be
of a certain type; or rather, a string that can be converted to a
certain type.  These options require a \member{type} attribute to the
Option constructor.
\end{description}

These are overlapping sets: some default ``store'' actions are \code{store},
\code{store{\_}const}, \code{append}, and \code{count}, while the default ``typed''
actions are \code{store}, \code{append}, and \code{callback}.

When you add an action, you need to categorize it by listing it in at
least one of the following class attributes of Option (all are lists of
strings):
\begin{description}
\item[\member{ACTIONS}]
all actions must be listed in ACTIONS
\item[\member{STORE{\_}ACTIONS}]
``store'' actions are additionally listed here
\item[\member{TYPED{\_}ACTIONS}]
``typed'' actions are additionally listed here
\item[\code{ALWAYS{\_}TYPED{\_}ACTIONS}]
actions that always take a type (i.e. whose options always take a
value) are additionally listed here.  The only effect of this is
that \module{optparse} assigns the default type, \code{string}, to options with no
explicit type whose action is listed in \code{ALWAYS{\_}TYPED{\_}ACTIONS}.
\end{description}

In order to actually implement your new action, you must override
Option's \method{take{\_}action()} method and add a case that recognizes your
action.

For example, let's add an \code{extend} action.  This is similar to the
standard \code{append} action, but instead of taking a single value from
the command-line and appending it to an existing list, \code{extend} will
take multiple values in a single comma-delimited string, and extend an
existing list with them.  That is, if \code{"-{}-names"} is an \code{extend}
option of type \code{string}, the command line
\begin{verbatim}
--names=foo,bar --names blah --names ding,dong
\end{verbatim}

would result in a list
\begin{verbatim}
["foo", "bar", "blah", "ding", "dong"]
\end{verbatim}

Again we define a subclass of Option:
\begin{verbatim}
class MyOption (Option):

    ACTIONS = Option.ACTIONS + ("extend",)
    STORE_ACTIONS = Option.STORE_ACTIONS + ("extend",)
    TYPED_ACTIONS = Option.TYPED_ACTIONS + ("extend",)
    ALWAYS_TYPED_ACTIONS = Option.ALWAYS_TYPED_ACTIONS + ("extend",)

    def take_action(self, action, dest, opt, value, values, parser):
        if action == "extend":
            lvalue = value.split(",")
            values.ensure_value(dest, []).extend(lvalue)
        else:
            Option.take_action(
                self, action, dest, opt, value, values, parser)
\end{verbatim}

Features of note:
\begin{itemize}
\item {} 
\code{extend} both expects a value on the command-line and stores that
value somewhere, so it goes in both \member{STORE{\_}ACTIONS} and
\member{TYPED{\_}ACTIONS}

\item {} 
to ensure that \module{optparse} assigns the default type of \code{string} to
\code{extend} actions, we put the \code{extend} action in
\code{ALWAYS{\_}TYPED{\_}ACTIONS} as well

\item {} 
\method{MyOption.take{\_}action()} implements just this one new action, and
passes control back to \method{Option.take{\_}action()} for the standard
\module{optparse} actions

\item {} 
\code{values} is an instance of the optparse{\_}parser.Values class,
which provides the very useful \method{ensure{\_}value()} method.
\method{ensure{\_}value()} is essentially \function{getattr()} with a safety valve;
it is called as
\begin{verbatim}
values.ensure_value(attr, value)
\end{verbatim}

If the \code{attr} attribute of \code{values} doesn't exist or is None, then
ensure{\_}value() first sets it to \code{value}, and then returns 'value.
This is very handy for actions like \code{extend}, \code{append}, and
\code{count}, all of which accumulate data in a variable and expect that
variable to be of a certain type (a list for the first two, an integer
for the latter).  Using \method{ensure{\_}value()} means that scripts using
your action don't have to worry about setting a default value for the
option destinations in question; they can just leave the default as
None and \method{ensure{\_}value()} will take care of getting it right when
it's needed.

\end{itemize}
% $Id: extending.txt 517 2006-06-10 16:18:11Z gward $ 


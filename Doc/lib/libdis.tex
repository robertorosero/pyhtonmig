\section{\module{dis} ---
         Disassembler for Python byte code}

\declaremodule{standard}{dis}
\modulesynopsis{Disassembler for Python byte code.}


The \module{dis} module supports the analysis of Python byte code by
disassembling it.  Since there is no Python assembler, this module
defines the Python assembly language.  The Python byte code which
this module takes as an input is defined in the file 
\file{Include/opcode.h} and used by the compiler and the interpreter.

Example: Given the function \function{myfunc}:

\begin{verbatim}
def myfunc(alist):
    return len(alist)
\end{verbatim}

the following command can be used to get the disassembly of
\function{myfunc()}:

\begin{verbatim}
>>> dis.dis(myfunc)
  2           0 LOAD_GLOBAL              0 (len)
              3 LOAD_FAST                0 (alist)
              6 CALL_FUNCTION            1
              9 RETURN_VALUE        
             10 RETURN_NONE
\end{verbatim}

(The ``2'' is a line number).

The \module{dis} module defines the following functions and constants:

\begin{funcdesc}{dis}{\optional{bytesource}}
Disassemble the \var{bytesource} object. \var{bytesource} can denote
either a module, a class, a method, a function, or a code object.  
For a module, it disassembles all functions.  For a class,
it disassembles all methods.  For a single code sequence, it prints
one line per byte code instruction.  If no object is provided, it
disassembles the last traceback.
\end{funcdesc}

\begin{funcdesc}{distb}{\optional{tb}}
Disassembles the top-of-stack function of a traceback, using the last
traceback if none was passed.  The instruction causing the exception
is indicated.
\end{funcdesc}

\begin{funcdesc}{disassemble}{code\optional{, lasti}}
Disassembles a code object, indicating the last instruction if \var{lasti}
was provided.  The output is divided in the following columns:

\begin{enumerate}
\item the line number, for the first instruction of each line
\item the current instruction, indicated as \samp{-->},
\item a labelled instruction, indicated with \samp{>\code{>}},
\item the address of the instruction,
\item the operation code name,
\item operation parameters, and
\item interpretation of the parameters in parentheses.
\end{enumerate}

The parameter interpretation recognizes local and global
variable names, constant values, branch targets, and compare
operators.
\end{funcdesc}

\begin{funcdesc}{disco}{code\optional{, lasti}}
A synonym for disassemble.  It is more convenient to type, and kept
for compatibility with earlier Python releases.
\end{funcdesc}

\begin{datadesc}{opname}
Sequence of operation names, indexable using the byte code.
\end{datadesc}

\begin{datadesc}{cmp_op}
Sequence of all compare operation names.
\end{datadesc}

\begin{datadesc}{hasconst}
Sequence of byte codes that have a constant parameter.
\end{datadesc}

\begin{datadesc}{hasfree}
Sequence of byte codes that access a free variable.
\end{datadesc}

\begin{datadesc}{hasname}
Sequence of byte codes that access an attribute by name.
\end{datadesc}

\begin{datadesc}{hasjrel}
Sequence of byte codes that have a relative jump target.
\end{datadesc}

\begin{datadesc}{hasjabs}
Sequence of byte codes that have an absolute jump target.
\end{datadesc}

\begin{datadesc}{haslocal}
Sequence of byte codes that access a local variable.
\end{datadesc}

\begin{datadesc}{hascompare}
Sequence of byte codes of Boolean operations.
\end{datadesc}

\subsection{Python Byte Code Instructions}
\label{bytecodes}

The Python compiler currently generates the following byte code
instructions.

\setindexsubitem{(byte code insns)}

\begin{opcodedesc}{STOP_CODE}{}
Indicates end-of-code to the compiler, not used by the interpreter.
\end{opcodedesc}

\begin{opcodedesc}{POP_TOP}{}
Removes the top-of-stack (TOS) item.
\end{opcodedesc}

\begin{opcodedesc}{ROT_TWO}{}
Swaps the two top-most stack items.
\end{opcodedesc}

\begin{opcodedesc}{ROT_THREE}{}
Lifts second and third stack item one position up, moves top down
to position three.
\end{opcodedesc}

\begin{opcodedesc}{ROT_FOUR}{}
Lifts second, third and forth stack item one position up, moves top down to
position four.
\end{opcodedesc}

\begin{opcodedesc}{DUP_TOP}{}
Duplicates the reference on top of the stack.
\end{opcodedesc}

Unary Operations take the top of the stack, apply the operation, and
push the result back on the stack.

\begin{opcodedesc}{UNARY_POSITIVE}{}
Implements \code{TOS = +TOS}.
\end{opcodedesc}

\begin{opcodedesc}{UNARY_NEGATIVE}{}
Implements \code{TOS = -TOS}.
\end{opcodedesc}

\begin{opcodedesc}{UNARY_NOT}{}
Implements \code{TOS = not TOS}.
\end{opcodedesc}

\begin{opcodedesc}{UNARY_CONVERT}{}
Implements \code{TOS = `TOS`}.
\end{opcodedesc}

\begin{opcodedesc}{UNARY_INVERT}{}
Implements \code{TOS = \~{}TOS}.
\end{opcodedesc}

\begin{opcodedesc}{GET_ITER}{}
Implements \code{TOS = iter(TOS)}.
\end{opcodedesc}

Binary operations remove the top of the stack (TOS) and the second top-most
stack item (TOS1) from the stack.  They perform the operation, and put the
result back on the stack.

\begin{opcodedesc}{BINARY_POWER}{}
Implements \code{TOS = TOS1 ** TOS}.
\end{opcodedesc}

\begin{opcodedesc}{BINARY_MULTIPLY}{}
Implements \code{TOS = TOS1 * TOS}.
\end{opcodedesc}

\begin{opcodedesc}{BINARY_DIVIDE}{}
Implements \code{TOS = TOS1 / TOS} when
\code{from __future__ import division} is not in effect.
\end{opcodedesc}

\begin{opcodedesc}{BINARY_FLOOR_DIVIDE}{}
Implements \code{TOS = TOS1 // TOS}.
\end{opcodedesc}

\begin{opcodedesc}{BINARY_TRUE_DIVIDE}{}
Implements \code{TOS = TOS1 / TOS} when
\code{from __future__ import division} is in effect.
\end{opcodedesc}

\begin{opcodedesc}{BINARY_MODULO}{}
Implements \code{TOS = TOS1 \%{} TOS}.
\end{opcodedesc}

\begin{opcodedesc}{BINARY_ADD}{}
Implements \code{TOS = TOS1 + TOS}.
\end{opcodedesc}

\begin{opcodedesc}{BINARY_SUBTRACT}{}
Implements \code{TOS = TOS1 - TOS}.
\end{opcodedesc}

\begin{opcodedesc}{BINARY_SUBSCR}{}
Implements \code{TOS = TOS1[TOS]}.
\end{opcodedesc}

\begin{opcodedesc}{BINARY_LSHIFT}{}
Implements \code{TOS = TOS1 <\code{}< TOS}.
\end{opcodedesc}

\begin{opcodedesc}{BINARY_RSHIFT}{}
Implements \code{TOS = TOS1 >\code{}> TOS}.
\end{opcodedesc}

\begin{opcodedesc}{BINARY_AND}{}
Implements \code{TOS = TOS1 \&\ TOS}.
\end{opcodedesc}

\begin{opcodedesc}{BINARY_XOR}{}
Implements \code{TOS = TOS1 \^\ TOS}.
\end{opcodedesc}

\begin{opcodedesc}{BINARY_OR}{}
Implements \code{TOS = TOS1 | TOS}.
\end{opcodedesc}

In-place operations are like binary operations, in that they remove TOS and
TOS1, and push the result back on the stack, but the operation is done
in-place when TOS1 supports it, and the resulting TOS may be (but does not
have to be) the original TOS1.

\begin{opcodedesc}{INPLACE_POWER}{}
Implements in-place \code{TOS = TOS1 ** TOS}.
\end{opcodedesc}

\begin{opcodedesc}{INPLACE_MULTIPLY}{}
Implements in-place \code{TOS = TOS1 * TOS}.
\end{opcodedesc}

\begin{opcodedesc}{INPLACE_DIVIDE}{}
Implements in-place \code{TOS = TOS1 / TOS} when
\code{from __future__ import division} is not in effect.
\end{opcodedesc}

\begin{opcodedesc}{INPLACE_FLOOR_DIVIDE}{}
Implements in-place \code{TOS = TOS1 // TOS}.
\end{opcodedesc}

\begin{opcodedesc}{INPLACE_TRUE_DIVIDE}{}
Implements in-place \code{TOS = TOS1 / TOS} when
\code{from __future__ import division} is in effect.
\end{opcodedesc}

\begin{opcodedesc}{INPLACE_MODULO}{}
Implements in-place \code{TOS = TOS1 \%{} TOS}.
\end{opcodedesc}

\begin{opcodedesc}{INPLACE_ADD}{}
Implements in-place \code{TOS = TOS1 + TOS}.
\end{opcodedesc}

\begin{opcodedesc}{INPLACE_SUBTRACT}{}
Implements in-place \code{TOS = TOS1 - TOS}.
\end{opcodedesc}

\begin{opcodedesc}{INPLACE_LSHIFT}{}
Implements in-place \code{TOS = TOS1 <\code{}< TOS}.
\end{opcodedesc}

\begin{opcodedesc}{INPLACE_RSHIFT}{}
Implements in-place \code{TOS = TOS1 >\code{}> TOS}.
\end{opcodedesc}

\begin{opcodedesc}{INPLACE_AND}{}
Implements in-place \code{TOS = TOS1 \&\ TOS}.
\end{opcodedesc}

\begin{opcodedesc}{INPLACE_XOR}{}
Implements in-place \code{TOS = TOS1 \^\ TOS}.
\end{opcodedesc}

\begin{opcodedesc}{INPLACE_OR}{}
Implements in-place \code{TOS = TOS1 | TOS}.
\end{opcodedesc}

The slice opcodes take up to three parameters.

\begin{opcodedesc}{SLICE+0}{}
Implements \code{TOS = TOS[:]}.
\end{opcodedesc}

\begin{opcodedesc}{SLICE+1}{}
Implements \code{TOS = TOS1[TOS:]}.
\end{opcodedesc}

\begin{opcodedesc}{SLICE+2}{}
Implements \code{TOS = TOS1[:TOS]}.
\end{opcodedesc}

\begin{opcodedesc}{SLICE+3}{}
Implements \code{TOS = TOS2[TOS1:TOS]}.
\end{opcodedesc}

Slice assignment needs even an additional parameter.  As any statement,
they put nothing on the stack.

\begin{opcodedesc}{STORE_SLICE+0}{}
Implements \code{TOS[:] = TOS1}.
\end{opcodedesc}

\begin{opcodedesc}{STORE_SLICE+1}{}
Implements \code{TOS1[TOS:] = TOS2}.
\end{opcodedesc}

\begin{opcodedesc}{STORE_SLICE+2}{}
Implements \code{TOS1[:TOS] = TOS2}.
\end{opcodedesc}

\begin{opcodedesc}{STORE_SLICE+3}{}
Implements \code{TOS2[TOS1:TOS] = TOS3}.
\end{opcodedesc}

\begin{opcodedesc}{DELETE_SLICE+0}{}
Implements \code{del TOS[:]}.
\end{opcodedesc}

\begin{opcodedesc}{DELETE_SLICE+1}{}
Implements \code{del TOS1[TOS:]}.
\end{opcodedesc}

\begin{opcodedesc}{DELETE_SLICE+2}{}
Implements \code{del TOS1[:TOS]}.
\end{opcodedesc}

\begin{opcodedesc}{DELETE_SLICE+3}{}
Implements \code{del TOS2[TOS1:TOS]}.
\end{opcodedesc}

\begin{opcodedesc}{STORE_SUBSCR}{}
Implements \code{TOS1[TOS] = TOS2}.
\end{opcodedesc}

\begin{opcodedesc}{DELETE_SUBSCR}{}
Implements \code{del TOS1[TOS]}.
\end{opcodedesc}

Miscellaneous opcodes.

\begin{opcodedesc}{PRINT_EXPR}{}
Implements the expression statement for the interactive mode.  TOS is
removed from the stack and printed.  In non-interactive mode, an
expression statement is terminated with \code{POP_STACK}.
\end{opcodedesc}

\begin{opcodedesc}{PRINT_ITEM}{}
Prints TOS to the file-like object bound to \code{sys.stdout}.  There
is one such instruction for each item in the \keyword{print} statement.
\end{opcodedesc}

\begin{opcodedesc}{PRINT_ITEM_TO}{}
Like \code{PRINT_ITEM}, but prints the item second from TOS to the
file-like object at TOS.  This is used by the extended print statement.
\end{opcodedesc}

\begin{opcodedesc}{PRINT_NEWLINE}{}
Prints a new line on \code{sys.stdout}.  This is generated as the
last operation of a \keyword{print} statement, unless the statement
ends with a comma.
\end{opcodedesc}

\begin{opcodedesc}{PRINT_NEWLINE_TO}{}
Like \code{PRINT_NEWLINE}, but prints the new line on the file-like
object on the TOS.  This is used by the extended print statement.
\end{opcodedesc}

\begin{opcodedesc}{BREAK_LOOP}{}
Terminates a loop due to a \keyword{break} statement.
\end{opcodedesc}

\begin{opcodedesc}{CONTINUE_LOOP}{target}
Continues a loop due to a \keyword{continue} statement.  \var{target}
is the address to jump to (which should be a \code{FOR_ITER}
instruction).
\end{opcodedesc}

\begin{opcodedesc}{LOAD_LOCALS}{}
Pushes a reference to the locals of the current scope on the stack.
This is used in the code for a class definition: After the class body
is evaluated, the locals are passed to the class definition.
\end{opcodedesc}

\begin{opcodedesc}{RETURN_VALUE}{}
Returns with TOS to the caller of the function.
\end{opcodedesc}

\begin{opcodedesc}{RETURN_NONE}{}
Returns \constant{None} to the caller of the function.  This opcode is
generated as the last opcode of every function and only then, for
reasons to do with tracing support.  See the comments in the function
\cfunction{maybe_call_line_trace} in \file{Python/ceval.c} for the
gory details.  \versionadded{2.3}.
\end{opcodedesc}

\begin{opcodedesc}{YIELD_VALUE}{}
Pops \code{TOS} and yields it from a generator.
\end{opcodedesc}

\begin{opcodedesc}{IMPORT_STAR}{}
Loads all symbols not starting with \character{_} directly from the module TOS
to the local namespace. The module is popped after loading all names.
This opcode implements \code{from module import *}.
\end{opcodedesc}

\begin{opcodedesc}{EXEC_STMT}{}
Implements \code{exec TOS2,TOS1,TOS}.  The compiler fills
missing optional parameters with \code{None}.
\end{opcodedesc}

\begin{opcodedesc}{POP_BLOCK}{}
Removes one block from the block stack.  Per frame, there is a 
stack of blocks, denoting nested loops, try statements, and such.
\end{opcodedesc}

\begin{opcodedesc}{END_FINALLY}{}
Terminates a \keyword{finally} clause.  The interpreter recalls
whether the exception has to be re-raised, or whether the function
returns, and continues with the outer-next block.
\end{opcodedesc}

\begin{opcodedesc}{BUILD_CLASS}{}
Creates a new class object.  TOS is the methods dictionary, TOS1
the tuple of the names of the base classes, and TOS2 the class name.
\end{opcodedesc}

All of the following opcodes expect arguments.  An argument is two
bytes, with the more significant byte last.

\begin{opcodedesc}{STORE_NAME}{namei}
Implements \code{name = TOS}. \var{namei} is the index of \var{name}
in the attribute \member{co_names} of the code object.
The compiler tries to use \code{STORE_LOCAL} or \code{STORE_GLOBAL}
if possible.
\end{opcodedesc}

\begin{opcodedesc}{DELETE_NAME}{namei}
Implements \code{del name}, where \var{namei} is the index into
\member{co_names} attribute of the code object.
\end{opcodedesc}

\begin{opcodedesc}{UNPACK_SEQUENCE}{count}
Unpacks TOS into \var{count} individual values, which are put onto
the stack right-to-left.
\end{opcodedesc}

%\begin{opcodedesc}{UNPACK_LIST}{count}
%This opcode is obsolete.
%\end{opcodedesc}

%\begin{opcodedesc}{UNPACK_ARG}{count}
%This opcode is obsolete.
%\end{opcodedesc}

\begin{opcodedesc}{DUP_TOPX}{count}
Duplicate \var{count} items, keeping them in the same order. Due to
implementation limits, \var{count} should be between 1 and 5 inclusive.
\end{opcodedesc}

\begin{opcodedesc}{STORE_ATTR}{namei}
Implements \code{TOS.name = TOS1}, where \var{namei} is the index
of name in \member{co_names}.
\end{opcodedesc}

\begin{opcodedesc}{DELETE_ATTR}{namei}
Implements \code{del TOS.name}, using \var{namei} as index into
\member{co_names}.
\end{opcodedesc}

\begin{opcodedesc}{STORE_GLOBAL}{namei}
Works as \code{STORE_NAME}, but stores the name as a global.
\end{opcodedesc}

\begin{opcodedesc}{DELETE_GLOBAL}{namei}
Works as \code{DELETE_NAME}, but deletes a global name.
\end{opcodedesc}

%\begin{opcodedesc}{UNPACK_VARARG}{argc}
%This opcode is obsolete.
%\end{opcodedesc}

\begin{opcodedesc}{LOAD_CONST}{consti}
Pushes \samp{co_consts[\var{consti}]} onto the stack.
\end{opcodedesc}

\begin{opcodedesc}{LOAD_NAME}{namei}
Pushes the value associated with \samp{co_names[\var{namei}]} onto the stack.
\end{opcodedesc}

\begin{opcodedesc}{BUILD_TUPLE}{count}
Creates a tuple consuming \var{count} items from the stack, and pushes
the resulting tuple onto the stack.
\end{opcodedesc}

\begin{opcodedesc}{BUILD_LIST}{count}
Works as \code{BUILD_TUPLE}, but creates a list.
\end{opcodedesc}

\begin{opcodedesc}{BUILD_MAP}{zero}
Pushes a new empty dictionary object onto the stack.  The argument is
ignored and set to zero by the compiler.
\end{opcodedesc}

\begin{opcodedesc}{LOAD_ATTR}{namei}
Replaces TOS with \code{getattr(TOS, co_names[\var{namei}]}.
\end{opcodedesc}

\begin{opcodedesc}{COMPARE_OP}{opname}
Performs a Boolean operation.  The operation name can be found
in \code{cmp_op[\var{opname}]}.
\end{opcodedesc}

\begin{opcodedesc}{IMPORT_NAME}{namei}
Imports the module \code{co_names[\var{namei}]}.  The module object is
pushed onto the stack.  The current namespace is not affected: for a
proper import statement, a subsequent \code{STORE_FAST} instruction
modifies the namespace.
\end{opcodedesc}

\begin{opcodedesc}{IMPORT_FROM}{namei}
Loads the attribute \code{co_names[\var{namei}]} from the module found in
TOS. The resulting object is pushed onto the stack, to be subsequently
stored by a \code{STORE_FAST} instruction.
\end{opcodedesc}

\begin{opcodedesc}{JUMP_FORWARD}{delta}
Increments byte code counter by \var{delta}.
\end{opcodedesc}

\begin{opcodedesc}{JUMP_IF_TRUE}{delta}
If TOS is true, increment the byte code counter by \var{delta}.  TOS is
left on the stack.
\end{opcodedesc}

\begin{opcodedesc}{JUMP_IF_FALSE}{delta}
If TOS is false, increment the byte code counter by \var{delta}.  TOS
is not changed. 
\end{opcodedesc}

\begin{opcodedesc}{JUMP_ABSOLUTE}{target}
Set byte code counter to \var{target}.
\end{opcodedesc}

\begin{opcodedesc}{FOR_ITER}{delta}
\code{TOS} is an iterator.  Call its \method{next()} method.  If this
yields a new value, push it on the stack (leaving the iterator below
it).  If the iterator indicates it is exhausted  \code{TOS} is
popped, and the byte code counter is incremented by \var{delta}.
\end{opcodedesc}

%\begin{opcodedesc}{FOR_LOOP}{delta}
%This opcode is obsolete.
%\end{opcodedesc}

%\begin{opcodedesc}{LOAD_LOCAL}{namei}
%This opcode is obsolete.
%\end{opcodedesc}

\begin{opcodedesc}{LOAD_GLOBAL}{namei}
Loads the global named \code{co_names[\var{namei}]} onto the stack.
\end{opcodedesc}

%\begin{opcodedesc}{SET_FUNC_ARGS}{argc}
%This opcode is obsolete.
%\end{opcodedesc}

\begin{opcodedesc}{SETUP_LOOP}{delta}
Pushes a block for a loop onto the block stack.  The block spans
from the current instruction with a size of \var{delta} bytes.
\end{opcodedesc}

\begin{opcodedesc}{SETUP_EXCEPT}{delta}
Pushes a try block from a try-except clause onto the block stack.
\var{delta} points to the first except block.
\end{opcodedesc}

\begin{opcodedesc}{SETUP_FINALLY}{delta}
Pushes a try block from a try-except clause onto the block stack.
\var{delta} points to the finally block.
\end{opcodedesc}

\begin{opcodedesc}{LOAD_FAST}{var_num}
Pushes a reference to the local \code{co_varnames[\var{var_num}]} onto
the stack.
\end{opcodedesc}

\begin{opcodedesc}{STORE_FAST}{var_num}
Stores TOS into the local \code{co_varnames[\var{var_num}]}.
\end{opcodedesc}

\begin{opcodedesc}{DELETE_FAST}{var_num}
Deletes local \code{co_varnames[\var{var_num}]}.
\end{opcodedesc}

\begin{opcodedesc}{LOAD_CLOSURE}{i}
Pushes a reference to the cell contained in slot \var{i} of the
cell and free variable storage.  The name of the variable is 
\code{co_cellvars[\var{i}]} if \var{i} is less than the length of
\var{co_cellvars}.  Otherwise it is 
\code{co_freevars[\var{i} - len(co_cellvars)]}.
\end{opcodedesc}

\begin{opcodedesc}{LOAD_DEREF}{i}
Loads the cell contained in slot \var{i} of the cell and free variable
storage.  Pushes a reference to the object the cell contains on the
stack. 
\end{opcodedesc}

\begin{opcodedesc}{STORE_DEREF}{i}
Stores TOS into the cell contained in slot \var{i} of the cell and
free variable storage.
\end{opcodedesc}

\begin{opcodedesc}{SET_LINENO}{lineno}
This opcode is obsolete.
\end{opcodedesc}

\begin{opcodedesc}{RAISE_VARARGS}{argc}
Raises an exception. \var{argc} indicates the number of parameters
to the raise statement, ranging from 0 to 3.  The handler will find
the traceback as TOS2, the parameter as TOS1, and the exception
as TOS.
\end{opcodedesc}

\begin{opcodedesc}{CALL_FUNCTION}{argc}
Calls a function.  The low byte of \var{argc} indicates the number of
positional parameters, the high byte the number of keyword parameters.
On the stack, the opcode finds the keyword parameters first.  For each
keyword argument, the value is on top of the key.  Below the keyword
parameters, the positional parameters are on the stack, with the
right-most parameter on top.  Below the parameters, the function object
to call is on the stack.
\end{opcodedesc}

\begin{opcodedesc}{MAKE_FUNCTION}{argc}
Pushes a new function object on the stack.  TOS is the code associated
with the function.  The function object is defined to have \var{argc}
default parameters, which are found below TOS.
\end{opcodedesc}

\begin{opcodedesc}{MAKE_CLOSURE}{argc}
Creates a new function object, sets its \var{func_closure} slot, and
pushes it on the stack.  TOS is the code associated with the function.
If the code object has N free variables, the next N items on the stack
are the cells for these variables.  The function also has \var{argc}
default parameters, where are found before the cells.
\end{opcodedesc}

\begin{opcodedesc}{BUILD_SLICE}{argc}
Pushes a slice object on the stack.  \var{argc} must be 2 or 3.  If it
is 2, \code{slice(TOS1, TOS)} is pushed; if it is 3,
\code{slice(TOS2, TOS1, TOS)} is pushed.
See the \code{slice()}\bifuncindex{slice} built-in function for more
information.
\end{opcodedesc}

\begin{opcodedesc}{EXTENDED_ARG}{ext}
Prefixes any opcode which has an argument too big to fit into the
default two bytes.  \var{ext} holds two additional bytes which, taken
together with the subsequent opcode's argument, comprise a four-byte
argument, \var{ext} being the two most-significant bytes.
\end{opcodedesc}

\begin{opcodedesc}{CALL_FUNCTION_VAR}{argc}
Calls a function. \var{argc} is interpreted as in \code{CALL_FUNCTION}.
The top element on the stack contains the variable argument list, followed
by keyword and positional arguments.
\end{opcodedesc}

\begin{opcodedesc}{CALL_FUNCTION_KW}{argc}
Calls a function. \var{argc} is interpreted as in \code{CALL_FUNCTION}.
The top element on the stack contains the keyword arguments dictionary, 
followed by explicit keyword and positional arguments.
\end{opcodedesc}

\begin{opcodedesc}{CALL_FUNCTION_VAR_KW}{argc}
Calls a function. \var{argc} is interpreted as in
\code{CALL_FUNCTION}.  The top element on the stack contains the
keyword arguments dictionary, followed by the variable-arguments
tuple, followed by explicit keyword and positional arguments.
\end{opcodedesc}

\section{\module{curses} ---
         Terminal handling for character-cell displays}

\declaremodule{standard}{curses}
\sectionauthor{Moshe Zadka}{moshez@zadka.site.co.il}
\sectionauthor{Eric Raymond}{esr@thyrsus.com}
\modulesynopsis{An interface to the curses library, providing portable
                terminal handling.}

\versionchanged[Added support for the \code{ncurses} library and
                converted to a package]{1.6}

The \module{curses} module provides an interface to the curses
library, the de-facto standard for portable advanced terminal
handling.

While curses is most widely used in the \UNIX{} environment, versions
are available for DOS, OS/2, and possibly other systems as well.  This
extension module is designed to match the API of ncurses, an
open-source curses library hosted on Linux and the BSD variants of
\UNIX.

\begin{seealso}
  \seemodule{curses.ascii}{Utilities for working with \ASCII{}
                           characters, regardless of your locale
                           settings.}
  \seemodule{curses.panel}{A panel stack extension that adds depth to 
                           curses windows.}
  \seemodule{curses.textpad}{Editable text widget for curses supporting 
                             \program{Emacs}-like bindings.}
  \seemodule{curses.wrapper}{Convenience function to ensure proper
                             terminal setup and resetting on
                             application entry and exit.}
  \seetitle[http://www.python.org/doc/howto/curses/curses.html]{Curses
            Programming with Python}{Tutorial material on using curses
            with Python, by Andrew Kuchling and Eric Raymond, is
            available on the Python Web site.}
  \seetext{The \file{Demo/curses/} directory in the Python source
           distribution contains some example programs using the
           curses bindings provided by this module.}
\end{seealso}


\subsection{Functions \label{curses-functions}}

The module \module{curses} defines the following exception:

\begin{excdesc}{error}
Exception raised when a curses library function returns an error.
\end{excdesc}

\note{Whenever \var{x} or \var{y} arguments to a function
or a method are optional, they default to the current cursor location.
Whenever \var{attr} is optional, it defaults to \constant{A_NORMAL}.}

The module \module{curses} defines the following functions:

\begin{funcdesc}{baudrate}{}
Returns the output speed of the terminal in bits per second.  On
software terminal emulators it will have a fixed high value.
Included for historical reasons; in former times, it was used to 
write output loops for time delays and occasionally to change
interfaces depending on the line speed.
\end{funcdesc}

\begin{funcdesc}{beep}{}
Emit a short attention sound.
\end{funcdesc}

\begin{funcdesc}{can_change_color}{}
Returns true or false, depending on whether the programmer can change
the colors displayed by the terminal.
\end{funcdesc}

\begin{funcdesc}{cbreak}{}
Enter cbreak mode.  In cbreak mode (sometimes called ``rare'' mode)
normal tty line buffering is turned off and characters are available
to be read one by one.  However, unlike raw mode, special characters
(interrupt, quit, suspend, and flow control) retain their effects on
the tty driver and calling program.  Calling first \function{raw()}
then \function{cbreak()} leaves the terminal in cbreak mode.
\end{funcdesc}

\begin{funcdesc}{color_content}{color_number}
Returns the intensity of the red, green, and blue (RGB) components in
the color \var{color_number}, which must be between \code{0} and
\constant{COLORS}.  A 3-tuple is returned, containing the R,G,B values
for the given color, which will be between \code{0} (no component) and
\code{1000} (maximum amount of component).
\end{funcdesc}

\begin{funcdesc}{color_pair}{color_number}
Returns the attribute value for displaying text in the specified
color.  This attribute value can be combined with
\constant{A_STANDOUT}, \constant{A_REVERSE}, and the other
\constant{A_*} attributes.  \function{pair_number()} is the
counterpart to this function.
\end{funcdesc}

\begin{funcdesc}{curs_set}{visibility}
Sets the cursor state.  \var{visibility} can be set to 0, 1, or 2, for
invisible, normal, or very visible.  If the terminal supports the
visibility requested, the previous cursor state is returned;
otherwise, an exception is raised.  On many terminals, the ``visible''
mode is an underline cursor and the ``very visible'' mode is a block cursor.
\end{funcdesc}

\begin{funcdesc}{def_prog_mode}{}
Saves the current terminal mode as the ``program'' mode, the mode when
the running program is using curses.  (Its counterpart is the
``shell'' mode, for when the program is not in curses.)  Subsequent calls
to \function{reset_prog_mode()} will restore this mode.
\end{funcdesc}

\begin{funcdesc}{def_shell_mode}{}
Saves the current terminal mode as the ``shell'' mode, the mode when
the running program is not using curses.  (Its counterpart is the
``program'' mode, when the program is using curses capabilities.)
Subsequent calls
to \function{reset_shell_mode()} will restore this mode.
\end{funcdesc}

\begin{funcdesc}{delay_output}{ms}
Inserts an \var{ms} millisecond pause in output.  
\end{funcdesc}

\begin{funcdesc}{doupdate}{}
Update the physical screen.  The curses library keeps two data
structures, one representing the current physical screen contents
and a virtual screen representing the desired next state.  The
\function{doupdate()} ground updates the physical screen to match the
virtual screen.

The virtual screen may be updated by a \method{noutrefresh()} call
after write operations such as \method{addstr()} have been performed
on a window.  The normal \method{refresh()} call is simply
\method{noutrefresh()} followed by \function{doupdate()}; if you have
to update multiple windows, you can speed performance and perhaps
reduce screen flicker by issuing \method{noutrefresh()} calls on
all windows, followed by a single \function{doupdate()}.
\end{funcdesc}

\begin{funcdesc}{echo}{}
Enter echo mode.  In echo mode, each character input is echoed to the
screen as it is entered.  
\end{funcdesc}

\begin{funcdesc}{endwin}{}
De-initialize the library, and return terminal to normal status.
\end{funcdesc}

\begin{funcdesc}{erasechar}{}
Returns the user's current erase character.  Under \UNIX{} operating
systems this is a property of the controlling tty of the curses
program, and is not set by the curses library itself.
\end{funcdesc}

\begin{funcdesc}{filter}{}
The \function{filter()} routine, if used, must be called before
\function{initscr()} is  called.  The effect is that, during those
calls, LINES is set to 1; the capabilities clear, cup, cud, cud1,
cuu1, cuu, vpa are disabled; and the home string is set to the value of cr.
The effect is that the cursor is confined to the current line, and so
are screen updates.  This may be used for enabling cgaracter-at-a-time 
line editing without touching the rest of the screen.
\end{funcdesc}

\begin{funcdesc}{flash}{}
Flash the screen.  That is, change it to reverse-video and then change
it back in a short interval.  Some people prefer such as `visible bell'
to the audible attention signal produced by \function{beep()}.
\end{funcdesc}

\begin{funcdesc}{flushinp}{}
Flush all input buffers.  This throws away any  typeahead  that  has
been typed by the user and has not yet been processed by the program.
\end{funcdesc}

\begin{funcdesc}{getmouse}{}
After \method{getch()} returns \constant{KEY_MOUSE} to signal a mouse
event, this method should be call to retrieve the queued mouse event,
represented as a 5-tuple
\code{(\var{id}, \var{x}, \var{y}, \var{z}, \var{bstate})}.
\var{id} is an ID value used to distinguish multiple devices,
and \var{x}, \var{y}, \var{z} are the event's coordinates.  (\var{z}
is currently unused.).  \var{bstate} is an integer value whose bits
will be set to indicate the type of event, and will be the bitwise OR
of one or more of the following constants, where \var{n} is the button
number from 1 to 4:
\constant{BUTTON\var{n}_PRESSED},
\constant{BUTTON\var{n}_RELEASED},
\constant{BUTTON\var{n}_CLICKED},
\constant{BUTTON\var{n}_DOUBLE_CLICKED},
\constant{BUTTON\var{n}_TRIPLE_CLICKED},
\constant{BUTTON_SHIFT},
\constant{BUTTON_CTRL},
\constant{BUTTON_ALT}.
\end{funcdesc}

\begin{funcdesc}{getsyx}{}
Returns the current coordinates of the virtual screen cursor in y and
x.  If leaveok is currently true, then -1,-1 is returned.
\end{funcdesc}

\begin{funcdesc}{getwin}{file}
Reads window related data stored in the file by an earlier
\function{putwin()} call.  The routine then creates and initializes a
new window using that data, returning the new window object.
\end{funcdesc}

\begin{funcdesc}{has_colors}{}
Returns true if the terminal can display colors; otherwise, it
returns false. 
\end{funcdesc}

\begin{funcdesc}{has_ic}{}
Returns true if the terminal has insert- and delete- character
capabilities.  This function is included for historical reasons only,
as all modern software terminal emulators have such capabilities.
\end{funcdesc}

\begin{funcdesc}{has_il}{}
Returns true if the terminal has insert- and
delete-line  capabilities,  or  can  simulate  them  using
scrolling regions. This function is included for historical reasons only,
as all modern software terminal emulators have such capabilities.
\end{funcdesc}

\begin{funcdesc}{has_key}{ch}
Takes a key value \var{ch}, and returns true if the current terminal
type recognizes a key with that value.
\end{funcdesc}

\begin{funcdesc}{halfdelay}{tenths}
Used for half-delay mode, which is similar to cbreak mode in that
characters typed by the user are immediately available to the program.
However, after blocking for \var{tenths} tenths of seconds, an
exception is raised if nothing has been typed.  The value of
\var{tenths} must be a number between 1 and 255.  Use
\function{nocbreak()} to leave half-delay mode.
\end{funcdesc}

\begin{funcdesc}{init_color}{color_number, r, g, b}
Changes the definition of a color, taking the number of the color to
be changed followed by three RGB values (for the amounts of red,
green, and blue components).  The value of \var{color_number} must be
between \code{0} and \constant{COLORS}.  Each of \var{r}, \var{g},
\var{b}, must be a value between \code{0} and \code{1000}.  When
\function{init_color()} is used, all occurrences of that color on the
screen immediately change to the new definition.  This function is a
no-op on most terminals; it is active only if
\function{can_change_color()} returns \code{1}.
\end{funcdesc}

\begin{funcdesc}{init_pair}{pair_number, fg, bg}
Changes the definition of a color-pair.  It takes three arguments: the
number of the color-pair to be changed, the foreground color number,
and the background color number.  The value of \var{pair_number} must
be between \code{1} and \code{COLOR_PAIRS - 1} (the \code{0} color
pair is wired to white on black and cannot be changed).  The value of
\var{fg} and \var{bg} arguments must be between \code{0} and
\constant{COLORS}.  If the color-pair was previously initialized, the
screen is refreshed and all occurrences of that color-pair are changed
to the new definition.
\end{funcdesc}

\begin{funcdesc}{initscr}{}
Initialize the library. Returns a \class{WindowObject} which represents
the whole screen.  \note{If there is an error opening the terminal,
the underlying curses library may cause the interpreter to exit.}
\end{funcdesc}

\begin{funcdesc}{isendwin}{}
Returns true if \function{endwin()} has been called (that is, the 
curses library has been deinitialized).
\end{funcdesc}

\begin{funcdesc}{keyname}{k}
Return the name of the key numbered \var{k}.  The name of a key
generating printable ASCII character is the key's character.  The name
of a control-key combination is a two-character string consisting of a
caret followed by the corresponding printable ASCII character.  The
name of an alt-key combination (128-255) is a string consisting of the
prefix `M-' followed by the name of the corresponding ASCII character.
\end{funcdesc}

\begin{funcdesc}{killchar}{}
Returns the user's current line kill character. Under \UNIX{} operating
systems this is a property of the controlling tty of the curses
program, and is not set by the curses library itself.
\end{funcdesc}

\begin{funcdesc}{longname}{}
Returns a string containing the terminfo long name field describing the current
terminal.  The maximum length of a verbose description is 128
characters.  It is defined only after the call to
\function{initscr()}.
\end{funcdesc}

\begin{funcdesc}{meta}{yes}
If \var{yes} is 1, allow 8-bit characters to be input. If \var{yes} is 0, 
allow only 7-bit chars.
\end{funcdesc}

\begin{funcdesc}{mouseinterval}{interval}
Sets the maximum time in milliseconds that can elapse between press and
release events in order for them to be recognized as a click, and
returns the previous interval value.  The default value is 200 msec,
or one fifth of a second.
\end{funcdesc}

\begin{funcdesc}{mousemask}{mousemask}
Sets the mouse events to be reported, and returns a tuple
\code{(\var{availmask}, \var{oldmask})}.  
\var{availmask} indicates which of the
specified mouse events can be reported; on complete failure it returns
0.  \var{oldmask} is the previous value of the given window's mouse
event mask.  If this function is never called, no mouse events are
ever reported.
\end{funcdesc}

\begin{funcdesc}{napms}{ms}
Sleep for \var{ms} milliseconds.
\end{funcdesc}

\begin{funcdesc}{newpad}{nlines, ncols}
Creates and returns a pointer to a new pad data structure with the
given number of lines and columns.  A pad is returned as a
window object.

A pad is like a window, except that it is not restricted by the screen
size, and is not necessarily associated with a particular part of the
screen.  Pads can be used when a large window is needed, and only a
part of the window will be on the screen at one time.  Automatic
refreshes of pads (such as from scrolling or echoing of input) do not
occur.  The \method{refresh()} and \method{noutrefresh()} methods of a
pad require 6 arguments to specify the part of the pad to be
displayed and the location on the screen to be used for the display.
The arguments are pminrow, pmincol, sminrow, smincol, smaxrow,
smaxcol; the p arguments refer to the upper left corner of the pad
region to be displayed and the s arguments define a clipping box on
the screen within which the pad region is to be displayed.
\end{funcdesc}

\begin{funcdesc}{newwin}{\optional{nlines, ncols,} begin_y, begin_x}
Return a new window, whose left-upper corner is at 
\code{(\var{begin_y}, \var{begin_x})}, and whose height/width is 
\var{nlines}/\var{ncols}.  

By default, the window will extend from the 
specified position to the lower right corner of the screen.
\end{funcdesc}

\begin{funcdesc}{nl}{}
Enter newline mode.  This mode translates the return key into newline
on input, and translates newline into return and line-feed on output.
Newline mode is initially on.
\end{funcdesc}

\begin{funcdesc}{nocbreak}{}
Leave cbreak mode.  Return to normal ``cooked'' mode with line buffering.
\end{funcdesc}

\begin{funcdesc}{noecho}{}
Leave echo mode.  Echoing of input characters is turned off,
\end{funcdesc}

\begin{funcdesc}{nonl}{}
Leave newline mode.  Disable translation of return into newline on
input, and disable low-level translation of newline into
newline/return on output (but this does not change the behavior of
\code{addch('\e n')}, which always does the equivalent of return and
line feed on the virtual screen).  With translation off, curses can
sometimes speed up vertical motion a little; also, it will be able to
detect the return key on input.
\end{funcdesc}

\begin{funcdesc}{noqiflush}{}
When the noqiflush routine is used, normal flush of input and
output queues associated with the INTR, QUIT and SUSP
characters will not be done.  You may want to call
\function{noqiflush()} in a signal handler if you want output
to continue as though the interrupt had not occurred, after the
handler exits.
\end{funcdesc}

\begin{funcdesc}{noraw}{}
Leave raw mode. Return to normal ``cooked'' mode with line buffering.
\end{funcdesc}

\begin{funcdesc}{pair_content}{pair_number}
Returns a tuple \code{(\var{fg}, \var{bg})} containing the colors for
the requested color pair.  The value of \var{pair_number} must be
between \code{0} and \code{\constant{COLOR_PAIRS} - 1}.
\end{funcdesc}

\begin{funcdesc}{pair_number}{attr}
Returns the number of the color-pair set by the attribute value
\var{attr}.  \function{color_pair()} is the counterpart to this
function.
\end{funcdesc}

\begin{funcdesc}{putp}{string}
Equivalent to \code{tputs(str, 1, putchar)}; emits the value of a
specified terminfo capability for the current terminal.  Note that the
output of putp always goes to standard output.
\end{funcdesc}

\begin{funcdesc}{qiflush}{ \optional{flag} }
If \var{flag} is false, the effect is the same as calling
\function{noqiflush()}. If \var{flag} is true, or no argument is
provided, the queues will be flushed when these control characters are
read.
\end{funcdesc}

\begin{funcdesc}{raw}{}
Enter raw mode.  In raw mode, normal line buffering and 
processing of interrupt, quit, suspend, and flow control keys are
turned off; characters are presented to curses input functions one
by one.
\end{funcdesc}

\begin{funcdesc}{reset_prog_mode}{}
Restores the  terminal  to ``program'' mode, as previously saved 
by \function{def_prog_mode()}.
\end{funcdesc}

\begin{funcdesc}{reset_shell_mode}{}
Restores the  terminal  to ``shell'' mode, as previously saved 
by \function{def_shell_mode()}.
\end{funcdesc}

\begin{funcdesc}{setsyx}{y, x}
Sets the virtual screen cursor to \var{y}, \var{x}.
If \var{y} and \var{x} are both -1, then leaveok is set.  
\end{funcdesc}

\begin{funcdesc}{setupterm}{\optional{termstr, fd}}
Initializes the terminal.  \var{termstr} is a string giving the
terminal name; if omitted, the value of the TERM environment variable
will be used.  \var{fd} is the file descriptor to which any
initialization sequences will be sent; if not supplied, the file
descriptor for \code{sys.stdout} will be used.
\end{funcdesc}

\begin{funcdesc}{start_color}{}
Must be called if the programmer wants to use colors, and before any
other color manipulation routine is called.  It is good
practice to call this routine right after \function{initscr()}.

\function{start_color()} initializes eight basic colors (black, red, 
green, yellow, blue, magenta, cyan, and white), and two global
variables in the \module{curses} module, \constant{COLORS} and
\constant{COLOR_PAIRS}, containing the maximum number of colors and
color-pairs the terminal can support.  It also restores the colors on
the terminal to the values they had when the terminal was just turned
on.
\end{funcdesc}

\begin{funcdesc}{termattrs}{}
Returns a logical OR of all video attributes supported by the
terminal.  This information is useful when a curses program needs
complete control over the appearance of the screen.
\end{funcdesc}

\begin{funcdesc}{termname}{}
Returns the value of the environment variable TERM, truncated to 14
characters.
\end{funcdesc}

\begin{funcdesc}{tigetflag}{capname}
Returns the value of the Boolean capability corresponding to the
terminfo capability name \var{capname}.  The value \code{-1} is
returned if \var{capname} is not a Boolean capability, or \code{0} if
it is canceled or absent from the terminal description.
\end{funcdesc}

\begin{funcdesc}{tigetnum}{capname}
Returns the value of the numeric capability corresponding to the
terminfo capability name \var{capname}.  The value \code{-2} is
returned if \var{capname} is not a numeric capability, or \code{-1} if
it is canceled or absent from the terminal description.  
\end{funcdesc}

\begin{funcdesc}{tigetstr}{capname}
Returns the value of the string capability corresponding to the
terminfo capability name \var{capname}.  \code{None} is returned if
\var{capname} is not a string capability, or is canceled or absent
from the terminal description.
\end{funcdesc}

\begin{funcdesc}{tparm}{str\optional{,...}}
Instantiates the string \var{str} with the supplied parameters, where 
\var{str} should be a parameterized string obtained from the terminfo 
database.  E.g. \code{tparm(tigetstr("cup"), 5, 3)} could result in 
\code{'\e{}033[6;4H'}, the exact result depending on terminal type.
\end{funcdesc}

\begin{funcdesc}{typeahead}{fd}
Specifies that the file descriptor \var{fd} be used for typeahead
checking.  If \var{fd} is \code{-1}, then no typeahead checking is
done.

The curses library does ``line-breakout optimization'' by looking for
typeahead periodically while updating the screen.  If input is found,
and it is coming from a tty, the current update is postponed until
refresh or doupdate is called again, allowing faster response to
commands typed in advance. This function allows specifying a different
file descriptor for typeahead checking.
\end{funcdesc}

\begin{funcdesc}{unctrl}{ch}
Returns a string which is a printable representation of the character
\var{ch}.  Control characters are displayed as a caret followed by the
character, for example as \code{\textasciicircum C}. Printing
characters are left as they are.
\end{funcdesc}

\begin{funcdesc}{ungetch}{ch}
Push \var{ch} so the next \method{getch()} will return it.
\note{Only one \var{ch} can be pushed before \method{getch()}
is called.}
\end{funcdesc}

\begin{funcdesc}{ungetmouse}{id, x, y, z, bstate}
Push a \constant{KEY_MOUSE} event onto the input queue, associating
the given state data with it.
\end{funcdesc}

\begin{funcdesc}{use_env}{flag}
If used, this function should be called before \function{initscr()} or
newterm are called.  When \var{flag} is false, the values of
lines and columns specified in the terminfo database will be
used, even if environment variables \envvar{LINES} and
\envvar{COLUMNS} (used by default) are set, or if curses is running in
a window (in which case default behavior would be to use the window
size if \envvar{LINES} and \envvar{COLUMNS} are not set).
\end{funcdesc}

\begin{funcdesc}{use_default_colors}{}
Allow use of default values for colors on terminals supporting this
feature. Use this to support transparency in your
application.  The default color is assigned to the color number -1.
After calling this function, 
\code{init_pair(x, curses.COLOR_RED, -1)} initializes, for instance,
color pair \var{x} to a red foreground color on the default background.
\end{funcdesc}

\subsection{Window Objects \label{curses-window-objects}}

Window objects, as returned by \function{initscr()} and
\function{newwin()} above, have the
following methods:

\begin{methoddesc}[window]{addch}{\optional{y, x,} ch\optional{, attr}}
\note{A \emph{character} means a C character (an
\ASCII{} code), rather then a Python character (a string of length 1).
(This note is true whenever the documentation mentions a character.)
The builtin \function{ord()} is handy for conveying strings to codes.}

Paint character \var{ch} at \code{(\var{y}, \var{x})} with attributes
\var{attr}, overwriting any character previously painter at that
location.  By default, the character position and attributes are the
current settings for the window object.
\end{methoddesc}

\begin{methoddesc}[window]{addnstr}{\optional{y, x,} str, n\optional{, attr}}
Paint at most \var{n} characters of the 
string \var{str} at \code{(\var{y}, \var{x})} with attributes
\var{attr}, overwriting anything previously on the display.
\end{methoddesc}

\begin{methoddesc}[window]{addstr}{\optional{y, x,} str\optional{, attr}}
Paint the string \var{str} at \code{(\var{y}, \var{x})} with attributes
\var{attr}, overwriting anything previously on the display.
\end{methoddesc}

\begin{methoddesc}[window]{attroff}{attr}
Remove attribute \var{attr} from the ``background'' set applied to all
writes to the current window.
\end{methoddesc}

\begin{methoddesc}[window]{attron}{attr}
Add attribute \var{attr} from the ``background'' set applied to all
writes to the current window.
\end{methoddesc}

\begin{methoddesc}[window]{attrset}{attr}
Set the ``background'' set of attributes to \var{attr}.  This set is
initially 0 (no attributes).
\end{methoddesc}

\begin{methoddesc}[window]{bkgd}{ch\optional{, attr}}
Sets the background property of the window to the character \var{ch},
with attributes \var{attr}.  The change is then applied to every
character position in that window:
\begin{itemize}
\item  
The attribute of every character in the window  is
changed to the new background attribute.
\item
Wherever  the  former background character appears,
it is changed to the new background character.
\end{itemize}

\end{methoddesc}

\begin{methoddesc}[window]{bkgdset}{ch\optional{, attr}}
Sets the window's background.  A window's background consists of a
character and any combination of attributes.  The attribute part of
the background is combined (OR'ed) with all non-blank characters that
are written into the window.  Both the character and attribute parts
of the background are combined with the blank characters.  The
background becomes a property of the character and moves with the
character through any scrolling and insert/delete line/character
operations.
\end{methoddesc}

\begin{methoddesc}[window]{border}{\optional{ls\optional{, rs\optional{,
                                   ts\optional{, bs\optional{, tl\optional{,
                                   tr\optional{, bl\optional{, br}}}}}}}}}
Draw a border around the edges of the window. Each parameter specifies 
the character to use for a specific part of the border; see the table
below for more details.  The characters can be specified as integers
or as one-character strings.

\note{A \code{0} value for any parameter will cause the
default character to be used for that parameter.  Keyword parameters
can \emph{not} be used.  The defaults are listed in this table:}

\begin{tableiii}{l|l|l}{var}{Parameter}{Description}{Default value}
  \lineiii{ls}{Left side}{\constant{ACS_VLINE}}
  \lineiii{rs}{Right side}{\constant{ACS_VLINE}}
  \lineiii{ts}{Top}{\constant{ACS_HLINE}}
  \lineiii{bs}{Bottom}{\constant{ACS_HLINE}}
  \lineiii{tl}{Upper-left corner}{\constant{ACS_ULCORNER}}
  \lineiii{tr}{Upper-right corner}{\constant{ACS_URCORNER}}
  \lineiii{bl}{Bottom-left corner}{\constant{ACS_BLCORNER}}
  \lineiii{br}{Bottom-right corner}{\constant{ACS_BRCORNER}}
\end{tableiii}
\end{methoddesc}

\begin{methoddesc}[window]{box}{\optional{vertch, horch}}
Similar to \method{border()}, but both \var{ls} and \var{rs} are
\var{vertch} and both \var{ts} and {bs} are \var{horch}.  The default
corner characters are always used by this function.
\end{methoddesc}

\begin{methoddesc}[window]{clear}{}
Like \method{erase()}, but also causes the whole window to be repainted
upon next call to \method{refresh()}.
\end{methoddesc}

\begin{methoddesc}[window]{clearok}{yes}
If \var{yes} is 1, the next call to \method{refresh()}
will clear the window completely.
\end{methoddesc}

\begin{methoddesc}[window]{clrtobot}{}
Erase from cursor to the end of the window: all lines below the cursor
are deleted, and then the equivalent of \method{clrtoeol()} is performed.
\end{methoddesc}

\begin{methoddesc}[window]{clrtoeol}{}
Erase from cursor to the end of the line.
\end{methoddesc}

\begin{methoddesc}[window]{cursyncup}{}
Updates the current cursor position of all the ancestors of the window
to reflect the current cursor position of the window.
\end{methoddesc}

\begin{methoddesc}[window]{delch}{\optional{x, y}}
Delete any character at \code{(\var{y}, \var{x})}.
\end{methoddesc}

\begin{methoddesc}[window]{deleteln}{}
Delete the line under the cursor. All following lines are moved up
by 1 line.
\end{methoddesc}

\begin{methoddesc}[window]{derwin}{\optional{nlines, ncols,} begin_y, begin_x}
An abbreviation for ``derive window'', \method{derwin()} is the same
as calling \method{subwin()}, except that \var{begin_y} and
\var{begin_x} are relative to the origin of the window, rather than
relative to the entire screen.  Returns a window object for the
derived window.
\end{methoddesc}

\begin{methoddesc}[window]{echochar}{ch\optional{, attr}}
Add character \var{ch} with attribute \var{attr}, and immediately 
call \method{refresh()} on the window.
\end{methoddesc}

\begin{methoddesc}[window]{enclose}{y, x}
Tests whether the given pair of screen-relative character-cell
coordinates are enclosed by the given window, returning true or
false.  It is useful for determining what subset of the screen
windows enclose the location of a mouse event.
\end{methoddesc}

\begin{methoddesc}[window]{erase}{}
Clear the window.
\end{methoddesc}

\begin{methoddesc}[window]{getbegyx}{}
Return a tuple \code{(\var{y}, \var{x})} of co-ordinates of upper-left
corner.
\end{methoddesc}

\begin{methoddesc}[window]{getch}{\optional{x, y}}
Get a character. Note that the integer returned does \emph{not} have to
be in \ASCII{} range: function keys, keypad keys and so on return numbers
higher than 256. In no-delay mode, -1 is returned if there is 
no input.
\end{methoddesc}

\begin{methoddesc}[window]{getkey}{\optional{x, y}}
Get a character, returning a string instead of an integer, as
\method{getch()} does. Function keys, keypad keys and so on return a
multibyte string containing the key name.  In no-delay mode, an
exception is raised if there is no input.
\end{methoddesc}

\begin{methoddesc}[window]{getmaxyx}{}
Return a tuple \code{(\var{y}, \var{x})} of the height and width of
the window.
\end{methoddesc}

\begin{methoddesc}[window]{getparyx}{}
Returns the beginning coordinates of this window relative to its
parent window into two integer variables y and x.  Returns
\code{-1,-1} if this window has no parent.
\end{methoddesc}

\begin{methoddesc}[window]{getstr}{\optional{x, y}}
Read a string from the user, with primitive line editing capacity.
\end{methoddesc}

\begin{methoddesc}[window]{getyx}{}
Return a tuple \code{(\var{y}, \var{x})} of current cursor position 
relative to the window's upper-left corner.
\end{methoddesc}

\begin{methoddesc}[window]{hline}{\optional{y, x,} ch, n}
Display a horizontal line starting at \code{(\var{y}, \var{x})} with
length \var{n} consisting of the character \var{ch}.
\end{methoddesc}

\begin{methoddesc}[window]{idcok}{flag}
If \var{flag} is false, curses no longer considers using the hardware
insert/delete character feature of the terminal; if \var{flag} is
true, use of character insertion and deletion is enabled.  When curses
is first initialized, use of character insert/delete is enabled by
default.
\end{methoddesc}

\begin{methoddesc}[window]{idlok}{yes}
If called with \var{yes} equal to 1, \module{curses} will try and use
hardware line editing facilities. Otherwise, line insertion/deletion
are disabled.
\end{methoddesc}

\begin{methoddesc}[window]{immedok}{flag}
If \var{flag} is true, any change in the window image
automatically causes the window to be refreshed; you no longer
have to call \method{refresh()} yourself.  However, it may
degrade performance considerably, due to repeated calls to
wrefresh.  This option is disabled by default.
\end{methoddesc}

\begin{methoddesc}[window]{inch}{\optional{x, y}}
Return the character at the given position in the window. The bottom
8 bits are the character proper, and upper bits are the attributes.
\end{methoddesc}

\begin{methoddesc}[window]{insch}{\optional{y, x,} ch\optional{, attr}}
Paint character \var{ch} at \code{(\var{y}, \var{x})} with attributes
\var{attr}, moving the line from position \var{x} right by one
character.
\end{methoddesc}

\begin{methoddesc}[window]{insdelln}{nlines}
Inserts \var{nlines} lines into the specified window above the current
line.  The \var{nlines} bottom lines are lost.  For negative
\var{nlines}, delete \var{nlines} lines starting with the one under
the cursor, and move the remaining lines up.  The bottom \var{nlines}
lines are cleared.  The current cursor position remains the same.
\end{methoddesc}

\begin{methoddesc}[window]{insertln}{}
Insert a blank line under the cursor. All following lines are moved
down by 1 line.
\end{methoddesc}

\begin{methoddesc}[window]{insnstr}{\optional{y, x,} str, n \optional{, attr}}
Insert a character string (as many characters as will fit on the line)
before the character under the cursor, up to \var{n} characters.  
If \var{n} is zero or negative,
the entire string is inserted.
All characters to the right of
the cursor are shifted right, with the rightmost characters on the
line being lost.  The cursor position does not change (after moving to
\var{y}, \var{x}, if specified). 
\end{methoddesc}

\begin{methoddesc}[window]{insstr}{\optional{y, x, } str \optional{, attr}}
Insert a character string (as many characters as will fit on the line)
before the character under the cursor.  All characters to the right of
the cursor are shifted right, with the rightmost characters on the
line being lost.  The cursor position does not change (after moving to
\var{y}, \var{x}, if specified). 
\end{methoddesc}

\begin{methoddesc}[window]{instr}{\optional{y, x} \optional{, n}}
Returns a string of characters, extracted from the window starting at
the current cursor position, or at \var{y}, \var{x} if specified.
Attributes are stripped from the characters.  If \var{n} is specified,
\method{instr()} returns return a string at most \var{n} characters
long (exclusive of the trailing NUL).
\end{methoddesc}

\begin{methoddesc}[window]{is_linetouched}{\var{line}}
Returns true if the specified line was modified since the last call to
\method{refresh()}; otherwise returns false.  Raises a
\exception{curses.error} exception if \var{line} is not valid
for the given window.
\end{methoddesc}

\begin{methoddesc}[window]{is_wintouched}{}
Returns true if the specified window was modified since the last call to
\method{refresh()}; otherwise returns false.
\end{methoddesc}

\begin{methoddesc}[window]{keypad}{yes}
If \var{yes} is 1, escape sequences generated by some keys (keypad, 
function keys) will be interpreted by \module{curses}.
If \var{yes} is 0, escape sequences will be left as is in the input
stream.
\end{methoddesc}

\begin{methoddesc}[window]{leaveok}{yes}
If \var{yes} is 1, cursor is left where it is on update, instead of
being at ``cursor position.''  This reduces cursor movement where
possible. If possible the cursor will be made invisible.

If \var{yes} is 0, cursor will always be at ``cursor position'' after
an update.
\end{methoddesc}

\begin{methoddesc}[window]{move}{new_y, new_x}
Move cursor to \code{(\var{new_y}, \var{new_x})}.
\end{methoddesc}

\begin{methoddesc}[window]{mvderwin}{y, x}
Moves the window inside its parent window.  The screen-relative
parameters of the window are not changed.  This routine is used to
display different parts of the parent window at the same physical
position on the screen.
\end{methoddesc}

\begin{methoddesc}[window]{mvwin}{new_y, new_x}
Move the window so its upper-left corner is at
\code{(\var{new_y}, \var{new_x})}.
\end{methoddesc}

\begin{methoddesc}[window]{nodelay}{yes}
If \var{yes} is \code{1}, \method{getch()} will be non-blocking.
\end{methoddesc}

\begin{methoddesc}[window]{notimeout}{yes}
If \var{yes} is \code{1}, escape sequences will not be timed out.

If \var{yes} is \code{0}, after a few milliseconds, an escape sequence
will not be interpreted, and will be left in the input stream as is.
\end{methoddesc}

\begin{methoddesc}[window]{noutrefresh}{}
Mark for refresh but wait.  This function updates the data structure
representing the desired state of the window, but does not force
an update of the physical screen.  To accomplish that, call 
\function{doupdate()}.
\end{methoddesc}

\begin{methoddesc}[window]{overlay}{destwin\optional{, sminrow, smincol,
                                    dminrow, dmincol, dmaxrow, dmaxcol}}
Overlay the window on top of \var{destwin}. The windows need not be
the same size, only the overlapping region is copied. This copy is
non-destructive, which means that the current background character
does not overwrite the old contents of \var{destwin}.

To get fine-grained control over the copied region, the second form
of \method{overlay()} can be used. \var{sminrow} and \var{smincol} are
the upper-left coordinates of the source window, and the other variables
mark a rectangle in the destination window.
\end{methoddesc}

\begin{methoddesc}[window]{overwrite}{destwin\optional{, sminrow, smincol,
                                      dminrow, dmincol, dmaxrow, dmaxcol}}
Overwrite the window on top of \var{destwin}. The windows need not be
the same size, in which case only the overlapping region is
copied. This copy is destructive, which means that the current
background character overwrites the old contents of \var{destwin}.

To get fine-grained control over the copied region, the second form
of \method{overwrite()} can be used. \var{sminrow} and \var{smincol} are
the upper-left coordinates of the source window, the other variables
mark a rectangle in the destination window.
\end{methoddesc}

\begin{methoddesc}[window]{putwin}{file}
Writes all data associated with the window into the provided file
object.  This information can be later retrieved using the
\function{getwin()} function.
\end{methoddesc}

\begin{methoddesc}[window]{redrawln}{beg, num}
Indicates that the \var{num} screen lines, starting at line \var{beg},
are corrupted and should be completely redrawn on the next
\method{refresh()} call.
\end{methoddesc}

\begin{methoddesc}[window]{redrawwin}{}
Touches the entire window, causing it to be completely redrawn on the
next \method{refresh()} call.
\end{methoddesc}

\begin{methoddesc}[window]{refresh}{\optional{pminrow, pmincol, sminrow,
                                    smincol, smaxrow, smaxcol}}
Update the display immediately (sync actual screen with previous
drawing/deleting methods).

The 6 optional arguments can only be specified when the window is a
pad created with \function{newpad()}.  The additional parameters are
needed to indicate what part of the pad and screen are involved.
\var{pminrow} and \var{pmincol} specify the upper left-hand corner of the
rectangle to be displayed in the pad.  \var{sminrow}, \var{smincol},
\var{smaxrow}, and \var{smaxcol} specify the edges of the rectangle to
be displayed on the screen.  The lower right-hand corner of the
rectangle to be displayed in the pad is calculated from the screen
coordinates, since the rectangles must be the same size.  Both
rectangles must be entirely contained within their respective
structures.  Negative values of \var{pminrow}, \var{pmincol},
\var{sminrow}, or \var{smincol} are treated as if they were zero.
\end{methoddesc}

\begin{methoddesc}[window]{scroll}{\optional{lines\code{ = 1}}}
Scroll the screen or scrolling region upward by \var{lines} lines.
\end{methoddesc}

\begin{methoddesc}[window]{scrollok}{flag}
Controls what happens when the cursor of a window is moved off the
edge of the window or scrolling region, either as a result of a
newline action on the bottom line, or typing the last character
of the last line.  If \var{flag} is false, the cursor is left
on the bottom line.  If \var{flag} is true, the window is
scrolled up one line.  Note that in order to get the physical
scrolling effect on the terminal, it is also necessary to call
\method{idlok()}.
\end{methoddesc}

\begin{methoddesc}[window]{setscrreg}{top, bottom}
Set the scrolling region from line \var{top} to line \var{bottom}. All
scrolling actions will take place in this region.
\end{methoddesc}

\begin{methoddesc}[window]{standend}{}
Turn off the standout attribute.  On some terminals this has the
side effect of turning off all attributes.
\end{methoddesc}

\begin{methoddesc}[window]{standout}{}
Turn on attribute \var{A_STANDOUT}.
\end{methoddesc}

\begin{methoddesc}[window]{subpad}{\optional{nlines, ncols,} begin_y, begin_x}
Return a sub-window, whose upper-left corner is at
\code{(\var{begin_y}, \var{begin_x})}, and whose width/height is
\var{ncols}/\var{nlines}.
\end{methoddesc}

\begin{methoddesc}[window]{subwin}{\optional{nlines, ncols,} begin_y, begin_x}
Return a sub-window, whose upper-left corner is at
\code{(\var{begin_y}, \var{begin_x})}, and whose width/height is
\var{ncols}/\var{nlines}.

By default, the sub-window will extend from the
specified position to the lower right corner of the window.
\end{methoddesc}

\begin{methoddesc}[window]{syncdown}{}
Touches each location in the window that has been touched in any of
its ancestor windows.  This routine is called by \method{refresh()},
so it should almost never be necessary to call it manually.
\end{methoddesc}

\begin{methoddesc}[window]{syncok}{flag}
If called with \var{flag} set to true, then \method{syncup()} is
called automatically whenever there is a change in the window.
\end{methoddesc}

\begin{methoddesc}[window]{syncup}{}
Touches all locations in ancestors of the window that have been changed in 
the window.  
\end{methoddesc}

\begin{methoddesc}[window]{timeout}{delay}
Sets blocking or non-blocking read behavior for the window.  If
\var{delay} is negative, blocking read is used (which will wait
indefinitely for input).  If \var{delay} is zero, then non-blocking
read is used, and -1 will be returned by \method{getch()} if no input
is waiting.  If \var{delay} is positive, then \method{getch()} will
block for \var{delay} milliseconds, and return -1 if there is still no
input at the end of that time.
\end{methoddesc}

\begin{methoddesc}[window]{touchline}{start, count}
Pretend \var{count} lines have been changed, starting with line
\var{start}.
\end{methoddesc}

\begin{methoddesc}[window]{touchwin}{}
Pretend the whole window has been changed, for purposes of drawing
optimizations.
\end{methoddesc}

\begin{methoddesc}[window]{untouchwin}{}
Marks all lines in  the  window  as unchanged since the last call to
\method{refresh()}. 
\end{methoddesc}

\begin{methoddesc}[window]{vline}{\optional{y, x,} ch, n}
Display a vertical line starting at \code{(\var{y}, \var{x})} with
length \var{n} consisting of the character \var{ch}.
\end{methoddesc}

\subsection{Constants}

The \module{curses} module defines the following data members:

\begin{datadesc}{ERR}
Some curses routines  that  return  an integer, such as 
\function{getch()}, return \constant{ERR} upon failure.  
\end{datadesc}

\begin{datadesc}{OK}
Some curses routines  that  return  an integer, such as 
\function{napms()}, return \constant{OK} upon success.  
\end{datadesc}

\begin{datadesc}{version}
A string representing the current version of the module. 
Also available as \constant{__version__}.
\end{datadesc}

Several constants are available to specify character cell attributes:

\begin{tableii}{l|l}{code}{Attribute}{Meaning}
  \lineii{A_ALTCHARSET}{Alternate character set mode.}
  \lineii{A_BLINK}{Blink mode.}
  \lineii{A_BOLD}{Bold mode.}
  \lineii{A_DIM}{Dim mode.}
  \lineii{A_NORMAL}{Normal attribute.}
  \lineii{A_STANDOUT}{Standout mode.}
  \lineii{A_UNDERLINE}{Underline mode.}
\end{tableii}

Keys are referred to by integer constants with names starting with 
\samp{KEY_}.   The exact keycaps available are system dependent.

% XXX this table is far too large!
% XXX should this table be alphabetized?

\begin{longtableii}{l|l}{code}{Key constant}{Key}
  \lineii{KEY_MIN}{Minimum key value}
  \lineii{KEY_BREAK}{ Break key (unreliable) }
  \lineii{KEY_DOWN}{ Down-arrow }
  \lineii{KEY_UP}{ Up-arrow }
  \lineii{KEY_LEFT}{ Left-arrow }
  \lineii{KEY_RIGHT}{ Right-arrow }
  \lineii{KEY_HOME}{ Home key (upward+left arrow) }
  \lineii{KEY_BACKSPACE}{ Backspace (unreliable) }
  \lineii{KEY_F0}{ Function keys.  Up to 64 function keys are supported. }
  \lineii{KEY_F\var{n}}{ Value of function key \var{n} }
  \lineii{KEY_DL}{ Delete line }
  \lineii{KEY_IL}{ Insert line }
  \lineii{KEY_DC}{ Delete character }
  \lineii{KEY_IC}{ Insert char or enter insert mode }
  \lineii{KEY_EIC}{ Exit insert char mode }
  \lineii{KEY_CLEAR}{ Clear screen }
  \lineii{KEY_EOS}{ Clear to end of screen }
  \lineii{KEY_EOL}{ Clear to end of line }
  \lineii{KEY_SF}{ Scroll 1 line forward }
  \lineii{KEY_SR}{ Scroll 1 line backward (reverse) }
  \lineii{KEY_NPAGE}{ Next page }
  \lineii{KEY_PPAGE}{ Previous page }
  \lineii{KEY_STAB}{ Set tab }
  \lineii{KEY_CTAB}{ Clear tab }
  \lineii{KEY_CATAB}{ Clear all tabs }
  \lineii{KEY_ENTER}{ Enter or send (unreliable) }
  \lineii{KEY_SRESET}{ Soft (partial) reset (unreliable) }
  \lineii{KEY_RESET}{ Reset or hard reset (unreliable) }
  \lineii{KEY_PRINT}{ Print }
  \lineii{KEY_LL}{ Home down or bottom (lower left) }
  \lineii{KEY_A1}{ Upper left of keypad }
  \lineii{KEY_A3}{ Upper right of keypad }
  \lineii{KEY_B2}{ Center of keypad }
  \lineii{KEY_C1}{ Lower left of keypad }
  \lineii{KEY_C3}{ Lower right of keypad }
  \lineii{KEY_BTAB}{ Back tab }
  \lineii{KEY_BEG}{ Beg (beginning) }
  \lineii{KEY_CANCEL}{ Cancel }
  \lineii{KEY_CLOSE}{ Close }
  \lineii{KEY_COMMAND}{ Cmd (command) }
  \lineii{KEY_COPY}{ Copy }
  \lineii{KEY_CREATE}{ Create }
  \lineii{KEY_END}{ End }
  \lineii{KEY_EXIT}{ Exit }
  \lineii{KEY_FIND}{ Find }
  \lineii{KEY_HELP}{ Help }
  \lineii{KEY_MARK}{ Mark }
  \lineii{KEY_MESSAGE}{ Message }
  \lineii{KEY_MOVE}{ Move }
  \lineii{KEY_NEXT}{ Next }
  \lineii{KEY_OPEN}{ Open }
  \lineii{KEY_OPTIONS}{ Options }
  \lineii{KEY_PREVIOUS}{ Prev (previous) }
  \lineii{KEY_REDO}{ Redo }
  \lineii{KEY_REFERENCE}{ Ref (reference) }
  \lineii{KEY_REFRESH}{ Refresh }
  \lineii{KEY_REPLACE}{ Replace }
  \lineii{KEY_RESTART}{ Restart }
  \lineii{KEY_RESUME}{ Resume }
  \lineii{KEY_SAVE}{ Save }
  \lineii{KEY_SBEG}{ Shifted Beg (beginning) }
  \lineii{KEY_SCANCEL}{ Shifted Cancel }
  \lineii{KEY_SCOMMAND}{ Shifted Command }
  \lineii{KEY_SCOPY}{ Shifted Copy }
  \lineii{KEY_SCREATE}{ Shifted Create }
  \lineii{KEY_SDC}{ Shifted Delete char }
  \lineii{KEY_SDL}{ Shifted Delete line }
  \lineii{KEY_SELECT}{ Select }
  \lineii{KEY_SEND}{ Shifted End }
  \lineii{KEY_SEOL}{ Shifted Clear line }
  \lineii{KEY_SEXIT}{ Shifted Dxit }
  \lineii{KEY_SFIND}{ Shifted Find }
  \lineii{KEY_SHELP}{ Shifted Help }
  \lineii{KEY_SHOME}{ Shifted Home }
  \lineii{KEY_SIC}{ Shifted Input }
  \lineii{KEY_SLEFT}{ Shifted Left arrow }
  \lineii{KEY_SMESSAGE}{ Shifted Message }
  \lineii{KEY_SMOVE}{ Shifted Move }
  \lineii{KEY_SNEXT}{ Shifted Next }
  \lineii{KEY_SOPTIONS}{ Shifted Options }
  \lineii{KEY_SPREVIOUS}{ Shifted Prev }
  \lineii{KEY_SPRINT}{ Shifted Print }
  \lineii{KEY_SREDO}{ Shifted Redo }
  \lineii{KEY_SREPLACE}{ Shifted Replace }
  \lineii{KEY_SRIGHT}{ Shifted Right arrow }
  \lineii{KEY_SRSUME}{ Shifted Resume }
  \lineii{KEY_SSAVE}{ Shifted Save }
  \lineii{KEY_SSUSPEND}{ Shifted Suspend }
  \lineii{KEY_SUNDO}{ Shifted Undo }
  \lineii{KEY_SUSPEND}{ Suspend }
  \lineii{KEY_UNDO}{ Undo }
  \lineii{KEY_MOUSE}{ Mouse event has occurred }
  \lineii{KEY_RESIZE}{ Terminal resize event }
  \lineii{KEY_MAX}{Maximum key value}
\end{longtableii}

On VT100s and their software emulations, such as X terminal emulators,
there are normally at least four function keys (\constant{KEY_F1},
\constant{KEY_F2}, \constant{KEY_F3}, \constant{KEY_F4}) available,
and the arrow keys mapped to \constant{KEY_UP}, \constant{KEY_DOWN},
\constant{KEY_LEFT} and \constant{KEY_RIGHT} in the obvious way.  If
your machine has a PC keybboard, it is safe to expect arrow keys and
twelve function keys (older PC keyboards may have only ten function
keys); also, the following keypad mappings are standard:

\begin{tableii}{l|l}{kbd}{Keycap}{Constant}
   \lineii{Insert}{KEY_IC}
   \lineii{Delete}{KEY_DC}
   \lineii{Home}{KEY_HOME}
   \lineii{End}{KEY_END}
   \lineii{Page Up}{KEY_NPAGE}
   \lineii{Page Down}{KEY_PPAGE}
\end{tableii}

The following table lists characters from the alternate character set.
These are inherited from the VT100 terminal, and will generally be 
available on software emulations such as X terminals.  When there
is no graphic available, curses falls back on a crude printable ASCII
approximation.
\note{These are available only after \function{initscr()} has 
been called.}

\begin{longtableii}{l|l}{code}{ACS code}{Meaning}
  \lineii{ACS_BBSS}{alternate name for upper right corner}
  \lineii{ACS_BLOCK}{solid square block}
  \lineii{ACS_BOARD}{board of squares}
  \lineii{ACS_BSBS}{alternate name for horizontal line}
  \lineii{ACS_BSSB}{alternate name for upper left corner}
  \lineii{ACS_BSSS}{alternate name for top tee}
  \lineii{ACS_BTEE}{bottom tee}
  \lineii{ACS_BULLET}{bullet}
  \lineii{ACS_CKBOARD}{checker board (stipple)}
  \lineii{ACS_DARROW}{arrow pointing down}
  \lineii{ACS_DEGREE}{degree symbol}
  \lineii{ACS_DIAMOND}{diamond}
  \lineii{ACS_GEQUAL}{greater-than-or-equal-to}
  \lineii{ACS_HLINE}{horizontal line}
  \lineii{ACS_LANTERN}{lantern symbol}
  \lineii{ACS_LARROW}{left arrow}
  \lineii{ACS_LEQUAL}{less-than-or-equal-to}
  \lineii{ACS_LLCORNER}{lower left-hand corner}
  \lineii{ACS_LRCORNER}{lower right-hand corner}
  \lineii{ACS_LTEE}{left tee}
  \lineii{ACS_NEQUAL}{not-equal sign}
  \lineii{ACS_PI}{letter pi}
  \lineii{ACS_PLMINUS}{plus-or-minus sign}
  \lineii{ACS_PLUS}{big plus sign}
  \lineii{ACS_RARROW}{right arrow}
  \lineii{ACS_RTEE}{right tee}
  \lineii{ACS_S1}{scan line 1}
  \lineii{ACS_S3}{scan line 3}
  \lineii{ACS_S7}{scan line 7}
  \lineii{ACS_S9}{scan line 9}
  \lineii{ACS_SBBS}{alternate name for lower right corner}
  \lineii{ACS_SBSB}{alternate name for vertical line}
  \lineii{ACS_SBSS}{alternate name for right tee}
  \lineii{ACS_SSBB}{alternate name for lower left corner}
  \lineii{ACS_SSBS}{alternate name for bottom tee}
  \lineii{ACS_SSSB}{alternate name for left tee}
  \lineii{ACS_SSSS}{alternate name for crossover or big plus}
  \lineii{ACS_STERLING}{pound sterling}
  \lineii{ACS_TTEE}{top tee}
  \lineii{ACS_UARROW}{up arrow}
  \lineii{ACS_ULCORNER}{upper left corner}
  \lineii{ACS_URCORNER}{upper right corner}
  \lineii{ACS_VLINE}{vertical line}
\end{longtableii}

The following table lists the predefined colors:

\begin{tableii}{l|l}{code}{Constant}{Color}
  \lineii{COLOR_BLACK}{Black}
  \lineii{COLOR_BLUE}{Blue}
  \lineii{COLOR_CYAN}{Cyan (light greenish blue)}
  \lineii{COLOR_GREEN}{Green}
  \lineii{COLOR_MAGENTA}{Magenta (purplish red)}
  \lineii{COLOR_RED}{Red}
  \lineii{COLOR_WHITE}{White}
  \lineii{COLOR_YELLOW}{Yellow}
\end{tableii}

\section{\module{curses.textpad} ---
         Text input widget for curses programs}

\declaremodule{standard}{curses.textpad}
\sectionauthor{Eric Raymond}{esr@thyrsus.com}
\moduleauthor{Eric Raymond}{esr@thyrsus.com}
\modulesynopsis{Emacs-like input editing in a curses window.}
\versionadded{1.6}

The \module{curses.textpad} module provides a \class{Textbox} class
that handles elementary text editing in a curses window, supporting a
set of keybindings resembling those of Emacs (thus, also of Netscape
Navigator, BBedit 6.x, FrameMaker, and many other programs).  The
module also provides a rectangle-drawing function useful for framing
text boxes or for other purposes.

The module \module{curses.textpad} defines the following function:

\begin{funcdesc}{rectangle}{win, uly, ulx, lry, lrx}
Draw a rectangle.  The first argument must be a window object; the
remaining arguments are coordinates relative to that window.  The
second and third arguments are the y and x coordinates of the upper
left hand corner of the rectangle To be drawn; the fourth and fifth
arguments are the y and x coordinates of the lower right hand corner.
The rectangle will be drawn using VT100/IBM PC forms characters on
terminals that make this possible (including xterm and most other
software terminal emulators).  Otherwise it will be drawn with ASCII 
dashes, vertical bars, and plus signs.
\end{funcdesc}


\subsection{Textbox objects \label{curses-textpad-objects}}

You can instantiate a \class{Textbox} object as follows:

\begin{classdesc}{Textbox}{win}
Return a textbox widget object.  The \var{win} argument should be a
curses \class{WindowObject} in which the textbox is to be contained.
The edit cursor of the textbox is initially located at the upper left
hand corner of the containin window, with coordinates \code{(0, 0)}.
The instance's \member{stripspaces} flag is initially on.
\end{classdesc}

\class{Textbox} objects have the following methods:

\begin{methoddesc}{edit}{\optional{validator}}
This is the entry point you will normally use.  It accepts editing
keystrokes until one of the termination keystrokes is entered.  If
\var{validator} is supplied, it must be a function.  It will be called
for each keystroke entered with the keystroke as a parameter; command
dispatch is done on the result. This method returns the window
contents as a string; whether blanks in the window are included is
affected by the \member{stripspaces} member.
\end{methoddesc}

\begin{methoddesc}{do_command}{ch}
Process a single command keystroke.  Here are the supported special
keystrokes: 

\begin{tableii}{l|l}{kbd}{Keystroke}{Action}
  \lineii{Control-A}{Go to left edge of window.}
  \lineii{Control-B}{Cursor left, wrapping to previous line if appropriate.}
  \lineii{Control-D}{Delete character under cursor.}
  \lineii{Control-E}{Go to right edge (stripspaces off) or end of line
                  (stripspaces on).}
  \lineii{Control-F}{Cursor right, wrapping to next line when appropriate.}
  \lineii{Control-G}{Terminate, returning the window contents.}
  \lineii{Control-H}{Delete character backward.}
  \lineii{Control-J}{Terminate if the window is 1 line, otherwise
                     insert newline.}
  \lineii{Control-K}{If line is blank, delete it, otherwise clear to
                     end of line.}
  \lineii{Control-L}{Refresh screen.}
  \lineii{Control-N}{Cursor down; move down one line.}
  \lineii{Control-O}{Insert a blank line at cursor location.}
  \lineii{Control-P}{Cursor up; move up one line.}
\end{tableii}

Move operations do nothing if the cursor is at an edge where the
movement is not possible.  The following synonyms are supported where
possible:

\begin{tableii}{l|l}{constant}{Constant}{Keystroke}
  \lineii{KEY_LEFT}{\kbd{Control-B}}
  \lineii{KEY_RIGHT}{\kbd{Control-F}}
  \lineii{KEY_UP}{\kbd{Control-P}}
  \lineii{KEY_DOWN}{\kbd{Control-N}}
  \lineii{KEY_BACKSPACE}{\kbd{Control-h}}
\end{tableii}

All other keystrokes are treated as a command to insert the given
character and move right (with line wrapping).
\end{methoddesc}

\begin{methoddesc}{gather}{}
This method returns the window contents as a string; whether blanks in
the window are included is affected by the \member{stripspaces}
member.
\end{methoddesc}

\begin{memberdesc}{stripspaces}
This data member is a flag which controls the interpretation of blanks in
the window.  When it is on, trailing blanks on each line are ignored;
any cursor motion that would land the cursor on a trailing blank goes
to the end of that line instead, and trailing blanks are stripped when
the window contents is gathered.
\end{memberdesc}


\section{\module{curses.wrapper} ---
         Terminal handler for curses programs}

\declaremodule{standard}{curses.wrapper}
\sectionauthor{Eric Raymond}{esr@thyrsus.com}
\moduleauthor{Eric Raymond}{esr@thyrsus.com}
\modulesynopsis{Terminal configuration wrapper for curses programs.}
\versionadded{1.6}

This module supplies one function, \function{wrapper()}, which runs
another function which should be the rest of your curses-using
application.  If the application raises an exception,
\function{wrapper()} will restore the terminal to a sane state before
passing it further up the stack and generating a traceback.

\begin{funcdesc}{wrapper}{func, \moreargs}
Wrapper function that initializes curses and calls another function,
\var{func}, restoring normal keyboard/screen behavior on error.
The callable object \var{func} is then passed the main window 'stdscr'
as its first argument, followed by any other arguments passed to
\function{wrapper()}.
\end{funcdesc}

Before calling the hook function, \function{wrapper()} turns on cbreak
mode, turns off echo, enables the terminal keypad, and initializes
colors if the terminal has color support.  On exit (whether normally
or by exception) it restores cooked mode, turns on echo, and disables
the terminal keypad.


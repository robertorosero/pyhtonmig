\documentclass{manual}

% NOTE: this file controls which chapters/sections of the library
% manual are actually printed.  It is easy to customize your manual
% by commenting out sections that you're not interested in.

\title{Python Library Reference}

\author{
	Guido van Rossum \\
	Dept. AA, CWI, P.O. Box 94079 \\
	1090 GB Amsterdam, The Netherlands \\
	E-mail: {\tt guido@cwi.nl}
}

\date{17 March 1995 \\ Release 1.2-proof-2} % XXX update before release!


\makeindex			% tell \index to actually write the
				% .idx file
\makemodindex			% ... and the module index as well.


\begin{document}

\maketitle

\strong{BEOPEN.COM TERMS AND CONDITIONS FOR PYTHON 2.0}

\centerline{\strong{BEOPEN PYTHON OPEN SOURCE LICENSE AGREEMENT VERSION 1}}

\begin{enumerate}

\item
This LICENSE AGREEMENT is between BeOpen.com (``BeOpen''), having an
office at 160 Saratoga Avenue, Santa Clara, CA 95051, and the
Individual or Organization (``Licensee'') accessing and otherwise
using this software in source or binary form and its associated
documentation (``the Software'').

\item
Subject to the terms and conditions of this BeOpen Python License
Agreement, BeOpen hereby grants Licensee a non-exclusive,
royalty-free, world-wide license to reproduce, analyze, test, perform
and/or display publicly, prepare derivative works, distribute, and
otherwise use the Software alone or in any derivative version,
provided, however, that the BeOpen Python License is retained in the
Software, alone or in any derivative version prepared by Licensee.

\item
BeOpen is making the Software available to Licensee on an ``AS IS''
basis.  BEOPEN MAKES NO REPRESENTATIONS OR WARRANTIES, EXPRESS OR
IMPLIED.  BY WAY OF EXAMPLE, BUT NOT LIMITATION, BEOPEN MAKES NO AND
DISCLAIMS ANY REPRESENTATION OR WARRANTY OF MERCHANTABILITY OR FITNESS
FOR ANY PARTICULAR PURPOSE OR THAT THE USE OF THE SOFTWARE WILL NOT
INFRINGE ANY THIRD PARTY RIGHTS.

\item
BEOPEN SHALL NOT BE LIABLE TO LICENSEE OR ANY OTHER USERS OF THE
SOFTWARE FOR ANY INCIDENTAL, SPECIAL, OR CONSEQUENTIAL DAMAGES OR LOSS
AS A RESULT OF USING, MODIFYING OR DISTRIBUTING THE SOFTWARE, OR ANY
DERIVATIVE THEREOF, EVEN IF ADVISED OF THE POSSIBILITY THEREOF.

\item
This License Agreement will automatically terminate upon a material
breach of its terms and conditions.

\item
This License Agreement shall be governed by and interpreted in all
respects by the law of the State of California, excluding conflict of
law provisions.  Nothing in this License Agreement shall be deemed to
create any relationship of agency, partnership, or joint venture
between BeOpen and Licensee.  This License Agreement does not grant
permission to use BeOpen trademarks or trade names in a trademark
sense to endorse or promote products or services of Licensee, or any
third party.  As an exception, the ``BeOpen Python'' logos available
at http://www.pythonlabs.com/logos.html may be used according to the
permissions granted on that web page.

\item
By copying, installing or otherwise using the software, Licensee
agrees to be bound by the terms and conditions of this License
Agreement.
\end{enumerate}


\centerline{\strong{CNRI OPEN SOURCE LICENSE AGREEMENT}}

Python 1.6 is made available subject to the terms and conditions in
CNRI's License Agreement.  This Agreement together with Python 1.6 may
be located on the Internet using the following unique, persistent
identifier (known as a handle): 1895.22/1012.  This Agreement may also
be obtained from a proxy server on the Internet using the following
URL: \url{http://hdl.handle.net/1895.22/1012}.


\centerline{\strong{CWI PERMISSIONS STATEMENT AND DISCLAIMER}}

Copyright \copyright{} 1991 - 1995, Stichting Mathematisch Centrum
Amsterdam, The Netherlands.  All rights reserved.

Permission to use, copy, modify, and distribute this software and its
documentation for any purpose and without fee is hereby granted,
provided that the above copyright notice appear in all copies and that
both that copyright notice and this permission notice appear in
supporting documentation, and that the name of Stichting Mathematisch
Centrum or CWI not be used in advertising or publicity pertaining to
distribution of the software without specific, written prior
permission.

STICHTING MATHEMATISCH CENTRUM DISCLAIMS ALL WARRANTIES WITH REGARD TO
THIS SOFTWARE, INCLUDING ALL IMPLIED WARRANTIES OF MERCHANTABILITY AND
FITNESS, IN NO EVENT SHALL STICHTING MATHEMATISCH CENTRUM BE LIABLE
FOR ANY SPECIAL, INDIRECT OR CONSEQUENTIAL DAMAGES OR ANY DAMAGES
WHATSOEVER RESULTING FROM LOSS OF USE, DATA OR PROFITS, WHETHER IN AN
ACTION OF CONTRACT, NEGLIGENCE OR OTHER TORTIOUS ACTION, ARISING OUT
OF OR IN CONNECTION WITH THE USE OR PERFORMANCE OF THIS SOFTWARE.


\begin{abstract}

\noindent
Python is an extensible, interpreted, object-oriented programming
language.  It supports a wide range of applications, from simple text
processing scripts to interactive WWW browsers.

While the \emph{Python Reference Manual} describes the exact syntax and
semantics of the language, it does not describe the standard library
that is distributed with the language, and which greatly enhances its
immediate usability.  This library contains built-in modules (written
in C) that provide access to system functionality such as file I/O
that would otherwise be inaccessible to Python programmers, as well as
modules written in Python that provide standardized solutions for many
problems that occur in everyday programming.  Some of these modules
are explicitly designed to encourage and enhance the portability of
Python programs.

This library reference manual documents Python's standard library, as
well as many optional library modules (which may or may not be
available, depending on whether the underlying platform supports them
and on the configuration choices made at compile time).  It also
documents the standard types of the language and its built-in
functions and exceptions, many of which are not or incompletely
documented in the Reference Manual.

This manual assumes basic knowledge about the Python language.  For an
informal introduction to Python, see the \emph{Python Tutorial}; the
\emph{Python Reference Manual} remains the highest authority on
syntactic and semantic questions.  Finally, the manual entitled
\emph{Extending and Embedding the Python Interpreter} describes how to
add new extensions to Python and how to embed it in other applications.

\end{abstract}

\tableofcontents

				% Chapter title:

\chapter{Introduction}

The Python library consists of three parts, with different levels of
integration with the interpreter.
Closest to the interpreter are built-in types, exceptions and functions.
Next are built-in modules, which are written in \C{} and linked statically
with the interpreter.
Finally there are standard modules that are implemented entirely in
Python, but are always available.
For efficiency, some standard modules may become built-in modules in
future versions of the interpreter.
\indexii{built-in}{types}
\indexii{built-in}{exceptions}
\indexii{built-in}{functions}
\indexii{built-in}{modules}
\indexii{standard}{modules}
\indexii{\C{}}{language}
		% Introduction

\chapter{Built-in Types, Exceptions and Functions}
\nodename{Built-in Objects}
\label{builtin}

Names for built-in exceptions and functions are found in a separate
symbol table.  This table is searched last when the interpreter looks
up the meaning of a name, so local and global
user-defined names can override built-in names.  Built-in types are
described together here for easy reference.\footnote{
	Most descriptions sorely lack explanations of the exceptions
	that may be raised --- this will be fixed in a future version of
	this manual.}
\indexii{built-in}{types}
\indexii{built-in}{exceptions}
\indexii{built-in}{functions}
\index{symbol table}

The tables in this chapter document the priorities of operators by
listing them in order of ascending priority (within a table) and
grouping operators that have the same priority in the same box.
Binary operators of the same priority group from left to right.
(Unary operators group from right to left, but there you have no real
choice.)  See chapter 5 of the \citetitle[../ref/ref.html]{Python
Reference Manual} for the complete picture on operator priorities.
			% Built-in Types, Exceptions and Functions
\section{Built-in Types}

The following sections describe the standard types that are built into
the interpreter.  These are the numeric types, sequence types, and
several others, including types themselves.  There is no explicit
Boolean type; use integers instead.
\indexii{built-in}{types}
\indexii{Boolean}{type}

Some operations are supported by several object types; in particular,
all objects can be compared, tested for truth value, and converted to
a string (with the \code{`{\rm \ldots}`} notation).  The latter conversion is
implicitly used when an object is written by the \code{print} statement.
\stindex{print}

\subsection{Truth Value Testing}

Any object can be tested for truth value, for use in an \code{if} or
\code{while} condition or as operand of the Boolean operations below.
The following values are false:
\stindex{if}
\stindex{while}
\indexii{truth}{value}
\indexii{Boolean}{operations}
\index{false}

\begin{itemize}
\renewcommand{\indexsubitem}{(Built-in object)}

\item	\code{None}
	\ttindex{None}

\item	zero of any numeric type, e.g., \code{0}, \code{0L}, \code{0.0}.

\item	any empty sequence, e.g., \code{''}, \code{()}, \code{[]}.

\item	any empty mapping, e.g., \code{\{\}}.

\end{itemize}

\emph{All} other values are true --- so objects of many types are
always true.
\index{true}

\subsection{Boolean Operations}

These are the Boolean operations:
\indexii{Boolean}{operations}

\begin{tableiii}{|c|l|c|}{code}{Operation}{Result}{Notes}
  \lineiii{\var{x} or \var{y}}{if \var{x} is false, then \var{y}, else \var{x}}{(1)}
  \lineiii{\var{x} and \var{y}}{if \var{x} is false, then \var{x}, else \var{y}}{(1)}
  \lineiii{not \var{x}}{if \var{x} is false, then \code{1}, else \code{0}}{}
\end{tableiii}
\opindex{and}
\opindex{or}
\opindex{not}

\noindent
Notes:

\begin{description}

\item[(1)]
These only evaluate their second argument if needed for their outcome.

\end{description}

\subsection{Comparisons}

Comparison operations are supported by all objects:

\begin{tableiii}{|c|l|c|}{code}{Operation}{Meaning}{Notes}
  \lineiii{<}{strictly less than}{}
  \lineiii{<=}{less than or equal}{}
  \lineiii{>}{strictly greater than}{}
  \lineiii{>=}{greater than or equal}{}
  \lineiii{==}{equal}{}
  \lineiii{<>}{not equal}{(1)}
  \lineiii{!=}{not equal}{(1)}
  \lineiii{is}{object identity}{}
  \lineiii{is not}{negated object identity}{}
\end{tableiii}
\indexii{operator}{comparison}
\opindex{==} % XXX *All* others have funny characters < ! >
\opindex{is}
\opindex{is not}

\noindent
Notes:

\begin{description}

\item[(1)]
\code{<>} and \code{!=} are alternate spellings for the same operator.
(I couldn't choose between \ABC{} and \C{}! :-)
\indexii{\ABC{}}{language}
\indexii{\C{}}{language}

\end{description}

Objects of different types, except different numeric types, never
compare equal; such objects are ordered consistently but arbitrarily
(so that sorting a heterogeneous array yields a consistent result).
Furthermore, some types (e.g., windows) support only a degenerate
notion of comparison where any two objects of that type are unequal.
Again, such objects are ordered arbitrarily but consistently.
\indexii{types}{numeric}
\indexii{objects}{comparing}

(Implementation note: objects of different types except numbers are
ordered by their type names; objects of the same types that don't
support proper comparison are ordered by their address.)

Two more operations with the same syntactic priority, \code{in} and
\code{not in}, are supported only by sequence types (below).
\opindex{in}
\opindex{not in}

\subsection{Numeric Types}

There are three numeric types: \dfn{plain integers}, \dfn{long integers}, and
\dfn{floating point numbers}.  Plain integers (also just called \dfn{integers})
are implemented using \code{long} in \C{}, which gives them at least 32
bits of precision.  Long integers have unlimited precision.  Floating
point numbers are implemented using \code{double} in \C{}.  All bets on
their precision are off unless you happen to know the machine you are
working with.
\indexii{numeric}{types}
\indexii{integer}{types}
\indexii{integer}{type}
\indexiii{long}{integer}{type}
\indexii{floating point}{type}
\indexii{\C{}}{language}

Numbers are created by numeric literals or as the result of built-in
functions and operators.  Unadorned integer literals (including hex
and octal numbers) yield plain integers.  Integer literals with an \samp{L}
or \samp{l} suffix yield long integers
(\samp{L} is preferred because \code{1l} looks too much like eleven!).
Numeric literals containing a decimal point or an exponent sign yield
floating point numbers.
\indexii{numeric}{literals}
\indexii{integer}{literals}
\indexiii{long}{integer}{literals}
\indexii{floating point}{literals}
\indexii{hexadecimal}{literals}
\indexii{octal}{literals}

Python fully supports mixed arithmetic: when a binary arithmetic
operator has operands of different numeric types, the operand with the
``smaller'' type is converted to that of the other, where plain
integer is smaller than long integer is smaller than floating point.
Comparisons between numbers of mixed type use the same rule.%
\footnote{As a consequence, the list \code{[1, 2]} is considered equal
	to \code{[1.0, 2.0]}, and similar for tuples.}
The functions \code{int()}, \code{long()} and \code{float()} can be used
to coerce numbers to a specific type.
\index{arithmetic}
\bifuncindex{int}
\bifuncindex{long}
\bifuncindex{float}

All numeric types support the following operations:

\begin{tableiii}{|c|l|c|}{code}{Operation}{Result}{Notes}
  \lineiii{abs(\var{x})}{absolute value of \var{x}}{}
  \lineiii{int(\var{x})}{\var{x} converted to integer}{(1)}
  \lineiii{long(\var{x})}{\var{x} converted to long integer}{(1)}
  \lineiii{float(\var{x})}{\var{x} converted to floating point}{}
  \lineiii{-\var{x}}{\var{x} negated}{}
  \lineiii{+\var{x}}{\var{x} unchanged}{}
  \lineiii{\var{x} + \var{y}}{sum of \var{x} and \var{y}}{}
  \lineiii{\var{x} - \var{y}}{difference of \var{x} and \var{y}}{}
  \lineiii{\var{x} * \var{y}}{product of \var{x} and \var{y}}{}
  \lineiii{\var{x} / \var{y}}{quotient of \var{x} and \var{y}}{(2)}
  \lineiii{\var{x} \%{} \var{y}}{remainder of \code{\var{x} / \var{y}}}{}
  \lineiii{divmod(\var{x}, \var{y})}{the pair \code{(\var{x} / \var{y}, \var{x} \%{} \var{y})}}{(3)}
  \lineiii{pow(\var{x}, \var{y})}{\var{x} to the power \var{y}}{}
\end{tableiii}
\indexiii{operations on}{numeric}{types}

\noindent
Notes:
\begin{description}
\item[(1)]
Conversion from floating point to (long or plain) integer may round or
% XXXJH xref here
truncate as in \C{}; see functions \code{floor} and \code{ceil} in module
\code{math} for well-defined conversions.
\indexii{numeric}{conversions}
\ttindex{math}
\indexii{\C{}}{language}

\item[(2)]
For (plain or long) integer division, the result is an integer; it
always truncates towards zero.
% XXXJH integer division is better defined nowadays
\indexii{integer}{division}
\indexiii{long}{integer}{division}

\item[(3)]
See the section on built-in functions for an exact definition.

\end{description}
% XXXJH exceptions: overflow (when? what operations?) zerodivision

\subsubsection{Bit-string Operations on Integer Types.}

Plain and long integer types support additional operations that make
sense only for bit-strings.  Negative numbers are treated as their 2's
complement value:

\begin{tableiii}{|c|l|c|}{code}{Operation}{Result}{Notes}
  \lineiii{\~\var{x}}{the bits of \var{x} inverted}{}
  \lineiii{\var{x} \^{} \var{y}}{bitwise \dfn{exclusive or} of \var{x} and \var{y}}{}
  \lineiii{\var{x} \&{} \var{y}}{bitwise \dfn{and} of \var{x} and \var{y}}{}
  \lineiii{\var{x} | \var{y}}{bitwise \dfn{or} of \var{x} and \var{y}}{}
  \lineiii{\var{x} << \var{n}}{\var{x} shifted left by \var{n} bits}{}
  \lineiii{\var{x} >> \var{n}}{\var{x} shifted right by \var{n} bits}{}
\end{tableiii}
% XXXJH what's `left'? `right'? maybe better use lsb or msb or something
\indexiii{operations on}{integer}{types}
\indexii{bit-string}{operations}
\indexii{shifting}{operations}
\indexii{masking}{operations}

\subsection{Sequence Types}

There are three sequence types: strings, lists and tuples.
Strings literals are written in single quotes: \code{'xyzzy'}.
Lists are constructed with square brackets,
separating items with commas:
\code{[a, b, c]}.
Tuples are constructed by the comma operator
(not within square brackets), with or without enclosing parentheses,
but an empty tuple must have the enclosing parentheses, e.g.,
\code{a, b, c} or \code{()}.  A single item tuple must have a trailing comma,
e.g., \code{(d,)}.
\indexii{sequence}{types}
\indexii{string}{type}
\indexii{tuple}{type}
\indexii{list}{type}

Sequence types support the following operations (\var{s} and \var{t} are
sequences of the same type; \var{n}, \var{i} and \var{j} are integers):

\begin{tableiii}{|c|l|c|}{code}{Operation}{Result}{Notes}
  \lineiii{len(\var{s})}{length of \var{s}}{}
  \lineiii{min(\var{s})}{smallest item of \var{s}}{}
  \lineiii{max(\var{s})}{largest item of \var{s}}{}
  \lineiii{\var{x} in \var{s}}{\code{1} if an item of \var{s} is equal to \var{x}, else \code{0}}{}
  \lineiii{\var{x} not in \var{s}}{\code{0} if an item of \var{s} is equal to \var{x}, else \code{1}}{}
  \lineiii{\var{s} + \var{t}}{the concatenation of \var{s} and \var{t}}{}
  \lineiii{\var{s} * \var{n}{\rm ,} \var{n} * \var{s}}{\var{n} copies of \var{s} concatenated}{}
  \lineiii{\var{s}[\var{i}]}{\var{i}'th item of \var{s}, origin 0}{(1)}
  \lineiii{\var{s}[\var{i}:\var{j}]}{slice of \var{s} from \var{i} to \var{j}}{(1), (2)}
\end{tableiii}
\indexiii{operations on}{sequence}{types}
\bifuncindex{len}
\bifuncindex{min}
\bifuncindex{max}
\indexii{concatenation}{operation}
\indexii{repetition}{operation}
\indexii{subscript}{operation}
\indexii{slice}{operation}
\opindex{in}
\opindex{not in}

\noindent
Notes:

% XXXJH all TeX-math expressions replaced by python-syntax expressions
\begin{description}
  
\item[(1)] If \var{i} or \var{j} is negative, the index is relative to
  the end of the string, i.e., \code{len(\var{s}) + \var{i}} or
  \code{len(\var{s}) + \var{j}} is substituted.  But note that \code{-0} is
  still \code{0}.
  
\item[(2)] The slice of \var{s} from \var{i} to \var{j} is defined as
  the sequence of items with index \var{k} such that \code{\var{i} <=
  \var{k} < \var{j}}.  If \var{i} or \var{j} is greater than
  \code{len(\var{s})}, use \code{len(\var{s})}.  If \var{i} is omitted,
  use \code{0}.  If \var{j} is omitted, use \code{len(\var{s})}.  If
  \var{i} is greater than or equal to \var{j}, the slice is empty.

\end{description}

\subsubsection{More String Operations.}

String objects have one unique built-in operation: the \code{\%}
operator (modulo) with a string left argument interprets this string
as a C sprintf format string to be applied to the right argument, and
returns the string resulting from this formatting operation.

The right argument should be a tuple with one item for each argument
required by the format string; if the string requires a single
argument, the right argument may also be a single non-tuple object.%
\footnote{A tuple object in this case should be a singleton.}
The following format characters are understood:
\%, c, s, i, d, u, o, x, X, e, E, f, g, G.
Width and precision may be a * to specify that an integer argument
specifies the actual width or precision.  The flag characters -, +,
blank, \# and 0 are understood.  The size specifiers h, l or L may be
present but are ignored.  The \code{\%s} conversion takes any Python
object and converts it to a string using \code{str()} before
formatting it.  The ANSI features \code{\%p} and \code{\%n}
are not supported.  Since Python strings have an explicit length,
\code{\%s} conversions don't assume that \code{'\\0'} is the end of
the string.

For safety reasons, floating point precisions are clipped to 50;
\code{\%f} conversions for numbers whose absolute value is over 1e25
are replaced by \code{\%g} conversions.%
\footnote{These numbers are fairly arbitrary.  They are intended to
avoid printing endless strings of meaningless digits without hampering
correct use and without having to know the exact precision of floating
point values on a particular machine.}
All other errors raise exceptions.

If the right argument is a dictionary (or any kind of mapping), then
the formats in the string must have a parenthesized key into that
dictionary inserted immediately after the \code{\%} character, and
each format formats the corresponding entry from the mapping.  E.g.
\begin{verbatim}
    >>> count = 2
    >>> language = 'Python'
    >>> print '%(language)s has %(count)03d quote types.' % vars()
    Python has 002 quote types.
    >>> 
\end{verbatim}
In this case no * specifiers may occur in a format.

Additional string operations are defined in standard module
\code{string} and in built-in module \code{regex}.
\index{string}
\index{regex}

\subsubsection{Mutable Sequence Types.}

List objects support additional operations that allow in-place
modification of the object.
These operations would be supported by other mutable sequence types
(when added to the language) as well.
Strings and tuples are immutable sequence types and such objects cannot
be modified once created.
The following operations are defined on mutable sequence types (where
\var{x} is an arbitrary object):
\indexiii{mutable}{sequence}{types}
\indexii{list}{type}

\begin{tableiii}{|c|l|c|}{code}{Operation}{Result}{Notes}
  \lineiii{\var{s}[\var{i}] = \var{x}}
	{item \var{i} of \var{s} is replaced by \var{x}}{}
  \lineiii{\var{s}[\var{i}:\var{j}] = \var{t}}
  	{slice of \var{s} from \var{i} to \var{j} is replaced by \var{t}}{}
  \lineiii{del \var{s}[\var{i}:\var{j}]}
	{same as \code{\var{s}[\var{i}:\var{j}] = []}}{}
  \lineiii{\var{s}.append(\var{x})}
	{same as \code{\var{s}[len(\var{s}):len(\var{s})] = [\var{x}]}}{}
  \lineiii{\var{s}.count(\var{x})}
	{return number of \var{i}'s for which \code{\var{s}[\var{i}] == \var{x}}}{}
  \lineiii{\var{s}.index(\var{x})}
	{return smallest \var{i} such that \code{\var{s}[\var{i}] == \var{x}}}{(1)}
  \lineiii{\var{s}.insert(\var{i}, \var{x})}
	{same as \code{\var{s}[\var{i}:\var{i}] = [\var{x}]}}{}
  \lineiii{\var{s}.remove(\var{x})}
	{same as \code{del \var{s}[\var{s}.index(\var{x})]}}{(1)}
  \lineiii{\var{s}.reverse()}
	{reverses the items of \var{s} in place}{}
  \lineiii{\var{s}.sort()}
	{permutes the items of \var{s} to satisfy
        \code{\var{s}[\var{i}] <= \var{s}[\var{j}]},
        for \code{\var{i} < \var{j}}}{(2)}
\end{tableiii}
\indexiv{operations on}{mutable}{sequence}{types}
\indexiii{operations on}{sequence}{types}
\indexiii{operations on}{list}{type}
\indexii{subscript}{assignment}
\indexii{slice}{assignment}
\stindex{del}
\renewcommand{\indexsubitem}{(list method)}
\ttindex{append}
\ttindex{count}
\ttindex{index}
\ttindex{insert}
\ttindex{remove}
\ttindex{reverse}
\ttindex{sort}

\noindent
Notes:
\begin{description}
\item[(1)] Raises an exception when \var{x} is not found in \var{s}.
  
\item[(2)] The \code{sort()} method takes an optional argument
  specifying a comparison function of two arguments (list items) which
  should return \code{-1}, \code{0} or \code{1} depending on whether the
  first argument is considered smaller than, equal to, or larger than the
  second argument.  Note that this slows the sorting process down
  considerably; e.g. to sort an array in reverse order it is much faster
  to use calls to \code{sort()} and \code{reverse()} than to use
  \code{sort()} with a comparison function that reverses the ordering of
  the elements.
\end{description}

\subsection{Mapping Types}

A \dfn{mapping} object maps values of one type (the key type) to
arbitrary objects.  Mappings are mutable objects.  There is currently
only one mapping type, the \dfn{dictionary}.  A dictionary's keys are
almost arbitrary values.  The only types of values not acceptable as
keys are values containing lists or dictionaries or other mutable
types that are compared by value rather than by object identity.
Numeric types used for keys obey the normal rules for numeric
comparison: if two numbers compare equal (e.g. 1 and 1.0) then they
can be used interchangeably to index the same dictionary entry.

\indexii{mapping}{types}
\indexii{dictionary}{type}

Dictionaries are created by placing a comma-separated list of
\code{\var{key}: \var{value}} pairs within braces, for example:
\code{\{'jack': 4098, 'sjoerd: 4127\}} or
\code{\{4098: 'jack', 4127: 'sjoerd\}}.

The following operations are defined on mappings (where \var{a} is a
mapping, \var{k} is a key and \var{x} is an arbitrary object):

\begin{tableiii}{|c|l|c|}{code}{Operation}{Result}{Notes}
  \lineiii{len(\var{a})}{the number of items in \var{a}}{}
  \lineiii{\var{a}[\var{k}]}{the item of \var{a} with key \var{k}}{(1)}
  \lineiii{\var{a}[\var{k}] = \var{x}}{set \code{\var{a}[\var{k}]} to \var{x}}{}
  \lineiii{del \var{a}[\var{k}]}{remove \code{\var{a}[\var{k}]} from \var{a}}{(1)}
  \lineiii{\var{a}.items()}{a copy of \var{a}'s list of (key, item) pairs}{(2)}
  \lineiii{\var{a}.keys()}{a copy of \var{a}'s list of keys}{(2)}
  \lineiii{\var{a}.values()}{a copy of \var{a}'s list of values}{(2)}
  \lineiii{\var{a}.has_key(\var{k})}{\code{1} if \var{a} has a key \var{k}, else \code{0}}{}
\end{tableiii}
\indexiii{operations on}{mapping}{types}
\indexiii{operations on}{dictionary}{type}
\stindex{del}
\bifuncindex{len}
\renewcommand{\indexsubitem}{(dictionary method)}
\ttindex{keys}
\ttindex{has_key}

% XXXJH some lines above, you talk about `true', elsewhere you
% explicitely states \code{0} or \code{1}.
\noindent
Notes:
\begin{description}
\item[(1)] Raises an exception if \var{k} is not in the map.

\item[(2)] Keys and values are listed in random order, but at any
moment the ordering of the \code{keys()}, \code{values()} and
\code{items()} lists is the consistent with each other.
\end{description}

\subsection{Other Built-in Types}

The interpreter supports several other kinds of objects.
Most of these support only one or two operations.

\subsubsection{Modules.}

The only special operation on a module is attribute access:
\code{\var{m}.\var{name}}, where \var{m} is a module and \var{name} accesses
a name defined in \var{m}'s symbol table.  Module attributes can be
assigned to.  (Note that the \code{import} statement is not, strictly
spoken, an operation on a module object; \code{import \var{foo}} does not
require a module object named \var{foo} to exist, rather it requires
an (external) \emph{definition} for a module named \var{foo}
somewhere.)

A special member of every module is \code{__dict__}.
This is the dictionary containing the module's symbol table.
Modifying this dictionary will actually change the module's symbol
table, but direct assignment to the \code{__dict__} attribute is not
possible (i.e., you can write \code{\var{m}.__dict__['a'] = 1}, which
defines \code{\var{m}.a} to be \code{1}, but you can't write \code{\var{m}.__dict__ = \{\}}.

Modules are written like this: \code{<module 'sys'>}.

\subsubsection{Classes and Class Instances.}
% XXXJH cross ref here
(See the Python Reference Manual for these.)

\subsubsection{Functions.}

Function objects are created by function definitions.  The only
operation on a function object is to call it:
\code{\var{func}(\var{argument-list})}.

There are really two flavors of function objects: built-in functions
and user-defined functions.  Both support the same operation (to call
the function), but the implementation is different, hence the
different object types.

The implementation adds two special read-only attributes:
\code{\var{f}.func_code} is a function's \dfn{code object} (see below) and
\code{\var{f}.func_globals} is the dictionary used as the function's
global name space (this is the same as \code{\var{m}.__dict__} where
\var{m} is the module in which the function \var{f} was defined).

\subsubsection{Methods.}
\obindex{method}

Methods are functions that are called using the attribute notation.
There are two flavors: built-in methods (such as \code{append()} on
lists) and class instance methods.  Built-in methods are described
with the types that support them.

The implementation adds two special read-only attributes to class
instance methods: \code{\var{m}.im_self} is the object whose method this
is, and \code{\var{m}.im_func} is the function implementing the method.
Calling \code{\var{m}(\var{arg-1}, \var{arg-2}, {\rm \ldots},
\var{arg-n})} is completely equivalent to calling
\code{\var{m}.im_func(\var{m}.im_self, \var{arg-1}, \var{arg-2}, {\rm
\ldots}, \var{arg-n})}.

(See the Python Reference Manual for more info.)

\subsubsection{Code Objects.}
\obindex{code}

Code objects are used by the implementation to represent
``pseudo-compiled'' executable Python code such as a function body.
They differ from function objects because they don't contain a
reference to their global execution environment.  Code objects are
returned by the built-in \code{compile()} function and can be
extracted from function objects through their \code{func_code}
attribute.
\bifuncindex{compile}
\ttindex{func_code}

A code object can be executed or evaluated by passing it (instead of a
source string) to the \code{exec} statement or the built-in
\code{eval()} function.
\stindex{exec}
\bifuncindex{eval}

(See the Python Reference Manual for more info.)

\subsubsection{Type Objects.}

Type objects represent the various object types.  An object's type is
% XXXJH xref here
accessed by the built-in function \code{type()}.  There are no special
operations on types.

Types are written like this: \code{<type 'int'>}.

\subsubsection{The Null Object.}

This object is returned by functions that don't explicitly return a
value.  It supports no special operations.  There is exactly one null
object, named \code{None} (a built-in name).

It is written as \code{None}.

\subsubsection{File Objects.}

File objects are implemented using \C{}'s \code{stdio} package and can be
% XXXJH xref here
created with the built-in function \code{open()} described under
Built-in Functions below.

When a file operation fails for an I/O-related reason, the exception
\code{IOError} is raised.  This includes situations where the
operation is not defined for some reason, like \code{seek()} on a tty
device or writing a file opened for reading.

Files have the following methods:


\renewcommand{\indexsubitem}{(file method)}

\begin{funcdesc}{close}{}
  Close the file.  A closed file cannot be read or written anymore.
\end{funcdesc}

\begin{funcdesc}{flush}{}
  Flush the internal buffer, like \code{stdio}'s \code{fflush()}.
\end{funcdesc}

\begin{funcdesc}{isatty}{}
  Return \code{1} if the file is connected to a tty(-like) device, else
  \code{0}.
\end{funcdesc}

\begin{funcdesc}{read}{size}
  Read at most \var{size} bytes from the file (less if the read hits
  \EOF{} or no more data is immediately available on a pipe, tty or
  similar device).  If the \var{size} argument is omitted, read all
  data until \EOF{} is reached.  The bytes are returned as a string
  object.  An empty string is returned when \EOF{} is encountered
  immediately.  (For certain files, like ttys, it makes sense to
  continue reading after an \EOF{} is hit.)
\end{funcdesc}

\begin{funcdesc}{readline}{}
  Read one entire line from the file.  A trailing newline character is
  kept in the string%
\footnote{The advantage of leaving the newline on is that an empty string 
	can be returned to mean \EOF{} without being ambiguous.  Another 
	advantage is that (in cases where it might matter, e.g. if you 
	want to make an exact copy of a file while scanning its lines) 
	you can tell whether the last line of a file ended in a newline
	or not (yes this happens!).}
  (but may be absent when a file ends with an
  incomplete line).  An empty string is returned when \EOF{} is hit
  immediately.  Note: unlike \code{stdio}'s \code{fgets()}, the returned
  string contains null characters (\code{'\e 0'}) if they occurred in the
  input.
\end{funcdesc}

\begin{funcdesc}{readlines}{}
  Read until \EOF{} using \code{readline()} and return a list containing
  the lines thus read.
\end{funcdesc}

\begin{funcdesc}{seek}{offset\, whence}
  Set the file's current position, like \code{stdio}'s \code{fseek()}.
  The \var{whence} argument is optional and defaults to \code{0}
  (absolute file positioning); other values are \code{1} (seek
  relative to the current position) and \code{2} (seek relative to the
  file's end).  There is no return value.
\end{funcdesc}

\begin{funcdesc}{tell}{}
  Return the file's current position, like \code{stdio}'s \code{ftell()}.
\end{funcdesc}

\begin{funcdesc}{write}{str}
  Write a string to the file.  There is no return value.
\end{funcdesc}

\begin{funcdesc}{writelines}{list}
Write a list of strings to the file.  There is no return value.
(The name is intended to match \code{readlines}; \code{writelines}
does not add line separators.)
\end{funcdesc}

\subsubsection{Internal Objects.}

(See the Python Reference Manual for these.)

\subsection{Special Attributes}

The implementation adds a few special read-only attributes to several
object types, where they are relevant:

\begin{itemize}

\item
\code{\var{x}.__dict__} is a dictionary of some sort used to store an
object's (writable) attributes;

\item
\code{\var{x}.__methods__} lists the methods of many built-in object types,
e.g., \code{[].__methods__} is
% XXXJH results in?, yields?, written down as an example
\code{['append', 'count', 'index', 'insert', 'remove', 'reverse', 'sort']};

\item
\code{\var{x}.__members__} lists data attributes;

\item
\code{\var{x}.__class__} is the class to which a class instance belongs;

\item
\code{\var{x}.__bases__} is the tuple of base classes of a class object.

\end{itemize}

\section{Built-in Exceptions}

\declaremodule{standard}{exceptions}
\modulesynopsis{Standard exceptions classes.}


Exceptions can be class objects or string objects.  Though most
exceptions have been string objects in past versions of Python, in
Python 1.5 and newer versions, all standard exceptions have been
converted to class objects, and users are encouraged to do the same.
The exceptions are defined in the module \module{exceptions}.  This
module never needs to be imported explicitly: the exceptions are
provided in the built-in namespace as well as the \module{exceptions}
module.

Two distinct string objects with the same value are considered different
exceptions.  This is done to force programmers to use exception names
rather than their string value when specifying exception handlers.
The string value of all built-in exceptions is their name, but this is
not a requirement for user-defined exceptions or exceptions defined by
library modules.

For class exceptions, in a \keyword{try}\stindex{try} statement with
an \keyword{except}\stindex{except} clause that mentions a particular
class, that clause also handles any exception classes derived from
that class (but not exception classes from which \emph{it} is
derived).  Two exception classes that are not related via subclassing
are never equivalent, even if they have the same name.

The built-in exceptions listed below can be generated by the
interpreter or built-in functions.  Except where mentioned, they have
an ``associated value'' indicating the detailed cause of the error.
This may be a string or a tuple containing several items of
information (e.g., an error code and a string explaining the code).
The associated value is the second argument to the
\keyword{raise}\stindex{raise} statement.  For string exceptions, the
associated value itself will be stored in the variable named as the
second argument of the \keyword{except} clause (if any).  For class
exceptions, that variable receives the exception instance.  If the
exception class is derived from the standard root class
\exception{Exception}, the associated value is present as the
exception instance's \member{args} attribute, and possibly on other
attributes as well.

User code can raise built-in exceptions.  This can be used to test an
exception handler or to report an error condition ``just like'' the
situation in which the interpreter raises the same exception; but
beware that there is nothing to prevent user code from raising an
inappropriate error.

The built-in exception classes can be sub-classed to define new
exceptions; programmers are encouraged to at least derive new
exceptions from the \exception{Exception} base class.  More
information on defining exceptions is available in the
\citetitle[../tut/tut.html]{Python Tutorial} under the heading
``User-defined Exceptions.''

\setindexsubitem{(built-in exception base class)}

The following exceptions are only used as base classes for other
exceptions.

\begin{excdesc}{Exception}
The root class for exceptions.  All built-in exceptions are derived
from this class.  All user-defined exceptions should also be derived
from this class, but this is not (yet) enforced.  The \function{str()}
function, when applied to an instance of this class (or most derived
classes) returns the string value of the argument or arguments, or an
empty string if no arguments were given to the constructor.  When used
as a sequence, this accesses the arguments given to the constructor
(handy for backward compatibility with old code).  The arguments are
also available on the instance's \member{args} attribute, as a tuple.
\end{excdesc}

\begin{excdesc}{StandardError}
The base class for all built-in exceptions except
\exception{StopIteration} and \exception{SystemExit}.
\exception{StandardError} itself is derived from the root class
\exception{Exception}.
\end{excdesc}

\begin{excdesc}{ArithmeticError}
The base class for those built-in exceptions that are raised for
various arithmetic errors: \exception{OverflowError},
\exception{ZeroDivisionError}, \exception{FloatingPointError}.
\end{excdesc}

\begin{excdesc}{LookupError}
The base class for the exceptions that are raised when a key or
index used on a mapping or sequence is invalid: \exception{IndexError},
\exception{KeyError}.  This can be raised directly by
\function{sys.setdefaultencoding()}.
\end{excdesc}

\begin{excdesc}{EnvironmentError}
The base class for exceptions that
can occur outside the Python system: \exception{IOError},
\exception{OSError}.  When exceptions of this type are created with a
2-tuple, the first item is available on the instance's \member{errno}
attribute (it is assumed to be an error number), and the second item
is available on the \member{strerror} attribute (it is usually the
associated error message).  The tuple itself is also available on the
\member{args} attribute.
\versionadded{1.5.2}

When an \exception{EnvironmentError} exception is instantiated with a
3-tuple, the first two items are available as above, while the third
item is available on the \member{filename} attribute.  However, for
backwards compatibility, the \member{args} attribute contains only a
2-tuple of the first two constructor arguments.

The \member{filename} attribute is \code{None} when this exception is
created with other than 3 arguments.  The \member{errno} and
\member{strerror} attributes are also \code{None} when the instance was
created with other than 2 or 3 arguments.  In this last case,
\member{args} contains the verbatim constructor arguments as a tuple.
\end{excdesc}


\setindexsubitem{(built-in exception)}

The following exceptions are the exceptions that are actually raised.

\begin{excdesc}{AssertionError}
\stindex{assert}
Raised when an \keyword{assert} statement fails.
\end{excdesc}

\begin{excdesc}{AttributeError}
% xref to attribute reference?
  Raised when an attribute reference or assignment fails.  (When an
  object does not support attribute references or attribute assignments
  at all, \exception{TypeError} is raised.)
\end{excdesc}

\begin{excdesc}{EOFError}
% XXXJH xrefs here
  Raised when one of the built-in functions (\function{input()} or
  \function{raw_input()}) hits an end-of-file condition (\EOF{}) without
  reading any data.
% XXXJH xrefs here
  (N.B.: the \method{read()} and \method{readline()} methods of file
  objects return an empty string when they hit \EOF{}.)
\end{excdesc}

\begin{excdesc}{FloatingPointError}
  Raised when a floating point operation fails.  This exception is
  always defined, but can only be raised when Python is configured
  with the \longprogramopt{with-fpectl} option, or the
  \constant{WANT_SIGFPE_HANDLER} symbol is defined in the
  \file{pyconfig.h} file.
\end{excdesc}

\begin{excdesc}{IOError}
% XXXJH xrefs here
  Raised when an I/O operation (such as a \keyword{print} statement,
  the built-in \function{open()} function or a method of a file
  object) fails for an I/O-related reason, e.g., ``file not found'' or
  ``disk full''.

  This class is derived from \exception{EnvironmentError}.  See the
  discussion above for more information on exception instance
  attributes.
\end{excdesc}

\begin{excdesc}{ImportError}
% XXXJH xref to import statement?
  Raised when an \keyword{import} statement fails to find the module
  definition or when a \code{from \textrm{\ldots} import} fails to find a
  name that is to be imported.
\end{excdesc}

\begin{excdesc}{IndexError}
% XXXJH xref to sequences
  Raised when a sequence subscript is out of range.  (Slice indices are
  silently truncated to fall in the allowed range; if an index is not a
  plain integer, \exception{TypeError} is raised.)
\end{excdesc}

\begin{excdesc}{KeyError}
% XXXJH xref to mapping objects?
  Raised when a mapping (dictionary) key is not found in the set of
  existing keys.
\end{excdesc}

\begin{excdesc}{KeyboardInterrupt}
  Raised when the user hits the interrupt key (normally
  \kbd{Control-C} or \kbd{Delete}).  During execution, a check for
  interrupts is made regularly.
% XXXJH xrefs here
  Interrupts typed when a built-in function \function{input()} or
  \function{raw_input()}) is waiting for input also raise this
  exception.
\end{excdesc}

\begin{excdesc}{MemoryError}
  Raised when an operation runs out of memory but the situation may
  still be rescued (by deleting some objects).  The associated value is
  a string indicating what kind of (internal) operation ran out of memory.
  Note that because of the underlying memory management architecture
  (C's \cfunction{malloc()} function), the interpreter may not
  always be able to completely recover from this situation; it
  nevertheless raises an exception so that a stack traceback can be
  printed, in case a run-away program was the cause.
\end{excdesc}

\begin{excdesc}{NameError}
  Raised when a local or global name is not found.  This applies only
  to unqualified names.  The associated value is the name that could
  not be found.
\end{excdesc}

\begin{excdesc}{NotImplementedError}
  This exception is derived from \exception{RuntimeError}.  In user
  defined base classes, abstract methods should raise this exception
  when they require derived classes to override the method.
  \versionadded{1.5.2}
\end{excdesc}

\begin{excdesc}{OSError}
  %xref for os module
  This class is derived from \exception{EnvironmentError} and is used
  primarily as the \refmodule{os} module's \code{os.error} exception.
  See \exception{EnvironmentError} above for a description of the
  possible associated values.
  \versionadded{1.5.2}
\end{excdesc}

\begin{excdesc}{OverflowError}
% XXXJH reference to long's and/or int's?
  Raised when the result of an arithmetic operation is too large to be
  represented.  This cannot occur for long integers (which would rather
  raise \exception{MemoryError} than give up).  Because of the lack of
  standardization of floating point exception handling in C, most
  floating point operations also aren't checked.  For plain integers,
  all operations that can overflow are checked except left shift, where
  typical applications prefer to drop bits than raise an exception.
\end{excdesc}

\begin{excdesc}{RuntimeError}
  Raised when an error is detected that doesn't fall in any of the
  other categories.  The associated value is a string indicating what
  precisely went wrong.  (This exception is mostly a relic from a
  previous version of the interpreter; it is not used very much any
  more.)
\end{excdesc}

\begin{excdesc}{StopIteration}
  Raised by an iterator's \method{next()} method to signal that there
  are no further values.
  This is derived from \exception{Exception} rather than
  \exception{StandardError}, since this is not considered an error in
  its normal application.
  \versionadded{2.2}
\end{excdesc}

\begin{excdesc}{SyntaxError}
% XXXJH xref to these functions?
  Raised when the parser encounters a syntax error.  This may occur in
  an \keyword{import} statement, in an \keyword{exec} statement, in a call
  to the built-in function \function{eval()} or \function{input()}, or
  when reading the initial script or standard input (also
  interactively).

  Instances of this class have atttributes \member{filename},
  \member{lineno}, \member{offset} and \member{text} for easier access
  to the details.  \function{str()} of the exception instance returns
  only the message.
\end{excdesc}

\begin{excdesc}{SystemError}
  Raised when the interpreter finds an internal error, but the
  situation does not look so serious to cause it to abandon all hope.
  The associated value is a string indicating what went wrong (in
  low-level terms).
  
  You should report this to the author or maintainer of your Python
  interpreter.  Be sure to report the version of the Python
  interpreter (\code{sys.version}; it is also printed at the start of an
  interactive Python session), the exact error message (the exception's
  associated value) and if possible the source of the program that
  triggered the error.
\end{excdesc}

\begin{excdesc}{SystemExit}
% XXXJH xref to module sys?
  This exception is raised by the \function{sys.exit()} function.  When it
  is not handled, the Python interpreter exits; no stack traceback is
  printed.  If the associated value is a plain integer, it specifies the
  system exit status (passed to C's \cfunction{exit()} function); if it is
  \code{None}, the exit status is zero; if it has another type (such as
  a string), the object's value is printed and the exit status is one.

  Instances have an attribute \member{code} which is set to the
  proposed exit status or error message (defaulting to \code{None}).
  Also, this exception derives directly from \exception{Exception} and
  not \exception{StandardError}, since it is not technically an error.

  A call to \function{sys.exit()} is translated into an exception so that
  clean-up handlers (\keyword{finally} clauses of \keyword{try} statements)
  can be executed, and so that a debugger can execute a script without
  running the risk of losing control.  The \function{os._exit()} function
  can be used if it is absolutely positively necessary to exit
  immediately (for example, in the child process after a call to
  \function{fork()}).
\end{excdesc}

\begin{excdesc}{TypeError}
  Raised when a built-in operation or function is applied to an object
  of inappropriate type.  The associated value is a string giving
  details about the type mismatch.
\end{excdesc}

\begin{excdesc}{UnboundLocalError}
  Raised when a reference is made to a local variable in a function or
  method, but no value has been bound to that variable.  This is a
  subclass of \exception{NameError}.
\versionadded{2.0}
\end{excdesc}

\begin{excdesc}{UnicodeError}
  Raised when a Unicode-related encoding or decoding error occurs.  It
  is a subclass of \exception{ValueError}.
\versionadded{2.0}
\end{excdesc}

\begin{excdesc}{ValueError}
  Raised when a built-in operation or function receives an argument
  that has the right type but an inappropriate value, and the
  situation is not described by a more precise exception such as
  \exception{IndexError}.
\end{excdesc}

\begin{excdesc}{WindowsError}
  Raised when a Windows-specific error occurs or when the error number
  does not correspond to an \cdata{errno} value.  The
  \member{errno} and \member{strerror} values are created from the
  return values of the \cfunction{GetLastError()} and
  \cfunction{FormatMessage()} functions from the Windows Platform API.
  This is a subclass of \exception{OSError}.
\versionadded{2.0}
\end{excdesc}

\begin{excdesc}{ZeroDivisionError}
  Raised when the second argument of a division or modulo operation is
  zero.  The associated value is a string indicating the type of the
  operands and the operation.
\end{excdesc}


\setindexsubitem{(built-in warning)}

The following exceptions are used as warning categories; see the
\module{warnings} module for more information.

\begin{excdesc}{Warning}
Base class for warning categories.
\end{excdesc}

\begin{excdesc}{UserWarning}
Base class for warnings generated by user code.
\end{excdesc}

\begin{excdesc}{DeprecationWarning}
Base class for warnings about deprecated features.
\end{excdesc}

\begin{excdesc}{SyntaxWarning}
Base class for warnings about dubious syntax
\end{excdesc}

\begin{excdesc}{RuntimeWarning}
Base class for warnings about dubious runtime behavior.
\end{excdesc}

\section{Built-in Functions}

The Python interpreter has a number of functions built into it that
are always available.  They are listed here in alphabetical order.


\renewcommand{\indexsubitem}{(built-in function)}
\begin{funcdesc}{abs}{x}
  Return the absolute value of a number.  The argument may be a plain
  or long integer or a floating point number.
\end{funcdesc}

\begin{funcdesc}{apply}{function\, args}
The \var{function} argument must be a callable object (a user-defined or
built-in function or method, or a class object) and the \var{args}
argument must be a tuple.  The \var{function} is called with
\var{args} as argument list; the number of arguments is the the length
of the tuple.  (This is different from just calling
\code{\var{func}(\var{args})}, since in that case there is always
exactly one argument.)
\end{funcdesc}

\begin{funcdesc}{chr}{i}
  Return a string of one character whose \ASCII{} code is the integer
  \var{i}, e.g., \code{chr(97)} returns the string \code{'a'}.  This is the
  inverse of \code{ord()}.  The argument must be in the range [0..255],
  inclusive.
\end{funcdesc}

\begin{funcdesc}{cmp}{x\, y}
  Compare the two objects \var{x} and \var{y} and return an integer
  according to the outcome.  The return value is negative if \code{\var{x}
  < \var{y}}, zero if \code{\var{x} == \var{y}} and strictly positive if
  \code{\var{x} > \var{y}}.
\end{funcdesc}

\begin{funcdesc}{coerce}{x\, y}
  Return a tuple consisting of the two numeric arguments converted to
  a common type, using the same rules as used by arithmetic
  operations.
\end{funcdesc}

\begin{funcdesc}{compile}{string\, filename\, kind}
  Compile the \var{string} into a code object.  Code objects can be
  executed by an \code{exec} statement or evaluated by a call to
  \code{eval()}.  The \var{filename} argument should
  give the file from which the code was read; pass e.g. \code{'<string>'}
  if it wasn't read from a file.  The \var{kind} argument specifies
  what kind of code must be compiled; it can be \code{'exec'} if
  \var{string} consists of a sequence of statements, \code{'eval'}
  if it consists of a single expression, or \code{'single'} if
  it consists of a single interactive statement (in the latter case,
  expression statements that evaluate to something else than
  \code{None} will printed).
\end{funcdesc}

\begin{funcdesc}{delattr}{object\, name}
  This is a relative of \code{setattr}.  The arguments are an
  object and a string.  The string must be the name
  of one of the object's attributes.  The function deletes
  the named attribute, provided the object allows it.  For example,
  \code{delattr(\var{x}, '\var{foobar}')} is equivalent to
  \code{del \var{x}.\var{foobar}}.
\end{funcdesc}

\begin{funcdesc}{dir}{}
  Without arguments, return the list of names in the current local
  symbol table.  With a module, class or class instance object as
  argument (or anything else that has a \code{__dict__} attribute),
  returns the list of names in that object's attribute dictionary.
  The resulting list is sorted.  For example:

\bcode\begin{verbatim}
>>> import sys
>>> dir()
['sys']
>>> dir(sys)
['argv', 'exit', 'modules', 'path', 'stderr', 'stdin', 'stdout']
>>> 
\end{verbatim}\ecode
\end{funcdesc}

\begin{funcdesc}{divmod}{a\, b}
  Take two numbers as arguments and return a pair of integers
  consisting of their integer quotient and remainder.  With mixed
  operand types, the rules for binary arithmetic operators apply.  For
  plain and long integers, the result is the same as
  \code{(\var{a} / \var{b}, \var{a} \%{} \var{b})}.
  For floating point numbers the result is the same as
  \code{(math.floor(\var{a} / \var{b}), \var{a} \%{} \var{b})}.
\end{funcdesc}

\begin{funcdesc}{eval}{expression\optional{\, globals\optional{\, locals}}}
  The arguments are a string and two optional dictionaries.  The
  \var{expression} argument is parsed and evaluated as a Python
  expression (technically speaking, a condition list) using the
  \var{globals} and \var{locals} dictionaries as global and local name
  space.  If the \var{locals} dictionary is omitted it defaults to
  the \var{globals} dictionary.  If both dictionaries are omitted, the
  expression is executed in the environment where \code{eval} is
  called.  The return value is the result of the evaluated expression.
  Syntax errors are reported as exceptions.  Example:

\bcode\begin{verbatim}
>>> x = 1
>>> print eval('x+1')
2
>>> 
\end{verbatim}\ecode

  This function can also be used to execute arbitrary code objects
  (e.g.\ created by \code{compile()}).  In this case pass a code
  object instead of a string.  The code object must have been compiled
  passing \code{'eval'} to the \var{kind} argument.

  Hints: dynamic execution of statements is supported by the
  \code{exec} statement.  Execution of statements from a file is
  supported by the \code{execfile()} function.  The \code{globals()}
  and \code{locals()} functions returns the current global and local
  dictionary, respectively, which may be useful
  to pass around for use by \code{eval()} or \code{execfile()}.

\end{funcdesc}

\begin{funcdesc}{execfile}{file\optional{\, globals\optional{\, locals}}}
  This function is similar to the
  \code{exec} statement, but parses a file instead of a string.  It is
  different from the \code{import} statement in that it does not use
  the module administration --- it reads the file unconditionally and
  does not create a new module.\footnote{It is used relatively rarely
  so does not warrant being made into a statement.}

  The arguments are a file name and two optional dictionaries.  The
  file is parsed and evaluated as a sequence of Python statements
  (similarly to a module) using the \var{globals} and \var{locals}
  dictionaries as global and local name space.  If the \var{locals}
  dictionary is omitted it defaults to the \var{globals} dictionary.
  If both dictionaries are omitted, the expression is executed in the
  environment where \code{execfile()} is called.  The return value is
  \code{None}.
\end{funcdesc}

\begin{funcdesc}{filter}{function\, list}
Construct a list from those elements of \var{list} for which
\var{function} returns true.  If \var{list} is a string or a tuple,
the result also has that type; otherwise it is always a list.  If
\var{function} is \code{None}, the identity function is assumed,
i.e.\ all elements of \var{list} that are false (zero or empty) are
removed.
\end{funcdesc}

\begin{funcdesc}{float}{x}
  Convert a number to floating point.  The argument may be a plain or
  long integer or a floating point number.
\end{funcdesc}

\begin{funcdesc}{getattr}{object\, name}
  The arguments are an object and a string.  The string must be the
  name
  of one of the object's attributes.  The result is the value of that
  attribute.  For example, \code{getattr(\var{x}, '\var{foobar}')} is equivalent to
  \code{\var{x}.\var{foobar}}.
\end{funcdesc}

\begin{funcdesc}{globals}{}
Return a dictionary representing the current global symbol table.
This is always the dictionary of the current module (inside a
function or method, this is the module where it is defined, not the
module from which it is called).
\end{funcdesc}

\begin{funcdesc}{hasattr}{object\, name}
  The arguments are an object and a string.  The result is 1 if the
  string is the name of one of the object's attributes, 0 if not.
  (This is implemented by calling \code{getattr(object, name)} and
  seeing whether it raises an exception or not.)
\end{funcdesc}

\begin{funcdesc}{hash}{object}
  Return the hash value of the object (if it has one).  Hash values
  are 32-bit integers.  They are used to quickly compare dictionary
  keys during a dictionary lookup.  Numeric values that compare equal
  have the same hash value (even if they are of different types, e.g.
  1 and 1.0).
\end{funcdesc}

\begin{funcdesc}{hex}{x}
  Convert an integer number (of any size) to a hexadecimal string.
  The result is a valid Python expression.
\end{funcdesc}

\begin{funcdesc}{id}{object}
  Return the `identity' of an object.  This is an integer which is
  guaranteed to be unique and constant for this object during its
  lifetime.  (Two objects whose lifetimes are disjunct may have the
  same id() value.)  (Implementation note: this is the address of the
  object.)
\end{funcdesc}

\begin{funcdesc}{input}{\optional{prompt}}
  Almost equivalent to \code{eval(raw_input(\var{prompt}))}.  Like
  \code{raw_input()}, the \var{prompt} argument is optional.  The difference
  is that a long input expression may be broken over multiple lines using
  the backslash convention.
\end{funcdesc}

\begin{funcdesc}{int}{x}
  Convert a number to a plain integer.  The argument may be a plain or
  long integer or a floating point number.  Conversion of floating
  point numbers to integers is defined by the C semantics; normally
  the conversion truncates towards zero.\footnote{This is ugly --- the
  language definition should require truncation towards zero.}
\end{funcdesc}

\begin{funcdesc}{len}{s}
  Return the length (the number of items) of an object.  The argument
  may be a sequence (string, tuple or list) or a mapping (dictionary).
\end{funcdesc}

\begin{funcdesc}{locals}{}
Return a dictionary representing the current local symbol table.
Inside a function, modifying this dictionary does not always have the
desired effect.
\end{funcdesc}

\begin{funcdesc}{long}{x}
  Convert a number to a long integer.  The argument may be a plain or
  long integer or a floating point number.
\end{funcdesc}

\begin{funcdesc}{map}{function\, list\, ...}
Apply \var{function} to every item of \var{list} and return a list
of the results.  If additional \var{list} arguments are passed, 
\var{function} must take that many arguments and is applied to
the items of all lists in parallel; if a list is shorter than another
it is assumed to be extended with \code{None} items.  If
\var{function} is \code{None}, the identity function is assumed; if
there are multiple list arguments, \code{map} returns a list
consisting of tuples containing the corresponding items from all lists
(i.e. a kind of transpose operation).  The \var{list} arguments may be
any kind of sequence; the result is always a list.
\end{funcdesc}

\begin{funcdesc}{max}{s}
  Return the largest item of a non-empty sequence (string, tuple or
  list).
\end{funcdesc}

\begin{funcdesc}{min}{s}
  Return the smallest item of a non-empty sequence (string, tuple or
  list).
\end{funcdesc}

\begin{funcdesc}{oct}{x}
  Convert an integer number (of any size) to an octal string.  The
  result is a valid Python expression.
\end{funcdesc}

\begin{funcdesc}{open}{filename\optional{\, mode\optional{\, bufsize}}}
  Return a new file object (described earlier under Built-in Types).
  The first two arguments are the same as for \code{stdio}'s
  \code{fopen()}: \var{filename} is the file name to be opened,
  \var{mode} indicates how the file is to be opened: \code{'r'} for
  reading, \code{'w'} for writing (truncating an existing file), and
  \code{'a'} opens it for appending.  Modes \code{'r+'}, \code{'w+'} and
  \code{'a+'} open the file for updating, provided the underlying
  \code{stdio} library understands this.  On systems that differentiate
  between binary and text files, \code{'b'} appended to the mode opens
  the file in binary mode.  If the file cannot be opened, \code{IOError}
  is raised.
If \var{mode} is omitted, it defaults to \code{'r'}.
The optional \var{bufsize} argument specifies the file's desired
buffer size: 0 means unbuffered, 1 means line buffered, any other
positive value means use a buffer of (approximately) that size.  A
negative \var{bufsize} means to use the system default, which is
usually line buffered for for tty devices and fully buffered for other
files.%
\footnote{Specifying a buffer size currently has no effect on systems
that don't have \code{setvbuf()}.  The interface to specify the buffer
size is not done using a method that calls \code{setvbuf()}, because
that may dump core when called after any I/O has been performed, and
there's no reliable way to determine whether this is the case.}
\end{funcdesc}

\begin{funcdesc}{ord}{c}
  Return the \ASCII{} value of a string of one character.  E.g.,
  \code{ord('a')} returns the integer \code{97}.  This is the inverse of
  \code{chr()}.
\end{funcdesc}

\begin{funcdesc}{pow}{x\, y\optional{\, z}}
  Return \var{x} to the power \var{y}; if \var{z} is present, return
  \var{x} to the power \var{y}, modulo \var{z} (computed more
  efficiently than \code{pow(\var{x}, \var{y}) \% \var{z}}).
  The arguments must have
  numeric types.  With mixed operand types, the rules for binary
  arithmetic operators apply.  The effective operand type is also the
  type of the result; if the result is not expressible in this type, the
  function raises an exception; e.g., \code{pow(2, -1)} or \code{pow(2,
  35000)} is not allowed.
\end{funcdesc}

\begin{funcdesc}{range}{\optional{start\,} end\optional{\, step}}
  This is a versatile function to create lists containing arithmetic
  progressions.  It is most often used in \code{for} loops.  The
  arguments must be plain integers.  If the \var{step} argument is
  omitted, it defaults to \code{1}.  If the \var{start} argument is
  omitted, it defaults to \code{0}.  The full form returns a list of
  plain integers \code{[\var{start}, \var{start} + \var{step},
  \var{start} + 2 * \var{step}, \ldots]}.  If \var{step} is positive,
  the last element is the largest \code{\var{start} + \var{i} *
  \var{step}} less than \var{end}; if \var{step} is negative, the last
  element is the largest \code{\var{start} + \var{i} * \var{step}}
  greater than \var{end}.  \var{step} must not be zero (or else an
  exception is raised).  Example:

\bcode\begin{verbatim}
>>> range(10)
[0, 1, 2, 3, 4, 5, 6, 7, 8, 9]
>>> range(1, 11)
[1, 2, 3, 4, 5, 6, 7, 8, 9, 10]
>>> range(0, 30, 5)
[0, 5, 10, 15, 20, 25]
>>> range(0, 10, 3)
[0, 3, 6, 9]
>>> range(0, -10, -1)
[0, -1, -2, -3, -4, -5, -6, -7, -8, -9]
>>> range(0)
[]
>>> range(1, 0)
[]
>>> 
\end{verbatim}\ecode
\end{funcdesc}

\begin{funcdesc}{raw_input}{\optional{prompt}}
  If the \var{prompt} argument is present, it is written to standard output
  without a trailing newline.  The function then reads a line from input,
  converts it to a string (stripping a trailing newline), and returns that.
  When \EOF{} is read, \code{EOFError} is raised. Example:

\bcode\begin{verbatim}
>>> s = raw_input('--> ')
--> Monty Python's Flying Circus
>>> s
"Monty Python's Flying Circus"
>>> 
\end{verbatim}\ecode
\end{funcdesc}

\begin{funcdesc}{reduce}{function\, list\optional{\, initializer}}
Apply the binary \var{function} to the items of \var{list} so as to
reduce the list to a single value.  E.g.,
\code{reduce(lambda x, y: x*y, \var{list}, 1)} returns the product of
the elements of \var{list}.  The optional \var{initializer} can be
thought of as being prepended to \var{list} so as to allow reduction
of an empty \var{list}.  The \var{list} arguments may be any kind of
sequence.
\end{funcdesc}

\begin{funcdesc}{reload}{module}
Re-parse and re-initialize an already imported \var{module}.  The
argument must be a module object, so it must have been successfully
imported before.  This is useful if you have edited the module source
file using an external editor and want to try out the new version
without leaving the Python interpreter.  The return value is the
module object (i.e.\ the same as the \var{module} argument).

There are a number of caveats:

If a module is syntactically correct but its initialization fails, the
first \code{import} statement for it does not bind its name locally,
but does store a (partially initialized) module object in
\code{sys.modules}.  To reload the module you must first
\code{import} it again (this will bind the name to the partially
initialized module object) before you can \code{reload()} it.

When a module is reloaded, its dictionary (containing the module's
global variables) is retained.  Redefinitions of names will override
the old definitions, so this is generally not a problem.  If the new
version of a module does not define a name that was defined by the old
version, the old definition remains.  This feature can be used to the
module's advantage if it maintains a global table or cache of objects
--- with a \code{try} statement it can test for the table's presence
and skip its initialization if desired.

It is legal though generally not very useful to reload built-in or
dynamically loaded modules, except for \code{sys}, \code{__main__} and
\code{__builtin__}.  In certain cases, however, extension modules are
not designed to be initialized more than once, and may fail in
arbitrary ways when reloaded.

If a module imports objects from another module using \code{from}
{\ldots} \code{import} {\ldots}, calling \code{reload()} for the other
module does not redefine the objects imported from it --- one way
around this is to re-execute the \code{from} statement, another is to
use \code{import} and qualified names (\var{module}.\var{name})
instead.

If a module instantiates instances of a class, reloading the module
that defines the class does not affect the method definitions of the
instances --- they continue to use the old class definition.  The same
is true for derived classes.
\end{funcdesc}

\begin{funcdesc}{repr}{object}
Return a string containing a printable representation of an object.
This is the same value yielded by conversions (reverse quotes).
It is sometimes useful to be able to access this operation as an
ordinary function.  For many types, this function makes an attempt
to return a string that would yield an object with the same value
when passed to \code{eval()}.
\end{funcdesc}

\begin{funcdesc}{round}{x\, n}
  Return the floating point value \var{x} rounded to \var{n} digits
  after the decimal point.  If \var{n} is omitted, it defaults to zero.
  The result is a floating point number.  Values are rounded to the
  closest multiple of 10 to the power minus \var{n}; if two multiples
  are equally close, rounding is done away from 0 (so e.g.
  \code{round(0.5)} is \code{1.0} and \code{round(-0.5)} is \code{-1.0}).
\end{funcdesc}

\begin{funcdesc}{setattr}{object\, name\, value}
  This is the counterpart of \code{getattr}.  The arguments are an
  object, a string and an arbitrary value.  The string must be the name
  of one of the object's attributes.  The function assigns the value to
  the attribute, provided the object allows it.  For example,
  \code{setattr(\var{x}, '\var{foobar}', 123)} is equivalent to
  \code{\var{x}.\var{foobar} = 123}.
\end{funcdesc}

\begin{funcdesc}{str}{object}
Return a string containing a nicely printable representation of an
object.  For strings, this returns the string itself.  The difference
with \code{repr(\var{object})} is that \code{str(\var{object})} does not
always attempt to return a string that is acceptable to \code{eval()};
its goal is to return a printable string.
\end{funcdesc}

\begin{funcdesc}{tuple}{sequence}
Return a tuple whose items are the same and in the same order as
\var{sequence}'s items.  If \var{sequence} is alread a tuple, it
is returned unchanged.  For instance, \code{tuple('abc')} returns
returns \code{('a', 'b', 'c')} and \code{tuple([1, 2, 3])} returns
\code{(1, 2, 3)}.
\end{funcdesc}

\begin{funcdesc}{type}{object}
Return the type of an \var{object}.  The return value is a type
object.  The standard module \code{types} defines names for all
built-in types.
\stmodindex{types}
\obindex{type}
For instance:

\bcode\begin{verbatim}
>>> import types
>>> if type(x) == types.StringType: print "It's a string"
\end{verbatim}\ecode
\end{funcdesc}

\begin{funcdesc}{vars}{\optional{object}}
Without arguments, return a dictionary corresponding to the current
local symbol table.  With a module, class or class instance object as
argument (or anything else that has a \code{__dict__} attribute),
returns a dictionary corresponding to the object's symbol table.
The returned dictionary should not be modified: the effects on the
corresponding symbol table are undefined.%
\footnote{In the current implementation, local variable bindings
cannot normally be affected this way, but variables retrieved from
other scopes (e.g. modules) can be.  This may change.}
\end{funcdesc}

\begin{funcdesc}{xrange}{\optional{start\,} end\optional{\, step}}
This function is very similar to \code{range()}, but returns an
``xrange object'' instead of a list.  This is an opaque sequence type
which yields the same values as the corresponding list, without
actually storing them all simultaneously.  The advantage of
\code{xrange()} over \code{range()} is minimal (since \code{xrange()}
still has to create the values when asked for them) except when a very
large range is used on a memory-starved machine (e.g. MS-DOS) or when all
of the range's elements are never used (e.g. when the loop is usually
terminated with \code{break}).
\end{funcdesc}


\chapter{Python Services}
\label{python}

The modules described in this chapter provide a wide range of services
related to the Python interpreter and its interaction with its
environment.  Here's an overview:

\begin{description}

\item[sys]
--- Access system specific parameters and functions.

\item[types]
--- Names for all built-in types.

\item[UserDict]
\item[UserList]
--- Class wrappers for dictionary and list objects.

\item[operator]
--- All python's standard operators as built-in functions.

\item[traceback]
--- Print or retrieve a stack traceback.

\item[pickle]
--- Convert Python objects to streams of bytes and back.

\item[copy_reg]
--- Register \module{pickle} support functions.

\item[shelve]
--- Python object persistency.

\item[copy]
--- Shallow and deep copy operations.

\item[marshal]
--- Convert Python objects to streams of bytes and back (with
different constraints).

\item[imp]
--- Access the implementation of the \keyword{import} statement.

\item[ni]
--- New import (obsolete).

\item[parser]
--- Retrieve and submit parse trees from and to the runtime support
environment.

\item[symbol]
--- Constants representing internal nodes of the parse tree.

\item[token]
--- Constants representing terminal nodes of the parse tree.

\item[keyword]
--- Test whether a string is a keyword in the Python language.

\item[code]
--- Code object services.

\item[pprint]
--- Data pretty printer.

\item[dis]
--- Disassembler.

\item[site]
--- A standard way to reference site-specific modules.

\item[user]
--- A standard way to reference user-specific modules.

\item[__builtin__]
--- The set of built-in functions.

\item[__main__]
--- The environment where the top-level script is run.

\end{description}
		% Python Services
\section{\module{sys} ---
         System-specific parameters and functions}

\declaremodule{builtin}{sys}
\modulesynopsis{Access system-specific parameters and functions.}

This module provides access to some variables used or maintained by the
interpreter and to functions that interact strongly with the interpreter.
It is always available.


\begin{datadesc}{argv}
  The list of command line arguments passed to a Python script.
  \code{argv[0]} is the script name (it is operating system
  dependent whether this is a full pathname or not).
  If the command was executed using the \programopt{-c} command line
  option to the interpreter, \code{argv[0]} is set to the string
  \code{'-c'}.
  If no script name was passed to the Python interpreter,
  \code{argv} has zero length.
\end{datadesc}

\begin{datadesc}{byteorder}
  An indicator of the native byte order.  This will have the value
  \code{'big'} on big-endian (most-signigicant byte first) platforms,
  and \code{'little'} on little-endian (least-significant byte first)
  platforms.
  \versionadded{2.0}
\end{datadesc}

\begin{datadesc}{builtin_module_names}
  A tuple of strings giving the names of all modules that are compiled
  into this Python interpreter.  (This information is not available in
  any other way --- \code{modules.keys()} only lists the imported
  modules.)
\end{datadesc}

\begin{datadesc}{copyright}
A string containing the copyright pertaining to the Python interpreter.
\end{datadesc}

\begin{datadesc}{dllhandle}
Integer specifying the handle of the Python DLL.
Availability: Windows.
\end{datadesc}

\begin{funcdesc}{exc_info}{}
This function returns a tuple of three values that give information
about the exception that is currently being handled.  The information
returned is specific both to the current thread and to the current
stack frame.  If the current stack frame is not handling an exception,
the information is taken from the calling stack frame, or its caller,
and so on until a stack frame is found that is handling an exception.
Here, ``handling an exception'' is defined as ``executing or having
executed an except clause.''  For any stack frame, only
information about the most recently handled exception is accessible.

If no exception is being handled anywhere on the stack, a tuple
containing three \code{None} values is returned.  Otherwise, the
values returned are
\code{(\var{type}, \var{value}, \var{traceback})}.
Their meaning is: \var{type} gets the exception type of the exception
being handled (a string or class object); \var{value} gets the
exception parameter (its \dfn{associated value} or the second argument
to \keyword{raise}, which is always a class instance if the exception
type is a class object); \var{traceback} gets a traceback object (see
the Reference Manual) which encapsulates the call stack at the point
where the exception originally occurred.
\obindex{traceback}

\strong{Warning:} assigning the \var{traceback} return value to a
local variable in a function that is handling an exception will cause
a circular reference. This will prevent anything referenced by a local
variable in the same function or by the traceback from being garbage
collected.  Since most functions don't need access to the traceback,
the best solution is to use something like
\code{type, value = sys.exc_info()[:2]}
to extract only the exception type and value.  If you do need the
traceback, make sure to delete it after use (best done with a
\keyword{try} ... \keyword{finally} statement) or to call
\function{exc_info()} in a function that does not itself handle an
exception.
\end{funcdesc}

\begin{datadesc}{exc_type}
\dataline{exc_value}
\dataline{exc_traceback}
\deprecated {1.5}
            {Use \function{exc_info()} instead.}
Since they are global variables, they are not specific to the current
thread, so their use is not safe in a multi-threaded program.  When no
exception is being handled, \code{exc_type} is set to \code{None} and
the other two are undefined.
\end{datadesc}

\begin{datadesc}{exec_prefix}
A string giving the site-specific directory prefix where the
platform-dependent Python files are installed; by default, this is
also \code{'/usr/local'}.  This can be set at build time with the
\longprogramopt{exec-prefix} argument to the
\program{configure} script.  Specifically, all configuration files
(e.g. the \file{config.h} header file) are installed in the directory
\code{exec_prefix + '/lib/python\var{version}/config'}, and shared
library modules are installed in \code{exec_prefix +
'/lib/python\var{version}/lib-dynload'}, where \var{version} is equal
to \code{version[:3]}.
\end{datadesc}

\begin{datadesc}{executable}
A string giving the name of the executable binary for the Python
interpreter, on systems where this makes sense.
\end{datadesc}

\begin{funcdesc}{exit}{\optional{arg}}
Exit from Python.  This is implemented by raising the
\exception{SystemExit} exception, so cleanup actions specified by
finally clauses of \keyword{try} statements are honored, and it is
possible to intercept the exit attempt at an outer level.  The
optional argument \var{arg} can be an integer giving the exit status
(defaulting to zero), or another type of object.  If it is an integer,
zero is considered ``successful termination'' and any nonzero value is
considered ``abnormal termination'' by shells and the like.  Most
systems require it to be in the range 0-127, and produce undefined
results otherwise.  Some systems have a convention for assigning
specific meanings to specific exit codes, but these are generally
underdeveloped; Unix programs generally use 2 for command line syntax
errors and 1 for all other kind of errors.  If another type of object
is passed, \code{None} is equivalent to passing zero, and any other
object is printed to \code{sys.stderr} and results in an exit code of
1.  In particular, \code{sys.exit("some error message")} is a quick
way to exit a program when an error occurs.
\end{funcdesc}

\begin{datadesc}{exitfunc}
  This value is not actually defined by the module, but can be set by
  the user (or by a program) to specify a clean-up action at program
  exit.  When set, it should be a parameterless function.  This function
  will be called when the interpreter exits.  Only one function may be
  installed in this way; to allow multiple functions which will be called
  at termination, use the \refmodule{atexit} module.  Note: the exit function
  is not called when the program is killed by a signal, when a Python
  fatal internal error is detected, or when \code{os._exit()} is called.
\end{datadesc}

\begin{funcdesc}{getrefcount}{object}
Return the reference count of the \var{object}.  The count returned is
generally one higher than you might expect, because it includes the
(temporary) reference as an argument to \function{getrefcount()}.
\end{funcdesc}

\begin{funcdesc}{getrecursionlimit}{}
Return the current value of the recursion limit, the maximum depth of
the Python interpreter stack.  This limit prevents infinite recursion
from causing an overflow of the C stack and crashing Python.  It can
be set by \function{setrecursionlimit()}.
\end{funcdesc}

\begin{datadesc}{hexversion}
The version number encoded as a single integer.  This is guaranteed to
increase with each version, including proper support for
non-production releases.  For example, to test that the Python
interpreter is at least version 1.5.2, use:

\begin{verbatim}
if sys.hexversion >= 0x010502F0:
    # use some advanced feature
    ...
else:
    # use an alternative implementation or warn the user
    ...
\end{verbatim}

This is called \samp{hexversion} since it only really looks meaningful
when viewed as the result of passing it to the built-in
\function{hex()} function.  The \code{version_info} value may be used
for a more human-friendly encoding of the same information.
\versionadded{1.5.2}
\end{datadesc}

\begin{datadesc}{last_type}
\dataline{last_value}
\dataline{last_traceback}
These three variables are not always defined; they are set when an
exception is not handled and the interpreter prints an error message
and a stack traceback.  Their intended use is to allow an interactive
user to import a debugger module and engage in post-mortem debugging
without having to re-execute the command that caused the error.
(Typical use is \samp{import pdb; pdb.pm()} to enter the post-mortem
debugger; see the chapter ``The Python Debugger'' for more
information.)
\refstmodindex{pdb}

The meaning of the variables is the same
as that of the return values from \function{exc_info()} above.
(Since there is only one interactive thread, thread-safety is not a
concern for these variables, unlike for \code{exc_type} etc.)
\end{datadesc}

\begin{datadesc}{maxint}
The largest positive integer supported by Python's regular integer
type.  This is at least 2**31-1.  The largest negative integer is
\code{-maxint-1} -- the asymmetry results from the use of 2's
complement binary arithmetic.
\end{datadesc}

\begin{datadesc}{modules}
  This is a dictionary that maps module names to modules which have
  already been loaded.  This can be manipulated to force reloading of
  modules and other tricks.  Note that removing a module from this
  dictionary is \emph{not} the same as calling
  \function{reload()}\bifuncindex{reload} on the corresponding module
  object.
\end{datadesc}

\begin{datadesc}{path}
\indexiii{module}{search}{path}
  A list of strings that specifies the search path for modules.
  Initialized from the environment variable \envvar{PYTHONPATH}, or an
  installation-dependent default.  

The first item of this list, \code{path[0]}, is the 
directory containing the script that was used to invoke the Python 
interpreter.  If the script directory is not available (e.g.  if the 
interpreter is invoked interactively or if the script is read from 
standard input), \code{path[0]} is the empty string, which directs 
Python to search modules in the current directory first.  Notice that 
the script directory is inserted \emph{before} the entries inserted as 
a result of \envvar{PYTHONPATH}.  
\end{datadesc}

\begin{datadesc}{platform}
This string contains a platform identifier, e.g. \code{'sunos5'} or
\code{'linux1'}.  This can be used to append platform-specific
components to \code{path}, for instance. 
\end{datadesc}

\begin{datadesc}{prefix}
A string giving the site-specific directory prefix where the platform
independent Python files are installed; by default, this is the string
\code{'/usr/local'}.  This can be set at build time with the
\longprogramopt{prefix} argument to the
\program{configure} script.  The main collection of Python library
modules is installed in the directory \code{prefix +
'/lib/python\var{version}'} while the platform independent header
files (all except \file{config.h}) are stored in \code{prefix +
'/include/python\var{version}'}, where \var{version} is equal to
\code{version[:3]}.
\end{datadesc}

\begin{datadesc}{ps1}
\dataline{ps2}
\index{interpreter prompts}
\index{prompts, interpreter}
  Strings specifying the primary and secondary prompt of the
  interpreter.  These are only defined if the interpreter is in
  interactive mode.  Their initial values in this case are
  \code{'>>> '} and \code{'... '}.  If a non-string object is assigned
  to either variable, its \function{str()} is re-evaluated each time
  the interpreter prepares to read a new interactive command; this can
  be used to implement a dynamic prompt.
\end{datadesc}

\begin{funcdesc}{setcheckinterval}{interval}
Set the interpreter's ``check interval''.  This integer value
determines how often the interpreter checks for periodic things such
as thread switches and signal handlers.  The default is \code{10}, meaning
the check is performed every 10 Python virtual instructions.  Setting
it to a larger value may increase performance for programs using
threads.  Setting it to a value \code{<=} 0 checks every virtual instruction,
maximizing responsiveness as well as overhead.
\end{funcdesc}

\begin{funcdesc}{setprofile}{profilefunc}
  Set the system's profile function, which allows you to implement a
  Python source code profiler in Python.  See the chapter on the
  Python Profiler.  The system's profile function
  is called similarly to the system's trace function (see
  \function{settrace()}), but it isn't called for each executed line of
  code (only on call and return and when an exception occurs).  Also,
  its return value is not used, so it can just return \code{None}.
\end{funcdesc}
\index{profile function}
\index{profiler}

\begin{funcdesc}{setrecursionlimit}{limit}
Set the maximum depth of the Python interpreter stack to \var{limit}.
This limit prevents infinite recursion from causing an overflow of the
C stack and crashing Python.  

The highest possible limit is platform-dependent.  A user may need to
set the limit higher when she has a program that requires deep
recursion and a platform that supports a higher limit.  This should be
done with care, because a too-high limit can lead to a crash.
\end{funcdesc}

\begin{funcdesc}{settrace}{tracefunc}
  Set the system's trace function, which allows you to implement a
  Python source code debugger in Python.  See section ``How It Works''
  in the chapter on the Python Debugger.
\end{funcdesc}
\index{trace function}
\index{debugger}

\begin{datadesc}{stdin}
\dataline{stdout}
\dataline{stderr}
  File objects corresponding to the interpreter's standard input,
  output and error streams.  \code{stdin} is used for all
  interpreter input except for scripts but including calls to
  \function{input()}\bifuncindex{input} and
  \function{raw_input()}\bifuncindex{raw_input}.  \code{stdout} is used
  for the output of \keyword{print} and expression statements and for the
  prompts of \function{input()} and \function{raw_input()}.  The interpreter's
  own prompts and (almost all of) its error messages go to
  \code{stderr}.  \code{stdout} and \code{stderr} needn't
  be built-in file objects: any object is acceptable as long as it has
  a \method{write()} method that takes a string argument.  (Changing these
  objects doesn't affect the standard I/O streams of processes
  executed by \function{os.popen()}, \function{os.system()} or the
  \function{exec*()} family of functions in the \refmodule{os} module.)
\refstmodindex{os}
\end{datadesc}

\begin{datadesc}{__stdin__}
\dataline{__stdout__}
\dataline{__stderr__}
These objects contain the original values of \code{stdin},
\code{stderr} and \code{stdout} at the start of the program.  They are 
used during finalization, and could be useful to restore the actual
files to known working file objects in case they have been overwritten
with a broken object.
\end{datadesc}

\begin{datadesc}{tracebacklimit}
When this variable is set to an integer value, it determines the
maximum number of levels of traceback information printed when an
unhandled exception occurs.  The default is \code{1000}.  When set to
0 or less, all traceback information is suppressed and only the
exception type and value are printed.
\end{datadesc}

\begin{datadesc}{version}
A string containing the version number of the Python interpreter plus
additional information on the build number and compiler used.  It has
a value of the form \code{'\var{version} (\#\var{build_number},
\var{build_date}, \var{build_time}) [\var{compiler}]'}.  The first
three characters are used to identify the version in the installation
directories (where appropriate on each platform).  An example:

\begin{verbatim}
>>> import sys
>>> sys.version
'1.5.2 (#0 Apr 13 1999, 10:51:12) [MSC 32 bit (Intel)]'
\end{verbatim}
\end{datadesc}

\begin{datadesc}{version_info}
A tuple containing the five components of the version number:
\var{major}, \var{minor}, \var{micro}, \var{releaselevel}, and
\var{serial}.  All values except \var{releaselevel} are integers; the
release level is \code{'alpha'}, \code{'beta'},
\code{'candidate'}, or \code{'final'}.  The \code{version_info} value
corresponding to the Python version 2.0 is
\code{(2, 0, 0, 'final', 0)}.
\versionadded{2.0}
\end{datadesc}

\begin{datadesc}{winver}
The version number used to form registry keys on Windows platforms.
This is stored as string resource 1000 in the Python DLL.  The value
is normally the first three characters of \constant{version}.  It is
provided in the \module{sys} module for informational purposes;
modifying this value has no effect on the registry keys used by
Python.
Availability: Windows.
\end{datadesc}

\section{Built-in module \sectcode{types}}
\stmodindex{types}

\renewcommand{\indexsubitem}{(in module types)}

This module defines names for all object types that are used by the
standard Python interpreter (but not for the types defined by various
extension modules).  It is safe to use ``\code{from types import *}'' ---
the module does not export any other names besides the ones listed
here.  New names exported by future versions of this module will
all end in \code{Type}.

Typical use is for functions that do different things depending on
their argument types, like the following:

\begin{verbatim}
from types import *
def delete(list, item):
    if type(item) is IntType:
       del list[item]
    else:
       list.remove(item)
\end{verbatim}

The module defines the following names:

\begin{datadesc}{NoneType}
The type of \code{None}.
\end{datadesc}

\begin{datadesc}{TypeType}
The type of type objects (such as returned by \code{type()}).
\end{datadesc}

\begin{datadesc}{IntType}
The type of integers (e.g. \code{1}).
\end{datadesc}

\begin{datadesc}{LongType}
The type of long integers (e.g. \code{1L}).
\end{datadesc}

\begin{datadesc}{FloatType}
The type of floating point numbers (e.g. \code{1.0}).
\end{datadesc}

\begin{datadesc}{StringType}
The type of character strings (e.g. \code{'Spam'}).
\end{datadesc}

\begin{datadesc}{TupleType}
The type of tuples (e.g. \code{(1, 2, 3, 'Spam')}).
\end{datadesc}

\begin{datadesc}{ListType}
The type of lists (e.g. \code{[0, 1, 2, 3]}).
\end{datadesc}

\begin{datadesc}{DictType}
The type of dictionaries (e.g. \code{\{'Bacon': 1, 'Ham': 0\}}).
\end{datadesc}

\begin{datadesc}{DictionaryType}
An alternative name for \code{DictType}.
\end{datadesc}

\begin{datadesc}{FunctionType}
The type of user-defined functions and lambdas.
\end{datadesc}

\begin{datadesc}{LambdaType}
	An alternative name for \code{FunctionType}.
\end{datadesc}

\begin{datadesc}{CodeType}
The type for code objects such as returned by \code{compile()}.
\end{datadesc}

\begin{datadesc}{ClassType}
The type of user-defined classes.
\end{datadesc}

\begin{datadesc}{InstanceType}
The type of instances of user-defined classes.
\end{datadesc}

\begin{datadesc}{MethodType}
The type of methods of user-defined class instances.
\end{datadesc}

\begin{datadesc}{UnboundMethodType}
An alternative name for \code{MethodType}.
\end{datadesc}

\begin{datadesc}{BuiltinFunctionType}
The type of built-in functions like \code{len} or \code{sys.exit}.
\end{datadesc}

\begin{datadesc}{BuiltinMethodType}
An alternative name for \code{BuiltinFunction}.
\end{datadesc}

\begin{datadesc}{ModuleType}
The type of modules.
\end{datadesc}

\begin{datadesc}{FileType}
The type of open file objects such as \code{sys.stdout}.
\end{datadesc}

\begin{datadesc}{XRangeType}
The type of range objects returned by \code{xrange()}.
\end{datadesc}

\begin{datadesc}{TracebackType}
The type of traceback objects such as found in \code{sys.exc_traceback}.
\end{datadesc}

\begin{datadesc}{FrameType}
The type of frame objects such as found in \code{tb.tb_frame} if
\code{tb} is a traceback object.
\end{datadesc}
		% types is already taken :-(
\section{\module{UserDict} ---
         Class wrapper for dictionary objects}

\declaremodule{standard}{UserDict}
\modulesynopsis{Class wrapper for dictionary objects.}

\note{This module is available for backward compatibility only.  If
you are writing code that does not need to work with versions of
Python earlier than Python 2.2, please consider subclassing directly
from the built-in \class{dict} type.}

This module defines a class that acts as a wrapper around
dictionary objects.  It is a useful base class for
your own dictionary-like classes, which can inherit from
them and override existing methods or add new ones.  In this way one
can add new behaviors to dictionaries.

The module also defines a mixin defining all dictionary methods for
classes that already have a minimum mapping interface.  This greatly
simplifies writing classes that need to be substitutable for
dictionaries (such as the shelve module).

The \module{UserDict} module defines the \class{UserDict} class
and \class{DictMixin}:

\begin{classdesc}{UserDict}{\optional{initialdata}}
Class that simulates a dictionary.  The instance's
contents are kept in a regular dictionary, which is accessible via the
\member{data} attribute of \class{UserDict} instances.  If
\var{initialdata} is provided, \member{data} is initialized with its
contents; note that a reference to \var{initialdata} will not be kept, 
allowing it be used for other purposes.
\end{classdesc}

In addition to supporting the methods and operations of mappings (see
section \ref{typesmapping}), \class{UserDict} instances provide the
following attribute:

\begin{memberdesc}{data}
A real dictionary used to store the contents of the \class{UserDict}
class.
\end{memberdesc}

\begin{classdesc}{DictMixin}{}
Mixin defining all dictionary methods for classes that already have
a minimum dictionary interface including \method{__getitem__()},
\method{__setitem__()}, \method{__delitem__()}, and \method{keys()}.

This mixin should be used as a superclass.  Adding each of the
above methods adds progressively more functionality.  For instance,
defining all but \method{__delitem__} will preclude only \method{pop}
and \method{popitem} from the full interface.

In addition to the four base methods, progressively more efficiency
comes with defining \method{__contains__()}, \method{__iter__()}, and
\method{iteritems()}.

Since the mixin has no knowledge of the subclass constructor, it
does not define \method{__init__()} or \method{copy()}.
\end{classdesc}


\section{\module{UserList} ---
         Class wrapper for list objects}

\declaremodule{standard}{UserList}
\modulesynopsis{Class wrapper for list objects.}


\note{This module is available for backward compatibility only.  If
you are writing code that does not need to work with versions of
Python earlier than Python 2.2, please consider subclassing directly
from the built-in \class{list} type.}

This module defines a class that acts as a wrapper around
list objects.  It is a useful base class for
your own list-like classes, which can inherit from
them and override existing methods or add new ones.  In this way one
can add new behaviors to lists.

The \module{UserList} module defines the \class{UserList} class:

\begin{classdesc}{UserList}{\optional{list}}
Class that simulates a list.  The instance's
contents are kept in a regular list, which is accessible via the
\member{data} attribute of \class{UserList} instances.  The instance's
contents are initially set to a copy of \var{list}, defaulting to the
empty list \code{[]}.  \var{list} can be either a regular Python list,
or an instance of \class{UserList} (or a subclass).
\end{classdesc}

In addition to supporting the methods and operations of mutable
sequences (see section \ref{typesseq}), \class{UserList} instances
provide the following attribute:

\begin{memberdesc}{data}
A real Python list object used to store the contents of the
\class{UserList} class.
\end{memberdesc}

\strong{Subclassing requirements:}
Subclasses of \class{UserList} are expect to offer a constructor which
can be called with either no arguments or one argument.  List
operations which return a new sequence attempt to create an instance
of the actual implementation class.  To do so, it assumes that the
constructor can be called with a single parameter, which is a sequence
object used as a data source.

If a derived class does not wish to comply with this requirement, all
of the special methods supported by this class will need to be
overridden; please consult the sources for information about the
methods which need to be provided in that case.

\versionchanged[Python versions 1.5.2 and 1.6 also required that the
                constructor be callable with no parameters, and offer
                a mutable \member{data} attribute.  Earlier versions
                of Python did not attempt to create instances of the
                derived class]{2.0}


\section{\module{UserString} ---
         Class wrapper for string objects}

\declaremodule{standard}{UserString}
\modulesynopsis{Class wrapper for string objects.}
\moduleauthor{Peter Funk}{pf@artcom-gmbh.de}
\sectionauthor{Peter Funk}{pf@artcom-gmbh.de}

\note{This \class{UserString} class from this module is available for
backward compatibility only.  If you are writing code that does not
need to work with versions of Python earlier than Python 2.2, please
consider subclassing directly from the built-in \class{str} type
instead of using \class{UserString} (there is no built-in equivalent
to \class{MutableString}).}

This module defines a class that acts as a wrapper around string
objects.  It is a useful base class for your own string-like classes,
which can inherit from them and override existing methods or add new
ones.  In this way one can add new behaviors to strings.

It should be noted that these classes are highly inefficient compared
to real string or Unicode objects; this is especially the case for
\class{MutableString}.

The \module{UserString} module defines the following classes:

\begin{classdesc}{UserString}{\optional{sequence}}
Class that simulates a string or a Unicode string
object.  The instance's content is kept in a regular string or Unicode
string object, which is accessible via the \member{data} attribute of
\class{UserString} instances.  The instance's contents are initially
set to a copy of \var{sequence}.  \var{sequence} can be either a
regular Python string or Unicode string, an instance of
\class{UserString} (or a subclass) or an arbitrary sequence which can
be converted into a string using the built-in \function{str()} function.
\end{classdesc}

\begin{classdesc}{MutableString}{\optional{sequence}}
This class is derived from the \class{UserString} above and redefines
strings to be \emph{mutable}.  Mutable strings can't be used as
dictionary keys, because dictionaries require \emph{immutable} objects as
keys.  The main intention of this class is to serve as an educational
example for inheritance and necessity to remove (override) the
\method{__hash__()} method in order to trap attempts to use a
mutable object as dictionary key, which would be otherwise very
error prone and hard to track down.
\end{classdesc}

In addition to supporting the methods and operations of string and
Unicode objects (see section \ref{string-methods}, ``String
Methods''), \class{UserString} instances provide the following
attribute:

\begin{memberdesc}{data}
A real Python string or Unicode object used to store the content of the
\class{UserString} class.
\end{memberdesc}

% Contributed by Skip Montanaro, from the module's doc strings.

\section{Built-in Module \sectcode{operator}}
\bimodindex{operator}

The \code{operator} module exports a set of functions implemented in C
corresponding to the intrinsic operators of Python.  For example,
\code{operator.add(x, y)} is equivalent to the expression \code{x+y}.  The
function names are those used for special class methods; variants without
leading and trailing \samp{__} are also provided for convenience.

The \code{operator} module defines the following functions:

\setindexsubitem{(in module operator)}

\begin{funcdesc}{add}{a, b}
Return \var{a} \code{+} \var{b}, for \var{a} and \var{b} numbers.
\end{funcdesc}

\begin{funcdesc}{__add__}{a, b}
Return \var{a} \code{+} \var{b}, for \var{a} and \var{b} numbers.
\end{funcdesc}

\begin{funcdesc}{sub}{a, b}
Return \var{a} \code{-} \var{b}.
\end{funcdesc}

\begin{funcdesc}{__sub__}{a, b}
Return \var{a} \code{-} \var{b}.
\end{funcdesc}

\begin{funcdesc}{mul}{a, b}
Return \var{a} \code{*} \var{b}, for \var{a} and \var{b} numbers.
\end{funcdesc}

\begin{funcdesc}{__mul__}{a, b}
Return \var{a} \code{*} \var{b}, for \var{a} and \var{b} numbers.
\end{funcdesc}

\begin{funcdesc}{div}{a, b}
Return \var{a} \code{/} \var{b}.
\end{funcdesc}

\begin{funcdesc}{__div__}{a, b}
Return \var{a} \code{/} \var{b}.
\end{funcdesc}

\begin{funcdesc}{mod}{a, b}
Return \var{a} \code{\%} \var{b}.
\end{funcdesc}

\begin{funcdesc}{__mod__}{a, b}
Return \var{a} \code{\%} \var{b}.
\end{funcdesc}

\begin{funcdesc}{neg}{o}
Return \var{o} negated.
\end{funcdesc}

\begin{funcdesc}{__neg__}{o}
Return \var{o} negated.
\end{funcdesc}

\begin{funcdesc}{pos}{o}
Return \var{o} positive.
\end{funcdesc}

\begin{funcdesc}{__pos__}{o}
Return \var{o} positive.
\end{funcdesc}

\begin{funcdesc}{abs}{o}
Return the absolute value of \var{o}.
\end{funcdesc}

\begin{funcdesc}{__abs__}{o}
Return the absolute value of \var{o}.
\end{funcdesc}

\begin{funcdesc}{inv}{o}
Return the inverse of \var{o}.
\end{funcdesc}

\begin{funcdesc}{__inv__}{o}
Return the inverse of \var{o}.
\end{funcdesc}

\begin{funcdesc}{lshift}{a, b}
Return \var{a} shifted left by \var{b}.
\end{funcdesc}

\begin{funcdesc}{__lshift__}{a, b}
Return \var{a} shifted left by \var{b}.
\end{funcdesc}

\begin{funcdesc}{rshift}{a, b}
Return \var{a} shifted right by \var{b}.
\end{funcdesc}

\begin{funcdesc}{__rshift__}{a, b}
Return \var{a} shifted right by \var{b}.
\end{funcdesc}

\begin{funcdesc}{and_}{a, b}
Return the bitwise and of \var{a} and \var{b}.
\end{funcdesc}

\begin{funcdesc}{__and__}{a, b}
Return the bitwise and of \var{a} and \var{b}.
\end{funcdesc}

\begin{funcdesc}{or_}{a, b}
Return the bitwise or of \var{a} and \var{b}.
\end{funcdesc}

\begin{funcdesc}{__or__}{a, b}
Return the bitwise or of \var{a} and \var{b}.
\end{funcdesc}

\begin{funcdesc}{concat}{a, b}
Return \var{a} \code{+} \var{b} for \var{a} and \var{b} sequences.
\end{funcdesc}

\begin{funcdesc}{__concat__}{a, b}
Return \var{a} \code{+} \var{b} for \var{a} and \var{b} sequences.
\end{funcdesc}

\begin{funcdesc}{repeat}{a, b}
Return \var{a} \code{*} \var{b} where \var{a} is a sequence and
\var{b} is an integer.
\end{funcdesc}

\begin{funcdesc}{__repeat__}{a, b}
Return \var{a} \code{*} \var{b} where \var{a} is a sequence and
\var{b} is an integer.
\end{funcdesc}

\begin{funcdesc}{getitem}{a, b}
Return the value of \var{a} at index \var{b}.
\end{funcdesc}

\begin{funcdesc}{__getitem__}{a, b}
Return the value of \var{a} at index \var{b}.
\end{funcdesc}

\begin{funcdesc}{setitem}{a, b, c}
Set the value of \var{a} at index \var{b} to \var{c}.
\end{funcdesc}

\begin{funcdesc}{__setitem__}{a, b, c}
Set the value of \var{a} at index \var{b} to \var{c}.
\end{funcdesc}

\begin{funcdesc}{delitem}{a, b}
Remove the value of \var{a} at index \var{b}.
\end{funcdesc}

\begin{funcdesc}{__delitem__}{a, b}
Remove the value of \var{a} at index \var{b}.
\end{funcdesc}

\begin{funcdesc}{getslice}{a, b, c}
Return the slice of \var{a} from index \var{b} to index \var{c}\code{-1}.
\end{funcdesc}

\begin{funcdesc}{__getslice__}{a, b, c}
Return the slice of \var{a} from index \var{b} to index \var{c}\code{-1}.
\end{funcdesc}

\begin{funcdesc}{setslice}{a, b, c, v}
Set the slice of \var{a} from index \var{b} to index \var{c}\code{-1} to the
sequence \var{v}.
\end{funcdesc}

\begin{funcdesc}{__setslice__}{a, b, c, v}
Set the slice of \var{a} from index \var{b} to index \var{c}\code{-1} to the
sequence \var{v}.
\end{funcdesc}

\begin{funcdesc}{delslice}{a, b, c}
Delete the slice of \var{a} from index \var{b} to index \var{c}\code{-1}.
\end{funcdesc}

\begin{funcdesc}{__delslice__}{a, b, c}
Delete the slice of \var{a} from index \var{b} to index \var{c}\code{-1}.
\end{funcdesc}

Example: Build a dictionary that maps the ordinals from \code{0} to
\code{256} to their character equivalents.

\begin{verbatim}
>>> import operator
>>> d = {}
>>> keys = range(256)
>>> vals = map(chr, keys)
>>> map(operator.setitem, [d]*len(keys), keys, vals)
\end{verbatim}

\section{Standard Module \sectcode{traceback}}
\label{module-traceback}
\stmodindex{traceback}

\renewcommand{\indexsubitem}{(in module traceback)}

This module provides a standard interface to format and print stack
traces of Python programs.  It exactly mimics the behavior of the
Python interpreter when it prints a stack trace.  This is useful when
you want to print stack traces under program control, e.g. in a
``wrapper'' around the interpreter.

The module uses traceback objects --- this is the object type
that is stored in the variables \code{sys.exc_traceback} and
\code{sys.last_traceback}.

The module defines the following functions:

\begin{funcdesc}{print_tb}{traceback\optional{\, limit}}
Print up to \var{limit} stack trace entries from \var{traceback}.  If
\var{limit} is omitted or \code{None}, all entries are printed.
\end{funcdesc}

\begin{funcdesc}{extract_tb}{traceback\optional{\, limit}}
Return a list of up to \var{limit} ``pre-processed'' stack trace
entries extracted from \var{traceback}.  It is useful for alternate
formatting of stack traces.  If \var{limit} is omitted or \code{None},
all entries are extracted.  A ``pre-processed'' stack trace entry is a
quadruple (\var{filename}, \var{line number}, \var{function name},
\var{line text}) representing the information that is usually printed
for a stack trace.  The \var{line text} is a string with leading and
trailing whitespace stripped; if the source is not available it is
\code{None}.
\end{funcdesc}

\begin{funcdesc}{print_exception}{type\, value\, traceback\optional{\, limit}}
Print exception information and up to \var{limit} stack trace entries
from \var{traceback}.  This differs from \code{print_tb} in the
following ways: (1) if \var{traceback} is not \code{None}, it prints a
header ``\code{Traceback (innermost last):}''; (2) it prints the
exception \var{type} and \var{value} after the stack trace; (3) if
\var{type} is \code{SyntaxError} and \var{value} has the appropriate
format, it prints the line where the syntax error occurred with a
caret indication the approximate position of the error.
\end{funcdesc}

\begin{funcdesc}{print_exc}{\optional{limit}}
This is a shorthand for \code{print_exception(sys.exc_type,}
\code{sys.exc_value,} \code{sys.exc_traceback,} \code{limit)}.
\end{funcdesc}

\begin{funcdesc}{print_last}{\optional{limit}}
This is a shorthand for \code{print_exception(sys.last_type,}
\code{sys.last_value,} \code{sys.last_traceback,} \code{limit)}.
\end{funcdesc}

\section{\module{pickle} ---
         Python object serialization}

\declaremodule{standard}{pickle}
\modulesynopsis{Convert Python objects to streams of bytes and back.}
% Substantial improvements by Jim Kerr <jbkerr@sr.hp.com>.

\index{persistency}
\indexii{persistent}{objects}
\indexii{serializing}{objects}
\indexii{marshalling}{objects}
\indexii{flattening}{objects}
\indexii{pickling}{objects}


The \module{pickle} module implements a basic but powerful algorithm
for ``pickling'' (a.k.a.\ serializing, marshalling or flattening)
nearly arbitrary Python objects.  This is the act of converting
objects to a stream of bytes (and back: ``unpickling'').  This is a
more primitive notion than persistency --- although \module{pickle}
reads and writes file objects, it does not handle the issue of naming
persistent objects, nor the (even more complicated) area of concurrent
access to persistent objects.  The \module{pickle} module can
transform a complex object into a byte stream and it can transform the
byte stream into an object with the same internal structure.  The most
obvious thing to do with these byte streams is to write them onto a
file, but it is also conceivable to send them across a network or
store them in a database.  The module
\refmodule{shelve}\refstmodindex{shelve} provides a simple interface
to pickle and unpickle objects on DBM-style database files.


\strong{Note:} The \module{pickle} module is rather slow.  A
reimplementation of the same algorithm in C, which is up to 1000 times
faster, is available as the
\refmodule{cPickle}\refbimodindex{cPickle} module.  This has the same
interface except that \class{Pickler} and \class{Unpickler} are
factory functions, not classes (so they cannot be used as base classes
for inheritance).

Although the \module{pickle} module can use the built-in module
\refmodule{marshal}\refbimodindex{marshal} internally, it differs from 
\refmodule{marshal} in the way it handles certain kinds of data:

\begin{itemize}

\item Recursive objects (objects containing references to themselves): 
      \module{pickle} keeps track of the objects it has already
      serialized, so later references to the same object won't be
      serialized again.  (The \refmodule{marshal} module breaks for
      this.)

\item Object sharing (references to the same object in different
      places):  This is similar to self-referencing objects;
      \module{pickle} stores the object once, and ensures that all
      other references point to the master copy.  Shared objects
      remain shared, which can be very important for mutable objects.

\item User-defined classes and their instances:  \refmodule{marshal}
      does not support these at all, but \module{pickle} can save
      and restore class instances transparently.  The class definition 
      must be importable and live in the same module as when the
      object was stored.

\end{itemize}

The data format used by \module{pickle} is Python-specific.  This has
the advantage that there are no restrictions imposed by external
standards such as
XDR\index{XDR}\index{External Data Representation} (which can't
represent pointer sharing); however it means that non-Python programs
may not be able to reconstruct pickled Python objects.

By default, the \module{pickle} data format uses a printable \ASCII{}
representation.  This is slightly more voluminous than a binary
representation.  The big advantage of using printable \ASCII{} (and of
some other characteristics of \module{pickle}'s representation) is that
for debugging or recovery purposes it is possible for a human to read
the pickled file with a standard text editor.

A binary format, which is slightly more efficient, can be chosen by
specifying a nonzero (true) value for the \var{bin} argument to the
\class{Pickler} constructor or the \function{dump()} and \function{dumps()}
functions.  The binary format is not the default because of backwards
compatibility with the Python 1.4 pickle module.  In a future version,
the default may change to binary.

The \module{pickle} module doesn't handle code objects, which the
\refmodule{marshal}\refbimodindex{marshal} module does.  I suppose
\module{pickle} could, and maybe it should, but there's probably no
great need for it right now (as long as \refmodule{marshal} continues
to be used for reading and writing code objects), and at least this
avoids the possibility of smuggling Trojan horses into a program.

For the benefit of persistency modules written using \module{pickle}, it
supports the notion of a reference to an object outside the pickled
data stream.  Such objects are referenced by a name, which is an
arbitrary string of printable \ASCII{} characters.  The resolution of
such names is not defined by the \module{pickle} module --- the
persistent object module will have to implement a method
\method{persistent_load()}.  To write references to persistent objects,
the persistent module must define a method \method{persistent_id()} which
returns either \code{None} or the persistent ID of the object.

There are some restrictions on the pickling of class instances.

First of all, the class must be defined at the top level in a module.
Furthermore, all its instance variables must be picklable.

\setindexsubitem{(pickle protocol)}

When a pickled class instance is unpickled, its \method{__init__()} method
is normally \emph{not} invoked.  \strong{Note:} This is a deviation
from previous versions of this module; the change was introduced in
Python 1.5b2.  The reason for the change is that in many cases it is
desirable to have a constructor that requires arguments; it is a
(minor) nuisance to have to provide a \method{__getinitargs__()} method.

If it is desirable that the \method{__init__()} method be called on
unpickling, a class can define a method \method{__getinitargs__()},
which should return a \emph{tuple} containing the arguments to be
passed to the class constructor (\method{__init__()}).  This method is
called at pickle time; the tuple it returns is incorporated in the
pickle for the instance.
\withsubitem{(copy protocol)}{\ttindex{__getinitargs__()}}
\withsubitem{(instance constructor)}{\ttindex{__init__()}}

Classes can further influence how their instances are pickled --- if
the class
\withsubitem{(copy protocol)}{
  \ttindex{__getstate__()}\ttindex{__setstate__()}}
\withsubitem{(instance attribute)}{
  \ttindex{__dict__}}
defines the method \method{__getstate__()}, it is called and the return
state is pickled as the contents for the instance, and if the class
defines the method \method{__setstate__()}, it is called with the
unpickled state.  (Note that these methods can also be used to
implement copying class instances.)  If there is no
\method{__getstate__()} method, the instance's \member{__dict__} is
pickled.  If there is no \method{__setstate__()} method, the pickled
object must be a dictionary and its items are assigned to the new
instance's dictionary.  (If a class defines both \method{__getstate__()}
and \method{__setstate__()}, the state object needn't be a dictionary
--- these methods can do what they want.)  This protocol is also used
by the shallow and deep copying operations defined in the
\refmodule{copy}\refstmodindex{copy} module.

Note that when class instances are pickled, their class's code and
data are not pickled along with them.  Only the instance data are
pickled.  This is done on purpose, so you can fix bugs in a class or
add methods and still load objects that were created with an earlier
version of the class.  If you plan to have long-lived objects that
will see many versions of a class, it may be worthwhile to put a version
number in the objects so that suitable conversions can be made by the
class's \method{__setstate__()} method.

When a class itself is pickled, only its name is pickled --- the class
definition is not pickled, but re-imported by the unpickling process.
Therefore, the restriction that the class must be defined at the top
level in a module applies to pickled classes as well.

\setindexsubitem{(in module pickle)}

The interface can be summarized as follows.

To pickle an object \code{x} onto a file \code{f}, open for writing:

\begin{verbatim}
p = pickle.Pickler(f)
p.dump(x)
\end{verbatim}

A shorthand for this is:

\begin{verbatim}
pickle.dump(x, f)
\end{verbatim}

To unpickle an object \code{x} from a file \code{f}, open for reading:

\begin{verbatim}
u = pickle.Unpickler(f)
x = u.load()
\end{verbatim}

A shorthand is:

\begin{verbatim}
x = pickle.load(f)
\end{verbatim}

The \class{Pickler} class only calls the method \code{f.write()} with a
\withsubitem{(class in pickle)}{\ttindex{Unpickler}\ttindex{Pickler}}
string argument.  The \class{Unpickler} calls the methods \code{f.read()}
(with an integer argument) and \code{f.readline()} (without argument),
both returning a string.  It is explicitly allowed to pass non-file
objects here, as long as they have the right methods.

The constructor for the \class{Pickler} class has an optional second
argument, \var{bin}.  If this is present and true, the binary
pickle format is used; if it is absent or false, the (less efficient,
but backwards compatible) text pickle format is used.  The
\class{Unpickler} class does not have an argument to distinguish
between binary and text pickle formats; it accepts either format.

The following types can be pickled:

\begin{itemize}

\item \code{None}

\item integers, long integers, floating point numbers

\item normal and Unicode strings

\item tuples, lists and dictionaries containing only picklable objects

\item functions defined at the top level of a module (by name
      reference, not storage of the implementation)

\item built-in functions

\item classes that are defined at the top level in a module

\item instances of such classes whose \member{__dict__} or
\method{__setstate__()} is picklable

\end{itemize}

Attempts to pickle unpicklable objects will raise the
\exception{PicklingError} exception; when this happens, an unspecified
number of bytes may have been written to the file.

It is possible to make multiple calls to the \method{dump()} method of
the same \class{Pickler} instance.  These must then be matched to the
same number of calls to the \method{load()} method of the
corresponding \class{Unpickler} instance.  If the same object is
pickled by multiple \method{dump()} calls, the \method{load()} will all
yield references to the same object.  \emph{Warning}: this is intended
for pickling multiple objects without intervening modifications to the
objects or their parts.  If you modify an object and then pickle it
again using the same \class{Pickler} instance, the object is not
pickled again --- a reference to it is pickled and the
\class{Unpickler} will return the old value, not the modified one.
(There are two problems here: (a) detecting changes, and (b)
marshalling a minimal set of changes.  I have no answers.  Garbage
Collection may also become a problem here.)

Apart from the \class{Pickler} and \class{Unpickler} classes, the
module defines the following functions, and an exception:

\begin{funcdesc}{dump}{object, file\optional{, bin}}
Write a pickled representation of \var{obect} to the open file object
\var{file}.  This is equivalent to
\samp{Pickler(\var{file}, \var{bin}).dump(\var{object})}.
If the optional \var{bin} argument is present and nonzero, the binary
pickle format is used; if it is zero or absent, the (less efficient)
text pickle format is used.
\end{funcdesc}

\begin{funcdesc}{load}{file}
Read a pickled object from the open file object \var{file}.  This is
equivalent to \samp{Unpickler(\var{file}).load()}.
\end{funcdesc}

\begin{funcdesc}{dumps}{object\optional{, bin}}
Return the pickled representation of the object as a string, instead
of writing it to a file.  If the optional \var{bin} argument is
present and nonzero, the binary pickle format is used; if it is zero
or absent, the (less efficient) text pickle format is used.
\end{funcdesc}

\begin{funcdesc}{loads}{string}
Read a pickled object from a string instead of a file.  Characters in
the string past the pickled object's representation are ignored.
\end{funcdesc}

\begin{excdesc}{PicklingError}
This exception is raised when an unpicklable object is passed to
\method{Pickler.dump()}.
\end{excdesc}


\begin{seealso}
  \seemodule[copyreg]{copy_reg}{pickle interface constructor
                                registration}

  \seemodule{shelve}{indexed databases of objects; uses \module{pickle}}

  \seemodule{copy}{shallow and deep object copying}

  \seemodule{marshal}{high-performance serialization of built-in types}
\end{seealso}


\subsection{Example \label{pickle-example}}

Here's a simple example of how to modify pickling behavior for a
class.  The \class{TextReader} class opens a text file, and returns
the line number and line contents each time its \method{readline()}
method is called. If a \class{TextReader} instance is pickled, all
attributes \emph{except} the file object member are saved. When the
instance is unpickled, the file is reopened, and reading resumes from
the last location. The \method{__setstate__()} and
\method{__getstate__()} methods are used to implement this behavior.

\begin{verbatim}
# illustrate __setstate__ and __getstate__  methods
# used in pickling.

class TextReader:
    "Print and number lines in a text file."
    def __init__(self,file):
        self.file = file
        self.fh = open(file,'r')
        self.lineno = 0

    def readline(self):
        self.lineno = self.lineno + 1
        line = self.fh.readline()
        if not line:
            return None
        return "%d: %s" % (self.lineno,line[:-1])

    # return data representation for pickled object
    def __getstate__(self):
        odict = self.__dict__    # get attribute dictionary
        del odict['fh']          # remove filehandle entry
        return odict

    # restore object state from data representation generated 
    # by __getstate__
    def __setstate__(self,dict):
        fh = open(dict['file'])  # reopen file
        count = dict['lineno']   # read from file...
        while count:             # until line count is restored
            fh.readline()
            count = count - 1
        dict['fh'] = fh          # create filehandle entry
        self.__dict__ = dict     # make dict our attribute dictionary
\end{verbatim}

A sample usage might be something like this:

\begin{verbatim}
>>> import TextReader
>>> obj = TextReader.TextReader("TextReader.py")
>>> obj.readline()
'1: #!/usr/local/bin/python'
>>> # (more invocations of obj.readline() here)
... obj.readline()
'7: class TextReader:'
>>> import pickle
>>> pickle.dump(obj,open('save.p','w'))

  (start another Python session)

>>> import pickle
>>> reader = pickle.load(open('save.p'))
>>> reader.readline()
'8:     "Print and number lines in a text file."'
\end{verbatim}


\section{\module{cPickle} ---
         Alternate implementation of \module{pickle}}

\declaremodule{builtin}{cPickle}
\modulesynopsis{Faster version of \refmodule{pickle}, but not subclassable.}
\moduleauthor{Jim Fulton}{jfulton@digicool.com}
\sectionauthor{Fred L. Drake, Jr.}{fdrake@acm.org}


The \module{cPickle} module provides a similar interface and identical
functionality as the \refmodule{pickle}\refstmodindex{pickle} module,
but can be up to 1000 times faster since it is implemented in C.  The
only other important difference to note is that \function{Pickler()}
and \function{Unpickler()} are functions and not classes, and so
cannot be subclassed.  This should not be an issue in most cases.

The format of the pickle data is identical to that produced using the
\refmodule{pickle} module, so it is possible to use \refmodule{pickle} and
\module{cPickle} interchangably with existing pickles.

(Since the pickle data format is actually a tiny stack-oriented
programming language, and there are some freedoms in the encodings of
certain objects, it's possible that the two modules produce different
pickled data for the same input objects; however they will always be
able to read each others pickles back in.)

\section{\module{copy_reg} ---
         Register \module{pickle} support functions}

\declaremodule[copyreg]{standard}{copy_reg}
\modulesynopsis{Register \module{pickle} support functions.}


The \module{copy_reg} module provides support for the
\refmodule{pickle}\refstmodindex{pickle}\ and
\refmodule{cPickle}\refbimodindex{cPickle}\ modules.  The
\refmodule{copy}\refstmodindex{copy}\ module is likely to use this in the
future as well.  It provides configuration information about object
constructors which are not classes.  Such constructors may be factory
functions or class instances.


\begin{funcdesc}{constructor}{object}
  Declares \var{object} to be a valid constructor.  If \var{object} is
  not callable (and hence not valid as a constructor), raises
  \exception{TypeError}.
\end{funcdesc}

\begin{funcdesc}{pickle}{type, function\optional{, constructor}}
  Declares that \var{function} should be used as a ``reduction''
  function for objects of type \var{type}; \var{type} must not be a
  ``classic'' class object.  (Classic classes are handled differently;
  see the documentation for the \refmodule{pickle} module for
  details.)  \var{function} should return either a string or a tuple
  containing two or three elements.

  The optional \var{constructor} parameter, if provided, is a
  callable object which can be used to reconstruct the object when
  called with the tuple of arguments returned by \var{function} at
  pickling time.  \exception{TypeError} will be raised if
  \var{object} is a class or \var{constructor} is not callable.

  See the \refmodule{pickle} module for more
  details on the interface expected of \var{function} and
  \var{constructor}.
\end{funcdesc}

\section{\module{shelve} ---
         Python object persistence}

\declaremodule{standard}{shelve}
\modulesynopsis{Python object persistence.}


A ``shelf'' is a persistent, dictionary-like object.  The difference
with ``dbm'' databases is that the values (not the keys!) in a shelf
can be essentially arbitrary Python objects --- anything that the
\refmodule{pickle} module can handle.  This includes most class
instances, recursive data types, and objects containing lots of shared 
sub-objects.  The keys are ordinary strings.
\refstmodindex{pickle}

To summarize the interface (\code{key} is a string, \code{data} is an
arbitrary object):

\begin{verbatim}
import shelve

d = shelve.open(filename) # open -- file may get suffix added by low-level
                          # library

d[key] = data   # store data at key (overwrites old data if
                # using an existing key)
data = d[key]   # retrieve data at key (raise KeyError if no
                # such key)
del d[key]      # delete data stored at key (raises KeyError
                # if no such key)
flag = d.has_key(key)   # true if the key exists
list = d.keys() # a list of all existing keys (slow!)

d.close()       # close it
\end{verbatim}

In addition to the above, shelve supports all methods that are
supported by dictionaries.  This eases the transition from dictionary
based scripts to those requiring persistent storage.

Restrictions:

\begin{itemize}

\item
The choice of which database package will be used
(such as \refmodule{dbm} or \refmodule{gdbm}) depends on which interface
is available.  Therefore it is not safe to open the database directly
using \refmodule{dbm}.  The database is also (unfortunately) subject
to the limitations of \refmodule{dbm}, if it is used --- this means
that (the pickled representation of) the objects stored in the
database should be fairly small, and in rare cases key collisions may
cause the database to refuse updates.
\refbimodindex{dbm}
\refbimodindex{gdbm}

\item
Depending on the implementation, closing a persistent dictionary may
or may not be necessary to flush changes to disk.  The \method{__del__}
method of the \class{Shelf} class calls the \method{close} method, so the
programmer generally need not do this explicitly.

\item
The \module{shelve} module does not support \emph{concurrent} read/write
access to shelved objects.  (Multiple simultaneous read accesses are
safe.)  When a program has a shelf open for writing, no other program
should have it open for reading or writing.  \UNIX{} file locking can
be used to solve this, but this differs across \UNIX{} versions and
requires knowledge about the database implementation used.

\end{itemize}

\begin{classdesc}{Shelf}{dict\optional{, binary=False}}
A subclass of \class{UserDict.DictMixin} which stores pickled values in the
\var{dict} object.  If the \var{binary} parameter is \code{True}, binary
pickles will be used.  This can provide much more compact storage than plain
text pickles, depending on the nature of the objects stored in the database.
\end{classdesc}

\begin{classdesc}{BsdDbShelf}{dict\optional{, binary=False}}
A subclass of \class{Shelf} which exposes \method{first}, \method{next},
\method{previous}, \method{last} and \method{set_location} which are
available in the \module{bsddb} module but not in other database modules.
The \var{dict} object passed to the constructor must support those methods.
This is generally accomplished by calling one of \function{bsddb.hashopen},
\function{bsddb.btopen} or \function{bsddb.rnopen}.  The optional
\var{binary} parameter has the same interpretation as for the \class{Shelf}
class. 
\end{classdesc}

\begin{classdesc}{DbfilenameShelf}{filename\optional{, flag='c'\optional{, binary=False}}}
A subclass of \class{Shelf} which accepts a \var{filename} instead of a dict-like
object.  The underlying file will be opened using \function{anydbm.open}.
By default, the file will be created and opened for both read and write.
The optional \var{binary} parameter has the same interpretation as for the
\class{Shelf} class.
\end{classdesc}

\begin{seealso}
  \seemodule{anydbm}{Generic interface to \code{dbm}-style databases.}
  \seemodule{bsddb}{BSD \code{db} database interface.}
  \seemodule{dbhash}{Thin layer around the \module{bsddb} which provides an
  \function{open} function like the other database modules.}
  \seemodule{dbm}{Standard \UNIX{} database interface.}
  \seemodule{dumbdbm}{Portable implementation of the \code{dbm} interface.}
  \seemodule{gdbm}{GNU database interface, based on the \code{dbm} interface.}
  \seemodule{pickle}{Object serialization used by \module{shelve}.}
  \seemodule{cPickle}{High-performance version of \refmodule{pickle}.}
\end{seealso}

\section{\module{copy} ---
         Shallow and deep copy operations}

\declaremodule{standard}{copy}
\modulesynopsis{Shallow and deep copy operations.}


This module provides generic (shallow and deep) copying operations.
\withsubitem{(in copy)}{\ttindex{copy()}\ttindex{deepcopy()}}

Interface summary:

\begin{verbatim}
import copy

x = copy.copy(y)        # make a shallow copy of y
x = copy.deepcopy(y)    # make a deep copy of y
\end{verbatim}
%
For module specific errors, \exception{copy.error} is raised.

The difference between shallow and deep copying is only relevant for
compound objects (objects that contain other objects, like lists or
class instances):

\begin{itemize}

\item
A \emph{shallow copy} constructs a new compound object and then (to the
extent possible) inserts \emph{references} into it to the objects found
in the original.

\item
A \emph{deep copy} constructs a new compound object and then,
recursively, inserts \emph{copies} into it of the objects found in the
original.

\end{itemize}

Two problems often exist with deep copy operations that don't exist
with shallow copy operations:

\begin{itemize}

\item
Recursive objects (compound objects that, directly or indirectly,
contain a reference to themselves) may cause a recursive loop.

\item
Because deep copy copies \emph{everything} it may copy too much,
e.g., administrative data structures that should be shared even
between copies.

\end{itemize}

The \function{deepcopy()} function avoids these problems by:

\begin{itemize}

\item
keeping a ``memo'' dictionary of objects already copied during the current
copying pass; and

\item
letting user-defined classes override the copying operation or the
set of components copied.

\end{itemize}

This version does not copy types like module, class, function, method,
stack trace, stack frame, file, socket, window, array, or any similar
types.

Classes can use the same interfaces to control copying that they use
to control pickling: they can define methods called
\method{__getinitargs__()}, \method{__getstate__()} and
\method{__setstate__()}.  See the description of module
\refmodule{pickle}\refstmodindex{pickle} for information on these
methods.  The \module{copy} module does not use the
\refmodule[copyreg]{copy_reg} registration module.
\withsubitem{(copy protocol)}{\ttindex{__getinitargs__()}
  \ttindex{__getstate__()}\ttindex{__setstate__()}}

In order for a class to define its own copy implementation, it can
define special methods \method{__copy__()} and
\method{__deepcopy__()}.  The former is called to implement the
shallow copy operation; no additional arguments are passed.  The
latter is called to implement the deep copy operation; it is passed
one argument, the memo dictionary.  If the \method{__deepcopy__()}
implementation needs to make a deep copy of a component, it should
call the \function{deepcopy()} function with the component as first
argument and the memo dictionary as second argument.
\withsubitem{(copy protocol)}{\ttindex{__copy__()}\ttindex{__deepcopy__()}}

\begin{seealso}
\seemodule{pickle}{Discussion of the special methods used to
support object state retrieval and restoration.}
\end{seealso}

\section{\module{marshal} ---
         Alternate Python object serialization}
\declaremodule{builtin}{marshal}

\modulesynopsis{Convert Python objects to streams of bytes and back
(with different constraints).}


This module contains functions that can read and write Python
values in a binary format.  The format is specific to Python, but
independent of machine architecture issues (e.g., you can write a
Python value to a file on a PC, transport the file to a Sun, and read
it back there).  Details of the format are undocumented on purpose;
it may change between Python versions (although it rarely does).%
\footnote{The name of this module stems from a bit of terminology used
by the designers of Modula-3 (amongst others), who use the term
``marshalling'' for shipping of data around in a self-contained form.
Strictly speaking, ``to marshal'' means to convert some data from
internal to external form (in an RPC buffer for instance) and
``unmarshalling'' for the reverse process.}

This is not a general ``persistency'' module.  For general persistency
and transfer of Python objects through RPC calls, see the modules
\module{pickle} and \module{shelve}.  The \module{marshal} module exists
mainly to support reading and writing the ``pseudo-compiled'' code for
Python modules of \file{.pyc} files.
\refstmodindex{pickle}
\refstmodindex{shelve}
\obindex{code}

Not all Python object types are supported; in general, only objects
whose value is independent from a particular invocation of Python can
be written and read by this module.  The following types are supported:
\code{None}, integers, long integers, floating point numbers,
strings, tuples, lists, dictionaries, and code objects, where it
should be understood that tuples, lists and dictionaries are only
supported as long as the values contained therein are themselves
supported; and recursive lists and dictionaries should not be written
(they will cause infinite loops).

\strong{Caveat:} On machines where C's \code{long int} type has more than
32 bits (such as the DEC Alpha), it
is possible to create plain Python integers that are longer than 32
bits.  Since the current \module{marshal} module uses 32 bits to
transfer plain Python integers, such values are silently truncated.
This particularly affects the use of very long integer literals in
Python modules --- these will be accepted by the parser on such
machines, but will be silently be truncated when the module is read
from the \file{.pyc} instead.%
\footnote{A solution would be to refuse such literals in the parser,
since they are inherently non-portable.  Another solution would be to
let the \module{marshal} module raise an exception when an integer
value would be truncated.  At least one of these solutions will be
implemented in a future version.}

There are functions that read/write files as well as functions
operating on strings.

The module defines these functions:

\begin{funcdesc}{dump}{value, file}
  Write the value on the open file.  The value must be a supported
  type.  The file must be an open file object such as
  \code{sys.stdout} or returned by \function{open()} or
  \function{posix.popen()}.

  If the value has (or contains an object that has) an unsupported type,
  a \exception{ValueError} exception is raised --- but garbage data
  will also be written to the file.  The object will not be properly
  read back by \function{load()}.
\end{funcdesc}

\begin{funcdesc}{load}{file}
  Read one value from the open file and return it.  If no valid value
  is read, raise \exception{EOFError}, \exception{ValueError} or
  \exception{TypeError}.  The file must be an open file object.

  \strong{Warning:} If an object containing an unsupported type was
  marshalled with \function{dump()}, \function{load()} will substitute
  \code{None} for the unmarshallable type.
\end{funcdesc}

\begin{funcdesc}{dumps}{value}
  Return the string that would be written to a file by
  \code{dump(\var{value}, \var{file})}.  The value must be a supported
  type.  Raise a \exception{ValueError} exception if value has (or
  contains an object that has) an unsupported type.
\end{funcdesc}

\begin{funcdesc}{loads}{string}
  Convert the string to a value.  If no valid value is found, raise
  \exception{EOFError}, \exception{ValueError} or
  \exception{TypeError}.  Extra characters in the string are ignored.
\end{funcdesc}

\section{\module{imp} ---
         Access the \keyword{import} internals}

\declaremodule{builtin}{imp}
\modulesynopsis{Access the implementation of the \keyword{import} statement.}


This\stindex{import} module provides an interface to the mechanisms
used to implement the \keyword{import} statement.  It defines the
following constants and functions:


\begin{funcdesc}{get_magic}{}
\indexii{file}{byte-code}
Return the magic string value used to recognize byte-compiled code
files (\file{.pyc} files).  (This value may be different for each
Python version.)
\end{funcdesc}

\begin{funcdesc}{get_suffixes}{}
Return a list of triples, each describing a particular type of module.
Each triple has the form \code{(\var{suffix}, \var{mode},
\var{type})}, where \var{suffix} is a string to be appended to the
module name to form the filename to search for, \var{mode} is the mode
string to pass to the built-in \function{open()} function to open the
file (this can be \code{'r'} for text files or \code{'rb'} for binary
files), and \var{type} is the file type, which has one of the values
\constant{PY_SOURCE}, \constant{PY_COMPILED}, or
\constant{C_EXTENSION}, described below.
\end{funcdesc}

\begin{funcdesc}{find_module}{name\optional{, path}}
Try to find the module \var{name} on the search path \var{path}.  If
\var{path} is a list of directory names, each directory is searched
for files with any of the suffixes returned by \function{get_suffixes()}
above.  Invalid names in the list are silently ignored (but all list
items must be strings).  If \var{path} is omitted or \code{None}, the
list of directory names given by \code{sys.path} is searched, but
first it searches a few special places: it tries to find a built-in
module with the given name (\constant{C_BUILTIN}), then a frozen module
(\constant{PY_FROZEN}), and on some systems some other places are looked
in as well (on the Mac, it looks for a resource (\constant{PY_RESOURCE});
on Windows, it looks in the registry which may point to a specific
file).

If search is successful, the return value is a triple
\code{(\var{file}, \var{pathname}, \var{description})} where
\var{file} is an open file object positioned at the beginning,
\var{pathname} is the pathname of the
file found, and \var{description} is a triple as contained in the list
returned by \function{get_suffixes()} describing the kind of module found.
If the module does not live in a file, the returned \var{file} is
\code{None}, \var{filename} is the empty string, and the
\var{description} tuple contains empty strings for its suffix and
mode; the module type is as indicate in parentheses above.  If the
search is unsuccessful, \exception{ImportError} is raised.  Other
exceptions indicate problems with the arguments or environment.

This function does not handle hierarchical module names (names
containing dots).  In order to find \var{P}.\var{M}, that is, submodule
\var{M} of package \var{P}, use \function{find_module()} and
\function{load_module()} to find and load package \var{P}, and then use
\function{find_module()} with the \var{path} argument set to
\code{\var{P}.__path__}.  When \var{P} itself has a dotted name, apply
this recipe recursively.
\end{funcdesc}

\begin{funcdesc}{load_module}{name, file, filename, description}
Load a module that was previously found by \function{find_module()} (or by
an otherwise conducted search yielding compatible results).  This
function does more than importing the module: if the module was
already imported, it is equivalent to a
\function{reload()}\bifuncindex{reload}!  The \var{name} argument
indicates the full module name (including the package name, if this is
a submodule of a package).  The \var{file} argument is an open file,
and \var{filename} is the corresponding file name; these can be
\code{None} and \code{''}, respectively, when the module is not being
loaded from a file.  The \var{description} argument is a tuple, as
would be returned by \function{get_suffixes()}, describing what kind
of module must be loaded.

If the load is successful, the return value is the module object;
otherwise, an exception (usually \exception{ImportError}) is raised.

\strong{Important:} the caller is responsible for closing the
\var{file} argument, if it was not \code{None}, even when an exception
is raised.  This is best done using a \keyword{try}
... \keyword{finally} statement.
\end{funcdesc}

\begin{funcdesc}{new_module}{name}
Return a new empty module object called \var{name}.  This object is
\emph{not} inserted in \code{sys.modules}.
\end{funcdesc}

\begin{funcdesc}{lock_held}{}
Return \code{True} if the import lock is currently held, else \code{False}.
On platforms without threads, always return \code{False}.

On platforms with threads, a thread executing an import holds an internal
lock until the import is complete.
This lock blocks other threads from doing an import until the original
import completes, which in turn prevents other threads from seeing
incomplete module objects constructed by the original thread while in
the process of completing its import (and the imports, if any,
triggered by that).
\end{funcdesc}

\begin{funcdesc}{acquire_lock}{}
Acquires the interpreter's import lock for the current thread.  This lock
should be used by import hooks to ensure thread-safety when importing modules.
On platforms without threads, this function does nothing.
\versionadded{2.3}
\end{funcdesc}

\begin{funcdesc}{release_lock}{}
Release the interpreter's import lock.
On platforms without threads, this function does nothing.
\versionadded{2.3}
\end{funcdesc}

The following constants with integer values, defined in this module,
are used to indicate the search result of \function{find_module()}.

\begin{datadesc}{PY_SOURCE}
The module was found as a source file.
\end{datadesc}

\begin{datadesc}{PY_COMPILED}
The module was found as a compiled code object file.
\end{datadesc}

\begin{datadesc}{C_EXTENSION}
The module was found as dynamically loadable shared library.
\end{datadesc}

\begin{datadesc}{PY_RESOURCE}
The module was found as a Macintosh resource.  This value can only be
returned on a Macintosh.
\end{datadesc}

\begin{datadesc}{PKG_DIRECTORY}
The module was found as a package directory.
\end{datadesc}

\begin{datadesc}{C_BUILTIN}
The module was found as a built-in module.
\end{datadesc}

\begin{datadesc}{PY_FROZEN}
The module was found as a frozen module (see \function{init_frozen()}).
\end{datadesc}

The following constant and functions are obsolete; their functionality
is available through \function{find_module()} or \function{load_module()}.
They are kept around for backward compatibility:

\begin{datadesc}{SEARCH_ERROR}
Unused.
\end{datadesc}

\begin{funcdesc}{init_builtin}{name}
Initialize the built-in module called \var{name} and return its module
object.  If the module was already initialized, it will be initialized
\emph{again}.  A few modules cannot be initialized twice --- attempting
to initialize these again will raise an \exception{ImportError}
exception.  If there is no
built-in module called \var{name}, \code{None} is returned.
\end{funcdesc}

\begin{funcdesc}{init_frozen}{name}
Initialize the frozen module called \var{name} and return its module
object.  If the module was already initialized, it will be initialized
\emph{again}.  If there is no frozen module called \var{name},
\code{None} is returned.  (Frozen modules are modules written in
Python whose compiled byte-code object is incorporated into a
custom-built Python interpreter by Python's \program{freeze} utility.
See \file{Tools/freeze/} for now.)
\end{funcdesc}

\begin{funcdesc}{is_builtin}{name}
Return \code{1} if there is a built-in module called \var{name} which
can be initialized again.  Return \code{-1} if there is a built-in
module called \var{name} which cannot be initialized again (see
\function{init_builtin()}).  Return \code{0} if there is no built-in
module called \var{name}.
\end{funcdesc}

\begin{funcdesc}{is_frozen}{name}
Return \code{True} if there is a frozen module (see
\function{init_frozen()}) called \var{name}, or \code{False} if there is
no such module.
\end{funcdesc}

\begin{funcdesc}{load_compiled}{name, pathname, file}
\indexii{file}{byte-code}
Load and initialize a module implemented as a byte-compiled code file
and return its module object.  If the module was already initialized,
it will be initialized \emph{again}.  The \var{name} argument is used
to create or access a module object.  The \var{pathname} argument
points to the byte-compiled code file.  The \var{file}
argument is the byte-compiled code file, open for reading in binary
mode, from the beginning.
It must currently be a real file object, not a
user-defined class emulating a file.
\end{funcdesc}

\begin{funcdesc}{load_dynamic}{name, pathname\optional{, file}}
Load and initialize a module implemented as a dynamically loadable
shared library and return its module object.  If the module was
already initialized, it will be initialized \emph{again}.  Some modules
don't like that and may raise an exception.  The \var{pathname}
argument must point to the shared library.  The \var{name} argument is
used to construct the name of the initialization function: an external
C function called \samp{init\var{name}()} in the shared library is
called.  The optional \var{file} argument is ignored.  (Note: using
shared libraries is highly system dependent, and not all systems
support it.)
\end{funcdesc}

\begin{funcdesc}{load_source}{name, pathname, file}
Load and initialize a module implemented as a Python source file and
return its module object.  If the module was already initialized, it
will be initialized \emph{again}.  The \var{name} argument is used to
create or access a module object.  The \var{pathname} argument points
to the source file.  The \var{file} argument is the source
file, open for reading as text, from the beginning.
It must currently be a real file
object, not a user-defined class emulating a file.  Note that if a
properly matching byte-compiled file (with suffix \file{.pyc} or
\file{.pyo}) exists, it will be used instead of parsing the given
source file.
\end{funcdesc}


\subsection{Examples}
\label{examples-imp}

The following function emulates what was the standard import statement
up to Python 1.4 (no hierarchical module names).  (This
\emph{implementation} wouldn't work in that version, since
\function{find_module()} has been extended and
\function{load_module()} has been added in 1.4.)

\begin{verbatim}
import imp
import sys

def __import__(name, globals=None, locals=None, fromlist=None):
    # Fast path: see if the module has already been imported.
    try:
        return sys.modules[name]
    except KeyError:
        pass

    # If any of the following calls raises an exception,
    # there's a problem we can't handle -- let the caller handle it.

    fp, pathname, description = imp.find_module(name)
    
    try:
        return imp.load_module(name, fp, pathname, description)
    finally:
        # Since we may exit via an exception, close fp explicitly.
        if fp:
            fp.close()
\end{verbatim}

A more complete example that implements hierarchical module names and
includes a \function{reload()}\bifuncindex{reload} function can be
found in the module \module{knee}\refmodindex{knee}.  The
\module{knee} module can be found in \file{Demo/imputil/} in the
Python source distribution.

\section{Built-in Module \sectcode{ni}}
\label{module-ni}
\bimodindex{ni}

\strong{Warning: This module is obsolete.}  As of Python 1.5a4,
package support (with different semantics for \code{__init__} and no
support for \code{__domain__} or\code{f__}) is built in the
interpreter.  The ni module is retained only for backward
compatibility.

The \code{ni} module defines a new importing scheme, which supports
packages containing several Python modules.  To enable package
support, execute \code{import ni} before importing any packages.  Importing
this module automatically installs the relevant import hooks.  There
are no publicly-usable functions or variables in the \code{ni} module.

To create a package named \code{spam} containing sub-modules \code{ham}, \code{bacon} and
\code{eggs}, create a directory \file{spam} somewhere on Python's module search
path, as given in \code{sys.path}.  Then, create files called \file{ham.py}, \file{bacon.py} and
\file{eggs.py} inside \file{spam}.

To import module \code{ham} from package \code{spam} and use function
\code{hamneggs()} from that module, you can use any of the following
possibilities:

\bcode\begin{verbatim}
import spam.ham		# *not* "import spam" !!!
spam.ham.hamneggs()
\end{verbatim}\ecode
%
\bcode\begin{verbatim}
from spam import ham
ham.hamneggs()
\end{verbatim}\ecode
%
\bcode\begin{verbatim}
from spam.ham import hamneggs
hamneggs()
\end{verbatim}\ecode
%
\code{import spam} creates an
empty package named \code{spam} if one does not already exist, but it does
\emph{not} automatically import \code{spam}'s submodules.  
The only submodule that is guaranteed to be imported is
\code{spam.__init__}, if it exists; it would be in a file named
\file{__init__.py} in the \file{spam} directory.  Note that
\code{spam.__init__} is a submodule of package spam.  It can refer to
spam's namespace as \code{__} (two underscores):

\bcode\begin{verbatim}
__.spam_inited = 1		# Set a package-level variable
\end{verbatim}\ecode
%
Additional initialization code (setting up variables, importing other
submodules) can be performed in \file{spam/__init__.py}.

% libparser.tex
%
% Copyright 1995 Virginia Polytechnic Institute and State University
% and Fred L. Drake, Jr.  This copyright notice must be distributed on
% all copies, but this document otherwise may be distributed as part
% of the Python distribution.  No fee may be charged for this document
% in any representation, either on paper or electronically.  This
% restriction does not affect other elements in a distributed package
% in any way.
%

\section{Built-in Module \sectcode{parser}}
\label{module-parser}
\bimodindex{parser}
\index{parsing!Python source code}

The \module{parser} module provides an interface to Python's internal
parser and byte-code compiler.  The primary purpose for this interface
is to allow Python code to edit the parse tree of a Python expression
and create executable code from this.  This is better than trying
to parse and modify an arbitrary Python code fragment as a string
because parsing is performed in a manner identical to the code
forming the application.  It is also faster.

There are a few things to note about this module which are important
to making use of the data structures created.  This is not a tutorial
on editing the parse trees for Python code, but some examples of using
the \module{parser} module are presented.

Most importantly, a good understanding of the Python grammar processed
by the internal parser is required.  For full information on the
language syntax, refer to the \emph{Python Language Reference}.  The
parser itself is created from a grammar specification defined in the file
\file{Grammar/Grammar} in the standard Python distribution.  The parse
trees stored in the AST objects created by this module are the
actual output from the internal parser when created by the
\function{expr()} or \function{suite()} functions, described below.  The AST
objects created by \function{sequence2ast()} faithfully simulate those
structures.  Be aware that the values of the sequences which are
considered ``correct'' will vary from one version of Python to another
as the formal grammar for the language is revised.  However,
transporting code from one Python version to another as source text
will always allow correct parse trees to be created in the target
version, with the only restriction being that migrating to an older
version of the interpreter will not support more recent language
constructs.  The parse trees are not typically compatible from one
version to another, whereas source code has always been
forward-compatible.

Each element of the sequences returned by \function{ast2list()} or
\function{ast2tuple()} has a simple form.  Sequences representing
non-terminal elements in the grammar always have a length greater than
one.  The first element is an integer which identifies a production in
the grammar.  These integers are given symbolic names in the C header
file \file{Include/graminit.h} and the Python module
\module{symbol}.  Each additional element of the sequence represents
a component of the production as recognized in the input string: these
are always sequences which have the same form as the parent.  An
important aspect of this structure which should be noted is that
keywords used to identify the parent node type, such as the keyword
\keyword{if} in an \constant{if_stmt}, are included in the node tree without
any special treatment.  For example, the \keyword{if} keyword is
represented by the tuple \code{(1, 'if')}, where \code{1} is the
numeric value associated with all \code{NAME} tokens, including
variable and function names defined by the user.  In an alternate form
returned when line number information is requested, the same token
might be represented as \code{(1, 'if', 12)}, where the \code{12}
represents the line number at which the terminal symbol was found.

Terminal elements are represented in much the same way, but without
any child elements and the addition of the source text which was
identified.  The example of the \keyword{if} keyword above is
representative.  The various types of terminal symbols are defined in
the C header file \file{Include/token.h} and the Python module
\module{token}.

The AST objects are not required to support the functionality of this
module, but are provided for three purposes: to allow an application
to amortize the cost of processing complex parse trees, to provide a
parse tree representation which conserves memory space when compared
to the Python list or tuple representation, and to ease the creation
of additional modules in C which manipulate parse trees.  A simple
``wrapper'' class may be created in Python to hide the use of AST
objects.

The \module{parser} module defines functions for a few distinct
purposes.  The most important purposes are to create AST objects and
to convert AST objects to other representations such as parse trees
and compiled code objects, but there are also functions which serve to
query the type of parse tree represented by an AST object.

\setindexsubitem{(in module parser)}


\subsection{Creating AST Objects}
\label{Creating ASTs}

AST objects may be created from source code or from a parse tree.
When creating an AST object from source, different functions are used
to create the \code{'eval'} and \code{'exec'} forms.

\begin{funcdesc}{expr}{string}
The \function{expr()} function parses the parameter \code{\var{string}}
as if it were an input to \samp{compile(\var{string}, 'eval')}.  If
the parse succeeds, an AST object is created to hold the internal
parse tree representation, otherwise an appropriate exception is
thrown.
\end{funcdesc}

\begin{funcdesc}{suite}{string}
The \function{suite()} function parses the parameter \code{\var{string}}
as if it were an input to \samp{compile(\var{string}, 'exec')}.  If
the parse succeeds, an AST object is created to hold the internal
parse tree representation, otherwise an appropriate exception is
thrown.
\end{funcdesc}

\begin{funcdesc}{sequence2ast}{sequence}
This function accepts a parse tree represented as a sequence and
builds an internal representation if possible.  If it can validate
that the tree conforms to the Python grammar and all nodes are valid
node types in the host version of Python, an AST object is created
from the internal representation and returned to the called.  If there
is a problem creating the internal representation, or if the tree
cannot be validated, a \exception{ParserError} exception is thrown.  An AST
object created this way should not be assumed to compile correctly;
normal exceptions thrown by compilation may still be initiated when
the AST object is passed to \function{compileast()}.  This may indicate
problems not related to syntax (such as a \exception{MemoryError}
exception), but may also be due to constructs such as the result of
parsing \code{del f(0)}, which escapes the Python parser but is
checked by the bytecode compiler.

Sequences representing terminal tokens may be represented as either
two-element lists of the form \code{(1, 'name')} or as three-element
lists of the form \code{(1, 'name', 56)}.  If the third element is
present, it is assumed to be a valid line number.  The line number
may be specified for any subset of the terminal symbols in the input
tree.
\end{funcdesc}

\begin{funcdesc}{tuple2ast}{sequence}
This is the same function as \function{sequence2ast()}.  This entry point
is maintained for backward compatibility.
\end{funcdesc}


\subsection{Converting AST Objects}
\label{Converting ASTs}

AST objects, regardless of the input used to create them, may be
converted to parse trees represented as list- or tuple- trees, or may
be compiled into executable code objects.  Parse trees may be
extracted with or without line numbering information.

\begin{funcdesc}{ast2list}{ast\optional{\, line_info\code{ = 0}}}
This function accepts an AST object from the caller in
\code{\var{ast}} and returns a Python list representing the
equivelent parse tree.  The resulting list representation can be used
for inspection or the creation of a new parse tree in list form.  This
function does not fail so long as memory is available to build the
list representation.  If the parse tree will only be used for
inspection, \function{ast2tuple()} should be used instead to reduce memory
consumption and fragmentation.  When the list representation is
required, this function is significantly faster than retrieving a
tuple representation and converting that to nested lists.

If \code{\var{line_info}} is true, line number information will be
included for all terminal tokens as a third element of the list
representing the token.  Note that the line number provided specifies
the line on which the token \emph{ends}.  This information is
omitted if the flag is false or omitted.
\end{funcdesc}

\begin{funcdesc}{ast2tuple}{ast\optional{\, line_info\code{ = 0}}}
This function accepts an AST object from the caller in
\code{\var{ast}} and returns a Python tuple representing the
equivelent parse tree.  Other than returning a tuple instead of a
list, this function is identical to \function{ast2list()}.

If \code{\var{line_info}} is true, line number information will be
included for all terminal tokens as a third element of the list
representing the token.  This information is omitted if the flag is
false or omitted.
\end{funcdesc}

\begin{funcdesc}{compileast}{ast\optional{\, filename\code{ = '<ast>'}}}
The Python byte compiler can be invoked on an AST object to produce
code objects which can be used as part of an \code{exec} statement or
a call to the built-in \function{eval()}\bifuncindex{eval} function.
This function provides the interface to the compiler, passing the
internal parse tree from \code{\var{ast}} to the parser, using the
source file name specified by the \code{\var{filename}} parameter.
The default value supplied for \code{\var{filename}} indicates that
the source was an AST object.

Compiling an AST object may result in exceptions related to
compilation; an example would be a \exception{SyntaxError} caused by the
parse tree for \code{del f(0)}: this statement is considered legal
within the formal grammar for Python but is not a legal language
construct.  The \exception{SyntaxError} raised for this condition is
actually generated by the Python byte-compiler normally, which is why
it can be raised at this point by the \module{parser} module.  Most
causes of compilation failure can be diagnosed programmatically by
inspection of the parse tree.
\end{funcdesc}


\subsection{Queries on AST Objects}
\label{Querying ASTs}

Two functions are provided which allow an application to determine if
an AST was create as an expression or a suite.  Neither of these
functions can be used to determine if an AST was created from source
code via \function{expr()} or \function{suite()} or from a parse tree
via \function{sequence2ast()}.

\begin{funcdesc}{isexpr}{ast}
When \code{\var{ast}} represents an \code{'eval'} form, this function
returns true, otherwise it returns false.  This is useful, since code
objects normally cannot be queried for this information using existing
built-in functions.  Note that the code objects created by
\function{compileast()} cannot be queried like this either, and are
identical to those created by the built-in
\function{compile()}\bifuncindex{compile} function.
\end{funcdesc}


\begin{funcdesc}{issuite}{ast}
This function mirrors \function{isexpr()} in that it reports whether an
AST object represents an \code{'exec'} form, commonly known as a
``suite.''  It is not safe to assume that this function is equivelent
to \samp{not isexpr(\var{ast})}, as additional syntactic fragments may
be supported in the future.
\end{funcdesc}


\subsection{Exceptions and Error Handling}
\label{AST Errors}

The parser module defines a single exception, but may also pass other
built-in exceptions from other portions of the Python runtime
environment.  See each function for information about the exceptions
it can raise.

\begin{excdesc}{ParserError}
Exception raised when a failure occurs within the parser module.  This
is generally produced for validation failures rather than the built in
\exception{SyntaxError} thrown during normal parsing.
The exception argument is either a string describing the reason of the
failure or a tuple containing a sequence causing the failure from a parse
tree passed to \function{sequence2ast()} and an explanatory string.  Calls to
\function{sequence2ast()} need to be able to handle either type of exception,
while calls to other functions in the module will only need to be
aware of the simple string values.
\end{excdesc}

Note that the functions \function{compileast()}, \function{expr()}, and
\function{suite()} may throw exceptions which are normally thrown by the
parsing and compilation process.  These include the built in
exceptions \exception{MemoryError}, \exception{OverflowError},
\exception{SyntaxError}, and \exception{SystemError}.  In these cases, these
exceptions carry all the meaning normally associated with them.  Refer
to the descriptions of each function for detailed information.


\subsection{AST Objects}
\label{AST Objects}

AST objects returned by \function{expr()}, \function{suite()} and
\function{sequence2ast()} have no methods of their own.
Some of the functions defined which accept an AST object as their
first argument may change to object methods in the future.

\begin{datadesc}{ASTType}
The type of the objects returned by \function{expr()},
\function{suite()} and \function{sequence2ast()}.

Ordered and equality comparisons are supported between AST objects.
\end{datadesc}


\subsection{Examples}
\nodename{AST Examples}

The parser modules allows operations to be performed on the parse tree
of Python source code before the bytecode is generated, and provides
for inspection of the parse tree for information gathering purposes.
Two examples are presented.  The simple example demonstrates emulation
of the \function{compile()}\bifuncindex{compile} built-in function and
the complex example shows the use of a parse tree for information
discovery.

\subsubsection{Emulation of \sectcode{compile()}}

While many useful operations may take place between parsing and
bytecode generation, the simplest operation is to do nothing.  For
this purpose, using the \module{parser} module to produce an
intermediate data structure is equivelent to the code

\begin{verbatim}
>>> code = compile('a + 5', 'eval')
>>> a = 5
>>> eval(code)
10
\end{verbatim}
%
The equivelent operation using the \module{parser} module is somewhat
longer, and allows the intermediate internal parse tree to be retained
as an AST object:

\begin{verbatim}
>>> import parser
>>> ast = parser.expr('a + 5')
>>> code = parser.compileast(ast)
>>> a = 5
>>> eval(code)
10
\end{verbatim}
%
An application which needs both AST and code objects can package this
code into readily available functions:

\begin{verbatim}
import parser

def load_suite(source_string):
    ast = parser.suite(source_string)
    code = parser.compileast(ast)
    return ast, code

def load_expression(source_string):
    ast = parser.expr(source_string)
    code = parser.compileast(ast)
    return ast, code
\end{verbatim}
%
\subsubsection{Information Discovery}

Some applications benefit from direct access to the parse tree.  The
remainder of this section demonstrates how the parse tree provides
access to module documentation defined in docstrings without requiring
that the code being examined be loaded into a running interpreter via
\keyword{import}.  This can be very useful for performing analyses of
untrusted code.

Generally, the example will demonstrate how the parse tree may be
traversed to distill interesting information.  Two functions and a set
of classes are developed which provide programmatic access to high
level function and class definitions provided by a module.  The
classes extract information from the parse tree and provide access to
the information at a useful semantic level, one function provides a
simple low-level pattern matching capability, and the other function
defines a high-level interface to the classes by handling file
operations on behalf of the caller.  All source files mentioned here
which are not part of the Python installation are located in the
\file{Demo/parser/} directory of the distribution.

The dynamic nature of Python allows the programmer a great deal of
flexibility, but most modules need only a limited measure of this when
defining classes, functions, and methods.  In this example, the only
definitions that will be considered are those which are defined in the
top level of their context, e.g., a function defined by a \keyword{def}
statement at column zero of a module, but not a function defined
within a branch of an \code{if} ... \code{else} construct, though
there are some good reasons for doing so in some situations.  Nesting
of definitions will be handled by the code developed in the example.

To construct the upper-level extraction methods, we need to know what
the parse tree structure looks like and how much of it we actually
need to be concerned about.  Python uses a moderately deep parse tree
so there are a large number of intermediate nodes.  It is important to
read and understand the formal grammar used by Python.  This is
specified in the file \file{Grammar/Grammar} in the distribution.
Consider the simplest case of interest when searching for docstrings:
a module consisting of a docstring and nothing else.  (See file
\file{docstring.py}.)

\begin{verbatim}
"""Some documentation.
"""
\end{verbatim}
%
Using the interpreter to take a look at the parse tree, we find a
bewildering mass of numbers and parentheses, with the documentation
buried deep in nested tuples.

\begin{verbatim}
>>> import parser
>>> import pprint
>>> ast = parser.suite(open('docstring.py').read())
>>> tup = parser.ast2tuple(ast)
>>> pprint.pprint(tup)
(257,
 (264,
  (265,
   (266,
    (267,
     (307,
      (287,
       (288,
        (289,
         (290,
          (292,
           (293,
            (294,
             (295,
              (296,
               (297,
                (298,
                 (299,
                  (300, (3, '"""Some documentation.\012"""'))))))))))))))))),
   (4, ''))),
 (4, ''),
 (0, ''))
\end{verbatim}
%
The numbers at the first element of each node in the tree are the node
types; they map directly to terminal and non-terminal symbols in the
grammar.  Unfortunately, they are represented as integers in the
internal representation, and the Python structures generated do not
change that.  However, the \module{symbol} and \module{token} modules
provide symbolic names for the node types and dictionaries which map
from the integers to the symbolic names for the node types.

In the output presented above, the outermost tuple contains four
elements: the integer \code{257} and three additional tuples.  Node
type \code{257} has the symbolic name \constant{file_input}.  Each of
these inner tuples contains an integer as the first element; these
integers, \code{264}, \code{4}, and \code{0}, represent the node types
\constant{stmt}, \constant{NEWLINE}, and \constant{ENDMARKER},
respectively.
Note that these values may change depending on the version of Python
you are using; consult \file{symbol.py} and \file{token.py} for
details of the mapping.  It should be fairly clear that the outermost
node is related primarily to the input source rather than the contents
of the file, and may be disregarded for the moment.  The \constant{stmt}
node is much more interesting.  In particular, all docstrings are
found in subtrees which are formed exactly as this node is formed,
with the only difference being the string itself.  The association
between the docstring in a similar tree and the defined entity (class,
function, or module) which it describes is given by the position of
the docstring subtree within the tree defining the described
structure.

By replacing the actual docstring with something to signify a variable
component of the tree, we allow a simple pattern matching approach to
check any given subtree for equivelence to the general pattern for
docstrings.  Since the example demonstrates information extraction, we
can safely require that the tree be in tuple form rather than list
form, allowing a simple variable representation to be
\code{['variable_name']}.  A simple recursive function can implement
the pattern matching, returning a boolean and a dictionary of variable
name to value mappings.  (See file \file{example.py}.)

\begin{verbatim}
from types import ListType, TupleType

def match(pattern, data, vars=None):
    if vars is None:
        vars = {}
    if type(pattern) is ListType:
        vars[pattern[0]] = data
        return 1, vars
    if type(pattern) is not TupleType:
        return (pattern == data), vars
    if len(data) != len(pattern):
        return 0, vars
    for pattern, data in map(None, pattern, data):
        same, vars = match(pattern, data, vars)
        if not same:
            break
    return same, vars
\end{verbatim}
%
Using this simple representation for syntactic variables and the symbolic
node types, the pattern for the candidate docstring subtrees becomes
fairly readable.  (See file \file{example.py}.)

\begin{verbatim}
import symbol
import token

DOCSTRING_STMT_PATTERN = (
    symbol.stmt,
    (symbol.simple_stmt,
     (symbol.small_stmt,
      (symbol.expr_stmt,
       (symbol.testlist,
        (symbol.test,
         (symbol.and_test,
          (symbol.not_test,
           (symbol.comparison,
            (symbol.expr,
             (symbol.xor_expr,
              (symbol.and_expr,
               (symbol.shift_expr,
                (symbol.arith_expr,
                 (symbol.term,
                  (symbol.factor,
                   (symbol.power,
                    (symbol.atom,
                     (token.STRING, ['docstring'])
                     )))))))))))))))),
     (token.NEWLINE, '')
     ))
\end{verbatim}
%
Using the \function{match()} function with this pattern, extracting the
module docstring from the parse tree created previously is easy:

\begin{verbatim}
>>> found, vars = match(DOCSTRING_STMT_PATTERN, tup[1])
>>> found
1
>>> vars
{'docstring': '"""Some documentation.\012"""'}
\end{verbatim}
%
Once specific data can be extracted from a location where it is
expected, the question of where information can be expected
needs to be answered.  When dealing with docstrings, the answer is
fairly simple: the docstring is the first \constant{stmt} node in a code
block (\constant{file_input} or \constant{suite} node types).  A module
consists of a single \constant{file_input} node, and class and function
definitions each contain exactly one \constant{suite} node.  Classes and
functions are readily identified as subtrees of code block nodes which
start with \code{(stmt, (compound_stmt, (classdef, ...} or
\code{(stmt, (compound_stmt, (funcdef, ...}.  Note that these subtrees
cannot be matched by \function{match()} since it does not support multiple
sibling nodes to match without regard to number.  A more elaborate
matching function could be used to overcome this limitation, but this
is sufficient for the example.

Given the ability to determine whether a statement might be a
docstring and extract the actual string from the statement, some work
needs to be performed to walk the parse tree for an entire module and
extract information about the names defined in each context of the
module and associate any docstrings with the names.  The code to
perform this work is not complicated, but bears some explanation.

The public interface to the classes is straightforward and should
probably be somewhat more flexible.  Each ``major'' block of the
module is described by an object providing several methods for inquiry
and a constructor which accepts at least the subtree of the complete
parse tree which it represents.  The \class{ModuleInfo} constructor
accepts an optional \var{name} parameter since it cannot
otherwise determine the name of the module.

The public classes include \class{ClassInfo}, \class{FunctionInfo},
and \class{ModuleInfo}.  All objects provide the
methods \method{get_name()}, \method{get_docstring()},
\method{get_class_names()}, and \method{get_class_info()}.  The
\class{ClassInfo} objects support \method{get_method_names()} and
\method{get_method_info()} while the other classes provide
\method{get_function_names()} and \method{get_function_info()}.

Within each of the forms of code block that the public classes
represent, most of the required information is in the same form and is
accessed in the same way, with classes having the distinction that
functions defined at the top level are referred to as ``methods.''
Since the difference in nomenclature reflects a real semantic
distinction from functions defined outside of a class, the
implementation needs to maintain the distinction.
Hence, most of the functionality of the public classes can be
implemented in a common base class, \class{SuiteInfoBase}, with the
accessors for function and method information provided elsewhere.
Note that there is only one class which represents function and method
information; this parallels the use of the \keyword{def} statement to
define both types of elements.

Most of the accessor functions are declared in \class{SuiteInfoBase}
and do not need to be overriden by subclasses.  More importantly, the
extraction of most information from a parse tree is handled through a
method called by the \class{SuiteInfoBase} constructor.  The example
code for most of the classes is clear when read alongside the formal
grammar, but the method which recursively creates new information
objects requires further examination.  Here is the relevant part of
the \class{SuiteInfoBase} definition from \file{example.py}:

\begin{verbatim}
class SuiteInfoBase:
    _docstring = ''
    _name = ''

    def __init__(self, tree = None):
        self._class_info = {}
        self._function_info = {}
        if tree:
            self._extract_info(tree)

    def _extract_info(self, tree):
        # extract docstring
        if len(tree) == 2:
            found, vars = match(DOCSTRING_STMT_PATTERN[1], tree[1])
        else:
            found, vars = match(DOCSTRING_STMT_PATTERN, tree[3])
        if found:
            self._docstring = eval(vars['docstring'])
        # discover inner definitions
        for node in tree[1:]:
            found, vars = match(COMPOUND_STMT_PATTERN, node)
            if found:
                cstmt = vars['compound']
                if cstmt[0] == symbol.funcdef:
                    name = cstmt[2][1]
                    self._function_info[name] = FunctionInfo(cstmt)
                elif cstmt[0] == symbol.classdef:
                    name = cstmt[2][1]
                    self._class_info[name] = ClassInfo(cstmt)
\end{verbatim}
%
After initializing some internal state, the constructor calls the
\method{_extract_info()} method.  This method performs the bulk of the
information extraction which takes place in the entire example.  The
extraction has two distinct phases: the location of the docstring for
the parse tree passed in, and the discovery of additional definitions
within the code block represented by the parse tree.

The initial \keyword{if} test determines whether the nested suite is of
the ``short form'' or the ``long form.''  The short form is used when
the code block is on the same line as the definition of the code
block, as in

\begin{verbatim}
def square(x): "Square an argument."; return x ** 2
\end{verbatim}
%
while the long form uses an indented block and allows nested
definitions:

\begin{verbatim}
def make_power(exp):
    "Make a function that raises an argument to the exponent `exp'."
    def raiser(x, y=exp):
        return x ** y
    return raiser
\end{verbatim}
%
When the short form is used, the code block may contain a docstring as
the first, and possibly only, \constant{small_stmt} element.  The
extraction of such a docstring is slightly different and requires only
a portion of the complete pattern used in the more common case.  As
implemented, the docstring will only be found if there is only
one \constant{small_stmt} node in the \constant{simple_stmt} node.
Since most functions and methods which use the short form do not
provide a docstring, this may be considered sufficient.  The
extraction of the docstring proceeds using the \function{match()} function
as described above, and the value of the docstring is stored as an
attribute of the \class{SuiteInfoBase} object.

After docstring extraction, a simple definition discovery
algorithm operates on the \constant{stmt} nodes of the
\constant{suite} node.  The special case of the short form is not
tested; since there are no \constant{stmt} nodes in the short form,
the algorithm will silently skip the single \constant{simple_stmt}
node and correctly not discover any nested definitions.

Each statement in the code block is categorized as
a class definition, function or method definition, or
something else.  For the definition statements, the name of the
element defined is extracted and a representation object
appropriate to the definition is created with the defining subtree
passed as an argument to the constructor.  The repesentation objects
are stored in instance variables and may be retrieved by name using
the appropriate accessor methods.

The public classes provide any accessors required which are more
specific than those provided by the \class{SuiteInfoBase} class, but
the real extraction algorithm remains common to all forms of code
blocks.  A high-level function can be used to extract the complete set
of information from a source file.  (See file \file{example.py}.)

\begin{verbatim}
def get_docs(fileName):
    source = open(fileName).read()
    import os
    basename = os.path.basename(os.path.splitext(fileName)[0])
    import parser
    ast = parser.suite(source)
    tup = parser.ast2tuple(ast)
    return ModuleInfo(tup, basename)
\end{verbatim}
%
This provides an easy-to-use interface to the documentation of a
module.  If information is required which is not extracted by the code
of this example, the code may be extended at clearly defined points to
provide additional capabilities.

\begin{seealso}

\seemodule{symbol}{
  useful constants representing internal nodes of the parse tree}

\seemodule{token}{
  useful constants representing leaf nodes of the parse tree and
  functions for testing node values}

\end{seealso}

\section{\module{symbol} ---
         Constants used with Python parse trees}

\declaremodule{standard}{symbol}
\modulesynopsis{Constants representing internal nodes of the parse tree.}
\sectionauthor{Fred L. Drake, Jr.}{fdrake@acm.org}


This module provides constants which represent the numeric values of
internal nodes of the parse tree.  Unlike most Python constants, these
use lower-case names.  Refer to the file \file{Grammar/Grammar} in the
Python distribution for the defintions of the names in the context of
the language grammar.  The specific numeric values which the names map
to may change between Python versions.

This module also provides one additional data object:



\begin{datadesc}{sym_name}
Dictionary mapping the numeric values of the constants defined in this
module back to name strings, allowing more human-readable
representation of parse trees to be generated.
\end{datadesc}

\begin{seealso}
\seemodule{parser}{second example uses this module}
\end{seealso}

\section{\module{token} ---
         Constants used with Python parse trees}

\declaremodule{standard}{token}
\modulesynopsis{Constants representing terminal nodes of the parse tree.}
\sectionauthor{Fred L. Drake, Jr.}{fdrake@acm.org}


This module provides constants which represent the numeric values of
leaf nodes of the parse tree (terminal tokens).  Refer to the file
\file{Grammar/Grammar} in the Python distribution for the defintions
of the names in the context of the language grammar.  The specific
numeric values which the names map to may change between Python
versions.

This module also provides one data object and some functions.  The
functions mirror definitions in the Python C header files.



\begin{datadesc}{tok_name}
Dictionary mapping the numeric values of the constants defined in this
module back to name strings, allowing more human-readable
representation of parse trees to be generated.
\end{datadesc}

\begin{funcdesc}{ISTERMINAL}{x}
Return true for terminal token values.
\end{funcdesc}

\begin{funcdesc}{ISNONTERMINAL}{x}
Return true for non-terminal token values.
\end{funcdesc}

\begin{funcdesc}{ISEOF}{x}
Return true if \var{x} is the marker indicating the end of input.
\end{funcdesc}

\begin{seealso}
\seemodule{parser}{second example uses this module}
\end{seealso}

\section{Standard Module \sectcode{keyword}}
\label{module-keyword}
\stmodindex{keyword}

This module allows a Python program to determine if a string is a
keyword.  A single function is provided:

\begin{funcdesc}{iskeyword}{s}
Return true if \var{s} is a Python keyword.
\end{funcdesc}

\section{\module{code} ---
         Interpreter base classes}
\declaremodule{standard}{code}

\modulesynopsis{Base classes for interactive Python interpreters.}


The \code{code} module provides facilities to implement
read-eval-print loops in Python.  Two classes and convenience
functions are included which can be used to build applications which
provide an interactive interpreter prompt.


\begin{classdesc}{InteractiveInterpreter}{\optional{locals}}
This class deals with parsing and interpreter state (the user's
namespace); it does not deal with input buffering or prompting or
input file naming (the filename is always passed in explicitly).
The optional \var{locals} argument specifies the dictionary in
which code will be executed; it defaults to a newly created
dictionary with key \code{'__name__'} set to \code{'__console__'}
and key \code{'__doc__'} set to \code{None}.
\end{classdesc}

\begin{classdesc}{InteractiveConsole}{\optional{locals\optional{, filename}}}
Closely emulate the behavior of the interactive Python interpreter.
This class builds on \class{InteractiveInterpreter} and adds
prompting using the familiar \code{sys.ps1} and \code{sys.ps2}, and
input buffering.
\end{classdesc}


\begin{funcdesc}{interact}{\optional{banner\optional{,
                           readfunc\optional{, local}}}}
Convenience function to run a read-eval-print loop.  This creates a
new instance of \class{InteractiveConsole} and sets \var{readfunc}
to be used as the \method{raw_input()} method, if provided.  If
\var{local} is provided, it is passed to the
\class{InteractiveConsole} constructor for use as the default
namespace for the interpreter loop.  The \method{interact()} method
of the instance is then run with \var{banner} passed as the banner
to use, if provided.  The console object is discarded after use.
\end{funcdesc}

\begin{funcdesc}{compile_command}{source\optional{,
                                  filename\optional{, symbol}}}
This function is useful for programs that want to emulate Python's
interpreter main loop (a.k.a. the read-eval-print loop).  The tricky
part is to determine when the user has entered an incomplete command
that can be completed by entering more text (as opposed to a
complete command or a syntax error).  This function
\emph{almost} always makes the same decision as the real interpreter
main loop.

\var{source} is the source string; \var{filename} is the optional
filename from which source was read, defaulting to \code{'<input>'};
and \var{symbol} is the optional grammar start symbol, which should
be either \code{'single'} (the default) or \code{'eval'}.

Returns a code object (the same as \code{compile(\var{source},
\var{filename}, \var{symbol})}) if the command is complete and
valid; \code{None} if the command is incomplete; raises
\exception{SyntaxError} if the command is complete and contains a
syntax error, or raises \exception{OverflowError} or
\exception{ValueError} if the command contains an invalid literal.
\end{funcdesc}


\subsection{Interactive Interpreter Objects
            \label{interpreter-objects}}

\begin{methoddesc}{runsource}{source\optional{, filename\optional{, symbol}}}
Compile and run some source in the interpreter.
Arguments are the same as for \function{compile_command()}; the
default for \var{filename} is \code{'<input>'}, and for
\var{symbol} is \code{'single'}.  One several things can happen:

\begin{itemize}
\item
The input is incorrect; \function{compile_command()} raised an
exception (\exception{SyntaxError} or \exception{OverflowError}).  A
syntax traceback will be printed by calling the
\method{showsyntaxerror()} method.  \method{runsource()} returns
\code{False}.

\item
The input is incomplete, and more input is required;
\function{compile_command()} returned \code{None}.
\method{runsource()} returns \code{True}.

\item
The input is complete; \function{compile_command()} returned a code
object.  The code is executed by calling the \method{runcode()} (which
also handles run-time exceptions, except for \exception{SystemExit}).
\method{runsource()} returns \code{False}.
\end{itemize}

The return value can be used to decide whether to use
\code{sys.ps1} or \code{sys.ps2} to prompt the next line.
\end{methoddesc}

\begin{methoddesc}{runcode}{code}
Execute a code object.
When an exception occurs, \method{showtraceback()} is called to
display a traceback.  All exceptions are caught except
\exception{SystemExit}, which is allowed to propagate.

A note about \exception{KeyboardInterrupt}: this exception may occur
elsewhere in this code, and may not always be caught.  The caller
should be prepared to deal with it.
\end{methoddesc}

\begin{methoddesc}{showsyntaxerror}{\optional{filename}}
Display the syntax error that just occurred.  This does not display
a stack trace because there isn't one for syntax errors.
If \var{filename} is given, it is stuffed into the exception instead
of the default filename provided by Python's parser, because it
always uses \code{'<string>'} when reading from a string.
The output is written by the \method{write()} method.
\end{methoddesc}

\begin{methoddesc}{showtraceback}{}
Display the exception that just occurred.  We remove the first stack
item because it is within the interpreter object implementation.
The output is written by the \method{write()} method.
\end{methoddesc}

\begin{methoddesc}{write}{data}
Write a string to the standard error stream (\code{sys.stderr}).
Derived classes should override this to provide the appropriate output
handling as needed.
\end{methoddesc}


\subsection{Interactive Console Objects
            \label{console-objects}}

The \class{InteractiveConsole} class is a subclass of
\class{InteractiveInterpreter}, and so offers all the methods of the
interpreter objects as well as the following additions.

\begin{methoddesc}{interact}{\optional{banner}}
Closely emulate the interactive Python console.
The optional banner argument specify the banner to print before the
first interaction; by default it prints a banner similar to the one
printed by the standard Python interpreter, followed by the class
name of the console object in parentheses (so as not to confuse this
with the real interpreter -- since it's so close!).
\end{methoddesc}

\begin{methoddesc}{push}{line}
Push a line of source text to the interpreter.
The line should not have a trailing newline; it may have internal
newlines.  The line is appended to a buffer and the interpreter's
\method{runsource()} method is called with the concatenated contents
of the buffer as source.  If this indicates that the command was
executed or invalid, the buffer is reset; otherwise, the command is
incomplete, and the buffer is left as it was after the line was
appended.  The return value is \code{True} if more input is required,
\code{False} if the line was dealt with in some way (this is the same as
\method{runsource()}).
\end{methoddesc}

\begin{methoddesc}{resetbuffer}{}
Remove any unhandled source text from the input buffer.
\end{methoddesc}

\begin{methoddesc}{raw_input}{\optional{prompt}}
Write a prompt and read a line.  The returned line does not include
the trailing newline.  When the user enters the \EOF{} key sequence,
\exception{EOFError} is raised.  The base implementation uses the
built-in function \function{raw_input()}; a subclass may replace this
with a different implementation.
\end{methoddesc}

%%  Author:  Fred L. Drake, Jr.		<fdrake@acm.org>

\section{Standard Module \sectcode{pprint}}
\stmodindex{pprint}
\label{module-pprint}

The \module{pprint} module provides a capability to ``pretty-print''
arbitrary Python data structures in a form which can be used as input
to the interpreter.  If the formatted structures include objects which
are not fundamental Python types, the representation may not be
loadable.  This may be the case if objects such as files, sockets,
classes, or instances are included, as well as many other builtin
objects which are not representable as Python constants.

The formatted representation keeps objects on a single line if it can,
and breaks them onto multiple lines if they don't fit within the
allowed width.  Construct \class{PrettyPrinter} objects explicitly if
you need to adjust the width constraint.

The \module{pprint} module defines one class:


% First the implementation class:

\begin{classdesc}{PrettyPrinter}{...}
Construct a \class{PrettyPrinter} instance.  This constructor
understands several keyword parameters.  An output stream may be set
using the \var{stream} keyword; the only method used on the stream
object is the file protocol's \method{write()} method.  If not
specified, the \class{PrettyPrinter} adopts \code{sys.stdout}.  Three
additional parameters may be used to control the formatted
representation.  The keywords are \var{indent}, \var{depth}, and
\var{width}.  The amount of indentation added for each recursive level
is specified by \var{indent}; the default is one.  Other values can
cause output to look a little odd, but can make nesting easier to
spot.  The number of levels which may be printed is controlled by
\var{depth}; if the data structure being printed is too deep, the next
contained level is replaced by \samp{...}.  By default, there is no
constraint on the depth of the objects being formatted.  The desired
output width is constrained using the \var{width} parameter; the
default is eighty characters.  If a structure cannot be formatted
within the constrained width, a best effort will be made.

\begin{verbatim}
>>> import pprint, sys
>>> stuff = sys.path[:]
>>> stuff.insert(0, stuff[:])
>>> pp = pprint.PrettyPrinter(indent=4)
>>> pp.pprint(stuff)
[   [   '',
        '/usr/local/lib/python1.5',
        '/usr/local/lib/python1.5/test',
        '/usr/local/lib/python1.5/sunos5',
        '/usr/local/lib/python1.5/sharedmodules',
        '/usr/local/lib/python1.5/tkinter'],
    '',
    '/usr/local/lib/python1.5',
    '/usr/local/lib/python1.5/test',
    '/usr/local/lib/python1.5/sunos5',
    '/usr/local/lib/python1.5/sharedmodules',
    '/usr/local/lib/python1.5/tkinter']
>>>
>>> import parser
>>> tup = parser.ast2tuple(
...     parser.suite(open('pprint.py').read()))[1][1][1]
>>> pp = pprint.PrettyPrinter(depth=6)
>>> pp.pprint(tup)
(266, (267, (307, (287, (288, (...))))))
\end{verbatim}
\end{classdesc}


% Now the derivative functions:

The \class{PrettyPrinter} class supports several derivative functions:

\begin{funcdesc}{pformat}{object}
Return the formatted representation of \var{object} as a string.  The
default parameters for formatting are used.
\end{funcdesc}

\begin{funcdesc}{pprint}{object\optional{, stream}}
Prints the formatted representation of \var{object} on \var{stream},
followed by a newline.  If \var{stream} is omitted, \code{sys.stdout}
is used.  This may be used in the interactive interpreter instead of a
\keyword{print} statement for inspecting values.  The default
parameters for formatting are used.

\begin{verbatim}
>>> stuff = sys.path[:]
>>> stuff.insert(0, stuff)
>>> pprint.pprint(stuff)
[<Recursion on list with id=869440>,
 '',
 '/usr/local/lib/python1.5',
 '/usr/local/lib/python1.5/test',
 '/usr/local/lib/python1.5/sunos5',
 '/usr/local/lib/python1.5/sharedmodules',
 '/usr/local/lib/python1.5/tkinter']
\end{verbatim}
\end{funcdesc}

\begin{funcdesc}{isreadable}{object}
Determine if the formatted representation of \var{object} is
``readable,'' or can be used to reconstruct the value using
\function{eval()}\bifuncindex{eval}.  Note that this returns false for
recursive objects.

\begin{verbatim}
>>> pprint.isreadable(stuff)
0
\end{verbatim}
\end{funcdesc}

\begin{funcdesc}{isrecursive}{object}
Determine if \var{object} requires a recursive representation.
\end{funcdesc}


One more support function is also defined:

\begin{funcdesc}{saferepr}{object}
Return a string representation of \var{object}, protected against
recursive data structures.  If the representation of \var{object}
exposes a recursive entry, the recursive reference will be represented
as \samp{<Recursion on \var{typename} with id=\var{number}>}.  The
representation is not otherwise formatted.
\end{funcdesc}

% This example is outside the {funcdesc} to keep it from running over
% the right margin.
\begin{verbatim}
>>> pprint.saferepr(stuff)
"[<Recursion on list with id=682968>, '', '/usr/local/lib/python1.4', '/usr/loca
l/lib/python1.4/test', '/usr/local/lib/python1.4/sunos5', '/usr/local/lib/python
1.4/sharedmodules', '/usr/local/lib/python1.4/tkinter']"
\end{verbatim}


\subsection{PrettyPrinter Objects}
\label{PrettyPrinter Objects}

\class{PrettyPrinter} instances have the following methods:


\begin{methoddesc}{pformat}{object}
Return the formatted representation of \var{object}.  This takes into
Account the options passed to the \class{PrettyPrinter} constructor.
\end{methoddesc}

\begin{methoddesc}{pprint}{object}
Print the formatted representation of \var{object} on the configured
stream, followed by a newline.
\end{methoddesc}

The following methods provide the implementations for the
corresponding functions of the same names.  Using these methods on an
instance is slightly more efficient since new \class{PrettyPrinter}
objects don't need to be created.

\begin{methoddesc}{isreadable}{object}
Determine if the formatted representation of the object is
``readable,'' or can be used to reconstruct the value using
\function{eval()}\bifuncindex{eval}.  Note that this returns false for
recursive objects.  If the \var{depth} parameter of the
\class{PrettyPrinter} is set and the object is deeper than allowed,
this returns false.
\end{methoddesc}

\begin{methoddesc}{isrecursive}{object}
Determine if the object requires a recursive representation.
\end{methoddesc}

\section{\module{dis} ---
         Disassembler for Python byte code}

\declaremodule{standard}{dis}
\modulesynopsis{Disassembler for Python byte code.}


The \module{dis} module supports the analysis of Python byte code by
disassembling it.  Since there is no Python assembler, this module
defines the Python assembly language.  The Python byte code which
this module takes as an input is defined in the file 
\file{Include/opcode.h} and used by the compiler and the interpreter.

Example: Given the function \function{myfunc}:

\begin{verbatim}
def myfunc(alist):
    return len(alist)
\end{verbatim}

the following command can be used to get the disassembly of
\function{myfunc()}:

\begin{verbatim}
>>> dis.dis(myfunc)
          0 SET_LINENO          1

          3 SET_LINENO          2
          6 LOAD_GLOBAL         0 (len)
          9 LOAD_FAST           0 (alist)
         12 CALL_FUNCTION       1
         15 RETURN_VALUE   
         16 LOAD_CONST          0 (None)
         19 RETURN_VALUE   
\end{verbatim}

The \module{dis} module defines the following functions and constants:

\begin{funcdesc}{dis}{\optional{bytesource}}
Disassemble the \var{bytesource} object. \var{bytesource} can denote
either a class, a method, a function, or a code object.  For a class,
it disassembles all methods.  For a single code sequence, it prints
one line per byte code instruction.  If no object is provided, it
disassembles the last traceback.
\end{funcdesc}

\begin{funcdesc}{distb}{\optional{tb}}
Disassembles the top-of-stack function of a traceback, using the last
traceback if none was passed.  The instruction causing the exception
is indicated.
\end{funcdesc}

\begin{funcdesc}{disassemble}{code\optional{, lasti}}
Disassembles a code object, indicating the last instruction if \var{lasti}
was provided.  The output is divided in the following columns:

\begin{enumerate}
\item the current instruction, indicated as \samp{-->},
\item a labelled instruction, indicated with \samp{>>},
\item the address of the instruction,
\item the operation code name,
\item operation parameters, and
\item interpretation of the parameters in parentheses.
\end{enumerate}

The parameter interpretation recognizes local and global
variable names, constant values, branch targets, and compare
operators.
\end{funcdesc}

\begin{funcdesc}{disco}{code\optional{, lasti}}
A synonym for disassemble.  It is more convenient to type, and kept
for compatibility with earlier Python releases.
\end{funcdesc}

\begin{datadesc}{opname}
Sequence of operation names, indexable using the byte code.
\end{datadesc}

\begin{datadesc}{cmp_op}
Sequence of all compare operation names.
\end{datadesc}

\begin{datadesc}{hasconst}
Sequence of byte codes that have a constant parameter.
\end{datadesc}

\begin{datadesc}{hasname}
Sequence of byte codes that access an attribute by name.
\end{datadesc}

\begin{datadesc}{hasjrel}
Sequence of byte codes that have a relative jump target.
\end{datadesc}

\begin{datadesc}{hasjabs}
Sequence of byte codes that have an absolute jump target.
\end{datadesc}

\begin{datadesc}{haslocal}
Sequence of byte codes that access a local variable.
\end{datadesc}

\begin{datadesc}{hascompare}
Sequence of byte codes of boolean operations.
\end{datadesc}

\subsection{Python Byte Code Instructions}
\label{bytecodes}

The Python compiler currently generates the following byte code
instructions.

\setindexsubitem{(byte code insns)}

\begin{opcodedesc}{STOP_CODE}{}
Indicates end-of-code to the compiler, not used by the interpreter.
\end{opcodedesc}

\begin{opcodedesc}{POP_TOP}{}
Removes the top-of-stack (TOS) item.
\end{opcodedesc}

\begin{opcodedesc}{ROT_TWO}{}
Swaps the two top-most stack items.
\end{opcodedesc}

\begin{opcodedesc}{ROT_THREE}{}
Lifts second and third stack item one position up, moves top down
to position three.
\end{opcodedesc}

\begin{opcodedesc}{ROT_FOUR}{}
Lifts second, third and forth stack item one position up, moves top down to
position four.
\end{opcodedesc}

\begin{opcodedesc}{DUP_TOP}{}
Duplicates the reference on top of the stack.
\end{opcodedesc}

Unary Operations take the top of the stack, apply the operation, and
push the result back on the stack.

\begin{opcodedesc}{UNARY_POSITIVE}{}
Implements \code{TOS = +TOS}.
\end{opcodedesc}

\begin{opcodedesc}{UNARY_NEGATIVE}{}
Implements \code{TOS = -TOS}.
\end{opcodedesc}

\begin{opcodedesc}{UNARY_NOT}{}
Implements \code{TOS = not TOS}.
\end{opcodedesc}

\begin{opcodedesc}{UNARY_CONVERT}{}
Implements \code{TOS = `TOS`}.
\end{opcodedesc}

\begin{opcodedesc}{UNARY_INVERT}{}
Implements \code{TOS = \~{}TOS}.
\end{opcodedesc}

Binary operations remove the top of the stack (TOS) and the second top-most
stack item (TOS1) from the stack.  They perform the operation, and put the
result back on the stack.

\begin{opcodedesc}{BINARY_POWER}{}
Implements \code{TOS = TOS1 ** TOS}.
\end{opcodedesc}

\begin{opcodedesc}{BINARY_MULTIPLY}{}
Implements \code{TOS = TOS1 * TOS}.
\end{opcodedesc}

\begin{opcodedesc}{BINARY_DIVIDE}{}
Implements \code{TOS = TOS1 / TOS}.
\end{opcodedesc}

\begin{opcodedesc}{BINARY_MODULO}{}
Implements \code{TOS = TOS1 \%{} TOS}.
\end{opcodedesc}

\begin{opcodedesc}{BINARY_ADD}{}
Implements \code{TOS = TOS1 + TOS}.
\end{opcodedesc}

\begin{opcodedesc}{BINARY_SUBTRACT}{}
Implements \code{TOS = TOS1 - TOS}.
\end{opcodedesc}

\begin{opcodedesc}{BINARY_SUBSCR}{}
Implements \code{TOS = TOS1[TOS]}.
\end{opcodedesc}

\begin{opcodedesc}{BINARY_LSHIFT}{}
Implements \code{TOS = TOS1 <\code{}< TOS}.
\end{opcodedesc}

\begin{opcodedesc}{BINARY_RSHIFT}{}
Implements \code{TOS = TOS1 >\code{}> TOS}.
\end{opcodedesc}

\begin{opcodedesc}{BINARY_AND}{}
Implements \code{TOS = TOS1 \&\ TOS}.
\end{opcodedesc}

\begin{opcodedesc}{BINARY_XOR}{}
Implements \code{TOS = TOS1 \^\ TOS}.
\end{opcodedesc}

\begin{opcodedesc}{BINARY_OR}{}
Implements \code{TOS = TOS1 | TOS}.
\end{opcodedesc}

In-place operations are like binary operations, in that they remove TOS and
TOS1, and push the result back on the stack, but the operation is done
in-place when TOS1 supports it, and the resulting TOS may be (but does not
have to be) the original TOS1.

\begin{opcodedesc}{INPLACE_POWER}{}
Implements in-place \code{TOS = TOS1 ** TOS}.
\end{opcodedesc}

\begin{opcodedesc}{INPLACE_MULTIPLY}{}
Implements in-place \code{TOS = TOS1 * TOS}.
\end{opcodedesc}

\begin{opcodedesc}{INPLACE_DIVIDE}{}
Implements in-place \code{TOS = TOS1 / TOS}.
\end{opcodedesc}

\begin{opcodedesc}{INPLACE_MODULO}{}
Implements in-place \code{TOS = TOS1 \%{} TOS}.
\end{opcodedesc}

\begin{opcodedesc}{INPLACE_ADD}{}
Implements in-place \code{TOS = TOS1 + TOS}.
\end{opcodedesc}

\begin{opcodedesc}{INPLACE_SUBTRACT}{}
Implements in-place \code{TOS = TOS1 - TOS}.
\end{opcodedesc}

\begin{opcodedesc}{INPLACE_LSHIFT}{}
Implements in-place \code{TOS = TOS1 <\code{}< TOS}.
\end{opcodedesc}

\begin{opcodedesc}{INPLACE_RSHIFT}{}
Implements in-place \code{TOS = TOS1 >\code{}> TOS}.
\end{opcodedesc}

\begin{opcodedesc}{INPLACE_AND}{}
Implements in-place \code{TOS = TOS1 \&\ TOS}.
\end{opcodedesc}

\begin{opcodedesc}{INPLACE_XOR}{}
Implements in-place \code{TOS = TOS1 \^\ TOS}.
\end{opcodedesc}

\begin{opcodedesc}{INPLACE_OR}{}
Implements in-place \code{TOS = TOS1 | TOS}.
\end{opcodedesc}

The slice opcodes take up to three parameters.

\begin{opcodedesc}{SLICE+0}{}
Implements \code{TOS = TOS[:]}.
\end{opcodedesc}

\begin{opcodedesc}{SLICE+1}{}
Implements \code{TOS = TOS1[TOS:]}.
\end{opcodedesc}

\begin{opcodedesc}{SLICE+2}{}
Implements \code{TOS = TOS1[:TOS1]}.
\end{opcodedesc}

\begin{opcodedesc}{SLICE+3}{}
Implements \code{TOS = TOS2[TOS1:TOS]}.
\end{opcodedesc}

Slice assignment needs even an additional parameter.  As any statement,
they put nothing on the stack.

\begin{opcodedesc}{STORE_SLICE+0}{}
Implements \code{TOS[:] = TOS1}.
\end{opcodedesc}

\begin{opcodedesc}{STORE_SLICE+1}{}
Implements \code{TOS1[TOS:] = TOS2}.
\end{opcodedesc}

\begin{opcodedesc}{STORE_SLICE+2}{}
Implements \code{TOS1[:TOS] = TOS2}.
\end{opcodedesc}

\begin{opcodedesc}{STORE_SLICE+3}{}
Implements \code{TOS2[TOS1:TOS] = TOS3}.
\end{opcodedesc}

\begin{opcodedesc}{DELETE_SLICE+0}{}
Implements \code{del TOS[:]}.
\end{opcodedesc}

\begin{opcodedesc}{DELETE_SLICE+1}{}
Implements \code{del TOS1[TOS:]}.
\end{opcodedesc}

\begin{opcodedesc}{DELETE_SLICE+2}{}
Implements \code{del TOS1[:TOS]}.
\end{opcodedesc}

\begin{opcodedesc}{DELETE_SLICE+3}{}
Implements \code{del TOS2[TOS1:TOS]}.
\end{opcodedesc}

\begin{opcodedesc}{STORE_SUBSCR}{}
Implements \code{TOS1[TOS] = TOS2}.
\end{opcodedesc}

\begin{opcodedesc}{DELETE_SUBSCR}{}
Implements \code{del TOS1[TOS]}.
\end{opcodedesc}

\begin{opcodedesc}{PRINT_EXPR}{}
Implements the expression statement for the interactive mode.  TOS is
removed from the stack and printed.  In non-interactive mode, an
expression statement is terminated with \code{POP_STACK}.
\end{opcodedesc}

\begin{opcodedesc}{PRINT_ITEM}{}
Prints TOS to the file-like object bound to \code{sys.stdout}.  There
is one such instruction for each item in the \keyword{print} statement.
\end{opcodedesc}

\begin{opcodedesc}{PRINT_ITEM_TO}{}
Like \code{PRINT_ITEM}, but prints the item second from TOS to the
file-like object at TOS.  This is used by the extended print statement.
\end{opcodedesc}

\begin{opcodedesc}{PRINT_NEWLINE}{}
Prints a new line on \code{sys.stdout}.  This is generated as the
last operation of a \keyword{print} statement, unless the statement
ends with a comma.
\end{opcodedesc}

\begin{opcodedesc}{PRINT_NEWLINE_TO}{}
Like \code{PRINT_NEWLINE}, but prints the new line on the file-like
object on the TOS.  This is used by the extended print statement.
\end{opcodedesc}

\begin{opcodedesc}{BREAK_LOOP}{}
Terminates a loop due to a \keyword{break} statement.
\end{opcodedesc}

\begin{opcodedesc}{LOAD_LOCALS}{}
Pushes a reference to the locals of the current scope on the stack.
This is used in the code for a class definition: After the class body
is evaluated, the locals are passed to the class definition.
\end{opcodedesc}

\begin{opcodedesc}{RETURN_VALUE}{}
Returns with TOS to the caller of the function.
\end{opcodedesc}

\begin{opcodedesc}{IMPORT_STAR}{}
Loads all symbols not starting with \character{_} directly from the module TOS
to the local namespace. The module is popped after loading all names.
This opcode implements \code{from module import *}.
\end{opcodedesc}

\begin{opcodedesc}{EXEC_STMT}{}
Implements \code{exec TOS2,TOS1,TOS}.  The compiler fills
missing optional parameters with \code{None}.
\end{opcodedesc}

\begin{opcodedesc}{POP_BLOCK}{}
Removes one block from the block stack.  Per frame, there is a 
stack of blocks, denoting nested loops, try statements, and such.
\end{opcodedesc}

\begin{opcodedesc}{END_FINALLY}{}
Terminates a \keyword{finally} clause.  The interpreter recalls
whether the exception has to be re-raised, or whether the function
returns, and continues with the outer-next block.
\end{opcodedesc}

\begin{opcodedesc}{BUILD_CLASS}{}
Creates a new class object.  TOS is the methods dictionary, TOS1
the tuple of the names of the base classes, and TOS2 the class name.
\end{opcodedesc}

All of the following opcodes expect arguments.  An argument is two
bytes, with the more significant byte last.

\begin{opcodedesc}{STORE_NAME}{namei}
Implements \code{name = TOS}. \var{namei} is the index of \var{name}
in the attribute \member{co_names} of the code object.
The compiler tries to use \code{STORE_LOCAL} or \code{STORE_GLOBAL}
if possible.
\end{opcodedesc}

\begin{opcodedesc}{DELETE_NAME}{namei}
Implements \code{del name}, where \var{namei} is the index into
\member{co_names} attribute of the code object.
\end{opcodedesc}

\begin{opcodedesc}{UNPACK_SEQUENCE}{count}
Unpacks TOS into \var{count} individual values, which are put onto
the stack right-to-left.
\end{opcodedesc}

%\begin{opcodedesc}{UNPACK_LIST}{count}
%This opcode is obsolete.
%\end{opcodedesc}

%\begin{opcodedesc}{UNPACK_ARG}{count}
%This opcode is obsolete.
%\end{opcodedesc}

\begin{opcodedesc}{DUP_TOPX}{count}
Duplicate \var{count} items, keeping them in the same order. Due to
implementation limits, \var{count} should be between 1 and 5 inclusive.
\end{opcodedesc}

\begin{opcodedesc}{STORE_ATTR}{namei}
Implements \code{TOS.name = TOS1}, where \var{namei} is the index
of name in \member{co_names}.
\end{opcodedesc}

\begin{opcodedesc}{DELETE_ATTR}{namei}
Implements \code{del TOS.name}, using \var{namei} as index into
\member{co_names}.
\end{opcodedesc}

\begin{opcodedesc}{STORE_GLOBAL}{namei}
Works as \code{STORE_NAME}, but stores the name as a global.
\end{opcodedesc}

\begin{opcodedesc}{DELETE_GLOBAL}{namei}
Works as \code{DELETE_NAME}, but deletes a global name.
\end{opcodedesc}

%\begin{opcodedesc}{UNPACK_VARARG}{argc}
%This opcode is obsolete.
%\end{opcodedesc}

\begin{opcodedesc}{LOAD_CONST}{consti}
Pushes \samp{co_consts[\var{consti}]} onto the stack.
\end{opcodedesc}

\begin{opcodedesc}{LOAD_NAME}{namei}
Pushes the value associated with \samp{co_names[\var{namei}]} onto the stack.
\end{opcodedesc}

\begin{opcodedesc}{BUILD_TUPLE}{count}
Creates a tuple consuming \var{count} items from the stack, and pushes
the resulting tuple onto the stack.
\end{opcodedesc}

\begin{opcodedesc}{BUILD_LIST}{count}
Works as \code{BUILD_TUPLE}, but creates a list.
\end{opcodedesc}

\begin{opcodedesc}{BUILD_MAP}{zero}
Pushes a new empty dictionary object onto the stack.  The argument is
ignored and set to zero by the compiler.
\end{opcodedesc}

\begin{opcodedesc}{LOAD_ATTR}{namei}
Replaces TOS with \code{getattr(TOS, co_names[\var{namei}]}.
\end{opcodedesc}

\begin{opcodedesc}{COMPARE_OP}{opname}
Performs a boolean operation.  The operation name can be found
in \code{cmp_op[\var{opname}]}.
\end{opcodedesc}

\begin{opcodedesc}{IMPORT_NAME}{namei}
Imports the module \code{co_names[\var{namei}]}.  The module object is
pushed onto the stack.  The current namespace is not affected: for a
proper import statement, a subsequent \code{STORE_FAST} instruction
modifies the namespace.
\end{opcodedesc}

\begin{opcodedesc}{IMPORT_FROM}{namei}
Loads the attribute \code{co_names[\var{namei}]} from the module found in
TOS. The resulting object is pushed onto the stack, to be subsequently
stored by a \code{STORE_FAST} instruction.
\end{opcodedesc}

\begin{opcodedesc}{JUMP_FORWARD}{delta}
Increments byte code counter by \var{delta}.
\end{opcodedesc}

\begin{opcodedesc}{JUMP_IF_TRUE}{delta}
If TOS is true, increment the byte code counter by \var{delta}.  TOS is
left on the stack.
\end{opcodedesc}

\begin{opcodedesc}{JUMP_IF_FALSE}{delta}
If TOS is false, increment the byte code counter by \var{delta}.  TOS
is not changed. 
\end{opcodedesc}

\begin{opcodedesc}{JUMP_ABSOLUTE}{target}
Set byte code counter to \var{target}.
\end{opcodedesc}

\begin{opcodedesc}{FOR_LOOP}{delta}
Iterate over a sequence.  TOS is the current index, TOS1 the sequence.
First, the next element is computed.  If the sequence is exhausted,
increment byte code counter by \var{delta}.  Otherwise, push the
sequence, the incremented counter, and the current item onto the stack.
\end{opcodedesc}

%\begin{opcodedesc}{LOAD_LOCAL}{namei}
%This opcode is obsolete.
%\end{opcodedesc}

\begin{opcodedesc}{LOAD_GLOBAL}{namei}
Loads the global named \code{co_names[\var{namei}]} onto the stack.
\end{opcodedesc}

%\begin{opcodedesc}{SET_FUNC_ARGS}{argc}
%This opcode is obsolete.
%\end{opcodedesc}

\begin{opcodedesc}{SETUP_LOOP}{delta}
Pushes a block for a loop onto the block stack.  The block spans
from the current instruction with a size of \var{delta} bytes.
\end{opcodedesc}

\begin{opcodedesc}{SETUP_EXCEPT}{delta}
Pushes a try block from a try-except clause onto the block stack.
\var{delta} points to the first except block.
\end{opcodedesc}

\begin{opcodedesc}{SETUP_FINALLY}{delta}
Pushes a try block from a try-except clause onto the block stack.
\var{delta} points to the finally block.
\end{opcodedesc}

\begin{opcodedesc}{LOAD_FAST}{var_num}
Pushes a reference to the local \code{co_varnames[\var{var_num}]} onto
the stack.
\end{opcodedesc}

\begin{opcodedesc}{STORE_FAST}{var_num}
Stores TOS into the local \code{co_varnames[\var{var_num}]}.
\end{opcodedesc}

\begin{opcodedesc}{DELETE_FAST}{var_num}
Deletes local \code{co_varnames[\var{var_num}]}.
\end{opcodedesc}

\begin{opcodedesc}{SET_LINENO}{lineno}
Sets the current line number to \var{lineno}.
\end{opcodedesc}

\begin{opcodedesc}{RAISE_VARARGS}{argc}
Raises an exception. \var{argc} indicates the number of parameters
to the raise statement, ranging from 0 to 3.  The handler will find
the traceback as TOS2, the parameter as TOS1, and the exception
as TOS.
\end{opcodedesc}

\begin{opcodedesc}{CALL_FUNCTION}{argc}
Calls a function.  The low byte of \var{argc} indicates the number of
positional parameters, the high byte the number of keyword parameters.
On the stack, the opcode finds the keyword parameters first.  For each
keyword argument, the value is on top of the key.  Below the keyword
parameters, the positional parameters are on the stack, with the
right-most parameter on top.  Below the parameters, the function object
to call is on the stack.
\end{opcodedesc}

\begin{opcodedesc}{MAKE_FUNCTION}{argc}
Pushes a new function object on the stack.  TOS is the code associated
with the function.  The function object is defined to have \var{argc}
default parameters, which are found below TOS.
\end{opcodedesc}

\begin{opcodedesc}{BUILD_SLICE}{argc}
Pushes a slice object on the stack.  \var{argc} must be 2 or 3.  If it
is 2, \code{slice(TOS1, TOS)} is pushed; if it is 3,
\code{slice(TOS2, TOS1, TOS)} is pushed.
See the \code{slice()}\bifuncindex{slice} built-in function for more
information.
\end{opcodedesc}

\begin{opcodedesc}{EXTENDED_ARG}{ext}
Prefixes any opcode which has an argument too big to fit into the
default two bytes.  \var{ext} holds two additional bytes which, taken
together with the subsequent opcode's argument, comprise a four-byte
argument, \var{ext} being the two most-significant bytes.
\end{opcodedesc}

\begin{opcodedesc}{CALL_FUNCTION_VAR}{argc}
Calls a function. \var{argc} is interpreted as in \code{CALL_FUNCTION}.
The top element on the stack contains the variable argument list, followed
by keyword and positional arguments.
\end{opcodedesc}

\begin{opcodedesc}{CALL_FUNCTION_KW}{argc}
Calls a function. \var{argc} is interpreted as in \code{CALL_FUNCTION}.
The top element on the stack contains the keyword arguments dictionary, 
followed by explicit keyword and positional arguments.
\end{opcodedesc}

\begin{opcodedesc}{CALL_FUNCTION_VAR_KW}{argc}
Calls a function. \var{argc} is interpreted as in
\code{CALL_FUNCTION}.  The top element on the stack contains the
keyword arguments dictionary, followed by the variable-arguments
tuple, followed by explicit keyword and positional arguments.
\end{opcodedesc}

\section{\module{site} ---
         Site-specific configuration hook}

\declaremodule{standard}{site}
\modulesynopsis{A standard way to reference site-specific modules.}


\strong{This module is automatically imported during initialization.}

In earlier versions of Python (up to and including 1.5a3), scripts or
modules that needed to use site-specific modules would place
\samp{import site} somewhere near the top of their code.  This is no
longer necessary.

This will append site-specific paths to the module search path.
\indexiii{module}{search}{path}

It starts by constructing up to four directories from a head and a
tail part.  For the head part, it uses \code{sys.prefix} and
\code{sys.exec_prefix}; empty heads are skipped.  For
the tail part, it uses the empty string (on Macintosh or Windows) or
it uses first \file{lib/python\shortversion/site-packages} and then
\file{lib/site-python} (on \UNIX).  For each of the distinct
head-tail combinations, it sees if it refers to an existing directory,
and if so, adds it to \code{sys.path} and also inspects the newly added 
path for configuration files.
\indexii{site-python}{directory}
\indexii{site-packages}{directory}

A path configuration file is a file whose name has the form
\file{\var{package}.pth}; its contents are additional items (one
per line) to be added to \code{sys.path}.  Non-existing items are
never added to \code{sys.path}, but no check is made that the item
refers to a directory (rather than a file).  No item is added to
\code{sys.path} more than once.  Blank lines and lines beginning with
\code{\#} are skipped.  Lines starting with \code{import} are executed.
\index{package}
\indexiii{path}{configuration}{file}

For example, suppose \code{sys.prefix} and \code{sys.exec_prefix} are
set to \file{/usr/local}.  The Python \version\ library is then
installed in \file{/usr/local/lib/python\shortversion} (where only the
first three characters of \code{sys.version} are used to form the
installation path name).  Suppose this has a subdirectory
\file{/usr/local/lib/python\shortversion/site-packages} with three
subsubdirectories, \file{foo}, \file{bar} and \file{spam}, and two
path configuration files, \file{foo.pth} and \file{bar.pth}.  Assume
\file{foo.pth} contains the following:

\begin{verbatim}
# foo package configuration

foo
bar
bletch
\end{verbatim}

and \file{bar.pth} contains:

\begin{verbatim}
# bar package configuration

bar
\end{verbatim}

Then the following directories are added to \code{sys.path}, in this
order:

\begin{verbatim}
/usr/local/lib/python2.3/site-packages/bar
/usr/local/lib/python2.3/site-packages/foo
\end{verbatim}

Note that \file{bletch} is omitted because it doesn't exist; the
\file{bar} directory precedes the \file{foo} directory because
\file{bar.pth} comes alphabetically before \file{foo.pth}; and
\file{spam} is omitted because it is not mentioned in either path
configuration file.

After these path manipulations, an attempt is made to import a module
named \module{sitecustomize}\refmodindex{sitecustomize}, which can
perform arbitrary site-specific customizations.  If this import fails
with an \exception{ImportError} exception, it is silently ignored.

Note that for some non-\UNIX{} systems, \code{sys.prefix} and
\code{sys.exec_prefix} are empty, and the path manipulations are
skipped; however the import of
\module{sitecustomize}\refmodindex{sitecustomize} is still attempted.

\section{Standard Module \sectcode{user}}
\label{module-user}
\stmodindex{user}
\indexii{.pythonrc.py}{file}
\indexiii{user}{configuration}{file}

As a policy, Python doesn't run user-specified code on startup of
Python programs.  (Only interactive sessions execute the script
specified in the \code{PYTHONSTARTUP} environment variable if it exists).

However, some programs or sites may find it convenient to allow users
to have a standard customization file, which gets run when a program
requests it.  This module implements such a mechanism.  A program
that wishes to use the mechanism must execute the statement

\begin{verbatim}
import user
\end{verbatim}

The \code{user} module looks for a file \file{.pythonrc.py} in the user's
home directory and if it can be opened, exececutes it (using
\code{execfile()}) in its own (i.e. the module \code{user}'s) global
namespace.  Errors during this phase are not caught; that's up to the
program that imports the \code{user} module, if it wishes.  The home
directory is assumed to be named by the \code{HOME} environment
variable; if this is not set, the current directory is used.

The user's \file{.pythonrc.py} could conceivably test for
\code{sys.version} if it wishes to do different things depending on
the Python version.

A warning to users: be very conservative in what you place in your
\file{.pythonrc.py} file.  Since you don't know which programs will
use it, changing the behavior of standard modules or functions is
generally not a good idea.

A suggestion for programmers who wish to use this mechanism: a simple
way to let users specify options for your package is to have them
define variables in their \file{.pythonrc.py} file that you test in
your module.  For example, a module \code{spam} that has a verbosity
level can look for a variable \code{user.spam_verbose}, as follows:

\bcode\begin{verbatim}
import user
try:
    verbose = user.spam_verbose  # user's verbosity preference
except AttributeError:
    verbose = 0                  # default verbosity
\end{verbatim}\ecode

Programs with extensive customization needs are better off reading a
program-specific customization file.

Programs with security or privacy concerns should \emph{not} import
this module; a user can easily break into a a program by placing
arbitrary code in the \file{.pythonrc.py} file.

Modules for general use should \emph{not} import this module; it may
interfere with the operation of the importing program.

\begin{seealso}
\seemodule{site}{site-wide customization mechanism}
\refstmodindex{site}
\end{seealso}

\section{\module{__builtin__} ---
         Built-in objects}

\declaremodule[builtin]{builtin}{__builtin__}
\modulesynopsis{The module that provides the built-in namespace.}


This module provides direct access to all `built-in' identifiers of
Python; for example, \code{__builtin__.open} is the full name for the
built-in function \function{open()}.  See chapter~\ref{builtin},
``Built-in Objects.''

This module is not normally accessed explicitly by most applications,
but can be useful in modules that provide objects with the same name
as a built-in value, but in which the built-in of that name is also
needed.  For example, in a module that wants to implement an
\function{open()} function that wraps the built-in \function{open()},
this module can be used directly:

\begin{verbatim}
import __builtin__

def open(path):
    f = __builtin__.open(path, 'r')
    return UpperCaser(f)

class UpperCaser:
    '''Wrapper around a file that converts output to upper-case.'''

    def __init__(self, f):
        self._f = f

    def read(self, count=-1):
        return self._f.read(count).upper()

    # ...
\end{verbatim}

As an implementation detail, most modules have the name
\code{__builtins__} (note the \character{s}) made available as part of
their globals.  The value of \code{__builtins__} is normally either
this module or the value of this modules's \member{__dict__}
attribute.  Since this is an implementation detail, it may not be used
by alternate implementations of Python.
		% really __builtin__
\section{\module{__main__} ---
         Top-level script environment.}
\declaremodule[main]{builtin}{__main__}

\modulesynopsis{The environment where the top-level script is run.}

This module represents the (otherwise anonymous) scope in which the
interpreter's main program executes --- commands read either from
standard input or from a script file.
			% really __main__

\chapter{String Services}

The modules described in this chapter provide a wide range of string
manipulation operations.  Here's an overview:

\begin{description}

\item[string]
--- Common string operations.

\item[re]
--- New Perl-style regular expression search and match operations.

\item[regex]
--- Regular expression search and match operations.

\item[regsub]
--- Substitution and splitting operations that use regular expressions.

\item[struct]
--- Interpret strings as packed binary data.

\item[StringIO]
--- Read and write strings as if they were files.

\item[soundex]
--- Compute hash values for English words.

\end{description}
		% String Services
\section{\module{string} ---
         Common string operations}

\declaremodule{standard}{string}
\modulesynopsis{Common string operations.}

The \module{string} module contains a number of useful constants and classes,
as well as some deprecated legacy functions that are also available as methods
on strings.  See the module \refmodule{re}\refstmodindex{re} for string
functions based on regular expressions.

\subsection{String constants}

The constants defined in this module are:

\begin{datadesc}{ascii_letters}
  The concatenation of the \constant{ascii_lowercase} and
  \constant{ascii_uppercase} constants described below.  This value is
  not locale-dependent.
\end{datadesc}

\begin{datadesc}{ascii_lowercase}
  The lowercase letters \code{'abcdefghijklmnopqrstuvwxyz'}.  This
  value is not locale-dependent and will not change.
\end{datadesc}

\begin{datadesc}{ascii_uppercase}
  The uppercase letters \code{'ABCDEFGHIJKLMNOPQRSTUVWXYZ'}.  This
  value is not locale-dependent and will not change.
\end{datadesc}

\begin{datadesc}{digits}
  The string \code{'0123456789'}.
\end{datadesc}

\begin{datadesc}{hexdigits}
  The string \code{'0123456789abcdefABCDEF'}.
\end{datadesc}

\begin{datadesc}{letters}
  The concatenation of the strings \constant{lowercase} and
  \constant{uppercase} described below.  The specific value is
  locale-dependent, and will be updated when
  \function{locale.setlocale()} is called.
\end{datadesc}

\begin{datadesc}{lowercase}
  A string containing all the characters that are considered lowercase
  letters.  On most systems this is the string
  \code{'abcdefghijklmnopqrstuvwxyz'}.  Do not change its definition ---
  the effect on the routines \function{upper()} and
  \function{swapcase()} is undefined.  The specific value is
  locale-dependent, and will be updated when
  \function{locale.setlocale()} is called.
\end{datadesc}

\begin{datadesc}{octdigits}
  The string \code{'01234567'}.
\end{datadesc}

\begin{datadesc}{punctuation}
  String of \ASCII{} characters which are considered punctuation
  characters in the \samp{C} locale.
\end{datadesc}

\begin{datadesc}{printable}
  String of characters which are considered printable.  This is a
  combination of \constant{digits}, \constant{letters},
  \constant{punctuation}, and \constant{whitespace}.
\end{datadesc}

\begin{datadesc}{uppercase}
  A string containing all the characters that are considered uppercase
  letters.  On most systems this is the string
  \code{'ABCDEFGHIJKLMNOPQRSTUVWXYZ'}.  Do not change its definition ---
  the effect on the routines \function{lower()} and
  \function{swapcase()} is undefined.  The specific value is
  locale-dependent, and will be updated when
  \function{locale.setlocale()} is called.
\end{datadesc}

\begin{datadesc}{whitespace}
  A string containing all characters that are considered whitespace.
  On most systems this includes the characters space, tab, linefeed,
  return, formfeed, and vertical tab.  Do not change its definition ---
  the effect on the routines \function{strip()} and \function{split()}
  is undefined.
\end{datadesc}

\subsection{Template strings}

Templates are Unicode strings that can be used to provide string substitutions
as described in \pep{292}.  There is a \class{Template} class that is a
subclass of \class{unicode}, overriding the default \method{__mod__()} method.
Instead of the normal \samp{\%}-based substitutions, Template strings support
\samp{\$}-based substitutions, using the following rules:

\begin{itemize}
\item \samp{\$\$} is an escape; it is replaced with a single \samp{\$}.

\item \samp{\$identifier} names a substitution placeholder matching a mapping
       key of "identifier".  By default, "identifier" must spell a Python
       identifier.  The first non-identifier character after the \samp{\$}
       character terminates this placeholder specification.

\item \samp{\$\{identifier\}} is equivalent to \samp{\$identifier}.  It is
      required when valid identifier characters follow the placeholder but are
      not part of the placeholder, such as "\$\{noun\}ification".
\end{itemize}

Any other appearance of \samp{\$} in the string will result in a
\exception{ValueError} being raised.

\versionadded{2.4}

Template strings are used just like normal strings, in that the modulus
operator is used to interpolate a dictionary of values into a Template string,
for example:

\begin{verbatim}
>>> from string import Template
>>> s = Template('$who likes $what')
>>> print s % dict(who='tim', what='kung pao')
tim likes kung pao
>>> Template('Give $who $100') % dict(who='tim')
Traceback (most recent call last):
[...]
ValueError: Invalid placeholder at index 10
\end{verbatim}

There is also a \class{SafeTemplate} class, derived from \class{Template}
which acts the same as \class{Template}, except that if placeholders are
missing in the interpolation dictionary, no \exception{KeyError} will be
raised.  Instead the original placeholder (with or without the braces, as
appropriate) will be used:

\begin{verbatim}
>>> from string import SafeTemplate
>>> s = SafeTemplate('$who likes $what for ${meal}')
>>> print s % dict(who='tim')
tim likes $what for ${meal}
\end{verbatim}

The values in the mapping will automatically be converted to Unicode strings,
using the built-in \function{unicode()} function, which will be called without
optional arguments \var{encoding} or \var{errors}.

Advanced usage: you can derive subclasses of \class{Template} or
\class{SafeTemplate} to use application-specific placeholder rules.  To do
this, you override the class attribute \member{pattern}; the value must be a
compiled regular expression object with four named capturing groups.  The
capturing groups correspond to the rules given above, along with the invalid
placeholder rule:

\begin{itemize}
\item \var{escaped} -- This group matches the escape sequence, \samp{\$\$},
      in the default pattern.
\item \var{named} -- This group matches the unbraced placeholder name; it
      should not include the \samp{\$} in capturing group.
\item \var{braced} -- This group matches the brace delimited placeholder name;
      it should not include either the \samp{\$} or braces in the capturing
      group.
\item \var{bogus} -- This group matches any other \samp{\$}.  It usually just
      matches a single \samp{\$} and should appear last.
\end{itemize}

\subsection{String functions}

The following functions are available to operate on string and Unicode
objects.  They are not available as string methods.

\begin{funcdesc}{capwords}{s}
  Split the argument into words using \function{split()}, capitalize
  each word using \function{capitalize()}, and join the capitalized
  words using \function{join()}.  Note that this replaces runs of
  whitespace characters by a single space, and removes leading and
  trailing whitespace.
\end{funcdesc}

\begin{funcdesc}{maketrans}{from, to}
  Return a translation table suitable for passing to
  \function{translate()} or \function{regex.compile()}, that will map
  each character in \var{from} into the character at the same position
  in \var{to}; \var{from} and \var{to} must have the same length.

  \warning{Don't use strings derived from \constant{lowercase}
  and \constant{uppercase} as arguments; in some locales, these don't have
  the same length.  For case conversions, always use
  \function{lower()} and \function{upper()}.}
\end{funcdesc}

\subsection{Deprecated string functions}

The following list of functions are also defined as methods of string and
Unicode objects; see ``String Methods'' (section
\ref{string-methods}) for more information on those.  You should consider
these functions as deprecated, although they will not be removed until Python
3.0.  The functions defined in this module are:

\begin{funcdesc}{atof}{s}
  \deprecated{2.0}{Use the \function{float()} built-in function.}
  Convert a string to a floating point number.  The string must have
  the standard syntax for a floating point literal in Python,
  optionally preceded by a sign (\samp{+} or \samp{-}).  Note that
  this behaves identical to the built-in function
  \function{float()}\bifuncindex{float} when passed a string.

  \note{When passing in a string, values for NaN\index{NaN}
  and Infinity\index{Infinity} may be returned, depending on the
  underlying C library.  The specific set of strings accepted which
  cause these values to be returned depends entirely on the C library
  and is known to vary.}
\end{funcdesc}

\begin{funcdesc}{atoi}{s\optional{, base}}
  \deprecated{2.0}{Use the \function{int()} built-in function.}
  Convert string \var{s} to an integer in the given \var{base}.  The
  string must consist of one or more digits, optionally preceded by a
  sign (\samp{+} or \samp{-}).  The \var{base} defaults to 10.  If it
  is 0, a default base is chosen depending on the leading characters
  of the string (after stripping the sign): \samp{0x} or \samp{0X}
  means 16, \samp{0} means 8, anything else means 10.  If \var{base}
  is 16, a leading \samp{0x} or \samp{0X} is always accepted, though
  not required.  This behaves identically to the built-in function
  \function{int()} when passed a string.  (Also note: for a more
  flexible interpretation of numeric literals, use the built-in
  function \function{eval()}\bifuncindex{eval}.)
\end{funcdesc}

\begin{funcdesc}{atol}{s\optional{, base}}
  \deprecated{2.0}{Use the \function{long()} built-in function.}
  Convert string \var{s} to a long integer in the given \var{base}.
  The string must consist of one or more digits, optionally preceded
  by a sign (\samp{+} or \samp{-}).  The \var{base} argument has the
  same meaning as for \function{atoi()}.  A trailing \samp{l} or
  \samp{L} is not allowed, except if the base is 0.  Note that when
  invoked without \var{base} or with \var{base} set to 10, this
  behaves identical to the built-in function
  \function{long()}\bifuncindex{long} when passed a string.
\end{funcdesc}

\begin{funcdesc}{capitalize}{word}
  Return a copy of \var{word} with only its first character capitalized.
\end{funcdesc}

\begin{funcdesc}{expandtabs}{s\optional{, tabsize}}
  Expand tabs in a string replacing them by one or more spaces,
  depending on the current column and the given tab size.  The column
  number is reset to zero after each newline occurring in the string.
  This doesn't understand other non-printing characters or escape
  sequences.  The tab size defaults to 8.
\end{funcdesc}

\begin{funcdesc}{find}{s, sub\optional{, start\optional{,end}}}
  Return the lowest index in \var{s} where the substring \var{sub} is
  found such that \var{sub} is wholly contained in
  \code{\var{s}[\var{start}:\var{end}]}.  Return \code{-1} on failure.
  Defaults for \var{start} and \var{end} and interpretation of
  negative values is the same as for slices.
\end{funcdesc}

\begin{funcdesc}{rfind}{s, sub\optional{, start\optional{, end}}}
  Like \function{find()} but find the highest index.
\end{funcdesc}

\begin{funcdesc}{index}{s, sub\optional{, start\optional{, end}}}
  Like \function{find()} but raise \exception{ValueError} when the
  substring is not found.
\end{funcdesc}

\begin{funcdesc}{rindex}{s, sub\optional{, start\optional{, end}}}
  Like \function{rfind()} but raise \exception{ValueError} when the
  substring is not found.
\end{funcdesc}

\begin{funcdesc}{count}{s, sub\optional{, start\optional{, end}}}
  Return the number of (non-overlapping) occurrences of substring
  \var{sub} in string \code{\var{s}[\var{start}:\var{end}]}.
  Defaults for \var{start} and \var{end} and interpretation of
  negative values are the same as for slices.
\end{funcdesc}

\begin{funcdesc}{lower}{s}
  Return a copy of \var{s}, but with upper case letters converted to
  lower case.
\end{funcdesc}

\begin{funcdesc}{split}{s\optional{, sep\optional{, maxsplit}}}
  Return a list of the words of the string \var{s}.  If the optional
  second argument \var{sep} is absent or \code{None}, the words are
  separated by arbitrary strings of whitespace characters (space, tab, 
  newline, return, formfeed).  If the second argument \var{sep} is
  present and not \code{None}, it specifies a string to be used as the 
  word separator.  The returned list will then have one more item
  than the number of non-overlapping occurrences of the separator in
  the string.  The optional third argument \var{maxsplit} defaults to
  0.  If it is nonzero, at most \var{maxsplit} number of splits occur,
  and the remainder of the string is returned as the final element of
  the list (thus, the list will have at most \code{\var{maxsplit}+1}
  elements).

  The behavior of split on an empty string depends on the value of \var{sep}.
  If \var{sep} is not specified, or specified as \code{None}, the result will
  be an empty list.  If \var{sep} is specified as any string, the result will
  be a list containing one element which is an empty string.
\end{funcdesc}

\begin{funcdesc}{rsplit}{s\optional{, sep\optional{, maxsplit}}}
  Return a list of the words of the string \var{s}, scanning \var{s}
  from the end.  To all intents and purposes, the resulting list of
  words is the same as returned by \function{split()}, except when the
  optional third argument \var{maxsplit} is explicitly specified and
  nonzero.  When \var{maxsplit} is nonzero, at most \var{maxsplit}
  number of splits -- the \emph{rightmost} ones -- occur, and the remainder
  of the string is returned as the first element of the list (thus, the
  list will have at most \code{\var{maxsplit}+1} elements).
  \versionadded{2.4}
\end{funcdesc}

\begin{funcdesc}{splitfields}{s\optional{, sep\optional{, maxsplit}}}
  This function behaves identically to \function{split()}.  (In the
  past, \function{split()} was only used with one argument, while
  \function{splitfields()} was only used with two arguments.)
\end{funcdesc}

\begin{funcdesc}{join}{words\optional{, sep}}
  Concatenate a list or tuple of words with intervening occurrences of 
  \var{sep}.  The default value for \var{sep} is a single space
  character.  It is always true that
  \samp{string.join(string.split(\var{s}, \var{sep}), \var{sep})}
  equals \var{s}.
\end{funcdesc}

\begin{funcdesc}{joinfields}{words\optional{, sep}}
  This function behaves identically to \function{join()}.  (In the past, 
  \function{join()} was only used with one argument, while
  \function{joinfields()} was only used with two arguments.)
  Note that there is no \method{joinfields()} method on string
  objects; use the \method{join()} method instead.
\end{funcdesc}

\begin{funcdesc}{lstrip}{s\optional{, chars}}
Return a copy of the string with leading characters removed.  If
\var{chars} is omitted or \code{None}, whitespace characters are
removed.  If given and not \code{None}, \var{chars} must be a string;
the characters in the string will be stripped from the beginning of
the string this method is called on.
\versionchanged[The \var{chars} parameter was added.  The \var{chars}
parameter cannot be passed in earlier 2.2 versions]{2.2.3}
\end{funcdesc}

\begin{funcdesc}{rstrip}{s\optional{, chars}}
Return a copy of the string with trailing characters removed.  If
\var{chars} is omitted or \code{None}, whitespace characters are
removed.  If given and not \code{None}, \var{chars} must be a string;
the characters in the string will be stripped from the end of the
string this method is called on.
\versionchanged[The \var{chars} parameter was added.  The \var{chars}
parameter cannot be passed in earlier 2.2 versions]{2.2.3}
\end{funcdesc}

\begin{funcdesc}{strip}{s\optional{, chars}}
Return a copy of the string with leading and trailing characters
removed.  If \var{chars} is omitted or \code{None}, whitespace
characters are removed.  If given and not \code{None}, \var{chars}
must be a string; the characters in the string will be stripped from
the both ends of the string this method is called on.
\versionchanged[The \var{chars} parameter was added.  The \var{chars}
parameter cannot be passed in earlier 2.2 versions]{2.2.3}
\end{funcdesc}

\begin{funcdesc}{swapcase}{s}
  Return a copy of \var{s}, but with lower case letters
  converted to upper case and vice versa.
\end{funcdesc}

\begin{funcdesc}{translate}{s, table\optional{, deletechars}}
  Delete all characters from \var{s} that are in \var{deletechars} (if 
  present), and then translate the characters using \var{table}, which 
  must be a 256-character string giving the translation for each
  character value, indexed by its ordinal.
\end{funcdesc}

\begin{funcdesc}{upper}{s}
  Return a copy of \var{s}, but with lower case letters converted to
  upper case.
\end{funcdesc}

\begin{funcdesc}{ljust}{s, width}
\funcline{rjust}{s, width}
\funcline{center}{s, width}
  These functions respectively left-justify, right-justify and center
  a string in a field of given width.  They return a string that is at
  least \var{width} characters wide, created by padding the string
  \var{s} with spaces until the given width on the right, left or both
  sides.  The string is never truncated.
\end{funcdesc}

\begin{funcdesc}{zfill}{s, width}
  Pad a numeric string on the left with zero digits until the given
  width is reached.  Strings starting with a sign are handled
  correctly.
\end{funcdesc}

\begin{funcdesc}{replace}{str, old, new\optional{, maxreplace}}
  Return a copy of string \var{str} with all occurrences of substring
  \var{old} replaced by \var{new}.  If the optional argument
  \var{maxreplace} is given, the first \var{maxreplace} occurrences are
  replaced.
\end{funcdesc}

\section{\module{re} ---
         New Perl-style regular expression search and match operations.}
\declaremodule{standard}{re}
\moduleauthor{Andrew M. Kuchling}{akuchling@acm.org}
\sectionauthor{Andrew M. Kuchling}{akuchling@acm.org}


\modulesynopsis{New Perl-style regular expression search and match
operations.}


This module provides regular expression matching operations similar to
those found in Perl.  It's 8-bit clean: the strings being processed
may contain both null bytes and characters whose high bit is set.  Regular
expression patterns may not contain null bytes, but they may contain
characters with the high bit set.  The \module{re} module is always
available.

Regular expressions use the backslash character (\character{\e}) to
indicate special forms or to allow special characters to be used
without invoking their special meaning.  This collides with Python's
usage of the same character for the same purpose in string literals;
for example, to match a literal backslash, one might have to write
\code{'\e\e\e\e'} as the pattern string, because the regular expression
must be \samp{\e\e}, and each backslash must be expressed as
\samp{\e\e} inside a regular Python string literal. 

The solution is to use Python's raw string notation for regular
expression patterns; backslashes are not handled in any special way in
a string literal prefixed with \character{r}.  So \code{r"\e n"} is a
two-character string containing \character{\e} and \character{n},
while \code{"\e n"} is a one-character string containing a newline.
Usually patterns will be expressed in Python code using this raw
string notation.

\subsection{Regular Expression Syntax \label{re-syntax}}

A regular expression (or RE) specifies a set of strings that matches
it; the functions in this module let you check if a particular string
matches a given regular expression (or if a given regular expression
matches a particular string, which comes down to the same thing).

Regular expressions can be concatenated to form new regular
expressions; if \emph{A} and \emph{B} are both regular expressions,
then \emph{AB} is also an regular expression.  If a string \emph{p}
matches A and another string \emph{q} matches B, the string \emph{pq}
will match AB.  Thus, complex expressions can easily be constructed
from simpler primitive expressions like the ones described here.  For
details of the theory and implementation of regular expressions,
consult the Friedl book referenced below, or almost any textbook about
compiler construction.

A brief explanation of the format of regular expressions follows.  
%For further information and a gentler presentation, consult XXX somewhere.

Regular expressions can contain both special and ordinary characters.
Most ordinary characters, like \character{A}, \character{a}, or \character{0},
are the simplest regular expressions; they simply match themselves.  
You can concatenate ordinary characters, so \regexp{last} matches the
string \code{'last'}.  (In the rest of this section, we'll write RE's in
\regexp{this special style}, usually without quotes, and strings to be
matched \code{'in single quotes'}.)

Some characters, like \character{|} or \character{(}, are special.  Special
characters either stand for classes of ordinary characters, or affect
how the regular expressions around them are interpreted.

The special characters are:
% define these since they're used twice:
\newcommand{\MyLeftMargin}{0.7in}
\newcommand{\MyLabelWidth}{0.65in}

\begin{list}{}{\leftmargin \MyLeftMargin \labelwidth \MyLabelWidth}

\item[\character{.}] (Dot.)  In the default mode, this matches any
character except a newline.  If the \constant{DOTALL} flag has been
specified, this matches any character including a newline.

\item[\character{\^}] (Caret.)  Matches the start of the string, and in
\constant{MULTILINE} mode also matches immediately after each newline.

\item[\character{\$}] Matches the end of the string, and in
\constant{MULTILINE} mode also matches before a newline.
\regexp{foo} matches both 'foo' and 'foobar', while the regular
expression \regexp{foo\$} matches only 'foo'.

\item[\character{*}] Causes the resulting RE to
match 0 or more repetitions of the preceding RE, as many repetitions
as are possible.  \regexp{ab*} will
match 'a', 'ab', or 'a' followed by any number of 'b's.

\item[\character{+}] Causes the
resulting RE to match 1 or more repetitions of the preceding RE.
\regexp{ab+} will match 'a' followed by any non-zero number of 'b's; it
will not match just 'a'.

\item[\character{?}] Causes the resulting RE to
match 0 or 1 repetitions of the preceding RE.  \regexp{ab?} will
match either 'a' or 'ab'.
\item[\code{*?}, \code{+?}, \code{??}] The \character{*}, \character{+}, and
\character{?} qualifiers are all \dfn{greedy}; they match as much text as
possible.  Sometimes this behaviour isn't desired; if the RE
\regexp{<.*>} is matched against \code{'<H1>title</H1>'}, it will match the
entire string, and not just \code{'<H1>'}.
Adding \character{?} after the qualifier makes it perform the match in
\dfn{non-greedy} or \dfn{minimal} fashion; as \emph{few} characters as
possible will be matched.  Using \regexp{.*?} in the previous
expression will match only \code{'<H1>'}.

\item[\code{\{\var{m},\var{n}\}}] Causes the resulting RE to match from
\var{m} to \var{n} repetitions of the preceding RE, attempting to
match as many repetitions as possible.   For example, \regexp{a\{3,5\}}  
will match from 3 to 5 \character{a} characters.  Omitting \var{m} is the same
as specifying 0 for the lower bound; omitting \var{n} specifies an
infinite upper bound. 

\item[\code{\{\var{m},\var{n}\}?}] Causes the resulting RE to
match from \var{m} to \var{n} repetitions of the preceding RE,
attempting to match as \emph{few} repetitions as possible.  This is
the non-greedy version of the previous qualifier.  For example, on the
6-character string \code{'aaaaaa'}, \regexp{a\{3,5\}} will match 5
\character{a} characters, while \regexp{a\{3,5\}?} will only match 3
characters.

\item[\character{\e}] Either escapes special characters (permitting
you to match characters like \character{*}, \character{?}, and so
forth), or signals a special sequence; special sequences are discussed
below.

If you're not using a raw string to
express the pattern, remember that Python also uses the
backslash as an escape sequence in string literals; if the escape
sequence isn't recognized by Python's parser, the backslash and
subsequent character are included in the resulting string.  However,
if Python would recognize the resulting sequence, the backslash should
be repeated twice.  This is complicated and hard to understand, so
it's highly recommended that you use raw strings for all but the
simplest expressions.

\item[\code{[]}] Used to indicate a set of characters.  Characters can
be listed individually, or a range of characters can be indicated by
giving two characters and separating them by a \character{-}.  Special
characters are not active inside sets.  For example, \regexp{[akm\$]}
will match any of the characters \character{a}, \character{k},
\character{m}, or \character{\$}; \regexp{[a-z]}
will match any lowercase letter, and \code{[a-zA-Z0-9]} matches any
letter or digit.  Character classes such as \code{\e w} or \code {\e
S} (defined below) are also acceptable inside a range.  If you want to
include a \character{]} or a \character{-} inside a set, precede it with a
backslash, or place it as the first character.  The 
pattern \regexp{[]]} will match \code{']'}, for example.  

You can match the characters not within a range by \dfn{complementing}
the set.  This is indicated by including a
\character{\^} as the first character of the set; \character{\^} elsewhere will
simply match the \character{\^} character.  For example, \regexp{[\^5]}
will match any character except \character{5}.

\item[\character{|}]\code{A|B}, where A and B can be arbitrary REs,
creates a regular expression that will match either A or B.  This can
be used inside groups (see below) as well.  To match a literal \character{|},
use \regexp{\e|}, or enclose it inside a character class, as in  \regexp{[|]}.

\item[\code{(...)}] Matches whatever regular expression is inside the
parentheses, and indicates the start and end of a group; the contents
of a group can be retrieved after a match has been performed, and can
be matched later in the string with the \regexp{\e \var{number}} special
sequence, described below.  To match the literals \character{(} or
\character{')}, use \regexp{\e(} or \regexp{\e)}, or enclose them
inside a character class: \regexp{[(] [)]}.

\item[\code{(?...)}] This is an extension notation (a \character{?}
following a \character{(} is not meaningful otherwise).  The first
character after the \character{?} 
determines what the meaning and further syntax of the construct is.
Extensions usually do not create a new group;
\regexp{(?P<\var{name}>...)} is the only exception to this rule.
Following are the currently supported extensions.

\item[\code{(?iLmsx)}] (One or more letters from the set \character{i},
\character{L}, \character{m}, \character{s}, \character{x}.)  The group matches
the empty string; the letters set the corresponding flags
(\constant{re.I}, \constant{re.L}, \constant{re.M}, \constant{re.S},
\constant{re.X}) for the entire regular expression.  This is useful if
you wish to include the flags as part of the regular expression, instead
of passing a \var{flag} argument to the \function{compile()} function. 

\item[\code{(?:...)}] A non-grouping version of regular parentheses.
Matches whatever regular expression is inside the parentheses, but the
substring matched by the 
group \emph{cannot} be retrieved after performing a match or
referenced later in the pattern. 

\item[\code{(?P<\var{name}>...)}] Similar to regular parentheses, but
the substring matched by the group is accessible via the symbolic group
name \var{name}.  Group names must be valid Python identifiers.  A
symbolic group is also a numbered group, just as if the group were not
named.  So the group named 'id' in the example above can also be
referenced as the numbered group 1.

For example, if the pattern is
\regexp{(?P<id>[a-zA-Z_]\e w*)}, the group can be referenced by its
name in arguments to methods of match objects, such as \code{m.group('id')}
or \code{m.end('id')}, and also by name in pattern text
(e.g. \regexp{(?P=id)}) and replacement text (e.g. \code{\e g<id>}).

\item[\code{(?P=\var{name})}] Matches whatever text was matched by the
earlier group named \var{name}.

\item[\code{(?\#...)}] A comment; the contents of the parentheses are
simply ignored.

\item[\code{(?=...)}] Matches if \regexp{...} matches next, but doesn't
consume any of the string.  This is called a lookahead assertion.  For
example, \regexp{Isaac (?=Asimov)} will match \code{'Isaac~'} only if it's
followed by \code{'Asimov'}.

\item[\code{(?!...)}] Matches if \regexp{...} doesn't match next.  This
is a negative lookahead assertion.  For example,
\regexp{Isaac (?!Asimov)} will match \code{'Isaac~'} only if it's \emph{not}
followed by \code{'Asimov'}.

\end{list}

The special sequences consist of \character{\e} and a character from the
list below.  If the ordinary character is not on the list, then the
resulting RE will match the second character.  For example,
\regexp{\e\$} matches the character \character{\$}.

\begin{list}{}{\leftmargin \MyLeftMargin \labelwidth \MyLabelWidth}

%
\item[\code{\e \var{number}}] Matches the contents of the group of the
same number.  Groups are numbered starting from 1.  For example,
\regexp{(.+) \e 1} matches \code{'the the'} or \code{'55 55'}, but not
\code{'the end'} (note 
the space after the group).  This special sequence can only be used to
match one of the first 99 groups.  If the first digit of \var{number}
is 0, or \var{number} is 3 octal digits long, it will not be interpreted
as a group match, but as the character with octal value \var{number}.
Inside the \character{[} and \character{]} of a character class, all numeric
escapes are treated as characters. 
%
\item[\code{\e A}] Matches only at the start of the string.
%
\item[\code{\e b}] Matches the empty string, but only at the
beginning or end of a word.  A word is defined as a sequence of
alphanumeric characters, so the end of a word is indicated by
whitespace or a non-alphanumeric character.  Inside a character range,
\regexp{\e b} represents the backspace character, for compatibility with
Python's string literals.
%
\item[\code{\e B}] Matches the empty string, but only when it is
\emph{not} at the beginning or end of a word.
%
\item[\code{\e d}]Matches any decimal digit; this is
equivalent to the set \regexp{[0-9]}.
%
\item[\code{\e D}]Matches any non-digit character; this is
equivalent to the set \regexp{[\^0-9]}.
%
\item[\code{\e s}]Matches any whitespace character; this is
equivalent to the set \regexp{[ \e t\e n\e r\e f\e v]}.
%
\item[\code{\e S}]Matches any non-whitespace character; this is
equivalent to the set \regexp{[\^\ \e t\e n\e r\e f\e v]}.
%
\item[\code{\e w}]When the \constant{LOCALE} flag is not specified,
matches any alphanumeric character; this is equivalent to the set
\regexp{[a-zA-Z0-9_]}.  With \constant{LOCALE}, it will match the set
\regexp{[0-9_]} plus whatever characters are defined as letters for the
current locale.
%
\item[\code{\e W}]When the \constant{LOCALE} flag is not specified,
matches any non-alphanumeric character; this is equivalent to the set
\regexp{[\^a-zA-Z0-9_]}.   With \constant{LOCALE}, it will match any
character not in the set \regexp{[0-9_]}, and not defined as a letter
for the current locale.

\item[\code{\e Z}]Matches only at the end of the string.
%

\item[\code{\e \e}] Matches a literal backslash.

\end{list}


\subsection{Module Contents}
\nodename{Contents of Module re}

The module defines the following functions and constants, and an exception:


\begin{funcdesc}{compile}{pattern\optional{, flags}}
  Compile a regular expression pattern into a regular expression
  object, which can be used for matching using its \function{match()} and
  \function{search()} methods, described below.  

  The expression's behaviour can be modified by specifying a
  \var{flags} value.  Values can be any of the following variables,
  combined using bitwise OR (the \code{|} operator).

The sequence

\begin{verbatim}
prog = re.compile(pat)
result = prog.match(str)
\end{verbatim}

is equivalent to

\begin{verbatim}
result = re.match(pat, str)
\end{verbatim}

but the version using \function{compile()} is more efficient when the
expression will be used several times in a single program.
%(The compiled version of the last pattern passed to
%\function{regex.match()} or \function{regex.search()} is cached, so
%programs that use only a single regular expression at a time needn't
%worry about compiling regular expressions.)
\end{funcdesc}

\begin{datadesc}{I}
\dataline{IGNORECASE}
Perform case-insensitive matching; expressions like \regexp{[A-Z]} will match
lowercase letters, too.  This is not affected by the current locale.
\end{datadesc}

\begin{datadesc}{L}
\dataline{LOCALE}
Make \regexp{\e w}, \regexp{\e W}, \regexp{\e b},
\regexp{\e B}, dependent on the current locale. 
\end{datadesc}

\begin{datadesc}{M}
\dataline{MULTILINE}
When specified, the pattern character \character{\^} matches at the
beginning of the string and at the beginning of each line
(immediately following each newline); and the pattern character
\character{\$} matches at the end of the string and at the end of each line
(immediately preceding each newline).
By default, \character{\^} matches only at the beginning of the string, and
\character{\$} only at the end of the string and immediately before the
newline (if any) at the end of the string. 
\end{datadesc}

\begin{datadesc}{S}
\dataline{DOTALL}
Make the \character{.} special character match any character at all, including a
newline; without this flag, \character{.} will match anything \emph{except}
a newline.
\end{datadesc}

\begin{datadesc}{X}
\dataline{VERBOSE}
This flag allows you to write regular expressions that look nicer.
Whitespace within the pattern is ignored, 
except when in a character class or preceded by an unescaped
backslash, and, when a line contains a \character{\#} neither in a character
class or preceded by an unescaped backslash, all characters from the
leftmost such \character{\#} through the end of the line are ignored.
% XXX should add an example here
\end{datadesc}


\begin{funcdesc}{escape}{string}
  Return \var{string} with all non-alphanumerics backslashed; this is
  useful if you want to match an arbitrary literal string that may have
  regular expression metacharacters in it.
\end{funcdesc}

\begin{funcdesc}{match}{pattern, string\optional{, flags}}
  If zero or more characters at the beginning of \var{string} match
  the regular expression \var{pattern}, return a corresponding
  \class{MatchObject} instance.  Return \code{None} if the string does not
  match the pattern; note that this is different from a zero-length
  match.
\end{funcdesc}

\begin{funcdesc}{search}{pattern, string\optional{, flags}}
  Scan through \var{string} looking for a location where the regular
  expression \var{pattern} produces a match, and return a
  corresponding \class{MatchObject} instance.
  Return \code{None} if no
  position in the string matches the pattern; note that this is
  different from finding a zero-length match at some point in the string.
\end{funcdesc}

\begin{funcdesc}{split}{pattern, string, \optional{, maxsplit\code{ = 0}}}
  Split \var{string} by the occurrences of \var{pattern}.  If
  capturing parentheses are used in \var{pattern}, then the text of all
  groups in the pattern are also returned as part of the resulting list.
  If \var{maxsplit} is nonzero, at most \var{maxsplit} splits
  occur, and the remainder of the string is returned as the final
  element of the list.  (Incompatibility note: in the original Python
  1.5 release, \var{maxsplit} was ignored.  This has been fixed in
  later releases.)
%
\begin{verbatim}
>>> re.split('\W+', 'Words, words, words.')
['Words', 'words', 'words', '']
>>> re.split('(\W+)', 'Words, words, words.')
['Words', ', ', 'words', ', ', 'words', '.', '']
>>> re.split('\W+', 'Words, words, words.', 1)
['Words', 'words, words.']
\end{verbatim}
%
  This function combines and extends the functionality of
  the old \function{regsub.split()} and \function{regsub.splitx()}.
\end{funcdesc}

\begin{funcdesc}{findall}{pattern, string}
\versionadded{1.5.2}
Return a list of all non-overlapping matches of \var{pattern} in
\var{string}.  If one or more groups are present in the pattern,
return a list of groups; this will be a list of tuples if the pattern
has more than one group.  Empty matches are included in the result.
\end{funcdesc}

\begin{funcdesc}{sub}{pattern, repl, string\optional{, count\code{ = 0}}}
Return the string obtained by replacing the leftmost non-overlapping
occurrences of \var{pattern} in \var{string} by the replacement
\var{repl}.  If the pattern isn't found, \var{string} is returned
unchanged.  \var{repl} can be a string or a function; if a function,
it is called for every non-overlapping occurance of \var{pattern}.
The function takes a single match object argument, and returns the
replacement string.  For example:
%
\begin{verbatim}
>>> def dashrepl(matchobj):
....    if matchobj.group(0) == '-': return ' '
....    else: return '-'
>>> re.sub('-{1,2}', dashrepl, 'pro----gram-files')
'pro--gram files'
\end{verbatim}
%
The pattern may be a string or a 
regex object; if you need to specify
regular expression flags, you must use a regex object, or use
embedded modifiers in a pattern; e.g.
\samp{sub("(?i)b+", "x", "bbbb BBBB")} returns \code{'x x'}.

The optional argument \var{count} is the maximum number of pattern
occurrences to be replaced; \var{count} must be a non-negative integer, and
the default value of 0 means to replace all occurrences.

Empty matches for the pattern are replaced only when not adjacent to a
previous match, so \samp{sub('x*', '-', 'abc')} returns \code{'-a-b-c-'}.

If \var{repl} is a string, any backslash escapes in it are processed.
That is, \samp{\e n} is converted to a single newline character,
\samp{\e r} is converted to a linefeed, and so forth.  Unknown escapes
such as \samp{\e j} are left alone.  Backreferences, such as \samp{\e 6}, are
replaced with the substring matched by group 6 in the pattern. 

In addition to character escapes and backreferences as described
above, \samp{\e g<name>} will use the substring matched by the group
named \samp{name}, as defined by the \regexp{(?P<name>...)} syntax.
\samp{\e g<number>} uses the corresponding group number; \samp{\e
g<2>} is therefore equivalent to \samp{\e 2}, but isn't ambiguous in a
replacement such as \samp{\e g<2>0}.  \samp{\e 20} would be
interpreted as a reference to group 20, not a reference to group 2
followed by the literal character \character{0}.  
\end{funcdesc}

\begin{funcdesc}{subn}{pattern, repl, string\optional{, count\code{ = 0}}}
Perform the same operation as \function{sub()}, but return a tuple
\code{(\var{new_string}, \var{number_of_subs_made})}.
\end{funcdesc}

\begin{excdesc}{error}
  Exception raised when a string passed to one of the functions here
  is not a valid regular expression (e.g., unmatched parentheses) or
  when some other error occurs during compilation or matching.  It is
  never an error if a string contains no match for a pattern.
\end{excdesc}


\subsection{Regular Expression Objects \label{re-objects}}

Compiled regular expression objects support the following methods and
attributes:

\begin{methoddesc}[RegexObject]{match}{string\optional{, pos}\optional{,
                                       endpos}}
  If zero or more characters at the beginning of \var{string} match
  this regular expression, return a corresponding
  \class{MatchObject} instance.  Return \code{None} if the string does not
  match the pattern; note that this is different from a zero-length
  match.
  
  The optional second parameter \var{pos} gives an index in the string
  where the search is to start; it defaults to \code{0}.  This is not
  completely equivalent to slicing the string; the \code{'\^'} pattern
  character matches at the real beginning of the string and at positions
  just after a newline, but not necessarily at the index where the search
  is to start.

  The optional parameter \var{endpos} limits how far the string will
  be searched; it will be as if the string is \var{endpos} characters
  long, so only the characters from \var{pos} to \var{endpos} will be
  searched for a match.
\end{methoddesc}

\begin{methoddesc}[RegexObject]{search}{string\optional{, pos}\optional{,
                                        endpos}}
  Scan through \var{string} looking for a location where this regular
  expression produces a match.  Return \code{None} if no
  position in the string matches the pattern; note that this is
  different from finding a zero-length match at some point in the string.
  
  The optional \var{pos} and \var{endpos} parameters have the same
  meaning as for the \method{match()} method.
\end{methoddesc}

\begin{methoddesc}[RegexObject]{split}{string, \optional{,
                                       maxsplit\code{ = 0}}}
Identical to the \function{split()} function, using the compiled pattern.
\end{methoddesc}

\begin{methoddesc}[RegexObject]{findall}{string}
Identical to the \function{findall()} function, using the compiled pattern.
\end{methoddesc}

\begin{methoddesc}[RegexObject]{sub}{repl, string\optional{, count\code{ = 0}}}
Identical to the \function{sub()} function, using the compiled pattern.
\end{methoddesc}

\begin{methoddesc}[RegexObject]{subn}{repl, string\optional{,
                                      count\code{ = 0}}}
Identical to the \function{subn()} function, using the compiled pattern.
\end{methoddesc}


\begin{memberdesc}[RegexObject]{flags}
The flags argument used when the regex object was compiled, or
\code{0} if no flags were provided.
\end{memberdesc}

\begin{memberdesc}[RegexObject]{groupindex}
A dictionary mapping any symbolic group names defined by 
\regexp{(?P<\var{id}>)} to group numbers.  The dictionary is empty if no
symbolic groups were used in the pattern.
\end{memberdesc}

\begin{memberdesc}[RegexObject]{pattern}
The pattern string from which the regex object was compiled.
\end{memberdesc}


\subsection{Match Objects \label{match-objects}}

\class{MatchObject} instances support the following methods and attributes:

\begin{methoddesc}[MatchObject]{group}{\optional{group1, group2, ...}}
Returns one or more subgroups of the match.  If there is a single
argument, the result is a single string; if there are
multiple arguments, the result is a tuple with one item per argument.
Without arguments, \var{group1} defaults to zero (i.e. the whole match
is returned).
If a \var{groupN} argument is zero, the corresponding return value is the
entire matching string; if it is in the inclusive range [1..99], it is
the string matching the the corresponding parenthesized group.  If a
group number is negative or larger than the number of groups defined
in the pattern, an \exception{IndexError} exception is raised.
If a group is contained in a part of the pattern that did not match,
the corresponding result is \code{None}.  If a group is contained in a 
part of the pattern that matched multiple times, the last match is
returned.

If the regular expression uses the \regexp{(?P<\var{name}>...)} syntax,
the \var{groupN} arguments may also be strings identifying groups by
their group name.  If a string argument is not used as a group name in 
the pattern, an \exception{IndexError} exception is raised.

A moderately complicated example:

\begin{verbatim}
m = re.match(r"(?P<int>\d+)\.(\d*)", '3.14')
\end{verbatim}

After performing this match, \code{m.group(1)} is \code{'3'}, as is
\code{m.group('int')}, and \code{m.group(2)} is \code{'14'}.
\end{methoddesc}

\begin{methoddesc}[MatchObject]{groups}{\optional{default}}
Return a tuple containing all the subgroups of the match, from 1 up to
however many groups are in the pattern.  The \var{default} argument is
used for groups that did not participate in the match; it defaults to
\code{None}.  (Incompatibility note: in the original Python 1.5
release, if the tuple was one element long, a string would be returned
instead.  In later versions (from 1.5.1 on), a singleton tuple is
returned in such cases.)
\end{methoddesc}

\begin{methoddesc}[MatchObject]{groupdict}{\optional{default}}
Return a dictionary containing all the \emph{named} subgroups of the
match, keyed by the subgroup name.  The \var{default} argument is
used for groups that did not participate in the match; it defaults to
\code{None}.
\end{methoddesc}

\begin{methoddesc}[MatchObject]{start}{\optional{group}}
\funcline{end}{\optional{group}}
Return the indices of the start and end of the substring
matched by \var{group}; \var{group} defaults to zero (meaning the whole
matched substring).
Return \code{None} if \var{group} exists but
did not contribute to the match.  For a match object
\var{m}, and a group \var{g} that did contribute to the match, the
substring matched by group \var{g} (equivalent to
\code{\var{m}.group(\var{g})}) is

\begin{verbatim}
m.string[m.start(g):m.end(g)]
\end{verbatim}

Note that
\code{m.start(\var{group})} will equal \code{m.end(\var{group})} if
\var{group} matched a null string.  For example, after \code{\var{m} =
re.search('b(c?)', 'cba')}, \code{\var{m}.start(0)} is 1,
\code{\var{m}.end(0)} is 2, \code{\var{m}.start(1)} and
\code{\var{m}.end(1)} are both 2, and \code{\var{m}.start(2)} raises
an \exception{IndexError} exception.
\end{methoddesc}

\begin{methoddesc}[MatchObject]{span}{\optional{group}}
For \class{MatchObject} \var{m}, return the 2-tuple
\code{(\var{m}.start(\var{group}), \var{m}.end(\var{group}))}.
Note that if \var{group} did not contribute to the match, this is
\code{(None, None)}.  Again, \var{group} defaults to zero.
\end{methoddesc}

\begin{memberdesc}[MatchObject]{pos}
The value of \var{pos} which was passed to the
\function{search()} or \function{match()} function.  This is the index into
the string at which the regex engine started looking for a match. 
\end{memberdesc}

\begin{memberdesc}[MatchObject]{endpos}
The value of \var{endpos} which was passed to the
\function{search()} or \function{match()} function.  This is the index into
the string beyond which the regex engine will not go.
\end{memberdesc}

\begin{memberdesc}[MatchObject]{re}
The regular expression object whose \method{match()} or
\method{search()} method produced this \class{MatchObject} instance.
\end{memberdesc}

\begin{memberdesc}[MatchObject]{string}
The string passed to \function{match()} or \function{search()}.
\end{memberdesc}

\begin{seealso}
\seetext{Jeffrey Friedl, \emph{Mastering Regular Expressions},
O'Reilly.  The Python material in this book dates from before the
\module{re} module, but it covers writing good regular expression
patterns in great detail.}
\end{seealso}


\section{\module{regex} ---
         Regular expression search and match operations.}
\declaremodule{builtin}{regex}

\modulesynopsis{Regular expression search and match operations.}


This module provides regular expression matching operations similar to
those found in Emacs.

\strong{Obsolescence note:}
This module is obsolete as of Python version 1.5; it is still being
maintained because much existing code still uses it.  All new code in
need of regular expressions should use the new
\code{re}\refstmodindex{re} module, which supports the more powerful
and regular Perl-style regular expressions.  Existing code should be
converted.  The standard library module
\code{reconvert}\refstmodindex{reconvert} helps in converting
\code{regex} style regular expressions to \code{re}\refstmodindex{re}
style regular expressions.  (For more conversion help, see Andrew
Kuchling's\index{Kuchling, Andrew} ``\module{regex-to-re} HOWTO'' at
\url{http://www.python.org/doc/howto/regex-to-re/}.)

By default the patterns are Emacs-style regular expressions
(with one exception).  There is
a way to change the syntax to match that of several well-known
\UNIX{} utilities.  The exception is that Emacs' \samp{\e s}
pattern is not supported, since the original implementation references
the Emacs syntax tables.

This module is 8-bit clean: both patterns and strings may contain null
bytes and characters whose high bit is set.

\strong{Please note:} There is a little-known fact about Python string
literals which means that you don't usually have to worry about
doubling backslashes, even though they are used to escape special
characters in string literals as well as in regular expressions.  This
is because Python doesn't remove backslashes from string literals if
they are followed by an unrecognized escape character.
\emph{However}, if you want to include a literal \dfn{backslash} in a
regular expression represented as a string literal, you have to
\emph{quadruple} it or enclose it in a singleton character class.
E.g.\  to extract \LaTeX\ \samp{\e section\{\textrm{\ldots}\}} headers
from a document, you can use this pattern:
\code{'[\e ]section\{\e (.*\e )\}'}.  \emph{Another exception:}
the escape sequece \samp{\e b} is significant in string literals
(where it means the ASCII bell character) as well as in Emacs regular
expressions (where it stands for a word boundary), so in order to
search for a word boundary, you should use the pattern \code{'\e \e b'}.
Similarly, a backslash followed by a digit 0-7 should be doubled to
avoid interpretation as an octal escape.

\subsection{Regular Expressions}

A regular expression (or RE) specifies a set of strings that matches
it; the functions in this module let you check if a particular string
matches a given regular expression (or if a given regular expression
matches a particular string, which comes down to the same thing).

Regular expressions can be concatenated to form new regular
expressions; if \emph{A} and \emph{B} are both regular expressions,
then \emph{AB} is also an regular expression.  If a string \emph{p}
matches A and another string \emph{q} matches B, the string \emph{pq}
will match AB.  Thus, complex expressions can easily be constructed
from simpler ones like the primitives described here.  For details of
the theory and implementation of regular expressions, consult almost
any textbook about compiler construction.

% XXX The reference could be made more specific, say to 
% "Compilers: Principles, Techniques and Tools", by Alfred V. Aho, 
% Ravi Sethi, and Jeffrey D. Ullman, or some FA text.   

A brief explanation of the format of regular expressions follows.

Regular expressions can contain both special and ordinary characters.
Ordinary characters, like '\code{A}', '\code{a}', or '\code{0}', are
the simplest regular expressions; they simply match themselves.  You
can concatenate ordinary characters, so '\code{last}' matches the
characters 'last'.  (In the rest of this section, we'll write RE's in
\code{this special font}, usually without quotes, and strings to be
matched 'in single quotes'.)

Special characters either stand for classes of ordinary characters, or
affect how the regular expressions around them are interpreted.

The special characters are:
\begin{itemize}
\item[\code{.}] (Dot.)  Matches any character except a newline.
\item[\code{\^}] (Caret.)  Matches the start of the string.
\item[\code{\$}] Matches the end of the string.  
\code{foo} matches both 'foo' and 'foobar', while the regular
expression '\code{foo\$}' matches only 'foo'.
\item[\code{*}] Causes the resulting RE to
match 0 or more repetitions of the preceding RE.  \code{ab*} will
match 'a', 'ab', or 'a' followed by any number of 'b's.
\item[\code{+}] Causes the
resulting RE to match 1 or more repetitions of the preceding RE.
\code{ab+} will match 'a' followed by any non-zero number of 'b's; it
will not match just 'a'.
\item[\code{?}] Causes the resulting RE to
match 0 or 1 repetitions of the preceding RE.  \code{ab?} will
match either 'a' or 'ab'.

\item[\code{\e}] Either escapes special characters (permitting you to match
characters like '*?+\&\$'), or signals a special sequence; special
sequences are discussed below.  Remember that Python also uses the
backslash as an escape sequence in string literals; if the escape
sequence isn't recognized by Python's parser, the backslash and
subsequent character are included in the resulting string.  However,
if Python would recognize the resulting sequence, the backslash should
be repeated twice.  

\item[\code{[]}] Used to indicate a set of characters.  Characters can
be listed individually, or a range is indicated by giving two
characters and separating them by a '-'.  Special characters are
not active inside sets.  For example, \code{[akm\$]}
will match any of the characters 'a', 'k', 'm', or '\$'; \code{[a-z]} will
match any lowercase letter.  

If you want to include a \code{]} inside a
set, it must be the first character of the set; to include a \code{-},
place it as the first or last character. 

Characters \emph{not} within a range can be matched by including a
\code{\^} as the first character of the set; \code{\^} elsewhere will
simply match the '\code{\^}' character.  
\end{itemize}

The special sequences consist of '\code{\e}' and a character
from the list below.  If the ordinary character is not on the list,
then the resulting RE will match the second character.  For example,
\code{\e\$} matches the character '\$'.  Ones where the backslash
should be doubled in string literals are indicated.

\begin{itemize}
\item[\code{\e|}]\code{A\e|B}, where A and B can be arbitrary REs,
creates a regular expression that will match either A or B.  This can
be used inside groups (see below) as well.
%
\item[\code{\e( \e)}] Indicates the start and end of a group; the
contents of a group can be matched later in the string with the
\code{\e [1-9]} special sequence, described next.
\end{itemize}

\begin{fulllineitems}
\item[\code{\e \e 1, ... \e \e 7, \e 8, \e 9}]
Matches the contents of the group of the same
number.  For example, \code{\e (.+\e ) \e \e 1} matches 'the the' or
'55 55', but not 'the end' (note the space after the group).  This
special sequence can only be used to match one of the first 9 groups;
groups with higher numbers can be matched using the \code{\e v}
sequence.  (\code{\e 8} and \code{\e 9} don't need a double backslash
because they are not octal digits.)
\end{fulllineitems}

\begin{itemize}
\item[\code{\e \e b}] Matches the empty string, but only at the
beginning or end of a word.  A word is defined as a sequence of
alphanumeric characters, so the end of a word is indicated by
whitespace or a non-alphanumeric character.
%
\item[\code{\e B}] Matches the empty string, but when it is \emph{not} at the
beginning or end of a word.
%
\item[\code{\e v}] Must be followed by a two digit decimal number, and
matches the contents of the group of the same number.  The group
number must be between 1 and 99, inclusive.
%
\item[\code{\e w}]Matches any alphanumeric character; this is
equivalent to the set \code{[a-zA-Z0-9]}.
%
\item[\code{\e W}] Matches any non-alphanumeric character; this is
equivalent to the set \code{[\^a-zA-Z0-9]}.
\item[\code{\e <}] Matches the empty string, but only at the beginning of a
word.  A word is defined as a sequence of alphanumeric characters, so
the end of a word is indicated by whitespace or a non-alphanumeric 
character.
\item[\code{\e >}] Matches the empty string, but only at the end of a
word.

\item[\code{\e \e \e \e}] Matches a literal backslash.

% In Emacs, the following two are start of buffer/end of buffer.  In
% Python they seem to be synonyms for ^$.
\item[\code{\e `}] Like \code{\^}, this only matches at the start of the
string.
\item[\code{\e \e '}] Like \code{\$}, this only matches at the end of
the string.
% end of buffer
\end{itemize}

\subsection{Module Contents}
\nodename{Contents of Module regex}

The module defines these functions, and an exception:


\begin{funcdesc}{match}{pattern, string}
  Return how many characters at the beginning of \var{string} match
  the regular expression \var{pattern}.  Return \code{-1} if the
  string does not match the pattern (this is different from a
  zero-length match!).
\end{funcdesc}

\begin{funcdesc}{search}{pattern, string}
  Return the first position in \var{string} that matches the regular
  expression \var{pattern}.  Return \code{-1} if no position in the string
  matches the pattern (this is different from a zero-length match
  anywhere!).
\end{funcdesc}

\begin{funcdesc}{compile}{pattern\optional{, translate}}
  Compile a regular expression pattern into a regular expression
  object, which can be used for matching using its \code{match()} and
  \code{search()} methods, described below.  The optional argument
  \var{translate}, if present, must be a 256-character string
  indicating how characters (both of the pattern and of the strings to
  be matched) are translated before comparing them; the \var{i}-th
  element of the string gives the translation for the character with
  \ASCII{} code \var{i}.  This can be used to implement
  case-insensitive matching; see the \code{casefold} data item below.

  The sequence

\begin{verbatim}
prog = regex.compile(pat)
result = prog.match(str)
\end{verbatim}
%
is equivalent to

\begin{verbatim}
result = regex.match(pat, str)
\end{verbatim}

but the version using \code{compile()} is more efficient when multiple
regular expressions are used concurrently in a single program.  (The
compiled version of the last pattern passed to \code{regex.match()} or
\code{regex.search()} is cached, so programs that use only a single
regular expression at a time needn't worry about compiling regular
expressions.)
\end{funcdesc}

\begin{funcdesc}{set_syntax}{flags}
  Set the syntax to be used by future calls to \code{compile()},
  \code{match()} and \code{search()}.  (Already compiled expression
  objects are not affected.)  The argument is an integer which is the
  OR of several flag bits.  The return value is the previous value of
  the syntax flags.  Names for the flags are defined in the standard 
  module \code{regex_syntax}\refstmodindex{regex_syntax}; read the
  file \file{regex_syntax.py} for more information.
\end{funcdesc}

\begin{funcdesc}{get_syntax}{}
  Returns the current value of the syntax flags as an integer.
\end{funcdesc}

\begin{funcdesc}{symcomp}{pattern\optional{, translate}}
This is like \code{compile()}, but supports symbolic group names: if a
parenthesis-enclosed group begins with a group name in angular
brackets, e.g. \code{'\e(<id>[a-z][a-z0-9]*\e)'}, the group can
be referenced by its name in arguments to the \code{group()} method of
the resulting compiled regular expression object, like this:
\code{p.group('id')}.  Group names may contain alphanumeric characters
and \code{'_'} only.
\end{funcdesc}

\begin{excdesc}{error}
  Exception raised when a string passed to one of the functions here
  is not a valid regular expression (e.g., unmatched parentheses) or
  when some other error occurs during compilation or matching.  (It is
  never an error if a string contains no match for a pattern.)
\end{excdesc}

\begin{datadesc}{casefold}
A string suitable to pass as the \var{translate} argument to
\code{compile()} to map all upper case characters to their lowercase
equivalents.
\end{datadesc}

\noindent
Compiled regular expression objects support these methods:

\setindexsubitem{(regex method)}
\begin{funcdesc}{match}{string\optional{, pos}}
  Return how many characters at the beginning of \var{string} match
  the compiled regular expression.  Return \code{-1} if the string
  does not match the pattern (this is different from a zero-length
  match!).
  
  The optional second parameter, \var{pos}, gives an index in the string
  where the search is to start; it defaults to \code{0}.  This is not
  completely equivalent to slicing the string; the \code{'\^'} pattern
  character matches at the real beginning of the string and at positions
  just after a newline, not necessarily at the index where the search
  is to start.
\end{funcdesc}

\begin{funcdesc}{search}{string\optional{, pos}}
  Return the first position in \var{string} that matches the regular
  expression \code{pattern}.  Return \code{-1} if no position in the
  string matches the pattern (this is different from a zero-length
  match anywhere!).
  
  The optional second parameter has the same meaning as for the
  \code{match()} method.
\end{funcdesc}

\begin{funcdesc}{group}{index, index, ...}
This method is only valid when the last call to the \code{match()}
or \code{search()} method found a match.  It returns one or more
groups of the match.  If there is a single \var{index} argument,
the result is a single string; if there are multiple arguments, the
result is a tuple with one item per argument.  If the \var{index} is
zero, the corresponding return value is the entire matching string; if
it is in the inclusive range [1..99], it is the string matching the
the corresponding parenthesized group (using the default syntax,
groups are parenthesized using \code{{\e}(} and \code{{\e})}).  If no
such group exists, the corresponding result is \code{None}.

If the regular expression was compiled by \code{symcomp()} instead of
\code{compile()}, the \var{index} arguments may also be strings
identifying groups by their group name.
\end{funcdesc}

\noindent
Compiled regular expressions support these data attributes:

\setindexsubitem{(regex attribute)}

\begin{datadesc}{regs}
When the last call to the \code{match()} or \code{search()} method found a
match, this is a tuple of pairs of indexes corresponding to the
beginning and end of all parenthesized groups in the pattern.  Indices
are relative to the string argument passed to \code{match()} or
\code{search()}.  The 0-th tuple gives the beginning and end or the
whole pattern.  When the last match or search failed, this is
\code{None}.
\end{datadesc}

\begin{datadesc}{last}
When the last call to the \code{match()} or \code{search()} method found a
match, this is the string argument passed to that method.  When the
last match or search failed, this is \code{None}.
\end{datadesc}

\begin{datadesc}{translate}
This is the value of the \var{translate} argument to
\code{regex.compile()} that created this regular expression object.  If
the \var{translate} argument was omitted in the \code{regex.compile()}
call, this is \code{None}.
\end{datadesc}

\begin{datadesc}{givenpat}
The regular expression pattern as passed to \code{compile()} or
\code{symcomp()}.
\end{datadesc}

\begin{datadesc}{realpat}
The regular expression after stripping the group names for regular
expressions compiled with \code{symcomp()}.  Same as \code{givenpat}
otherwise.
\end{datadesc}

\begin{datadesc}{groupindex}
A dictionary giving the mapping from symbolic group names to numerical
group indexes for regular expressions compiled with \code{symcomp()}.
\code{None} otherwise.
\end{datadesc}

\section{Standard Module \sectcode{regsub}}

\stmodindex{regsub}
This module defines a number of functions useful for working with
regular expressions (see built-in module \code{regex}).

\renewcommand{\indexsubitem}{(in module regsub)}
\begin{funcdesc}{sub}{pat\, repl\, str}
Replace the first occurrence of pattern \var{pat} in string
\var{str} by replacement \var{repl}.  If the pattern isn't found,
the string is returned unchanged.  The pattern may be a string or an
already compiled pattern.  The replacement may contain references
\samp{\e \var{digit}} to subpatterns and escaped backslashes.
\end{funcdesc}

\begin{funcdesc}{gsub}{pat\, repl\, str}
Replace all (non-overlapping) occurrences of pattern \var{pat} in
string \var{str} by replacement \var{repl}.  The same rules as for
\code{sub()} apply.  Empty matches for the pattern are replaced only
when not adjacent to a previous match, so e.g.
\code{gsub('', '-', 'abc')} returns \code{'-a-b-c-'}.
\end{funcdesc}

\begin{funcdesc}{split}{str\, pat}
Split the string \var{str} in fields separated by delimiters matching
the pattern \var{pat}, and return a list containing the fields.  Only
non-empty matches for the pattern are considered, so e.g.
\code{split('a:b', ':*')} returns \code{['a', 'b']} and
\code{split('abc', '')} returns \code{['abc']}.
\end{funcdesc}

\section{\module{struct} ---
         Interpret strings as packed binary data}
\declaremodule{builtin}{struct}

\modulesynopsis{Interpret strings as packed binary data.}

\indexii{C}{structures}
\indexiii{packing}{binary}{data}

This module performs conversions between Python values and C
structs represented as Python strings.  It uses \dfn{format strings}
(explained below) as compact descriptions of the lay-out of the C
structs and the intended conversion to/from Python values.  This can
be used in handling binary data stored in files or from network
connections, among other sources.

The module defines the following exception and functions:


\begin{excdesc}{error}
  Exception raised on various occasions; argument is a string
  describing what is wrong.
\end{excdesc}

\begin{funcdesc}{pack}{fmt, v1, v2, \textrm{\ldots}}
  Return a string containing the values
  \code{\var{v1}, \var{v2}, \textrm{\ldots}} packed according to the given
  format.  The arguments must match the values required by the format
  exactly.
\end{funcdesc}

\begin{funcdesc}{pack_into}{fmt, buffer, offset, v1, v2, \moreargs}
  Pack the values \code{\var{v1}, \var{v2}, \textrm{\ldots}} according to the given
  format, write the packed bytes into the writable \var{buffer} starting at
  \var{offset}.
  Note that the offset is not an optional argument.
\end{funcdesc}

\begin{funcdesc}{unpack}{fmt, string}
  Unpack the string (presumably packed by \code{pack(\var{fmt},
  \textrm{\ldots})}) according to the given format.  The result is a
  tuple even if it contains exactly one item.  The string must contain
  exactly the amount of data required by the format
  (\code{len(\var{string})} must equal \code{calcsize(\var{fmt})}).
\end{funcdesc}

\begin{funcdesc}{unpack_from}{fmt, buffer\optional{,offset \code{= 0}}}
  Unpack the \var{buffer} according to tthe given format.
  The result is a tuple even if it contains exactly one item. The
  \var{buffer} must contain at least the amount of data required by the
  format (\code{len(buffer[offset:])} must be at least
  \code{calcsize(\var{fmt})}).
\end{funcdesc}

\begin{funcdesc}{calcsize}{fmt}
  Return the size of the struct (and hence of the string)
  corresponding to the given format.
\end{funcdesc}

Format characters have the following meaning; the conversion between
C and Python values should be obvious given their types:

\begin{tableiv}{c|l|l|c}{samp}{Format}{C Type}{Python}{Notes}
  \lineiv{x}{pad byte}{no value}{}
  \lineiv{c}{\ctype{char}}{string of length 1}{}
  \lineiv{b}{\ctype{signed char}}{integer}{}
  \lineiv{B}{\ctype{unsigned char}}{integer}{}
  \lineiv{t}{\ctype{_Bool}}{bool}{(1)}
  \lineiv{h}{\ctype{short}}{integer}{}
  \lineiv{H}{\ctype{unsigned short}}{integer}{}
  \lineiv{i}{\ctype{int}}{integer}{}
  \lineiv{I}{\ctype{unsigned int}}{long}{}
  \lineiv{l}{\ctype{long}}{integer}{}
  \lineiv{L}{\ctype{unsigned long}}{long}{}
  \lineiv{q}{\ctype{long long}}{long}{(2)}
  \lineiv{Q}{\ctype{unsigned long long}}{long}{(2)}
  \lineiv{f}{\ctype{float}}{float}{}
  \lineiv{d}{\ctype{double}}{float}{}
  \lineiv{s}{\ctype{char[]}}{string}{}
  \lineiv{p}{\ctype{char[]}}{string}{}
  \lineiv{P}{\ctype{void *}}{integer}{}
\end{tableiv}

\noindent
Notes:

\begin{description}
\item[(1)]
  The \character{t} conversion code corresponds to the \ctype{_Bool} type
  defined by C99. If this type is not available, it is simulated using a
  \ctype{char}. In standard mode, it is always represented by one byte.
  \versionadded{2.6}
\item[(2)]
  The \character{q} and \character{Q} conversion codes are available in
  native mode only if the platform C compiler supports C \ctype{long long},
  or, on Windows, \ctype{__int64}.  They are always available in standard
  modes.
  \versionadded{2.2}
\end{description}


A format character may be preceded by an integral repeat count.  For
example, the format string \code{'4h'} means exactly the same as
\code{'hhhh'}.

Whitespace characters between formats are ignored; a count and its
format must not contain whitespace though.

For the \character{s} format character, the count is interpreted as the
size of the string, not a repeat count like for the other format
characters; for example, \code{'10s'} means a single 10-byte string, while
\code{'10c'} means 10 characters.  For packing, the string is
truncated or padded with null bytes as appropriate to make it fit.
For unpacking, the resulting string always has exactly the specified
number of bytes.  As a special case, \code{'0s'} means a single, empty
string (while \code{'0c'} means 0 characters).

The \character{p} format character encodes a "Pascal string", meaning
a short variable-length string stored in a fixed number of bytes.
The count is the total number of bytes stored.  The first byte stored is
the length of the string, or 255, whichever is smaller.  The bytes
of the string follow.  If the string passed in to \function{pack()} is too
long (longer than the count minus 1), only the leading count-1 bytes of the
string are stored.  If the string is shorter than count-1, it is padded
with null bytes so that exactly count bytes in all are used.  Note that
for \function{unpack()}, the \character{p} format character consumes count
bytes, but that the string returned can never contain more than 255
characters.

For the \character{I}, \character{L}, \character{q} and \character{Q}
format characters, the return value is a Python long integer.

For the \character{P} format character, the return value is a Python
integer or long integer, depending on the size needed to hold a
pointer when it has been cast to an integer type.  A \NULL{} pointer will
always be returned as the Python integer \code{0}. When packing pointer-sized
values, Python integer or long integer objects may be used.  For
example, the Alpha and Merced processors use 64-bit pointer values,
meaning a Python long integer will be used to hold the pointer; other
platforms use 32-bit pointers and will use a Python integer.

For the \character{t} format character, the return value is either
\constant{True} or \constant{False}. When packing, the truth value
of the argument object is used. Either 0 or 1 in the native or standard
bool representation will be packed, and any non-zero value will be True
when unpacking.

By default, C numbers are represented in the machine's native format
and byte order, and properly aligned by skipping pad bytes if
necessary (according to the rules used by the C compiler).

Alternatively, the first character of the format string can be used to
indicate the byte order, size and alignment of the packed data,
according to the following table:

\begin{tableiii}{c|l|l}{samp}{Character}{Byte order}{Size and alignment}
  \lineiii{@}{native}{native}
  \lineiii{=}{native}{standard}
  \lineiii{<}{little-endian}{standard}
  \lineiii{>}{big-endian}{standard}
  \lineiii{!}{network (= big-endian)}{standard}
\end{tableiii}

If the first character is not one of these, \character{@} is assumed.

Native byte order is big-endian or little-endian, depending on the
host system.  For example, Motorola and Sun processors are big-endian;
Intel and DEC processors are little-endian.

Native size and alignment are determined using the C compiler's
\keyword{sizeof} expression.  This is always combined with native byte
order.

Standard size and alignment are as follows: no alignment is required
for any type (so you have to use pad bytes);
\ctype{short} is 2 bytes;
\ctype{int} and \ctype{long} are 4 bytes;
\ctype{long long} (\ctype{__int64} on Windows) is 8 bytes;
\ctype{float} and \ctype{double} are 32-bit and 64-bit
IEEE floating point numbers, respectively.
\ctype{_Bool} is 1 byte.

Note the difference between \character{@} and \character{=}: both use
native byte order, but the size and alignment of the latter is
standardized.

The form \character{!} is available for those poor souls who claim they
can't remember whether network byte order is big-endian or
little-endian.

There is no way to indicate non-native byte order (force
byte-swapping); use the appropriate choice of \character{<} or
\character{>}.

The \character{P} format character is only available for the native
byte ordering (selected as the default or with the \character{@} byte
order character). The byte order character \character{=} chooses to
use little- or big-endian ordering based on the host system. The
struct module does not interpret this as native ordering, so the
\character{P} format is not available.

Examples (all using native byte order, size and alignment, on a
big-endian machine):

\begin{verbatim}
>>> from struct import *
>>> pack('hhl', 1, 2, 3)
'\x00\x01\x00\x02\x00\x00\x00\x03'
>>> unpack('hhl', '\x00\x01\x00\x02\x00\x00\x00\x03')
(1, 2, 3)
>>> calcsize('hhl')
8
\end{verbatim}

Hint: to align the end of a structure to the alignment requirement of
a particular type, end the format with the code for that type with a
repeat count of zero.  For example, the format \code{'llh0l'}
specifies two pad bytes at the end, assuming longs are aligned on
4-byte boundaries.  This only works when native size and alignment are
in effect; standard size and alignment does not enforce any alignment.

\begin{seealso}
  \seemodule{array}{Packed binary storage of homogeneous data.}
  \seemodule{xdrlib}{Packing and unpacking of XDR data.}
\end{seealso}

\subsection{Struct Objects \label{struct-objects}}

The \module{struct} module also defines the following type:

\begin{classdesc}{Struct}{format}
  Return a new Struct object which writes and reads binary data according to
  the format string \var{format}.  Creating a Struct object once and calling
  its methods is more efficient than calling the \module{struct} functions
  with the same format since the format string only needs to be compiled once.

 \versionadded{2.5}
\end{classdesc}

Compiled Struct objects support the following methods and attributes:

\begin{methoddesc}[Struct]{pack}{v1, v2, \moreargs}
  Identical to the \function{pack()} function, using the compiled format.
  (\code{len(result)} will equal \member{self.size}.)
\end{methoddesc}

\begin{methoddesc}[Struct]{pack_into}{buffer, offset, v1, v2, \moreargs}
  Identical to the \function{pack_into()} function, using the compiled format.
\end{methoddesc}

\begin{methoddesc}[Struct]{unpack}{string}
  Identical to the \function{unpack()} function, using the compiled format.
  (\code{len(string)} must equal \member{self.size}).
\end{methoddesc}

\begin{methoddesc}[Struct]{unpack_from}{buffer\optional{,offset
                                              \code{= 0}}}
  Identical to the \function{unpack_from()} function, using the compiled format.
  (\code{len(buffer[offset:])} must be at least \member{self.size}).
\end{methoddesc}

\begin{memberdesc}[Struct]{format}
  The format string used to construct this Struct object.
\end{memberdesc}


\section{Standard Module \sectcode{StringIO}}

\stmodindex{StringIO}

This module implements a file-like class, \code{StringIO},
that reads and writes a string buffer (also known as {\em memory
files}). See the description on file objects for operations.

When a \code{StringIO} object is created, it can be initialized
to an existing string by passing the string to the constructor.
If no string is given, the \code{StringIO} will start empty.

The method \code{getvalue()} can be called to retrieve the
entire contents of the ``file'' at any time.

\section{Built-in Module \sectcode{soundex}}
\label{module-soundex}
\bimodindex{soundex}

\setindexsubitem{(in module soundex)}
The soundex algorithm takes an English word, and returns an
easily-computed hash of it; this hash is intended to be the same for
words that sound alike.  This module provides an interface to the
soundex algorithm.

Note that the soundex algorithm is quite simple-minded, and isn't
perfect by any measure.  Its main purpose is to help looking up names
in databases, when the name may be misspelled --- soundex hashes common
misspellings together.

\begin{funcdesc}{get_soundex}{string}
Return the soundex hash value for a word; it will always be a
6-character string.  \var{string} must contain the word to be hashed,
with no leading whitespace; the case of the word is ignored.
\end{funcdesc}

\begin{funcdesc}{sound_similar}{string1, string2}
Compare the word in \var{string1} with the word in \var{string2}; this
is equivalent to 
\code{get_soundex(\var{string1})} \code{==}
\code{get_soundex(\var{string2})}.
\end{funcdesc}


\chapter{Miscellaneous Services}
\label{misc}

The modules described in this chapter provide miscellaneous services
that are available in all Python versions.  Here's an overview:

\begin{description}

\item[math]
--- Mathematical functions (\function{sin()} etc.).

\item[cmath]
--- Mathematical functions for complex numbers.

\item[whrandom]
--- Floating point pseudo-random number generator.

\item[random]
--- Generate pseudo-random numbers with various common distributions.

\item[rand]
--- Integer pseudo-random number generator (obsolete).

\item[array]
--- Efficient arrays of uniformly typed numeric values.

\end{description}
			% Miscellaneous Services
\section{\module{math} ---
         Mathematical functions}

\declaremodule{builtin}{math}
\modulesynopsis{Mathematical functions (\function{sin()} etc.).}

This module is always available.  It provides access to the
mathematical functions defined by the C standard.

These functions cannot be used with complex numbers; use the functions
of the same name from the \refmodule{cmath} module if you require
support for complex numbers.  The distinction between functions which
support complex numbers and those which don't is made since most users
do not want to learn quite as much mathematics as required to
understand complex numbers.  Receiving an exception instead of a
complex result allows earlier detection of the unexpected complex
number used as a parameter, so that the programmer can determine how
and why it was generated in the first place.

The following functions provided by this module:

\begin{funcdesc}{acos}{x}
Return the arc cosine of \var{x}.
\end{funcdesc}

\begin{funcdesc}{asin}{x}
Return the arc sine of \var{x}.
\end{funcdesc}

\begin{funcdesc}{atan}{x}
Return the arc tangent of \var{x}.
\end{funcdesc}

\begin{funcdesc}{atan2}{y, x}
Return \code{atan(\var{y} / \var{x})}.
\end{funcdesc}

\begin{funcdesc}{ceil}{x}
Return the ceiling of \var{x} as a real.
\end{funcdesc}

\begin{funcdesc}{cos}{x}
Return the cosine of \var{x}.
\end{funcdesc}

\begin{funcdesc}{cosh}{x}
Return the hyperbolic cosine of \var{x}.
\end{funcdesc}

\begin{funcdesc}{exp}{x}
Return \code{e**\var{x}}.
\end{funcdesc}

\begin{funcdesc}{fabs}{x}
Return the absolute value of the real \var{x}.
\end{funcdesc}

\begin{funcdesc}{floor}{x}
Return the floor of \var{x} as a real.
\end{funcdesc}

\begin{funcdesc}{fmod}{x, y}
Return \code{\var{x} \%\ \var{y}}.
\end{funcdesc}

\begin{funcdesc}{frexp}{x}
% Blessed by Tim.
Return the mantissa and exponent of \var{x} as the pair
\code{(\var{m}, \var{e})}.  \var{m} is a float and \var{e} is an
integer such that \code{\var{x} == \var{m} * 2**\var{e}}.
If \var{x} is zero, returns \code{(0.0, 0)}, otherwise
\code{0.5 <= abs(\var{m}) < 1}.
\end{funcdesc}

\begin{funcdesc}{hypot}{x, y}
Return the Euclidean distance, \code{sqrt(\var{x}*\var{x} + \var{y}*\var{y})}.
\end{funcdesc}

\begin{funcdesc}{ldexp}{x, i}
Return \code{\var{x} * (2**\var{i})}.
\end{funcdesc}

\begin{funcdesc}{log}{x}
Return the natural logarithm of \var{x}.
\end{funcdesc}

\begin{funcdesc}{log10}{x}
Return the base-10 logarithm of \var{x}.
\end{funcdesc}

\begin{funcdesc}{modf}{x}
Return the fractional and integer parts of \var{x}.  Both results
carry the sign of \var{x}.  The integer part is returned as a real.
\end{funcdesc}

\begin{funcdesc}{pow}{x, y}
Return \code{\var{x}**\var{y}}.
\end{funcdesc}

\begin{funcdesc}{rint}{x, y}
Return the integer nearest to \var{x} as a real.
(Only available on platforms where this is in the standard C math library.)
\end{funcdesc}

\begin{funcdesc}{sin}{x}
Return the sine of \var{x}.
\end{funcdesc}

\begin{funcdesc}{sinh}{x}
Return the hyperbolic sine of \var{x}.
\end{funcdesc}

\begin{funcdesc}{sqrt}{x}
Return the square root of \var{x}.
\end{funcdesc}

\begin{funcdesc}{tan}{x}
Return the tangent of \var{x}.
\end{funcdesc}

\begin{funcdesc}{tanh}{x}
Return the hyperbolic tangent of \var{x}.
\end{funcdesc}

Note that \function{frexp()} and \function{modf()} have a different
call/return pattern than their C equivalents: they take a single
argument and return a pair of values, rather than returning their
second return value through an `output parameter' (there is no such
thing in Python).

The module also defines two mathematical constants:

\begin{datadesc}{pi}
The mathematical constant \emph{pi}.
\end{datadesc}

\begin{datadesc}{e}
The mathematical constant \emph{e}.
\end{datadesc}

\begin{seealso}
  \seemodule{cmath}{Complex number versions of many of these functions.}
\end{seealso}

\section{\module{cmath} ---
         Mathematical functions for complex numbers}

\declaremodule{builtin}{cmath}
\modulesynopsis{Mathematical functions for complex numbers.}

This module is always available.  It provides access to mathematical
functions for complex numbers.  The functions are:

\begin{funcdesc}{acos}{x}
Return the arc cosine of \var{x}.
There are two branch cuts:
One extends right from 1 along the real axis to \infinity, continuous
from below.
The other extends left from -1 along the real axis to -\infinity,
continuous from above.
\end{funcdesc}

\begin{funcdesc}{acosh}{x}
Return the hyperbolic arc cosine of \var{x}.
There is one branch cut, extending left from 1 along the real axis
to -\infinity, continuous from above.
\end{funcdesc}

\begin{funcdesc}{asin}{x}
Return the arc sine of \var{x}.
This has the same branch cuts as \function{acos()}.
\end{funcdesc}

\begin{funcdesc}{asinh}{x}
Return the hyperbolic arc sine of \var{x}.
There are two branch cuts, extending left from \plusminus\code{1j} to
\plusminus-\infinity\code{j}, both continuous from above.
These branch cuts should be considered a bug to be corrected in a
future release.
The correct branch cuts should extend along the imaginary axis,
one from \code{1j} up to \infinity\code{j} and continuous from the
right, and one from -\code{1j} down to -\infinity\code{j} and
continuous from the left.
\end{funcdesc}

\begin{funcdesc}{atan}{x}
Return the arc tangent of \var{x}.
There are two branch cuts:
One extends from \code{1j} along the imaginary axis to
\infinity\code{j}, continuous from the left.
The other extends from -\code{1j} along the imaginary axis to
-\infinity\code{j}, continuous from the left.
(This should probably be changed so the upper cut becomes continuous
from the other side.)
\end{funcdesc}

\begin{funcdesc}{atanh}{x}
Return the hyperbolic arc tangent of \var{x}.
There are two branch cuts:
One extends from 1 along the real axis to \infinity, continuous
from above.
The other extends from -1 along the real axis to -\infinity,
continuous from above.
(This should probably be changed so the right cut becomes continuous from
the other side.)
\end{funcdesc}

\begin{funcdesc}{cos}{x}
Return the cosine of \var{x}.
\end{funcdesc}

\begin{funcdesc}{cosh}{x}
Return the hyperbolic cosine of \var{x}.
\end{funcdesc}

\begin{funcdesc}{exp}{x}
Return the exponential value \code{e**\var{x}}.
\end{funcdesc}

\begin{funcdesc}{log}{x\optional{, base}}
Returns the logarithm of \var{x} to the given \var{base}.
If the \var{base} is not specified, returns the natural logarithm of \var{x}.
There is one branch cut, from 0 along the negative real axis to
-\infinity, continuous from above.
\versionchanged[\var{base} argument added]{2.4}
\end{funcdesc}

\begin{funcdesc}{log10}{x}
Return the base-10 logarithm of \var{x}.
This has the same branch cut as \function{log()}.
\end{funcdesc}

\begin{funcdesc}{sin}{x}
Return the sine of \var{x}.
\end{funcdesc}

\begin{funcdesc}{sinh}{x}
Return the hyperbolic sine of \var{x}.
\end{funcdesc}

\begin{funcdesc}{sqrt}{x}
Return the square root of \var{x}.
This has the same branch cut as \function{log()}.
\end{funcdesc}

\begin{funcdesc}{tan}{x}
Return the tangent of \var{x}.
\end{funcdesc}

\begin{funcdesc}{tanh}{x}
Return the hyperbolic tangent of \var{x}.
\end{funcdesc}

The module also defines two mathematical constants:

\begin{datadesc}{pi}
The mathematical constant \emph{pi}, as a real.
\end{datadesc}

\begin{datadesc}{e}
The mathematical constant \emph{e}, as a real.
\end{datadesc}

Note that the selection of functions is similar, but not identical, to
that in module \refmodule{math}\refbimodindex{math}.  The reason for having
two modules is that some users aren't interested in complex numbers,
and perhaps don't even know what they are.  They would rather have
\code{math.sqrt(-1)} raise an exception than return a complex number.
Also note that the functions defined in \module{cmath} always return a
complex number, even if the answer can be expressed as a real number
(in which case the complex number has an imaginary part of zero).

A note on branch cuts: They are curves along which the given function
fails to be continuous.  They are a necessary feature of many complex
functions.  It is assumed that if you need to compute with complex
functions, you will understand about branch cuts.  Consult almost any
(not too elementary) book on complex variables for enlightenment.  For
information of the proper choice of branch cuts for numerical
purposes, a good reference should be the following:

\begin{seealso}
  \seetext{Kahan, W:  Branch cuts for complex elementary functions;
           or, Much ado about nothing's sign bit.  In Iserles, A.,
           and Powell, M. (eds.), \citetitle{The state of the art in
           numerical analysis}. Clarendon Press (1987) pp165-211.}
\end{seealso}

\section{Standard Module \sectcode{whrandom}}
\label{module-whrandom}
\stmodindex{whrandom}

This module implements a Wichmann-Hill pseudo-random number generator
class that is also named \code{whrandom}.  Instances of the
\code{whrandom} class have the following methods:

\begin{funcdesc}{choice}{seq}
Chooses a random element from the non-empty sequence \var{seq} and returns it.
\end{funcdesc}

\begin{funcdesc}{randint}{a, b}
Returns a random integer \var{N} such that \code{\var{a}<=\var{N}<=\var{b}}.
\end{funcdesc}

\begin{funcdesc}{random}{}
Returns the next random floating point number in the range [0.0 ... 1.0).
\end{funcdesc}

\begin{funcdesc}{seed}{x, y, z}
Initializes the random number generator from the integers
\var{x},
\var{y}
and
\var{z}.
When the module is first imported, the random number is initialized
using values derived from the current time.
\end{funcdesc}

\begin{funcdesc}{uniform}{a, b}
Returns a random real number \var{N} such that \code{\var{a}<=\var{N}<\var{b}}.
\end{funcdesc}

When imported, the \code{whrandom} module also creates an instance of
the \code{whrandom} class, and makes the methods of that instance
available at the module level.  Therefore one can write either 
\code{N = whrandom.random()} or:
\begin{verbatim}
generator = whrandom.whrandom()
N = generator.random()
\end{verbatim}
%
\begin{seealso}
\seemodule{random}{generators for various random distributions}
\seetext{Wichmann, B. A. \& Hill, I. D., ``Algorithm AS 183: 
An efficient and portable pseudo-random number generator'', 
\emph{Applied Statistics} 31 (1982) 188-190}
\end{seealso}

\section{\module{random} ---
         Generate pseudo-random numbers}

\declaremodule{standard}{random}
\modulesynopsis{Generate pseudo-random numbers with various common
                distributions.}


This module implements pseudo-random number generators for various
distributions: on the real line, there are functions to compute normal
or Gaussian, lognormal, negative exponential, gamma, and beta
distributions.  For generating distribution of angles, the circular
uniform and von Mises distributions are available.


The \module{random} module supports the \emph{Random Number
Generator} interface, described in section \ref{rng-objects}.  This
interface of the module, as well as the distribution-specific
functions described below, all use the pseudo-random generator
provided by the \refmodule{whrandom} module.


The following functions are defined to support specific distributions,
and all return real values.  Function parameters are named after the
corresponding variables in the distribution's equation, as used in
common mathematical practice; most of these equations can be found in
any statistics text.  These are expected to become part of the Random
Number Generator interface in a future release.

\begin{funcdesc}{betavariate}{alpha, beta}
Beta distribution.  Conditions on the parameters are
$\var{alpha} > -1$ and $\var{beta} > -1$.
Returned values range between 0 and 1.
\end{funcdesc}

\begin{funcdesc}{cunifvariate}{mean, arc}
Circular uniform distribution.  \var{mean} is the mean angle, and
\var{arc} is the range of the distribution, centered around the mean
angle.  Both values must be expressed in radians, and can range
between 0 and \emph{pi}.  Returned values will range between
$\var{mean} - \var{arc}/2$ and $\var{mean} + \var{arc}/2$.
\end{funcdesc}

\begin{funcdesc}{expovariate}{lambd}
Exponential distribution.  \var{lambd} is 1.0 divided by the desired
mean.  (The parameter would be called ``lambda'', but that is a
reserved word in Python.)  Returned values will range from 0 to
positive infinity.
\end{funcdesc}

\begin{funcdesc}{gamma}{alpha, beta}
Gamma distribution.  (\emph{Not} the gamma function!)  Conditions on
the parameters are $\var{alpha} > -1$ and $\var{beta} > 0$.
\end{funcdesc}

\begin{funcdesc}{gauss}{mu, sigma}
Gaussian distribution.  \var{mu} is the mean, and \var{sigma} is the
standard deviation.  This is slightly faster than the
\function{normalvariate()} function defined below.
\end{funcdesc}

\begin{funcdesc}{lognormvariate}{mu, sigma}
Log normal distribution.  If you take the natural logarithm of this
distribution, you'll get a normal distribution with mean \var{mu} and
standard deviation \var{sigma}.  \var{mu} can have any value, and
\var{sigma} must be greater than zero.  
\end{funcdesc}

\begin{funcdesc}{normalvariate}{mu, sigma}
Normal distribution.  \var{mu} is the mean, and \var{sigma} is the
standard deviation.
\end{funcdesc}

\begin{funcdesc}{vonmisesvariate}{mu, kappa}
\var{mu} is the mean angle, expressed in radians between 0 and 2*\emph{pi},
and \var{kappa} is the concentration parameter, which must be greater
than or equal to zero.  If \var{kappa} is equal to zero, this
distribution reduces to a uniform random angle over the range 0 to
2*\emph{pi}.
\end{funcdesc}

\begin{funcdesc}{paretovariate}{alpha}
Pareto distribution.  \var{alpha} is the shape parameter.
\end{funcdesc}

\begin{funcdesc}{weibullvariate}{alpha, beta}
Weibull distribution.  \var{alpha} is the scale parameter and
\var{beta} is the shape parameter.
\end{funcdesc}

\begin{seealso}
  \seemodule{whrandom}{The standard Python random number generator.}
\end{seealso}


\subsection{The Random Number Generator Interface
            \label{rng-objects}}

% XXX This *must* be updated before a future release!

The \dfn{Random Number Generator} interface describes the methods
which are available for all random number generators.  This will be
enhanced in future releases of Python.

In this release of Python, the modules \refmodule{random},
\refmodule{whrandom}, and instances of the
\class{whrandom.whrandom} class all conform to this interface.


\begin{funcdesc}{choice}{seq}
Chooses a random element from the non-empty sequence \var{seq} and
returns it.
\end{funcdesc}

\begin{funcdesc}{randint}{a, b}
Returns a random integer \var{N} such that
\code{\var{a}<=\var{N}<=\var{b}}.
\end{funcdesc}

\begin{funcdesc}{random}{}
Returns the next random floating point number in the range [0.0
... 1.0).
\end{funcdesc}

\begin{funcdesc}{uniform}{a, b}
Returns a random real number \var{N} such that
\code{\var{a}<=\var{N}<\var{b}}.
\end{funcdesc}

\section{\module{rand} ---
         None}
\declaremodule{standard}{rand}

\modulesynopsis{None}


The \code{rand} module simulates the C library's \code{rand()}
interface, though the results aren't necessarily compatible with any
given library's implementation.  While still supported for
compatibility, the \code{rand} module is now considered obsolete; if
possible, use the \code{whrandom} module instead.


\begin{funcdesc}{choice}{seq}
Returns a random element from the sequence \var{seq}.
\end{funcdesc}

\begin{funcdesc}{rand}{}
Return a random integer between 0 and 32767, inclusive.
\end{funcdesc}

\begin{funcdesc}{srand}{seed}
Set a starting seed value for the random number generator; \var{seed}
can be an arbitrary integer. 
\end{funcdesc}

\begin{seealso}
  \seemodule{random}{Python's interface to random number generators.}
  \seemodule{whrandom}{The random number generator used by default.}
\end{seealso}

\section{Built-in Module \sectcode{array}}
\label{module-array}
\bimodindex{array}
\index{arrays}

This module defines a new object type which can efficiently represent
an array of basic values: characters, integers, floating point
numbers.  Arrays are sequence types and behave very much like lists,
except that the type of objects stored in them is constrained.  The
type is specified at object creation time by using a \dfn{type code},
which is a single character.  The following type codes are defined:

\begin{tableiii}{|c|c|c|}{code}{Typecode}{Type}{Minimal size in bytes}
\lineiii{'c'}{character}{1}
\lineiii{'b'}{signed integer}{1}
\lineiii{'B'}{unsigned integer}{1}
\lineiii{'h'}{signed integer}{2}
\lineiii{'H'}{unsigned integer}{2}
\lineiii{'i'}{signed integer}{2}
\lineiii{'I'}{unsigned integer}{2}
\lineiii{'l'}{signed integer}{4}
\lineiii{'L'}{unsigned integer}{4}
\lineiii{'f'}{floating point}{4}
\lineiii{'d'}{floating point}{8}
\end{tableiii}

The actual representation of values is determined by the machine
architecture (strictly speaking, by the C implementation).  The actual
size can be accessed through the \var{itemsize} attribute.  The values
stored  for \code{'L'} and \code{'I'} items will be represented as
Python long integers when retrieved, because Python's plain integer
type can't represent the full range of C's unsigned (long) integers.

See also built-in module \code{struct}.
\refbimodindex{struct}

The module defines the following function:

\setindexsubitem{(in module array)}

\begin{funcdesc}{array}{typecode\optional{\, initializer}}
Return a new array whose items are restricted by \var{typecode}, and
initialized from the optional \var{initializer} value, which must be a
list or a string.  The list or string is passed to the new array's
\code{fromlist()} or \code{fromstring()} method (see below) to add
initial items to the array.
\end{funcdesc}

Array objects support the following data items and methods:

\begin{datadesc}{typecode}
The typecode character used to create the array.
\end{datadesc}

\begin{datadesc}{itemsize}
The length in bytes of one array item in the internal representation.
\end{datadesc}

\begin{funcdesc}{append}{x}
Append a new item with value \var{x} to the end of the array.
\end{funcdesc}

\begin{funcdesc}{buffer_info}{}
Return a tuple \code{(\var{address}, \var{length})} giving the current
memory address and the length in bytes of the buffer used to hold
array's contents.  This is occasionally useful when working with
low-level (and inherently unsafe) I/O interfaces that require memory
addresses, such as certain \code{ioctl} operations.  The returned
numbers are valid as long as the array exists and no length-changing
operations are applied to it.
\end{funcdesc}

\begin{funcdesc}{byteswap}{x}
``Byteswap'' all items of the array.  This is only supported for
integer values.  It is useful when reading data from a file written
on a machine with a different byte order.
\end{funcdesc}

\begin{funcdesc}{fromfile}{f\, n}
Read \var{n} items (as machine values) from the file object \var{f}
and append them to the end of the array.  If less than \var{n} items
are available, \code{EOFError} is raised, but the items that were
available are still inserted into the array.  \var{f} must be a real
built-in file object; something else with a \code{read()} method won't
do.
\end{funcdesc}

\begin{funcdesc}{fromlist}{list}
Append items from the list.  This is equivalent to
\code{for x in \var{list}:\ a.append(x)}
except that if there is a type error, the array is unchanged.
\end{funcdesc}

\begin{funcdesc}{fromstring}{s}
Appends items from the string, interpreting the string as an
array of machine values (i.e. as if it had been read from a
file using the \code{fromfile()} method).
\end{funcdesc}

\begin{funcdesc}{insert}{i\, x}
Insert a new item with value \var{x} in the array before position
\var{i}.
\end{funcdesc}

\begin{funcdesc}{tofile}{f}
Write all items (as machine values) to the file object \var{f}.
\end{funcdesc}

\begin{funcdesc}{tolist}{}
Convert the array to an ordinary list with the same items.
\end{funcdesc}

\begin{funcdesc}{tostring}{}
Convert the array to an array of machine values and return the
string representation (the same sequence of bytes that would
be written to a file by the \code{tofile()} method.)
\end{funcdesc}

When an array object is printed or converted to a string, it is
represented as \code{array(\var{typecode}, \var{initializer})}.  The
\var{initializer} is omitted if the array is empty, otherwise it is a
string if the \var{typecode} is \code{'c'}, otherwise it is a list of
numbers.  The string is guaranteed to be able to be converted back to
an array with the same type and value using reverse quotes
(\code{``}).  Examples:

\begin{verbatim}
array('l')
array('c', 'hello world')
array('l', [1, 2, 3, 4, 5])
array('d', [1.0, 2.0, 3.14])
\end{verbatim}


\chapter{Generic Operating System Services}

The modules described in this chapter provide interfaces to operating
system features that are available on (almost) all operating systems,
such as files and a clock.  The interfaces are generally modelled
after the \UNIX{} or C interfaces but they are available on most other
systems as well.  Here's an overview:

\begin{description}

\item[os]
--- Miscellaneous OS interfaces.

\item[time]
--- Time access and conversions.

\item[getopt]
--- Parser for command line options.

\item[tempfile]
--- Generate temporary file names.

\end{description}
		% Generic Operating System Services
\section{\module{os} ---
         Miscellaneous operating system interfaces}

\declaremodule{standard}{os}
\modulesynopsis{Miscellaneous operating system interfaces.}


This module provides a more portable way of using operating system
dependent functionality than importing a operating system dependent
built-in module like \refmodule{posix} or \module{nt}.

This module searches for an operating system dependent built-in module like
\module{mac} or \refmodule{posix} and exports the same functions and data
as found there.  The design of all Python's built-in operating system dependent
modules is such that as long as the same functionality is available,
it uses the same interface; for example, the function
\code{os.stat(\var{path})} returns stat information about \var{path} in
the same format (which happens to have originated with the
\POSIX{} interface).

Extensions peculiar to a particular operating system are also
available through the \module{os} module, but using them is of course a
threat to portability!

Note that after the first time \module{os} is imported, there is
\emph{no} performance penalty in using functions from \module{os}
instead of directly from the operating system dependent built-in module,
so there should be \emph{no} reason not to use \module{os}!


% Frank Stajano <fstajano@uk.research.att.com> complained that it
% wasn't clear that the entries described in the subsections were all
% available at the module level (most uses of subsections are
% different); I think this is only a problem for the HTML version,
% where the relationship may not be as clear.
%
\ifhtml
The \module{os} module contains many functions and data values.
The items below and in the following sub-sections are all available
directly from the \module{os} module.
\fi


\begin{excdesc}{error}
This exception is raised when a function returns a system-related
error (not for illegal argument types or other incidental errors).
This is also known as the built-in exception \exception{OSError}.  The
accompanying value is a pair containing the numeric error code from
\cdata{errno} and the corresponding string, as would be printed by the
C function \cfunction{perror()}.  See the module
\refmodule{errno}\refbimodindex{errno}, which contains names for the
error codes defined by the underlying operating system.

When exceptions are classes, this exception carries two attributes,
\member{errno} and \member{strerror}.  The first holds the value of
the C \cdata{errno} variable, and the latter holds the corresponding
error message from \cfunction{strerror()}.  For exceptions that
involve a file system path (such as \function{chdir()} or
\function{unlink()}), the exception instance will contain a third
attribute, \member{filename}, which is the file name passed to the
function.
\end{excdesc}

\begin{datadesc}{name}
The name of the operating system dependent module imported.  The
following names have currently been registered: \code{'posix'},
\code{'nt'}, \code{'dos'}, \code{'mac'}, \code{'os2'}, \code{'ce'},
\code{'java'}, \code{'riscos'}.
\end{datadesc}

\begin{datadesc}{path}
The corresponding operating system dependent standard module for pathname
operations, such as \module{posixpath} or \module{macpath}.  Thus,
given the proper imports, \code{os.path.split(\var{file})} is
equivalent to but more portable than
\code{posixpath.split(\var{file})}.  Note that this is also an
importable module: it may be imported directly as
\refmodule{os.path}.
\end{datadesc}



\subsection{Process Parameters \label{os-procinfo}}

These functions and data items provide information and operate on the
current process and user.

\begin{datadesc}{environ}
A mapping object representing the string environment. For example,
\code{environ['HOME']} is the pathname of your home directory (on some
platforms), and is equivalent to \code{getenv("HOME")} in C.

If the platform supports the \function{putenv()} function, this
mapping may be used to modify the environment as well as query the
environment.  \function{putenv()} will be called automatically when
the mapping is modified.

If \function{putenv()} is not provided, this mapping may be passed to
the appropriate process-creation functions to cause child processes to
use a modified environment.
\end{datadesc}

\begin{funcdescni}{chdir}{path}
\funclineni{fchdir}{fd}
\funclineni{getcwd}{}
These functions are described in ``Files and Directories'' (section
\ref{os-file-dir}).
\end{funcdescni}

\begin{funcdesc}{ctermid}{}
Return the filename corresponding to the controlling terminal of the
process.
Availability: \UNIX.
\end{funcdesc}

\begin{funcdesc}{getegid}{}
Return the effective group id of the current process.  This
corresponds to the `set id' bit on the file being executed in the
current process.
Availability: \UNIX.
\end{funcdesc}

\begin{funcdesc}{geteuid}{}
\index{user!effective id}
Return the current process' effective user id.
Availability: \UNIX.
\end{funcdesc}

\begin{funcdesc}{getgid}{}
\index{process!group}
Return the real group id of the current process.
Availability: \UNIX.
\end{funcdesc}

\begin{funcdesc}{getgroups}{}
Return list of supplemental group ids associated with the current
process.
Availability: \UNIX.
\end{funcdesc}

\begin{funcdesc}{getlogin}{}
Return the name of the user logged in on the controlling terminal of
the process.  For most purposes, it is more useful to use the
environment variable \envvar{LOGNAME} to find out who the user is.
Availability: \UNIX.
\end{funcdesc}

\begin{funcdesc}{getpgid}{pid}
Return the process group id of the process with process id \var{pid}.
If \var{pid} is 0, the process group id of the current process is
returned. Availability: \UNIX.
\versionadded{2.3}
\end{funcdesc}

\begin{funcdesc}{getpgrp}{}
\index{process!group}
Return the id of the current process group.
Availability: \UNIX.
\end{funcdesc}

\begin{funcdesc}{getpid}{}
\index{process!id}
Return the current process id.
Availability: \UNIX, Windows.
\end{funcdesc}

\begin{funcdesc}{getppid}{}
\index{process!id of parent}
Return the parent's process id.
Availability: \UNIX.
\end{funcdesc}

\begin{funcdesc}{getuid}{}
\index{user!id}
Return the current process' user id.
Availability: \UNIX.
\end{funcdesc}

\begin{funcdesc}{getenv}{varname\optional{, value}}
Return the value of the environment variable \var{varname} if it
exists, or \var{value} if it doesn't.  \var{value} defaults to
\code{None}.
Availability: most flavors of \UNIX, Windows.
\end{funcdesc}

\begin{funcdesc}{putenv}{varname, value}
\index{environment variables!setting}
Set the environment variable named \var{varname} to the string
\var{value}.  Such changes to the environment affect subprocesses
started with \function{os.system()}, \function{popen()} or
\function{fork()} and \function{execv()}.
Availability: most flavors of \UNIX, Windows.

When \function{putenv()} is
supported, assignments to items in \code{os.environ} are automatically
translated into corresponding calls to \function{putenv()}; however,
calls to \function{putenv()} don't update \code{os.environ}, so it is
actually preferable to assign to items of \code{os.environ}.
\end{funcdesc}

\begin{funcdesc}{setegid}{egid}
Set the current process's effective group id.
Availability: \UNIX.
\end{funcdesc}

\begin{funcdesc}{seteuid}{euid}
Set the current process's effective user id.
Availability: \UNIX.
\end{funcdesc}

\begin{funcdesc}{setgid}{gid}
Set the current process' group id.
Availability: \UNIX.
\end{funcdesc}

\begin{funcdesc}{setgroups}{groups}
Set the list of supplemental group ids associated with the current
process to \var{groups}. \var{groups} must be a sequence, and each
element must be an integer identifying a group. This operation is
typical available only to the superuser.
Availability: \UNIX.
\versionadded{2.2}
\end{funcdesc}

\begin{funcdesc}{setpgrp}{}
Calls the system call \cfunction{setpgrp()} or \cfunction{setpgrp(0,
0)} depending on which version is implemented (if any).  See the
\UNIX{} manual for the semantics.
Availability: \UNIX.
\end{funcdesc}

\begin{funcdesc}{setpgid}{pid, pgrp} Calls the system call
\cfunction{setpgid()} to set the process group id of the process with
id \var{pid} to the process group with id \var{pgrp}.  See the \UNIX{}
manual for the semantics.
Availability: \UNIX.
\end{funcdesc}

\begin{funcdesc}{setreuid}{ruid, euid}
Set the current process's real and effective user ids.
Availability: \UNIX.
\end{funcdesc}

\begin{funcdesc}{setregid}{rgid, egid}
Set the current process's real and effective group ids.
Availability: \UNIX.
\end{funcdesc}

\begin{funcdesc}{setsid}{}
Calls the system call \cfunction{setsid()}.  See the \UNIX{} manual
for the semantics.
Availability: \UNIX.
\end{funcdesc}

\begin{funcdesc}{setuid}{uid}
\index{user!id, setting}
Set the current process' user id.
Availability: \UNIX.
\end{funcdesc}

% placed in this section since it relates to errno.... a little weak ;-(
\begin{funcdesc}{strerror}{code}
Return the error message corresponding to the error code in
\var{code}.
Availability: \UNIX, Windows.
\end{funcdesc}

\begin{funcdesc}{umask}{mask}
Set the current numeric umask and returns the previous umask.
Availability: \UNIX, Windows.
\end{funcdesc}

\begin{funcdesc}{uname}{}
Return a 5-tuple containing information identifying the current
operating system.  The tuple contains 5 strings:
\code{(\var{sysname}, \var{nodename}, \var{release}, \var{version},
\var{machine})}.  Some systems truncate the nodename to 8
characters or to the leading component; a better way to get the
hostname is \function{socket.gethostname()}
\withsubitem{(in module socket)}{\ttindex{gethostname()}}
or even
\withsubitem{(in module socket)}{\ttindex{gethostbyaddr()}}
\code{socket.gethostbyaddr(socket.gethostname())}.
Availability: recent flavors of \UNIX.
\end{funcdesc}



\subsection{File Object Creation \label{os-newstreams}}

These functions create new file objects.


\begin{funcdesc}{fdopen}{fd\optional{, mode\optional{, bufsize}}}
Return an open file object connected to the file descriptor \var{fd}.
\index{I/O control!buffering}
The \var{mode} and \var{bufsize} arguments have the same meaning as
the corresponding arguments to the built-in \function{open()}
function.
Availability: Macintosh, \UNIX, Windows.
\end{funcdesc}

\begin{funcdesc}{popen}{command\optional{, mode\optional{, bufsize}}}
Open a pipe to or from \var{command}.  The return value is an open
file object connected to the pipe, which can be read or written
depending on whether \var{mode} is \code{'r'} (default) or \code{'w'}.
The \var{bufsize} argument has the same meaning as the corresponding
argument to the built-in \function{open()} function.  The exit status of
the command (encoded in the format specified for \function{wait()}) is
available as the return value of the \method{close()} method of the file
object, except that when the exit status is zero (termination without
errors), \code{None} is returned.
Availability: \UNIX, Windows.

\versionchanged[This function worked unreliably under Windows in
  earlier versions of Python.  This was due to the use of the
  \cfunction{_popen()} function from the libraries provided with
  Windows.  Newer versions of Python do not use the broken
  implementation from the Windows libraries]{2.0}
\end{funcdesc}

\begin{funcdesc}{tmpfile}{}
Return a new file object opened in update mode (\samp{w+b}).  The file
has no directory entries associated with it and will be automatically
deleted once there are no file descriptors for the file.
Availability: \UNIX, Windows.
\end{funcdesc}


For each of these \function{popen()} variants, if \var{bufsize} is
specified, it specifies the buffer size for the I/O pipes.
\var{mode}, if provided, should be the string \code{'b'} or
\code{'t'}; on Windows this is needed to determine whether the file
objects should be opened in binary or text mode.  The default value
for \var{mode} is \code{'t'}.

These methods do not make it possible to retrieve the return code from
the child processes.  The only way to control the input and output
streams and also retrieve the return codes is to use the
\class{Popen3} and \class{Popen4} classes from the \refmodule{popen2}
module; these are only available on \UNIX.

For a discussion of possible dead lock conditions related to the use
of these functions, see ``\ulink{Flow Control
Issues}{popen2-flow-control.html}''
(section~\ref{popen2-flow-control}).

\begin{funcdesc}{popen2}{cmd\optional{, mode\optional{, bufsize}}}
Executes \var{cmd} as a sub-process.  Returns the file objects
\code{(\var{child_stdin}, \var{child_stdout})}.
Availability: \UNIX, Windows.
\versionadded{2.0}
\end{funcdesc}

\begin{funcdesc}{popen3}{cmd\optional{, mode\optional{, bufsize}}}
Executes \var{cmd} as a sub-process.  Returns the file objects
\code{(\var{child_stdin}, \var{child_stdout}, \var{child_stderr})}.
Availability: \UNIX, Windows.
\versionadded{2.0}
\end{funcdesc}

\begin{funcdesc}{popen4}{cmd\optional{, mode\optional{, bufsize}}}
Executes \var{cmd} as a sub-process.  Returns the file objects
\code{(\var{child_stdin}, \var{child_stdout_and_stderr})}.
Availability: \UNIX, Windows.
\versionadded{2.0}
\end{funcdesc}

This functionality is also available in the \refmodule{popen2} module
using functions of the same names, but the return values of those
functions have a different order.


\subsection{File Descriptor Operations \label{os-fd-ops}}

These functions operate on I/O streams referred to
using file descriptors.


\begin{funcdesc}{close}{fd}
Close file descriptor \var{fd}.
Availability: Macintosh, \UNIX, Windows.

Note: this function is intended for low-level I/O and must be applied
to a file descriptor as returned by \function{open()} or
\function{pipe()}.  To close a ``file object'' returned by the
built-in function \function{open()} or by \function{popen()} or
\function{fdopen()}, use its \method{close()} method.
\end{funcdesc}

\begin{funcdesc}{dup}{fd}
Return a duplicate of file descriptor \var{fd}.
Availability: Macintosh, \UNIX, Windows.
\end{funcdesc}

\begin{funcdesc}{dup2}{fd, fd2}
Duplicate file descriptor \var{fd} to \var{fd2}, closing the latter
first if necessary.
Availability: \UNIX, Windows.
\end{funcdesc}

\begin{funcdesc}{fpathconf}{fd, name}
Return system configuration information relevant to an open file.
\var{name} specifies the configuration value to retrieve; it may be a
string which is the name of a defined system value; these names are
specified in a number of standards (\POSIX.1, \UNIX 95, \UNIX 98, and
others).  Some platforms define additional names as well.  The names
known to the host operating system are given in the
\code{pathconf_names} dictionary.  For configuration variables not
included in that mapping, passing an integer for \var{name} is also
accepted.
Availability: \UNIX.

If \var{name} is a string and is not known, \exception{ValueError} is
raised.  If a specific value for \var{name} is not supported by the
host system, even if it is included in \code{pathconf_names}, an
\exception{OSError} is raised with \constant{errno.EINVAL} for the
error number.
\end{funcdesc}

\begin{funcdesc}{fstat}{fd}
Return status for file descriptor \var{fd}, like \function{stat()}.
Availability: \UNIX, Windows.
\end{funcdesc}

\begin{funcdesc}{fstatvfs}{fd}
Return information about the filesystem containing the file associated
with file descriptor \var{fd}, like \function{statvfs()}.
Availability: \UNIX.
\end{funcdesc}

\begin{funcdesc}{ftruncate}{fd, length}
Truncate the file corresponding to file descriptor \var{fd},
so that it is at most \var{length} bytes in size.
Availability: \UNIX.
\end{funcdesc}

\begin{funcdesc}{isatty}{fd}
Return \code{True} if the file descriptor \var{fd} is open and
connected to a tty(-like) device, else \code{False}.
Availability: \UNIX.
\end{funcdesc}

\begin{funcdesc}{lseek}{fd, pos, how}
Set the current position of file descriptor \var{fd} to position
\var{pos}, modified by \var{how}: \code{0} to set the position
relative to the beginning of the file; \code{1} to set it relative to
the current position; \code{2} to set it relative to the end of the
file.
Availability: Macintosh, \UNIX, Windows.
\end{funcdesc}

\begin{funcdesc}{open}{file, flags\optional{, mode}}
Open the file \var{file} and set various flags according to
\var{flags} and possibly its mode according to \var{mode}.
The default \var{mode} is \code{0777} (octal), and the current umask
value is first masked out.  Return the file descriptor for the newly
opened file.
Availability: Macintosh, \UNIX, Windows.

For a description of the flag and mode values, see the C run-time
documentation; flag constants (like \constant{O_RDONLY} and
\constant{O_WRONLY}) are defined in this module too (see below).

Note: this function is intended for low-level I/O.  For normal usage,
use the built-in function \function{open()}, which returns a ``file
object'' with \method{read()} and \method{write()} methods (and many
more).
\end{funcdesc}

\begin{funcdesc}{openpty}{}
Open a new pseudo-terminal pair. Return a pair of file descriptors
\code{(\var{master}, \var{slave})} for the pty and the tty,
respectively. For a (slightly) more portable approach, use the
\refmodule{pty}\refstmodindex{pty} module.
Availability: Some flavors of \UNIX.
\end{funcdesc}

\begin{funcdesc}{pipe}{}
Create a pipe.  Return a pair of file descriptors \code{(\var{r},
\var{w})} usable for reading and writing, respectively.
Availability: \UNIX, Windows.
\end{funcdesc}

\begin{funcdesc}{read}{fd, n}
Read at most \var{n} bytes from file descriptor \var{fd}.
Return a string containing the bytes read.  If the end of the file
referred to by \var{fd} has been reached, an empty string is
returned.
Availability: Macintosh, \UNIX, Windows.

Note: this function is intended for low-level I/O and must be applied
to a file descriptor as returned by \function{open()} or
\function{pipe()}.  To read a ``file object'' returned by the
built-in function \function{open()} or by \function{popen()} or
\function{fdopen()}, or \code{sys.stdin}, use its
\method{read()} or \method{readline()} methods.
\end{funcdesc}

\begin{funcdesc}{tcgetpgrp}{fd}
Return the process group associated with the terminal given by
\var{fd} (an open file descriptor as returned by \function{open()}).
Availability: \UNIX.
\end{funcdesc}

\begin{funcdesc}{tcsetpgrp}{fd, pg}
Set the process group associated with the terminal given by
\var{fd} (an open file descriptor as returned by \function{open()})
to \var{pg}.
Availability: \UNIX.
\end{funcdesc}

\begin{funcdesc}{ttyname}{fd}
Return a string which specifies the terminal device associated with
file-descriptor \var{fd}.  If \var{fd} is not associated with a terminal
device, an exception is raised.
Availability: \UNIX.
\end{funcdesc}

\begin{funcdesc}{write}{fd, str}
Write the string \var{str} to file descriptor \var{fd}.
Return the number of bytes actually written.
Availability: Macintosh, \UNIX, Windows.

Note: this function is intended for low-level I/O and must be applied
to a file descriptor as returned by \function{open()} or
\function{pipe()}.  To write a ``file object'' returned by the
built-in function \function{open()} or by \function{popen()} or
\function{fdopen()}, or \code{sys.stdout} or \code{sys.stderr}, use
its \method{write()} method.
\end{funcdesc}


The following data items are available for use in constructing the
\var{flags} parameter to the \function{open()} function.

\begin{datadesc}{O_RDONLY}
\dataline{O_WRONLY}
\dataline{O_RDWR}
\dataline{O_NDELAY}
\dataline{O_NONBLOCK}
\dataline{O_APPEND}
\dataline{O_DSYNC}
\dataline{O_RSYNC}
\dataline{O_SYNC}
\dataline{O_NOCTTY}
\dataline{O_CREAT}
\dataline{O_EXCL}
\dataline{O_TRUNC}
Options for the \var{flag} argument to the \function{open()} function.
These can be bit-wise OR'd together.
Availability: Macintosh, \UNIX, Windows.
% XXX O_NDELAY, O_NONBLOCK, O_DSYNC, O_RSYNC, O_SYNC, O_NOCTTY are not on Windows.
\end{datadesc}

\begin{datadesc}{O_BINARY}
Option for the \var{flag} argument to the \function{open()} function.
This can be bit-wise OR'd together with those listed above.
Availability: Macintosh, Windows.
% XXX need to check on the availability of this one.
\end{datadesc}

\begin{datadesc}{O_NOINHERIT}
\dataline{O_SHORT_LIVED}
\dataline{O_TEMPORARY}
\dataline{O_RANDOM}
\dataline{O_SEQUENTIAL}
\dataline{O_TEXT}
Options for the \var{flag} argument to the \function{open()} function.
These can be bit-wise OR'd together.
Availability: Windows.
\end{datadesc}

\subsection{Files and Directories \label{os-file-dir}}

\begin{funcdesc}{access}{path, mode}
Use the real uid/gid to test for access to \var{path}.  Note that most
operations will use the effective uid/gid, therefore this routine can
be used in a suid/sgid environment to test if the invoking user has the
specified access to \var{path}.  \var{mode} should be \constant{F_OK}
to test the existence of \var{path}, or it can be the inclusive OR of
one or more of \constant{R_OK}, \constant{W_OK}, and \constant{X_OK} to
test permissions.  Return \code{1} if access is allowed, \code{0} if not.
See the \UNIX{} man page \manpage{access}{2} for more information.
Availability: \UNIX, Windows.
\end{funcdesc}

\begin{datadesc}{F_OK}
  Value to pass as the \var{mode} parameter of \function{access()} to
  test the existence of \var{path}.
\end{datadesc}

\begin{datadesc}{R_OK}
  Value to include in the \var{mode} parameter of \function{access()}
  to test the readability of \var{path}.
\end{datadesc}

\begin{datadesc}{W_OK}
  Value to include in the \var{mode} parameter of \function{access()}
  to test the writability of \var{path}.
\end{datadesc}

\begin{datadesc}{X_OK}
  Value to include in the \var{mode} parameter of \function{access()}
  to determine if \var{path} can be executed.
\end{datadesc}

\begin{funcdesc}{chdir}{path}
\index{directory!changing}
Change the current working directory to \var{path}.
Availability: Macintosh, \UNIX, Windows.
\end{funcdesc}

\begin{funcdesc}{fchdir}{fd}
Change the current working directory to the directory represented by
the file descriptor \var{fd}.  The descriptor must refer to an opened
directory, not an open file.
Availability: \UNIX.
\versionadded{2.3}
\end{funcdesc}

\begin{funcdesc}{getcwd}{}
Return a string representing the current working directory.
Availability: Macintosh, \UNIX, Windows.
\end{funcdesc}

\begin{funcdesc}{chroot}{path}
Change the root directory of the current process to \var{path}.
Availability: \UNIX.
\versionadded{2.2}
\end{funcdesc}

\begin{funcdesc}{chmod}{path, mode}
Change the mode of \var{path} to the numeric \var{mode}.
Availability: \UNIX, Windows.
\end{funcdesc}

\begin{funcdesc}{chown}{path, uid, gid}
Change the owner and group id of \var{path} to the numeric \var{uid}
and \var{gid}.
Availability: \UNIX.
\end{funcdesc}

\begin{funcdesc}{link}{src, dst}
Create a hard link pointing to \var{src} named \var{dst}.
Availability: \UNIX.
\end{funcdesc}

\begin{funcdesc}{listdir}{path}
Return a list containing the names of the entries in the directory.
The list is in arbitrary order.  It does not include the special
entries \code{'.'} and \code{'..'} even if they are present in the
directory.
Availability: Macintosh, \UNIX, Windows.
\end{funcdesc}

\begin{funcdesc}{lstat}{path}
Like \function{stat()}, but do not follow symbolic links.
Availability: \UNIX.
\end{funcdesc}

\begin{funcdesc}{mkfifo}{path\optional{, mode}}
Create a FIFO (a named pipe) named \var{path} with numeric mode
\var{mode}.  The default \var{mode} is \code{0666} (octal).  The current
umask value is first masked out from the mode.
Availability: \UNIX.

FIFOs are pipes that can be accessed like regular files.  FIFOs exist
until they are deleted (for example with \function{os.unlink()}).
Generally, FIFOs are used as rendezvous between ``client'' and
``server'' type processes: the server opens the FIFO for reading, and
the client opens it for writing.  Note that \function{mkfifo()}
doesn't open the FIFO --- it just creates the rendezvous point.
\end{funcdesc}

\begin{funcdesc}{mknod}{path\optional{, mode=0600, major, minor}}
Create a filesystem node (file, device special file or named pipe)
named filename. mode specifies both the permissions to use and the
type of node to be created, being combined (bitwise OR) with one of
S_IFREG, S_IFCHR, S_IFBLK, and S_IFIFO (those constants are available
in \module{stat}). For S_IFCHR and S_IFBLK, major and minor define the
newly created device special file, otherwise they are ignored.

\versionadded{2.3}
\end{funcdesc}

\begin{funcdesc}{mkdir}{path\optional{, mode}}
Create a directory named \var{path} with numeric mode \var{mode}.
The default \var{mode} is \code{0777} (octal).  On some systems,
\var{mode} is ignored.  Where it is used, the current umask value is
first masked out.
Availability: Macintosh, \UNIX, Windows.
\end{funcdesc}

\begin{funcdesc}{makedirs}{path\optional{, mode}}
\index{directory!creating}
Recursive directory creation function.  Like \function{mkdir()},
but makes all intermediate-level directories needed to contain the
leaf directory.  Throws an \exception{error} exception if the leaf
directory already exists or cannot be created.  The default \var{mode}
is \code{0777} (octal).  This function does not properly handle UNC
paths (only relevant on Windows systems).
\versionadded{1.5.2}
\end{funcdesc}

\begin{funcdesc}{pathconf}{path, name}
Return system configuration information relevant to a named file.
\var{name} specifies the configuration value to retrieve; it may be a
string which is the name of a defined system value; these names are
specified in a number of standards (\POSIX.1, \UNIX 95, \UNIX 98, and
others).  Some platforms define additional names as well.  The names
known to the host operating system are given in the
\code{pathconf_names} dictionary.  For configuration variables not
included in that mapping, passing an integer for \var{name} is also
accepted.
Availability: \UNIX.

If \var{name} is a string and is not known, \exception{ValueError} is
raised.  If a specific value for \var{name} is not supported by the
host system, even if it is included in \code{pathconf_names}, an
\exception{OSError} is raised with \constant{errno.EINVAL} for the
error number.
\end{funcdesc}

\begin{datadesc}{pathconf_names}
Dictionary mapping names accepted by \function{pathconf()} and
\function{fpathconf()} to the integer values defined for those names
by the host operating system.  This can be used to determine the set
of names known to the system.
Availability: \UNIX.
\end{datadesc}

\begin{funcdesc}{readlink}{path}
Return a string representing the path to which the symbolic link
points.  The result may be either an absolute or relative pathname; if
it is relative, it may be converted to an absolute pathname using
\code{os.path.join(os.path.dirname(\var{path}), \var{result})}.
Availability: \UNIX.
\end{funcdesc}

\begin{funcdesc}{remove}{path}
Remove the file \var{path}.  If \var{path} is a directory,
\exception{OSError} is raised; see \function{rmdir()} below to remove
a directory.  This is identical to the \function{unlink()} function
documented below.  On Windows, attempting to remove a file that is in
use causes an exception to be raised; on \UNIX, the directory entry is
removed but the storage allocated to the file is not made available
until the original file is no longer in use.
Availability: Macintosh, \UNIX, Windows.
\end{funcdesc}

\begin{funcdesc}{removedirs}{path}
\index{directory!deleting}
Removes directories recursively.  Works like
\function{rmdir()} except that, if the leaf directory is
successfully removed, directories corresponding to rightmost path
segments will be pruned way until either the whole path is consumed or
an error is raised (which is ignored, because it generally means that
a parent directory is not empty).  Throws an \exception{error}
exception if the leaf directory could not be successfully removed.
\versionadded{1.5.2}
\end{funcdesc}

\begin{funcdesc}{rename}{src, dst}
Rename the file or directory \var{src} to \var{dst}.  If \var{dst} is
a directory, \exception{OSError} will be raised.  On \UNIX, if
\var{dst} exists and is a file, it will be removed silently if the
user has permission.  The operation may fail on some \UNIX{} flavors
if \var{src} and \var{dst} are on different filesystems.  If
successful, the renaming will be an atomic operation (this is a
\POSIX{} requirement).  On Windows, if \var{dst} already exists,
\exception{OSError} will be raised even if it is a file; there may be
no way to implement an atomic rename when \var{dst} names an existing
file.
Availability: Macintosh, \UNIX, Windows.
\end{funcdesc}

\begin{funcdesc}{renames}{old, new}
Recursive directory or file renaming function.
Works like \function{rename()}, except creation of any intermediate
directories needed to make the new pathname good is attempted first.
After the rename, directories corresponding to rightmost path segments
of the old name will be pruned away using \function{removedirs()}.

Note: this function can fail with the new directory structure made if
you lack permissions needed to remove the leaf directory or file.
\versionadded{1.5.2}
\end{funcdesc}

\begin{funcdesc}{rmdir}{path}
Remove the directory \var{path}.
Availability: Macintosh, \UNIX, Windows.
\end{funcdesc}

\begin{funcdesc}{stat}{path}
Perform a \cfunction{stat()} system call on the given path.  The
return value is an object whose attributes correspond to the members of
the \ctype{stat} structure, namely:
\member{st_mode} (protection bits),
\member{st_ino} (inode number),
\member{st_dev} (device),
\member{st_nlink} (number of hard links,
\member{st_uid} (user ID of owner),
\member{st_gid} (group ID of owner),
\member{st_size} (size of file, in bytes),
\member{st_atime} (time of most recent access),
\member{st_mtime} (time of most recent content modification),
\member{st_ctime}
(time of most recent content modification or metadata change).

On some Unix systems (such as Linux), the following attributes may
also be available:
\member{st_blocks} (number of blocks allocated for file),
\member{st_blksize} (filesystem blocksize),
\member{st_rdev} (type of device if an inode device).

On Mac OS systems, the following attributes may also be available:
\member{st_rsize},
\member{st_creator},
\member{st_type}.

On RISCOS systems, the following attributes are also available:
\member{st_ftype} (file type),
\member{st_attrs} (attributes),
\member{st_obtype} (object type).

For backward compatibility, the return value of \function{stat()} is
also accessible as a tuple of at least 10 integers giving the most
important (and portable) members of the \ctype{stat} structure, in the
order
\member{st_mode},
\member{st_ino},
\member{st_dev},
\member{st_nlink},
\member{st_uid},
\member{st_gid},
\member{st_size},
\member{st_atime},
\member{st_mtime},
\member{st_ctime}.
More items may be added at the end by some implementations.  Note that
on the Mac OS, the time values are floating point values, like all
time values on the Mac OS.
The standard module \refmodule{stat}\refstmodindex{stat} defines
functions and constants that are useful for extracting information
from a \ctype{stat} structure.
(On Windows, some items are filled with dummy values.)
Availability: Macintosh, \UNIX, Windows.

\versionchanged
[Added access to values as attributes of the returned object]{2.2}
\end{funcdesc}

\begin{funcdesc}{statvfs}{path}
Perform a \cfunction{statvfs()} system call on the given path.  The
return value is an object whose attributes describe the filesystem on
the given path, and correspond to the members of the
\ctype{statvfs} structure, namely:
\member{f_frsize},
\member{f_blocks},
\member{f_bfree},
\member{f_bavail},
\member{f_files},
\member{f_ffree},
\member{f_favail},
\member{f_flag},
\member{f_namemax}.
Availability: \UNIX.

For backward compatibility, the return value is also accessible as a
tuple whose values correspond to the attributes, in the order given above.
The standard module \refmodule{statvfs}\refstmodindex{statvfs}
defines constants that are useful for extracting information
from a \ctype{statvfs} structure when accessing it as a sequence; this
remains useful when writing code that needs to work with versions of
Python that don't support accessing the fields as attributes.

\versionchanged
[Added access to values as attributes of the returned object]{2.2}
\end{funcdesc}

\begin{funcdesc}{symlink}{src, dst}
Create a symbolic link pointing to \var{src} named \var{dst}.
Availability: \UNIX.
\end{funcdesc}

\begin{funcdesc}{tempnam}{\optional{dir\optional{, prefix}}}
Return a unique path name that is reasonable for creating a temporary
file.  This will be an absolute path that names a potential directory
entry in the directory \var{dir} or a common location for temporary
files if \var{dir} is omitted or \code{None}.  If given and not
\code{None}, \var{prefix} is used to provide a short prefix to the
filename.  Applications are responsible for properly creating and
managing files created using paths returned by \function{tempnam()};
no automatic cleanup is provided.
\warning{Use of \function{tempnam()} is vulnerable to symlink attacks;
consider using \function{tmpfile()} instead.}
Availability: \UNIX, Windows.
\end{funcdesc}

\begin{funcdesc}{tmpnam}{}
Return a unique path name that is reasonable for creating a temporary
file.  This will be an absolute path that names a potential directory
entry in a common location for temporary files.  Applications are
responsible for properly creating and managing files created using
paths returned by \function{tmpnam()}; no automatic cleanup is
provided.
\warning{Use of \function{tmpnam()} is vulnerable to symlink attacks;
consider using \function{tmpfile()} instead.}
Availability: \UNIX, Windows.
\end{funcdesc}

\begin{datadesc}{TMP_MAX}
The maximum number of unique names that \function{tmpnam()} will
generate before reusing names.
\end{datadesc}

\begin{funcdesc}{unlink}{path}
Remove the file \var{path}.  This is the same function as
\function{remove()}; the \function{unlink()} name is its traditional
\UNIX{} name.
Availability: Macintosh, \UNIX, Windows.
\end{funcdesc}

\begin{funcdesc}{utime}{path, times}
Set the access and modified times of the file specified by \var{path}.
If \var{times} is \code{None}, then the file's access and modified
times are set to the current time.  Otherwise, \var{times} must be a
2-tuple of numbers, of the form \code{(\var{atime}, \var{mtime})}
which is used to set the access and modified times, respectively.
\versionchanged[Added support for \code{None} for \var{times}]{2.0}
Availability: Macintosh, \UNIX, Windows.
\end{funcdesc}


\subsection{Process Management \label{os-process}}

These functions may be used to create and manage processes.

The various \function{exec*()} functions take a list of arguments for
the new program loaded into the process.  In each case, the first of
these arguments is passed to the new program as its own name rather
than as an argument a user may have typed on a command line.  For the
C programmer, this is the \code{argv[0]} passed to a program's
\cfunction{main()}.  For example, \samp{os.execv('/bin/echo', ['foo',
'bar'])} will only print \samp{bar} on standard output; \samp{foo}
will seem to be ignored.


\begin{funcdesc}{abort}{}
Generate a \constant{SIGABRT} signal to the current process.  On
\UNIX, the default behavior is to produce a core dump; on Windows, the
process immediately returns an exit code of \code{3}.  Be aware that
programs which use \function{signal.signal()} to register a handler
for \constant{SIGABRT} will behave differently.
Availability: \UNIX, Windows.
\end{funcdesc}

\begin{funcdesc}{execl}{path, arg0, arg1, \moreargs}
\funcline{execle}{path, arg0, arg1, \moreargs, env}
\funcline{execlp}{file, arg0, arg1, \moreargs}
\funcline{execlpe}{file, arg0, arg1, \moreargs, env}
\funcline{execv}{path, args}
\funcline{execve}{path, args, env}
\funcline{execvp}{file, args}
\funcline{execvpe}{file, args, env}
These functions all execute a new program, replacing the current
process; they do not return.  On \UNIX, the new executable is loaded
into the current process, and will have the same process ID as the
caller.  Errors will be reported as \exception{OSError} exceptions.

The \character{l} and \character{v} variants of the
\function{exec*()} functions differ in how command-line arguments are
passed.  The \character{l} variants are perhaps the easiest to work
with if the number of parameters is fixed when the code is written;
the individual parameters simply become additional parameters to the
\function{execl*()} functions.  The \character{v} variants are good
when the number of parameters is variable, with the arguments being
passed in a list or tuple as the \var{args} parameter.  In either
case, the arguments to the child process must start with the name of
the command being run.

The variants which include a \character{p} near the end
(\function{execlp()}, \function{execlpe()}, \function{execvp()},
and \function{execvpe()}) will use the \envvar{PATH} environment
variable to locate the program \var{file}.  When the environment is
being replaced (using one of the \function{exec*e()} variants,
discussed in the next paragraph), the
new environment is used as the source of the \envvar{PATH} variable.
The other variants, \function{execl()}, \function{execle()},
\function{execv()}, and \function{execve()}, will not use the
\envvar{PATH} variable to locate the executable; \var{path} must
contain an appropriate absolute or relative path.

For \function{execle()}, \function{execlpe()}, \function{execve()},
and \function{execvpe()} (note that these all end in \character{e}),
the \var{env} parameter must be a mapping which is used to define the
environment variables for the new process; the \function{execl()},
\function{execlp()}, \function{execv()}, and \function{execvp()}
all cause the new process to inherit the environment of the current
process.
Availability: \UNIX, Windows.
\end{funcdesc}

\begin{funcdesc}{_exit}{n}
Exit to the system with status \var{n}, without calling cleanup
handlers, flushing stdio buffers, etc.
Availability: \UNIX, Windows.

Note: the standard way to exit is \code{sys.exit(\var{n})}.
\function{_exit()} should normally only be used in the child process
after a \function{fork()}.
\end{funcdesc}

\begin{funcdesc}{fork}{}
Fork a child process.  Return \code{0} in the child, the child's
process id in the parent.
Availability: \UNIX.
\end{funcdesc}

\begin{funcdesc}{forkpty}{}
Fork a child process, using a new pseudo-terminal as the child's
controlling terminal. Return a pair of \code{(\var{pid}, \var{fd})},
where \var{pid} is \code{0} in the child, the new child's process id
in the parent, and \var{fd} is the file descriptor of the master end
of the pseudo-terminal.  For a more portable approach, use the
\refmodule{pty} module.
Availability: Some flavors of \UNIX.
\end{funcdesc}

\begin{funcdesc}{kill}{pid, sig}
\index{process!killing}
\index{process!signalling}
Kill the process \var{pid} with signal \var{sig}.  Constants for the
specific signals available on the host platform are defined in the
\refmodule{signal} module.
Availability: \UNIX.
\end{funcdesc}

\begin{funcdesc}{nice}{increment}
Add \var{increment} to the process's ``niceness''.  Return the new
niceness.
Availability: \UNIX.
\end{funcdesc}

\begin{funcdesc}{plock}{op}
Lock program segments into memory.  The value of \var{op}
(defined in \code{<sys/lock.h>}) determines which segments are locked.
Availability: \UNIX.
\end{funcdesc}

\begin{funcdescni}{popen}{\unspecified}
\funclineni{popen2}{\unspecified}
\funclineni{popen3}{\unspecified}
\funclineni{popen4}{\unspecified}
Run child processes, returning opened pipes for communications.  These
functions are described in section \ref{os-newstreams}.
\end{funcdescni}

\begin{funcdesc}{spawnl}{mode, path, \moreargs}
\funcline{spawnle}{mode, path, \moreargs, env}
\funcline{spawnlp}{mode, file, \moreargs}
\funcline{spawnlpe}{mode, file, \moreargs, env}
\funcline{spawnv}{mode, path, args}
\funcline{spawnve}{mode, path, args, env}
\funcline{spawnvp}{mode, file, args}
\funcline{spawnvpe}{mode, file, args, env}
Execute the program \var{path} in a new process.  If \var{mode} is
\constant{P_NOWAIT}, this function returns the process ID of the new
process; if \var{mode} is \constant{P_WAIT}, returns the process's
exit code if it exits normally, or \code{-\var{signal}}, where
\var{signal} is the signal that killed the process.  On Windows, the
process ID will actually be the process handle, so can be used with
the \function{waitpid()} function.

The \character{l} and \character{v} variants of the
\function{spawn*()} functions differ in how command-line arguments are
passed.  The \character{l} variants are perhaps the easiest to work
with if the number of parameters is fixed when the code is written;
the individual parameters simply become additional parameters to the
\function{spawnl*()} functions.  The \character{v} variants are good
when the number of parameters is variable, with the arguments being
passed in a list or tuple as the \var{args} parameter.  In either
case, the arguments to the child process must start with the name of
the command being run.

The variants which include a second \character{p} near the end
(\function{spawnlp()}, \function{spawnlpe()}, \function{spawnvp()},
and \function{spawnvpe()}) will use the \envvar{PATH} environment
variable to locate the program \var{file}.  When the environment is
being replaced (using one of the \function{spawn*e()} variants,
discussed in the next paragraph), the new environment is used as the
source of the \envvar{PATH} variable.  The other variants,
\function{spawnl()}, \function{spawnle()}, \function{spawnv()}, and
\function{spawnve()}, will not use the \envvar{PATH} variable to
locate the executable; \var{path} must contain an appropriate absolute
or relative path.

For \function{spawnle()}, \function{spawnlpe()}, \function{spawnve()},
and \function{spawnvpe()} (note that these all end in \character{e}),
the \var{env} parameter must be a mapping which is used to define the
environment variables for the new process; the \function{spawnl()},
\function{spawnlp()}, \function{spawnv()}, and \function{spawnvp()}
all cause the new process to inherit the environment of the current
process.

As an example, the following calls to \function{spawnlp()} and
\function{spawnvpe()} are equivalent:

\begin{verbatim}
import os
os.spawnlp(os.P_WAIT, 'cp', 'cp', 'index.html', '/dev/null')

L = ['cp', 'index.html', '/dev/null']
os.spawnvpe(os.P_WAIT, 'cp', L, os.environ)
\end{verbatim}

Availability: \UNIX, Windows.  \function{spawnlp()},
\function{spawnlpe()}, \function{spawnvp()} and \function{spawnvpe()}
are not available on Windows.
\versionadded{1.6}
\end{funcdesc}

\begin{datadesc}{P_NOWAIT}
\dataline{P_NOWAITO}
Possible values for the \var{mode} parameter to the \function{spawn*()}
family of functions.  If either of these values is given, the
\function{spawn*()} functions will return as soon as the new process
has been created, with the process ID as the return value.
Availability: \UNIX, Windows.
\versionadded{1.6}
\end{datadesc}

\begin{datadesc}{P_WAIT}
Possible value for the \var{mode} parameter to the \function{spawn*()}
family of functions.  If this is given as \var{mode}, the
\function{spawn*()} functions will not return until the new process
has run to completion and will return the exit code of the process the
run is successful, or \code{-\var{signal}} if a signal kills the
process.
Availability: \UNIX, Windows.
\versionadded{1.6}
\end{datadesc}

\begin{datadesc}{P_DETACH}
\dataline{P_OVERLAY}
Possible values for the \var{mode} parameter to the
\function{spawn*()} family of functions.  These are less portable than
those listed above.
\constant{P_DETACH} is similar to \constant{P_NOWAIT}, but the new
process is detached from the console of the calling process.
If \constant{P_OVERLAY} is used, the current process will be replaced;
the \function{spawn*()} function will not return.
Availability: Windows.
\versionadded{1.6}
\end{datadesc}

\begin{funcdesc}{startfile}{path}
Start a file with its associated application.  This acts like
double-clicking the file in Windows Explorer, or giving the file name
as an argument to the \program{start} command from the interactive
command shell: the file is opened with whatever application (if any)
its extension is associated.

\function{startfile()} returns as soon as the associated application
is launched.  There is no option to wait for the application to close,
and no way to retrieve the application's exit status.  The \var{path}
parameter is relative to the current directory.  If you want to use an
absolute path, make sure the first character is not a slash
(\character{/}); the underlying Win32 \cfunction{ShellExecute()}
function doesn't work if it is.  Use the \function{os.path.normpath()}
function to ensure that the path is properly encoded for Win32.
Availability: Windows.
\versionadded{2.0}
\end{funcdesc}

\begin{funcdesc}{system}{command}
Execute the command (a string) in a subshell.  This is implemented by
calling the Standard C function \cfunction{system()}, and has the
same limitations.  Changes to \code{posix.environ}, \code{sys.stdin},
etc.\ are not reflected in the environment of the executed command.
The return value is the exit status of the process encoded in the
format specified for \function{wait()}, except on Windows 95 and 98,
where it is always \code{0}.  Note that \POSIX{} does not specify the
meaning of the return value of the C \cfunction{system()} function,
so the return value of the Python function is system-dependent.
Availability: \UNIX, Windows.
\end{funcdesc}

\begin{funcdesc}{times}{}
Return a 5-tuple of floating point numbers indicating accumulated
(processor or other)
times, in seconds.  The items are: user time, system time, children's
user time, children's system time, and elapsed real time since a fixed
point in the past, in that order.  See the \UNIX{} manual page
\manpage{times}{2} or the corresponding Windows Platform API
documentation.
Availability: \UNIX, Windows.
\end{funcdesc}

\begin{funcdesc}{wait}{}
Wait for completion of a child process, and return a tuple containing
its pid and exit status indication: a 16-bit number, whose low byte is
the signal number that killed the process, and whose high byte is the
exit status (if the signal number is zero); the high bit of the low
byte is set if a core file was produced.
Availability: \UNIX.
\end{funcdesc}

\begin{funcdesc}{waitpid}{pid, options}
The details of this function differ on \UNIX{} and Windows.

On \UNIX:
Wait for completion of a child process given by process id \var{pid},
and return a tuple containing its process id and exit status
indication (encoded as for \function{wait()}).  The semantics of the
call are affected by the value of the integer \var{options}, which
should be \code{0} for normal operation.

If \var{pid} is greater than \code{0}, \function{waitpid()} requests
status information for that specific process.  If \var{pid} is
\code{0}, the request is for the status of any child in the process
group of the current process.  If \var{pid} is \code{-1}, the request
pertains to any child of the current process.  If \var{pid} is less
than \code{-1}, status is requested for any process in the process
group \code{-\var{pid}} (the absolute value of \var{pid}).

On Windows:
Wait for completion of a process given by process handle \var{pid},
and return a tuple containing \var{pid},
and its exit status shifted left by 8 bits (shifting makes cross-platform
use of the function easier).
A \var{pid} less than or equal to \code{0} has no special meaning on
Windows, and raises an exception.
The value of integer \var{options} has no effect.
\var{pid} can refer to any process whose id is known, not necessarily a
child process.
The \function{spawn()} functions called with \constant{P_NOWAIT}
return suitable process handles.
\end{funcdesc}

\begin{datadesc}{WNOHANG}
The option for \function{waitpid()} to avoid hanging if no child
process status is available immediately.
Availability: \UNIX.
\end{datadesc}

\begin{datadesc}{WCONTINUED}
This option causes child processes to be reported if they have been
continued from a job control stop since their status was last
reported.
Availability: Some \UNIX{} systems.
\versionadded{2.3}
\end{datadesc}

\begin{datadesc}{WUNTRACED}
This option causes child processes to be reported if they have been
stopped but their current state has not been reported since they were
stopped.
Availability: \UNIX.
\versionadded{2.3}
\end{datadesc}

The following functions take a process status code as returned by
\function{system()}, \function{wait()}, or \function{waitpid()} as a
parameter.  They may be used to determine the disposition of a
process.

\begin{funcdesc}{WCOREDUMP}{status}
Returns \code{True} if a core dump was generated for the process,
otherwise it returns \code{False}.
Availability: \UNIX.
\versionadded{2.3}
\end{funcdesc}

\begin{funcdesc}{WIFCONTINUED}{status}
Returns \code{True} if the process has been continued from a job
control stop, otherwise it returns \code{False}.
Availability: \UNIX.
\versionadded{2.3}
\end{funcdesc}

\begin{funcdesc}{WIFSTOPPED}{status}
Returns \code{True} if the process has been stopped, otherwise it
returns \code{False}.
Availability: \UNIX.
\end{funcdesc}

\begin{funcdesc}{WIFSIGNALED}{status}
Returns \code{True} if the process exited due to a signal, otherwise
it returns \code{False}.
Availability: \UNIX.
\end{funcdesc}

\begin{funcdesc}{WIFEXITED}{status}
Returns \code{True} if the process exited using the \manpage{exit}{2}
system call, otherwise it returns \code{False}.
Availability: \UNIX.
\end{funcdesc}

\begin{funcdesc}{WEXITSTATUS}{status}
If \code{WIFEXITED(\var{status})} is true, return the integer
parameter to the \manpage{exit}{2} system call.  Otherwise, the return
value is meaningless.
Availability: \UNIX.
\end{funcdesc}

\begin{funcdesc}{WSTOPSIG}{status}
Return the signal which caused the process to stop.
Availability: \UNIX.
\end{funcdesc}

\begin{funcdesc}{WTERMSIG}{status}
Return the signal which caused the process to exit.
Availability: \UNIX.
\end{funcdesc}


\subsection{Miscellaneous System Information \label{os-path}}


\begin{funcdesc}{confstr}{name}
Return string-valued system configuration values.
\var{name} specifies the configuration value to retrieve; it may be a
string which is the name of a defined system value; these names are
specified in a number of standards (\POSIX, \UNIX 95, \UNIX 98, and
others).  Some platforms define additional names as well.  The names
known to the host operating system are given in the
\code{confstr_names} dictionary.  For configuration variables not
included in that mapping, passing an integer for \var{name} is also
accepted.
Availability: \UNIX.

If the configuration value specified by \var{name} isn't defined, the
empty string is returned.

If \var{name} is a string and is not known, \exception{ValueError} is
raised.  If a specific value for \var{name} is not supported by the
host system, even if it is included in \code{confstr_names}, an
\exception{OSError} is raised with \constant{errno.EINVAL} for the
error number.
\end{funcdesc}

\begin{datadesc}{confstr_names}
Dictionary mapping names accepted by \function{confstr()} to the
integer values defined for those names by the host operating system.
This can be used to determine the set of names known to the system.
Availability: \UNIX.
\end{datadesc}

\begin{funcdesc}{sysconf}{name}
Return integer-valued system configuration values.
If the configuration value specified by \var{name} isn't defined,
\code{-1} is returned.  The comments regarding the \var{name}
parameter for \function{confstr()} apply here as well; the dictionary
that provides information on the known names is given by
\code{sysconf_names}.
Availability: \UNIX.
\end{funcdesc}

\begin{datadesc}{sysconf_names}
Dictionary mapping names accepted by \function{sysconf()} to the
integer values defined for those names by the host operating system.
This can be used to determine the set of names known to the system.
Availability: \UNIX.
\end{datadesc}


The follow data values are used to support path manipulation
operations.  These are defined for all platforms.

Higher-level operations on pathnames are defined in the
\refmodule{os.path} module.


\begin{datadesc}{curdir}
The constant string used by the operating system to refer to the current
directory.
For example: \code{'.'} for \POSIX{} or \code{':'} for the Macintosh.
\end{datadesc}

\begin{datadesc}{pardir}
The constant string used by the operating system to refer to the parent
directory.
For example: \code{'..'} for \POSIX{} or \code{'::'} for the Macintosh.
\end{datadesc}

\begin{datadesc}{sep}
The character used by the operating system to separate pathname components,
for example, \character{/} for \POSIX{} or \character{:} for the
Macintosh.  Note that knowing this is not sufficient to be able to
parse or concatenate pathnames --- use \function{os.path.split()} and
\function{os.path.join()} --- but it is occasionally useful.
\end{datadesc}

\begin{datadesc}{altsep}
An alternative character used by the operating system to separate pathname
components, or \code{None} if only one separator character exists.  This is
set to \character{/} on DOS and Windows systems where \code{sep} is a
backslash.
\end{datadesc}

\begin{datadesc}{pathsep}
The character conventionally used by the operating system to separate
search patch components (as in \envvar{PATH}), such as \character{:} for
\POSIX{} or \character{;} for DOS and Windows.
\end{datadesc}

\begin{datadesc}{defpath}
The default search path used by \function{exec*p*()} and
\function{spawn*p*()} if the environment doesn't have a \code{'PATH'}
key.
\end{datadesc}

\begin{datadesc}{linesep}
The string used to separate (or, rather, terminate) lines on the
current platform.  This may be a single character, such as \code{'\e
n'} for \POSIX{} or \code{'\e r'} for Mac OS, or multiple characters,
for example, \code{'\e r\e n'} for DOS and Windows.
\end{datadesc}

\section{\module{time} ---
         Time access and conversions}

\declaremodule{builtin}{time}
\modulesynopsis{Time access and conversions.}


This module provides various time-related functions.  It is always
available, but not all functions are available on all platforms.  Most
of the functions defined in this module call platform C library
functions with the same name.  It may sometimes be helpful to consult
the platform documentation, because the semantics of these functions
varies among platforms.

An explanation of some terminology and conventions is in order.

\begin{itemize}

\item
The \dfn{epoch}\index{epoch} is the point where the time starts.  On
January 1st of that year, at 0 hours, the ``time since the epoch'' is
zero.  For \UNIX, the epoch is 1970.  To find out what the epoch is,
look at \code{gmtime(0)}.

\item
The functions in this module do not handle dates and times before the
epoch or far in the future.  The cut-off point in the future is
determined by the C library; for \UNIX, it is typically in
2038\index{Year 2038}.

\item
\strong{Year 2000 (Y2K) issues}:\index{Year 2000}\index{Y2K}  Python
depends on the platform's C library, which generally doesn't have year
2000 issues, since all dates and times are represented internally as
seconds since the epoch.  Functions accepting a \class{struct_time}
(see below) generally require a 4-digit year.  For backward
compatibility, 2-digit years are supported if the module variable
\code{accept2dyear} is a non-zero integer; this variable is
initialized to \code{1} unless the environment variable
\envvar{PYTHONY2K} is set to a non-empty string, in which case it is
initialized to \code{0}.  Thus, you can set
\envvar{PYTHONY2K} to a non-empty string in the environment to require 4-digit
years for all year input.  When 2-digit years are accepted, they are
converted according to the \POSIX{} or X/Open standard: values 69-99
are mapped to 1969-1999, and values 0--68 are mapped to 2000--2068.
Values 100--1899 are always illegal.  Note that this is new as of
Python 1.5.2(a2); earlier versions, up to Python 1.5.1 and 1.5.2a1,
would add 1900 to year values below 1900.

\item
UTC\index{UTC} is Coordinated Universal Time\index{Coordinated
Universal Time} (formerly known as Greenwich Mean
Time,\index{Greenwich Mean Time} or GMT).  The acronym UTC is not a
mistake but a compromise between English and French.

\item
DST is Daylight Saving Time,\index{Daylight Saving Time} an adjustment
of the timezone by (usually) one hour during part of the year.  DST
rules are magic (determined by local law) and can change from year to
year.  The C library has a table containing the local rules (often it
is read from a system file for flexibility) and is the only source of
True Wisdom in this respect.

\item
The precision of the various real-time functions may be less than
suggested by the units in which their value or argument is expressed.
E.g.\ on most \UNIX{} systems, the clock ``ticks'' only 50 or 100 times a
second, and on the Mac, times are only accurate to whole seconds.

\item
On the other hand, the precision of \function{time()} and
\function{sleep()} is better than their \UNIX{} equivalents: times are
expressed as floating point numbers, \function{time()} returns the
most accurate time available (using \UNIX{} \cfunction{gettimeofday()}
where available), and \function{sleep()} will accept a time with a
nonzero fraction (\UNIX{} \cfunction{select()} is used to implement
this, where available).

\item
The time value as returned by \function{gmtime()},
\function{localtime()}, and \function{strptime()}, and accepted by
\function{asctime()}, \function{mktime()} and \function{strftime()},
is a sequence of 9 integers.  The return values of \function{gmtime()},
\function{localtime()}, and \function{strptime()} also offer attribute
names for individual fields.

\begin{tableiii}{c|l|l}{textrm}{Index}{Attribute}{Values}
  \lineiii{0}{\member{tm_year}}{(for example, 1993)}
  \lineiii{1}{\member{tm_mon}}{range [1,12]}
  \lineiii{2}{\member{tm_mday}}{range [1,31]}
  \lineiii{3}{\member{tm_hour}}{range [0,23]}
  \lineiii{4}{\member{tm_min}}{range [0,59]}
  \lineiii{5}{\member{tm_sec}}{range [0,61]; see \strong{(1)} in \function{strftime()} description}
  \lineiii{6}{\member{tm_wday}}{range [0,6], Monday is 0}
  \lineiii{7}{\member{tm_yday}}{range [1,366]}
  \lineiii{8}{\member{tm_isdst}}{0, 1 or -1; see below}
\end{tableiii}

Note that unlike the C structure, the month value is a
range of 1-12, not 0-11.  A year value will be handled as described
under ``Year 2000 (Y2K) issues'' above.  A \code{-1} argument as the
daylight savings flag, passed to \function{mktime()} will usually
result in the correct daylight savings state to be filled in.

When a tuple with an incorrect length is passed to a function
expecting a \class{struct_time}, or having elements of the wrong type, a
\exception{TypeError} is raised.

\versionchanged[The time value sequence was changed from a tuple to a
                \class{struct_time}, with the addition of attribute names
                for the fields]{2.2}
\end{itemize}

The module defines the following functions and data items:


\begin{datadesc}{accept2dyear}
Boolean value indicating whether two-digit year values will be
accepted.  This is true by default, but will be set to false if the
environment variable \envvar{PYTHONY2K} has been set to a non-empty
string.  It may also be modified at run time.
\end{datadesc}

\begin{datadesc}{altzone}
The offset of the local DST timezone, in seconds west of UTC, if one
is defined.  This is negative if the local DST timezone is east of UTC
(as in Western Europe, including the UK).  Only use this if
\code{daylight} is nonzero.
\end{datadesc}

\begin{funcdesc}{asctime}{\optional{t}}
Convert a tuple or \class{struct_time} representing a time as returned
by \function{gmtime()}
or \function{localtime()} to a 24-character string of the following form:
\code{'Sun Jun 20 23:21:05 1993'}.  If \var{t} is not provided, the
current time as returned by \function{localtime()} is used.
Locale information is not used by \function{asctime()}.
\note{Unlike the C function of the same name, there is no trailing
newline.}
\versionchanged[Allowed \var{t} to be omitted]{2.1}
\end{funcdesc}

\begin{funcdesc}{clock}{}
On \UNIX, return
the current processor time as a floating point number expressed in
seconds.  The precision, and in fact the very definition of the meaning
of ``processor time''\index{CPU time}\index{processor time}, depends
on that of the C function of the same name, but in any case, this is
the function to use for benchmarking\index{benchmarking} Python or
timing algorithms.

On Windows, this function returns wall-clock seconds elapsed since the
first call to this function, as a floating point number,
based on the Win32 function \cfunction{QueryPerformanceCounter()}.
The resolution is typically better than one microsecond.
\end{funcdesc}

\begin{funcdesc}{ctime}{\optional{secs}}
Convert a time expressed in seconds since the epoch to a string
representing local time. If \var{secs} is not provided or
\constant{None}, the current time as returned by \function{time()} is
used.  \code{ctime(\var{secs})} is equivalent to
\code{asctime(localtime(\var{secs}))}.
Locale information is not used by \function{ctime()}.
\versionchanged[Allowed \var{secs} to be omitted]{2.1}
\versionchanged[If \var{secs} is \constant{None}, the current time is
                used]{2.4}
\end{funcdesc}

\begin{datadesc}{daylight}
Nonzero if a DST timezone is defined.
\end{datadesc}

\begin{funcdesc}{gmtime}{\optional{secs}}
Convert a time expressed in seconds since the epoch to a \class{struct_time}
in UTC in which the dst flag is always zero.  If \var{secs} is not
provided or \constant{None}, the current time as returned by
\function{time()} is used.  Fractions of a second are ignored.  See
above for a description of the \class{struct_time} object. See
\function{calendar.timegm()} for the inverse of this function.
\versionchanged[Allowed \var{secs} to be omitted]{2.1}
\versionchanged[If \var{secs} is \constant{None}, the current time is
                used]{2.4}
\end{funcdesc}

\begin{funcdesc}{localtime}{\optional{secs}}
Like \function{gmtime()} but converts to local time.  If \var{secs} is
not provided or \constant{None}, the current time as returned by
\function{time()} is used.  The dst flag is set to \code{1} when DST
applies to the given time.
\versionchanged[Allowed \var{secs} to be omitted]{2.1}
\versionchanged[If \var{secs} is \constant{None}, the current time is
                used]{2.4}
\end{funcdesc}

\begin{funcdesc}{mktime}{t}
This is the inverse function of \function{localtime()}.  Its argument
is the \class{struct_time} or full 9-tuple (since the dst flag is
needed; use \code{-1} as the dst flag if it is unknown) which
expresses the time in
\emph{local} time, not UTC.  It returns a floating point number, for
compatibility with \function{time()}.  If the input value cannot be
represented as a valid time, either \exception{OverflowError} or
\exception{ValueError} will be raised (which depends on whether the
invalid value is caught by Python or the underlying C libraries).  The
earliest date for which it can generate a time is platform-dependent.
\end{funcdesc}

\begin{funcdesc}{sleep}{secs}
Suspend execution for the given number of seconds.  The argument may
be a floating point number to indicate a more precise sleep time.
The actual suspension time may be less than that requested because any
caught signal will terminate the \function{sleep()} following
execution of that signal's catching routine.  Also, the suspension
time may be longer than requested by an arbitrary amount because of
the scheduling of other activity in the system.
\end{funcdesc}

\begin{funcdesc}{strftime}{format\optional{, t}}
Convert a tuple or \class{struct_time} representing a time as returned
by \function{gmtime()} or \function{localtime()} to a string as
specified by the \var{format} argument.  If \var{t} is not
provided, the current time as returned by \function{localtime()} is
used.  \var{format} must be a string.  \exception{ValueError} is raised
if any field in \var{t} is outside of the allowed range.
\versionchanged[Allowed \var{t} to be omitted]{2.1}
\versionchanged[\exception{ValueError} raised if a field in \var{t} is
out of range]{2.4}


The following directives can be embedded in the \var{format} string.
They are shown without the optional field width and precision
specification, and are replaced by the indicated characters in the
\function{strftime()} result:

\begin{tableiii}{c|p{24em}|c}{code}{Directive}{Meaning}{Notes}
  \lineiii{\%a}{Locale's abbreviated weekday name.}{}
  \lineiii{\%A}{Locale's full weekday name.}{}
  \lineiii{\%b}{Locale's abbreviated month name.}{}
  \lineiii{\%B}{Locale's full month name.}{}
  \lineiii{\%c}{Locale's appropriate date and time representation.}{}
  \lineiii{\%d}{Day of the month as a decimal number [01,31].}{}
  \lineiii{\%H}{Hour (24-hour clock) as a decimal number [00,23].}{}
  \lineiii{\%I}{Hour (12-hour clock) as a decimal number [01,12].}{}
  \lineiii{\%j}{Day of the year as a decimal number [001,366].}{}
  \lineiii{\%m}{Month as a decimal number [01,12].}{}
  \lineiii{\%M}{Minute as a decimal number [00,59].}{}
  \lineiii{\%p}{Locale's equivalent of either AM or PM.}{(1)}
  \lineiii{\%S}{Second as a decimal number [00,61].}{(2)}
  \lineiii{\%U}{Week number of the year (Sunday as the first day of the
                week) as a decimal number [00,53].  All days in a new year
                preceding the first Sunday are considered to be in week 0.}{(3)}
  \lineiii{\%w}{Weekday as a decimal number [0(Sunday),6].}{}
  \lineiii{\%W}{Week number of the year (Monday as the first day of the
                week) as a decimal number [00,53].  All days in a new year
                preceding the first Monday are considered to be in week 0.}{(3)}
  \lineiii{\%x}{Locale's appropriate date representation.}{}
  \lineiii{\%X}{Locale's appropriate time representation.}{}
  \lineiii{\%y}{Year without century as a decimal number [00,99].}{}
  \lineiii{\%Y}{Year with century as a decimal number.}{}
  \lineiii{\%Z}{Time zone name (no characters if no time zone exists).}{}
  \lineiii{\%\%}{A literal \character{\%} character.}{}
\end{tableiii}

\noindent
Notes:

\begin{description}
  \item[(1)]
    When used with the \function{strptime()} function, the \code{\%p}
    directive only affects the output hour field if the \code{\%I} directive
    is used to parse the hour.
  \item[(2)]
    The range really is \code{0} to \code{61}; this accounts for leap
    seconds and the (very rare) double leap seconds.
  \item[(3)]
    When used with the \function{strptime()} function, \code{\%U} and \code{\%W}
    are only used in calculations when the day of the week and the year are
    specified.
\end{description}

Here is an example, a format for dates compatible with that specified 
in the \rfc{2822} Internet email standard.
	\footnote{The use of \code{\%Z} is now
	deprecated, but the \code{\%z} escape that expands to the preferred 
	hour/minute offset is not supported by all ANSI C libraries. Also,
	a strict reading of the original 1982 \rfc{822} standard calls for
	a two-digit year (\%y rather than \%Y), but practice moved to
	4-digit years long before the year 2000.  The 4-digit year has
        been mandated by \rfc{2822}, which obsoletes \rfc{822}.}

\begin{verbatim}
>>> from time import gmtime, strftime
>>> strftime("%a, %d %b %Y %H:%M:%S +0000", gmtime())
'Thu, 28 Jun 2001 14:17:15 +0000'
\end{verbatim}

Additional directives may be supported on certain platforms, but
only the ones listed here have a meaning standardized by ANSI C.

On some platforms, an optional field width and precision
specification can immediately follow the initial \character{\%} of a
directive in the following order; this is also not portable.
The field width is normally 2 except for \code{\%j} where it is 3.
\end{funcdesc}

\begin{funcdesc}{strptime}{string\optional{, format}}
Parse a string representing a time according to a format.  The return 
value is a \class{struct_time} as returned by \function{gmtime()} or
\function{localtime()}.  The \var{format} parameter uses the same
directives as those used by \function{strftime()}; it defaults to
\code{"\%a \%b \%d \%H:\%M:\%S \%Y"} which matches the formatting
returned by \function{ctime()}.  If \var{string} cannot be parsed
according to \var{format}, \exception{ValueError} is raised.  If the
string to be parsed has excess data after parsing,
\exception{ValueError} is raised.  The default values used to fill in
any missing data when more accurate values cannot be inferred are
\code{(1900, 1, 1, 0, 0, 0, 0, 1, -1)} .

Support for the \code{\%Z} directive is based on the values contained in
\code{tzname} and whether \code{daylight} is true.  Because of this,
it is platform-specific except for recognizing UTC and GMT which are
always known (and are considered to be non-daylight savings
timezones).
\end{funcdesc}

\begin{datadesc}{struct_time}
The type of the time value sequence returned by \function{gmtime()},
\function{localtime()}, and \function{strptime()}.
\versionadded{2.2}
\end{datadesc}

\begin{funcdesc}{time}{}
Return the time as a floating point number expressed in seconds since
the epoch, in UTC.  Note that even though the time is always returned
as a floating point number, not all systems provide time with a better
precision than 1 second.  While this function normally returns
non-decreasing values, it can return a lower value than a previous
call if the system clock has been set back between the two calls.
\end{funcdesc}

\begin{datadesc}{timezone}
The offset of the local (non-DST) timezone, in seconds west of UTC
(negative in most of Western Europe, positive in the US, zero in the
UK).
\end{datadesc}

\begin{datadesc}{tzname}
A tuple of two strings: the first is the name of the local non-DST
timezone, the second is the name of the local DST timezone.  If no DST
timezone is defined, the second string should not be used.
\end{datadesc}

\begin{funcdesc}{tzset}{}
Resets the time conversion rules used by the library routines.
The environment variable \envvar{TZ} specifies how this is done.
\versionadded{2.3}

Availability: \UNIX.

\begin{notice}
Although in many cases, changing the \envvar{TZ} environment variable
may affect the output of functions like \function{localtime} without calling 
\function{tzset}, this behavior should not be relied on.

The \envvar{TZ} environment variable should contain no whitespace.
\end{notice}

The standard format of the \envvar{TZ} environment variable is:
(whitespace added for clarity)
\begin{itemize}
    \item[std offset [dst [offset] [,start[/time], end[/time]]]]
\end{itemize}

Where:

\begin{itemize}
  \item[std and dst]
    Three or more alphanumerics giving the timezone abbreviations.
    These will be propagated into time.tzname

  \item[offset]
    The offset has the form: \plusminus{} hh[:mm[:ss]].
    This indicates the value added the local time to arrive at UTC. 
    If preceded by a '-', the timezone is east of the Prime 
    Meridian; otherwise, it is west. If no offset follows
    dst, summer time is assumed to be one hour ahead of standard time.

  \item[start[/time],end[/time]]
    Indicates when to change to and back from DST. The format of the
    start and end dates are one of the following:

    \begin{itemize}
      \item[J\var{n}]
        The Julian day \var{n} (1 <= \var{n} <= 365). Leap days are not 
        counted, so in all years February 28 is day 59 and
        March 1 is day 60.

    \item[\var{n}]
        The zero-based Julian day (0 <= \var{n} <= 365). Leap days are
        counted, and it is possible to refer to February 29.

      \item[M\var{m}.\var{n}.\var{d}]
        The \var{d}'th day (0 <= \var{d} <= 6) or week \var{n} 
        of month \var{m} of the year (1 <= \var{n} <= 5, 
        1 <= \var{m} <= 12, where week 5 means "the last \var{d} day
        in month \var{m}" which may occur in either the fourth or 
        the fifth week). Week 1 is the first week in which the 
        \var{d}'th day occurs. Day zero is Sunday.
    \end{itemize}

    time has the same format as offset except that no leading sign ('-' or
    '+') is allowed. The default, if time is not given, is 02:00:00.
\end{itemize}


\begin{verbatim}
>>> os.environ['TZ'] = 'EST+05EDT,M4.1.0,M10.5.0'
>>> time.tzset()
>>> time.strftime('%X %x %Z')
'02:07:36 05/08/03 EDT'
>>> os.environ['TZ'] = 'AEST-10AEDT-11,M10.5.0,M3.5.0'
>>> time.tzset()
>>> time.strftime('%X %x %Z')
'16:08:12 05/08/03 AEST'
\end{verbatim}

On many Unix systems (including *BSD, Linux, Solaris, and Darwin), it
is more convenient to use the system's zoneinfo (\manpage{tzfile}{5}) 
database to specify the timezone rules. To do this, set the 
\envvar{TZ} environment variable to the path of the required timezone 
datafile, relative to the root of the systems 'zoneinfo' timezone database,
usually located at \file{/usr/share/zoneinfo}. For example, 
\code{'US/Eastern'}, \code{'Australia/Melbourne'}, \code{'Egypt'} or 
\code{'Europe/Amsterdam'}.

\begin{verbatim}
>>> os.environ['TZ'] = 'US/Eastern'
>>> time.tzset()
>>> time.tzname
('EST', 'EDT')
>>> os.environ['TZ'] = 'Egypt'
>>> time.tzset()
>>> time.tzname
('EET', 'EEST')
\end{verbatim}

\end{funcdesc}


\begin{seealso}
  \seemodule{datetime}{More object-oriented interface to dates and times.}
  \seemodule{locale}{Internationalization services.  The locale
                     settings can affect the return values for some of 
                     the functions in the \module{time} module.}
  \seemodule{calendar}{General calendar-related functions.  
                       \function{timegm()} is the inverse of
                       \function{gmtime()} from this module.}
\end{seealso}

\section{\module{getopt} ---
         Parser for command line options}

\declaremodule{standard}{getopt}
\modulesynopsis{Portable parser for command line options; support both
                short and long option names.}


This module helps scripts to parse the command line arguments in
\code{sys.argv}.
It supports the same conventions as the \UNIX{} \cfunction{getopt()}
function (including the special meanings of arguments of the form
`\code{-}' and `\code{-}\code{-}').
% That's to fool latex2html into leaving the two hyphens alone!
Long options similar to those supported by
GNU software may be used as well via an optional third argument.
This module provides a single function and an exception:

\begin{funcdesc}{getopt}{args, options\optional{, long_options}}
Parses command line options and parameter list.  \var{args} is the
argument list to be parsed, without the leading reference to the
running program. Typically, this means \samp{sys.argv[1:]}.
\var{options} is the string of option letters that the script wants to
recognize, with options that require an argument followed by a colon
(\character{:}; i.e., the same format that \UNIX{}
\cfunction{getopt()} uses).

\note{Unlike GNU \cfunction{getopt()}, after a non-option
argument, all further arguments are considered also non-options.
This is similar to the way non-GNU \UNIX{} systems work.}

\var{long_options}, if specified, must be a list of strings with the
names of the long options which should be supported.  The leading
\code{'-}\code{-'} characters should not be included in the option
name.  Long options which require an argument should be followed by an
equal sign (\character{=}).  To accept only long options,
\var{options} should be an empty string.  Long options on the command
line can be recognized so long as they provide a prefix of the option
name that matches exactly one of the accepted options.  For example,
it \var{long_options} is \code{['foo', 'frob']}, the option
\longprogramopt{fo} will match as \longprogramopt{foo}, but
\longprogramopt{f} will not match uniquely, so \exception{GetoptError}
will be raised.

The return value consists of two elements: the first is a list of
\code{(\var{option}, \var{value})} pairs; the second is the list of
program arguments left after the option list was stripped (this is a
trailing slice of \var{args}).  Each option-and-value pair returned
has the option as its first element, prefixed with a hyphen for short
options (e.g., \code{'-x'}) or two hyphens for long options (e.g.,
\code{'-}\code{-long-option'}), and the option argument as its second
element, or an empty string if the option has no argument.  The
options occur in the list in the same order in which they were found,
thus allowing multiple occurrences.  Long and short options may be
mixed.
\end{funcdesc}

\begin{funcdesc}{gnu_getopt}{args, options\optional{, long_options}}
This function works like \function{getopt()}, except that GNU style
scanning mode is used by default. This means that option and
non-option arguments may be intermixed. The \function{getopt()}
function stops processing options as soon as a non-option argument is
encountered.

If the first character of the option string is `+', or if the
environment variable POSIXLY_CORRECT is set, then option processing
stops as soon as a non-option argument is encountered.
\end{funcdesc}

\begin{excdesc}{GetoptError}
This is raised when an unrecognized option is found in the argument
list or when an option requiring an argument is given none.
The argument to the exception is a string indicating the cause of the
error.  For long options, an argument given to an option which does
not require one will also cause this exception to be raised.  The
attributes \member{msg} and \member{opt} give the error message and
related option; if there is no specific option to which the exception
relates, \member{opt} is an empty string.

\versionchanged[Introduced \exception{GetoptError} as a synonym for
                \exception{error}]{1.6}
\end{excdesc}

\begin{excdesc}{error}
Alias for \exception{GetoptError}; for backward compatibility.
\end{excdesc}


An example using only \UNIX{} style options:

\begin{verbatim}
>>> import getopt
>>> args = '-a -b -cfoo -d bar a1 a2'.split()
>>> args
['-a', '-b', '-cfoo', '-d', 'bar', 'a1', 'a2']
>>> optlist, args = getopt.getopt(args, 'abc:d:')
>>> optlist
[('-a', ''), ('-b', ''), ('-c', 'foo'), ('-d', 'bar')]
>>> args
['a1', 'a2']
\end{verbatim}

Using long option names is equally easy:

\begin{verbatim}
>>> s = '--condition=foo --testing --output-file abc.def -x a1 a2'
>>> args = s.split()
>>> args
['--condition=foo', '--testing', '--output-file', 'abc.def', '-x', 'a1', 'a2']
>>> optlist, args = getopt.getopt(args, 'x', [
...     'condition=', 'output-file=', 'testing'])
>>> optlist
[('--condition', 'foo'), ('--testing', ''), ('--output-file', 'abc.def'), ('-x',
 '')]
>>> args
['a1', 'a2']
\end{verbatim}

In a script, typical usage is something like this:

\begin{verbatim}
import getopt, sys

def main():
    try:
        opts, args = getopt.getopt(sys.argv[1:], "ho:v", ["help", "output="])
    except getopt.GetoptError:
        # print help information and exit:
        usage()
        sys.exit(2)
    output = None
    verbose = False
    for o, a in opts:
        if o == "-v":
            verbose = True
        if o in ("-h", "--help"):
            usage()
            sys.exit()
        if o in ("-o", "--output"):
            output = a
    # ...

if __name__ == "__main__":
    main()
\end{verbatim}

\section{Standard Module \sectcode{tempfile}}
\label{module-tempfile}
\stmodindex{tempfile}
\indexii{temporary}{file name}
\indexii{temporary}{file}

\renewcommand{\indexsubitem}{(in module tempfile)}

This module generates temporary file names.  It is not \UNIX{} specific,
but it may require some help on non-\UNIX{} systems.

Note: the modules does not create temporary files, nor does it
automatically remove them when the current process exits or dies.

The module defines a single user-callable function:

\begin{funcdesc}{mktemp}{}
Return a unique temporary filename.  This is an absolute pathname of a
file that does not exist at the time the call is made.  No two calls
will return the same filename.
\end{funcdesc}

The module uses two global variables that tell it how to construct a
temporary name.  The caller may assign values to them; by default they
are initialized at the first call to \code{mktemp()}.

\begin{datadesc}{tempdir}
When set to a value other than \code{None}, this variable defines the
directory in which filenames returned by \code{mktemp()} reside.  The
default is taken from the environment variable \code{TMPDIR}; if this
is not set, either \code{/usr/tmp} is used (on \UNIX{}), or the current
working directory (all other systems).  No check is made to see
whether its value is valid.
\end{datadesc}
\ttindex{TMPDIR}

\begin{datadesc}{template}
When set to a value other than \code{None}, this variable defines the
prefix of the final component of the filenames returned by
\code{mktemp()}.  A string of decimal digits is added to generate
unique filenames.  The default is either ``\code{@\var{pid}.}'' where
\var{pid} is the current process ID (on \UNIX{}), or ``\code{tmp}'' (all
other systems).
\end{datadesc}

Warning: if a \UNIX{} process uses \code{mktemp()}, then calls
\code{fork()} and both parent and child continue to use
\code{mktemp()}, the processes will generate conflicting temporary
names.  To resolve this, the child process should assign \code{None}
to \code{template}, to force recomputing the default on the next call
to \code{mktemp()}.

\section{Standard Module \sectcode{errno}}
\stmodindex{errno}

\renewcommand{\indexsubitem}{(in module errno)}

This module makes available standard errno system symbols.
The value of each symbol is the corresponding integer value.
The names and descriptions are borrowed from \file{linux/include/errno.h},
which should be pretty all-inclusive.  Of the following list, symbols
that are not used on the current platform are not defined by the
module.

The module also defines the dictionary variable \code{errorcode} which
maps numeric error codes back to their symbol names, so that e.g.
\code{errno.errorcode[errno.EPERM] == 'EPERM'}.  To translate a
numeric error code to an error message, use \code{os.strerror()}.

Symbols available can include:
\begin{datadesc}{EPERM} Operation not permitted \end{datadesc}
\begin{datadesc}{ENOENT} No such file or directory \end{datadesc}
\begin{datadesc}{ESRCH} No such process \end{datadesc}
\begin{datadesc}{EINTR} Interrupted system call \end{datadesc}
\begin{datadesc}{EIO} I/O error \end{datadesc}
\begin{datadesc}{ENXIO} No such device or address \end{datadesc}
\begin{datadesc}{E2BIG} Arg list too long \end{datadesc}
\begin{datadesc}{ENOEXEC} Exec format error \end{datadesc}
\begin{datadesc}{EBADF} Bad file number \end{datadesc}
\begin{datadesc}{ECHILD} No child processes \end{datadesc}
\begin{datadesc}{EAGAIN} Try again \end{datadesc}
\begin{datadesc}{ENOMEM} Out of memory \end{datadesc}
\begin{datadesc}{EACCES} Permission denied \end{datadesc}
\begin{datadesc}{EFAULT} Bad address \end{datadesc}
\begin{datadesc}{ENOTBLK} Block device required \end{datadesc}
\begin{datadesc}{EBUSY} Device or resource busy \end{datadesc}
\begin{datadesc}{EEXIST} File exists \end{datadesc}
\begin{datadesc}{EXDEV} Cross-device link \end{datadesc}
\begin{datadesc}{ENODEV} No such device \end{datadesc}
\begin{datadesc}{ENOTDIR} Not a directory \end{datadesc}
\begin{datadesc}{EISDIR} Is a directory \end{datadesc}
\begin{datadesc}{EINVAL} Invalid argument \end{datadesc}
\begin{datadesc}{ENFILE} File table overflow \end{datadesc}
\begin{datadesc}{EMFILE} Too many open files \end{datadesc}
\begin{datadesc}{ENOTTY} Not a typewriter \end{datadesc}
\begin{datadesc}{ETXTBSY} Text file busy \end{datadesc}
\begin{datadesc}{EFBIG} File too large \end{datadesc}
\begin{datadesc}{ENOSPC} No space left on device \end{datadesc}
\begin{datadesc}{ESPIPE} Illegal seek \end{datadesc}
\begin{datadesc}{EROFS} Read-only file system \end{datadesc}
\begin{datadesc}{EMLINK} Too many links \end{datadesc}
\begin{datadesc}{EPIPE} Broken pipe \end{datadesc}
\begin{datadesc}{EDOM} Math argument out of domain of func \end{datadesc}
\begin{datadesc}{ERANGE} Math result not representable \end{datadesc}
\begin{datadesc}{EDEADLK} Resource deadlock would occur \end{datadesc}
\begin{datadesc}{ENAMETOOLONG} File name too long \end{datadesc}
\begin{datadesc}{ENOLCK} No record locks available \end{datadesc}
\begin{datadesc}{ENOSYS} Function not implemented \end{datadesc}
\begin{datadesc}{ENOTEMPTY} Directory not empty \end{datadesc}
\begin{datadesc}{ELOOP} Too many symbolic links encountered \end{datadesc}
\begin{datadesc}{EWOULDBLOCK} Operation would block \end{datadesc}
\begin{datadesc}{ENOMSG} No message of desired type \end{datadesc}
\begin{datadesc}{EIDRM} Identifier removed \end{datadesc}
\begin{datadesc}{ECHRNG} Channel number out of range \end{datadesc}
\begin{datadesc}{EL2NSYNC} Level 2 not synchronized \end{datadesc}
\begin{datadesc}{EL3HLT} Level 3 halted \end{datadesc}
\begin{datadesc}{EL3RST} Level 3 reset \end{datadesc}
\begin{datadesc}{ELNRNG} Link number out of range \end{datadesc}
\begin{datadesc}{EUNATCH} Protocol driver not attached \end{datadesc}
\begin{datadesc}{ENOCSI} No CSI structure available \end{datadesc}
\begin{datadesc}{EL2HLT} Level 2 halted \end{datadesc}
\begin{datadesc}{EBADE} Invalid exchange \end{datadesc}
\begin{datadesc}{EBADR} Invalid request descriptor \end{datadesc}
\begin{datadesc}{EXFULL} Exchange full \end{datadesc}
\begin{datadesc}{ENOANO} No anode \end{datadesc}
\begin{datadesc}{EBADRQC} Invalid request code \end{datadesc}
\begin{datadesc}{EBADSLT} Invalid slot \end{datadesc}
\begin{datadesc}{EDEADLOCK} File locking deadlock error \end{datadesc}
\begin{datadesc}{EBFONT} Bad font file format \end{datadesc}
\begin{datadesc}{ENOSTR} Device not a stream \end{datadesc}
\begin{datadesc}{ENODATA} No data available \end{datadesc}
\begin{datadesc}{ETIME} Timer expired \end{datadesc}
\begin{datadesc}{ENOSR} Out of streams resources \end{datadesc}
\begin{datadesc}{ENONET} Machine is not on the network \end{datadesc}
\begin{datadesc}{ENOPKG} Package not installed \end{datadesc}
\begin{datadesc}{EREMOTE} Object is remote \end{datadesc}
\begin{datadesc}{ENOLINK} Link has been severed \end{datadesc}
\begin{datadesc}{EADV} Advertise error \end{datadesc}
\begin{datadesc}{ESRMNT} Srmount error \end{datadesc}
\begin{datadesc}{ECOMM} Communication error on send \end{datadesc}
\begin{datadesc}{EPROTO} Protocol error \end{datadesc}
\begin{datadesc}{EMULTIHOP} Multihop attempted \end{datadesc}
\begin{datadesc}{EDOTDOT} RFS specific error \end{datadesc}
\begin{datadesc}{EBADMSG} Not a data message \end{datadesc}
\begin{datadesc}{EOVERFLOW} Value too large for defined data type \end{datadesc}
\begin{datadesc}{ENOTUNIQ} Name not unique on network \end{datadesc}
\begin{datadesc}{EBADFD} File descriptor in bad state \end{datadesc}
\begin{datadesc}{EREMCHG} Remote address changed \end{datadesc}
\begin{datadesc}{ELIBACC} Can not access a needed shared library \end{datadesc}
\begin{datadesc}{ELIBBAD} Accessing a corrupted shared library \end{datadesc}
\begin{datadesc}{ELIBSCN} .lib section in a.out corrupted \end{datadesc}
\begin{datadesc}{ELIBMAX} Attempting to link in too many shared libraries \end{datadesc}
\begin{datadesc}{ELIBEXEC} Cannot exec a shared library directly \end{datadesc}
\begin{datadesc}{EILSEQ} Illegal byte sequence \end{datadesc}
\begin{datadesc}{ERESTART} Interrupted system call should be restarted \end{datadesc}
\begin{datadesc}{ESTRPIPE} Streams pipe error \end{datadesc}
\begin{datadesc}{EUSERS} Too many users \end{datadesc}
\begin{datadesc}{ENOTSOCK} Socket operation on non-socket \end{datadesc}
\begin{datadesc}{EDESTADDRREQ} Destination address required \end{datadesc}
\begin{datadesc}{EMSGSIZE} Message too long \end{datadesc}
\begin{datadesc}{EPROTOTYPE} Protocol wrong type for socket \end{datadesc}
\begin{datadesc}{ENOPROTOOPT} Protocol not available \end{datadesc}
\begin{datadesc}{EPROTONOSUPPORT} Protocol not supported \end{datadesc}
\begin{datadesc}{ESOCKTNOSUPPORT} Socket type not supported \end{datadesc}
\begin{datadesc}{EOPNOTSUPP} Operation not supported on transport endpoint \end{datadesc}
\begin{datadesc}{EPFNOSUPPORT} Protocol family not supported \end{datadesc}
\begin{datadesc}{EAFNOSUPPORT} Address family not supported by protocol \end{datadesc}
\begin{datadesc}{EADDRINUSE} Address already in use \end{datadesc}
\begin{datadesc}{EADDRNOTAVAIL} Cannot assign requested address \end{datadesc}
\begin{datadesc}{ENETDOWN} Network is down \end{datadesc}
\begin{datadesc}{ENETUNREACH} Network is unreachable \end{datadesc}
\begin{datadesc}{ENETRESET} Network dropped connection because of reset \end{datadesc}
\begin{datadesc}{ECONNABORTED} Software caused connection abort \end{datadesc}
\begin{datadesc}{ECONNRESET} Connection reset by peer \end{datadesc}
\begin{datadesc}{ENOBUFS} No buffer space available \end{datadesc}
\begin{datadesc}{EISCONN} Transport endpoint is already connected \end{datadesc}
\begin{datadesc}{ENOTCONN} Transport endpoint is not connected \end{datadesc}
\begin{datadesc}{ESHUTDOWN} Cannot send after transport endpoint shutdown \end{datadesc}
\begin{datadesc}{ETOOMANYREFS} Too many references: cannot splice \end{datadesc}
\begin{datadesc}{ETIMEDOUT} Connection timed out \end{datadesc}
\begin{datadesc}{ECONNREFUSED} Connection refused \end{datadesc}
\begin{datadesc}{EHOSTDOWN} Host is down \end{datadesc}
\begin{datadesc}{EHOSTUNREACH} No route to host \end{datadesc}
\begin{datadesc}{EALREADY} Operation already in progress \end{datadesc}
\begin{datadesc}{EINPROGRESS} Operation now in progress \end{datadesc}
\begin{datadesc}{ESTALE} Stale NFS file handle \end{datadesc}
\begin{datadesc}{EUCLEAN} Structure needs cleaning \end{datadesc}
\begin{datadesc}{ENOTNAM} Not a XENIX named type file \end{datadesc}
\begin{datadesc}{ENAVAIL} No XENIX semaphores available \end{datadesc}
\begin{datadesc}{EISNAM} Is a named type file \end{datadesc}
\begin{datadesc}{EREMOTEIO} Remote I/O error \end{datadesc}
\begin{datadesc}{EDQUOT} Quota exceeded \end{datadesc}


\section{\module{glob} ---
         \UNIX{} style pathname pattern expansion}

\declaremodule{standard}{glob}
\modulesynopsis{\UNIX{} shell style pathname pattern expansion.}


The \module{glob} module finds all the pathnames matching a specified
pattern according to the rules used by the \UNIX{} shell.  No tilde
expansion is done, but \code{*}, \code{?}, and character ranges
expressed with \code{[]} will be correctly matched.  This is done by
using the \function{os.listdir()} and \function{fnmatch.fnmatch()}
functions in concert, and not by actually invoking a subshell.  (For
tilde and shell variable expansion, use \function{os.path.expanduser()}
and \function{os.path.expandvars()}.)
\index{filenames!pathname expansion}

\begin{funcdesc}{glob}{pathname}
Returns a possibly-empty list of path names that match \var{pathname},
which must be a string containing a path specification.
\var{pathname} can be either absolute (like
\file{/usr/src/Python-1.5/Makefile}) or relative (like
\file{../../Tools/*/*.gif}), and can contain shell-style wildcards.
\end{funcdesc}

For example, consider a directory containing only the following files:
\file{1.gif}, \file{2.txt}, and \file{card.gif}.  \function{glob()}
will produce the following results.  Notice how any leading components
of the path are preserved.

\begin{verbatim}
>>> import glob
>>> glob.glob('./[0-9].*')
['./1.gif', './2.txt']
>>> glob.glob('*.gif')
['1.gif', 'card.gif']
>>> glob.glob('?.gif')
['1.gif']
\end{verbatim}


\begin{seealso}
  \seemodule{fnmatch}{Shell-style filename (not path) expansion}
\end{seealso}

\section{Standard Module \sectcode{fnmatch}}
\label{module-fnmatch}
\stmodindex{fnmatch}

This module provides support for \UNIX{} shell-style wildcards, which
are \emph{not} the same as regular expressions (which are documented
in the \code{re}\refstmodindex{re} module).  The special characters
used in shell-style wildcards are:
\begin{itemize}
\item[\code{*}] matches everything
\item[\code{?}]	matches any single character
\item[\code{[}\var{seq}\code{]}] matches any character in \var{seq}
\item[\code{[!}\var{seq}\code{]}] matches any character not in \var{seq}
\end{itemize}

Note that the filename separator (\code{'/'} on \UNIX{}) is \emph{not}
special to this module.  See module \code{glob}\refstmodindex{glob}
for pathname expansion (\code{glob} uses \code{fnmatch()} to
match filename segments).

\renewcommand{\indexsubitem}{(in module fnmatch)}

\begin{funcdesc}{fnmatch}{filename, pattern}
Test whether the \var{filename} string matches the \var{pattern}
string, returning true or false.  If the operating system is
case-insensitive, then both parameters will be normalized to all
lower- or upper-case before the comparision is performed.  If you
require a case-sensitive comparision regardless of whether that's
standard for your operating system, use \code{fnmatchcase()} instead.
\end{funcdesc}

\begin{funcdesc}{fnmatchcase}{filename, pattern}
Test whether \var{filename} matches \var{pattern}, returning true or
false; the comparision is case-sensitive.
\end{funcdesc}

\begin{seealso}

\seemodule{glob}{Shell-style path expansion}
\end{seealso}

\section{\module{locale} ---
         Internationalization services.}
\declaremodule{standard}{locale}


\modulesynopsis{Internationalization services.}


The \code{locale} module opens access to the \POSIX{} locale database
and functionality. The \POSIX{} locale mechanism allows programmers
to deal with certain cultural issues in an application, without
requiring the programmer to know all the specifics of each country
where the software is executed.

The \module{locale} module is implemented on top of the
\module{_locale}\refbimodindex{_locale} module, which in turn uses an
ANSI \C{} locale implementation if available.

The \module{locale} module defines the following exception and
functions:


\begin{funcdesc}{setlocale}{category\optional{, value}}
If \var{value} is specified, modifies the locale setting for the
\var{category}. The available categories are listed in the data
description below. The value is the name of a locale. An empty string
specifies the user's default settings. If the modification of the
locale fails, the exception \exception{Error} is
raised. If successful, the new locale setting is returned.

If no \var{value} is specified, the current setting for the
\var{category} is returned.

\function{setlocale()} is not thread safe on most systems. Applications
typically start with a call of
\begin{verbatim}
import locale
locale.setlocale(locale.LC_ALL,"")
\end{verbatim}
This sets the locale for all categories to the user's default setting
(typically specified in the \code{LANG} environment variable). If the
locale is not changed thereafter, using multithreading should not
cause problems.
\end{funcdesc}

\begin{excdesc}{Error}
Exception raised when \function{setlocale()} fails.
\end{excdesc}

\begin{funcdesc}{localeconv}{}
Returns the database of of the local conventions as a dictionary. This
dictionary has the following strings as keys:
\begin{itemize}
\item \code{decimal_point} specifies the decimal point used in
floating point number representations for the \code{LC_NUMERIC}
category.
\item \code{grouping} is a sequence of numbers specifying at which
relative positions the \code{thousands_sep} is expected. If the
sequence is terminated with \code{locale.CHAR_MAX}, no further
grouping is performed. If the sequence terminates with a \code{0}, the last
group size is repeatedly used.
\item \code{thousands_sep} is the character used between groups.
\item \code{int_curr_symbol} specifies the international currency
symbol from the \code{LC_MONETARY} category.
\item \code{currency_symbol} is the local currency symbol.
\item \code{mon_decimal_point} is the decimal point used in monetary
values.
\item \code{mon_thousands_sep} is the separator for grouping of
monetary values.
\item \code{mon_grouping} has the same format as the \code{grouping}
key; it is used for monetary values.
\item \code{positive_sign} and \code{negative_sign} gives the sign
used for positive and negative monetary quantities.
\item \code{int_frac_digits} and \code{frac_digits} specify the number
of fractional digits used in the international and local formatting
of monetary values.
\item \code{p_cs_precedes} and \code{n_cs_precedes} specifies whether
the currency symbol precedes the value for positive or negative
values.
\item \code{p_sep_by_space} and \code{n_sep_by_space} specifies
whether there is a space between the positive or negative value and
the currency symbol.
\item \code{p_sign_posn} and \code{n_sign_posn} indicate how the
sign should be placed for positive and negative monetary values. 
\end{itemize}

The possible values for \code{p_sign_posn} and \code{n_sign_posn}
are given below.

\begin{tableii}{c|l}{code}{Value}{Explanation}
\lineii{0}{Currency and value are surrounded by parentheses.}
\lineii{1}{The sign should precede the value and currency symbol.}
\lineii{2}{The sign should follow the value and currency symbol.}
\lineii{3}{The sign should immediately precede the value.}
\lineii{4}{The sign should immediately follow the value.}
\lineii{LC_MAX}{Nothing is specified in this locale.}
\end{tableii}
\end{funcdesc}

\begin{funcdesc}{strcoll}{string1,string2}
Compares two strings according to the current \constant{LC_COLLATE}
setting. As any other compare function, returns a negative, or a
positive value, or \code{0}, depending on whether \var{string1}
collates before or after \var{string2} or is equal to it.
\end{funcdesc}

\begin{funcdesc}{strxfrm}{string}
Transforms a string to one that can be used for the built-in function
\function{cmp()}\bifuncindex{cmp}, and still returns locale-aware
results.  This function can be used when the same string is compared
repeatedly, e.g. when collating a sequence of strings.
\end{funcdesc}

\begin{funcdesc}{format}{format, val, \optional{grouping\code{ = 0}}}
Formats a number \var{val} according to the current
\constant{LC_NUMERIC} setting.  The format follows the conventions of
the \code{\%} operator.  For floating point values, the decimal point
is modified if appropriate.  If \var{grouping} is true, also takes the
grouping into account.
\end{funcdesc}

\begin{funcdesc}{str}{float}
Formats a floating point number using the same format as the built-in
function \code{str(\var{float})}, but takes the decimal point into
account.
\end{funcdesc}

\begin{funcdesc}{atof}{string}
Converts a string to a floating point number, following the
\constant{LC_NUMERIC} settings.
\end{funcdesc}

\begin{funcdesc}{atoi}{string}
Converts a string to an integer, following the \constant{LC_NUMERIC}
conventions.
\end{funcdesc}

\begin{datadesc}{LC_CTYPE}
\refstmodindex{string}
Locale category for the character type functions. Depending on the
settings of this category, the functions of module \module{string}
dealing with case change their behaviour.
\end{datadesc}

\begin{datadesc}{LC_COLLATE}
Locale category for sorting strings. The functions
\function{strcoll()} and \function{strxfrm()} of the \module{locale}
module are affected.
\end{datadesc}

\begin{datadesc}{LC_TIME}
Locale category for the formatting of time. The function
\function{time.strftime()} follows these conventions.
\end{datadesc}

\begin{datadesc}{LC_MONETARY}
Locale category for formatting of monetary values. The available
options are available from the \function{localeconv()} function.
\end{datadesc}

\begin{datadesc}{LC_MESSAGES}
Locale category for message display. Python currently does not support
application specific locale-aware messages. Messages displayed by the
operating system, like those returned by \function{os.strerror()}
might be affected by this category.
\end{datadesc}

\begin{datadesc}{LC_NUMERIC}
Locale category for formatting numbers. The functions
\function{format()}, \function{atoi()}, \function{atof()} and
\function{str()} of the \module{locale} module are affected by that
category. All other numeric formatting operations are not affected.
\end{datadesc}

\begin{datadesc}{LC_ALL}
Combination of all locale settings. If this flag is used when the
locale is changed, setting the locale for all categories is
attempted. If that fails for any category, no category is changed at
all. When the locale is retrieved using this flag, a string indicating
the setting for all categories is returned. This string can be later
used to restore the settings.
\end{datadesc}

\begin{datadesc}{CHAR_MAX}
This is a symbolic constant used for different values returned by
\function{localeconv()}.
\end{datadesc}

Example:

\begin{verbatim}
>>> import locale
>>> loc = locale.setlocale(locale.LC_ALL) # get current locale
>>> locale.setlocale(locale.LC_ALL, "de") # use German locale
>>> locale.strcoll("f\344n", "foo") # compare a string containing an umlaut 
>>> locale.setlocale(locale.LC_ALL, "") # use user's preferred locale
>>> locale.setlocale(locale.LC_ALL, "C") # use default (C) locale
>>> locale.setlocale(locale.LC_ALL, loc) # restore saved locale
\end{verbatim}

\subsection{Background, details, hints, tips and caveats}

The C standard defines the locale as a program-wide property that may
be relatively expensive to change.  On top of that, some
implementation are broken in such a way that frequent locale changes
may cause core dumps.  This makes the locale somewhat painful to use
correctly.

Initially, when a program is started, the locale is the \samp{C} locale, no
matter what the user's preferred locale is.  The program must
explicitly say that it wants the user's preferred locale settings by
calling \code{setlocale(LC_ALL, "")}.

It is generally a bad idea to call \function{setlocale()} in some library
routine, since as a side effect it affects the entire program.  Saving
and restoring it is almost as bad: it is expensive and affects other
threads that happen to run before the settings have been restored.

If, when coding a module for general use, you need a locale
independent version of an operation that is affected by the locale
(e.g. \function{string.lower()}, or certain formats used with
\function{time.strftime()})), you will have to find a way to do it
without using the standard library routine.  Even better is convincing
yourself that using locale settings is okay.  Only as a last resort
should you document that your module is not compatible with
non-\samp{C} locale settings.

The case conversion functions in the
\module{string}\refstmodindex{string} and
\module{strop}\refbimodindex{strop} modules are affected by the locale
settings.  When a call to the \function{setlocale()} function changes
the \constant{LC_CTYPE} settings, the variables
\code{string.lowercase}, \code{string.uppercase} and
\code{string.letters} (and their counterparts in \module{strop}) are
recalculated.  Note that this code that uses these variable through
`\keyword{from} ... \keyword{import} ...', e.g. \code{from string
import letters}, is not affected by subsequent \function{setlocale()}
calls.

The only way to perform numeric operations according to the locale
is to use the special functions defined by this module:
\function{atof()}, \function{atoi()}, \function{format()},
\function{str()}.

\subsection{For extension writers and programs that embed Python}
\label{embedding-locale}

Extension modules should never call \function{setlocale()}, except to
find out what the current locale is.  But since the return value can
only be used portably to restore it, that is not very useful (except
perhaps to find out whether or not the locale is \samp{C}).

When Python is embedded in an application, if the application sets the
locale to something specific before initializing Python, that is
generally okay, and Python will use whatever locale is set,
\emph{except} that the \constant{LC_NUMERIC} locale should always be
\samp{C}.

The \function{setlocale()} function in the \module{locale} module contains
gives the Python progammer the impression that you can manipulate the
\constant{LC_NUMERIC} locale setting, but this not the case at the \C{}
level: \C{} code will always find that the \constant{LC_NUMERIC} locale
setting is \samp{C}.  This is because too much would break when the
decimal point character is set to something else than a period
(e.g. the Python parser would break).  Caveat: threads that run
without holding Python's global interpreter lock may occasionally find
that the numeric locale setting differs; this is because the only
portable way to implement this feature is to set the numeric locale
settings to what the user requests, extract the relevant
characteristics, and then restore the \samp{C} numeric locale.

When Python code uses the \module{locale} module to change the locale,
this also affects the embedding application.  If the embedding
application doesn't want this to happen, it should remove the
\module{_locale} extension module (which does all the work) from the
table of built-in modules in the \file{config.c} file, and make sure
that the \module{_locale} module is not accessible as a shared library.


\chapter{Optional Operating System Services}

The modules described in this chapter provide interfaces to operating
system features that are available on selected operating systems only.
The interfaces are generally modelled after the \UNIX{} or C
interfaces but they are available on some other systems as well
(e.g. Windows or NT).  Here's an overview:

\begin{description}

\item[signal]
--- Set handlers for asynchronous events.

\item[socket]
--- Low-level networking interface.

\item[select]
--- Wait for I/O completion on multiple streams.

\item[thread]
--- Create multiple threads of control within one namespace.

\item[anydbm]
--- Generic interface to DBM-style database modules.

\item[whichdbm]
--- Guess which DBM-style module created a given database.

\item[zlib]
\item[gzip]
--- Compression and decompression compatible with the
\code{gzip} program (zlib is the low-level interface, gzip the
high-level one).

\end{description}
		% Optional Operating System Services
\section{\module{signal} ---
         Set handlers for asynchronous events}

\declaremodule{builtin}{signal}
\modulesynopsis{Set handlers for asynchronous events.}


This module provides mechanisms to use signal handlers in Python.
Some general rules for working with signals and their handlers:

\begin{itemize}

\item
A handler for a particular signal, once set, remains installed until
it is explicitly reset (i.e. Python emulates the BSD style interface
regardless of the underlying implementation), with the exception of
the handler for \constant{SIGCHLD}, which follows the underlying
implementation.

\item
There is no way to ``block'' signals temporarily from critical
sections (since this is not supported by all \UNIX{} flavors).

\item
Although Python signal handlers are called asynchronously as far as
the Python user is concerned, they can only occur between the
``atomic'' instructions of the Python interpreter.  This means that
signals arriving during long calculations implemented purely in \C{}
(e.g.\ regular expression matches on large bodies of text) may be
delayed for an arbitrary amount of time.

\item
When a signal arrives during an I/O operation, it is possible that the
I/O operation raises an exception after the signal handler returns.
This is dependent on the underlying \UNIX{} system's semantics regarding
interrupted system calls.

\item
Because the \C{} signal handler always returns, it makes little sense to
catch synchronous errors like \constant{SIGFPE} or \constant{SIGSEGV}.

\item
Python installs a small number of signal handlers by default:
\constant{SIGPIPE} is ignored (so write errors on pipes and sockets can be
reported as ordinary Python exceptions) and \constant{SIGINT} is translated
into a \exception{KeyboardInterrupt} exception.  All of these can be
overridden.

\item
Some care must be taken if both signals and threads are used in the
same program.  The fundamental thing to remember in using signals and
threads simultaneously is:\ always perform \function{signal()} operations
in the main thread of execution.  Any thread can perform an
\function{alarm()}, \function{getsignal()}, or \function{pause()};
only the main thread can set a new signal handler, and the main thread
will be the only one to receive signals (this is enforced by the
Python \module{signal} module, even if the underlying thread
implementation supports sending signals to individual threads).  This
means that signals can't be used as a means of inter-thread
communication.  Use locks instead.

\end{itemize}

The variables defined in the \module{signal} module are:

\begin{datadesc}{SIG_DFL}
  This is one of two standard signal handling options; it will simply
  perform the default function for the signal.  For example, on most
  systems the default action for \constant{SIGQUIT} is to dump core
  and exit, while the default action for \constant{SIGCLD} is to
  simply ignore it.
\end{datadesc}

\begin{datadesc}{SIG_IGN}
  This is another standard signal handler, which will simply ignore
  the given signal.
\end{datadesc}

\begin{datadesc}{SIG*}
  All the signal numbers are defined symbolically.  For example, the
  hangup signal is defined as \constant{signal.SIGHUP}; the variable names
  are identical to the names used in C programs, as found in
  \code{<signal.h>}.
  The \UNIX{} man page for `\cfunction{signal()}' lists the existing
  signals (on some systems this is \manpage{signal}{2}, on others the
  list is in \manpage{signal}{7}).
  Note that not all systems define the same set of signal names; only
  those names defined by the system are defined by this module.
\end{datadesc}

\begin{datadesc}{NSIG}
  One more than the number of the highest signal number.
\end{datadesc}

The \module{signal} module defines the following functions:

\begin{funcdesc}{alarm}{time}
  If \var{time} is non-zero, this function requests that a
  \constant{SIGALRM} signal be sent to the process in \var{time} seconds.
  Any previously scheduled alarm is canceled (i.e.\ only one alarm can
  be scheduled at any time).  The returned value is then the number of
  seconds before any previously set alarm was to have been delivered.
  If \var{time} is zero, no alarm id scheduled, and any scheduled
  alarm is canceled.  The return value is the number of seconds
  remaining before a previously scheduled alarm.  If the return value
  is zero, no alarm is currently scheduled.  (See the \UNIX{} man page
  \manpage{alarm}{2}.)
  Availability: \UNIX.
\end{funcdesc}

\begin{funcdesc}{getsignal}{signalnum}
  Return the current signal handler for the signal \var{signalnum}.
  The returned value may be a callable Python object, or one of the
  special values \constant{signal.SIG_IGN}, \constant{signal.SIG_DFL} or
  \constant{None}.  Here, \constant{signal.SIG_IGN} means that the
  signal was previously ignored, \constant{signal.SIG_DFL} means that the
  default way of handling the signal was previously in use, and
  \code{None} means that the previous signal handler was not installed
  from Python.
\end{funcdesc}

\begin{funcdesc}{pause}{}
  Cause the process to sleep until a signal is received; the
  appropriate handler will then be called.  Returns nothing.  (See the
  \UNIX{} man page \manpage{signal}{2}.)
\end{funcdesc}

\begin{funcdesc}{signal}{signalnum, handler}
  Set the handler for signal \var{signalnum} to the function
  \var{handler}.  \var{handler} can be a callable Python object
  taking two arguments (see below), or
  one of the special values \constant{signal.SIG_IGN} or
  \constant{signal.SIG_DFL}.  The previous signal handler will be returned
  (see the description of \function{getsignal()} above).  (See the
  \UNIX{} man page \manpage{signal}{2}.)

  When threads are enabled, this function can only be called from the
  main thread; attempting to call it from other threads will cause a
  \exception{ValueError} exception to be raised.

  The \var{handler} is called with two arguments: the signal number
  and the current stack frame (\code{None} or a frame object; see the
  reference manual for a description of frame objects).
\obindex{frame}
\end{funcdesc}

\subsection{Example}
\nodename{Signal Example}

Here is a minimal example program. It uses the \function{alarm()}
function to limit the time spent waiting to open a file; this is
useful if the file is for a serial device that may not be turned on,
which would normally cause the \function{os.open()} to hang
indefinitely.  The solution is to set a 5-second alarm before opening
the file; if the operation takes too long, the alarm signal will be
sent, and the handler raises an exception.

\begin{verbatim}
import signal, os, FCNTL

def handler(signum, frame):
    print 'Signal handler called with signal', signum
    raise IOError, "Couldn't open device!"

# Set the signal handler and a 5-second alarm
signal.signal(signal.SIGALRM, handler)
signal.alarm(5)

# This open() may hang indefinitely
fd = os.open('/dev/ttyS0', FCNTL.O_RDWR)  

signal.alarm(0)          # Disable the alarm
\end{verbatim}

\section{\module{socket} ---
         Low-level networking interface}

\declaremodule{builtin}{socket}
\modulesynopsis{Low-level networking interface.}


This module provides access to the BSD \emph{socket} interface.
It is available on all modern \UNIX{} systems, Windows, MacOS, BeOS,
OS/2, and probably additional platforms.

For an introduction to socket programming (in C), see the following
papers: \citetitle{An Introductory 4.3BSD Interprocess Communication
Tutorial}, by Stuart Sechrest and \citetitle{An Advanced 4.3BSD
Interprocess Communication Tutorial}, by Samuel J.  Leffler et al,
both in the \citetitle{\UNIX{} Programmer's Manual, Supplementary Documents 1}
(sections PS1:7 and PS1:8).  The platform-specific reference material
for the various socket-related system calls are also a valuable source
of information on the details of socket semantics.  For \UNIX, refer
to the manual pages; for Windows, see the WinSock (or Winsock 2)
specification.

The Python interface is a straightforward transliteration of the
\UNIX{} system call and library interface for sockets to Python's
object-oriented style: the \function{socket()} function returns a
\dfn{socket object}\obindex{socket} whose methods implement the
various socket system calls.  Parameter types are somewhat
higher-level than in the C interface: as with \method{read()} and
\method{write()} operations on Python files, buffer allocation on
receive operations is automatic, and buffer length is implicit on send
operations.

Socket addresses are represented as a single string for the
\constant{AF_UNIX} address family and as a pair
\code{(\var{host}, \var{port})} for the \constant{AF_INET} address
family, where \var{host} is a string representing
either a hostname in Internet domain notation like
\code{'daring.cwi.nl'} or an IP address like \code{'100.50.200.5'},
and \var{port} is an integral port number.  Other address families are
currently not supported.  The address format required by a particular
socket object is automatically selected based on the address family
specified when the socket object was created.

For IP addresses, two special forms are accepted instead of a host
address: the empty string represents \constant{INADDR_ANY}, and the string
\code{'<broadcast>'} represents \constant{INADDR_BROADCAST}.

All errors raise exceptions.  The normal exceptions for invalid
argument types and out-of-memory conditions can be raised; errors
related to socket or address semantics raise the error
\exception{socket.error}.

Non-blocking mode is supported through the
\method{setblocking()} method.

The module \module{socket} exports the following constants and functions:


\begin{excdesc}{error}
This exception is raised for socket- or address-related errors.
The accompanying value is either a string telling what went wrong or a
pair \code{(\var{errno}, \var{string})}
representing an error returned by a system
call, similar to the value accompanying \exception{os.error}.
See the module \refmodule{errno}\refbimodindex{errno}, which contains
names for the error codes defined by the underlying operating system.
\end{excdesc}

\begin{datadesc}{AF_UNIX}
\dataline{AF_INET}
These constants represent the address (and protocol) families,
used for the first argument to \function{socket()}.  If the
\constant{AF_UNIX} constant is not defined then this protocol is
unsupported.
\end{datadesc}

\begin{datadesc}{SOCK_STREAM}
\dataline{SOCK_DGRAM}
\dataline{SOCK_RAW}
\dataline{SOCK_RDM}
\dataline{SOCK_SEQPACKET}
These constants represent the socket types,
used for the second argument to \function{socket()}.
(Only \constant{SOCK_STREAM} and
\constant{SOCK_DGRAM} appear to be generally useful.)
\end{datadesc}

\begin{datadesc}{SO_*}
\dataline{SOMAXCONN}
\dataline{MSG_*}
\dataline{SOL_*}
\dataline{IPPROTO_*}
\dataline{IPPORT_*}
\dataline{INADDR_*}
\dataline{IP_*}
Many constants of these forms, documented in the \UNIX{} documentation on
sockets and/or the IP protocol, are also defined in the socket module.
They are generally used in arguments to the \method{setsockopt()} and
\method{getsockopt()} methods of socket objects.  In most cases, only
those symbols that are defined in the \UNIX{} header files are defined;
for a few symbols, default values are provided.
\end{datadesc}

\begin{funcdesc}{gethostbyname}{hostname}
Translate a host name to IP address format.  The IP address is
returned as a string, e.g.,  \code{'100.50.200.5'}.  If the host name
is an IP address itself it is returned unchanged.  See
\function{gethostbyname_ex()} for a more complete interface.
\end{funcdesc}

\begin{funcdesc}{gethostbyname_ex}{hostname}
Translate a host name to IP address format, extended interface.
Return a triple \code{(hostname, aliaslist, ipaddrlist)} where
\code{hostname} is the primary host name responding to the given
\var{ip_address}, \code{aliaslist} is a (possibly empty) list of
alternative host names for the same address, and \code{ipaddrlist} is
a list of IP addresses for the same interface on the same
host (often but not always a single address).
\end{funcdesc}

\begin{funcdesc}{gethostname}{}
Return a string containing the hostname of the machine where 
the Python interpreter is currently executing.  If you want to know the
current machine's IP address, use \code{gethostbyname(gethostname())}.
Note: \function{gethostname()} doesn't always return the fully qualified
domain name; use \code{gethostbyaddr(gethostname())}
(see below).
\end{funcdesc}

\begin{funcdesc}{gethostbyaddr}{ip_address}
Return a triple \code{(\var{hostname}, \var{aliaslist},
\var{ipaddrlist})} where \var{hostname} is the primary host name
responding to the given \var{ip_address}, \var{aliaslist} is a
(possibly empty) list of alternative host names for the same address,
and \var{ipaddrlist} is a list of IP addresses for the same interface
on the same host (most likely containing only a single address).
To find the fully qualified domain name, check \var{hostname} and the
items of \var{aliaslist} for an entry containing at least one period.
\end{funcdesc}

\begin{funcdesc}{getprotobyname}{protocolname}
Translate an Internet protocol name (e.g.\ \code{'icmp'}) to a constant
suitable for passing as the (optional) third argument to the
\function{socket()} function.  This is usually only needed for sockets
opened in ``raw'' mode (\constant{SOCK_RAW}); for the normal socket
modes, the correct protocol is chosen automatically if the protocol is
omitted or zero.
\end{funcdesc}

\begin{funcdesc}{getservbyname}{servicename, protocolname}
Translate an Internet service name and protocol name to a port number
for that service.  The protocol name should be \code{'tcp'} or
\code{'udp'}.
\end{funcdesc}

\begin{funcdesc}{socket}{family, type\optional{, proto}}
Create a new socket using the given address family, socket type and
protocol number.  The address family should be \constant{AF_INET} or
\constant{AF_UNIX}.  The socket type should be \constant{SOCK_STREAM},
\constant{SOCK_DGRAM} or perhaps one of the other \samp{SOCK_} constants.
The protocol number is usually zero and may be omitted in that case.
\end{funcdesc}

\begin{funcdesc}{fromfd}{fd, family, type\optional{, proto}}
Build a socket object from an existing file descriptor (an integer as
returned by a file object's \method{fileno()} method).  Address family,
socket type and protocol number are as for the \function{socket()} function
above.  The file descriptor should refer to a socket, but this is not
checked --- subsequent operations on the object may fail if the file
descriptor is invalid.  This function is rarely needed, but can be
used to get or set socket options on a socket passed to a program as
standard input or output (e.g.\ a server started by the \UNIX{} inet
daemon).
\end{funcdesc}

\begin{funcdesc}{ntohl}{x}
Convert 32-bit integers from network to host byte order.  On machines
where the host byte order is the same as network byte order, this is a
no-op; otherwise, it performs a 4-byte swap operation.
\end{funcdesc}

\begin{funcdesc}{ntohs}{x}
Convert 16-bit integers from network to host byte order.  On machines
where the host byte order is the same as network byte order, this is a
no-op; otherwise, it performs a 2-byte swap operation.
\end{funcdesc}

\begin{funcdesc}{htonl}{x}
Convert 32-bit integers from host to network byte order.  On machines
where the host byte order is the same as network byte order, this is a
no-op; otherwise, it performs a 4-byte swap operation.
\end{funcdesc}

\begin{funcdesc}{htons}{x}
Convert 16-bit integers from host to network byte order.  On machines
where the host byte order is the same as network byte order, this is a
no-op; otherwise, it performs a 2-byte swap operation.
\end{funcdesc}

\begin{funcdesc}{inet_aton}{ip_string}
Convert an IP address from dotted-quad string format
(e.g.\ '123.45.67.89') to 32-bit packed binary format, as a string four
characters in length.

Useful when conversing with a program that uses the standard C library
and needs objects of type \ctype{struct in_addr}, which is the C type
for the 32-bit packed binary this function returns.

If the IP address string passed to this function is invalid,
\exception{socket.error} will be raised. Note that exactly what is
valid depends on the underlying C implementation of
\cfunction{inet_aton()}.
\end{funcdesc}

\begin{funcdesc}{inet_ntoa}{packed_ip}
Convert a 32-bit packed IP address (a string four characters in
length) to its standard dotted-quad string representation
(e.g. '123.45.67.89').

Useful when conversing with a program that uses the standard C library
and needs objects of type \ctype{struct in_addr}, which is the C type
for the 32-bit packed binary this function takes as an argument.

If the string passed to this function is not exactly 4 bytes in
length, \exception{socket.error} will be raised.
\end{funcdesc}

\begin{datadesc}{SocketType}
This is a Python type object that represents the socket object type.
It is the same as \code{type(socket(...))}.
\end{datadesc}

\subsection{Socket Objects \label{socket-objects}}

Socket objects have the following methods.  Except for
\method{makefile()} these correspond to \UNIX{} system calls
applicable to sockets.

\begin{methoddesc}[socket]{accept}{}
Accept a connection.
The socket must be bound to an address and listening for connections.
The return value is a pair \code{(\var{conn}, \var{address})}
where \var{conn} is a \emph{new} socket object usable to send and
receive data on the connection, and \var{address} is the address bound
to the socket on the other end of the connection.
\end{methoddesc}

\begin{methoddesc}[socket]{bind}{address}
Bind the socket to \var{address}.  The socket must not already be bound.
(The format of \var{address} depends on the address family --- see above.)
\end{methoddesc}

\begin{methoddesc}[socket]{close}{}
Close the socket.  All future operations on the socket object will fail.
The remote end will receive no more data (after queued data is flushed).
Sockets are automatically closed when they are garbage-collected.
\end{methoddesc}

\begin{methoddesc}[socket]{connect}{address}
Connect to a remote socket at \var{address}.
(The format of \var{address} depends on the address family --- see
above.)
\end{methoddesc}

\begin{methoddesc}[socket]{connect_ex}{address}
Like \code{connect(\var{address})}, but return an error indicator
instead of raising an exception for errors returned by the C-level
\cfunction{connect()} call (other problems, such as ``host not found,''
can still raise exceptions).  The error indicator is \code{0} if the
operation succeeded, otherwise the value of the \cdata{errno}
variable.  This is useful, e.g., for asynchronous connects.
\end{methoddesc}

\begin{methoddesc}[socket]{fileno}{}
Return the socket's file descriptor (a small integer).  This is useful
with \function{select.select()}.
\end{methoddesc}

\begin{methoddesc}[socket]{getpeername}{}
Return the remote address to which the socket is connected.  This is
useful to find out the port number of a remote IP socket, for instance.
(The format of the address returned depends on the address family ---
see above.)  On some systems this function is not supported.
\end{methoddesc}

\begin{methoddesc}[socket]{getsockname}{}
Return the socket's own address.  This is useful to find out the port
number of an IP socket, for instance.
(The format of the address returned depends on the address family ---
see above.)
\end{methoddesc}

\begin{methoddesc}[socket]{getsockopt}{level, optname\optional{, buflen}}
Return the value of the given socket option (see the \UNIX{} man page
\manpage{getsockopt}{2}).  The needed symbolic constants
(\constant{SO_*} etc.) are defined in this module.  If \var{buflen}
is absent, an integer option is assumed and its integer value
is returned by the function.  If \var{buflen} is present, it specifies
the maximum length of the buffer used to receive the option in, and
this buffer is returned as a string.  It is up to the caller to decode
the contents of the buffer (see the optional built-in module
\refmodule{struct} for a way to decode C structures encoded as strings).
\end{methoddesc}

\begin{methoddesc}[socket]{listen}{backlog}
Listen for connections made to the socket.  The \var{backlog} argument
specifies the maximum number of queued connections and should be at
least 1; the maximum value is system-dependent (usually 5).
\end{methoddesc}

\begin{methoddesc}[socket]{makefile}{\optional{mode\optional{, bufsize}}}
Return a \dfn{file object} associated with the socket.  (File objects
are described in \ref{bltin-file-objects}, ``File Objects.'')
The file object references a \cfunction{dup()}ped version of the
socket file descriptor, so the file object and socket object may be
closed or garbage-collected independently.
\index{I/O control!buffering}The optional \var{mode}
and \var{bufsize} arguments are interpreted the same way as by the
built-in \function{open()} function.
\end{methoddesc}

\begin{methoddesc}[socket]{recv}{bufsize\optional{, flags}}
Receive data from the socket.  The return value is a string representing
the data received.  The maximum amount of data to be received
at once is specified by \var{bufsize}.  See the \UNIX{} manual page
\manpage{recv}{2} for the meaning of the optional argument
\var{flags}; it defaults to zero.
\end{methoddesc}

\begin{methoddesc}[socket]{recvfrom}{bufsize\optional{, flags}}
Receive data from the socket.  The return value is a pair
\code{(\var{string}, \var{address})} where \var{string} is a string
representing the data received and \var{address} is the address of the
socket sending the data.  The optional \var{flags} argument has the
same meaning as for \method{recv()} above.
(The format of \var{address} depends on the address family --- see above.)
\end{methoddesc}

\begin{methoddesc}[socket]{send}{string\optional{, flags}}
Send data to the socket.  The socket must be connected to a remote
socket.  The optional \var{flags} argument has the same meaning as for
\method{recv()} above.  Returns the number of bytes sent.
\end{methoddesc}

\begin{methoddesc}[socket]{sendto}{string\optional{, flags}, address}
Send data to the socket.  The socket should not be connected to a
remote socket, since the destination socket is specified by
\var{address}.  The optional \var{flags} argument has the same
meaning as for \method{recv()} above.  Return the number of bytes sent.
(The format of \var{address} depends on the address family --- see above.)
\end{methoddesc}

\begin{methoddesc}[socket]{setblocking}{flag}
Set blocking or non-blocking mode of the socket: if \var{flag} is 0,
the socket is set to non-blocking, else to blocking mode.  Initially
all sockets are in blocking mode.  In non-blocking mode, if a
\method{recv()} call doesn't find any data, or if a
\method{send()} call can't immediately dispose of the data, a
\exception{error} exception is raised; in blocking mode, the calls
block until they can proceed.
\end{methoddesc}

\begin{methoddesc}[socket]{setsockopt}{level, optname, value}
Set the value of the given socket option (see the \UNIX{} man page
\manpage{setsockopt}{2}).  The needed symbolic constants are defined in
the \module{socket} module (\code{SO_*} etc.).  The value can be an
integer or a string representing a buffer.  In the latter case it is
up to the caller to ensure that the string contains the proper bits
(see the optional built-in module
\refmodule{struct}\refbimodindex{struct} for a way to encode C
structures as strings). 
\end{methoddesc}

\begin{methoddesc}[socket]{shutdown}{how}
Shut down one or both halves of the connection.  If \var{how} is
\code{0}, further receives are disallowed.  If \var{how} is \code{1},
further sends are disallowed.  If \var{how} is \code{2}, further sends
and receives are disallowed.
\end{methoddesc}

Note that there are no methods \method{read()} or \method{write()};
use \method{recv()} and \method{send()} without \var{flags} argument
instead.

\subsection{Example}
\nodename{Socket Example}

Here are two minimal example programs using the TCP/IP protocol:\ a
server that echoes all data that it receives back (servicing only one
client), and a client using it.  Note that a server must perform the
sequence \function{socket()}, \method{bind()}, \method{listen()},
\method{accept()} (possibly repeating the \method{accept()} to service
more than one client), while a client only needs the sequence
\function{socket()}, \method{connect()}.  Also note that the server
does not \method{send()}/\method{recv()} on the 
socket it is listening on but on the new socket returned by
\method{accept()}.

\begin{verbatim}
# Echo server program
from socket import *
HOST = ''                 # Symbolic name meaning the local host
PORT = 50007              # Arbitrary non-privileged server
s = socket(AF_INET, SOCK_STREAM)
s.bind(HOST, PORT)
s.listen(1)
conn, addr = s.accept()
print 'Connected by', addr
while 1:
    data = conn.recv(1024)
    if not data: break
    conn.send(data)
conn.close()
\end{verbatim}

\begin{verbatim}
# Echo client program
from socket import *
HOST = 'daring.cwi.nl'    # The remote host
PORT = 50007              # The same port as used by the server
s = socket(AF_INET, SOCK_STREAM)
s.connect(HOST, PORT)
s.send('Hello, world')
data = s.recv(1024)
s.close()
print 'Received', `data`
\end{verbatim}

\begin{seealso}
\seemodule{SocketServer}{classes that simplify writing network servers}
\end{seealso}

\section{\module{select} ---
         Waiting for I/O completion}

\declaremodule{builtin}{select}
\modulesynopsis{Wait for I/O completion on multiple streams.}


This module provides access to the function \cfunction{select()}
available in most operating systems.  Note that on Windows, it only
works for sockets; on other operating systems, it also works for other
file types (in particular, on \UNIX{}, it works on pipes).  It cannot
be used or regular files to determine whether a file has grown since
it was last read.

The module defines the following:

\begin{excdesc}{error}
The exception raised when an error occurs.  The accompanying value is
a pair containing the numeric error code from \cdata{errno} and the
corresponding string, as would be printed by the \C{} function
\cfunction{perror()}.
\end{excdesc}

\begin{funcdesc}{select}{iwtd, owtd, ewtd\optional{, timeout}}
This is a straightforward interface to the \UNIX{} \cfunction{select()}
system call.  The first three arguments are lists of `waitable
objects': either integers representing \UNIX{} file descriptors or
objects with a parameterless method named \method{fileno()} returning
such an integer.  The three lists of waitable objects are for input,
output and `exceptional conditions', respectively.  Empty lists are
allowed.  The optional \var{timeout} argument specifies a time-out as a
floating point number in seconds.  When the \var{timeout} argument
is omitted the function blocks until at least one file descriptor is
ready.  A time-out value of zero specifies a poll and never blocks.

The return value is a triple of lists of objects that are ready:
subsets of the first three arguments.  When the time-out is reached
without a file descriptor becoming ready, three empty lists are
returned.

Amongst the acceptable object types in the lists are Python file
objects (e.g. \code{sys.stdin}, or objects returned by
\function{open()} or \function{os.popen()}), socket objects
returned by \function{socket.socket()},%
\withsubitem{(in module socket)}{\ttindex{socket()}}
\withsubitem{(in module posix)}{\ttindex{popen()}}
\withsubitem{(in module os)}{\ttindex{popen()}}
and the module \module{stdwin}\refbimodindex{stdwin} which happens to
define a function \function{fileno()}%
\withsubitem{(in module stdwin)}{\ttindex{fileno()}}
for just this purpose.  You may
also define a \dfn{wrapper} class yourself, as long as it has an
appropriate \method{fileno()} method (that really returns a \UNIX{}
file descriptor, not just a random integer).
\end{funcdesc}

\section{Built-in Module \sectcode{thread}}
\bimodindex{thread}

This module provides low-level primitives for working with multiple
threads (a.k.a. \dfn{light-weight processes} or \dfn{tasks}) --- multiple
threads of control sharing their global data space.  For
synchronization, simple locks (a.k.a. \dfn{mutexes} or \dfn{binary
semaphores}) are provided.

The module is optional and supported on SGI IRIX 4.x and 5.x and Sun
Solaris 2.x systems, as well as on systems that have a PTHREAD
implementation (e.g. KSR).

It defines the following constant and functions:

\renewcommand{\indexsubitem}{(in module thread)}
\begin{excdesc}{error}
Raised on thread-specific errors.
\end{excdesc}

\begin{funcdesc}{start_new_thread}{func\, arg}
Start a new thread.  The thread executes the function \var{func}
with the argument list \var{arg} (which must be a tuple).  When the
function returns, the thread silently exits.  When the function raises
terminates with an unhandled exception, a stack trace is printed and
then the thread exits (but other threads continue to run).
\end{funcdesc}

\begin{funcdesc}{exit_thread}{}
Exit the current thread silently.  Other threads continue to run.
\strong{Caveat:} code in pending \code{finally} clauses is not executed.
\end{funcdesc}

\begin{funcdesc}{exit_prog}{status}
Exit all threads and report the value of the integer argument
\var{status} as the exit status of the entire program.
\strong{Caveat:} code in pending \code{finally} clauses, in this thread
or in other threads, is not executed.
\end{funcdesc}

\begin{funcdesc}{allocate_lock}{}
Return a new lock object.  Methods of locks are described below.  The
lock is initially unlocked.
\end{funcdesc}

\begin{funcdesc}{get_ident}{}
Return the `thread identifier' of the current thread.  This is a
nonzero integer.  Its value has no direct meaning; it is intended as a
magic cookie to be used e.g. to index a dictionary of thread-specific
data.  Thread identifiers may be recycled when a thread exits and
another thread is created.
\end{funcdesc}

Lock objects have the following methods:

\renewcommand{\indexsubitem}{(lock method)}
\begin{funcdesc}{acquire}{waitflag}
Without the optional argument, this method acquires the lock
unconditionally, if necessary waiting until it is released by another
thread (only one thread at a time can acquire a lock --- that's their
reason for existence), and returns \code{None}.  If the integer
\var{waitflag} argument is present, the action depends on its value:
if it is zero, the lock is only acquired if it can be acquired
immediately without waiting, while if it is nonzero, the lock is
acquired unconditionally as before.  If an argument is present, the
return value is 1 if the lock is acquired successfully, 0 if not.
\end{funcdesc}

\begin{funcdesc}{release}{}
Releases the lock.  The lock must have been acquired earlier, but not
necessarily by the same thread.
\end{funcdesc}

\begin{funcdesc}{locked}{}
Return the status of the lock: 1 if it has been acquired by some
thread, 0 if not.
\end{funcdesc}

{\bf Caveats:}

\begin{itemize}
\item
Threads interact strangely with interrupts: the
\code{KeyboardInterrupt} exception will be received by an arbitrary
thread.

\item
Calling \code{sys.exit(\var{status})} or executing
\code{raise SystemExit, \var{status}} is almost equivalent to calling
\code{thread.exit_prog(\var{status})}, except that the former ways of
exiting the entire program do honor \code{finally} clauses in the
current thread (but not in other threads).

\item
Not all built-in functions that may block waiting for I/O allow other
threads to run, although the most popular ones (\code{sleep},
\code{read}, \code{select}) work as expected.

\end{itemize}

\section{\module{Queue} ---
         A synchronized queue class.}
\declaremodule{standard}{Queue}

\modulesynopsis{A synchronized queue class.}



The \module{Queue} module implements a multi-producer, multi-consumer
FIFO queue.  It is especially useful in threads programming when
information must be exchanged safely between multiple threads.  The
\class{Queue} class in this module implements all the required locking
semantics.  It depends on the availability of thread support in
Python.

The \module{Queue} module defines the following class and exception:


\begin{classdesc}{Queue}{maxsize}
Constructor for the class.  \var{maxsize} is an integer that sets the
upperbound limit on the number of items that can be placed in the
queue.  Insertion will block once this size has been reached, until
queue items are consumed.  If \var{maxsize} is less than or equal to
zero, the queue size is infinite.
\end{classdesc}

\begin{excdesc}{Empty}
Exception raised when non-blocking \method{get()} (or
\method{get_nowait()}) is called on a \class{Queue} object which is
empty or locked.
\end{excdesc}

\begin{excdesc}{Full}
Exception raised when non-blocking \method{put()} (or
\method{get_nowait()}) is called on a \class{Queue} object which is
full or locked.
\end{excdesc}

\subsection{Queue Objects}
\label{QueueObjects}

Class \class{Queue} implements queue objects and has the methods
described below.  This class can be derived from in order to implement
other queue organizations (e.g. stack) but the inheritable interface
is not described here.  See the source code for details.  The public
methods are:

\begin{methoddesc}{qsize}{}
Return the approximate size of the queue.  Because of multithreading
semantics, this number is not reliable.
\end{methoddesc}

\begin{methoddesc}{empty}{}
Return \code{1} if the queue is empty, \code{0} otherwise.  Because
of multithreading semantics, this is not reliable.
\end{methoddesc}

\begin{methoddesc}{full}{}
Return \code{1} if the queue is full, \code{0} otherwise.  Because of
multithreading semantics, this is not reliable.
\end{methoddesc}

\begin{methoddesc}{put}{item\optional{, block}}
Put \var{item} into the queue.  If optional argument \var{block} is 1
(the default), block if necessary until a free slot is available.
Otherwise (\var{block} is 0), put \var{item} on the queue if a free
slot is immediately available, else raise the \exception{Full}
exception.
\end{methoddesc}

\begin{methoddesc}{put_nowait}{item}
Equivalent to \code{put(\var{item}, 0)}.
\end{methoddesc}

\begin{methoddesc}{get}{\optional{block}}
Remove and return an item from the queue.  If optional argument
\var{block} is 1 (the default), block if necessary until an item is
available.  Otherwise (\var{block} is 0), return an item if one is
immediately available, else raise the
\exception{Empty} exception.
\end{methoddesc}

\begin{methoddesc}{get_nowait}{}
Equivalent to \code{get(0)}.
\end{methoddesc}

\section{\module{anydbm} ---
         Generic access to DBM-style databases}

\declaremodule{standard}{anydbm}
\modulesynopsis{Generic interface to DBM-style database modules.}


\module{anydbm} is a generic interface to variants of the DBM
database --- \refmodule{dbhash}\refstmodindex{dbhash} (requires
\refmodule{bsddb}\refbimodindex{bsddb}),
\refmodule{gdbm}\refbimodindex{gdbm}, or
\refmodule{dbm}\refbimodindex{dbm}.  If none of these modules is
installed, the slow-but-simple implementation in module
\refmodule{dumbdbm}\refstmodindex{dumbdbm} will be used.

\begin{funcdesc}{open}{filename\optional{, flag\optional{, mode}}}
Open the database file \var{filename} and return a corresponding object.

If the database file already exists, the \refmodule{whichdb} module is 
used to determine its type and the appropriate module is used; if it
does not exist, the first module listed above that can be imported is
used.

The optional \var{flag} argument can be
\code{'r'} to open an existing database for reading only,
\code{'w'} to open an existing database for reading and writing,
\code{'c'} to create the database if it doesn't exist, or
\code{'n'}, which will always create a new empty database.  If not
specified, the default value is \code{'r'}.

The optional \var{mode} argument is the \UNIX{} mode of the file, used
only when the database has to be created.  It defaults to octal
\code{0666} (and will be modified by the prevailing umask).
\end{funcdesc}

\begin{excdesc}{error}
A tuple containing the exceptions that can be raised by each of the
supported modules, with a unique exception \exception{anydbm.error} as
the first item --- the latter is used when \exception{anydbm.error} is
raised.
\end{excdesc}

The object returned by \function{open()} supports most of the same
functionality as dictionaries; keys and their corresponding values can
be stored, retrieved, and deleted, and the \method{has_key()} and
\method{keys()} methods are available.  Keys and values must always be
strings.


\begin{seealso}
  \seemodule{anydbm}{Generic interface to \code{dbm}-style databases.}
  \seemodule{dbhash}{BSD \code{db} database interface.}
  \seemodule{dbm}{Standard \UNIX{} database interface.}
  \seemodule{dumbdbm}{Portable implementation of the \code{dbm} interface.}
  \seemodule{gdbm}{GNU database interface, based on the \code{dbm} interface.}
  \seemodule{shelve}{General object persistence built on top of 
                     the Python \code{dbm} interface.}
  \seemodule{whichdb}{Utility module used to determine the type of an
                      existing database.}
\end{seealso}


\section{\module{dumbdbm} ---
         Portable DBM implementation}

\declaremodule{standard}{dumbdbm}
\modulesynopsis{Portable implementation of the simple DBM interface.}


A simple and slow database implemented entirely in Python.  This
should only be used when no other DBM-style database is available.


\begin{funcdesc}{open}{filename\optional{, flag\optional{, mode}}}
Open the database file \var{filename} and return a corresponding
object.  The \var{flag} argument, used to control how the database is
opened in the other DBM implementations, is ignored in
\module{dumbdbm}; the database is always opened for update, and will
be created if it does not exist.

The optional \var{mode} argument is ignored.
\end{funcdesc}

\begin{excdesc}{error}
Raised for errors not reported as \exception{KeyError} errors.
\end{excdesc}


\begin{seealso}
  \seemodule{anydbm}{Generic interface to \code{dbm}-style databases.}
  \seemodule{whichdb}{Utility module used to determine the type of an
                      existing database.}
\end{seealso}

\section{Standard Module \sectcode{whichdb}}
\label{module-whichdb}
\stmodindex{whichdb}

The single function in this module attempts to guess which of the
several simple database modules available--dbm, gdbm, or
dbhash--should be used to open a given file.

\renewcommand{\indexsubitem}{(in module whichdb)}
\begin{funcdesc}{whichdb}{filename}
Returns one of the following values: \code{None} if the file can't be
opened because it's unreadable or doesn't exist; the empty string
(\code{""}) if the file's format can't be guessed; or a string
containing the required module name, such as \code{"dbm"} or
\code{"gdbm"}.
\end{funcdesc}


\section{\module{zlib} ---
         Compression compatible with \program{gzip}}

\declaremodule{builtin}{zlib}
\modulesynopsis{Low-level interface to compression and decompression
                routines compatible with \program{gzip}.}


For applications that require data compression, the functions in this
module allow compression and decompression, using the zlib library.
The zlib library has its own home page at
\url{http://www.gzip.org/zlib/}.  Version 1.1.3 is the
most recent version as of September 2000; use a later version if one
is available.  There are known incompatibilities between the Python
module and earlier versions of the zlib library.

The available exception and functions in this module are:

\begin{excdesc}{error}
  Exception raised on compression and decompression errors.
\end{excdesc}


\begin{funcdesc}{adler32}{string\optional{, value}}
   Computes a Adler-32 checksum of \var{string}.  (An Adler-32
   checksum is almost as reliable as a CRC32 but can be computed much
   more quickly.)  If \var{value} is present, it is used as the
   starting value of the checksum; otherwise, a fixed default value is
   used.  This allows computing a running checksum over the
   concatenation of several input strings.  The algorithm is not
   cryptographically strong, and should not be used for
   authentication or digital signatures.  Since the algorithm is
   designed for use as a checksum algorithm, it is not suitable for
   use as a general hash algorithm.
\end{funcdesc}

\begin{funcdesc}{compress}{string\optional{, level}}
  Compresses the data in \var{string}, returning a string contained
  compressed data.  \var{level} is an integer from \code{1} to
  \code{9} controlling the level of compression; \code{1} is fastest
  and produces the least compression, \code{9} is slowest and produces
  the most.  The default value is \code{6}.  Raises the
  \exception{error} exception if any error occurs.
\end{funcdesc}

\begin{funcdesc}{compressobj}{\optional{level}}
  Returns a compression object, to be used for compressing data streams
  that won't fit into memory at once.  \var{level} is an integer from
  \code{1} to \code{9} controlling the level of compression; \code{1} is
  fastest and produces the least compression, \code{9} is slowest and
  produces the most.  The default value is \code{6}.
\end{funcdesc}

\begin{funcdesc}{crc32}{string\optional{, value}}
  Computes a CRC (Cyclic Redundancy Check)%
  \index{Cyclic Redundancy Check}
  \index{checksum!Cyclic Redundancy Check}
  checksum of \var{string}. If
  \var{value} is present, it is used as the starting value of the
  checksum; otherwise, a fixed default value is used.  This allows
  computing a running checksum over the concatenation of several
  input strings.  The algorithm is not cryptographically strong, and
  should not be used for authentication or digital signatures.  Since
  the algorithm is designed for use as a checksum algorithm, it is not
  suitable for use as a general hash algorithm.
\end{funcdesc}

\begin{funcdesc}{decompress}{string\optional{, wbits\optional{, bufsize}}}
  Decompresses the data in \var{string}, returning a string containing
  the uncompressed data.  The \var{wbits} parameter controls the size of
  the window buffer.  If \var{bufsize} is given, it is used as the
  initial size of the output buffer.  Raises the \exception{error}
  exception if any error occurs.

The absolute value of \var{wbits} is the base two logarithm of the
size of the history buffer (the ``window size'') used when compressing
data.  Its absolute value should be between 8 and 15 for the most
recent versions of the zlib library, larger values resulting in better
compression at the expense of greater memory usage.  The default value
is 15.  When \var{wbits} is negative, the standard
\program{gzip} header is suppressed; this is an undocumented feature
of the zlib library, used for compatibility with \program{unzip}'s
compression file format.

\var{bufsize} is the initial size of the buffer used to hold
decompressed data.  If more space is required, the buffer size will be
increased as needed, so you don't have to get this value exactly
right; tuning it will only save a few calls to \cfunction{malloc()}.  The
default size is 16384.
   
\end{funcdesc}

\begin{funcdesc}{decompressobj}{\optional{wbits}}
  Returns a decompression object, to be used for decompressing data
  streams that won't fit into memory at once.  The \var{wbits}
  parameter controls the size of the window buffer.
\end{funcdesc}

Compression objects support the following methods:

\begin{methoddesc}[Compress]{compress}{string}
Compress \var{string}, returning a string containing compressed data
for at least part of the data in \var{string}.  This data should be
concatenated to the output produced by any preceding calls to the
\method{compress()} method.  Some input may be kept in internal buffers
for later processing.
\end{methoddesc}

\begin{methoddesc}[Compress]{flush}{\optional{mode}}
All pending input is processed, and a string containing the remaining
compressed output is returned.  \var{mode} can be selected from the
constants \constant{Z_SYNC_FLUSH},  \constant{Z_FULL_FLUSH},  or 
\constant{Z_FINISH}, defaulting to \constant{Z_FINISH}.  \constant{Z_SYNC_FLUSH} and 
\constant{Z_FULL_FLUSH} allow compressing further strings of data and
are used to allow partial error recovery on decompression, while
\constant{Z_FINISH} finishes the compressed stream and 
prevents compressing any more data.  After calling
\method{flush()} with \var{mode} set to \constant{Z_FINISH}, the
\method{compress()} method cannot be called again; the only realistic
action is to delete the object.  
\end{methoddesc}

Decompression objects support the following methods, and two attributes:

\begin{memberdesc}{unused_data}
A string which contains any bytes past the end of the compressed data.
That is, this remains \code{""} until the last byte that contains
compression data is available.  If the whole string turned out to
contain compressed data, this is \code{""}, the empty string.

The only way to determine where a string of compressed data ends is by
actually decompressing it.  This means that when compressed data is
contained part of a larger file, you can only find the end of it by
reading data and feeding it followed by some non-empty string into a
decompression object's \method{decompress} method until the
\member{unused_data} attribute is no longer the empty string.
\end{memberdesc}

\begin{memberdesc}{unconsumed_tail}
A string that contains any data that was not consumed by the last
\method{decompress} call because it exceeded the limit for the
uncompressed data buffer.  This data has not yet been seen by the zlib
machinery, so you must feed it (possibly with further data
concatenated to it) back to a subsequent \method{decompress} method
call in order to get correct output.
\end{memberdesc}


\begin{methoddesc}[Decompress]{decompress}{string}{\optional{max_length}}
Decompress \var{string}, returning a string containing the
uncompressed data corresponding to at least part of the data in
\var{string}.  This data should be concatenated to the output produced
by any preceding calls to the
\method{decompress()} method.  Some of the input data may be preserved
in internal buffers for later processing.

If the optional parameter \var{max_length} is supplied then the return value
will be no longer than \var{max_length}. This may mean that not all of the
compressed input can be processed; and unconsumed data will be stored
in the attribute \member{unconsumed_tail}. This string must be passed
to a subsequent call to \method{decompress()} if decompression is to
continue.  If \var{max_length} is not supplied then the whole input is
decompressed, and \member{unconsumed_tail} is an empty string.
\end{methoddesc}

\begin{methoddesc}[Decompress]{flush}{}
All pending input is processed, and a string containing the remaining
uncompressed output is returned.  After calling \method{flush()}, the
\method{decompress()} method cannot be called again; the only realistic
action is to delete the object.
\end{methoddesc}

\begin{seealso}
  \seemodule{gzip}{Reading and writing \program{gzip}-format files.}
  \seeurl{http://www.gzip.org/zlib/}{The zlib library home page.}
\end{seealso}

\section{Standard Module \sectcode{gzip}}
\label{module-gzip}
\stmodindex{gzip}

The data compression provided by the \code{zlib} module is compatible
with that used by the GNU compression program \program{gzip}.
Accordingly, the \module{gzip} module provides the \class{GzipFile}
class to read and write \program{gzip}-format files, automatically
compressing or decompressing the data so it looks like an ordinary
file object.

\class{GzipFile} objects simulate most of the methods of a file
object, though it's not possible to use the \method{seek()} and
\method{tell()} methods to access the file randomly.


\begin{funcdesc}{open}{fileobj\optional{, filename\optional{, mode\optional{, compresslevel}}}}
  Returns a new \class{GzipFile} object on top of \var{fileobj}, which
  can be a regular file, a \class{StringIO} object, or any object which
  simulates a file.

  The \program{gzip} file format includes the original filename of the
  uncompressed file; when opening a \class{GzipFile} object for
  writing, it can be set by the \var{filename} argument.  The default
  value is an empty string.

  \var{mode} can be either \code{'r'} or \code{'w'} depending on
  whether the file will be read or written.  \var{compresslevel} is an
  integer from \code{1} to \code{9} controlling the level of
  compression; \code{1} is fastest and produces the least compression,
  and \code{9} is slowest and produces the most compression.  The
  default value of \var{compresslevel} is \code{9}.

  Calling a \class{GzipFile} object's \method{close()} method does not
  close \var{fileobj}, since you might wish to append more material
  after the compressed data.  This also allows you to pass a
  \class{StringIO} object opened for writing as \var{fileobj}, and
  retrieve the resulting memory buffer using the \class{StringIO}
  object's \method{getvalue()} method.
\end{funcdesc}

\begin{seealso}
\seemodule{zlib}{the basic data compression module}
\end{seealso}



\chapter{UNIX Specific Services}

The modules described in this chapter provide interfaces to features
that are unique to the \UNIX{} operating system, or in some cases to
some or many variants of it.  Here's an overview:

\begin{description}

\item[posix]
--- The most common Posix system calls (normally used via module \code{os}).

\item[posixpath]
--- Common Posix pathname manipulations (normally used via \code{os.path}).

\item[pwd]
--- The password database (\code{getpwnam()} and friends).

\item[grp]
--- The group database (\code{getgrnam()} and friends).

\item[crypt]
--- The \code{crypt()} function used to check Unix passwords.

\item[dbm]
--- The standard ``database'' interface, based on \code{ndbm}.

\item[gdbm]
--- GNU's reinterpretation of dbm.

\item[termios]
--- Posix style tty control.

\item[TERMIOS]
--- The symbolic constants required to use the \code{termios} module.

\item[fcntl]
--- The \code{fcntl()} and \code{ioctl()} system calls.

\item[posixfile]
--- A file-like object with support for locking.

\item[syslog]
--- An interface to the Unix \code{syslog} library routines.

\end{description}
			% UNIX Specific Services
\section{Built-in Module \sectcode{posix}}
\bimodindex{posix}

This module provides access to operating system functionality that is
standardized by the C Standard and the POSIX standard (a thinly disguised
\UNIX{} interface).

\strong{Do not import this module directly.}  Instead, import the
module \code{os}, which provides a \emph{portable} version of this
interface.  On \UNIX{}, the \code{os} module provides a superset of
the \code{posix} interface.  On non-\UNIX{} operating systems the
\code{posix} module is not available, but a subset is always available
through the \code{os} interface.  Once \code{os} is imported, there is
\emph{no} performance penalty in using it instead of
\code{posix}.
\stmodindex{os}

The descriptions below are very terse; refer to the
corresponding \UNIX{} manual entry for more information.  Arguments
called \var{path} refer to a pathname given as a string.

Errors are reported as exceptions; the usual exceptions are given
for type errors, while errors reported by the system calls raise
\code{posix.error}, described below.

Module \code{posix} defines the following data items:

\renewcommand{\indexsubitem}{(data in module posix)}
\begin{datadesc}{environ}
A dictionary representing the string environment at the time
the interpreter was started.
For example,
\code{posix.environ['HOME']}
is the pathname of your home directory, equivalent to
\code{getenv("HOME")}
in C.
Modifying this dictionary does not affect the string environment
passed on by \code{execv()}, \code{popen()} or \code{system()}; if you
need to change the environment, pass \code{environ} to \code{execve()}
or add variable assignments and export statements to the command
string for \code{system()} or \code{popen()}.%
\footnote{The problem with automatically passing on \code{environ} is
that there is no portable way of changing the environment.}
\end{datadesc}

\renewcommand{\indexsubitem}{(exception in module posix)}
\begin{excdesc}{error}
This exception is raised when a POSIX function returns a
POSIX-related error (e.g., not for illegal argument types).  Its
string value is \code{'posix.error'}.  The accompanying value is a
pair containing the numeric error code from \code{errno} and the
corresponding string, as would be printed by the C function
\code{perror()}.
\end{excdesc}

It defines the following functions and constants:

\renewcommand{\indexsubitem}{(in module posix)}
\begin{funcdesc}{chdir}{path}
Change the current working directory to \var{path}.
\end{funcdesc}

\begin{funcdesc}{chmod}{path\, mode}
Change the mode of \var{path} to the numeric \var{mode}.
\end{funcdesc}

\begin{funcdesc}{chown}{path\, uid, gid}
Change the owner and group id of \var{path} to the numeric \var{uid}
and \var{gid}.
(Not on MS-DOS.)
\end{funcdesc}

\begin{funcdesc}{close}{fd}
Close file descriptor \var{fd}.

Note: this function is intended for low-level I/O and must be applied
to a file descriptor as returned by \code{posix.open()} or
\code{posix.pipe()}.  To close a ``file object'' returned by the
built-in function \code{open} or by \code{posix.popen} or
\code{posix.fdopen}, use its \code{close()} method.
\end{funcdesc}

\begin{funcdesc}{dup}{fd}
Return a duplicate of file descriptor \var{fd}.
\end{funcdesc}

\begin{funcdesc}{dup2}{fd\, fd2}
Duplicate file descriptor \var{fd} to \var{fd2}, closing the latter
first if necessary.  Return \code{None}.
\end{funcdesc}

\begin{funcdesc}{execv}{path\, args}
Execute the executable \var{path} with argument list \var{args},
replacing the current process (i.e., the Python interpreter).
The argument list may be a tuple or list of strings.
(Not on MS-DOS.)
\end{funcdesc}

\begin{funcdesc}{execve}{path\, args\, env}
Execute the executable \var{path} with argument list \var{args},
and environment \var{env},
replacing the current process (i.e., the Python interpreter).
The argument list may be a tuple or list of strings.
The environment must be a dictionary mapping strings to strings.
(Not on MS-DOS.)
\end{funcdesc}

\begin{funcdesc}{_exit}{n}
Exit to the system with status \var{n}, without calling cleanup
handlers, flushing stdio buffers, etc.
(Not on MS-DOS.)

Note: the standard way to exit is \code{sys.exit(\var{n})}.
\code{posix._exit()} should normally only be used in the child process
after a \code{fork()}.
\end{funcdesc}

\begin{funcdesc}{fdopen}{fd\optional{\, mode\optional{\, bufsize}}}
Return an open file object connected to the file descriptor \var{fd}.
The \var{mode} and \var{bufsize} arguments have the same meaning as
the corresponding arguments to the built-in \code{open()} function.
\end{funcdesc}

\begin{funcdesc}{fork}{}
Fork a child process.  Return 0 in the child, the child's process id
in the parent.
(Not on MS-DOS.)
\end{funcdesc}

\begin{funcdesc}{fstat}{fd}
Return status for file descriptor \var{fd}, like \code{stat()}.
\end{funcdesc}

\begin{funcdesc}{getcwd}{}
Return a string representing the current working directory.
\end{funcdesc}

\begin{funcdesc}{getegid}{}
Return the current process's effective group id.
(Not on MS-DOS.)
\end{funcdesc}

\begin{funcdesc}{geteuid}{}
Return the current process's effective user id.
(Not on MS-DOS.)
\end{funcdesc}

\begin{funcdesc}{getgid}{}
Return the current process's group id.
(Not on MS-DOS.)
\end{funcdesc}

\begin{funcdesc}{getpgrp}{}
Return the current process group id.
(Not on MS-DOS.)
\end{funcdesc}

\begin{funcdesc}{getpid}{}
Return the current process id.
(Not on MS-DOS.)
\end{funcdesc}

\begin{funcdesc}{getppid}{}
Return the parent's process id.
(Not on MS-DOS.)
\end{funcdesc}

\begin{funcdesc}{getuid}{}
Return the current process's user id.
(Not on MS-DOS.)
\end{funcdesc}

\begin{funcdesc}{kill}{pid\, sig}
Kill the process \var{pid} with signal \var{sig}.
(Not on MS-DOS.)
\end{funcdesc}

\begin{funcdesc}{link}{src\, dst}
Create a hard link pointing to \var{src} named \var{dst}.
(Not on MS-DOS.)
\end{funcdesc}

\begin{funcdesc}{listdir}{path}
Return a list containing the names of the entries in the directory.
The list is in arbitrary order.  It does not include the special
entries \code{'.'} and \code{'..'} even if they are present in the
directory.
\end{funcdesc}

\begin{funcdesc}{lseek}{fd\, pos\, how}
Set the current position of file descriptor \var{fd} to position
\var{pos}, modified by \var{how}: 0 to set the position relative to
the beginning of the file; 1 to set it relative to the current
position; 2 to set it relative to the end of the file.
\end{funcdesc}

\begin{funcdesc}{lstat}{path}
Like \code{stat()}, but do not follow symbolic links.  (On systems
without symbolic links, this is identical to \code{posix.stat}.)
\end{funcdesc}

\begin{funcdesc}{mkfifo}{path\optional{\, mode}}
Create a FIFO (a POSIX named pipe) named \var{path} with numeric mode
\var{mode}.  The default \var{mode} is 0666 (octal).  The current
umask value is first masked out from the mode.
(Not on MS-DOS.)

FIFOs are pipes that can be accessed like regular files.  FIFOs exist
until they are deleted (for example with \code{os.unlink}).
Generally, FIFOs are used as rendez-vous between ``client'' and
``server'' type processes: the server opens the FIFO for reading, and
the client opens it for writing.  Note that \code{mkfifo()} doesn't
open the FIFO -- it just creates the rendez-vous point.
\end{funcdesc}

\begin{funcdesc}{mkdir}{path\optional{\, mode}}
Create a directory named \var{path} with numeric mode \var{mode}.
The default \var{mode} is 0777 (octal).  On some systems, \var{mode}
is ignored.  Where it is used, the current umask value is first
masked out.
\end{funcdesc}

\begin{funcdesc}{nice}{increment}
Add \var{incr} to the process' ``niceness''.  Return the new niceness.
(Not on MS-DOS.)
\end{funcdesc}

\begin{funcdesc}{open}{file\, flags\optional{\, mode}}
Open the file \var{file} and set various flags according to
\var{flags} and possibly its mode according to \var{mode}.
The default \var{mode} is 0777 (octal), and the current umask value is
first masked out.  Return the file descriptor for the newly opened
file.

Note: this function is intended for low-level I/O.  For normal usage,
use the built-in function \code{open}, which returns a ``file object''
with \code{read()} and  \code{write()} methods (and many more).
\end{funcdesc}

\begin{funcdesc}{pipe}{}
Create a pipe.  Return a pair of file descriptors \code{(r, w)}
usable for reading and writing, respectively.
(Not on MS-DOS.)
\end{funcdesc}

\begin{funcdesc}{plock}{op}
Lock program segments into memory.  The value of \var{op}
(defined in \code{<sys/lock.h>}) determines which segments are locked.
(Not on MS-DOS.)
\end{funcdesc}

\begin{funcdesc}{popen}{command\optional{\, mode\optional{\, bufsize}}}
Open a pipe to or from \var{command}.  The return value is an open
file object connected to the pipe, which can be read or written
depending on whether \var{mode} is \code{'r'} (default) or \code{'w'}.
The \var{bufsize} argument has the same meaning as the corresponding
argument to the built-in \code{open()} function.
(Not on MS-DOS.)
\end{funcdesc}

\begin{funcdesc}{read}{fd\, n}
Read at most \var{n} bytes from file descriptor \var{fd}.
Return a string containing the bytes read.

Note: this function is intended for low-level I/O and must be applied
to a file descriptor as returned by \code{posix.open()} or
\code{posix.pipe()}.  To read a ``file object'' returned by the
built-in function \code{open} or by \code{posix.popen} or
\code{posix.fdopen}, or \code{sys.stdin}, use its
\code{read()} or \code{readline()} methods.
\end{funcdesc}

\begin{funcdesc}{readlink}{path}
Return a string representing the path to which the symbolic link
points.  (On systems without symbolic links, this always raises
\code{posix.error}.)
\end{funcdesc}

\begin{funcdesc}{remove}{path}
Remove the file \var{path}.  See \code{rmdir} below to remove a directory.
\end{funcdesc}

\begin{funcdesc}{rename}{src\, dst}
Rename the file or directory \var{src} to \var{dst}.
\end{funcdesc}

\begin{funcdesc}{rmdir}{path}
Remove the directory \var{path}.
\end{funcdesc}

\begin{funcdesc}{setgid}{gid}
Set the current process's group id.
(Not on MS-DOS.)
\end{funcdesc}

\begin{funcdesc}{setpgrp}{}
Calls the system call \code{setpgrp()} or \code{setpgrp(0, 0)}
depending on which version is implemented (if any).  See the \UNIX{}
manual for the semantics.
(Not on MS-DOS.)
\end{funcdesc}

\begin{funcdesc}{setpgid}{pid\, pgrp}
Calls the system call \code{setpgid()}.  See the \UNIX{} manual for
the semantics.
(Not on MS-DOS.)
\end{funcdesc}

\begin{funcdesc}{setsid}{}
Calls the system call \code{setsid()}.  See the \UNIX{} manual for the
semantics.
(Not on MS-DOS.)
\end{funcdesc}

\begin{funcdesc}{setuid}{uid}
Set the current process's user id.
(Not on MS-DOS.)
\end{funcdesc}

\begin{funcdesc}{stat}{path}
Perform a {\em stat} system call on the given path.  The return value
is a tuple of at least 10 integers giving the most important (and
portable) members of the {\em stat} structure, in the order
\code{st_mode},
\code{st_ino},
\code{st_dev},
\code{st_nlink},
\code{st_uid},
\code{st_gid},
\code{st_size},
\code{st_atime},
\code{st_mtime},
\code{st_ctime}.
More items may be added at the end by some implementations.
(On MS-DOS, some items are filled with dummy values.)

Note: The standard module \code{stat} defines functions and constants
that are useful for extracting information from a stat structure.
\end{funcdesc}

\begin{funcdesc}{symlink}{src\, dst}
Create a symbolic link pointing to \var{src} named \var{dst}.  (On
systems without symbolic links, this always raises
\code{posix.error}.)
\end{funcdesc}

\begin{funcdesc}{system}{command}
Execute the command (a string) in a subshell.  This is implemented by
calling the Standard C function \code{system()}, and has the same
limitations.  Changes to \code{posix.environ}, \code{sys.stdin} etc.\ are
not reflected in the environment of the executed command.  The return
value is the exit status of the process as returned by Standard C
\code{system()}.
\end{funcdesc}

\begin{funcdesc}{tcgetpgrp}{fd}
Return the process group associated with the terminal given by
\var{fd} (an open file descriptor as returned by \code{posix.open()}).
(Not on MS-DOS.)
\end{funcdesc}

\begin{funcdesc}{tcsetpgrp}{fd\, pg}
Set the process group associated with the terminal given by
\var{fd} (an open file descriptor as returned by \code{posix.open()})
to \var{pg}.
(Not on MS-DOS.)
\end{funcdesc}

\begin{funcdesc}{times}{}
Return a 5-tuple of floating point numbers indicating accumulated (CPU
or other)
times, in seconds.  The items are: user time, system time, children's
user time, children's system time, and elapsed real time since a fixed
point in the past, in that order.  See the \UNIX{}
manual page {\it times}(2).  (Not on MS-DOS.)
\end{funcdesc}

\begin{funcdesc}{umask}{mask}
Set the current numeric umask and returns the previous umask.
(Not on MS-DOS.)
\end{funcdesc}

\begin{funcdesc}{uname}{}
Return a 5-tuple containing information identifying the current
operating system.  The tuple contains 5 strings:
\code{(\var{sysname}, \var{nodename}, \var{release}, \var{version}, \var{machine})}.
Some systems truncate the nodename to 8
characters or to the leading component; a better way to get the
hostname is \code{socket.gethostname()}.  (Not on MS-DOS, nor on older
\UNIX{} systems.)
\end{funcdesc}

\begin{funcdesc}{unlink}{path}
Remove the file \var{path}.  This is the same function as \code{remove};
the \code{unlink} name is its traditional \UNIX{} name.
\end{funcdesc}

\begin{funcdesc}{utime}{path\, \(atime\, mtime\)}
Set the access and modified time of the file to the given values.
(The second argument is a tuple of two items.)
\end{funcdesc}

\begin{funcdesc}{wait}{}
Wait for completion of a child process, and return a tuple containing
its pid and exit status indication (encoded as by \UNIX{}).
(Not on MS-DOS.)
\end{funcdesc}

\begin{funcdesc}{waitpid}{pid\, options}
Wait for completion of a child process given by proces id, and return
a tuple containing its pid and exit status indication (encoded as by
\UNIX{}).  The semantics of the call are affected by the value of
the integer options, which should be 0 for normal operation.  (If the
system does not support \code{waitpid()}, this always raises
\code{posix.error}.  Not on MS-DOS.)
\end{funcdesc}

\begin{funcdesc}{write}{fd\, str}
Write the string \var{str} to file descriptor \var{fd}.
Return the number of bytes actually written.

Note: this function is intended for low-level I/O and must be applied
to a file descriptor as returned by \code{posix.open()} or
\code{posix.pipe()}.  To write a ``file object'' returned by the
built-in function \code{open} or by \code{posix.popen} or
\code{posix.fdopen}, or \code{sys.stdout} or \code{sys.stderr}, use
its \code{write()} method.
\end{funcdesc}

\begin{datadesc}{WNOHANG}
The option for \code{waitpid()} to avoid hanging if no child process
status is available immediately.
\end{datadesc}

\section{Standard Module \sectcode{posixpath}}
\stmodindex{posixpath}

This module implements some useful functions on POSIX pathnames.

\strong{Do not import this module directly.}  Instead, import the
module \code{os} and use \code{os.path}.
\stmodindex{os}

\renewcommand{\indexsubitem}{(in module posixpath)}

\begin{funcdesc}{basename}{p}
Return the base name of pathname
\var{p}.
This is the second half of the pair returned by
\code{posixpath.split(\var{p})}.
\end{funcdesc}

\begin{funcdesc}{commonprefix}{list}
Return the longest string that is a prefix of all strings in
\var{list}.
If
\var{list}
is empty, return the empty string (\code{''}).
\end{funcdesc}

\begin{funcdesc}{exists}{p}
Return true if
\var{p}
refers to an existing path.
\end{funcdesc}

\begin{funcdesc}{expanduser}{p}
Return the argument with an initial component of \samp{\~} or
\samp{\~\var{user}} replaced by that \var{user}'s home directory.  An
initial \samp{\~{}} is replaced by the environment variable \code{\${}HOME};
an initial \samp{\~\var{user}} is looked up in the password directory through
the built-in module \code{pwd}.  If the expansion fails, or if the
path does not begin with a tilde, the path is returned unchanged.
\end{funcdesc}

\begin{funcdesc}{expandvars}{p}
Return the argument with environment variables expanded.  Substrings
of the form \samp{\$\var{name}} or \samp{\$\{\var{name}\}} are
replaced by the value of environment variable \var{name}.  Malformed
variable names and references to non-existing variables are left
unchanged.
\end{funcdesc}

\begin{funcdesc}{isabs}{p}
Return true if \var{p} is an absolute pathname (begins with a slash).
\end{funcdesc}

\begin{funcdesc}{isfile}{p}
Return true if \var{p} is an existing regular file.  This follows
symbolic links, so both \code{islink()} and \code{isfile()} can be true for the same
path.
\end{funcdesc}

\begin{funcdesc}{isdir}{p}
Return true if \var{p} is an existing directory.  This follows
symbolic links, so both \code{islink()} and \code{isdir()} can be true for the same
path.
\end{funcdesc}

\begin{funcdesc}{islink}{p}
Return true if
\var{p}
refers to a directory entry that is a symbolic link.
Always false if symbolic links are not supported.
\end{funcdesc}

\begin{funcdesc}{ismount}{p}
Return true if pathname \var{p} is a \dfn{mount point}: a point in a
file system where a different file system has been mounted.  The
function checks whether \var{p}'s parent, \file{\var{p}/..}, is on a
different device than \var{p}, or whether \file{\var{p}/..} and
\var{p} point to the same i-node on the same device --- this should
detect mount points for all \UNIX{} and POSIX variants.
\end{funcdesc}

\begin{funcdesc}{join}{p\, q}
Join the paths
\var{p}
and
\var{q} intelligently:
If
\var{q}
is an absolute path, the return value is
\var{q}.
Otherwise, the concatenation of
\var{p}
and
\var{q}
is returned, with a slash (\code{'/'}) inserted unless
\var{p}
is empty or ends in a slash.
\end{funcdesc}

\begin{funcdesc}{normcase}{p}
Normalize the case of a pathname.  This returns the path unchanged;
however, a similar function in \code{macpath} converts upper case to
lower case.
\end{funcdesc}

\begin{funcdesc}{samefile}{p\, q}
Return true if both pathname arguments refer to the same file or directory
(as indicated by device number and i-node number).
Raise an exception if a stat call on either pathname fails.
\end{funcdesc}

\begin{funcdesc}{split}{p}
Split the pathname \var{p} in a pair \code{(\var{head}, \var{tail})}, where
\var{tail} is the last pathname component and \var{head} is
everything leading up to that.  If \var{p} ends in a slash (except if
it is the root), the trailing slash is removed and the operation
applied to the result; otherwise, \code{join(\var{head}, \var{tail})} equals
\var{p}.  The \var{tail} part never contains a slash.  Some boundary
cases:\ if \var{p} is the root, \var{head} equals \var{p} and
\var{tail} is empty; if \var{p} is empty, both \var{head} and
\var{tail} are empty; if \var{p} contains no slash, \var{head} is
empty and \var{tail} equals \var{p}.
\end{funcdesc}

\begin{funcdesc}{splitext}{p}
Split the pathname \var{p} in a pair \code{(\var{root}, \var{ext})}
such that \code{\var{root} + \var{ext} == \var{p}},
the last component of \var{root} contains no periods,
and \var{ext} is empty or begins with a period.
\end{funcdesc}

\begin{funcdesc}{walk}{p\, visit\, arg}
Calls the function \var{visit} with arguments
\code{(\var{arg}, \var{dirname}, \var{names})} for each directory in the
directory tree rooted at \var{p} (including \var{p} itself, if it is a
directory).  The argument \var{dirname} specifies the visited directory,
the argument \var{names} lists the files in the directory (gotten from
\code{posix.listdir(\var{dirname})}, so including \samp{.} and
\samp{..}).  The \var{visit} function may modify \var{names} to
influence the set of directories visited below \var{dirname}, e.g., to
avoid visiting certain parts of the tree.  (The object referred to by
\var{names} must be modified in place, using \code{del} or slice
assignment.)
\end{funcdesc}
		% == posixpath
\section{\module{pwd} ---
         The password database.}
\declaremodule{builtin}{pwd}


\modulesynopsis{The password database (\function{getpwnam()} and friends).}

This module provides access to the \UNIX{} password database.
It is available on all \UNIX{} versions.

Password database entries are reported as 7-tuples containing the
following items from the password database (see \code{<pwd.h>}), in order:
\code{pw_name},
\code{pw_passwd},
\code{pw_uid},
\code{pw_gid},
\code{pw_gecos},
\code{pw_dir},
\code{pw_shell}.
The uid and gid items are integers, all others are strings.
A \code{KeyError} exception is raised if the entry asked for cannot be found.

It defines the following items:

\begin{funcdesc}{getpwuid}{uid}
Return the password database entry for the given numeric user ID.
\end{funcdesc}

\begin{funcdesc}{getpwnam}{name}
Return the password database entry for the given user name.
\end{funcdesc}

\begin{funcdesc}{getpwall}{}
Return a list of all available password database entries, in arbitrary order.
\end{funcdesc}

\section{\module{grp} ---
         The group database}

\declaremodule{builtin}{grp}
  \platform{Unix}
\modulesynopsis{The group database (\function{getgrnam()} and friends).}


This module provides access to the \UNIX{} group database.
It is available on all \UNIX{} versions.

Group database entries are reported as a tuple-like object, whose
attributes correspond to the members of the \code{group} structure
(Attribute field below, see \code{<pwd.h>}):

\begin{tableiii}{r|l|l}{textrm}{Index}{Attribute}{Meaning}
  \lineiii{0}{gr_name}{the name of the group}
  \lineiii{1}{gr_passwd}{the (encrypted) group password; often empty}
  \lineiii{2}{gr_gid}{the numerical group ID}
  \lineiii{3}{gr_mem}{all the group member's  user  names}
\end{tableiii}

The gid is an integer, name and password are strings, and the member
list is a list of strings.
(Note that most users are not explicitly listed as members of the
group they are in according to the password database.  Check both
databases to get complete membership information.)

It defines the following items:

\begin{funcdesc}{getgrgid}{gid}
Return the group database entry for the given numeric group ID.
\exception{KeyError} is raised if the entry asked for cannot be found.
\end{funcdesc}

\begin{funcdesc}{getgrnam}{name}
Return the group database entry for the given group name.
\exception{KeyError} is raised if the entry asked for cannot be found.
\end{funcdesc}

\begin{funcdesc}{getgrall}{}
Return a list of all available group entries, in arbitrary order.
\end{funcdesc}


\begin{seealso}
  \seemodule{pwd}{An interface to the user database, similar to this.}
\end{seealso}

\section{\module{crypt} ---
         Function to check \UNIX{} passwords}

\declaremodule{builtin}{crypt}
  \platform{Unix}
\modulesynopsis{The \cfunction{crypt()} function used to check
  \UNIX\ passwords.}
\moduleauthor{Steven D. Majewski}{sdm7g@virginia.edu}
\sectionauthor{Steven D. Majewski}{sdm7g@virginia.edu}
\sectionauthor{Peter Funk}{pf@artcom-gmbh.de}


This module implements an interface to the
\manpage{crypt}{3}\index{crypt(3)} routine, which is a one-way hash
function based upon a modified DES\indexii{cipher}{DES} algorithm; see
the \UNIX{} man page for further details.  Possible uses include
allowing Python scripts to accept typed passwords from the user, or
attempting to crack \UNIX{} passwords with a dictionary.

\begin{funcdesc}{crypt}{word, salt} 
  \var{word} will usually be a user's password as typed at a prompt or 
  in a graphical interface.  \var{salt} is usually a random
  two-character string which will be used to perturb the DES algorithm
  in one of 4096 ways.  The characters in \var{salt} must be in the
  set \regexp{[./a-zA-Z0-9]}.  Returns the hashed password as a
  string, which will be composed of characters from the same alphabet
   as the salt (the first two characters represent the salt itself).
\end{funcdesc}


A simple example illustrating typical use:

\begin{verbatim}
import crypt, getpass, pwd

def login():
    username = raw_input('Python login:')
    cryptedpasswd = pwd.getpwnam(username)[1]
    if cryptedpasswd:
        if cryptedpasswd == 'x' or cryptedpasswd == '*': 
            raise "Sorry, currently no support for shadow passwords"
        cleartext = getpass.getpass()
        return crypt.crypt(cleartext, cryptedpasswd[:2]) == cryptedpasswd
    else:
        return 1
\end{verbatim}

\section{\module{dbm} ---
         The standard ``database'' interface, based on ndbm.}
\declaremodule{builtin}{dbm}

\modulesynopsis{The standard ``database'' interface, based on ndbm.}


The \code{dbm} module provides an interface to the \UNIX{}
\code{(n)dbm} library.  Dbm objects behave like mappings
(dictionaries), except that keys and values are always strings.
Printing a dbm object doesn't print the keys and values, and the
\code{items()} and \code{values()} methods are not supported.

See also the \code{gdbm} module, which provides a similar interface
using the GNU GDBM library.
\refbimodindex{gdbm}

The module defines the following constant and functions:

\begin{excdesc}{error}
Raised on dbm-specific errors, such as I/O errors. \code{KeyError} is
raised for general mapping errors like specifying an incorrect key.
\end{excdesc}

\begin{funcdesc}{open}{filename, \optional{flag, \optional{mode}}}
Open a dbm database and return a dbm object.  The \var{filename}
argument is the name of the database file (without the \file{.dir} or
\file{.pag} extensions).

The optional \var{flag} argument can be
\code{'r'} (to open an existing database for reading only --- default),
\code{'w'} (to open an existing database for reading and writing),
\code{'c'} (which creates the database if it doesn't exist), or
\code{'n'} (which always creates a new empty database).

The optional \var{mode} argument is the \UNIX{} mode of the file, used
only when the database has to be created.  It defaults to octal
\code{0666}.
\end{funcdesc}

\section{Built-in Module \sectcode{gdbm}}
\bimodindex{gdbm}

This module is nearly identical to the \code{dbm} module, but uses
GDBM instead.  Its interface is identical, and not repeated here.

Warning: the file formats created by gdbm and dbm are incompatible.
\bimodindex{dbm}

\section{\module{termios} ---
         \POSIX{} style tty control}

\declaremodule{builtin}{termios}
  \platform{Unix}
\modulesynopsis{\POSIX\ style tty control.}

\indexii{\POSIX}{I/O control}
\indexii{tty}{I/O control}


This module provides an interface to the \POSIX{} calls for tty I/O
control.  For a complete description of these calls, see the \POSIX{} or
\UNIX{} manual pages.  It is only available for those \UNIX{} versions
that support \POSIX{} \emph{termios} style tty I/O control (and then
only if configured at installation time).

All functions in this module take a file descriptor \var{fd} as their
first argument.  This can be an integer file descriptor, such as
returned by \code{sys.stdin.fileno()}, or a file object, such as
\code{sys.stdin} itself.

This module also defines all the constants needed to work with the
functions provided here; these have the same name as their
counterparts in C.  Please refer to your system documentation for more
information on using these terminal control interfaces.

The module defines the following functions:

\begin{funcdesc}{tcgetattr}{fd}
Return a list containing the tty attributes for file descriptor
\var{fd}, as follows: \code{[}\var{iflag}, \var{oflag}, \var{cflag},
\var{lflag}, \var{ispeed}, \var{ospeed}, \var{cc}\code{]} where
\var{cc} is a list of the tty special characters (each a string of
length 1, except the items with indices \constant{VMIN} and
\constant{VTIME}, which are integers when these fields are
defined).  The interpretation of the flags and the speeds as well as
the indexing in the \var{cc} array must be done using the symbolic
constants defined in the \module{termios}
module.
\end{funcdesc}

\begin{funcdesc}{tcsetattr}{fd, when, attributes}
Set the tty attributes for file descriptor \var{fd} from the
\var{attributes}, which is a list like the one returned by
\function{tcgetattr()}.  The \var{when} argument determines when the
attributes are changed: \constant{TCSANOW} to change immediately,
\constant{TCSADRAIN} to change after transmitting all queued output,
or \constant{TCSAFLUSH} to change after transmitting all queued
output and discarding all queued input.
\end{funcdesc}

\begin{funcdesc}{tcsendbreak}{fd, duration}
Send a break on file descriptor \var{fd}.  A zero \var{duration} sends
a break for 0.25--0.5 seconds; a nonzero \var{duration} has a system
dependent meaning.
\end{funcdesc}

\begin{funcdesc}{tcdrain}{fd}
Wait until all output written to file descriptor \var{fd} has been
transmitted.
\end{funcdesc}

\begin{funcdesc}{tcflush}{fd, queue}
Discard queued data on file descriptor \var{fd}.  The \var{queue}
selector specifies which queue: \constant{TCIFLUSH} for the input
queue, \constant{TCOFLUSH} for the output queue, or
\constant{TCIOFLUSH} for both queues.
\end{funcdesc}

\begin{funcdesc}{tcflow}{fd, action}
Suspend or resume input or output on file descriptor \var{fd}.  The
\var{action} argument can be \constant{TCOOFF} to suspend output,
\constant{TCOON} to restart output, \constant{TCIOFF} to suspend
input, or \constant{TCION} to restart input.
\end{funcdesc}


\begin{seealso}
  \seemodule{tty}{Convenience functions for common terminal control
                  operations.}
\end{seealso}


\subsection{Example}
\nodename{termios Example}

Here's a function that prompts for a password with echoing turned
off.  Note the technique using a separate \function{tcgetattr()} call
and a \keyword{try} ... \keyword{finally} statement to ensure that the
old tty attributes are restored exactly no matter what happens:

\begin{verbatim}
def raw_input(prompt):
    import sys
    sys.stdout.write(prompt)
    sys.stdout.flush()
    return sys.stdin.readline()

def getpass(prompt = "Password: "):
    import termios, sys
    fd = sys.stdin.fileno()
    old = termios.tcgetattr(fd)
    new = termios.tcgetattr(fd)
    new[3] = new[3] & ~termios.ECHO          # lflags
    try:
        termios.tcsetattr(fd, termios.TCSADRAIN, new)
        passwd = raw_input(prompt)
    finally:
        termios.tcsetattr(fd, termios.TCSADRAIN, old)
    return passwd
\end{verbatim}

% Manual text by Jaap Vermeulen
\section{Built-in Module \sectcode{fcntl}}
\bimodindex{fcntl}
\indexii{\UNIX{}}{file control}
\indexii{\UNIX{}}{I/O control}

This module performs file control and I/O control on file descriptors.
It is an interface to the \dfn{fcntl()} and \dfn{ioctl()} \UNIX{} routines.
File descriptors can be obtained with the \dfn{fileno()} method of a
file or socket object.

The module defines the following functions:

\renewcommand{\indexsubitem}{(in module struct)}

\begin{funcdesc}{fcntl}{fd\, op\optional{\, arg}}
  Perform the requested operation on file descriptor \code{\var{fd}}.
  The operation is defined by \code{\var{op}} and is operating system
  dependent.  Typically these codes can be retrieved from the library
  module \code{FCNTL}. The argument \code{\var{arg}} is optional, and
  defaults to the integer value \code{0}.  When
  it is present, it can either be an integer value, or a string.  With
  the argument missing or an integer value, the return value of this
  function is the integer return value of the real \code{fcntl()}
  call.  When the argument is a string it represents a binary
  structure, e.g.\ created by \code{struct.pack()}. The binary data is
  copied to a buffer whose address is passed to the real \code{fcntl()}
  call.  The return value after a successful call is the contents of
  the buffer, converted to a string object.  In case the
  \code{fcntl()} fails, an \code{IOError} will be raised.
\end{funcdesc}

\begin{funcdesc}{ioctl}{fd\, op\, arg}
  This function is identical to the \code{fcntl()} function, except
  that the operations are typically defined in the library module
  \code{IOCTL}.
\end{funcdesc}

\begin{funcdesc}{flock}{fd\, op}
Perform the lock operation \var{op} on file descriptor \var{fd}.
See the Unix manual for details.  (On some systems, this function is
emulated using \code{fcntl}.)
\end{funcdesc}

\begin{funcdesc}{lockf}{fd\, code\, \optional{len\, \optional{start\, \optional{whence}}}}
This is a wrapper around the \code{F_SETLK} and \code{F_SETLKW}
\code{fcntl()} calls.  See the Unix manual for details.
\end{funcdesc}

If the library modules \code{FCNTL} or \code{IOCTL} are missing, you
can find the opcodes in the C include files \file{sys/fcntl.h} and
\file{sys/ioctl.h}. You can create the modules yourself with the h2py
script, found in the \file{Tools/scripts} directory.
\refstmodindex{FCNTL}
\refstmodindex{IOCTL}

Examples (all on a SVR4 compliant system):

\bcode\begin{verbatim}
import struct, FCNTL

file = open(...)
rv = fcntl(file.fileno(), FCNTL.O_NDELAY, 1)

lockdata = struct.pack('hhllhh', FCNTL.F_WRLCK, 0, 0, 0, 0, 0)
rv = fcntl(file.fileno(), FCNTL.F_SETLKW, lockdata)
\end{verbatim}\ecode
%
Note that in the first example the return value variable \code{rv} will
hold an integer value; in the second example it will hold a string
value.  The structure lay-out for the \var{lockadata} variable is
system dependent -- therefore using the \code{flock()} call may be
better.

% Manual text and implementation by Jaap Vermeulen
\section{Standard Module \module{posixfile}}
\label{module-posixfile}
\bimodindex{posixfile}
\indexii{\POSIX{}}{file object}

\emph{Note:} This module will become obsolete in a future release.
The locking operation that it provides is done better and more
portably by the \function{fcntl.lockf()} call.%
\withsubitem{(in module fcntl)}{\ttindex{lockf()}}

This module implements some additional functionality over the built-in
file objects.  In particular, it implements file locking, control over
the file flags, and an easy interface to duplicate the file object.
The module defines a new file object, the posixfile object.  It
has all the standard file object methods and adds the methods
described below.  This module only works for certain flavors of
\UNIX{}, since it uses \function{fcntl.fcntl()} for file locking.%
\withsubitem{(in module fcntl)}{\ttindex{fcntl()}}

To instantiate a posixfile object, use the \function{open()} function
in the \module{posixfile} module.  The resulting object looks and
feels roughly the same as a standard file object.

The \module{posixfile} module defines the following constants:


\begin{datadesc}{SEEK_SET}
Offset is calculated from the start of the file.
\end{datadesc}

\begin{datadesc}{SEEK_CUR}
Offset is calculated from the current position in the file.
\end{datadesc}

\begin{datadesc}{SEEK_END}
Offset is calculated from the end of the file.
\end{datadesc}

The \module{posixfile} module defines the following functions:


\begin{funcdesc}{open}{filename\optional{, mode\optional{, bufsize}}}
 Create a new posixfile object with the given filename and mode.  The
 \var{filename}, \var{mode} and \var{bufsize} arguments are
 interpreted the same way as by the built-in \function{open()}
 function.
\end{funcdesc}

\begin{funcdesc}{fileopen}{fileobject}
 Create a new posixfile object with the given standard file object.
 The resulting object has the same filename and mode as the original
 file object.
\end{funcdesc}

The posixfile object defines the following additional methods:

\setindexsubitem{(posixfile method)}
\begin{funcdesc}{lock}{fmt, \optional{len\optional{, start\optional{, whence}}}}
 Lock the specified section of the file that the file object is
 referring to.  The format is explained
 below in a table.  The \var{len} argument specifies the length of the
 section that should be locked. The default is \code{0}. \var{start}
 specifies the starting offset of the section, where the default is
 \code{0}.  The \var{whence} argument specifies where the offset is
 relative to. It accepts one of the constants \constant{SEEK_SET},
 \constant{SEEK_CUR} or \constant{SEEK_END}.  The default is
 \constant{SEEK_SET}.  For more information about the arguments refer
 to the \manpage{fcntl}{2} manual page on your system.
\end{funcdesc}

\begin{funcdesc}{flags}{\optional{flags}}
 Set the specified flags for the file that the file object is referring
 to.  The new flags are ORed with the old flags, unless specified
 otherwise.  The format is explained below in a table.  Without
 the \var{flags} argument
 a string indicating the current flags is returned (this is
 the same as the \samp{?} modifier).  For more information about the
 flags refer to the \manpage{fcntl}{2} manual page on your system.
\end{funcdesc}

\begin{funcdesc}{dup}{}
 Duplicate the file object and the underlying file pointer and file
 descriptor.  The resulting object behaves as if it were newly
 opened.
\end{funcdesc}

\begin{funcdesc}{dup2}{fd}
 Duplicate the file object and the underlying file pointer and file
 descriptor.  The new object will have the given file descriptor.
 Otherwise the resulting object behaves as if it were newly opened.
\end{funcdesc}

\begin{funcdesc}{file}{}
 Return the standard file object that the posixfile object is based
 on.  This is sometimes necessary for functions that insist on a
 standard file object.
\end{funcdesc}

All methods raise \exception{IOError} when the request fails.

Format characters for the \method{lock()} method have the following
meaning:

\begin{tableii}{|c|l|}{samp}{Format}{Meaning}
  \lineii{u}{unlock the specified region}
  \lineii{r}{request a read lock for the specified section}
  \lineii{w}{request a write lock for the specified section}
\end{tableii}

In addition the following modifiers can be added to the format:

\begin{tableiii}{|c|l|c|}{samp}{Modifier}{Meaning}{Notes}
  \lineiii{|}{wait until the lock has been granted}{}
  \lineiii{?}{return the first lock conflicting with the requested lock, or
              \code{None} if there is no conflict.}{(1)} 
\end{tableiii}

Note:

(1) The lock returned is in the format \code{(\var{mode}, \var{len},
\var{start}, \var{whence}, \var{pid})} where \var{mode} is a character
representing the type of lock ('r' or 'w').  This modifier prevents a
request from being granted; it is for query purposes only.

Format characters for the \method{flags()} method have the following
meanings:

\begin{tableii}{|c|l|}{samp}{Format}{Meaning}
  \lineii{a}{append only flag}
  \lineii{c}{close on exec flag}
  \lineii{n}{no delay flag (also called non-blocking flag)}
  \lineii{s}{synchronization flag}
\end{tableii}

In addition the following modifiers can be added to the format:

\begin{tableiii}{|c|l|c|}{samp}{Modifier}{Meaning}{Notes}
  \lineiii{!}{turn the specified flags 'off', instead of the default 'on'}{(1)}
  \lineiii{=}{replace the flags, instead of the default 'OR' operation}{(1)}
  \lineiii{?}{return a string in which the characters represent the flags that
  are set.}{(2)}
\end{tableiii}

Note:

(1) The \code{!} and \code{=} modifiers are mutually exclusive.

(2) This string represents the flags after they may have been altered
by the same call.

Examples:

\begin{verbatim}
import posixfile

file = posixfile.open('/tmp/test', 'w')
file.lock('w|')
...
file.lock('u')
file.close()
\end{verbatim}

\section{Built-in Module \module{resource}}
\label{module-resource}

\bimodindex{resource}
This module provides basic mechanisms for measuring and controlling
system resources utilized by a program.

Symbolic constants are used to specify particular system resources and
to request usage information about either the current process or its
children.

A single exception is defined for errors:


\begin{excdesc}{error}
  The functions described below may raise this error if the underlying
  system call failures unexpectedly.
\end{excdesc}

\subsection{Resource Limits}

Resources usage can be limited using the \function{setrlimit()} function
described below. Each resource is controlled by a pair of limits: a
soft limit and a hard limit. The soft limit is the current limit, and
may be lowered or raised by a process over time. The soft limit can
never exceed the hard limit. The hard limit can be lowered to any
value greater than the soft limit, but not raised. (Only processes with
the effective UID of the super-user can raise a hard limit.)

The specific resources that can be limited are system dependent. They
are described in the \manpage{getrlimit}{2} man page.  The resources
listed below are supported when the underlying operating system
supports them; resources which cannot be checked or controlled by the
operating system are not defined in this module for those platforms.

\begin{funcdesc}{getrlimit}{resource}
  Returns a tuple \code{(\var{soft}, \var{hard})} with the current
  soft and hard limits of \var{resource}. Raises \exception{ValueError} if
  an invalid resource is specified, or \exception{error} if the
  underyling system call fails unexpectedly.
\end{funcdesc}

\begin{funcdesc}{setrlimit}{resource, limits}
  Sets new limits of consumption of \var{resource}. The \var{limits}
  argument must be a tuple \code{(\var{soft}, \var{hard})} of two
  integers describing the new limits. A value of \code{-1} can be used to
  specify the maximum possible upper limit.

  Raises \exception{ValueError} if an invalid resource is specified,
  if the new soft limit exceeds the hard limit, or if a process tries
  to raise its hard limit (unless the process has an effective UID of
  super-user).  Can also raise \exception{error} if the underyling
  system call fails.
\end{funcdesc}

These symbols define resources whose consumption can be controlled
using the \function{setrlimit()} and \function{getrlimit()} functions
described below. The values of these symbols are exactly the constants
used by \C{} programs.

The \UNIX{} man page for \manpage{getrlimit}{2} lists the available
resources.  Note that not all systems use the same symbol or same
value to denote the same resource.

\begin{datadesc}{RLIMIT_CORE}
  The maximum size (in bytes) of a core file that the current process
  can create.  This may result in the creation of a partial core file
  if a larger core would be required to contain the entire process
  image.
\end{datadesc}

\begin{datadesc}{RLIMIT_CPU}
  The maximum amount of CPU time (in seconds) that a process can
  use. If this limit is exceeded, a \constant{SIGXCPU} signal is sent to
  the process. (See the \module{signal} module documentation for
  information about how to catch this signal and do something useful,
  e.g. flush open files to disk.)
\end{datadesc}

\begin{datadesc}{RLIMIT_FSIZE}
  The maximum size of a file which the process may create.  This only
  affects the stack of the main thread in a multi-threaded process.
\end{datadesc}

\begin{datadesc}{RLIMIT_DATA}
  The maximum size (in bytes) of the process's heap.
\end{datadesc}

\begin{datadesc}{RLIMIT_STACK}
  The maximum size (in bytes) of the call stack for the current
  process.
\end{datadesc}

\begin{datadesc}{RLIMIT_RSS}
  The maximum resident set size that should be made available to the
  process.
\end{datadesc}

\begin{datadesc}{RLIMIT_NPROC}
  The maximum number of processes the current process may create.
\end{datadesc}

\begin{datadesc}{RLIMIT_NOFILE}
  The maximum number of open file descriptors for the current
  process.
\end{datadesc}

\begin{datadesc}{RLIMIT_OFILE}
  The BSD name for \constant{RLIMIT_NOFILE}.
\end{datadesc}

\begin{datadesc}{RLIMIT_MEMLOC}
  The maximm address space which may be locked in memory.
\end{datadesc}

\begin{datadesc}{RLIMIT_VMEM}
  The largest area of mapped memory which the process may occupy.
\end{datadesc}

\begin{datadesc}{RLIMIT_AS}
  The maximum area (in bytes) of address space which may be taken by
  the process.
\end{datadesc}

\subsection{Resource Usage}

These functiona are used to retrieve resource usage information:

\begin{funcdesc}{getrusage}{who}
  This function returns a large tuple that describes the resources
  consumed by either the current process or its children, as specified
  by the \var{who} parameter.  The \var{who} parameter should be
  specified using one of the \code{RUSAGE_*} constants described
  below.

  The elements of the return value each
  describe how a particular system resource has been used, e.g. amount
  of time spent running is user mode or number of times the process was
  swapped out of main memory. Some values are dependent on the clock
  tick internal, e.g. the amount of memory the process is using.

  The first two elements of the return value are floating point values
  representing the amount of time spent executing in user mode and the
  amount of time spent executing in system mode, respectively. The
  remaining values are integers. Consult the \manpage{getrusage}{2}
  man page for detailed information about these values. A brief
  summary is presented here:

\begin{tableii}{|r|l|}{code}{Offset}{Resource}
  \lineii{0}{time in user mode (float)}
  \lineii{1}{time in system mode (float)}
  \lineii{2}{maximum resident set size}
  \lineii{3}{shared memory size}
  \lineii{4}{unshared memory size}
  \lineii{5}{unshared stack size}
  \lineii{6}{page faults not requiring I/O}
  \lineii{7}{page faults requiring I/O}
  \lineii{8}{number of swap outs}
  \lineii{9}{block input operations}
  \lineii{10}{block output operations}
  \lineii{11}{messages sent}
  \lineii{12}{messages received}
  \lineii{13}{signals received}
  \lineii{14}{voluntary context switches}
  \lineii{15}{involuntary context switches}
\end{tableii}

  This function will raise a \exception{ValueError} if an invalid
  \var{who} parameter is specified. It may also raise
  \exception{error} exception in unusual circumstances.
\end{funcdesc}

\begin{funcdesc}{getpagesize}{}
  Returns the number of bytes in a system page. (This need not be the
  same as the hardware page size.) This function is useful for
  determining the number of bytes of memory a process is using. The
  third element of the tuple returned by \function{getrusage()} describes
  memory usage in pages; multiplying by page size produces number of
  bytes. 
\end{funcdesc}

The following \code{RUSAGE_*} symbols are passed to the
\function{getrusage()} function to specify which processes information
should be provided for.

\begin{datadesc}{RUSAGE_SELF}
  \constant{RUSAGE_SELF} should be used to
  request information pertaining only to the process itself.
\end{datadesc}

\begin{datadesc}{RUSAGE_CHILDREN}
  Pass to \function{getrusage()} to request resource information for
  child processes of the calling process.
\end{datadesc}

\begin{datadesc}{RUSAGE_BOTH}
  Pass to \function{getrusage()} to request resources consumed by both
  the current process and child processes.  May not be available on all
  systems.
\end{datadesc}

\section{\module{syslog} ---
         \UNIX{} syslog library routines.}
\declaremodule{builtin}{syslog}

\modulesynopsis{An interface to the \UNIX{} syslog library routines.}


This module provides an interface to the \UNIX{} \code{syslog} library
routines.  Refer to the \UNIX{} manual pages for a detailed description
of the \code{syslog} facility.

The module defines the following functions:


\begin{funcdesc}{syslog}{\optional{priority,} message}
Send the string \var{message} to the system logger.  A trailing
newline is added if necessary.  Each message is tagged with a priority
composed of a \var{facility} and a \var{level}.  The optional
\var{priority} argument, which defaults to \constant{LOG_INFO},
determines the message priority.  If the facility is not encoded in
\var{priority} using logical-or (\code{LOG_INFO | LOG_USER}), the
value given in the \function{openlog()} call is used.
\end{funcdesc}

\begin{funcdesc}{openlog}{ident\optional{, logopt\optional{, facility}}}
Logging options other than the defaults can be set by explicitly
opening the log file with \function{openlog()} prior to calling
\function{syslog()}.  The defaults are (usually) \var{ident} =
\code{'syslog'}, \var{logopt} = \code{0}, \var{facility} =
\constant{LOG_USER}.  The \var{ident} argument is a string which is
prepended to every message.  The optional \var{logopt} argument is a
bit field - see below for possible values to combine.  The optional
\var{facility} argument sets the default facility for messages which
do not have a facility explicitly encoded.
\end{funcdesc}

\begin{funcdesc}{closelog}{}
Close the log file.
\end{funcdesc}

\begin{funcdesc}{setlogmask}{maskpri}
This function set the priority mask to \var{maskpri} and returns the
previous mask value.  Calls to \function{syslog()} with a priority
level not set in \var{maskpri} are ignored.  The default is to log all
priorities.  The function \code{LOG_MASK(\var{pri})} calculates the
mask for the individual priority \var{pri}.  The function
\code{LOG_UPTO(\var{pri})} calculates the mask for all priorities up
to and including \var{pri}.
\end{funcdesc}

The module defines the following constants:

\begin{description}

\item[Priority levels (high to low):]

\constant{LOG_EMERG}, \constant{LOG_ALERT}, \constant{LOG_CRIT},
\constant{LOG_ERR}, \constant{LOG_WARNING}, \constant{LOG_NOTICE},
\constant{LOG_INFO}, \constant{LOG_DEBUG}.

\item[Facilities:]

\constant{LOG_KERN}, \constant{LOG_USER}, \constant{LOG_MAIL},
\constant{LOG_DAEMON}, \constant{LOG_AUTH}, \constant{LOG_LPR},
\constant{LOG_NEWS}, \constant{LOG_UUCP}, \constant{LOG_CRON} and
\constant{LOG_LOCAL0} to \constant{LOG_LOCAL7}.

\item[Log options:]

\constant{LOG_PID}, \constant{LOG_CONS}, \constant{LOG_NDELAY},
\constant{LOG_NOWAIT} and \constant{LOG_PERROR} if defined in
\code{<syslog.h>}.

\end{description}

\section{\module{stat} ---
         Interpreting \function{stat()} results}

\declaremodule{standard}{stat}
  \platform{UNIX}
\modulesynopsis{Utilities for interpreting the results of
  \function{os.stat()}, \function{os.lstat()} and \function{os.fstat()}.}
\sectionauthor{Skip Montanaro}{skip@automatrix.com}


The \module{stat} module defines constants and functions for
interpreting the results of \function{os.stat()} and
\function{os.lstat()} (if they exist).  For complete details about the
\cfunction{stat()} and \cfunction{lstat()} system calls, consult your
local man pages.

The \module{stat} module defines the following functions:


\begin{funcdesc}{S_ISDIR}{mode}
Return non-zero if the mode was gotten from a directory.
\end{funcdesc}

\begin{funcdesc}{S_ISCHR}{mode}
Return non-zero if the mode was gotten from a character special device.
\end{funcdesc}

\begin{funcdesc}{S_ISBLK}{mode}
Return non-zero if the mode was gotten from a block special device.
\end{funcdesc}

\begin{funcdesc}{S_ISREG}{mode}
Return non-zero if the mode was gotten from a regular file.
\end{funcdesc}

\begin{funcdesc}{S_ISFIFO}{mode}
Return non-zero if the mode was gotten from a FIFO.
\end{funcdesc}

\begin{funcdesc}{S_ISLNK}{mode}
Return non-zero if the mode was gotten from a symbolic link.
\end{funcdesc}

\begin{funcdesc}{S_ISSOCK}{mode}
Return non-zero if the mode was gotten from a socket.
\end{funcdesc}

All the data items below are simply symbolic indexes into the 10-tuple
returned by \function{os.stat()} or \function{os.lstat()}.  

\begin{datadesc}{ST_MODE}
Inode protection mode.
\end{datadesc}

\begin{datadesc}{ST_INO}
Inode number.
\end{datadesc}

\begin{datadesc}{ST_DEV}
Device inode resides on.
\end{datadesc}

\begin{datadesc}{ST_NLINK}
Number of links to the inode.
\end{datadesc}

\begin{datadesc}{ST_UID}
User id of the owner.
\end{datadesc}

\begin{datadesc}{ST_GID}
Group id of the owner.
\end{datadesc}

\begin{datadesc}{ST_SIZE}
File size in bytes.
\end{datadesc}

\begin{datadesc}{ST_ATIME}
Time of last access.
\end{datadesc}

\begin{datadesc}{ST_MTIME}
Time of last modification.
\end{datadesc}

\begin{datadesc}{ST_CTIME}
Time of last status change (see manual pages for details).
\end{datadesc}

Example:

\begin{verbatim}
import os, sys
from stat import *

def process(dir, func):
    '''recursively descend the directory rooted at dir, calling func for
       each regular file'''

    for f in os.listdir(dir):
        mode = os.stat('%s/%s' % (dir, f))[ST_MODE]
        if S_ISDIR(mode):
            # recurse into directory
            process('%s/%s' % (dir, f), func)
        elif S_ISREG(mode):
            func('%s/%s' % (dir, f))
        else:
            print 'Skipping %s/%s' % (dir, f)

def f(file):
-Egon



    print 'frobbed', file

if __name__ == '__main__': process(sys.argv[1], f)
\end{verbatim}

-Egon



\section{\module{commands} ---
         Utility functions for external commands}

\declaremodule{standard}{commands}
  \platform{Unix}
\modulesynopsis{Utility functions for running external commands.}
\sectionauthor{Sue Williams}{sbw@provis.com}


The \module{commands} module contains wrapper functions for
\function{os.popen()} which take a system command as a string and
return any output generated by the command and, optionally, the exit
status.

The \module{commands} module defines the following functions:


\begin{funcdesc}{getstatusoutput}{cmd}
Execute the string \var{cmd} in a shell with \function{os.popen()} and
return a 2-tuple \code{(\var{status}, \var{output})}.  \var{cmd} is
actually run as \code{\{ \var{cmd} ; \} 2>\&1}, so that the returned
output will contain output or error messages. A trailing newline is
stripped from the output. The exit status for the command can be
interpreted according to the rules for the C function
\cfunction{wait()}.
\end{funcdesc}

\begin{funcdesc}{getoutput}{cmd}
Like \function{getstatusoutput()}, except the exit status is ignored
and the return value is a string containing the command's output.  
\end{funcdesc}

\begin{funcdesc}{getstatus}{file}
Return the output of \samp{ls -ld \var{file}} as a string.  This
function uses the \function{getoutput()} function, and properly
escapes backslashes and dollar signs in the argument.
\end{funcdesc}

Example:

\begin{verbatim}
>>> import commands
>>> commands.getstatusoutput('ls /bin/ls')
(0, '/bin/ls')
>>> commands.getstatusoutput('cat /bin/junk')
(256, 'cat: /bin/junk: No such file or directory')
>>> commands.getstatusoutput('/bin/junk')
(256, 'sh: /bin/junk: not found')
>>> commands.getoutput('ls /bin/ls')
'/bin/ls'
>>> commands.getstatus('/bin/ls')
'-rwxr-xr-x  1 root        13352 Oct 14  1994 /bin/ls'
\end{verbatim}


\chapter{The Python Debugger}
\stmodindex{pdb}
\index{debugging}

\setindexsubitem{(in module pdb)}

The module \code{pdb} defines an interactive source code debugger for
Python programs.  It supports setting
(conditional) breakpoints and single stepping
at the source line level, inspection of stack frames, source code
listing, and evaluation of arbitrary Python code in the context of any
stack frame.  It also supports post-mortem debugging and can be called
under program control.

The debugger is extensible --- it is actually defined as a class
\code{Pdb}.  This is currently undocumented but easily understood by
reading the source.  The extension interface uses the (also
undocumented) modules \code{bdb} and \code{cmd}.
\ttindex{Pdb}
\ttindex{bdb}
\ttindex{cmd}

A primitive windowing version of the debugger also exists --- this is
module \code{wdb}, which requires STDWIN (see the chapter on STDWIN
specific modules).
\index{stdwin}
\ttindex{wdb}

The debugger's prompt is ``\code{(Pdb) }''.
Typical usage to run a program under control of the debugger is:

\begin{verbatim}
>>> import pdb
>>> import mymodule
>>> pdb.run('mymodule.test()')
> <string>(0)?()
(Pdb) continue
> <string>(1)?()
(Pdb) continue
NameError: 'spam'
> <string>(1)?()
(Pdb) 
\end{verbatim}
%
\code{pdb.py} can also be invoked as
a script to debug other scripts.  For example:
\code{python /usr/local/lib/python1.4/pdb.py myscript.py}

Typical usage to inspect a crashed program is:

\begin{verbatim}
>>> import pdb
>>> import mymodule
>>> mymodule.test()
Traceback (innermost last):
  File "<stdin>", line 1, in ?
  File "./mymodule.py", line 4, in test
    test2()
  File "./mymodule.py", line 3, in test2
    print spam
NameError: spam
>>> pdb.pm()
> ./mymodule.py(3)test2()
-> print spam
(Pdb) 
\end{verbatim}
%
The module defines the following functions; each enters the debugger
in a slightly different way:

\begin{funcdesc}{run}{statement\optional{\, globals\optional{\, locals}}}
Execute the \var{statement} (given as a string) under debugger
control.  The debugger prompt appears before any code is executed; you
can set breakpoints and type \code{continue}, or you can step through
the statement using \code{step} or \code{next} (all these commands are
explained below).  The optional \var{globals} and \var{locals}
arguments specify the environment in which the code is executed; by
default the dictionary of the module \code{__main__} is used.  (See
the explanation of the \code{exec} statement or the \code{eval()}
built-in function.)
\end{funcdesc}

\begin{funcdesc}{runeval}{expression\optional{\, globals\optional{\, locals}}}
Evaluate the \var{expression} (given as a a string) under debugger
control.  When \code{runeval()} returns, it returns the value of the
expression.  Otherwise this function is similar to
\code{run()}.
\end{funcdesc}

\begin{funcdesc}{runcall}{function\optional{\, argument\, ...}}
Call the \var{function} (a function or method object, not a string)
with the given arguments.  When \code{runcall()} returns, it returns
whatever the function call returned.  The debugger prompt appears as
soon as the function is entered.
\end{funcdesc}

\begin{funcdesc}{set_trace}{}
Enter the debugger at the calling stack frame.  This is useful to
hard-code a breakpoint at a given point in a program, even if the code
is not otherwise being debugged (e.g. when an assertion fails).
\end{funcdesc}

\begin{funcdesc}{post_mortem}{traceback}
Enter post-mortem debugging of the given \var{traceback} object.
\end{funcdesc}

\begin{funcdesc}{pm}{}
Enter post-mortem debugging of the traceback found in
\code{sys.last_traceback}.
\end{funcdesc}

\section{Debugger Commands}

The debugger recognizes the following commands.  Most commands can be
abbreviated to one or two letters; e.g. ``\code{h(elp)}'' means that
either ``\code{h}'' or ``\code{help}'' can be used to enter the help
command (but not ``\code{he}'' or ``\code{hel}'', nor ``\code{H}'' or
``\code{Help} or ``\code{HELP}'').  Arguments to commands must be
separated by whitespace (spaces or tabs).  Optional arguments are
enclosed in square brackets (``\code{[]}'') in the command syntax; the
square brackets must not be typed.  Alternatives in the command syntax
are separated by a vertical bar (``\code{|}'').

Entering a blank line repeats the last command entered.  Exception: if
the last command was a ``\code{list}'' command, the next 11 lines are
listed.

Commands that the debugger doesn't recognize are assumed to be Python
statements and are executed in the context of the program being
debugged.  Python statements can also be prefixed with an exclamation
point (``\code{!}'').  This is a powerful way to inspect the program
being debugged; it is even possible to change a variable or call a
function.  When an
exception occurs in such a statement, the exception name is printed
but the debugger's state is not changed.

\begin{description}

\item[h(elp) \optional{\var{command}}]

Without argument, print the list of available commands.  With a
\var{command} as argument, print help about that command.  \samp{help
pdb} displays the full documentation file; if the environment variable
\code{PAGER} is defined, the file is piped through that command
instead.  Since the \var{command} argument must be an identifier,
\samp{help exec} must be entered to get help on the \samp{!} command.

\item[w(here)]

Print a stack trace, with the most recent frame at the bottom.  An
arrow indicates the current frame, which determines the context of
most commands.

\item[d(own)]

Move the current frame one level down in the stack trace
(to an older frame).

\item[u(p)]

Move the current frame one level up in the stack trace
(to a newer frame).

\item[b(reak) \optional{\var{lineno}{\Large\code{|}}\var{function}%
              \optional{, \code{'}\var{condition}\code{'}}}]

With a \var{lineno} argument, set a break there in the current
file.  With a \var{function} argument, set a break at the entry of
that function.  Without argument, list all breaks.
If a second argument is present, it is a string (included in string
quotes!) specifying an expression which must evaluate to true before
the breakpoint is honored.

\item[cl(ear) \optional{\var{lineno}}]

With a \var{lineno} argument, clear that break in the current file.
Without argument, clear all breaks (but first ask confirmation).

\item[s(tep)]

Execute the current line, stop at the first possible occasion
(either in a function that is called or on the next line in the
current function).

\item[n(ext)]

Continue execution until the next line in the current function
is reached or it returns.  (The difference between \code{next} and
\code{step} is that \code{step} stops inside a called function, while
\code{next} executes called functions at (nearly) full speed, only
stopping at the next line in the current function.)

\item[r(eturn)]

Continue execution until the current function returns.

\item[c(ont(inue))]

Continue execution, only stop when a breakpoint is encountered.

\item[l(ist) \optional{\var{first\optional{, last}}}]

List source code for the current file.  Without arguments, list 11
lines around the current line or continue the previous listing.  With
one argument, list 11 lines around at that line.  With two arguments,
list the given range; if the second argument is less than the first,
it is interpreted as a count.

\item[a(rgs)]

Print the argument list of the current function.

\item[p \var{expression}]

Evaluate the \var{expression} in the current context and print its
value.  (Note: \code{print} can also be used, but is not a debugger
command --- this executes the Python \code{print} statement.)

\item[\optional{!}\var{statement}]

Execute the (one-line) \var{statement} in the context of
the current stack frame.
The exclamation point can be omitted unless the first word
of the statement resembles a debugger command.
To set a global variable, you can prefix the assignment
command with a ``\code{global}'' command on the same line, e.g.:
\begin{verbatim}
(Pdb) global list_options; list_options = ['-l']
(Pdb)
\end{verbatim}
%
\item[q(uit)]

Quit from the debugger.
The program being executed is aborted.

\end{description}

\section{How It Works}

Some changes were made to the interpreter:

\begin{itemize}
\item \code{sys.settrace(\var{func})} sets the global trace function
\item there can also a local trace function (see later)
\end{itemize}

Trace functions have three arguments: (\var{frame}, \var{event}, \var{arg})

\begin{description}

\item[\var{frame}] is the current stack frame

\item[\var{event}] is a string: \code{'call'}, \code{'line'}, \code{'return'}
or \code{'exception'}

\item[\var{arg}] is dependent on the event type

\end{description}

The global trace function is invoked (with \var{event} set to
\code{'call'}) whenever a new local scope is entered; it should return
a reference to the local trace function to be used that scope, or
\code{None} if the scope shouldn't be traced.

The local trace function should return a reference to itself (or to
another function for further tracing in that scope), or \code{None} to
turn off tracing in that scope.

Instance methods are accepted (and very useful!) as trace functions.

The events have the following meaning:

\begin{description}

\item[\code{'call'}]
A function is called (or some other code block entered).  The global
trace function is called; arg is the argument list to the function;
the return value specifies the local trace function.

\item[\code{'line'}]
The interpreter is about to execute a new line of code (sometimes
multiple line events on one line exist).  The local trace function is
called; arg in None; the return value specifies the new local trace
function.

\item[\code{'return'}]
A function (or other code block) is about to return.  The local trace
function is called; arg is the value that will be returned.  The trace
function's return value is ignored.

\item[\code{'exception'}]
An exception has occurred.  The local trace function is called; arg is
a triple (exception, value, traceback); the return value specifies the
new local trace function

\end{description}

Note that as an exception is propagated down the chain of callers, an
\code{'exception'} event is generated at each level.

Stack frame objects have the following read-only attributes:

\begin{description}
\item[f_code]      the code object being executed
\item[f_lineno]    the current line number (\code{-1} for \code{'call'} events)
\item[f_back]      the stack frame of the caller, or None
\item[f_locals]    dictionary containing local name bindings
\item[f_globals]   dictionary containing global name bindings
\end{description}

Code objects have the following read-only attributes:

\begin{description}
\item[co_code]     the code string
\item[co_names]    the list of names used by the code
\item[co_consts]   the list of (literal) constants used by the code
\item[co_filename] the filename from which the code was compiled
\end{description}
			% The Python Debugger

\chapter{The Python Profiler}
\label{profile}

Copyright \copyright{} 1994, by InfoSeek Corporation, all rights reserved.

Written by James Roskind.%
\footnote{
Updated and converted to \LaTeX\ by Guido van Rossum.  The references to
the old profiler are left in the text, although it no longer exists.
}

Permission to use, copy, modify, and distribute this Python software
and its associated documentation for any purpose (subject to the
restriction in the following sentence) without fee is hereby granted,
provided that the above copyright notice appears in all copies, and
that both that copyright notice and this permission notice appear in
supporting documentation, and that the name of InfoSeek not be used in
advertising or publicity pertaining to distribution of the software
without specific, written prior permission.  This permission is
explicitly restricted to the copying and modification of the software
to remain in Python, compiled Python, or other languages (such as C)
wherein the modified or derived code is exclusively imported into a
Python module.

INFOSEEK CORPORATION DISCLAIMS ALL WARRANTIES WITH REGARD TO THIS
SOFTWARE, INCLUDING ALL IMPLIED WARRANTIES OF MERCHANTABILITY AND
FITNESS. IN NO EVENT SHALL INFOSEEK CORPORATION BE LIABLE FOR ANY
SPECIAL, INDIRECT OR CONSEQUENTIAL DAMAGES OR ANY DAMAGES WHATSOEVER
RESULTING FROM LOSS OF USE, DATA OR PROFITS, WHETHER IN AN ACTION OF
CONTRACT, NEGLIGENCE OR OTHER TORTIOUS ACTION, ARISING OUT OF OR IN
CONNECTION WITH THE USE OR PERFORMANCE OF THIS SOFTWARE.


The profiler was written after only programming in Python for 3 weeks.
As a result, it is probably clumsy code, but I don't know for sure yet
'cause I'm a beginner :-).  I did work hard to make the code run fast,
so that profiling would be a reasonable thing to do.  I tried not to
repeat code fragments, but I'm sure I did some stuff in really awkward
ways at times.  Please send suggestions for improvements to:
\email{jar@netscape.com}.  I won't promise \emph{any} support.  ...but
I'd appreciate the feedback.


\section{Introduction to the profiler}
\nodename{Profiler Introduction}

A \dfn{profiler} is a program that describes the run time performance
of a program, providing a variety of statistics.  This documentation
describes the profiler functionality provided in the modules
\module{profile} and \module{pstats}.  This profiler provides
\dfn{deterministic profiling} of any Python programs.  It also
provides a series of report generation tools to allow users to rapidly
examine the results of a profile operation.
\index{deterministic profiling}
\index{profiling, deterministic}


\section{How Is This Profiler Different From The Old Profiler?}
\nodename{Profiler Changes}

(This section is of historical importance only; the old profiler
discussed here was last seen in Python 1.1.)

The big changes from old profiling module are that you get more
information, and you pay less CPU time.  It's not a trade-off, it's a
trade-up.

To be specific:

\begin{description}

\item[Bugs removed:]
Local stack frame is no longer molested, execution time is now charged
to correct functions.

\item[Accuracy increased:]
Profiler execution time is no longer charged to user's code,
calibration for platform is supported, file reads are not done \emph{by}
profiler \emph{during} profiling (and charged to user's code!).

\item[Speed increased:]
Overhead CPU cost was reduced by more than a factor of two (perhaps a
factor of five), lightweight profiler module is all that must be
loaded, and the report generating module (\module{pstats}) is not needed
during profiling.

\item[Recursive functions support:]
Cumulative times in recursive functions are correctly calculated;
recursive entries are counted.

\item[Large growth in report generating UI:]
Distinct profiles runs can be added together forming a comprehensive
report; functions that import statistics take arbitrary lists of
files; sorting criteria is now based on keywords (instead of 4 integer
options); reports shows what functions were profiled as well as what
profile file was referenced; output format has been improved.

\end{description}


\section{Instant Users Manual}

This section is provided for users that ``don't want to read the
manual.'' It provides a very brief overview, and allows a user to
rapidly perform profiling on an existing application.

To profile an application with a main entry point of \samp{foo()}, you
would add the following to your module:

\begin{verbatim}
import profile
profile.run("foo()")
\end{verbatim}
%
The above action would cause \samp{foo()} to be run, and a series of
informative lines (the profile) to be printed.  The above approach is
most useful when working with the interpreter.  If you would like to
save the results of a profile into a file for later examination, you
can supply a file name as the second argument to the \function{run()}
function:

\begin{verbatim}
import profile
profile.run("foo()", 'fooprof')
\end{verbatim}
%
The file \file{profile.py} can also be invoked as
a script to profile another script.  For example:

\begin{verbatim}
python /usr/local/lib/python1.4/profile.py myscript.py
\end{verbatim}

When you wish to review the profile, you should use the methods in the
\module{pstats} module.  Typically you would load the statistics data as
follows:

\begin{verbatim}
import pstats
p = pstats.Stats('fooprof')
\end{verbatim}
%
The class \class{Stats} (the above code just created an instance of
this class) has a variety of methods for manipulating and printing the
data that was just read into \samp{p}.  When you ran
\function{profile.run()} above, what was printed was the result of three
method calls:

\begin{verbatim}
p.strip_dirs().sort_stats(-1).print_stats()
\end{verbatim}
%
The first method removed the extraneous path from all the module
names. The second method sorted all the entries according to the
standard module/line/name string that is printed (this is to comply
with the semantics of the old profiler).  The third method printed out
all the statistics.  You might try the following sort calls:

\begin{verbatim}
p.sort_stats('name')
p.print_stats()
\end{verbatim}
%
The first call will actually sort the list by function name, and the
second call will print out the statistics.  The following are some
interesting calls to experiment with:

\begin{verbatim}
p.sort_stats('cumulative').print_stats(10)
\end{verbatim}
%
This sorts the profile by cumulative time in a function, and then only
prints the ten most significant lines.  If you want to understand what
algorithms are taking time, the above line is what you would use.

If you were looking to see what functions were looping a lot, and
taking a lot of time, you would do:

\begin{verbatim}
p.sort_stats('time').print_stats(10)
\end{verbatim}
%
to sort according to time spent within each function, and then print
the statistics for the top ten functions.

You might also try:

\begin{verbatim}
p.sort_stats('file').print_stats('__init__')
\end{verbatim}
%
This will sort all the statistics by file name, and then print out
statistics for only the class init methods ('cause they are spelled
with \samp{__init__} in them).  As one final example, you could try:

\begin{verbatim}
p.sort_stats('time', 'cum').print_stats(.5, 'init')
\end{verbatim}
%
This line sorts statistics with a primary key of time, and a secondary
key of cumulative time, and then prints out some of the statistics.
To be specific, the list is first culled down to 50\% (re: \samp{.5})
of its original size, then only lines containing \code{init} are
maintained, and that sub-sub-list is printed.

If you wondered what functions called the above functions, you could
now (\samp{p} is still sorted according to the last criteria) do:

\begin{verbatim}
p.print_callers(.5, 'init')
\end{verbatim}

and you would get a list of callers for each of the listed functions. 

If you want more functionality, you're going to have to read the
manual, or guess what the following functions do:

\begin{verbatim}
p.print_callees()
p.add('fooprof')
\end{verbatim}
%
\section{What Is Deterministic Profiling?}
\nodename{Deterministic Profiling}

\dfn{Deterministic profiling} is meant to reflect the fact that all
\dfn{function call}, \dfn{function return}, and \dfn{exception} events
are monitored, and precise timings are made for the intervals between
these events (during which time the user's code is executing).  In
contrast, \dfn{statistical profiling} (which is not done by this
module) randomly samples the effective instruction pointer, and
deduces where time is being spent.  The latter technique traditionally
involves less overhead (as the code does not need to be instrumented),
but provides only relative indications of where time is being spent.

In Python, since there is an interpreter active during execution, the
presence of instrumented code is not required to do deterministic
profiling.  Python automatically provides a \dfn{hook} (optional
callback) for each event.  In addition, the interpreted nature of
Python tends to add so much overhead to execution, that deterministic
profiling tends to only add small processing overhead in typical
applications.  The result is that deterministic profiling is not that
expensive, yet provides extensive run time statistics about the
execution of a Python program.

Call count statistics can be used to identify bugs in code (surprising
counts), and to identify possible inline-expansion points (high call
counts).  Internal time statistics can be used to identify ``hot
loops'' that should be carefully optimized.  Cumulative time
statistics should be used to identify high level errors in the
selection of algorithms.  Note that the unusual handling of cumulative
times in this profiler allows statistics for recursive implementations
of algorithms to be directly compared to iterative implementations.


\section{Reference Manual}
\stmodindex{profile}
\label{module-profile}


The primary entry point for the profiler is the global function
\function{profile.run()}.  It is typically used to create any profile
information.  The reports are formatted and printed using methods of
the class \class{pstats.Stats}.  The following is a description of all
of these standard entry points and functions.  For a more in-depth
view of some of the code, consider reading the later section on
Profiler Extensions, which includes discussion of how to derive
``better'' profilers from the classes presented, or reading the source
code for these modules.

\begin{funcdesc}{run}{string\optional{, filename\optional{, ...}}}

This function takes a single argument that has can be passed to the
\keyword{exec} statement, and an optional file name.  In all cases this
routine attempts to \keyword{exec} its first argument, and gather profiling
statistics from the execution. If no file name is present, then this
function automatically prints a simple profiling report, sorted by the
standard name string (file/line/function-name) that is presented in
each line.  The following is a typical output from such a call:

\begin{verbatim}
      main()
      2706 function calls (2004 primitive calls) in 4.504 CPU seconds

Ordered by: standard name

ncalls  tottime  percall  cumtime  percall filename:lineno(function)
     2    0.006    0.003    0.953    0.477 pobject.py:75(save_objects)
  43/3    0.533    0.012    0.749    0.250 pobject.py:99(evaluate)
 ...
\end{verbatim}

The first line indicates that this profile was generated by the call:\\
\code{profile.run('main()')}, and hence the exec'ed string is
\code{'main()'}.  The second line indicates that 2706 calls were
monitored.  Of those calls, 2004 were \dfn{primitive}.  We define
\dfn{primitive} to mean that the call was not induced via recursion.
The next line: \code{Ordered by:\ standard name}, indicates that
the text string in the far right column was used to sort the output.
The column headings include:

\begin{description}

\item[ncalls ]
for the number of calls, 

\item[tottime ]
for the total time spent in the given function (and excluding time
made in calls to sub-functions),

\item[percall ]
is the quotient of \code{tottime} divided by \code{ncalls}

\item[cumtime ]
is the total time spent in this and all subfunctions (i.e., from
invocation till exit). This figure is accurate \emph{even} for recursive
functions.

\item[percall ]
is the quotient of \code{cumtime} divided by primitive calls

\item[filename:lineno(function) ]
provides the respective data of each function

\end{description}

When there are two numbers in the first column (e.g.: \samp{43/3}),
then the latter is the number of primitive calls, and the former is
the actual number of calls.  Note that when the function does not
recurse, these two values are the same, and only the single figure is
printed.

\end{funcdesc}

Analysis of the profiler data is done using this class from the
\module{pstats} module:

% now switch modules....
\stmodindex{pstats}

\begin{classdesc}{Stats}{filename\optional{, ...}}
This class constructor creates an instance of a ``statistics object''
from a \var{filename} (or set of filenames).  \class{Stats} objects are
manipulated by methods, in order to print useful reports.

The file selected by the above constructor must have been created by
the corresponding version of \module{profile}.  To be specific, there is
\emph{no} file compatibility guaranteed with future versions of this
profiler, and there is no compatibility with files produced by other
profilers (e.g., the old system profiler).

If several files are provided, all the statistics for identical
functions will be coalesced, so that an overall view of several
processes can be considered in a single report.  If additional files
need to be combined with data in an existing \class{Stats} object, the
\method{add()} method can be used.
\end{classdesc}


\subsection{The \sectcode{Stats} Class}

\setindexsubitem{(Stats method)}

\begin{methoddesc}{strip_dirs}{}
This method for the \class{Stats} class removes all leading path
information from file names.  It is very useful in reducing the size
of the printout to fit within (close to) 80 columns.  This method
modifies the object, and the stripped information is lost.  After
performing a strip operation, the object is considered to have its
entries in a ``random'' order, as it was just after object
initialization and loading.  If \method{strip_dirs()} causes two
function names to be indistinguishable (i.e., they are on the same
line of the same filename, and have the same function name), then the
statistics for these two entries are accumulated into a single entry.
\end{methoddesc}


\begin{methoddesc}{add}{filename\optional{, ...}}
This method of the \class{Stats} class accumulates additional
profiling information into the current profiling object.  Its
arguments should refer to filenames created by the corresponding
version of \function{profile.run()}.  Statistics for identically named
(re: file, line, name) functions are automatically accumulated into
single function statistics.
\end{methoddesc}

\begin{methoddesc}{sort_stats}{key\optional{, ...}}
This method modifies the \class{Stats} object by sorting it according
to the supplied criteria.  The argument is typically a string
identifying the basis of a sort (example: \code{"time"} or
\code{"name"}).

When more than one key is provided, then additional keys are used as
secondary criteria when the there is equality in all keys selected
before them.  For example, \samp{sort_stats('name', 'file')} will sort
all the entries according to their function name, and resolve all ties
(identical function names) by sorting by file name.

Abbreviations can be used for any key names, as long as the
abbreviation is unambiguous.  The following are the keys currently
defined: 

\begin{tableii}{|l|l|}{code}{Valid Arg}{Meaning}
\lineii{"calls"}{call count}
\lineii{"cumulative"}{cumulative time}
\lineii{"file"}{file name}
\lineii{"module"}{file name}
\lineii{"pcalls"}{primitive call count}
\lineii{"line"}{line number}
\lineii{"name"}{function name}
\lineii{"nfl"}{name/file/line}
\lineii{"stdname"}{standard name}
\lineii{"time"}{internal time}
\end{tableii}

Note that all sorts on statistics are in descending order (placing
most time consuming items first), where as name, file, and line number
searches are in ascending order (i.e., alphabetical). The subtle
distinction between \code{"nfl"} and \code{"stdname"} is that the
standard name is a sort of the name as printed, which means that the
embedded line numbers get compared in an odd way.  For example, lines
3, 20, and 40 would (if the file names were the same) appear in the
string order 20, 3 and 40.  In contrast, \code{"nfl"} does a numeric
compare of the line numbers.  In fact, \code{sort_stats("nfl")} is the
same as \code{sort_stats("name", "file", "line")}.

For compatibility with the old profiler, the numeric arguments
\samp{-1}, \samp{0}, \samp{1}, and \samp{2} are permitted.  They are
interpreted as \code{"stdname"}, \code{"calls"}, \code{"time"}, and
\code{"cumulative"} respectively.  If this old style format (numeric)
is used, only one sort key (the numeric key) will be used, and
additional arguments will be silently ignored.
\end{methoddesc}


\begin{methoddesc}{reverse_order}{}
This method for the \class{Stats} class reverses the ordering of the basic
list within the object.  This method is provided primarily for
compatibility with the old profiler.  Its utility is questionable
now that ascending vs descending order is properly selected based on
the sort key of choice.
\end{methoddesc}

\begin{methoddesc}{print_stats}{restriction\optional{, ...}}
This method for the \class{Stats} class prints out a report as described
in the \function{profile.run()} definition.

The order of the printing is based on the last \method{sort_stats()}
operation done on the object (subject to caveats in \method{add()} and
\method{strip_dirs()}.

The arguments provided (if any) can be used to limit the list down to
the significant entries.  Initially, the list is taken to be the
complete set of profiled functions.  Each restriction is either an
integer (to select a count of lines), or a decimal fraction between
0.0 and 1.0 inclusive (to select a percentage of lines), or a regular
expression (to pattern match the standard name that is printed; as of
Python 1.5b1, this uses the Perl-style regular expression syntax
defined by the \module{re} module).  If several restrictions are
provided, then they are applied sequentially.  For example:

\begin{verbatim}
print_stats(.1, "foo:")
\end{verbatim}

would first limit the printing to first 10\% of list, and then only
print functions that were part of filename \samp{.*foo:}.  In
contrast, the command:

\begin{verbatim}
print_stats("foo:", .1)
\end{verbatim}

would limit the list to all functions having file names \samp{.*foo:},
and then proceed to only print the first 10\% of them.
\end{methoddesc}


\begin{methoddesc}{print_callers}{restrictions\optional{, ...}}
This method for the \class{Stats} class prints a list of all functions
that called each function in the profiled database.  The ordering is
identical to that provided by \method{print_stats()}, and the definition
of the restricting argument is also identical.  For convenience, a
number is shown in parentheses after each caller to show how many
times this specific call was made.  A second non-parenthesized number
is the cumulative time spent in the function at the right.
\end{methoddesc}

\begin{methoddesc}{print_callees}{restrictions\optional{, ...}}
This method for the \class{Stats} class prints a list of all function
that were called by the indicated function.  Aside from this reversal
of direction of calls (re: called vs was called by), the arguments and
ordering are identical to the \method{print_callers()} method.
\end{methoddesc}

\begin{methoddesc}{ignore}{}
This method of the \class{Stats} class is used to dispose of the value
returned by earlier methods.  All standard methods in this class
return the instance that is being processed, so that the commands can
be strung together.  For example:

\begin{verbatim}
pstats.Stats('foofile').strip_dirs().sort_stats('cum') \
                       .print_stats().ignore()
\end{verbatim}

would perform all the indicated functions, but it would not return
the final reference to the \class{Stats} instance.%
\footnote{
This was once necessary, when Python would print any unused expression
result that was not \code{None}.  The method is still defined for
backward compatibility.
}
\end{methoddesc}


\section{Limitations}

There are two fundamental limitations on this profiler.  The first is
that it relies on the Python interpreter to dispatch \dfn{call},
\dfn{return}, and \dfn{exception} events.  Compiled \C{} code does not
get interpreted, and hence is ``invisible'' to the profiler.  All time
spent in \C{} code (including builtin functions) will be charged to the
Python function that invoked the \C{} code.  If the \C{} code calls out
to some native Python code, then those calls will be profiled
properly.

The second limitation has to do with accuracy of timing information.
There is a fundamental problem with deterministic profilers involving
accuracy.  The most obvious restriction is that the underlying ``clock''
is only ticking at a rate (typically) of about .001 seconds.  Hence no
measurements will be more accurate that that underlying clock.  If
enough measurements are taken, then the ``error'' will tend to average
out. Unfortunately, removing this first error induces a second source
of error...

The second problem is that it ``takes a while'' from when an event is
dispatched until the profiler's call to get the time actually
\emph{gets} the state of the clock.  Similarly, there is a certain lag
when exiting the profiler event handler from the time that the clock's
value was obtained (and then squirreled away), until the user's code
is once again executing.  As a result, functions that are called many
times, or call many functions, will typically accumulate this error.
The error that accumulates in this fashion is typically less than the
accuracy of the clock (i.e., less than one clock tick), but it
\emph{can} accumulate and become very significant.  This profiler
provides a means of calibrating itself for a given platform so that
this error can be probabilistically (i.e., on the average) removed.
After the profiler is calibrated, it will be more accurate (in a least
square sense), but it will sometimes produce negative numbers (when
call counts are exceptionally low, and the gods of probability work
against you :-). )  Do \emph{NOT} be alarmed by negative numbers in
the profile.  They should \emph{only} appear if you have calibrated
your profiler, and the results are actually better than without
calibration.


\section{Calibration}

The profiler class has a hard coded constant that is added to each
event handling time to compensate for the overhead of calling the time
function, and socking away the results.  The following procedure can
be used to obtain this constant for a given platform (see discussion
in section Limitations above).

\begin{verbatim}
import profile
pr = profile.Profile()
print pr.calibrate(100)
print pr.calibrate(100)
print pr.calibrate(100)
\end{verbatim}

The argument to \method{calibrate()} is the number of times to try to
do the sample calls to get the CPU times.  If your computer is
\emph{very} fast, you might have to do:

\begin{verbatim}
pr.calibrate(1000)
\end{verbatim}

or even:

\begin{verbatim}
pr.calibrate(10000)
\end{verbatim}

The object of this exercise is to get a fairly consistent result.
When you have a consistent answer, you are ready to use that number in
the source code.  For a Sun Sparcstation 1000 running Solaris 2.3, the
magical number is about .00053.  If you have a choice, you are better
off with a smaller constant, and your results will ``less often'' show
up as negative in profile statistics.

The following shows how the trace_dispatch() method in the Profile
class should be modified to install the calibration constant on a Sun
Sparcstation 1000:

\begin{verbatim}
def trace_dispatch(self, frame, event, arg):
    t = self.timer()
    t = t[0] + t[1] - self.t - .00053 # Calibration constant

    if self.dispatch[event](frame,t):
        t = self.timer()
        self.t = t[0] + t[1]
    else:
        r = self.timer()
        self.t = r[0] + r[1] - t # put back unrecorded delta
    return
\end{verbatim}

Note that if there is no calibration constant, then the line
containing the callibration constant should simply say:

\begin{verbatim}
t = t[0] + t[1] - self.t  # no calibration constant
\end{verbatim}

You can also achieve the same results using a derived class (and the
profiler will actually run equally fast!!), but the above method is
the simplest to use.  I could have made the profiler ``self
calibrating'', but it would have made the initialization of the
profiler class slower, and would have required some \emph{very} fancy
coding, or else the use of a variable where the constant \samp{.00053}
was placed in the code shown.  This is a \strong{VERY} critical
performance section, and there is no reason to use a variable lookup
at this point, when a constant can be used.


\section{Extensions --- Deriving Better Profilers}
\nodename{Profiler Extensions}

The \class{Profile} class of module \module{profile} was written so that
derived classes could be developed to extend the profiler.  Rather
than describing all the details of such an effort, I'll just present
the following two examples of derived classes that can be used to do
profiling.  If the reader is an avid Python programmer, then it should
be possible to use these as a model and create similar (and perchance
better) profile classes.

If all you want to do is change how the timer is called, or which
timer function is used, then the basic class has an option for that in
the constructor for the class.  Consider passing the name of a
function to call into the constructor:

\begin{verbatim}
pr = profile.Profile(your_time_func)
\end{verbatim}

The resulting profiler will call \code{your_time_func()} instead of
\function{os.times()}.  The function should return either a single number
or a list of numbers (like what \function{os.times()} returns).  If the
function returns a single time number, or the list of returned numbers
has length 2, then you will get an especially fast version of the
dispatch routine.

Be warned that you \emph{should} calibrate the profiler class for the
timer function that you choose.  For most machines, a timer that
returns a lone integer value will provide the best results in terms of
low overhead during profiling.  (\function{os.times()} is
\emph{pretty} bad, 'cause it returns a tuple of floating point values,
so all arithmetic is floating point in the profiler!).  If you want to
substitute a better timer in the cleanest fashion, you should derive a
class, and simply put in the replacement dispatch method that better
handles your timer call, along with the appropriate calibration
constant :-).


\subsection{OldProfile Class}

The following derived profiler simulates the old style profiler,
providing errant results on recursive functions. The reason for the
usefulness of this profiler is that it runs faster (i.e., less
overhead) than the old profiler.  It still creates all the caller
stats, and is quite useful when there is \emph{no} recursion in the
user's code.  It is also a lot more accurate than the old profiler, as
it does not charge all its overhead time to the user's code.

\begin{verbatim}
class OldProfile(Profile):

    def trace_dispatch_exception(self, frame, t):
        rt, rtt, rct, rfn, rframe, rcur = self.cur
        if rcur and not rframe is frame:
            return self.trace_dispatch_return(rframe, t)
        return 0

    def trace_dispatch_call(self, frame, t):
        fn = `frame.f_code`
        
        self.cur = (t, 0, 0, fn, frame, self.cur)
        if self.timings.has_key(fn):
            tt, ct, callers = self.timings[fn]
            self.timings[fn] = tt, ct, callers
        else:
            self.timings[fn] = 0, 0, {}
        return 1

    def trace_dispatch_return(self, frame, t):
        rt, rtt, rct, rfn, frame, rcur = self.cur
        rtt = rtt + t
        sft = rtt + rct

        pt, ptt, pct, pfn, pframe, pcur = rcur
        self.cur = pt, ptt+rt, pct+sft, pfn, pframe, pcur

        tt, ct, callers = self.timings[rfn]
        if callers.has_key(pfn):
            callers[pfn] = callers[pfn] + 1
        else:
            callers[pfn] = 1
        self.timings[rfn] = tt+rtt, ct + sft, callers

        return 1


    def snapshot_stats(self):
        self.stats = {}
        for func in self.timings.keys():
            tt, ct, callers = self.timings[func]
            nor_func = self.func_normalize(func)
            nor_callers = {}
            nc = 0
            for func_caller in callers.keys():
                nor_callers[self.func_normalize(func_caller)]=\
                      callers[func_caller]
                nc = nc + callers[func_caller]
            self.stats[nor_func] = nc, nc, tt, ct, nor_callers
\end{verbatim}

\subsection{HotProfile Class}

This profiler is the fastest derived profile example.  It does not
calculate caller-callee relationships, and does not calculate
cumulative time under a function.  It only calculates time spent in a
function, so it runs very quickly (re: very low overhead).  In truth,
the basic profiler is so fast, that is probably not worth the savings
to give up the data, but this class still provides a nice example.

\begin{verbatim}
class HotProfile(Profile):

    def trace_dispatch_exception(self, frame, t):
        rt, rtt, rfn, rframe, rcur = self.cur
        if rcur and not rframe is frame:
            return self.trace_dispatch_return(rframe, t)
        return 0

    def trace_dispatch_call(self, frame, t):
        self.cur = (t, 0, frame, self.cur)
        return 1

    def trace_dispatch_return(self, frame, t):
        rt, rtt, frame, rcur = self.cur

        rfn = `frame.f_code`

        pt, ptt, pframe, pcur = rcur
        self.cur = pt, ptt+rt, pframe, pcur

        if self.timings.has_key(rfn):
            nc, tt = self.timings[rfn]
            self.timings[rfn] = nc + 1, rt + rtt + tt
        else:
            self.timings[rfn] =      1, rt + rtt

        return 1


    def snapshot_stats(self):
        self.stats = {}
        for func in self.timings.keys():
            nc, tt = self.timings[func]
            nor_func = self.func_normalize(func)
            self.stats[nor_func] = nc, nc, tt, 0, {}
\end{verbatim}
		% The Python Profiler

\chapter{Internet and WWW Services}
\nodename{Internet and WWW}
\label{www}
\index{WWW}
\index{Internet}
\index{World-Wide Web}

The modules described in this chapter provide various services to
World-Wide Web (WWW) clients and/or services, and a few modules
related to news and email.  They are all implemented in Python.  Some
of these modules require the presence of the system-dependent module
\code{sockets}\refbimodindex{socket}, which is currently only fully
supported on \UNIX{} and Windows NT.  Here is an overview:

\begin{description}

\item[cgi]
--- Common Gateway Interface, used to interpret forms in server-side
scripts.

\item[urllib]
--- Open an arbitrary object given by URL (requires sockets).

\item[httplib]
--- HTTP protocol client (requires sockets).

\item[ftplib]
--- FTP protocol client (requires sockets).

\item[gopherlib]
--- Gopher protocol client (requires sockets).

\item[poplib]
--- POP3 protocol client (requires sockets).

\item[imaplib]
--- IMAP4 protocol client (requires sockets).

\item[nntplib]
--- NNTP protocol client (requires sockets).

\item[smtplib]
--- SMTP protocol client (requires sockets).

\item[urlparse]
--- Parse a URL string into a tuple (addressing scheme identifier, network
location, path, parameters, query string, fragment identifier).

\item[sgmllib]
--- Only as much of an SGML parser as needed to parse HTML.

\item[htmllib]
--- A parser for HTML documents.

\item[xmllib]
--- A parser for XML documents.

\item[formatter]
--- Generic output formatter and device interface.

\item[rfc822]
--- Parse \rfc{822} style mail headers.

\item[mimetools]
--- Tools for parsing MIME style message bodies.

\item[multifile]
--- Make each part of a MIME multipart message ``feel'' like a regular file.

\item[binhex]
--- Encode and decode files in binhex4 format.

\item[uu]
--- Encode and decode files in uuencode format.

\item[binascii]
--- Tools for converting between binary and various ascii-encoded binary 
representation

\item[xdrlib]
--- The External Data Representation Standard as described in \rfc{1014},
written by Sun Microsystems, Inc. June 1987.

\item[mailcap]
--- Mailcap file handling.  See \rfc{1524}.

\item[mimetypes]
--- Mapping of filename extensions to MIME types.

\item[base64]
--- Encode/decode binary files using the MIME base64 encoding.

\item[quopri]
--- Encode/decode binary files using the MIME quoted-printable encoding.

\item[SocketServer]
--- A framework for network servers.

\item[mailbox]
--- Read various mailbox formats.

\item[mimify]
--- Mimification and unmimification of mail messages.

\item[BaseHTTPServer]
--- Basic HTTP server (base class for SimpleHTTPServer and CGIHTTPServer).

\end{description}
			% Internet and WWW Services
\section{Standard Module \sectcode{cgi}}
\stmodindex{cgi}
\indexii{WWW}{server}
\indexii{CGI}{protocol}
\indexii{HTTP}{protocol}
\indexii{MIME}{headers}
\index{URL}

\renewcommand{\indexsubitem}{(in module cgi)}

This module makes it easy to write Python scripts that run in a WWW
server using the Common Gateway Interface.  It was written by Michael
McLay and subsequently modified by Steve Majewski and Guido van
Rossum.

When a WWW server finds that a URL contains a reference to a file in a
particular subdirectory (usually \code{/cgibin}), it runs the file as
a subprocess.  Information about the request such as the full URL, the
originating host etc., is passed to the subprocess in the shell
environment; additional input from the client may be read from
standard input.  Standard output from the subprocess is sent back
across the network to the client as the response from the request.
The CGI protocol describes what the environment variables passed to
the subprocess mean and how the output should be formatted.  The
official reference documentation for the CGI protocol can be found on
the World-Wide Web at
\code{<URL:http://hoohoo.ncsa.uiuc.edu/cgi/overview.html>}.  The
\code{cgi} module was based on version 1.1 of the protocol and should
also work with version 1.0.

The \code{cgi} module defines several classes that make it easy to
access the information passed to the subprocess from a Python script;
in particular, it knows how to parse the input sent by an HTML
``form'' using either a POST or a GET request (these are alternatives
for submitting forms in the HTTP protocol).

The formatting of the output is so trivial that no additional support
is needed.  All you need to do is print a minimal set of MIME headers
describing the output format, followed by a blank line and your actual
output.  E.g. if you want to generate HTML, your script could start as
follows:

\begin{verbatim}
# Header -- one or more lines:
print "Content-type: text/html"
# Blank line separating header from body:
print
# Body, in HTML format:
print "<TITLE>The Amazing SPAM Homepage!</TITLE>"
# etc...
\end{verbatim}

The server will add some header lines of its own, but it won't touch
the output following the header.

The \code{cgi} module defines the following functions:

\begin{funcdesc}{parse}{}
Read and parse the form submitted to the script and return a
dictionary containing the form's fields.  This should be called at
most once per script invocation, as it may consume standard input (if
the form was submitted through a POST request).  The keys in the
resulting dictionary are the field names used in the submission; the
values are {\em lists} of the field values (since field name may be
used multiple times in a single form).  \samp{\%} escapes in the
values are translated to their single-character equivalent using
\code{urllib.unquote()}.  As a side effect, this function sets
\code{environ['QUERY_STRING']} to the raw query string, if it isn't
already set.
\end{funcdesc}

\begin{funcdesc}{print_environ_usage}{}
Print a piece of HTML listing the environment variables that may be
set by the CGI protocol.
This is mainly useful when learning about writing CGI scripts.
\end{funcdesc}

\begin{funcdesc}{print_environ}{}
Print a piece of HTML text showing the entire contents of the shell
environment.  This is mainly useful when debugging a CGI script.
\end{funcdesc}

\begin{funcdesc}{print_form}{form}
Print a piece of HTML text showing the contents of the \var{form} (a
dictionary, an instance of the \code{FormContentDict} class defined
below, or a subclass thereof).
This is mainly useful when debugging a CGI script.
\end{funcdesc}

\begin{funcdesc}{escape}{string}
Convert special characters in \var{string} to HTML escapes.  In
particular, ``\code{\&}'' is replaced with ``\code{\&amp;}'',
``\code{<}'' is replaced with ``\code{\&lt;}'', and ``\code{>}'' is
replaced with ``\code{\&gt;}''.  This is useful when printing (almost)
arbitrary text in an HTML context.  Note that for inclusion in quoted
tag attributes (e.g. \code{<A HREF="...">}), some additional
characters would have to be converted --- in particular the string
quote.  There is currently no function that does this.
\end{funcdesc}

The module defines the following classes.  Since the base class
initializes itself by calling \code{parse()}, at most one instance of
at most one of these classes should be created per script invocation:

\begin{funcdesc}{FormContentDict}{}
This class behaves like a (read-only) dictionary and has the same keys
and values as the dictionary returned by \code{parse()} (i.e. each
field name maps to a list of values).  Additionally, it initializes
its data member \code{query_string} to the raw query sent from the
server.
\end{funcdesc}

\begin{funcdesc}{SvFormContentDict}{}
This class, derived from \code{FormContentDict}, is a little more
user-friendly when you are expecting that each field name is only used
once in the form.  When you access for a particular field (using
\code{form[fieldname]}), it will return the string value of that item
if it is unique, or raise \code{IndexError} if the field was specified
more than once in the form.  (If the field wasn't specified at all,
\code{KeyError} is raised.)  To access fields that are specified
multiple times, use \code{form.getlist(fieldname)}.  The
\code{values()} and \code{items()} methods return mixed lists ---
containing strings for singly-defined fields, and lists of strings for
multiply-defined fields.
\end{funcdesc}

(It currently defines some more classes, but these are experimental
and/or obsolescent, and are thus not documented --- see the source for
more informations.)

The module defines the following variable:

\begin{datadesc}{environ}
The shell environment, exactly as received from the http server.  See
the CGI documentation for a description of the various fields.
\end{datadesc}

\subsection{Example}

This example assumes that you have a WWW server up and running,
e.g.\ NCSA's \code{httpd}.

Place the following file in a convenient spot in the WWW server's
directory tree.  E.g., if you place it in the subdirectory \file{test}
of the root directory and call it \file{test.html}, its URL will be
\file{http://\var{yourservername}/test/test.html}.

\begin{verbatim}
<TITLE>Test Form Input</TITLE>
<H1>Test Form Input</H1>
<FORM METHOD="POST" ACTION="/cgi-bin/test.py">
<INPUT NAME=Name> (Name)<br>
<INPUT NAME=Address> (Address)<br>
<INPUT TYPE=SUBMIT>
</FORM>
\end{verbatim}

Selecting this file's URL from a forms-capable browser such as Mosaic
or Netscape will bring up a simple form with two text input fields and
a ``submit'' button.

But wait.  Before pressing ``submit'', a script that responds to the
form must also be installed.  The test file as shown assumes that the
script is called \file{test.py} and lives in the server's
\code{cgi-bin} directory.  Here's the test script:

\begin{verbatim}
#!/usr/local/bin/python

import cgi

print "Content-type: text/html"
print                                   # End of headers!
print "<TITLE>Test Form Output</TITLE>"
print "<H1>Test Form Output</H1>"

form = cgi.SvFormContentDict()          # Load the form

name = addr = None                      # Default: no name and address

# Extract name and address from the form, if given

if form.has_key('Name'):
        name = form['Name']
if form.has_key('Address'):
        addr = form['Address']
        
# Print an unnumbered list of the name and address, if present

print "<UL>"
if name is not None:
        print "<LI>Name:", cgi.escape(name)
if addr is not None:
        print "<LI>Address:", cgi.escape(addr)
print "</UL>"
\end{verbatim}

The script should be made executable (\samp{chmod +x \var{script}}).
If the Python interpreter is not located at
\file{/usr/local/bin/python} but somewhere else, the first line of the
script should be modified accordingly.

Now that everything is installed correctly, we can try out the form.
Bring up the test form in your WWW browser, fill in a name and address
in the form, and press the ``submit'' button.  The script should now
run and its output is sent back to your browser.  This should roughly
look as follows:

\strong{Test Form Output}

\begin{itemize}
\item Name: \var{the name you entered}
\item Address: \var{the address you entered}
\end{itemize}

If you didn't enter a name or address, the corresponding line will be
missing (since the browser doesn't send empty form fields to the
server).

\section{\module{urllib} ---
         Open an arbitrary resource by URL}

\declaremodule{standard}{urllib}
\modulesynopsis{Open an arbitrary network resource by URL (requires sockets).}

\index{WWW}
\index{World-Wide Web}
\index{URL}


This module provides a high-level interface for fetching data across
the World-Wide Web.  In particular, the \function{urlopen()} function
is similar to the built-in function \function{open()}, but accepts
Universal Resource Locators (URLs) instead of filenames.  Some
restrictions apply --- it can only open URLs for reading, and no seek
operations are available.

It defines the following public functions:

\begin{funcdesc}{urlopen}{url\optional{, data}}
Open a network object denoted by a URL for reading.  If the URL does
not have a scheme identifier, or if it has \file{file:} as its scheme
identifier, this opens a local file; otherwise it opens a socket to a
server somewhere on the network.  If the connection cannot be made, or
if the server returns an error code, the \exception{IOError} exception
is raised.  If all went well, a file-like object is returned.  This
supports the following methods: \method{read()}, \method{readline()},
\method{readlines()}, \method{fileno()}, \method{close()},
\method{info()} and \method{geturl()}.

Except for the \method{info()} and \method{geturl()} methods,
these methods have the same interface as for
file objects --- see section \ref{bltin-file-objects} in this
manual.  (It is not a built-in file object, however, so it can't be
used at those few places where a true built-in file object is
required.)

The \method{info()} method returns an instance of the class
\class{mimetools.Message} containing meta-information associated
with the URL.  When the method is HTTP, these headers are those
returned by the server at the head of the retrieved HTML page
(including Content-Length and Content-Type).  When the method is FTP,
a Content-Length header will be present if (as is now usual) the
server passed back a file length in response to the FTP retrieval
request.  When the method is local-file, returned headers will include
a Date representing the file's last-modified time, a Content-Length
giving file size, and a Content-Type containing a guess at the file's
type. See also the description of the
\refmodule{mimetools}\refstmodindex{mimetools} module.

The \method{geturl()} method returns the real URL of the page.  In
some cases, the HTTP server redirects a client to another URL.  The
\function{urlopen()} function handles this transparently, but in some
cases the caller needs to know which URL the client was redirected
to.  The \method{geturl()} method can be used to get at this
redirected URL.

If the \var{url} uses the \file{http:} scheme identifier, the optional
\var{data} argument may be given to specify a \code{POST} request
(normally the request type is \code{GET}).  The \var{data} argument
must in standard \file{application/x-www-form-urlencoded} format;
see the \function{urlencode()} function below.

The \function{urlopen()} function works transparently with proxies.
In a \UNIX{} or Windows environment, set the \envvar{http_proxy},
\envvar{ftp_proxy} or \envvar{gopher_proxy} environment variables to a
URL that identifies the proxy server before starting the Python
interpreter.  For example (the \character{\%} is the command prompt):

\begin{verbatim}
% http_proxy="http://www.someproxy.com:3128"
% export http_proxy
% python
...
\end{verbatim}

In a Macintosh environment, \function{urlopen()} will retrieve proxy
information from Internet\index{Internet Config} Config.

The \function{urlopen()} function works transparently with proxies.
In a \UNIX{} or Windows environment, set the \envvar{http_proxy},
\envvar{ftp_proxy} or \envvar{gopher_proxy} environment variables to a
URL that identifies the proxy server before starting the Python
interpreter, e.g.:

\begin{verbatim}
% http_proxy="http://www.someproxy.com:3128"
% export http_proxy
% python
...
\end{verbatim}

In a Macintosh environment, \function{urlopen()} will retrieve proxy
information from Internet Config.
\end{funcdesc}

\begin{funcdesc}{urlretrieve}{url\optional{, filename\optional{, hook}}}
Copy a network object denoted by a URL to a local file, if necessary.
If the URL points to a local file, or a valid cached copy of the
object exists, the object is not copied.  Return a tuple
\code{(\var{filename}, \var{headers})} where \var{filename} is the
local file name under which the object can be found, and \var{headers}
is either \code{None} (for a local object) or whatever the
\method{info()} method of the object returned by \function{urlopen()}
returned (for a remote object, possibly cached).  Exceptions are the
same as for \function{urlopen()}.

The second argument, if present, specifies the file location to copy
to (if absent, the location will be a tempfile with a generated name).
The third argument, if present, is a hook function that will be called
once on establishment of the network connection and once after each
block read thereafter.  The hook will be passed three arguments; a
count of blocks transferred so far, a block size in bytes, and the
total size of the file.  The third argument may be \code{-1} on older
FTP servers which do not return a file size in response to a retrieval 
request.
\end{funcdesc}

\begin{funcdesc}{urlcleanup}{}
Clear the cache that may have been built up by previous calls to
\function{urlretrieve()}.
\end{funcdesc}

\begin{funcdesc}{quote}{string\optional{, safe}}
Replace special characters in \var{string} using the \samp{\%xx} escape.
Letters, digits, and the characters \character{_,.-} are never quoted.
The optional \var{safe} parameter specifies additional characters
that should not be quoted --- its default value is \code{'/'}.

Example: \code{quote('/\~connolly/')} yields \code{'/\%7econnolly/'}.
\end{funcdesc}

\begin{funcdesc}{quote_plus}{string\optional{, safe}}
Like \function{quote()}, but also replaces spaces by plus signs, as
required for quoting HTML form values.  Plus signs in the original
string are escaped unless they are included in \var{safe}.
\end{funcdesc}

\begin{funcdesc}{unquote}{string}
Replace \samp{\%xx} escapes by their single-character equivalent.

Example: \code{unquote('/\%7Econnolly/')} yields \code{'/\~connolly/'}.
\end{funcdesc}

\begin{funcdesc}{unquote_plus}{string}
Like \function{unquote()}, but also replaces plus signs by spaces, as
required for unquoting HTML form values.
\end{funcdesc}

\begin{funcdesc}{urlencode}{dict}
Convert a dictionary to a ``url-encoded'' string, suitable to pass to
\function{urlopen()} above as the optional \var{data} argument.  This
is useful to pass a dictionary of form fields to a \code{POST}
request.  The resulting string is a series of
\code{\var{key}=\var{value}} pairs separated by \character{\&}
characters, where both \var{key} and \var{value} are quoted using
\function{quote_plus()} above.
\end{funcdesc}

The public functions \function{urlopen()} and \function{urlretrieve()}
create an instance of the \class{FancyURLopener} class and use it to perform
their requested actions.  To override this functionality, programmers can
create a subclass of \class{URLopener} or \class{FancyURLopener}, then
assign that class to the \var{urllib._urlopener} variable before calling the
desired function.  For example, applications may want to specify a different
\code{user-agent} header than \class{URLopener} defines.  This can be
accomplished with the following code:

\begin{verbatim}
class AppURLopener(urllib.FancyURLopener):
    def __init__(self, *args):
        apply(urllib.FancyURLopener.__init__, (self,) + args)
        self.version = "App/1.7"

urllib._urlopener = AppURLopener
\end{verbatim}

\begin{classdesc}{URLopener}{\optional{proxies\optional{, **x509}}}
Base class for opening and reading URLs.  Unless you need to support
opening objects using schemes other than \file{http:}, \file{ftp:},
\file{gopher:} or \file{file:}, you probably want to use
\class{FancyURLopener}.

By default, the \class{URLopener} class sends a
\code{user-agent} header of \samp{urllib/\var{VVV}}, where
\var{VVV} is the \module{urllib} version number.  Applications can
define their own \code{user-agent} header by subclassing
\class{URLopener} or \class{FancyURLopener} and setting the instance
attribute \var{version} to an appropriate string value before the
\method{open()} method is called.

Additional keyword parameters, collected in \var{x509}, are used for
authentication with the \file{https:} scheme.  The keywords
\var{key_file} and \var{cert_file} are supported; both are needed to
actually retrieve a resource at an \file{https:} URL.

\begin{methoddesc}{open}{fullurl\optional{, data}}
Open \var{fullurl} using the appropriate protocol.  This method sets 
up cache and proxy information, then calls the appropriate open method with
its input arguments.  If the scheme is not recognized,
\method{open_unknown()} is called.  The \var{data} argument 
has the same meaning as the \var{data} argument of \function{urlopen()}.
\end{methoddesc}

\begin{methoddesc}{open_unknown}{fullurl\optional{, data}}
Overridable interface to open unknown URL types.
\end{methoddesc}

\begin{methoddesc}{retrieve}{url\optional{, filename\optional{, reporthook}}}
Retrieves the contents of \var{url} and places it in \var{filename}.  The
return value is a tuple consisting of a local filename and either a
\class{mimetools.Message} object containing the response headers (for remote
URLs) or None (for local URLs).  The caller must then open and read the
contents of \var{filename}.  If \var{filename} is not given and the URL
refers to a local file, the input filename is returned.  If the URL is
non-local and \var{filename} is not given, the filename is the output of
\function{tempfile.mktemp()} with a suffix that matches the suffix of the last
path component of the input URL.  If \var{reporthook} is given, it must be
a function accepting three numeric parameters.  It will be called after each
chunk of data is read from the network.  \var{reporthook} is ignored for
local URLs.
\end{methoddesc}

\end{classdesc}

\begin{classdesc}{FancyURLopener}
\class{FancyURLopener} subclasses \class{URLopener} providing default handling 
for the following HTTP response codes: 301, 302 or 401.  For 301 and 302
response codes, the \code{location} header is used to fetch the actual URL.
For 401 response codes (authentication required), basic HTTP authentication
is performed.
\end{classdesc}

Restrictions:

\begin{itemize}

\item
Currently, only the following protocols are supported: HTTP, (versions
0.9 and 1.0), Gopher (but not Gopher-+), FTP, and local files.
\indexii{HTTP}{protocol}
\indexii{Gopher}{protocol}
\indexii{FTP}{protocol}

\item
The caching feature of \function{urlretrieve()} has been disabled
until I find the time to hack proper processing of Expiration time
headers.

\item
There should be a function to query whether a particular URL is in
the cache.

\item
For backward compatibility, if a URL appears to point to a local file
but the file can't be opened, the URL is re-interpreted using the FTP
protocol.  This can sometimes cause confusing error messages.

\item
The \function{urlopen()} and \function{urlretrieve()} functions can
cause arbitrarily long delays while waiting for a network connection
to be set up.  This means that it is difficult to build an interactive
web client using these functions without using threads.

\item
The data returned by \function{urlopen()} or \function{urlretrieve()}
is the raw data returned by the server.  This may be binary data
(e.g. an image), plain text or (for example) HTML\index{HTML}.  The
HTTP\indexii{HTTP}{protocol} protocol provides type information in the
reply header, which can be inspected by looking at the
\code{content-type} header.  For the Gopher\indexii{Gopher}{protocol}
protocol, type information is encoded in the URL; there is currently
no easy way to extract it.  If the returned data is HTML, you can use
the module \refmodule{htmllib}\refstmodindex{htmllib} to parse it.

\item
Although the \module{urllib} module contains (undocumented) routines
to parse and unparse URL strings, the recommended interface for URL
manipulation is in module \refmodule{urlparse}\refstmodindex{urlparse}.

\end{itemize}


\subsection{Examples}
\nodename{Urllib Examples}

Here is an example session that uses the \samp{GET} method to retrieve
a URL containing parameters:

\begin{verbatim}
>>> import urllib
>>> params = urllib.urlencode({'spam': 1, 'eggs': 2, 'bacon': 0})
>>> f = urllib.urlopen("http://www.musi-cal.com/cgi-bin/query?%s" % params)
>>> print f.read()
\end{verbatim}

The following example uses the \samp{POST} method instead:

\begin{verbatim}
>>> import urllib
>>> params = urllib.urlencode({'spam': 1, 'eggs': 2, 'bacon': 0})
>>> f = urllib.urlopen("http://www.musi-cal.com/cgi-bin/query", params)
>>> print f.read()
\end{verbatim}

\section{\module{httplib} ---
         HTTP protocol client}

\declaremodule{standard}{httplib}
\modulesynopsis{HTTP and HTTPS protocol client (requires sockets).}

\indexii{HTTP}{protocol}

This module defines classes which implement the client side of the
HTTP and HTTPS protocols.  It is normally not used directly --- the
module \refmodule{urllib}\refstmodindex{urllib} uses it to handle URLs
that use HTTP and HTTPS.  \note{HTTPS support is only
available if the \refmodule{socket} module was compiled with SSL
support.}

The module defines one class, \class{HTTP}:

\begin{classdesc}{HTTP}{\optional{host\optional{, port}}}
An \class{HTTP} instance
represents one transaction with an HTTP server.  It should be
instantiated passing it a host and optional port number.  If no port
number is passed, the port is extracted from the host string if it has
the form \code{\var{host}:\var{port}}, else the default HTTP port (80)
is used.  If no host is passed, no connection is made, and the
\method{connect()} method should be used to connect to a server.  For
example, the following calls all create instances that connect to the
server at the same host and port:

\begin{verbatim}
>>> h1 = httplib.HTTP('www.cwi.nl')
>>> h2 = httplib.HTTP('www.cwi.nl:80')
>>> h3 = httplib.HTTP('www.cwi.nl', 80)
\end{verbatim}

Once an \class{HTTP} instance has been connected to an HTTP server, it
should be used as follows:

\begin{enumerate}

\item Make exactly one call to the \method{putrequest()} method.

\item Make zero or more calls to the \method{putheader()} method.

\item Call the \method{endheaders()} method (this can be omitted if
step 4 makes no calls).

\item Optional calls to the \method{send()} method.

\item Call the \method{getreply()} method.

\item Call the \method{getfile()} method and read the data off the
file object that it returns.

\end{enumerate}
\end{classdesc}

\begin{datadesc}{HTTP_PORT}
  The default port for the HTTP protocol (always \code{80}).
\end{datadesc}

\begin{datadesc}{HTTPS_PORT}
  The default port for the HTTPS protocol (always \code{443}).
\end{datadesc}


\subsection{HTTP Objects \label{http-objects}}

\class{HTTP} instances have the following methods:


\begin{methoddesc}{set_debuglevel}{level}
Set the debugging level (the amount of debugging output printed).
The default debug level is \code{0}, meaning no debugging output is
printed.
\end{methoddesc}

\begin{methoddesc}{connect}{host\optional{, port}}
Connect to the server given by \var{host} and \var{port}.  See the
introduction to the \refmodule{httplib} module for information on the
default ports.  This should be called directly only if the instance
was instantiated without passing a host.
\end{methoddesc}

\begin{methoddesc}{send}{data}
Send data to the server.  This should be used directly only after the
\method{endheaders()} method has been called and before
\method{getreply()} has been called.
\end{methoddesc}

\begin{methoddesc}{putrequest}{request, selector}
This should be the first call after the connection to the server has
been made.  It sends a line to the server consisting of the
\var{request} string, the \var{selector} string, and the HTTP version
(\code{HTTP/1.0}).
\end{methoddesc}

\begin{methoddesc}{putheader}{header, argument\optional{, ...}}
Send an \rfc{822} style header to the server.  It sends a line to the
server consisting of the header, a colon and a space, and the first
argument.  If more arguments are given, continuation lines are sent,
each consisting of a tab and an argument.
\end{methoddesc}

\begin{methoddesc}{endheaders}{}
Send a blank line to the server, signalling the end of the headers.
\end{methoddesc}

\begin{methoddesc}{getreply}{}
Complete the request by shutting down the sending end of the socket,
read the reply from the server, and return a triple
\code{(\var{replycode}, \var{message}, \var{headers})}.  Here,
\var{replycode} is the integer reply code from the request (e.g.,
\code{200} if the request was handled properly); \var{message} is the
message string corresponding to the reply code; and \var{headers} is
an instance of the class \class{mimetools.Message} containing the
headers received from the server.  See the description of the
\refmodule{mimetools}\refstmodindex{mimetools} module.
\end{methoddesc}

\begin{methoddesc}{getfile}{}
Return a file object from which the data returned by the server can be
read, using the \method{read()}, \method{readline()} or
\method{readlines()} methods.
\end{methoddesc}


\subsection{Examples \label{httplib-examples}}

Here is an example session that uses the \samp{GET} method:

\begin{verbatim}
>>> import httplib
>>> h = httplib.HTTP('www.cwi.nl')
>>> h.putrequest('GET', '/index.html')
>>> h.putheader('Accept', 'text/html')
>>> h.putheader('Accept', 'text/plain')
>>> h.putheader('Host', 'www.cwi.nl')
>>> h.endheaders()
>>> errcode, errmsg, headers = h.getreply()
>>> print errcode # Should be 200
>>> f = h.getfile()
>>> data = f.read() # Get the raw HTML
>>> f.close()
\end{verbatim}

Here is an example session that shows how to \samp{POST} requests:

\begin{verbatim}
>>> import httplib, urllib
>>> params = urllib.urlencode({'spam': 1, 'eggs': 2, 'bacon': 0})
>>> h = httplib.HTTP("www.musi-cal.com:80")
>>> h.putrequest("POST", "/cgi-bin/query")
>>> h.putheader("Content-type", "application/x-www-form-urlencoded")
>>> h.putheader("Content-length", "%d" % len(params))
>>> h.putheader('Accept', 'text/plain')
>>> h.putheader('Host', 'www.musi-cal.com')
>>> h.endheaders()
>>> h.send(params)
>>> reply, msg, hdrs = h.getreply()
>>> print reply # should be 200
>>> data = h.getfile().read() # get the raw HTML
\end{verbatim}

\section{Standard Module \sectcode{ftplib}}
\label{module-ftplib}
\stmodindex{ftplib}
\indexii{FTP}{protocol}

\renewcommand{\indexsubitem}{(in module ftplib)}

This module defines the class \code{FTP} and a few related items.  The
\code{FTP} class implements the client side of the FTP protocol.  You
can use this to write Python programs that perform a variety of
automated FTP jobs, such as mirroring other ftp servers.  It is also
used by the module \code{urllib} to handle URLs that use FTP.  For
more information on FTP (File Transfer Protocol), see Internet \rfc{959}.

Here's a sample session using the \code{ftplib} module:

\bcode\begin{verbatim}
>>> from ftplib import FTP
>>> ftp = FTP('ftp.cwi.nl')   # connect to host, default port
>>> ftp.login()               # user anonymous, passwd user@hostname
>>> ftp.retrlines('LIST')     # list directory contents
total 24418
drwxrwsr-x   5 ftp-usr  pdmaint     1536 Mar 20 09:48 .
dr-xr-srwt 105 ftp-usr  pdmaint     1536 Mar 21 14:32 ..
-rw-r--r--   1 ftp-usr  pdmaint     5305 Mar 20 09:48 INDEX
 .
 .
 .
>>> ftp.quit()
\end{verbatim}\ecode
%
The module defines the following items:

\begin{funcdesc}{FTP}{\optional{host\optional{\, user\, passwd\, acct}}}
Return a new instance of the \code{FTP} class.  When
\var{host} is given, the method call \code{connect(\var{host})} is
made.  When \var{user} is given, additionally the method call
\code{login(\var{user}, \var{passwd}, \var{acct})} is made (where
\var{passwd} and \var{acct} default to the empty string when not given).
\end{funcdesc}

\begin{datadesc}{all_errors}
The set of all exceptions (as a tuple) that methods of \code{FTP}
instances may raise as a result of problems with the FTP connection
(as opposed to programming errors made by the caller).  This set
includes the four exceptions listed below as well as
\code{socket.error} and \code{IOError}.
\end{datadesc}

\begin{excdesc}{error_reply}
Exception raised when an unexpected reply is received from the server.
\end{excdesc}

\begin{excdesc}{error_temp}
Exception raised when an error code in the range 400--499 is received.
\end{excdesc}

\begin{excdesc}{error_perm}
Exception raised when an error code in the range 500--599 is received.
\end{excdesc}

\begin{excdesc}{error_proto}
Exception raised when a reply is received from the server that does
not begin with a digit in the range 1--5.
\end{excdesc}

\subsection{FTP Objects}

FTP instances have the following methods:

\renewcommand{\indexsubitem}{(FTP object method)}

\begin{funcdesc}{set_debuglevel}{level}
Set the instance's debugging level.  This controls the amount of
debugging output printed.  The default, 0, produces no debugging
output.  A value of 1 produces a moderate amount of debugging output,
generally a single line per request.  A value of 2 or higher produces
the maximum amount of debugging output, logging each line sent and
received on the control connection.
\end{funcdesc}

\begin{funcdesc}{connect}{host\optional{\, port}}
Connect to the given host and port.  The default port number is 21, as
specified by the FTP protocol specification.  It is rarely needed to
specify a different port number.  This function should be called only
once for each instance; it should not be called at all if a host was
given when the instance was created.  All other methods can only be
used after a connection has been made.
\end{funcdesc}

\begin{funcdesc}{getwelcome}{}
Return the welcome message sent by the server in reply to the initial
connection.  (This message sometimes contains disclaimers or help
information that may be relevant to the user.)
\end{funcdesc}

\begin{funcdesc}{login}{\optional{user\optional{\, passwd\optional{\, acct}}}}
Log in as the given \var{user}.  The \var{passwd} and \var{acct}
parameters are optional and default to the empty string.  If no
\var{user} is specified, it defaults to \samp{anonymous}.  If
\var{user} is \code{anonymous}, the default \var{passwd} is
\samp{\var{realuser}@\var{host}} where \var{realuser} is the real user
name (glanced from the \samp{LOGNAME} or \samp{USER} environment
variable) and \var{host} is the hostname as returned by
\code{socket.gethostname()}.  This function should be called only
once for each instance, after a connection has been established; it
should not be called at all if a host and user were given when the
instance was created.  Most FTP commands are only allowed after the
client has logged in.
\end{funcdesc}

\begin{funcdesc}{abort}{}
Abort a file transfer that is in progress.  Using this does not always
work, but it's worth a try.
\end{funcdesc}

\begin{funcdesc}{sendcmd}{command}
Send a simple command string to the server and return the response
string.
\end{funcdesc}

\begin{funcdesc}{voidcmd}{command}
Send a simple command string to the server and handle the response.
Return nothing if a response code in the range 200--299 is received.
Raise an exception otherwise.
\end{funcdesc}

\begin{funcdesc}{retrbinary}{command\, callback\optional{\, maxblocksize}}
Retrieve a file in binary transfer mode.  \var{command} should be an
appropriate \samp{RETR} command, i.e.\ \code{"RETR \var{filename}"}.
The \var{callback} function is called for each block of data received,
with a single string argument giving the data block.
The optional \var{maxblocksize} argument specifies the maximum chunk size to
read on the low-level socket object created to do the actual transfer
(which will also be the largest size of the data blocks passed to
\var{callback}).  A reasonable default is chosen.
\end{funcdesc}

\begin{funcdesc}{retrlines}{command\optional{\, callback}}
Retrieve a file or directory listing in \ASCII{} transfer mode.
\var{command} should be an appropriate \samp{RETR} command (see
\code{retrbinary()} or a \samp{LIST} command (usually just the string
\code{"LIST"}).  The \var{callback} function is called for each line,
with the trailing CRLF stripped.  The default \var{callback} prints
the line to \code{sys.stdout}.
\end{funcdesc}

\begin{funcdesc}{storbinary}{command\, file\, blocksize}
Store a file in binary transfer mode.  \var{command} should be an
appropriate \samp{STOR} command, i.e.\ \code{"STOR \var{filename}"}.
\var{file} is an open file object which is read until EOF using its
\code{read()} method in blocks of size \var{blocksize} to provide the
data to be stored.
\end{funcdesc}

\begin{funcdesc}{storlines}{command\, file}
Store a file in \ASCII{} transfer mode.  \var{command} should be an
appropriate \samp{STOR} command (see \code{storbinary()}).  Lines are
read until EOF from the open file object \var{file} using its
\code{readline()} method to privide the data to be stored.
\end{funcdesc}

\begin{funcdesc}{nlst}{argument\optional{\, \ldots}}
Return a list of files as returned by the \samp{NLST} command.  The
optional \var{argument} is a directory to list (default is the current
server directory).  Multiple arguments can be used to pass
non-standard options to the \samp{NLST} command.
\end{funcdesc}

\begin{funcdesc}{dir}{argument\optional{\, \ldots}}
Return a directory listing as returned by the \samp{LIST} command, as
a list of lines.  The optional \var{argument} is a directory to list
(default is the current server directory).  Multiple arguments can be
used to pass non-standard options to the \samp{LIST} command.  If the
last argument is a function, it is used as a \var{callback} function
as for \code{retrlines()}.
\end{funcdesc}

\begin{funcdesc}{rename}{fromname\, toname}
Rename file \var{fromname} on the server to \var{toname}.
\end{funcdesc}

\begin{funcdesc}{cwd}{pathname}
Set the current directory on the server.
\end{funcdesc}

\begin{funcdesc}{mkd}{pathname}
Create a new directory on the server.
\end{funcdesc}

\begin{funcdesc}{pwd}{}
Return the pathname of the current directory on the server.
\end{funcdesc}

\begin{funcdesc}{quit}{}
Send a \samp{QUIT} command to the server and close the connection.
This is the ``polite'' way to close a connection, but it may raise an
exception of the server reponds with an error to the \code{QUIT}
command.
\end{funcdesc}

\begin{funcdesc}{close}{}
Close the connection unilaterally.  This should not be applied to an
already closed connection (e.g.\ after a successful call to
\code{quit()}.
\end{funcdesc}

\section{\module{gopherlib} ---
         Gopher protocol client}

\declaremodule{standard}{gopherlib}
\modulesynopsis{Gopher protocol client (requires sockets).}

\deprecated{2.5}{The \code{gopher} protocol is not in active use
                 anymore.}

\indexii{Gopher}{protocol}

This module provides a minimal implementation of client side of the
Gopher protocol.  It is used by the module \refmodule{urllib} to
handle URLs that use the Gopher protocol.

The module defines the following functions:

\begin{funcdesc}{send_selector}{selector, host\optional{, port}}
Send a \var{selector} string to the gopher server at \var{host} and
\var{port} (default \code{70}).  Returns an open file object from
which the returned document can be read.
\end{funcdesc}

\begin{funcdesc}{send_query}{selector, query, host\optional{, port}}
Send a \var{selector} string and a \var{query} string to a gopher
server at \var{host} and \var{port} (default \code{70}).  Returns an
open file object from which the returned document can be read.
\end{funcdesc}

Note that the data returned by the Gopher server can be of any type,
depending on the first character of the selector string.  If the data
is text (first character of the selector is \samp{0}), lines are
terminated by CRLF, and the data is terminated by a line consisting of
a single \samp{.}, and a leading \samp{.} should be stripped from
lines that begin with \samp{..}.  Directory listings (first character
of the selector is \samp{1}) are transferred using the same protocol.

\section{Standard Module \sectcode{nntplib}}
\stmodindex{nntplib}

\renewcommand{\indexsubitem}{(in module nntplib)}

This module defines the class \code{NNTP} which implements the client
side of the NNTP protocol.  It can be used to implement a news reader
or poster, or automated news processors.  For more information on NNTP
(Network News Transfer Protocol), see Internet RFC 977.

Here are two small examples of how it can be used.  To list some
statistics about a newsgroup and print the subjects of the last 10
articles:

\small{
\begin{verbatim}
>>> s = NNTP('news.cwi.nl')
>>> resp, count, first, last, name = s.group('comp.lang.python')
>>> print 'Group', name, 'has', count, 'articles, range', first, 'to', last
Group comp.lang.python has 59 articles, range 3742 to 3803
>>> resp, subs = s.xhdr('subject', first + '-' + last)
>>> for id, sub in subs[-10:]: print id, sub
... 
3792 Re: Removing elements from a list while iterating...
3793 Re: Who likes Info files?
3794 Emacs and doc strings
3795 a few questions about the Mac implementation
3796 Re: executable python scripts
3797 Re: executable python scripts
3798 Re: a few questions about the Mac implementation 
3799 Re: PROPOSAL: A Generic Python Object Interface for Python C Modules
3802 Re: executable python scripts 
3803 Re: POSIX wait and SIGCHLD
>>> s.quit()
'205 news.cwi.nl closing connection.  Goodbye.'
>>> 
\end{verbatim}
}

To post an article from a file (this assumes that the article has
valid headers):

\begin{verbatim}
>>> s = NNTP('news.cwi.nl')
>>> f = open('/tmp/article')
>>> s.post(f)
'240 Article posted successfully.'
>>> s.quit()
'205 news.cwi.nl closing connection.  Goodbye.'
>>> 
\end{verbatim}

The module itself defines the following items:

\begin{funcdesc}{NNTP}{host\optional{\, port}}
Return a new instance of the \code{NNTP} class, representing a
connection to the NNTP server running on host \var{host}, listening at
port \var{port}.  The default \var{port} is 119.
\end{funcdesc}

\begin{excdesc}{error_reply}
Exception raised when an unexpected reply is received from the server.
\end{excdesc}

\begin{excdesc}{error_temp}
Exception raised when an error code in the range 400--499 is received.
\end{excdesc}

\begin{excdesc}{error_perm}
Exception raised when an error code in the range 500--599 is received.
\end{excdesc}

\begin{excdesc}{error_proto}
Exception raised when a reply is received from the server that does
not begin with a digit in the range 1--5.
\end{excdesc}

\subsection{NNTP Objects}

NNTP instances have the following methods.  The \var{response} that is
returned as the first item in the return tuple of almost all methods
is the server's response: a string beginning with a three-digit code.
If the server's response indicates an error, the method raises one of
the above exceptions.

\renewcommand{\indexsubitem}{(NNTP object method)}

\begin{funcdesc}{getwelcome}{}
Return the welcome message sent by the server in reply to the initial
connection.  (This message sometimes contains disclaimers or help
information that may be relevant to the user.)
\end{funcdesc}

\begin{funcdesc}{set_debuglevel}{level}
Set the instance's debugging level.  This controls the amount of
debugging output printed.  The default, 0, produces no debugging
output.  A value of 1 produces a moderate amount of debugging output,
generally a single line per request or response.  A value of 2 or
higher produces the maximum amount of debugging output, logging each
line sent and received on the connection (including message text).
\end{funcdesc}

\begin{funcdesc}{newgroups}{date\, time}
Send a \samp{NEWGROUPS} command.  The \var{date} argument should be a
string of the form \code{"\var{yy}\var{mm}\var{dd}"} indicating the
date, and \var{time} should be a string of the form
\code{"\var{hh}\var{mm}\var{ss}"} indicating the time.  Return a pair
\code{(\var{response}, \var{groups})} where \var{groups} is a list of
group names that are new since the given date and time.
\end{funcdesc}

\begin{funcdesc}{newnews}{group\, date\, time}
Send a \samp{NEWNEWS} command.  Here, \var{group} is a group name or
\code{"*"}, and \var{date} and \var{time} have the same meaning as for
\code{newgroups()}.  Return a pair \code{(\var{response},
\var{articles})} where \var{articles} is a list of article ids.
\end{funcdesc}

\begin{funcdesc}{list}{}
Send a \samp{LIST} command.  Return a pair \code{(\var{response},
\var{list})} where \var{list} is a list of tuples.  Each tuple has the
form \code{(\var{group}, \var{last}, \var{first}, \var{flag})}, where
\var{group} is a group name, \var{last} and \var{first} are the last
and first article numbers (as strings), and \var{flag} is \code{'y'}
if posting is allowed, \code{'n'} if not, and \code{'m'} if the
newsgroup is moderated.  (Note the ordering: \var{last}, \var{first}.)
\end{funcdesc}

\begin{funcdesc}{group}{name}
Send a \samp{GROUP} command, where \var{name} is the group name.
Return a tuple \code{(\var{response}, \var{count}, \var{first},
\var{last}, \var{name})} where \var{count} is the (estimated) number
of articles in the group, \var{first} is the first article number in
the group, \var{last} is the last article number in the group, and
\var{name} is the group name.  The numbers are returned as strings.
\end{funcdesc}

\begin{funcdesc}{help}{}
Send a \samp{HELP} command.  Return a pair \code{(\var{response},
\var{list})} where \var{list} is a list of help strings.
\end{funcdesc}

\begin{funcdesc}{stat}{id}
Send a \samp{STAT} command, where \var{id} is the message id (enclosed
in \samp{<} and \samp{>}) or an article number (as a string).
Return a triple \code{(var{response}, \var{number}, \var{id})} where
\var{number} is the article number (as a string) and \var{id} is the
article id  (enclosed in \samp{<} and \samp{>}).
\end{funcdesc}

\begin{funcdesc}{next}{}
Send a \samp{NEXT} command.  Return as for \code{stat()}.
\end{funcdesc}

\begin{funcdesc}{last}{}
Send a \samp{LAST} command.  Return as for \code{stat()}.
\end{funcdesc}

\begin{funcdesc}{head}{id}
Send a \samp{HEAD} command, where \var{id} has the same meaning as for
\code{stat()}.  Return a pair \code{(\var{response}, \var{list})}
where \var{list} is a list of the article's headers (an uninterpreted
list of lines, without trailing newlines).
\end{funcdesc}

\begin{funcdesc}{body}{id}
Send a \samp{BODY} command, where \var{id} has the same meaning as for
\code{stat()}.  Return a pair \code{(\var{response}, \var{list})}
where \var{list} is a list of the article's body text (an
uninterpreted list of lines, without trailing newlines).
\end{funcdesc}

\begin{funcdesc}{article}{id}
Send a \samp{ARTICLE} command, where \var{id} has the same meaning as
for \code{stat()}.  Return a pair \code{(\var{response}, \var{list})}
where \var{list} is a list of the article's header and body text (an
uninterpreted list of lines, without trailing newlines).
\end{funcdesc}

\begin{funcdesc}{slave}{}
Send a \samp{SLAVE} command.  Return the server's \var{response}.
\end{funcdesc}

\begin{funcdesc}{xhdr}{header\, string}
Send an \samp{XHDR} command.  This command is not defined in the RFC
but is a common extension.  The \var{header} argument is a header
keyword, e.g. \code{"subject"}.  The \var{string} argument should have
the form \code{"\var{first}-\var{last}"} where \var{first} and
\var{last} are the first and last article numbers to search.  Return a
pair \code{(\var{response}, \var{list})}, where \var{list} is a list of
pairs \code{(\var{id}, \var{text})}, where \var{id} is an article id
(as a string) and \var{text} is the text of the requested header for
that article.
\end{funcdesc}

\begin{funcdesc}{post}{file}
Post an article using the \samp{POST} command.  The \var{file}
argument is an open file object which is read until EOF using its
\code{readline()} method.  It should be a well-formed news article,
including the required headers.  The \code{post()} method
automatically escapes lines beginning with \samp{.}.
\end{funcdesc}

\begin{funcdesc}{ihave}{id\, file}
Send an \samp{IHAVE} command.  If the response is not an error, treat
\var{file} exactly as for the \code{post()} method.
\end{funcdesc}

\begin{funcdesc}{quit}{}
Send a \samp{QUIT} command and close the connection.  Once this method
has been called, no other methods of the NNTP object should be called.
\end{funcdesc}

\section{\module{urlparse} ---
         Parse URLs into components}
\declaremodule{standard}{urlparse}

\modulesynopsis{Parse URLs into components.}

\index{WWW}
\index{World Wide Web}
\index{URL}
\indexii{URL}{parsing}
\indexii{relative}{URL}


This module defines a standard interface to break Uniform Resource
Locator (URL) strings up in components (addressing scheme, network
location, path etc.), to combine the components back into a URL
string, and to convert a ``relative URL'' to an absolute URL given a
``base URL.''

The module has been designed to match the Internet RFC on Relative
Uniform Resource Locators (and discovered a bug in an earlier
draft!). It supports the following URL schemes:
\code{file}, \code{ftp}, \code{gopher}, \code{hdl}, \code{http}, 
\code{https}, \code{imap}, \code{mailto}, \code{mms}, \code{news}, 
\code{nntp}, \code{prospero}, \code{rsync}, \code{rtsp}, \code{rtspu}, 
\code{sftp}, \code{shttp}, \code{sip}, \code{sips}, \code{snews}, \code{svn}, 
\code{svn+ssh}, \code{telnet}, \code{wais}.

\versionadded[Support for the \code{sftp} and \code{sips} schemes]{2.5}

The \module{urlparse} module defines the following functions:

\begin{funcdesc}{urlparse}{urlstring\optional{,
                           default_scheme\optional{, allow_fragments}}}
Parse a URL into six components, returning a 6-tuple.  This
corresponds to the general structure of a URL:
\code{\var{scheme}://\var{netloc}/\var{path};\var{parameters}?\var{query}\#\var{fragment}}.
Each tuple item is a string, possibly empty.
The components are not broken up in smaller parts (for example, the network
location is a single string), and \% escapes are not expanded.
The delimiters as shown above are not part of the result,
except for a leading slash in the \var{path} component, which is
retained if present.  For example:

\begin{verbatim}
>>> from urlparse import urlparse
>>> o = urlparse('http://www.cwi.nl:80/%7Eguido/Python.html')
>>> o
('http', 'www.cwi.nl:80', '/%7Eguido/Python.html', '', '', '')
>>> o.scheme
'http'
>>> o.port
80
>>> o.geturl()
'http://www.cwi.nl:80/%7Eguido/Python.html'
\end{verbatim}

If the \var{default_scheme} argument is specified, it gives the
default addressing scheme, to be used only if the URL does not
specify one.  The default value for this argument is the empty string.

If the \var{allow_fragments} argument is false, fragment identifiers
are not allowed, even if the URL's addressing scheme normally does
support them.  The default value for this argument is \constant{True}.

The return value is actually an instance of a subclass of
\pytype{tuple}.  This class has the following additional read-only
convenience attributes:

\begin{tableiv}{l|c|l|c}{member}{Attribute}{Index}{Value}{Value if not present}
  \lineiv{scheme}  {0} {URL scheme specifier}             {empty string}
  \lineiv{netloc}  {1} {Network location part}            {empty string}
  \lineiv{path}    {2} {Hierarchical path}                {empty string}
  \lineiv{params}  {3} {Parameters for last path element} {empty string}
  \lineiv{query}   {4} {Query component}                  {empty string}
  \lineiv{fragment}{5} {Fragment identifier}              {empty string}
  \lineiv{username}{ } {User name}                        {\constant{None}}
  \lineiv{password}{ } {Password}                         {\constant{None}}
  \lineiv{hostname}{ } {Host name (lower case)}           {\constant{None}}
  \lineiv{port}    { } {Port number as integer, if present} {\constant{None}}
\end{tableiv}

See section~\ref{urlparse-result-object}, ``Results of
\function{urlparse()} and \function{urlsplit()},'' for more
information on the result object.

\versionchanged[Added attributes to return value]{2.5}
\end{funcdesc}

\begin{funcdesc}{urlunparse}{parts}
Construct a URL from a tuple as returned by \code{urlparse()}.
The \var{parts} argument be any six-item iterable.
This may result in a slightly different, but equivalent URL, if the
URL that was parsed originally had unnecessary delimiters (for example,
a ? with an empty query; the RFC states that these are equivalent).
\end{funcdesc}

\begin{funcdesc}{urlsplit}{urlstring\optional{,
                           default_scheme\optional{, allow_fragments}}}
This is similar to \function{urlparse()}, but does not split the
params from the URL.  This should generally be used instead of
\function{urlparse()} if the more recent URL syntax allowing
parameters to be applied to each segment of the \var{path} portion of
the URL (see \rfc{2396}) is wanted.  A separate function is needed to
separate the path segments and parameters.  This function returns a
5-tuple: (addressing scheme, network location, path, query, fragment
identifier).

The return value is actually an instance of a subclass of
\pytype{tuple}.  This class has the following additional read-only
convenience attributes:

\begin{tableiv}{l|c|l|c}{member}{Attribute}{Index}{Value}{Value if not present}
  \lineiv{scheme}   {0} {URL scheme specifier}   {empty string}
  \lineiv{netloc}   {1} {Network location part}  {empty string}
  \lineiv{path}     {2} {Hierarchical path}      {empty string}
  \lineiv{query}    {3} {Query component}        {empty string}
  \lineiv{fragment} {4} {Fragment identifier}    {empty string}
  \lineiv{username} { } {User name}              {\constant{None}}
  \lineiv{password} { } {Password}               {\constant{None}}
  \lineiv{hostname} { } {Host name (lower case)} {\constant{None}}
  \lineiv{port}     { } {Port number as integer, if present} {\constant{None}}
\end{tableiv}

See section~\ref{urlparse-result-object}, ``Results of
\function{urlparse()} and \function{urlsplit()},'' for more
information on the result object.

\versionadded{2.2}
\versionchanged[Added attributes to return value]{2.5}
\end{funcdesc}

\begin{funcdesc}{urlunsplit}{parts}
Combine the elements of a tuple as returned by \function{urlsplit()}
into a complete URL as a string.
The \var{parts} argument be any five-item iterable.
This may result in a slightly different, but equivalent URL, if the
URL that was parsed originally had unnecessary delimiters (for example,
a ? with an empty query; the RFC states that these are equivalent).
\versionadded{2.2}
\end{funcdesc}

\begin{funcdesc}{urljoin}{base, url\optional{, allow_fragments}}
Construct a full (``absolute'') URL by combining a ``base URL''
(\var{base}) with another URL (\var{url}).  Informally, this
uses components of the base URL, in particular the addressing scheme,
the network location and (part of) the path, to provide missing
components in the relative URL.  For example:

\begin{verbatim}
>>> from urlparse import urljoin
>>> urljoin('http://www.cwi.nl/%7Eguido/Python.html', 'FAQ.html')
'http://www.cwi.nl/%7Eguido/FAQ.html'
\end{verbatim}

The \var{allow_fragments} argument has the same meaning and default as
for \function{urlparse()}.

\note{If \var{url} is an absolute URL (that is, starting with \code{//}
      or \code{scheme://}, the \var{url}'s host name and/or scheme
      will be present in the result.  For example:}

\begin{verbatim}
>>> urljoin('http://www.cwi.nl/%7Eguido/Python.html',
...         '//www.python.org/%7Eguido')
'http://www.python.org/%7Eguido'
\end{verbatim}
      
If you do not want that behavior, preprocess
the \var{url} with \function{urlsplit()} and \function{urlunsplit()},
removing possible \em{scheme} and \em{netloc} parts.
\end{funcdesc}

\begin{funcdesc}{urldefrag}{url}
If \var{url} contains a fragment identifier, returns a modified
version of \var{url} with no fragment identifier, and the fragment
identifier as a separate string.  If there is no fragment identifier
in \var{url}, returns \var{url} unmodified and an empty string.
\end{funcdesc}


\begin{seealso}
  \seerfc{1738}{Uniform Resource Locators (URL)}{
        This specifies the formal syntax and semantics of absolute
        URLs.}
  \seerfc{1808}{Relative Uniform Resource Locators}{
        This Request For Comments includes the rules for joining an
        absolute and a relative URL, including a fair number of
        ``Abnormal Examples'' which govern the treatment of border
        cases.}
  \seerfc{2396}{Uniform Resource Identifiers (URI): Generic Syntax}{
        Document describing the generic syntactic requirements for
        both Uniform Resource Names (URNs) and Uniform Resource
        Locators (URLs).}
\end{seealso}


\subsection{Results of \function{urlparse()} and \function{urlsplit()}
            \label{urlparse-result-object}}

The result objects from the \function{urlparse()} and
\function{urlsplit()} functions are subclasses of the \pytype{tuple}
type.  These subclasses add the attributes described in those
functions, as well as provide an additional method:

\begin{methoddesc}[ParseResult]{geturl}{}
  Return the re-combined version of the original URL as a string.
  This may differ from the original URL in that the scheme will always
  be normalized to lower case and empty components may be dropped.
  Specifically, empty parameters, queries, and fragment identifiers
  will be removed.

  The result of this method is a fixpoint if passed back through the
  original parsing function:

\begin{verbatim}
>>> import urlparse
>>> url = 'HTTP://www.Python.org/doc/#'

>>> r1 = urlparse.urlsplit(url)
>>> r1.geturl()
'http://www.Python.org/doc/'

>>> r2 = urlparse.urlsplit(r1.geturl())
>>> r2.geturl()
'http://www.Python.org/doc/'
\end{verbatim}

\versionadded{2.5}
\end{methoddesc}

The following classes provide the implementations of the parse results::

\begin{classdesc*}{BaseResult}
  Base class for the concrete result classes.  This provides most of
  the attribute definitions.  It does not provide a \method{geturl()}
  method.  It is derived from \class{tuple}, but does not override the
  \method{__init__()} or \method{__new__()} methods.
\end{classdesc*}


\begin{classdesc}{ParseResult}{scheme, netloc, path, params, query, fragment}
  Concrete class for \function{urlparse()} results.  The
  \method{__new__()} method is overridden to support checking that the
  right number of arguments are passed.
\end{classdesc}


\begin{classdesc}{SplitResult}{scheme, netloc, path, query, fragment}
  Concrete class for \function{urlsplit()} results.  The
  \method{__new__()} method is overridden to support checking that the
  right number of arguments are passed.
\end{classdesc}

\section{Built-in module \sectcode{sgmllib}}
\stmodindex{sgmllib}
\index{SGML}

\renewcommand{\indexsubitem}{(in module sgmllib)}

This module defines a class \code{SGMLParser} which serves as the
basis for parsing text files formatted in SGML (Standard Generalized
Mark-up Language).  In fact, it does not provide a full SGML parser
--- it only parses SGML insofar as it is used by HTML, and the module only
exists as a basis for the \code{htmllib} module.
\stmodindex{htmllib}

In particular, the parser is hardcoded to recognize the following
elements:

\begin{itemize}

\item
Opening and closing tags of the form
``\code{<\var{tag} \var{attr}="\var{value}" ...>}'' and
``\code{</\var{tag}>}'', respectively.

\item
Character references of the form ``\code{\&\#\var{name};}''.

\item
Entity references of the form ``\code{\&\var{name};}''.

\item
SGML comments of the form ``\code{<!--\var{text}>}''.

\end{itemize}

The \code{SGMLParser} class must be instantiated without arguments.
It has the following interface methods:

\begin{funcdesc}{reset}{}
Reset the instance.  Loses all unprocessed data.  This is called
implicitly at instantiation time.
\end{funcdesc}

\begin{funcdesc}{setnomoretags}{}
Stop processing tags.  Treat all following input as literal input
(CDATA).  (This is only provided so the HTML tag \code{<PLAINTEXT>}
can be implemented.)
\end{funcdesc}

\begin{funcdesc}{setliteral}{}
Enter literal mode (CDATA mode).
\end{funcdesc}

\begin{funcdesc}{feed}{data}
Feed some text to the parser.  It is processed insofar as it consists
of complete elements; incomplete data is buffered until more data is
fed or \code{close()} is called.
\end{funcdesc}

\begin{funcdesc}{close}{}
Force processing of all buffered data as if it were followed by an
end-of-file mark.  This method may be redefined by a derived class to
define additional processing at the end of the input, but the
redefined version should always call \code{SGMLParser.close()}.
\end{funcdesc}

\begin{funcdesc}{handle_charref}{ref}
This method is called to process a character reference of the form
``\code{\&\#\var{ref};}'' where \var{ref} is a decimal number in the
range 0-255.  It translates the character to ASCII and calls the
method \code{handle_data()} with the character as argument.  If
\var{ref} is invalid or out of range, the method
\code{unknown_charref(\var{ref})} is called instead.
\end{funcdesc}

\begin{funcdesc}{handle_entityref}{ref}
This method is called to process an entity reference of the form
``\code{\&\var{ref};}'' where \var{ref} is an alphabetic entity
reference.  It looks for \var{ref} in the instance (or class)
variable \code{entitydefs} which should give the entity's translation.
If a translation is found, it calls the method \code{handle_data()}
with the translation; otherwise, it calls the method
\code{unknown_entityref(\var{ref})}.
\end{funcdesc}

\begin{funcdesc}{handle_data}{data}
This method is called to process arbitrary data.  It is intended to be
overridden by a derived class; the base class implementation does
nothing.
\end{funcdesc}

\begin{funcdesc}{unknown_starttag}{tag\, attributes}
This method is called to process an unknown start tag.  It is intended
to be overridden by a derived class; the base class implementation
does nothing.  The \var{attributes} argument is a list of
(\var{name}, \var{value}) pairs containing the attributes found inside
the tag's \code{<>} brackets.  The \var{name} has been translated to
lower case and double quotes and backslashes in the \var{value} have
been interpreted.  For instance, for the tag
\code{<A HREF="http://www.cwi.nl/">}, this method would be
called as \code{unknown_starttag('a', [('href', 'http://www.cwi.nl/')])}.
\end{funcdesc}

\begin{funcdesc}{unknown_endtag}{tag}
This method is called to process an unknown end tag.  It is intended
to be overridden by a derived class; the base class implementation
does nothing.
\end{funcdesc}

\begin{funcdesc}{unknown_charref}{ref}
This method is called to process an unknown character reference.  It
is intended to be overridden by a derived class; the base class
implementation does nothing.
\end{funcdesc}

\begin{funcdesc}{unknown_entityref}{ref}
This method is called to process an unknown entity reference.  It is
intended to be overridden by a derived class; the base class
implementation does nothing.
\end{funcdesc}

Apart from overriding or extending the methods listed above, derived
classes may also define methods of the following form to define
processing of specific tags.  Tag names in the input stream are case
independent; the \var{tag} occurring in method names must be in lower
case:

\begin{funcdesc}{start_\var{tag}}{attributes}
This method is called to process an opening tag \var{tag}.  It has
preference over \code{do_\var{tag}()}.  The \var{attributes} argument
has the same meaning as described for \code{unknown_tag()} above.
\end{funcdesc}

\begin{funcdesc}{do_\var{tag}}{attributes}
This method is called to process an opening tag \var{tag} that does
not come with a matching closing tag.  The \var{attributes} argument
has the same meaning as described for \code{unknown_tag()} above.
\end{funcdesc}

\begin{funcdesc}{end_\var{tag}}{}
This method is called to process a closing tag \var{tag}.
\end{funcdesc}

Note that the parser maintains a stack of opening tags for which no
matching closing tag has been found yet.  Only tags processed by
\code{start_\var{tag}()} are pushed on this stack.  Definition of a
\code{end_\var{tag}()} method is optional for these tags.  For tags
processed by \code{do_\var{tag}()} or by \code{unknown_tag()}, no
\code{end_\var{tag}()} method must be defined.

\section{\module{htmllib} ---
         A parser for HTML documents.}
\declaremodule{standard}{htmllib}

\modulesynopsis{A parser for HTML documents.}

\index{HTML}
\index{hypertext}


This module defines a class which can serve as a base for parsing text
files formatted in the HyperText Mark-up Language (HTML).  The class
is not directly concerned with I/O --- it must be provided with input
in string form via a method, and makes calls to methods of a
``formatter'' object in order to produce output.  The
\class{HTMLParser} class is designed to be used as a base class for
other classes in order to add functionality, and allows most of its
methods to be extended or overridden.  In turn, this class is derived
from and extends the \class{SGMLParser} class defined in module
\module{sgmllib}\refstmodindex{sgmllib}.  The \class{HTMLParser}
implementation supports the HTML 2.0 language as described in
\rfc{1866}.  Two implementations of formatter objects are provided in
the \module{formatter}\refstmodindex{formatter} module; refer to the
documentation for that module for information on the formatter
interface.
\index{SGML}
\withsubitem{(in module sgmllib)}{\ttindex{SGMLParser}}
\index{formatter}

The following is a summary of the interface defined by
\class{sgmllib.SGMLParser}:

\begin{itemize}

\item
The interface to feed data to an instance is through the \method{feed()}
method, which takes a string argument.  This can be called with as
little or as much text at a time as desired; \samp{p.feed(a);
p.feed(b)} has the same effect as \samp{p.feed(a+b)}.  When the data
contains complete HTML tags, these are processed immediately;
incomplete elements are saved in a buffer.  To force processing of all
unprocessed data, call the \method{close()} method.

For example, to parse the entire contents of a file, use:
\begin{verbatim}
parser.feed(open('myfile.html').read())
parser.close()
\end{verbatim}

\item
The interface to define semantics for HTML tags is very simple: derive
a class and define methods called \code{start_\var{tag}()},
\code{end_\var{tag}()}, or \code{do_\var{tag}()}.  The parser will
call these at appropriate moments: \code{start_\var{tag}} or
\code{do_\var{tag}()} is called when an opening tag of the form
\code{<\var{tag} ...>} is encountered; \code{end_\var{tag}()} is called
when a closing tag of the form \code{<\var{tag}>} is encountered.  If
an opening tag requires a corresponding closing tag, like \code{<H1>}
... \code{</H1>}, the class should define the \code{start_\var{tag}()}
method; if a tag requires no closing tag, like \code{<P>}, the class
should define the \code{do_\var{tag}()} method.

\end{itemize}

The module defines a single class:

\begin{classdesc}{HTMLParser}{formatter}
This is the basic HTML parser class.  It supports all entity names
required by the HTML 2.0 specification (\rfc{1866}).  It also defines
handlers for all HTML 2.0 and many HTML 3.0 and 3.2 elements.
\end{classdesc}

In addition to tag methods, the \class{HTMLParser} class provides some
additional methods and instance variables for use within tag methods.

\begin{memberdesc}{formatter}
This is the formatter instance associated with the parser.
\end{memberdesc}

\begin{memberdesc}{nofill}
Boolean flag which should be true when whitespace should not be
collapsed, or false when it should be.  In general, this should only
be true when character data is to be treated as ``preformatted'' text,
as within a \code{<PRE>} element.  The default value is false.  This
affects the operation of \method{handle_data()} and \method{save_end()}.
\end{memberdesc}


\begin{methoddesc}{anchor_bgn}{href, name, type}
This method is called at the start of an anchor region.  The arguments
correspond to the attributes of the \code{<A>} tag with the same
names.  The default implementation maintains a list of hyperlinks
(defined by the \code{href} attribute) within the document.  The list
of hyperlinks is available as the data attribute \code{anchorlist}.
\end{methoddesc}

\begin{methoddesc}{anchor_end}{}
This method is called at the end of an anchor region.  The default
implementation adds a textual footnote marker using an index into the
list of hyperlinks created by \method{anchor_bgn()}.
\end{methoddesc}

\begin{methoddesc}{handle_image}{source, alt\optional{, ismap\optional{, align\optional{, width\optional{, height}}}}}
This method is called to handle images.  The default implementation
simply passes the \var{alt} value to the \method{handle_data()}
method.
\end{methoddesc}

\begin{methoddesc}{save_bgn}{}
Begins saving character data in a buffer instead of sending it to the
formatter object.  Retrieve the stored data via \method{save_end()}.
Use of the \method{save_bgn()} / \method{save_end()} pair may not be
nested.
\end{methoddesc}

\begin{methoddesc}{save_end}{}
Ends buffering character data and returns all data saved since the
preceeding call to \method{save_bgn()}.  If the \code{nofill} flag is
false, whitespace is collapsed to single spaces.  A call to this
method without a preceeding call to \method{save_bgn()} will raise a
\exception{TypeError} exception.
\end{methoddesc}

\section{\module{xmllib} ---
         A parser for XML documents}

\declaremodule{standard}{xmllib}
\modulesynopsis{A parser for XML documents.}
\moduleauthor{Sjoerd Mullender}{Sjoerd.Mullender@cwi.nl}
\sectionauthor{Sjoerd Mullender}{Sjoerd.Mullender@cwi.nl}


\index{XML}
\index{Extensible Markup Language}

\versionchanged{1.5.2}

This module defines a class \class{XMLParser} which serves as the basis 
for parsing text files formatted in XML (Extensible Markup Language).

\begin{classdesc}{XMLParser}{}
The \class{XMLParser} class must be instantiated without arguments.
\end{classdesc}

This class provides the following interface methods and instance variables:

\begin{memberdesc}{attributes}
A mapping of element names to mappings.  The latter mapping maps
attribute names that are valid for the element to the default value of 
the attribute, or if there is no default to \code{None}.  The default
value is the empty dictionary.  This variable is meant to be
overridden, not extended since the default is shared by all instances
of \class{XMLParser}.
\end{memberdesc}

\begin{memberdesc}{elements} 
A mapping of element names to tuples.  The tuples contain a function
for handling the start and end tag respectively of the element, or
\code{None} if the method \method{unknown_starttag()} or
\method{unknown_endtag()} is to be called.  The default value is the
empty dictionary.  This variable is meant to be overridden, not
extended since the default is shared by all instances of
\class{XMLParser}.
\end{memberdesc}

\begin{memberdesc}{entitydefs}
A mapping of entitynames to their values.  The default value contains
definitions for \code{'lt'}, \code{'gt'}, \code{'amp'}, \code{'quot'}, 
and \code{'apos'}.
\end{memberdesc}

\begin{methoddesc}{reset}{}
Reset the instance.  Loses all unprocessed data.  This is called
implicitly at the instantiation time.
\end{methoddesc}

\begin{methoddesc}{setnomoretags}{}
Stop processing tags.  Treat all following input as literal input
(CDATA).
\end{methoddesc}

\begin{methoddesc}{setliteral}{}
Enter literal mode (CDATA mode).  This mode is automatically exited
when the close tag matching the last unclosed open tag is encountered.
\end{methoddesc}

\begin{methoddesc}{feed}{data}
Feed some text to the parser.  It is processed insofar as it consists
of complete tags; incomplete data is buffered until more data is
fed or \method{close()} is called.
\end{methoddesc}

\begin{methoddesc}{close}{}
Force processing of all buffered data as if it were followed by an
end-of-file mark.  This method may be redefined by a derived class to
define additional processing at the end of the input, but the
redefined version should always call \method{close()}.
\end{methoddesc}

\begin{methoddesc}{translate_references}{data}
Translate all entity and character references in \var{data} and
return the translated string.
\end{methoddesc}

\begin{methoddesc}{handle_xml}{encoding, standalone}
This method is called when the \samp{<?xml ...?>} tag is processed.
The arguments are the values of the encoding and standalone attributes 
in the tag.  Both encoding and standalone are optional.  The values
passed to \method{handle_xml()} default to \code{None} and the string
\code{'no'} respectively.
\end{methoddesc}

\begin{methoddesc}{handle_doctype}{tag, pubid, syslit, data}
This method is called when the \samp{<!DOCTYPE...>} tag is processed.
The arguments are the name of the root element, the Formal Public
Identifier (or \code{None} if not specified), the system identifier,
and the uninterpreted contents of the internal DTD subset as a string
(or \code{None} if not present).
\end{methoddesc}

\begin{methoddesc}{handle_starttag}{tag, method, attributes}
This method is called to handle start tags for which a start tag
handler is defined in the instance variable \member{elements}.  The
\var{tag} argument is the name of the tag, and the \var{method}
argument is the function (method) which should be used to support semantic
interpretation of the start tag.  The \var{attributes} argument is a
dictionary of attributes, the key being the \var{name} and the value
being the \var{value} of the attribute found inside the tag's
\code{<>} brackets.  Character and entity references in the
\var{value} have been interpreted.  For instance, for the start tag
\code{<A HREF="http://www.cwi.nl/">}, this method would be called as
\code{handle_starttag('A', self.elements['A'][0], \{'HREF': 'http://www.cwi.nl/'\})}.
The base implementation simply calls \var{method} with \var{attributes}
as the only argument.
\end{methoddesc}

\begin{methoddesc}{handle_endtag}{tag, method}
This method is called to handle endtags for which an end tag handler
is defined in the instance variable \member{elements}.  The \var{tag}
argument is the name of the tag, and the \var{method} argument is the
function (method) which should be used to support semantic
interpretation of the end tag.  For instance, for the endtag
\code{</A>}, this method would be called as \code{handle_endtag('A',
self.elements['A'][1])}.  The base implementation simply calls
\var{method}.
\end{methoddesc}

\begin{methoddesc}{handle_data}{data}
This method is called to process arbitrary data.  It is intended to be
overridden by a derived class; the base class implementation does
nothing.
\end{methoddesc}

\begin{methoddesc}{handle_charref}{ref}
This method is called to process a character reference of the form
\samp{\&\#\var{ref};}.  \var{ref} can either be a decimal number,
or a hexadecimal number when preceded by an \character{x}.
In the base implementation, \var{ref} must be a number in the
range 0-255.  It translates the character to \ASCII{} and calls the
method \method{handle_data()} with the character as argument.  If
\var{ref} is invalid or out of range, the method
\code{unknown_charref(\var{ref})} is called to handle the error.  A
subclass must override this method to provide support for character
references outside of the \ASCII{} range.
\end{methoddesc}

\begin{methoddesc}{handle_entityref}{ref}
This method is called to process a general entity reference of the
form \samp{\&\var{ref};} where \var{ref} is an general entity
reference.  It looks for \var{ref} in the instance (or class)
variable \member{entitydefs} which should be a mapping from entity
names to corresponding translations.
If a translation is found, it calls the method \method{handle_data()}
with the translation; otherwise, it calls the method
\code{unknown_entityref(\var{ref})}.  The default \member{entitydefs}
defines translations for \code{\&amp;}, \code{\&apos}, \code{\&gt;},
\code{\&lt;}, and \code{\&quot;}.
\end{methoddesc}

\begin{methoddesc}{handle_comment}{comment}
This method is called when a comment is encountered.  The
\var{comment} argument is a string containing the text between the
\samp{<!--} and \samp{-->} delimiters, but not the delimiters
themselves.  For example, the comment \samp{<!--text-->} will
cause this method to be called with the argument \code{'text'}.  The
default method does nothing.
\end{methoddesc}

\begin{methoddesc}{handle_cdata}{data}
This method is called when a CDATA element is encountered.  The
\var{data} argument is a string containing the text between the
\samp{<![CDATA[} and \samp{]]>} delimiters, but not the delimiters
themselves.  For example, the entity \samp{<![CDATA[text]]>} will
cause this method to be called with the argument \code{'text'}.  The
default method does nothing, and is intended to be overridden.
\end{methoddesc}

\begin{methoddesc}{handle_proc}{name, data}
This method is called when a processing instruction (PI) is
encountered.  The \var{name} is the PI target, and the \var{data}
argument is a string containing the text between the PI target and the
closing delimiter, but not the delimiter itself.  For example, the
instruction \samp{<?XML text?>} will cause this method to be called
with the arguments \code{'XML'} and \code{'text'}.  The default method
does nothing.  Note that if a document starts with \samp{<?xml
..?>}, \method{handle_xml()} is called to handle it.
\end{methoddesc}

\begin{methoddesc}{handle_special}{data}
This method is called when a declaration is encountered.  The
\var{data} argument is a string containing the text between the
\samp{<!} and \samp{>} delimiters, but not the delimiters
themselves.  For example, the entity \samp{<!ENTITY text>} will
cause this method to be called with the argument \code{'ENTITY text'}.  The
default method does nothing.  Note that \samp{<!DOCTYPE ...>} is
handled separately if it is located at the start of the document.
\end{methoddesc}

\begin{methoddesc}{syntax_error}{message}
This method is called when a syntax error is encountered.  The
\var{message} is a description of what was wrong.  The default method 
raises a \exception{RuntimeError} exception.  If this method is
overridden, it is permissable for it to return.  This method is only
called when the error can be recovered from.  Unrecoverable errors
raise a \exception{RuntimeError} without first calling
\method{syntax_error()}.
\end{methoddesc}

\begin{methoddesc}{unknown_starttag}{tag, attributes}
This method is called to process an unknown start tag.  It is intended
to be overridden by a derived class; the base class implementation
does nothing.
\end{methoddesc}

\begin{methoddesc}{unknown_endtag}{tag}
This method is called to process an unknown end tag.  It is intended
to be overridden by a derived class; the base class implementation
does nothing.
\end{methoddesc}

\begin{methoddesc}{unknown_charref}{ref}
This method is called to process unresolvable numeric character
references.  It is intended to be overridden by a derived class; the
base class implementation does nothing.
\end{methoddesc}

\begin{methoddesc}{unknown_entityref}{ref}
This method is called to process an unknown entity reference.  It is
intended to be overridden by a derived class; the base class
implementation does nothing.
\end{methoddesc}


\begin{seealso}
  \seetext{The XML specification, published by the World Wide Web
           Consortium (W3C), is available online at
           \url{http://www.w3.org/TR/REC-xml}.  References to
           additional material on XML are available at
           \url{http://www.w3.org/XML/}.}

  \seetext{The Python XML Topic Guide provides a great deal of information
           on using XML from Python and links to other sources of information
           on XML.  It's located on the Web at
           \url{http://www.python.org/topics/xml/}.}

  \seetext{The Python XML Special Interest Group is developing substantial
           support for processing XML from Python.  See
           \url{http://www.python.org/sigs/xml-sig/} for more information.}
\end{seealso}


\subsection{XML Namespaces \label{xml-namespace}}

This module has support for XML namespaces as defined in the XML
Namespaces proposed recommendation.
\indexii{XML}{namespaces}

Tag and attribute names that are defined in an XML namespace are
handled as if the name of the tag or element consisted of the
namespace (i.e. the URL that defines the namespace) followed by a
space and the name of the tag or attribute.  For instance, the tag
\code{<html xmlns='http://www.w3.org/TR/REC-html40'>} is treated as if 
the tag name was \code{'http://www.w3.org/TR/REC-html40 html'}, and
the tag \code{<html:a href='http://frob.com'>} inside the above
mentioned element is treated as if the tag name were
\code{'http://www.w3.org/TR/REC-html40 a'} and the attribute name as
if it were \code{'http://www.w3.org/TR/REC-html40 src'}.

An older draft of the XML Namespaces proposal is also recognized, but
triggers a warning.

\section{Standard Module \module{formatter}}
\declaremodule{standard}{formatter}

\modulesynopsis{Generic output formatter and device interface.}



This module supports two interface definitions, each with mulitple
implementations.  The \emph{formatter} interface is used by the
\class{HTMLParser} class of the \module{htmllib} module, and the
\emph{writer} interface is required by the formatter interface.
\withsubitem{(class in htmllib)}{\ttindex{HTMLParser}}

Formatter objects transform an abstract flow of formatting events into
specific output events on writer objects.  Formatters manage several
stack structures to allow various properties of a writer object to be
changed and restored; writers need not be able to handle relative
changes nor any sort of ``change back'' operation.  Specific writer
properties which may be controlled via formatter objects are
horizontal alignment, font, and left margin indentations.  A mechanism
is provided which supports providing arbitrary, non-exclusive style
settings to a writer as well.  Additional interfaces facilitate
formatting events which are not reversible, such as paragraph
separation.

Writer objects encapsulate device interfaces.  Abstract devices, such
as file formats, are supported as well as physical devices.  The
provided implementations all work with abstract devices.  The
interface makes available mechanisms for setting the properties which
formatter objects manage and inserting data into the output.


\subsection{The Formatter Interface}

Interfaces to create formatters are dependent on the specific
formatter class being instantiated.  The interfaces described below
are the required interfaces which all formatters must support once
initialized.

One data element is defined at the module level:


\begin{datadesc}{AS_IS}
Value which can be used in the font specification passed to the
\code{push_font()} method described below, or as the new value to any
other \code{push_\var{property}()} method.  Pushing the \code{AS_IS}
value allows the corresponding \code{pop_\var{property}()} method to
be called without having to track whether the property was changed.
\end{datadesc}

The following attributes are defined for formatter instance objects:


\begin{memberdesc}[formatter]{writer}
The writer instance with which the formatter interacts.
\end{memberdesc}


\begin{methoddesc}[formatter]{end_paragraph}{blanklines}
Close any open paragraphs and insert at least \var{blanklines}
before the next paragraph.
\end{methoddesc}

\begin{methoddesc}[formatter]{add_line_break}{}
Add a hard line break if one does not already exist.  This does not
break the logical paragraph.
\end{methoddesc}

\begin{methoddesc}[formatter]{add_hor_rule}{*args, **kw}
Insert a horizontal rule in the output.  A hard break is inserted if
there is data in the current paragraph, but the logical paragraph is
not broken.  The arguments and keywords are passed on to the writer's
\method{send_line_break()} method.
\end{methoddesc}

\begin{methoddesc}[formatter]{add_flowing_data}{data}
Provide data which should be formatted with collapsed whitespaces.
Whitespace from preceeding and successive calls to
\method{add_flowing_data()} is considered as well when the whitespace
collapse is performed.  The data which is passed to this method is
expected to be word-wrapped by the output device.  Note that any
word-wrapping still must be performed by the writer object due to the
need to rely on device and font information.
\end{methoddesc}

\begin{methoddesc}[formatter]{add_literal_data}{data}
Provide data which should be passed to the writer unchanged.
Whitespace, including newline and tab characters, are considered legal
in the value of \var{data}.  
\end{methoddesc}

\begin{methoddesc}[formatter]{add_label_data}{format, counter}
Insert a label which should be placed to the left of the current left
margin.  This should be used for constructing bulleted or numbered
lists.  If the \var{format} value is a string, it is interpreted as a
format specification for \var{counter}, which should be an integer.
The result of this formatting becomes the value of the label; if
\var{format} is not a string it is used as the label value directly.
The label value is passed as the only argument to the writer's
\method{send_label_data()} method.  Interpretation of non-string label
values is dependent on the associated writer.

Format specifications are strings which, in combination with a counter
value, are used to compute label values.  Each character in the format
string is copied to the label value, with some characters recognized
to indicate a transform on the counter value.  Specifically, the
character \character{1} represents the counter value formatter as an
arabic number, the characters \character{A} and \character{a}
represent alphabetic representations of the counter value in upper and
lower case, respectively, and \character{I} and \character{i}
represent the counter value in Roman numerals, in upper and lower
case.  Note that the alphabetic and roman transforms require that the
counter value be greater than zero.
\end{methoddesc}

\begin{methoddesc}[formatter]{flush_softspace}{}
Send any pending whitespace buffered from a previous call to
\method{add_flowing_data()} to the associated writer object.  This
should be called before any direct manipulation of the writer object.
\end{methoddesc}

\begin{methoddesc}[formatter]{push_alignment}{align}
Push a new alignment setting onto the alignment stack.  This may be
\constant{AS_IS} if no change is desired.  If the alignment value is
changed from the previous setting, the writer's \method{new_alignment()}
method is called with the \var{align} value.
\end{methoddesc}

\begin{methoddesc}[formatter]{pop_alignment}{}
Restore the previous alignment.
\end{methoddesc}

\begin{methoddesc}[formatter]{push_font}{\code{(}size, italic, bold, teletype\code{)}}
Change some or all font properties of the writer object.  Properties
which are not set to \constant{AS_IS} are set to the values passed in
while others are maintained at their current settings.  The writer's
\method{new_font()} method is called with the fully resolved font
specification.
\end{methoddesc}

\begin{methoddesc}[formatter]{pop_font}{}
Restore the previous font.
\end{methoddesc}

\begin{methoddesc}[formatter]{push_margin}{margin}
Increase the number of left margin indentations by one, associating
the logical tag \var{margin} with the new indentation.  The initial
margin level is \code{0}.  Changed values of the logical tag must be
true values; false values other than \constant{AS_IS} are not
sufficient to change the margin.
\end{methoddesc}

\begin{methoddesc}[formatter]{pop_margin}{}
Restore the previous margin.
\end{methoddesc}

\begin{methoddesc}[formatter]{push_style}{*styles}
Push any number of arbitrary style specifications.  All styles are
pushed onto the styles stack in order.  A tuple representing the
entire stack, including \constant{AS_IS} values, is passed to the
writer's \method{new_styles()} method.
\end{methoddesc}

\begin{methoddesc}[formatter]{pop_style}{\optional{n\code{ = 1}}}
Pop the last \var{n} style specifications passed to
\method{push_style()}.  A tuple representing the revised stack,
including \constant{AS_IS} values, is passed to the writer's
\method{new_styles()} method.
\end{methoddesc}

\begin{methoddesc}[formatter]{set_spacing}{spacing}
Set the spacing style for the writer.
\end{methoddesc}

\begin{methoddesc}[formatter]{assert_line_data}{\optional{flag\code{ = 1}}}
Inform the formatter that data has been added to the current paragraph
out-of-band.  This should be used when the writer has been manipulated
directly.  The optional \var{flag} argument can be set to false if
the writer manipulations produced a hard line break at the end of the
output.
\end{methoddesc}


\subsection{Formatter Implementations}

Two implementations of formatter objects are provided by this module.
Most applications may use one of these classes without modification or
subclassing.

\begin{classdesc}{NullFormatter}{\optional{writer}}
A formatter which does nothing.  If \var{writer} is omitted, a
\class{NullWriter} instance is created.  No methods of the writer are
called by \class{NullFormatter} instances.  Implementations should
inherit from this class if implementing a writer interface but don't
need to inherit any implementation.
\end{classdesc}

\begin{classdesc}{AbstractFormatter}{writer}
The standard formatter.  This implementation has demonstrated wide
applicability to many writers, and may be used directly in most
circumstances.  It has been used to implement a full-featured
world-wide web browser.
\end{classdesc}



\subsection{The Writer Interface}

Interfaces to create writers are dependent on the specific writer
class being instantiated.  The interfaces described below are the
required interfaces which all writers must support once initialized.
Note that while most applications can use the
\class{AbstractFormatter} class as a formatter, the writer must
typically be provided by the application.


\begin{methoddesc}[writer]{flush}{}
Flush any buffered output or device control events.
\end{methoddesc}

\begin{methoddesc}[writer]{new_alignment}{align}
Set the alignment style.  The \var{align} value can be any object,
but by convention is a string or \code{None}, where \code{None}
indicates that the writer's ``preferred'' alignment should be used.
Conventional \var{align} values are \code{'left'}, \code{'center'},
\code{'right'}, and \code{'justify'}.
\end{methoddesc}

\begin{methoddesc}[writer]{new_font}{font}
Set the font style.  The value of \var{font} will be \code{None},
indicating that the device's default font should be used, or a tuple
of the form \code{(}\var{size}, \var{italic}, \var{bold},
\var{teletype}\code{)}.  Size will be a string indicating the size of
font that should be used; specific strings and their interpretation
must be defined by the application.  The \var{italic}, \var{bold}, and
\var{teletype} values are boolean indicators specifying which of those
font attributes should be used.
\end{methoddesc}

\begin{methoddesc}[writer]{new_margin}{margin, level}
Set the margin level to the integer \var{level} and the logical tag
to \var{margin}.  Interpretation of the logical tag is at the
writer's discretion; the only restriction on the value of the logical
tag is that it not be a false value for non-zero values of
\var{level}.
\end{methoddesc}

\begin{methoddesc}[writer]{new_spacing}{spacing}
Set the spacing style to \var{spacing}.
\end{methoddesc}

\begin{methoddesc}[writer]{new_styles}{styles}
Set additional styles.  The \var{styles} value is a tuple of
arbitrary values; the value \constant{AS_IS} should be ignored.  The
\var{styles} tuple may be interpreted either as a set or as a stack
depending on the requirements of the application and writer
implementation.
\end{methoddesc}

\begin{methoddesc}[writer]{send_line_break}{}
Break the current line.
\end{methoddesc}

\begin{methoddesc}[writer]{send_paragraph}{blankline}
Produce a paragraph separation of at least \var{blankline} blank
lines, or the equivelent.  The \var{blankline} value will be an
integer.
\end{methoddesc}

\begin{methoddesc}[writer]{send_hor_rule}{*args, **kw}
Display a horizontal rule on the output device.  The arguments to this
method are entirely application- and writer-specific, and should be
interpreted with care.  The method implementation may assume that a
line break has already been issued via \method{send_line_break()}.
\end{methoddesc}

\begin{methoddesc}[writer]{send_flowing_data}{data}
Output character data which may be word-wrapped and re-flowed as
needed.  Within any sequence of calls to this method, the writer may
assume that spans of multiple whitespace characters have been
collapsed to single space characters.
\end{methoddesc}

\begin{methoddesc}[writer]{send_literal_data}{data}
Output character data which has already been formatted
for display.  Generally, this should be interpreted to mean that line
breaks indicated by newline characters should be preserved and no new
line breaks should be introduced.  The data may contain embedded
newline and tab characters, unlike data provided to the
\method{send_formatted_data()} interface.
\end{methoddesc}

\begin{methoddesc}[writer]{send_label_data}{data}
Set \var{data} to the left of the current left margin, if possible.
The value of \var{data} is not restricted; treatment of non-string
values is entirely application- and writer-dependent.  This method
will only be called at the beginning of a line.
\end{methoddesc}


\subsection{Writer Implementations}

Three implementations of the writer object interface are provided as
examples by this module.  Most applications will need to derive new
writer classes from the \class{NullWriter} class.

\begin{classdesc}{NullWriter}{}
A writer which only provides the interface definition; no actions are
taken on any methods.  This should be the base class for all writers
which do not need to inherit any implementation methods.
\end{classdesc}

\begin{classdesc}{AbstractWriter}{}
A writer which can be used in debugging formatters, but not much
else.  Each method simply announces itself by printing its name and
arguments on standard output.
\end{classdesc}

\begin{classdesc}{DumbWriter}{\optional{file\optional{, maxcol\code{ = 72}}}}
Simple writer class which writes output on the file object passed in
as \var{file} or, if \var{file} is omitted, on standard output.  The
output is simply word-wrapped to the number of columns specified by
\var{maxcol}.  This class is suitable for reflowing a sequence of
paragraphs.
\end{classdesc}

\section{Built-in module \sectcode{rfc822}}
\stmodindex{rfc822}

This module defines a class, \code{Message}, which represents a
collection of ``email headers'' as defined by the Internet standard
RFC 822.  It is used in various contexts, usually to read such headers
from a file.

A \code{Message} instance is instantiated with an open file object as
parameter.  Instantiation reads headers from the file up to a blank
line and stores them in the instance; after instantiation, the file is
positioned directly after the blank line that terminates the headers.

Input lines as read from the file may either be terminated by CR-LF or
by a single linefeed; a terminating CR-LF is replaced by a single
linefeed before the line is stored.

All header matching is done independent of upper or lower case;
e.g. \code{m['From']}, \code{m['from']} and \code{m['FROM']} all yield
the same result.

A \code{Message} instance has the following methods:

\begin{funcdesc}{rewindbody}{}
Seek to the start of the message body.  This only works if the file
object is seekable.
\end{funcdesc}

\begin{funcdesc}{getallmatchingheaders}{name}
Return a list of lines consisting of all headers whose header matches
\var{name}, if any.  Each physical line, whether it is a continuation
line or not, is a separate list item.  Return the empty list if no
header matches \var{name}.
\end{funcdesc}

\begin{funcdesc}{getfirstmatchingheader}{name}
Return a list of lines comprising the first header matching
\var{name}, and its continuation line(s), if any.  Return \code{None}
if there is no header matching \var{name}.
\end{funcdesc}

\begin{funcdesc}{getrawheader}{name}
Return a single string consisting of the text after the colon in the
first header matching \var{name}.  This includes leading whitespace,
the trailing linefeed, and internal linefeeds and whitespace if there
any continuation line(s) were present.  Return \code{None} if there is
no header matching \var{name}.
\end{funcdesc}

\begin{funcdesc}{getheader}{name}
Like \code{getrawheader(\var{name})}, but strip leading and trailing
whitespace (but not internal whitespace).
\end{funcdesc}

\begin{funcdesc}{getaddr}{name}
Return a pair (full name, email address) parsed from the string
returned by \code{getheader(\var{name})}.  If no header matching
\var{name} exists, return \code{None, None}; otherwise both the full
name and the address are (possibly empty )strings.

Example: if \code{m}'s first \code{From} header contains the string
\code{'guido@cwi.nl (Guido van Rossum)'}, then
\code{m.getaddr('From')} will yield the pair
\code{('Guido van Rossum', 'guido\@cwi.nl')}.
If the header contained
\code{'Guido van Rossum <guido\@cwi.nl>'} instead, it would yield the
exact same result.
\end{funcdesc}

\begin{funcdesc}{getaddrlist}{name}
This is similar to \code{getaddr(\var{list})}, but parses a header
containing a list of email addresses (e.g. a \code{To} header) and
returns a list of (full name, email address) pairs (even if there was
only one address in the header).  If there is no header matching
\var{name}, return an empty list.

XXX The current version of this function is not really correct.  It
yields bogus results if a full name contains a comma.
\end{funcdesc}

\begin{funcdesc}{getdate}{name}
Retrieve a header using \code{getheader} and parse it into a 9-tuple
compatible with \code{time.kmtime()}.  If there is no header matching
\var{name}, or it is unparsable, return \code{None}.

Date parsing appears to be a black art, and not all mailers adhere to
the standard.  While it has been tested and found correct on a large
collection of email from many sources, it is still possible that this
function may occasionally yield an incorrect result.
\end{funcdesc}

\code{Message} instances also support a read-only mapping interface.
In particular: \code{m[name]} is the same as \code{m.getheader(name)};
and \code{len(m)}, \code{m.has_key(name)}, \code{m.keys()},
\code{m.values()} and \code{m.items()} act as expected (and
consistently).

Finally, \code{Message} instances have two public instance variables:

\begin{datadesc}{headers}
A list containing the entire set of header lines, in the order in
which they were read.  Each line contains a trailing newline.  The
blank line terminating the headers is not contained in the list.
\end{datadesc}

\begin{datadesc}{fp}
The file object passed at instantiation time.
\end{datadesc}

\section{\module{mimetools} ---
         Tools for parsing MIME messages}

\declaremodule{standard}{mimetools}
\modulesynopsis{Tools for parsing MIME-style message bodies.}


This module defines a subclass of the
\refmodule{rfc822}\refstmodindex{rfc822} module's
\class{Message} class and a number of utility functions that are
useful for the manipulation for MIME multipart or encoded message.

It defines the following items:

\begin{classdesc}{Message}{fp\optional{, seekable}}
Return a new instance of the \class{Message} class.  This is a
subclass of the \class{rfc822.Message} class, with some additional
methods (see below).  The \var{seekable} argument has the same meaning
as for \class{rfc822.Message}.
\end{classdesc}

\begin{funcdesc}{choose_boundary}{}
Return a unique string that has a high likelihood of being usable as a
part boundary.  The string has the form
\code{'\var{hostipaddr}.\var{uid}.\var{pid}.\var{timestamp}.\var{random}'}.
\end{funcdesc}

\begin{funcdesc}{decode}{input, output, encoding}
Read data encoded using the allowed MIME \var{encoding} from open file
object \var{input} and write the decoded data to open file object
\var{output}.  Valid values for \var{encoding} include
\code{'base64'}, \code{'quoted-printable'}, \code{'uuencode'},
\code{'x-uuencode'}, \code{'uue'}, \code{'x-uue'}, \code{'7bit'}, and 
\code{'8bit'}.  Decoding messages encoded in \code{'7bit'} or \code{'8bit'}
has no effect.  The input is simply copied to the output.
\end{funcdesc}

\begin{funcdesc}{encode}{input, output, encoding}
Read data from open file object \var{input} and write it encoded using
the allowed MIME \var{encoding} to open file object \var{output}.
Valid values for \var{encoding} are the same as for \method{decode()}.
\end{funcdesc}

\begin{funcdesc}{copyliteral}{input, output}
Read lines from open file \var{input} until \EOF{} and write them to
open file \var{output}.
\end{funcdesc}

\begin{funcdesc}{copybinary}{input, output}
Read blocks until \EOF{} from open file \var{input} and write them to
open file \var{output}.  The block size is currently fixed at 8192.
\end{funcdesc}


\begin{seealso}
  \seemodule{email}{Comprehensive email handling package; supercedes
                    the \module{mimetools} module.}
  \seemodule{rfc822}{Provides the base class for
                     \class{mimetools.Message}.}
  \seemodule{multifile}{Support for reading files which contain
                        distinct parts, such as MIME data.}
  \seeurl{http://www.cs.uu.nl/wais/html/na-dir/mail/mime-faq/.html}{
          The MIME Frequently Asked Questions document.  For an
          overview of MIME, see the answer to question 1.1 in Part 1
          of this document.}
\end{seealso}


\subsection{Additional Methods of Message Objects
            \label{mimetools-message-objects}}

The \class{Message} class defines the following methods in
addition to the \class{rfc822.Message} methods:

\begin{methoddesc}{getplist}{}
Return the parameter list of the \mailheader{Content-Type} header.
This is a list of strings.  For parameters of the form
\samp{\var{key}=\var{value}}, \var{key} is converted to lower case but
\var{value} is not.  For example, if the message contains the header
\samp{Content-type: text/html; spam=1; Spam=2; Spam} then
\method{getplist()} will return the Python list \code{['spam=1',
'spam=2', 'Spam']}.
\end{methoddesc}

\begin{methoddesc}{getparam}{name}
Return the \var{value} of the first parameter (as returned by
\method{getplist()} of the form \samp{\var{name}=\var{value}} for the
given \var{name}.  If \var{value} is surrounded by quotes of the form
`\code{<}...\code{>}' or `\code{"}...\code{"}', these are removed.
\end{methoddesc}

\begin{methoddesc}{getencoding}{}
Return the encoding specified in the
\mailheader{Content-Transfer-Encoding} message header.  If no such
header exists, return \code{'7bit'}.  The encoding is converted to
lower case.
\end{methoddesc}

\begin{methoddesc}{gettype}{}
Return the message type (of the form \samp{\var{type}/\var{subtype}})
as specified in the \mailheader{Content-Type} header.  If no such
header exists, return \code{'text/plain'}.  The type is converted to
lower case.
\end{methoddesc}

\begin{methoddesc}{getmaintype}{}
Return the main type as specified in the \mailheader{Content-Type}
header.  If no such header exists, return \code{'text'}.  The main
type is converted to lower case.
\end{methoddesc}

\begin{methoddesc}{getsubtype}{}
Return the subtype as specified in the \mailheader{Content-Type}
header.  If no such header exists, return \code{'plain'}.  The subtype
is converted to lower case.
\end{methoddesc}

\section{Standard Module \sectcode{binhex}}
\label{module-binhex}
\stmodindex{binhex}

This module encodes and decodes files in binhex4 format, a format
allowing representation of Macintosh files in ASCII. On the macintosh,
both forks of a file and the finder information are encoded (or
decoded), on other platforms only the data fork is handled.

The \code{binhex} module defines the following functions:

\setindexsubitem{(in module binhex)}

\begin{funcdesc}{binhex}{input\, output}
Convert a binary file with filename \var{input} to binhex file
\var{output}. The \var{output} parameter can either be a filename or a
file-like object (any object supporting a \var{write} and \var{close}
method).
\end{funcdesc}

\begin{funcdesc}{hexbin}{input\optional{\, output}}
Decode a binhex file \var{input}. \var{input} may be a filename or a
file-like object supporting \var{read} and \var{close} methods.
The resulting file is written to a file named \var{output}, unless the
argument is empty in which case the output filename is read from the
binhex file.
\end{funcdesc}

\subsection{Notes}
There is an alternative, more powerful interface to the coder and
decoder, see the source for details.

If you code or decode textfiles on non-Macintosh platforms they will
still use the macintosh newline convention (carriage-return as end of
line).

As of this writing, \var{hexbin} appears to not work in all cases.

\section{Standard Module \sectcode{uu}}
\label{module-uu}
\stmodindex{uu}

This module encodes and decodes files in uuencode format, allowing
arbitrary binary data to be transferred over ascii-only connections.
Wherever a file argument is expected, the methods accept a file-like
object.  For backwards compatibility, a string containing a pathname
is also accepted, and the corresponding file will be opened for
reading and writing; the pathname \code{'-'} is understood to mean the
standard input or output.  However, this interface is deprecated; it's
better for the caller to open the file itself, and be sure that, when
required, the mode is \code{'rb'} or \code{'wb'} on Windows or DOS.

This code was contributed by Lance Ellinghouse, and modified by Jack
Jansen.

The \module{uu} module defines the following functions:

\setindexsubitem{(in module uu)}

\begin{funcdesc}{encode}{in_file, out_file\optional{, name, mode}}
Uuencode file \var{in_file} into file \var{out_file}.  The uuencoded
file will have the header specifying \var{name} and \var{mode} as the
defaults for the results of decoding the file. The default defaults
are taken from \var{in_file}, or \code{'-'} and \code{0666}
respectively. 
\end{funcdesc}

\begin{funcdesc}{decode}{in_file\optional{, out_file, mode}}
This call decodes uuencoded file \var{in_file} placing the result on
file \var{out_file}. If \var{out_file} is a pathname the \var{mode} is
also set. Defaults for \var{out_file} and \var{mode} are taken from
the uuencode header.
\end{funcdesc}

\section{\module{binascii} ---
         Convert between binary and \ASCII{}}

\declaremodule{builtin}{binascii}
\modulesynopsis{Tools for converting between binary and various
                \ASCII{}-encoded binary representations.}


The \module{binascii} module contains a number of methods to convert
between binary and various \ASCII{}-encoded binary
representations. Normally, you will not use these functions directly
but use wrapper modules like \refmodule{uu}\refstmodindex{uu} or
\refmodule{binhex}\refstmodindex{binhex} instead, this module solely
exists because bit-manipulation of large amounts of data is slow in
Python.

The \module{binascii} module defines the following functions:

\begin{funcdesc}{a2b_uu}{string}
Convert a single line of uuencoded data back to binary and return the
binary data. Lines normally contain 45 (binary) bytes, except for the
last line. Line data may be followed by whitespace.
\end{funcdesc}

\begin{funcdesc}{b2a_uu}{data}
Convert binary data to a line of \ASCII{} characters, the return value
is the converted line, including a newline char. The length of
\var{data} should be at most 45.
\end{funcdesc}

\begin{funcdesc}{a2b_base64}{string}
Convert a block of base64 data back to binary and return the
binary data. More than one line may be passed at a time.
\end{funcdesc}

\begin{funcdesc}{b2a_base64}{data}
Convert binary data to a line of \ASCII{} characters in base64 coding.
The return value is the converted line, including a newline char.
The length of \var{data} should be at most 57 to adhere to the base64
standard.
\end{funcdesc}

\begin{funcdesc}{a2b_hqx}{string}
Convert binhex4 formatted \ASCII{} data to binary, without doing
RLE-decompression. The string should contain a complete number of
binary bytes, or (in case of the last portion of the binhex4 data)
have the remaining bits zero.
\end{funcdesc}

\begin{funcdesc}{rledecode_hqx}{data}
Perform RLE-decompression on the data, as per the binhex4
standard. The algorithm uses \code{0x90} after a byte as a repeat
indicator, followed by a count. A count of \code{0} specifies a byte
value of \code{0x90}. The routine returns the decompressed data,
unless data input data ends in an orphaned repeat indicator, in which
case the \exception{Incomplete} exception is raised.
\end{funcdesc}

\begin{funcdesc}{rlecode_hqx}{data}
Perform binhex4 style RLE-compression on \var{data} and return the
result.
\end{funcdesc}

\begin{funcdesc}{b2a_hqx}{data}
Perform hexbin4 binary-to-\ASCII{} translation and return the
resulting string. The argument should already be RLE-coded, and have a
length divisible by 3 (except possibly the last fragment).
\end{funcdesc}

\begin{funcdesc}{crc_hqx}{data, crc}
Compute the binhex4 crc value of \var{data}, starting with an initial
\var{crc} and returning the result.
\end{funcdesc}

\begin{funcdesc}{crc32}{data\optional{, crc}}
Compute CRC-32, the 32-bit checksum of data, starting with an initial
crc.  This is consistent with the ZIP file checksum.  Use as follows:
\begin{verbatim}
    print binascii.crc32("hello world")
    # Or, in two pieces:
    crc = binascii.crc32("hello")
    crc = binascii.crc32(" world", crc)
    print crc
\end{verbatim}
\end{funcdesc}
 
\begin{funcdesc}{b2a_hex}{data}
Return the hexadecimal representation of the binary \var{data}.  Every
byte of \var{data} is converted into the corresponding 2-digit hex
representation.  The resulting string is therefore, twice as long as
the length of \var{data}.  This function is also available as
\function{hexlify()}.
\end{funcdesc}

\begin{funcdesc}{a2b_hex}{hexstr}
Return the binary data represented by the hexadecimal string
\var{hexstr}.  This function is the inverse of \function{b2a_hex()}.
\var{hexstr} must contain an even number of hexadecimal digits (which
can be upper or lower case), otherwise a \exception{TypeError} is
raised.  This function is also available as \function{unhexlify()}.

\begin{excdesc}{Error}
Exception raised on errors. These are usually programming errors.
\end{excdesc}

\begin{excdesc}{Incomplete}
Exception raised on incomplete data. These are usually not programming
errors, but may be handled by reading a little more data and trying
again.
\end{excdesc}


\begin{seealso}
  \seemodule{base64}{support for base64 encoding used in MIME email messages}

  \seemodule{binhex}{support for the binhex format used on the Macintosh}

  \seemodule{uu}{support for UU encoding used on \UNIX{}}
\end{seealso}

\section{Standard Module \sectcode{xdrlib}}
\label{module-xdrlib}
\stmodindex{xdrlib}
\index{XDR}
\index{RFC!1014}

\renewcommand{\indexsubitem}{(in module xdrlib)}


The \code{xdrlib} module supports the External Data Representation
Standard as described in RFC 1014, written by Sun Microsystems,
Inc. June 1987.  It supports most of the data types described in the
RFC.

The \code{xdrlib} module defines two classes, one for packing
variables into XDR representation, and another for unpacking from XDR
representation.  There are also two exception classes.


\subsection{Packer Objects}

\code{Packer} is the class for packing data into XDR representation.
The \code{Packer} class is instantiated with no arguments.

\begin{funcdesc}{get_buffer}{}
Returns the current pack buffer as a string.
\end{funcdesc}

\begin{funcdesc}{reset}{}
Resets the pack buffer to the empty string.
\end{funcdesc}

In general, you can pack any of the most common XDR data types by
calling the appropriate \code{pack_\var{type}} method.  Each method
takes a single argument, the value to pack.  The following simple data
type packing methods are supported: \code{pack_uint}, \code{pack_int},
\code{pack_enum}, \code{pack_bool}, \code{pack_uhyper},
and \code{pack_hyper}.

\begin{funcdesc}{pack_float}{value}
Packs the single-precision floating point number \var{value}.
\end{funcdesc}

\begin{funcdesc}{pack_double}{value}
Packs the double-precision floating point number \var{value}.
\end{funcdesc}

The following methods support packing strings, bytes, and opaque data:

\begin{funcdesc}{pack_fstring}{n\, s}
Packs a fixed length string, \var{s}.  \var{n} is the length of the
string but it is \emph{not} packed into the data buffer.  The string
is padded with null bytes if necessary to guaranteed 4 byte alignment.
\end{funcdesc}

\begin{funcdesc}{pack_fopaque}{n\, data}
Packs a fixed length opaque data stream, similarly to
\code{pack_fstring}.
\end{funcdesc}

\begin{funcdesc}{pack_string}{s}
Packs a variable length string, \var{s}.  The length of the string is
first packed as an unsigned integer, then the string data is packed
with \code{pack_fstring}.
\end{funcdesc}

\begin{funcdesc}{pack_opaque}{data}
Packs a variable length opaque data string, similarly to
\code{pack_string}.
\end{funcdesc}

\begin{funcdesc}{pack_bytes}{bytes}
Packs a variable length byte stream, similarly to \code{pack_string}.
\end{funcdesc}

The following methods support packing arrays and lists:

\begin{funcdesc}{pack_list}{list\, pack_item}
Packs a \var{list} of homogeneous items.  This method is useful for
lists with an indeterminate size; i.e. the size is not available until
the entire list has been walked.  For each item in the list, an
unsigned integer \code{1} is packed first, followed by the data value
from the list.  \var{pack_item} is the function that is called to pack
the individual item.  At the end of the list, an unsigned integer
\code{0} is packed.
\end{funcdesc}

\begin{funcdesc}{pack_farray}{n\, array\, pack_item}
Packs a fixed length list (\var{array}) of homogeneous items.  \var{n}
is the length of the list; it is \emph{not} packed into the buffer,
but a \code{ValueError} exception is raised if \code{len(array)} is not
equal to \var{n}.  As above, \var{pack_item} is the function used to
pack each element.
\end{funcdesc}

\begin{funcdesc}{pack_array}{list\, pack_item}
Packs a variable length \var{list} of homogeneous items.  First, the
length of the list is packed as an unsigned integer, then each element
is packed as in \code{pack_farray} above.
\end{funcdesc}

\subsection{Unpacker Objects}

\code{Unpacker} is the complementary class which unpacks XDR data
values from a string buffer, and has the following methods:

\begin{funcdesc}{__init__}{data}
Instantiates an \code{Unpacker} object with the string buffer
\var{data}.
\end{funcdesc}

\begin{funcdesc}{reset}{data}
Resets the string buffer with the given \var{data}.
\end{funcdesc}

\begin{funcdesc}{get_position}{}
Returns the current unpack position in the data buffer.
\end{funcdesc}

\begin{funcdesc}{set_position}{position}
Sets the data buffer unpack position to \var{position}.  You should be
careful about using \code{get_position()} and \code{set_position()}.
\end{funcdesc}

\begin{funcdesc}{get_buffer}{}
Returns the current unpack data buffer as a string.
\end{funcdesc}

\begin{funcdesc}{done}{}
Indicates unpack completion.  Raises an \code{xdrlib.Error} exception
if all of the data has not been unpacked.
\end{funcdesc}

In addition, every data type that can be packed with a \code{Packer},
can be unpacked with an \code{Unpacker}.  Unpacking methods are of the
form \code{unpack_\var{type}}, and take no arguments.  They return the
unpacked object.

\begin{funcdesc}{unpack_float}{}
Unpacks a single-precision floating point number.
\end{funcdesc}

\begin{funcdesc}{unpack_double}{}
Unpacks a double-precision floating point number, similarly to
\code{unpack_float}.
\end{funcdesc}

In addition, the following methods unpack strings, bytes, and opaque
data:

\begin{funcdesc}{unpack_fstring}{n}
Unpacks and returns a fixed length string.  \var{n} is the number of
characters expected.  Padding with null bytes to guaranteed 4 byte
alignment is assumed.
\end{funcdesc}

\begin{funcdesc}{unpack_fopaque}{n}
Unpacks and returns a fixed length opaque data stream, similarly to
\code{unpack_fstring}.
\end{funcdesc}

\begin{funcdesc}{unpack_string}{}
Unpacks and returns a variable length string.  The length of the
string is first unpacked as an unsigned integer, then the string data
is unpacked with \code{unpack_fstring}.
\end{funcdesc}

\begin{funcdesc}{unpack_opaque}{}
Unpacks and returns a variable length opaque data string, similarly to
\code{unpack_string}.
\end{funcdesc}

\begin{funcdesc}{unpack_bytes}{}
Unpacks and returns a variable length byte stream, similarly to
\code{unpack_string}.
\end{funcdesc}

The following methods support unpacking arrays and lists:

\begin{funcdesc}{unpack_list}{unpack_item}
Unpacks and returns a list of homogeneous items.  The list is unpacked
one element at a time
by first unpacking an unsigned integer flag.  If the flag is \code{1},
then the item is unpacked and appended to the list.  A flag of
\code{0} indicates the end of the list.  \var{unpack_item} is the
function that is called to unpack the items.
\end{funcdesc}

\begin{funcdesc}{unpack_farray}{n\, unpack_item}
Unpacks and returns (as a list) a fixed length array of homogeneous
items.  \var{n} is number of list elements to expect in the buffer.
As above, \var{unpack_item} is the function used to unpack each element.
\end{funcdesc}

\begin{funcdesc}{unpack_array}{unpack_item}
Unpacks and returns a variable length \var{list} of homogeneous items.
First, the length of the list is unpacked as an unsigned integer, then
each element is unpacked as in \code{unpack_farray} above.
\end{funcdesc}

\subsection{Exceptions}
\nodename{Exceptions in xdrlib module}

Exceptions in this module are coded as class instances:

\begin{excdesc}{Error}
The base exception class.  \code{Error} has a single public data
member \code{msg} containing the description of the error.
\end{excdesc}

\begin{excdesc}{ConversionError}
Class derived from \code{Error}.  Contains no additional instance
variables.
\end{excdesc}

Here is an example of how you would catch one of these exceptions:

\bcode\begin{verbatim}
import xdrlib
p = xdrlib.Packer()
try:
    p.pack_double(8.01)
except xdrlib.ConversionError, instance:
    print 'packing the double failed:', instance.msg
\end{verbatim}\ecode

\section{\module{mailcap} ---
         Mailcap file handling.}
\declaremodule{standard}{mailcap}

\modulesynopsis{Mailcap file handling.}


Mailcap files are used to configure how MIME-aware applications such
as mail readers and Web browsers react to files with different MIME
types. (The name ``mailcap'' is derived from the phrase ``mail
capability''.)  For example, a mailcap file might contain a line like
\samp{video/mpeg; xmpeg \%s}.  Then, if the user encounters an email
message or Web document with the MIME type \mimetype{video/mpeg},
\samp{\%s} will be replaced by a filename (usually one belonging to a
temporary file) and the \program{xmpeg} program can be automatically
started to view the file.

The mailcap format is documented in \rfc{1524}, ``A User Agent
Configuration Mechanism For Multimedia Mail Format Information,'' but
is not an Internet standard.  However, mailcap files are supported on
most \UNIX{} systems.

\begin{funcdesc}{findmatch}{caps, MIMEtype%
                            \optional{, key\optional{,
                            filename\optional{, plist}}}}
Return a 2-tuple; the first element is a string containing the command
line to be executed
(which can be passed to \function{os.system()}), and the second element is
the mailcap entry for a given MIME type.  If no matching MIME
type can be found, \code{(None, None)} is returned.

\var{key} is the name of the field desired, which represents the type
of activity to be performed; the default value is 'view', since in the 
most common case you simply want to view the body of the MIME-typed
data.  Other possible values might be 'compose' and 'edit', if you
wanted to create a new body of the given MIME type or alter the
existing body data.  See \rfc{1524} for a complete list of these
fields.

\var{filename} is the filename to be substituted for \samp{\%s} in the
command line; the default value is
\code{'/dev/null'} which is almost certainly not what you want, so
usually you'll override it by specifying a filename.

\var{plist} can be a list containing named parameters; the default
value is simply an empty list.  Each entry in the list must be a
string containing the parameter name, an equals sign (\character{=}),
and the parameter's value.  Mailcap entries can contain 
named parameters like \code{\%\{foo\}}, which will be replaced by the
value of the parameter named 'foo'.  For example, if the command line
\samp{showpartial \%\{id\}\ \%\{number\}\ \%\{total\}}
was in a mailcap file, and \var{plist} was set to \code{['id=1',
'number=2', 'total=3']}, the resulting command line would be 
\code{'showpartial 1 2 3'}.  

In a mailcap file, the ``test'' field can optionally be specified to
test some external condition (such as the machine architecture, or the
window system in use) to determine whether or not the mailcap line
applies.  \function{findmatch()} will automatically check such
conditions and skip the entry if the check fails.
\end{funcdesc}

\begin{funcdesc}{getcaps}{}
Returns a dictionary mapping MIME types to a list of mailcap file
entries. This dictionary must be passed to the \function{findmatch()}
function.  An entry is stored as a list of dictionaries, but it
shouldn't be necessary to know the details of this representation.

The information is derived from all of the mailcap files found on the
system. Settings in the user's mailcap file \file{\$HOME/.mailcap}
will override settings in the system mailcap files
\file{/etc/mailcap}, \file{/usr/etc/mailcap}, and
\file{/usr/local/etc/mailcap}.
\end{funcdesc}

An example usage:
\begin{verbatim}
>>> import mailcap
>>> d=mailcap.getcaps()
>>> mailcap.findmatch(d, 'video/mpeg', filename='/tmp/tmp1223')
('xmpeg /tmp/tmp1223', {'view': 'xmpeg %s'})
\end{verbatim}

\section{Standard Module \module{base64}}
\declaremodule{standard}{base64}

\modulesynopsis{Encode/decode binary files using the MIME base64 encoding.}

\indexii{base64}{encoding}
\index{MIME!base64 encoding}

This module performs base64 encoding and decoding of arbitrary binary
strings into text strings that can be safely emailed or posted.  The
encoding scheme is defined in \rfc{1421} (``Privacy Enhancement for
Internet Electronic Mail: Part I: Message Encryption and
Authentication Procedures'', section 4.3.2.4, ``Step 4: Printable
Encoding'') and is used for MIME email and
various other Internet-related applications; it is not the same as the
output produced by the \program{uuencode} program.  For example, the
string \code{'www.python.org'} is encoded as the string
\code{'d3d3LnB5dGhvbi5vcmc=\e n'}.  


\begin{funcdesc}{decode}{input, output}
Decode the contents of the \var{input} file and write the resulting
binary data to the \var{output} file.
\var{input} and \var{output} must either be file objects or objects that
mimic the file object interface. \var{input} will be read until
\code{\var{input}.read()} returns an empty string.
\end{funcdesc}

\begin{funcdesc}{decodestring}{s}
Decode the string \var{s}, which must contain one or more lines of
base64 encoded data, and return a string containing the resulting
binary data.
\end{funcdesc}

\begin{funcdesc}{encode}{input, output}
Encode the contents of the \var{input} file and write the resulting
base64 encoded data to the \var{output} file.
\var{input} and \var{output} must either be file objects or objects that
mimic the file object interface. \var{input} will be read until
\code{\var{input}.read()} returns an empty string.
\end{funcdesc}

\begin{funcdesc}{encodestring}{s}
Encode the string \var{s}, which can contain arbitrary binary data,
and return a string containing one or more lines of
base64 encoded data.
\end{funcdesc}

\section{\module{quopri} ---
         Encode and decode MIME quoted-printable data}

\declaremodule{standard}{quopri}
\modulesynopsis{Encode and decode files using the MIME
                quoted-printable encoding.}


This module performs quoted-printable transport encoding and decoding,
as defined in \rfc{1521}: ``MIME (Multipurpose Internet Mail Extensions)
Part One''.  The quoted-printable encoding is designed for data where
there are relatively few nonprintable characters; the base64 encoding
scheme available via the \refmodule{base64} module is more compact if there
are many such characters, as when sending a graphics file.
\indexii{quoted-printable}{encoding}
\index{MIME!quoted-printable encoding}


\begin{funcdesc}{decode}{input, output}
Decode the contents of the \var{input} file and write the resulting
decoded binary data to the \var{output} file.
\var{input} and \var{output} must either be file objects or objects that
mimic the file object interface. \var{input} will be read until
\code{\var{input}.readline()} returns an empty string.
\end{funcdesc}

\begin{funcdesc}{encode}{input, output, quotetabs}
Encode the contents of the \var{input} file and write the resulting
quoted-printable data to the \var{output} file.
\var{input} and \var{output} must either be file objects or objects that
mimic the file object interface. \var{input} will be read until
\code{\var{input}.readline()} returns an empty string.
\var{quotetabs} is a flag which controls whether to encode embedded
spaces and tabs; when true it encodes such embedded whitespace, and
when false it leaves them unencoded.  Note that spaces and tabs
appearing at the end of lines are always encoded, as per \rfc{1521}.
\end{funcdesc}

\begin{funcdesc}{decodestring}{s}
Like \function{decode()}, except that it accepts a source string and
returns the corresponding decoded string.
\end{funcdesc}

\begin{funcdesc}{encodestring}{s\optional{, quotetabs}}
Like \function{encode()}, except that it accepts a source string and
returns the corresponding encoded string.  \var{quotetabs} is optional
(defaulting to 0), and is passed straight through to
\function{encode()}.
\end{funcdesc}


\begin{seealso}
  \seemodule{mimify}{General utilities for processing of MIME messages.}
  \seemodule{base64}{Encode and decode MIME base64 data}
\end{seealso}

\section{Standard Module \sectcode{SocketServer}}
\label{module-SocketServer}
\stmodindex{SocketServer}

The \module{SocketServer} module simplifies the task of writing network
servers.

There are four basic server classes: \class{TCPServer} uses the
Internet TCP protocol, which provides for continuous streams of data
between the client and server.  \class{UDPServer} uses datagrams, which
are discrete packets of information that may arrive out of order or be
lost while in transit.  The more infrequently used
\class{UnixStreamServer} and \class{UnixDatagramServer} classes are
similar, but use \UNIX{} domain sockets; they're not available on
non-\UNIX{} platforms.  For more details on network programming, consult
a book such as W. Richard Steven's \emph{UNIX Network Programming}
or Ralph Davis's \emph{Win32 Network Programming}.

These four classes process requests \dfn{synchronously}; each request
must be completed before the next request can be started.  This isn't
suitable if each request takes a long time to complete, because it
requires a lot of computation, or because it returns a lot of data
which the client is slow to process.  The solution is to create a
separate process or thread to handle each request; the
\class{ForkingMixIn} and \class{ThreadingMixIn} mix-in classes can be
used to support asynchronous behaviour.

Creating a server requires several steps.  First, you must create a
request handler class by subclassing the \class{BaseRequestHandler}
class and overriding its \method{handle()} method; this method will
process incoming requests.  Second, you must instantiate one of the
server classes, passing it the server's address and the request
handler class.  Finally, call the \method{handle_request()} or
\method{serve_forever()} method of the server object to process one or
many requests.

Server classes have the same external methods and attributes, no
matter what network protocol they use:

\setindexsubitem{(SocketServer protocol)}

%XXX should data and methods be intermingled, or separate?
% how should the distinction between class and instance variables be
% drawn?

\begin{funcdesc}{fileno}{}
Return an integer file descriptor for the socket on which the server
is listening.  This function is most commonly passed to
\function{select.select()}, to allow monitoring multiple servers in the
same process.
\end{funcdesc}

\begin{funcdesc}{handle_request}{}
Process a single request.  This function calls the following methods
in order: \method{get_request()}, \method{verify_request()}, and
\method{process_request()}.  If the user-provided \method{handle()}
method of the handler class raises an exception, the server's
\method{handle_error()} method will be called.
\end{funcdesc}

\begin{funcdesc}{serve_forever}{}
Handle an infinite number of requests.  This simply calls
\method{handle_request()} inside an infinite loop.
\end{funcdesc}

\begin{datadesc}{address_family}
The family of protocols to which the server's socket belongs.
\constant{socket.AF_INET} and \constant{socket.AF_UNIX} are two
possible values.
\end{datadesc}

\begin{datadesc}{RequestHandlerClass}
The user-provided request handler class; an instance of this class is
created for each request.
\end{datadesc}

\begin{datadesc}{server_address}
The address on which the server is listening.  The format of addresses
varies depending on the protocol family; see the documentation for the
socket module for details.  For Internet protocols, this is a tuple
containing a string giving the address, and an integer port number:
\code{('127.0.0.1', 80)}, for example.
\end{datadesc}

\begin{datadesc}{socket}
The socket object on which the server will listen for incoming requests.
\end{datadesc}

% XXX should class variables be covered before instance variables, or
% vice versa?

The server classes support the following class variables:

\begin{datadesc}{request_queue_size}
The size of the request queue.  If it takes a long time to process a
single request, any requests that arrive while the server is busy are
placed into a queue, up to \member{request_queue_size} requests.  Once
the queue is full, further requests from clients will get a
``Connection denied'' error.  The default value is usually 5, but this
can be overridden by subclasses.
\end{datadesc}

\begin{datadesc}{socket_type}
The type of socket used by the server; \constant{socket.SOCK_STREAM}
and \constant{socket.SOCK_DGRAM} are two possible values.
\end{datadesc}

There are various server methods that can be overridden by subclasses
of base server classes like \class{TCPServer}; these methods aren't
useful to external users of the server object.

% should the default implementations of these be documented, or should
% it be assumed that the user will look at SocketServer.py?

\begin{funcdesc}{finish_request}{}
Actually processes the request by instantiating
\member{RequestHandlerClass} and calling its \method{handle()} method.
\end{funcdesc}

\begin{funcdesc}{get_request}{}
Must accept a request from the socket, and return a 2-tuple containing
the \emph{new} socket object to be used to communicate with the
client, and the client's address.
\end{funcdesc}

\begin{funcdesc}{handle_error}{request, client_address}
This function is called if the \member{RequestHandlerClass}'s
\method{handle()} method raises an exception.  The default action is
to print the traceback to standard output and continue handling
further requests.
\end{funcdesc}

\begin{funcdesc}{process_request}{request, client_address}
Calls \method{finish_request()} to create an instance of the
\member{RequestHandlerClass}.  If desired, this function can create a
new process or thread to handle the request; the \class{ForkingMixIn}
and \class{ThreadingMixIn} classes do this.
\end{funcdesc}

% Is there any point in documenting the following two functions?
% What would the purpose of overriding them be: initializing server
% instance variables, adding new network families?

\begin{funcdesc}{server_activate}{}
Called by the server's constructor to activate the server.
May be overridden.
\end{funcdesc}

\begin{funcdesc}{server_bind}{}
Called by the server's constructor to bind the socket to the desired
address.  May be overridden.
\end{funcdesc}

\begin{funcdesc}{verify_request}{request, client_address}
Must return a Boolean value; if the value is true, the request will be
processed, and if it's false, the request will be denied.
This function can be overridden to implement access controls for a server.
The default implementation always return true.
\end{funcdesc}

The request handler class must define a new \method{handle()} method,
and can override any of the following methods.  A new instance is
created for each request.

\begin{funcdesc}{finish}{}
Called after the \method{handle()} method to perform any clean-up
actions required.  The default implementation does nothing.  If
\method{setup()} or \method{handle()} raise an exception, this
function will not be called.
\end{funcdesc}

\begin{funcdesc}{handle}{}
This function must do all the work required to service a request.
Several instance attributes are available to it; the request is
available as \member{self.request}; the client address as
\member{self.client_request}; and the server instance as
\member{self.server}, in case it needs access to per-server
information.

The type of \member{self.request} is different for datagram or stream
services.  For stream services, \member{self.request} is a socket
object; for datagram services, \member{self.request} is a string.
However, this can be hidden by using the mix-in request handler
classes
\class{StreamRequestHandler} or \class{DatagramRequestHandler}, which
override the \method{setup()} and \method{finish()} methods, and
provides \member{self.rfile} and \member{self.wfile} attributes.
\member{self.rfile} and \member{self.wfile} can be read or written,
respectively, to get the request data or return data to the client.
\end{funcdesc}

\begin{funcdesc}{setup}{}
Called before the \method{handle()} method to perform any
initialization actions required.  The default implementation does
nothing.
\end{funcdesc}

\section{\module{mailbox} ---
          Manipulate mailboxes in various formats}

\declaremodule{}{mailbox}
\moduleauthor{Gregory K.~Johnson}{gkj@gregorykjohnson.com}
\sectionauthor{Gregory K.~Johnson}{gkj@gregorykjohnson.com}
\modulesynopsis{Manipulate mailboxes in various formats}


This module defines two classes, \class{Mailbox} and \class{Message}, for
accessing and manipulating on-disk mailboxes and the messages they contain.
\class{Mailbox} offers a dictionary-like mapping from keys to messages.
\class{Message} extends the \module{email.Message} module's \class{Message}
class with format-specific state and behavior. Supported mailbox formats are
Maildir, mbox, MH, Babyl, and MMDF.

\begin{seealso}
    \seemodule{email}{Represent and manipulate messages.}
\end{seealso}

\subsection{\class{Mailbox} objects}
\label{mailbox-objects}

\begin{classdesc*}{Mailbox}
A mailbox, which may be inspected and modified.
\end{classdesc*}

The \class{Mailbox} class defines an interface and
is not intended to be instantiated.  Instead, format-specific
subclasses should inherit from \class{Mailbox} and your code
should instantiate a particular subclass.

The \class{Mailbox} interface is dictionary-like, with small keys
corresponding to messages. Keys are issued by the \class{Mailbox}
instance with which they will be used and are only meaningful to that
\class{Mailbox} instance. A key continues to identify a message even
if the corresponding message is modified, such as by replacing it with
another message.

Messages may be added to a \class{Mailbox} instance using the set-like
method \method{add()} and removed using a \code{del} statement or the
set-like methods \method{remove()} and \method{discard()}.

\class{Mailbox} interface semantics differ from dictionary semantics in some
noteworthy ways. Each time a message is requested, a new
representation (typically a \class{Message} instance) is generated
based upon the current state of the mailbox. Similarly, when a message
is added to a \class{Mailbox} instance, the provided message
representation's contents are copied. In neither case is a reference
to the message representation kept by the \class{Mailbox} instance.

The default \class{Mailbox} iterator iterates over message representations, not
keys as the default dictionary iterator does. Moreover, modification of a
mailbox during iteration is safe and well-defined. Messages added to the
mailbox after an iterator is created will not be seen by the iterator. Messages
removed from the mailbox before the iterator yields them will be silently
skipped, though using a key from an iterator may result in a
\exception{KeyError} exception if the corresponding message is subsequently
removed.

\begin{notice}[warning]
Be very cautious when modifying mailboxes that might be
simultaneously changed by some other process.  The safest mailbox
format to use for such tasks is Maildir; try to avoid using
single-file formats such as mbox for concurrent writing.  If you're
modifying a mailbox, you
\emph{must} lock it by calling the \method{lock()} and
\method{unlock()} methods \emph{before} reading any messages in the file
or making any changes by adding or deleting a message.  Failing to
lock the mailbox runs the risk of losing messages or corrupting the entire
mailbox.
\end{notice}

\class{Mailbox} instances have the following methods:

\begin{methoddesc}{add}{message}
Add \var{message} to the mailbox and return the key that has been assigned to
it.

Parameter \var{message} may be a \class{Message} instance, an
\class{email.Message.Message} instance, a string, or a file-like object (which
should be open in text mode). If \var{message} is an instance of the
appropriate format-specific \class{Message} subclass (e.g., if it's an
\class{mboxMessage} instance and this is an \class{mbox} instance), its
format-specific information is used. Otherwise, reasonable defaults for
format-specific information are used.
\end{methoddesc}

\begin{methoddesc}{remove}{key}
\methodline{__delitem__}{key}
\methodline{discard}{key}
Delete the message corresponding to \var{key} from the mailbox.

If no such message exists, a \exception{KeyError} exception is raised if the
method was called as \method{remove()} or \method{__delitem__()} but no
exception is raised if the method was called as \method{discard()}. The
behavior of \method{discard()} may be preferred if the underlying mailbox
format supports concurrent modification by other processes.
\end{methoddesc}

\begin{methoddesc}{__setitem__}{key, message}
Replace the message corresponding to \var{key} with \var{message}. Raise a
\exception{KeyError} exception if no message already corresponds to \var{key}.

As with \method{add()}, parameter \var{message} may be a \class{Message}
instance, an \class{email.Message.Message} instance, a string, or a file-like
object (which should be open in text mode). If \var{message} is an instance of
the appropriate format-specific \class{Message} subclass (e.g., if it's an
\class{mboxMessage} instance and this is an \class{mbox} instance), its
format-specific information is used. Otherwise, the format-specific information
of the message that currently corresponds to \var{key} is left unchanged. 
\end{methoddesc}

\begin{methoddesc}{iterkeys}{}
\methodline{keys}{}
Return an iterator over all keys if called as \method{iterkeys()} or return a
list of keys if called as \method{keys()}.
\end{methoddesc}

\begin{methoddesc}{itervalues}{}
\methodline{__iter__}{}
\methodline{values}{}
Return an iterator over representations of all messages if called as
\method{itervalues()} or \method{__iter__()} or return a list of such
representations if called as \method{values()}. The messages are represented as
instances of the appropriate format-specific \class{Message} subclass unless a
custom message factory was specified when the \class{Mailbox} instance was
initialized. \note{The behavior of \method{__iter__()} is unlike that of
dictionaries, which iterate over keys.}
\end{methoddesc}

\begin{methoddesc}{iteritems}{}
\methodline{items}{}
Return an iterator over (\var{key}, \var{message}) pairs, where \var{key} is a
key and \var{message} is a message representation, if called as
\method{iteritems()} or return a list of such pairs if called as
\method{items()}. The messages are represented as instances of the appropriate
format-specific \class{Message} subclass unless a custom message factory was
specified when the \class{Mailbox} instance was initialized.
\end{methoddesc}

\begin{methoddesc}{get}{key\optional{, default=None}}
\methodline{__getitem__}{key}
Return a representation of the message corresponding to \var{key}. If no such
message exists, \var{default} is returned if the method was called as
\method{get()} and a \exception{KeyError} exception is raised if the method was
called as \method{__getitem__()}. The message is represented as an instance of
the appropriate format-specific \class{Message} subclass unless a custom
message factory was specified when the \class{Mailbox} instance was
initialized.
\end{methoddesc}

\begin{methoddesc}{get_message}{key}
Return a representation of the message corresponding to \var{key} as an
instance of the appropriate format-specific \class{Message} subclass, or raise
a \exception{KeyError} exception if no such message exists.
\end{methoddesc}

\begin{methoddesc}{get_string}{key}
Return a string representation of the message corresponding to \var{key}, or
raise a \exception{KeyError} exception if no such message exists.
\end{methoddesc}

\begin{methoddesc}{get_file}{key}
Return a file-like representation of the message corresponding to \var{key},
or raise a \exception{KeyError} exception if no such message exists. The
file-like object behaves as if open in binary mode. This file should be closed
once it is no longer needed.

\note{Unlike other representations of messages, file-like representations are
not necessarily independent of the \class{Mailbox} instance that created them
or of the underlying mailbox. More specific documentation is provided by each
subclass.}
\end{methoddesc}

\begin{methoddesc}{has_key}{key}
\methodline{__contains__}{key}
Return \code{True} if \var{key} corresponds to a message, \code{False}
otherwise.
\end{methoddesc}

\begin{methoddesc}{__len__}{}
Return a count of messages in the mailbox.
\end{methoddesc}

\begin{methoddesc}{clear}{}
Delete all messages from the mailbox.
\end{methoddesc}

\begin{methoddesc}{pop}{key\optional{, default}}
Return a representation of the message corresponding to \var{key} and delete
the message. If no such message exists, return \var{default} if it was supplied
or else raise a \exception{KeyError} exception. The message is represented as
an instance of the appropriate format-specific \class{Message} subclass unless
a custom message factory was specified when the \class{Mailbox} instance was
initialized.
\end{methoddesc}

\begin{methoddesc}{popitem}{}
Return an arbitrary (\var{key}, \var{message}) pair, where \var{key} is a key
and \var{message} is a message representation, and delete the corresponding
message. If the mailbox is empty, raise a \exception{KeyError} exception. The
message is represented as an instance of the appropriate format-specific
\class{Message} subclass unless a custom message factory was specified when the
\class{Mailbox} instance was initialized.
\end{methoddesc}

\begin{methoddesc}{update}{arg}
Parameter \var{arg} should be a \var{key}-to-\var{message} mapping or an
iterable of (\var{key}, \var{message}) pairs. Updates the mailbox so that, for
each given \var{key} and \var{message}, the message corresponding to \var{key}
is set to \var{message} as if by using \method{__setitem__()}. As with
\method{__setitem__()}, each \var{key} must already correspond to a message in
the mailbox or else a \exception{KeyError} exception will be raised, so in
general it is incorrect for \var{arg} to be a \class{Mailbox} instance.
\note{Unlike with dictionaries, keyword arguments are not supported.}
\end{methoddesc}

\begin{methoddesc}{flush}{}
Write any pending changes to the filesystem. For some \class{Mailbox}
subclasses, changes are always written immediately and \method{flush()} does
nothing, but you should still make a habit of calling this method.
\end{methoddesc}

\begin{methoddesc}{lock}{}
Acquire an exclusive advisory lock on the mailbox so that other processes know
not to modify it. An \exception{ExternalClashError} is raised if the lock is
not available. The particular locking mechanisms used depend upon the mailbox
format.  You should \emph{always} lock the mailbox before making any 
modifications to its contents.
\end{methoddesc}

\begin{methoddesc}{unlock}{}
Release the lock on the mailbox, if any.
\end{methoddesc}

\begin{methoddesc}{close}{}
Flush the mailbox, unlock it if necessary, and close any open files. For some
\class{Mailbox} subclasses, this method does nothing.
\end{methoddesc}


\subsubsection{\class{Maildir}}
\label{mailbox-maildir}

\begin{classdesc}{Maildir}{dirname\optional{, factory=rfc822.Message\optional{,
create=True}}}
A subclass of \class{Mailbox} for mailboxes in Maildir format. Parameter
\var{factory} is a callable object that accepts a file-like message
representation (which behaves as if opened in binary mode) and returns a custom
representation. If \var{factory} is \code{None}, \class{MaildirMessage} is used
as the default message representation. If \var{create} is \code{True}, the
mailbox is created if it does not exist.

It is for historical reasons that \var{factory} defaults to
\class{rfc822.Message} and that \var{dirname} is named as such rather than
\var{path}. For a \class{Maildir} instance that behaves like instances of other
\class{Mailbox} subclasses, set \var{factory} to \code{None}.
\end{classdesc}

Maildir is a directory-based mailbox format invented for the qmail mail
transfer agent and now widely supported by other programs. Messages in a
Maildir mailbox are stored in separate files within a common directory
structure. This design allows Maildir mailboxes to be accessed and modified by
multiple unrelated programs without data corruption, so file locking is
unnecessary.

Maildir mailboxes contain three subdirectories, namely: \file{tmp}, \file{new},
and \file{cur}. Messages are created momentarily in the \file{tmp} subdirectory
and then moved to the \file{new} subdirectory to finalize delivery. A mail user
agent may subsequently move the message to the \file{cur} subdirectory and
store information about the state of the message in a special "info" section
appended to its file name.

Folders of the style introduced by the Courier mail transfer agent are also
supported. Any subdirectory of the main mailbox is considered a folder if
\character{.} is the first character in its name. Folder names are represented
by \class{Maildir} without the leading \character{.}. Each folder is itself a
Maildir mailbox but should not contain other folders. Instead, a logical
nesting is indicated using \character{.} to delimit levels, e.g.,
"Archived.2005.07".

\begin{notice}
The Maildir specification requires the use of a colon (\character{:}) in
certain message file names. However, some operating systems do not permit this
character in file names, If you wish to use a Maildir-like format on such an
operating system, you should specify another character to use instead. The
exclamation point (\character{!}) is a popular choice. For example:
\begin{verbatim}
import mailbox
mailbox.Maildir.colon = '!'
\end{verbatim}
The \member{colon} attribute may also be set on a per-instance basis.
\end{notice}

\class{Maildir} instances have all of the methods of \class{Mailbox} in
addition to the following:

\begin{methoddesc}{list_folders}{}
Return a list of the names of all folders.
\end{methoddesc}

\begin{methoddesc}{get_folder}{folder}
Return a \class{Maildir} instance representing the folder whose name is
\var{folder}. A \exception{NoSuchMailboxError} exception is raised if the
folder does not exist.
\end{methoddesc}

\begin{methoddesc}{add_folder}{folder}
Create a folder whose name is \var{folder} and return a \class{Maildir}
instance representing it.
\end{methoddesc}

\begin{methoddesc}{remove_folder}{folder}
Delete the folder whose name is \var{folder}. If the folder contains any
messages, a \exception{NotEmptyError} exception will be raised and the folder
will not be deleted.
\end{methoddesc}

\begin{methoddesc}{clean}{}
Delete temporary files from the mailbox that have not been accessed in the
last 36 hours. The Maildir specification says that mail-reading programs
should do this occasionally.
\end{methoddesc}

Some \class{Mailbox} methods implemented by \class{Maildir} deserve special
remarks:

\begin{methoddesc}{add}{message}
\methodline[Maildir]{__setitem__}{key, message}
\methodline[Maildir]{update}{arg}
\warning{These methods generate unique file names based upon the current
process ID. When using multiple threads, undetected name clashes may occur and
cause corruption of the mailbox unless threads are coordinated to avoid using
these methods to manipulate the same mailbox simultaneously.}
\end{methoddesc}

\begin{methoddesc}{flush}{}
All changes to Maildir mailboxes are immediately applied, so this method does
nothing.
\end{methoddesc}

\begin{methoddesc}{lock}{}
\methodline{unlock}{}
Maildir mailboxes do not support (or require) locking, so these methods do
nothing. 
\end{methoddesc}

\begin{methoddesc}{close}{}
\class{Maildir} instances do not keep any open files and the underlying
mailboxes do not support locking, so this method does nothing.
\end{methoddesc}

\begin{methoddesc}{get_file}{key}
Depending upon the host platform, it may not be possible to modify or remove
the underlying message while the returned file remains open.
\end{methoddesc}

\begin{seealso}
    \seelink{http://www.qmail.org/man/man5/maildir.html}{maildir man page from
    qmail}{The original specification of the format.}
    \seelink{http://cr.yp.to/proto/maildir.html}{Using maildir format}{Notes
    on Maildir by its inventor. Includes an updated name-creation scheme and
    details on "info" semantics.}
    \seelink{http://www.courier-mta.org/?maildir.html}{maildir man page from
    Courier}{Another specification of the format. Describes a common extension
    for supporting folders.}
\end{seealso}

\subsubsection{\class{mbox}}
\label{mailbox-mbox}

\begin{classdesc}{mbox}{path\optional{, factory=None\optional{, create=True}}}
A subclass of \class{Mailbox} for mailboxes in mbox format. Parameter
\var{factory} is a callable object that accepts a file-like message
representation (which behaves as if opened in binary mode) and returns a custom
representation. If \var{factory} is \code{None}, \class{mboxMessage} is used as
the default message representation. If \var{create} is \code{True}, the mailbox
is created if it does not exist.
\end{classdesc}

The mbox format is the classic format for storing mail on \UNIX{} systems. All
messages in an mbox mailbox are stored in a single file with the beginning of
each message indicated by a line whose first five characters are "From~".

Several variations of the mbox format exist to address perceived shortcomings
in the original. In the interest of compatibility, \class{mbox} implements the
original format, which is sometimes referred to as \dfn{mboxo}. This means that
the \mailheader{Content-Length} header, if present, is ignored and that any
occurrences of "From~" at the beginning of a line in a message body are
transformed to ">From~" when storing the message, although occurences of
">From~" are not transformed to "From~" when reading the message.

Some \class{Mailbox} methods implemented by \class{mbox} deserve special
remarks:

\begin{methoddesc}{get_file}{key}
Using the file after calling \method{flush()} or \method{close()} on the
\class{mbox} instance may yield unpredictable results or raise an exception.
\end{methoddesc}

\begin{methoddesc}{lock}{}
\methodline{unlock}{}
Three locking mechanisms are used---dot locking and, if available, the
\cfunction{flock()} and \cfunction{lockf()} system calls.
\end{methoddesc}

\begin{seealso}
    \seelink{http://www.qmail.org/man/man5/mbox.html}{mbox man page from
    qmail}{A specification of the format and its variations.}
    \seelink{http://www.tin.org/bin/man.cgi?section=5\&topic=mbox}{mbox man
    page from tin}{Another specification of the format, with details on
    locking.}
    \seelink{http://home.netscape.com/eng/mozilla/2.0/relnotes/demo/content-length.html}
    {Configuring Netscape Mail on \UNIX{}: Why The Content-Length Format is
    Bad}{An argument for using the original mbox format rather than a
    variation.}
    \seelink{http://homepages.tesco.net./\tilde{}J.deBoynePollard/FGA/mail-mbox-formats.html}
    {"mbox" is a family of several mutually incompatible mailbox formats}{A
    history of mbox variations.}
\end{seealso}

\subsubsection{\class{MH}}
\label{mailbox-mh}

\begin{classdesc}{MH}{path\optional{, factory=None\optional{, create=True}}}
A subclass of \class{Mailbox} for mailboxes in MH format. Parameter
\var{factory} is a callable object that accepts a file-like message
representation (which behaves as if opened in binary mode) and returns a custom
representation. If \var{factory} is \code{None}, \class{MHMessage} is used as
the default message representation. If \var{create} is \code{True}, the mailbox
is created if it does not exist.
\end{classdesc}

MH is a directory-based mailbox format invented for the MH Message Handling
System, a mail user agent. Each message in an MH mailbox resides in its own
file. An MH mailbox may contain other MH mailboxes (called \dfn{folders}) in
addition to messages. Folders may be nested indefinitely. MH mailboxes also
support \dfn{sequences}, which are named lists used to logically group messages
without moving them to sub-folders. Sequences are defined in a file called
\file{.mh_sequences} in each folder.

The \class{MH} class manipulates MH mailboxes, but it does not attempt to
emulate all of \program{mh}'s behaviors. In particular, it does not modify and
is not affected by the \file{context} or \file{.mh_profile} files that are used
by \program{mh} to store its state and configuration.

\class{MH} instances have all of the methods of \class{Mailbox} in addition to
the following:

\begin{methoddesc}{list_folders}{}
Return a list of the names of all folders.
\end{methoddesc}

\begin{methoddesc}{get_folder}{folder}
Return an \class{MH} instance representing the folder whose name is
\var{folder}. A \exception{NoSuchMailboxError} exception is raised if the
folder does not exist.
\end{methoddesc}

\begin{methoddesc}{add_folder}{folder}
Create a folder whose name is \var{folder} and return an \class{MH} instance
representing it.
\end{methoddesc}

\begin{methoddesc}{remove_folder}{folder}
Delete the folder whose name is \var{folder}. If the folder contains any
messages, a \exception{NotEmptyError} exception will be raised and the folder
will not be deleted.
\end{methoddesc}

\begin{methoddesc}{get_sequences}{}
Return a dictionary of sequence names mapped to key lists. If there are no
sequences, the empty dictionary is returned.
\end{methoddesc}

\begin{methoddesc}{set_sequences}{sequences}
Re-define the sequences that exist in the mailbox based upon \var{sequences}, a
dictionary of names mapped to key lists, like returned by
\method{get_sequences()}.
\end{methoddesc}

\begin{methoddesc}{pack}{}
Rename messages in the mailbox as necessary to eliminate gaps in numbering.
Entries in the sequences list are updated correspondingly. \note{Already-issued
keys are invalidated by this operation and should not be subsequently used.}
\end{methoddesc}

Some \class{Mailbox} methods implemented by \class{MH} deserve special remarks:

\begin{methoddesc}{remove}{key}
\methodline{__delitem__}{key}
\methodline{discard}{key}
These methods immediately delete the message. The MH convention of marking a
message for deletion by prepending a comma to its name is not used.
\end{methoddesc}

\begin{methoddesc}{lock}{}
\methodline{unlock}{}
Three locking mechanisms are used---dot locking and, if available, the
\cfunction{flock()} and \cfunction{lockf()} system calls. For MH mailboxes,
locking the mailbox means locking the \file{.mh_sequences} file and, only for
the duration of any operations that affect them, locking individual message
files.
\end{methoddesc}

\begin{methoddesc}{get_file}{key}
Depending upon the host platform, it may not be possible to remove the
underlying message while the returned file remains open.
\end{methoddesc}

\begin{methoddesc}{flush}{}
All changes to MH mailboxes are immediately applied, so this method does
nothing.
\end{methoddesc}

\begin{methoddesc}{close}{}
\class{MH} instances do not keep any open files, so this method is equivelant
to \method{unlock()}.
\end{methoddesc}

\begin{seealso}
\seelink{http://www.nongnu.org/nmh/}{nmh - Message Handling System}{Home page
of \program{nmh}, an updated version of the original \program{mh}.}
\seelink{http://www.ics.uci.edu/\tilde{}mh/book/}{MH \& nmh: Email for Users \&
Programmers}{A GPL-licensed book on \program{mh} and \program{nmh}, with some
information on the mailbox format.}
\end{seealso}

\subsubsection{\class{Babyl}}
\label{mailbox-babyl}

\begin{classdesc}{Babyl}{path\optional{, factory=None\optional{, create=True}}}
A subclass of \class{Mailbox} for mailboxes in Babyl format. Parameter
\var{factory} is a callable object that accepts a file-like message
representation (which behaves as if opened in binary mode) and returns a custom
representation. If \var{factory} is \code{None}, \class{BabylMessage} is used
as the default message representation. If \var{create} is \code{True}, the
mailbox is created if it does not exist.
\end{classdesc}

Babyl is a single-file mailbox format used by the Rmail mail user agent
included with Emacs. The beginning of a message is indicated by a line
containing the two characters Control-Underscore
(\character{\textbackslash037}) and Control-L (\character{\textbackslash014}).
The end of a message is indicated by the start of the next message or, in the
case of the last message, a line containing a Control-Underscore
(\character{\textbackslash037}) character.

Messages in a Babyl mailbox have two sets of headers, original headers and
so-called visible headers. Visible headers are typically a subset of the
original headers that have been reformatted or abridged to be more attractive.
Each message in a Babyl mailbox also has an accompanying list of \dfn{labels},
or short strings that record extra information about the message, and a list of
all user-defined labels found in the mailbox is kept in the Babyl options
section.

\class{Babyl} instances have all of the methods of \class{Mailbox} in addition
to the following:

\begin{methoddesc}{get_labels}{}
Return a list of the names of all user-defined labels used in the mailbox.
\note{The actual messages are inspected to determine which labels exist in the
mailbox rather than consulting the list of labels in the Babyl options section,
but the Babyl section is updated whenever the mailbox is modified.}
\end{methoddesc}

Some \class{Mailbox} methods implemented by \class{Babyl} deserve special
remarks:

\begin{methoddesc}{get_file}{key}
In Babyl mailboxes, the headers of a message are not stored contiguously with
the body of the message. To generate a file-like representation, the headers
and body are copied together into a \class{StringIO} instance (from the
\module{StringIO} module), which has an API identical to that of a file. As a
result, the file-like object is truly independent of the underlying mailbox but
does not save memory compared to a string representation.
\end{methoddesc}

\begin{methoddesc}{lock}{}
\methodline{unlock}{}
Three locking mechanisms are used---dot locking and, if available, the
\cfunction{flock()} and \cfunction{lockf()} system calls.
\end{methoddesc}

\begin{seealso}
\seelink{http://quimby.gnus.org/notes/BABYL}{Format of Version 5 Babyl Files}{A
specification of the Babyl format.}
\seelink{http://www.gnu.org/software/emacs/manual/html_node/Rmail.html}{Reading
Mail with Rmail}{The Rmail manual, with some information on Babyl semantics.}
\end{seealso}

\subsubsection{\class{MMDF}}
\label{mailbox-mmdf}

\begin{classdesc}{MMDF}{path\optional{, factory=None\optional{, create=True}}}
A subclass of \class{Mailbox} for mailboxes in MMDF format. Parameter
\var{factory} is a callable object that accepts a file-like message
representation (which behaves as if opened in binary mode) and returns a custom
representation. If \var{factory} is \code{None}, \class{MMDFMessage} is used as
the default message representation. If \var{create} is \code{True}, the mailbox
is created if it does not exist.
\end{classdesc}

MMDF is a single-file mailbox format invented for the Multichannel Memorandum
Distribution Facility, a mail transfer agent. Each message is in the same form
as an mbox message but is bracketed before and after by lines containing four
Control-A (\character{\textbackslash001}) characters. As with the mbox format,
the beginning of each message is indicated by a line whose first five
characters are "From~", but additional occurrences of "From~" are not
transformed to ">From~" when storing messages because the extra message
separator lines prevent mistaking such occurrences for the starts of subsequent
messages.

Some \class{Mailbox} methods implemented by \class{MMDF} deserve special
remarks:

\begin{methoddesc}{get_file}{key}
Using the file after calling \method{flush()} or \method{close()} on the
\class{MMDF} instance may yield unpredictable results or raise an exception.
\end{methoddesc}

\begin{methoddesc}{lock}{}
\methodline{unlock}{}
Three locking mechanisms are used---dot locking and, if available, the
\cfunction{flock()} and \cfunction{lockf()} system calls.
\end{methoddesc}

\begin{seealso}
\seelink{http://www.tin.org/bin/man.cgi?section=5\&topic=mmdf}{mmdf man page
from tin}{A specification of MMDF format from the documentation of tin, a
newsreader.}
\seelink{http://en.wikipedia.org/wiki/MMDF}{MMDF}{A Wikipedia article
describing the Multichannel Memorandum Distribution Facility.}
\end{seealso}

\subsection{\class{Message} objects}
\label{mailbox-message-objects}

\begin{classdesc}{Message}{\optional{message}}
A subclass of the \module{email.Message} module's \class{Message}. Subclasses
of \class{mailbox.Message} add mailbox-format-specific state and behavior.

If \var{message} is omitted, the new instance is created in a default, empty
state. If \var{message} is an \class{email.Message.Message} instance, its
contents are copied; furthermore, any format-specific information is converted
insofar as possible if \var{message} is a \class{Message} instance. If
\var{message} is a string or a file, it should contain an \rfc{2822}-compliant
message, which is read and parsed.
\end{classdesc}

The format-specific state and behaviors offered by subclasses vary, but in
general it is only the properties that are not specific to a particular mailbox
that are supported (although presumably the properties are specific to a
particular mailbox format). For example, file offsets for single-file mailbox
formats and file names for directory-based mailbox formats are not retained,
because they are only applicable to the original mailbox. But state such as
whether a message has been read by the user or marked as important is retained,
because it applies to the message itself.

There is no requirement that \class{Message} instances be used to represent
messages retrieved using \class{Mailbox} instances. In some situations, the
time and memory required to generate \class{Message} representations might not
not acceptable. For such situations, \class{Mailbox} instances also offer
string and file-like representations, and a custom message factory may be
specified when a \class{Mailbox} instance is initialized. 

\subsubsection{\class{MaildirMessage}}
\label{mailbox-maildirmessage}

\begin{classdesc}{MaildirMessage}{\optional{message}}
A message with Maildir-specific behaviors. Parameter \var{message}
has the same meaning as with the \class{Message} constructor.
\end{classdesc}

Typically, a mail user agent application moves all of the messages in the
\file{new} subdirectory to the \file{cur} subdirectory after the first time the
user opens and closes the mailbox, recording that the messages are old whether
or not they've actually been read. Each message in \file{cur} has an "info"
section added to its file name to store information about its state. (Some mail
readers may also add an "info" section to messages in \file{new}.) The "info"
section may take one of two forms: it may contain "2," followed by a list of
standardized flags (e.g., "2,FR") or it may contain "1," followed by so-called
experimental information. Standard flags for Maildir messages are as follows:

\begin{tableiii}{l|l|l}{textrm}{Flag}{Meaning}{Explanation}
\lineiii{D}{Draft}{Under composition}
\lineiii{F}{Flagged}{Marked as important}
\lineiii{P}{Passed}{Forwarded, resent, or bounced}
\lineiii{R}{Replied}{Replied to}
\lineiii{S}{Seen}{Read}
\lineiii{T}{Trashed}{Marked for subsequent deletion}
\end{tableiii}

\class{MaildirMessage} instances offer the following methods:

\begin{methoddesc}{get_subdir}{}
Return either "new" (if the message should be stored in the \file{new}
subdirectory) or "cur" (if the message should be stored in the \file{cur}
subdirectory). \note{A message is typically moved from \file{new} to \file{cur}
after its mailbox has been accessed, whether or not the message is has been
read. A message \code{msg} has been read if \code{"S" in msg.get_flags()}
is \code{True}.}
\end{methoddesc}

\begin{methoddesc}{set_subdir}{subdir}
Set the subdirectory the message should be stored in. Parameter \var{subdir}
must be either "new" or "cur".
\end{methoddesc}

\begin{methoddesc}{get_flags}{}
Return a string specifying the flags that are currently set. If the message
complies with the standard Maildir format, the result is the concatenation in
alphabetical order of zero or one occurrence of each of \character{D},
\character{F}, \character{P}, \character{R}, \character{S}, and \character{T}.
The empty string is returned if no flags are set or if "info" contains
experimental semantics.
\end{methoddesc}

\begin{methoddesc}{set_flags}{flags}
Set the flags specified by \var{flags} and unset all others.
\end{methoddesc}

\begin{methoddesc}{add_flag}{flag}
Set the flag(s) specified by \var{flag} without changing other flags. To add
more than one flag at a time, \var{flag} may be a string of more than one
character. The current "info" is overwritten whether or not it contains
experimental information rather than
flags.
\end{methoddesc}

\begin{methoddesc}{remove_flag}{flag}
Unset the flag(s) specified by \var{flag} without changing other flags. To
remove more than one flag at a time, \var{flag} maybe a string of more than one
character. If "info" contains experimental information rather than flags, the
current "info" is not modified.
\end{methoddesc}

\begin{methoddesc}{get_date}{}
Return the delivery date of the message as a floating-point number representing
seconds since the epoch.
\end{methoddesc}

\begin{methoddesc}{set_date}{date}
Set the delivery date of the message to \var{date}, a floating-point number
representing seconds since the epoch.
\end{methoddesc}

\begin{methoddesc}{get_info}{}
Return a string containing the "info" for a message. This is useful for
accessing and modifying "info" that is experimental (i.e., not a list of
flags).
\end{methoddesc}

\begin{methoddesc}{set_info}{info}
Set "info" to \var{info}, which should be a string.
\end{methoddesc}

When a \class{MaildirMessage} instance is created based upon an
\class{mboxMessage} or \class{MMDFMessage} instance, the \mailheader{Status}
and \mailheader{X-Status} headers are omitted and the following conversions
take place:

\begin{tableii}{l|l}{textrm}
    {Resulting state}{\class{mboxMessage} or \class{MMDFMessage} state}
\lineii{"cur" subdirectory}{O flag}
\lineii{F flag}{F flag}
\lineii{R flag}{A flag}
\lineii{S flag}{R flag}
\lineii{T flag}{D flag}
\end{tableii}

When a \class{MaildirMessage} instance is created based upon an
\class{MHMessage} instance, the following conversions take place:

\begin{tableii}{l|l}{textrm}
    {Resulting state}{\class{MHMessage} state}
\lineii{"cur" subdirectory}{"unseen" sequence}
\lineii{"cur" subdirectory and S flag}{no "unseen" sequence}
\lineii{F flag}{"flagged" sequence}
\lineii{R flag}{"replied" sequence}
\end{tableii}

When a \class{MaildirMessage} instance is created based upon a
\class{BabylMessage} instance, the following conversions take place:

\begin{tableii}{l|l}{textrm}
    {Resulting state}{\class{BabylMessage} state}
\lineii{"cur" subdirectory}{"unseen" label}
\lineii{"cur" subdirectory and S flag}{no "unseen" label}
\lineii{P flag}{"forwarded" or "resent" label}
\lineii{R flag}{"answered" label}
\lineii{T flag}{"deleted" label}
\end{tableii}

\subsubsection{\class{mboxMessage}}
\label{mailbox-mboxmessage}

\begin{classdesc}{mboxMessage}{\optional{message}}
A message with mbox-specific behaviors. Parameter \var{message} has the same
meaning as with the \class{Message} constructor.
\end{classdesc}

Messages in an mbox mailbox are stored together in a single file. The sender's
envelope address and the time of delivery are typically stored in a line
beginning with "From~" that is used to indicate the start of a message, though
there is considerable variation in the exact format of this data among mbox
implementations. Flags that indicate the state of the message, such as whether
it has been read or marked as important, are typically stored in
\mailheader{Status} and \mailheader{X-Status} headers.

Conventional flags for mbox messages are as follows:

\begin{tableiii}{l|l|l}{textrm}{Flag}{Meaning}{Explanation}
\lineiii{R}{Read}{Read}
\lineiii{O}{Old}{Previously detected by MUA}
\lineiii{D}{Deleted}{Marked for subsequent deletion}
\lineiii{F}{Flagged}{Marked as important}
\lineiii{A}{Answered}{Replied to}
\end{tableiii}

The "R" and "O" flags are stored in the \mailheader{Status} header, and the
"D", "F", and "A" flags are stored in the \mailheader{X-Status} header. The
flags and headers typically appear in the order mentioned.

\class{mboxMessage} instances offer the following methods:

\begin{methoddesc}{get_from}{}
Return a string representing the "From~" line that marks the start of the
message in an mbox mailbox. The leading "From~" and the trailing newline are
excluded.
\end{methoddesc}

\begin{methoddesc}{set_from}{from_\optional{, time_=None}}
Set the "From~" line to \var{from_}, which should be specified without a
leading "From~" or trailing newline. For convenience, \var{time_} may be
specified and will be formatted appropriately and appended to \var{from_}. If
\var{time_} is specified, it should be a \class{struct_time} instance, a tuple
suitable for passing to \method{time.strftime()}, or \code{True} (to use
\method{time.gmtime()}).
\end{methoddesc}

\begin{methoddesc}{get_flags}{}
Return a string specifying the flags that are currently set. If the message
complies with the conventional format, the result is the concatenation in the
following order of zero or one occurrence of each of \character{R},
\character{O}, \character{D}, \character{F}, and \character{A}.
\end{methoddesc}

\begin{methoddesc}{set_flags}{flags}
Set the flags specified by \var{flags} and unset all others. Parameter
\var{flags} should be the concatenation in any order of zero or more
occurrences of each of \character{R}, \character{O}, \character{D},
\character{F}, and \character{A}.
\end{methoddesc}

\begin{methoddesc}{add_flag}{flag}
Set the flag(s) specified by \var{flag} without changing other flags. To add
more than one flag at a time, \var{flag} may be a string of more than one
character.
\end{methoddesc}

\begin{methoddesc}{remove_flag}{flag}
Unset the flag(s) specified by \var{flag} without changing other flags. To
remove more than one flag at a time, \var{flag} maybe a string of more than one
character.
\end{methoddesc}

When an \class{mboxMessage} instance is created based upon a
\class{MaildirMessage} instance, a "From~" line is generated based upon the
\class{MaildirMessage} instance's delivery date, and the following conversions
take place:

\begin{tableii}{l|l}{textrm}
    {Resulting state}{\class{MaildirMessage} state}
\lineii{R flag}{S flag}
\lineii{O flag}{"cur" subdirectory}
\lineii{D flag}{T flag}
\lineii{F flag}{F flag}
\lineii{A flag}{R flag}
\end{tableii}

When an \class{mboxMessage} instance is created based upon an \class{MHMessage}
instance, the following conversions take place:

\begin{tableii}{l|l}{textrm}
    {Resulting state}{\class{MHMessage} state}
\lineii{R flag and O flag}{no "unseen" sequence}
\lineii{O flag}{"unseen" sequence}
\lineii{F flag}{"flagged" sequence}
\lineii{A flag}{"replied" sequence}
\end{tableii}

When an \class{mboxMessage} instance is created based upon a
\class{BabylMessage} instance, the following conversions take place:

\begin{tableii}{l|l}{textrm}
    {Resulting state}{\class{BabylMessage} state}
\lineii{R flag and O flag}{no "unseen" label}
\lineii{O flag}{"unseen" label}
\lineii{D flag}{"deleted" label}
\lineii{A flag}{"answered" label}
\end{tableii}

When a \class{Message} instance is created based upon an \class{MMDFMessage}
instance, the "From~" line is copied and all flags directly correspond:

\begin{tableii}{l|l}{textrm}
    {Resulting state}{\class{MMDFMessage} state}
\lineii{R flag}{R flag}
\lineii{O flag}{O flag}
\lineii{D flag}{D flag}
\lineii{F flag}{F flag}
\lineii{A flag}{A flag}
\end{tableii}

\subsubsection{\class{MHMessage}}
\label{mailbox-mhmessage}

\begin{classdesc}{MHMessage}{\optional{message}}
A message with MH-specific behaviors. Parameter \var{message} has the same
meaning as with the \class{Message} constructor.
\end{classdesc}

MH messages do not support marks or flags in the traditional sense, but they do
support sequences, which are logical groupings of arbitrary messages. Some mail
reading programs (although not the standard \program{mh} and \program{nmh}) use
sequences in much the same way flags are used with other formats, as follows:

\begin{tableii}{l|l}{textrm}{Sequence}{Explanation}
\lineii{unseen}{Not read, but previously detected by MUA}
\lineii{replied}{Replied to}
\lineii{flagged}{Marked as important}
\end{tableii}

\class{MHMessage} instances offer the following methods:

\begin{methoddesc}{get_sequences}{}
Return a list of the names of sequences that include this message.
\end{methoddesc}

\begin{methoddesc}{set_sequences}{sequences}
Set the list of sequences that include this message.
\end{methoddesc}

\begin{methoddesc}{add_sequence}{sequence}
Add \var{sequence} to the list of sequences that include this message.
\end{methoddesc}

\begin{methoddesc}{remove_sequence}{sequence}
Remove \var{sequence} from the list of sequences that include this message.
\end{methoddesc}

When an \class{MHMessage} instance is created based upon a
\class{MaildirMessage} instance, the following conversions take place:

\begin{tableii}{l|l}{textrm}
    {Resulting state}{\class{MaildirMessage} state}
\lineii{"unseen" sequence}{no S flag}
\lineii{"replied" sequence}{R flag}
\lineii{"flagged" sequence}{F flag}
\end{tableii}

When an \class{MHMessage} instance is created based upon an \class{mboxMessage}
or \class{MMDFMessage} instance, the \mailheader{Status} and
\mailheader{X-Status} headers are omitted and the following conversions take
place:

\begin{tableii}{l|l}{textrm}
    {Resulting state}{\class{mboxMessage} or \class{MMDFMessage} state}
\lineii{"unseen" sequence}{no R flag}
\lineii{"replied" sequence}{A flag}
\lineii{"flagged" sequence}{F flag}
\end{tableii}

When an \class{MHMessage} instance is created based upon a \class{BabylMessage}
instance, the following conversions take place:

\begin{tableii}{l|l}{textrm}
    {Resulting state}{\class{BabylMessage} state}
\lineii{"unseen" sequence}{"unseen" label}
\lineii{"replied" sequence}{"answered" label}
\end{tableii}

\subsubsection{\class{BabylMessage}}
\label{mailbox-babylmessage}

\begin{classdesc}{BabylMessage}{\optional{message}}
A message with Babyl-specific behaviors. Parameter \var{message} has the same
meaning as with the \class{Message} constructor.
\end{classdesc}

Certain message labels, called \dfn{attributes}, are defined by convention to
have special meanings. The attributes are as follows:

\begin{tableii}{l|l}{textrm}{Label}{Explanation}
\lineii{unseen}{Not read, but previously detected by MUA}
\lineii{deleted}{Marked for subsequent deletion}
\lineii{filed}{Copied to another file or mailbox}
\lineii{answered}{Replied to}
\lineii{forwarded}{Forwarded}
\lineii{edited}{Modified by the user}
\lineii{resent}{Resent}
\end{tableii}

By default, Rmail displays only
visible headers. The \class{BabylMessage} class, though, uses the original
headers because they are more complete. Visible headers may be accessed
explicitly if desired.

\class{BabylMessage} instances offer the following methods:

\begin{methoddesc}{get_labels}{}
Return a list of labels on the message.
\end{methoddesc}

\begin{methoddesc}{set_labels}{labels}
Set the list of labels on the message to \var{labels}.
\end{methoddesc}

\begin{methoddesc}{add_label}{label}
Add \var{label} to the list of labels on the message.
\end{methoddesc}

\begin{methoddesc}{remove_label}{label}
Remove \var{label} from the list of labels on the message.
\end{methoddesc}

\begin{methoddesc}{get_visible}{}
Return an \class{Message} instance whose headers are the message's visible
headers and whose body is empty.
\end{methoddesc}

\begin{methoddesc}{set_visible}{visible}
Set the message's visible headers to be the same as the headers in
\var{message}. Parameter \var{visible} should be a \class{Message} instance, an
\class{email.Message.Message} instance, a string, or a file-like object (which
should be open in text mode).
\end{methoddesc}

\begin{methoddesc}{update_visible}{}
When a \class{BabylMessage} instance's original headers are modified, the
visible headers are not automatically modified to correspond. This method
updates the visible headers as follows: each visible header with a
corresponding original header is set to the value of the original header, each
visible header without a corresponding original header is removed, and any of
\mailheader{Date}, \mailheader{From}, \mailheader{Reply-To}, \mailheader{To},
\mailheader{CC}, and \mailheader{Subject} that are present in the original
headers but not the visible headers are added to the visible headers.
\end{methoddesc}

When a \class{BabylMessage} instance is created based upon a
\class{MaildirMessage} instance, the following conversions take place:

\begin{tableii}{l|l}{textrm}
    {Resulting state}{\class{MaildirMessage} state}
\lineii{"unseen" label}{no S flag}
\lineii{"deleted" label}{T flag}
\lineii{"answered" label}{R flag}
\lineii{"forwarded" label}{P flag}
\end{tableii}

When a \class{BabylMessage} instance is created based upon an
\class{mboxMessage} or \class{MMDFMessage} instance, the \mailheader{Status}
and \mailheader{X-Status} headers are omitted and the following conversions
take place:

\begin{tableii}{l|l}{textrm}
    {Resulting state}{\class{mboxMessage} or \class{MMDFMessage} state}
\lineii{"unseen" label}{no R flag}
\lineii{"deleted" label}{D flag}
\lineii{"answered" label}{A flag}
\end{tableii}

When a \class{BabylMessage} instance is created based upon an \class{MHMessage}
instance, the following conversions take place:

\begin{tableii}{l|l}{textrm}
    {Resulting state}{\class{MHMessage} state}
\lineii{"unseen" label}{"unseen" sequence}
\lineii{"answered" label}{"replied" sequence}
\end{tableii}

\subsubsection{\class{MMDFMessage}}
\label{mailbox-mmdfmessage}

\begin{classdesc}{MMDFMessage}{\optional{message}}
A message with MMDF-specific behaviors. Parameter \var{message} has the same
meaning as with the \class{Message} constructor.
\end{classdesc}

As with message in an mbox mailbox, MMDF messages are stored with the sender's
address and the delivery date in an initial line beginning with "From ".
Likewise, flags that indicate the state of the message are typically stored in
\mailheader{Status} and \mailheader{X-Status} headers.

Conventional flags for MMDF messages are identical to those of mbox message and
are as follows:

\begin{tableiii}{l|l|l}{textrm}{Flag}{Meaning}{Explanation}
\lineiii{R}{Read}{Read}
\lineiii{O}{Old}{Previously detected by MUA}
\lineiii{D}{Deleted}{Marked for subsequent deletion}
\lineiii{F}{Flagged}{Marked as important}
\lineiii{A}{Answered}{Replied to}
\end{tableiii}

The "R" and "O" flags are stored in the \mailheader{Status} header, and the
"D", "F", and "A" flags are stored in the \mailheader{X-Status} header. The
flags and headers typically appear in the order mentioned.

\class{MMDFMessage} instances offer the following methods, which are identical
to those offered by \class{mboxMessage}:

\begin{methoddesc}{get_from}{}
Return a string representing the "From~" line that marks the start of the
message in an mbox mailbox. The leading "From~" and the trailing newline are
excluded.
\end{methoddesc}

\begin{methoddesc}{set_from}{from_\optional{, time_=None}}
Set the "From~" line to \var{from_}, which should be specified without a
leading "From~" or trailing newline. For convenience, \var{time_} may be
specified and will be formatted appropriately and appended to \var{from_}. If
\var{time_} is specified, it should be a \class{struct_time} instance, a tuple
suitable for passing to \method{time.strftime()}, or \code{True} (to use
\method{time.gmtime()}).
\end{methoddesc}

\begin{methoddesc}{get_flags}{}
Return a string specifying the flags that are currently set. If the message
complies with the conventional format, the result is the concatenation in the
following order of zero or one occurrence of each of \character{R},
\character{O}, \character{D}, \character{F}, and \character{A}.
\end{methoddesc}

\begin{methoddesc}{set_flags}{flags}
Set the flags specified by \var{flags} and unset all others. Parameter
\var{flags} should be the concatenation in any order of zero or more
occurrences of each of \character{R}, \character{O}, \character{D},
\character{F}, and \character{A}.
\end{methoddesc}

\begin{methoddesc}{add_flag}{flag}
Set the flag(s) specified by \var{flag} without changing other flags. To add
more than one flag at a time, \var{flag} may be a string of more than one
character.
\end{methoddesc}

\begin{methoddesc}{remove_flag}{flag}
Unset the flag(s) specified by \var{flag} without changing other flags. To
remove more than one flag at a time, \var{flag} maybe a string of more than one
character.
\end{methoddesc}

When an \class{MMDFMessage} instance is created based upon a
\class{MaildirMessage} instance, a "From~" line is generated based upon the
\class{MaildirMessage} instance's delivery date, and the following conversions
take place:

\begin{tableii}{l|l}{textrm}
    {Resulting state}{\class{MaildirMessage} state}
\lineii{R flag}{S flag}
\lineii{O flag}{"cur" subdirectory}
\lineii{D flag}{T flag}
\lineii{F flag}{F flag}
\lineii{A flag}{R flag}
\end{tableii}

When an \class{MMDFMessage} instance is created based upon an \class{MHMessage}
instance, the following conversions take place:

\begin{tableii}{l|l}{textrm}
    {Resulting state}{\class{MHMessage} state}
\lineii{R flag and O flag}{no "unseen" sequence}
\lineii{O flag}{"unseen" sequence}
\lineii{F flag}{"flagged" sequence}
\lineii{A flag}{"replied" sequence}
\end{tableii}

When an \class{MMDFMessage} instance is created based upon a
\class{BabylMessage} instance, the following conversions take place:

\begin{tableii}{l|l}{textrm}
    {Resulting state}{\class{BabylMessage} state}
\lineii{R flag and O flag}{no "unseen" label}
\lineii{O flag}{"unseen" label}
\lineii{D flag}{"deleted" label}
\lineii{A flag}{"answered" label}
\end{tableii}

When an \class{MMDFMessage} instance is created based upon an
\class{mboxMessage} instance, the "From~" line is copied and all flags directly
correspond:

\begin{tableii}{l|l}{textrm}
    {Resulting state}{\class{mboxMessage} state}
\lineii{R flag}{R flag}
\lineii{O flag}{O flag}
\lineii{D flag}{D flag}
\lineii{F flag}{F flag}
\lineii{A flag}{A flag}
\end{tableii}

\subsection{Exceptions}
\label{mailbox-deprecated}

The following exception classes are defined in the \module{mailbox} module:

\begin{classdesc}{Error}{}
The based class for all other module-specific exceptions.
\end{classdesc}

\begin{classdesc}{NoSuchMailboxError}{}
Raised when a mailbox is expected but is not found, such as when instantiating
a \class{Mailbox} subclass with a path that does not exist (and with the
\var{create} parameter set to \code{False}), or when opening a folder that does
not exist.
\end{classdesc}

\begin{classdesc}{NotEmptyErrorError}{}
Raised when a mailbox is not empty but is expected to be, such as when deleting
a folder that contains messages.
\end{classdesc}

\begin{classdesc}{ExternalClashError}{}
Raised when some mailbox-related condition beyond the control of the program
causes it to be unable to proceed, such as when failing to acquire a lock that
another program already holds a lock, or when a uniquely-generated file name
already exists.
\end{classdesc}

\begin{classdesc}{FormatError}{}
Raised when the data in a file cannot be parsed, such as when an \class{MH}
instance attempts to read a corrupted \file{.mh_sequences} file.
\end{classdesc}

\subsection{Deprecated classes and methods}
\label{mailbox-deprecated}

Older versions of the \module{mailbox} module do not support modification of
mailboxes, such as adding or removing message, and do not provide classes to
represent format-specific message properties. For backward compatibility, the
older mailbox classes are still available, but the newer classes should be used
in preference to them.

Older mailbox objects support only iteration and provide a single public
method:

\begin{methoddesc}{next}{}
Return the next message in the mailbox, created with the optional \var{factory}
argument passed into the mailbox object's constructor. By default this is an
\class{rfc822.Message} object (see the \refmodule{rfc822} module).  Depending
on the mailbox implementation the \var{fp} attribute of this object may be a
true file object or a class instance simulating a file object, taking care of
things like message boundaries if multiple mail messages are contained in a
single file, etc.  If no more messages are available, this method returns
\code{None}.
\end{methoddesc}

Most of the older mailbox classes have names that differ from the current
mailbox class names, except for \class{Maildir}. For this reason, the new
\class{Maildir} class defines a \method{next()} method and its constructor
differs slightly from those of the other new mailbox classes.

The older mailbox classes whose names are not the same as their newer
counterparts are as follows:

\begin{classdesc}{UnixMailbox}{fp\optional{, factory}}
Access to a classic \UNIX-style mailbox, where all messages are
contained in a single file and separated by \samp{From }
(a.k.a.\ \samp{From_}) lines.  The file object \var{fp} points to the
mailbox file.  The optional \var{factory} parameter is a callable that
should create new message objects.  \var{factory} is called with one
argument, \var{fp} by the \method{next()} method of the mailbox
object.  The default is the \class{rfc822.Message} class (see the
\refmodule{rfc822} module -- and the note below).

\begin{notice}
  For reasons of this module's internal implementation, you will
  probably want to open the \var{fp} object in binary mode.  This is
  especially important on Windows.
\end{notice}

For maximum portability, messages in a \UNIX-style mailbox are
separated by any line that begins exactly with the string \code{'From
'} (note the trailing space) if preceded by exactly two newlines.
Because of the wide-range of variations in practice, nothing else on
the From_ line should be considered.  However, the current
implementation doesn't check for the leading two newlines.  This is
usually fine for most applications.

The \class{UnixMailbox} class implements a more strict version of
From_ line checking, using a regular expression that usually correctly
matched From_ delimiters.  It considers delimiter line to be separated
by \samp{From \var{name} \var{time}} lines.  For maximum portability,
use the \class{PortableUnixMailbox} class instead.  This class is
identical to \class{UnixMailbox} except that individual messages are
separated by only \samp{From } lines.

For more information, see
\citetitle[http://home.netscape.com/eng/mozilla/2.0/relnotes/demo/content-length.html]{Configuring
Netscape Mail on \UNIX: Why the Content-Length Format is Bad}.
\end{classdesc}

\begin{classdesc}{PortableUnixMailbox}{fp\optional{, factory}}
A less-strict version of \class{UnixMailbox}, which considers only the
\samp{From } at the beginning of the line separating messages.  The
``\var{name} \var{time}'' portion of the From line is ignored, to
protect against some variations that are observed in practice.  This
works since lines in the message which begin with \code{'From '} are
quoted by mail handling software at delivery-time.
\end{classdesc}

\begin{classdesc}{MmdfMailbox}{fp\optional{, factory}}
Access an MMDF-style mailbox, where all messages are contained
in a single file and separated by lines consisting of 4 control-A
characters.  The file object \var{fp} points to the mailbox file.
Optional \var{factory} is as with the \class{UnixMailbox} class.
\end{classdesc}

\begin{classdesc}{MHMailbox}{dirname\optional{, factory}}
Access an MH mailbox, a directory with each message in a separate
file with a numeric name.
The name of the mailbox directory is passed in \var{dirname}.
\var{factory} is as with the \class{UnixMailbox} class.
\end{classdesc}

\begin{classdesc}{BabylMailbox}{fp\optional{, factory}}
Access a Babyl mailbox, which is similar to an MMDF mailbox.  In
Babyl format, each message has two sets of headers, the
\emph{original} headers and the \emph{visible} headers.  The original
headers appear before a line containing only \code{'*** EOOH ***'}
(End-Of-Original-Headers) and the visible headers appear after the
\code{EOOH} line.  Babyl-compliant mail readers will show you only the
visible headers, and \class{BabylMailbox} objects will return messages
containing only the visible headers.  You'll have to do your own
parsing of the mailbox file to get at the original headers.  Mail
messages start with the EOOH line and end with a line containing only
\code{'\e{}037\e{}014'}.  \var{factory} is as with the
\class{UnixMailbox} class.
\end{classdesc}

If you wish to use the older mailbox classes with the \module{email} module
rather than the deprecated \module{rfc822} module, you can do so as follows:

\begin{verbatim}
import email
import email.Errors
import mailbox

def msgfactory(fp):
    try:
        return email.message_from_file(fp)
    except email.Errors.MessageParseError:
        # Don't return None since that will
        # stop the mailbox iterator
        return ''

mbox = mailbox.UnixMailbox(fp, msgfactory)
\end{verbatim}

Alternatively, if you know your mailbox contains only well-formed MIME
messages, you can simplify this to:

\begin{verbatim}
import email
import mailbox

mbox = mailbox.UnixMailbox(fp, email.message_from_file)
\end{verbatim}

\subsection{Examples}
\label{mailbox-examples}

A simple example of printing the subjects of all messages in a mailbox that
seem interesting:

\begin{verbatim}
import mailbox
for message in mailbox.mbox('~/mbox'):
    subject = message['subject']       # Could possibly be None.
    if subject and 'python' in subject.lower():
        print subject
\end{verbatim}

To copy all mail from a Babyl mailbox to an MH mailbox, converting all
of the format-specific information that can be converted:

\begin{verbatim}
import mailbox
destination = mailbox.MH('~/Mail')
destination.lock()
for message in mailbox.Babyl('~/RMAIL'):
    destination.add(MHMessage(message))
destination.flush()
destination.unlock()
\end{verbatim}

This example sorts mail from several mailing lists into different
mailboxes, being careful to avoid mail corruption due to concurrent
modification by other programs, mail loss due to interruption of the
program, or premature termination due to malformed messages in the
mailbox:

\begin{verbatim}
import mailbox
import email.Errors

list_names = ('python-list', 'python-dev', 'python-bugs')

boxes = dict((name, mailbox.mbox('~/email/%s' % name)) for name in list_names)
inbox = mailbox.Maildir('~/Maildir', factory=None)

for key in inbox.iterkeys():
    try:
        message = inbox[key]
    except email.Errors.MessageParseError:
        continue                # The message is malformed. Just leave it.

    for name in list_names:
        list_id = message['list-id']
        if list_id and name in list_id:
            # Get mailbox to use
            box = boxes[name]

            # Write copy to disk before removing original.
            # If there's a crash, you might duplicate a message, but
            # that's better than losing a message completely.
            box.lock()
            box.add(message)
            box.flush()         
            box.unlock()

            # Remove original message
            inbox.lock()
            inbox.discard(key)
            inbox.flush()
            inbox.unlock()
            break               # Found destination, so stop looking.

for box in boxes.itervalues():
    box.close()
\end{verbatim}

\section{Standard Module \sectcode{mimify}}
\stmodindex{mimify}
\renewcommand{\indexsubitem}{(in module mimify)}

The mimify module defines two functions to convert mail messages to
and from MIME format.  The mail message can be either a simple message
or a so-called multipart message.  Each part is treated separately.
Mimifying (a part of) a message entails encoding the message as
quoted-printable if it contains any characters that cannot be
represented using 7-bit ASCII.  Unmimifying (a part of) a message
entails undoing the quoted-printable encoding.  Mimify and unmimify
are especially useful when a message has to be edited before being
sent.  Typical use would be:

\begin{verbatim}
unmimify message
edit message
mimify message
send message
\end{verbatim}

The modules defines the following user-callable functions and
user-settable variables:

\begin{funcdesc}{mimify}{infile, outfile}
Copy the message in \var{infile} to \var{outfile}, converting parts to
quoted-printable and adding MIME mail headers when necessary.
\var{infile} and \var{outfile} can be file objects (actually, any
object that has a \code{readline} method (for \var{infile}) or a
\code{write} method (for \var{outfile})) or strings naming the files.
If \var{infile} and \var{outfile} are both strings, they may have the
same value.
\end{funcdesc}

\begin{funcdesc}{unmimify}{infile, outfile, decode_base64 = 0} 
Copy the message in \var{infile} to \var{outfile}, decoding all
quoted-printable parts.  \var{infile} and \var{outfile} can be file
objects (actually, any object that has a \code{readline} method (for
\var{infile}) or a \code{write} method (for \var{outfile})) or strings
naming the files.  If \var{infile} and \var{outfile} are both strings,
they may have the same value.
If the \var{decode_base64} argument is provided and tests true, any
parts that are coded in the base64 encoding are decoded as well.
\end{funcdesc}

\begin{datadesc}{MAXLEN}
By default, a part will be encoded as quoted-printable when it
contains any non-ASCII characters (i.e., characters with the 8th bit
set), or if there are any lines longer than \code{MAXLEN} characters
(default value 200).  
\end{datadesc}

\begin{datadesc}{CHARSET}
When not specified in the mail headers, a character set must be filled
in.  The string used is stored in \code{CHARSET}, and the default
value is ISO-8859-1 (also known as Latin1 (latin-one)).
\end{datadesc}

This module can also be used from the command line.  Usage is as
follows:
\begin{verbatim}
mimify.py -e [-l length] [infile [outfile]]
mimify.py -d [-b] [infile [outfile]]
\end{verbatim}
to encode (mimify) and decode (unmimify) respectively.  \var{infile}
defaults to standard input, \var{outfile} defaults to standard output.
The same file can be specified for input and output.

If the \code{-l} option is given when encoding, if there are any lines
longer than the specified \var{length}, the containing part will be
encoded.

If the \code{-b} option is given when decoding, any base64 parts will
be decoded as well.


\section{Standard Module \sectcode{BaseHTTPServer}}
\label{module-BaseHTTPServer}
\stmodindex{BaseHTTPServer}

\indexii{WWW}{server}
\indexii{HTTP}{protocol}
\index{URL}
\index{httpd}


This module defines two classes for implementing HTTP servers
(web servers). Usually, this module isn't used directly, but is used
as a basis for building functioning web servers. See the
\module{SimpleHTTPServer} and \module{CGIHTTPServer} modules.
\refstmodindex{SimpleHTTPServer}
\refstmodindex{CGIHTTPServer}

The first class, \class{HTTPServer}, is a
\class{SocketServer.TCPServer} subclass. It creates and listens at the
web socket, dispatching the requests to a handler. Code to create and
run the server looks like this:

\begin{verbatim}
def run(server_class=BaseHTTPServer.HTTPServer,
        handler_class=BaseHTTPServer.BaseHTTPRequestHandler):
  server_address = ('', 8000)
  httpd = server_class(server_address, handler_class)
  httpd.serve_forever()
\end{verbatim}

The \class{HTTPServer} class builds on the \class{TCPServer} class by
storing the server address as instance
variables named \member{server_name} and \member{server_port}. The
server is accessible by the handler, typically through the handler's
\member{server} instance variable.

The module's second class, \class{BaseHTTPRequestHandler}, is used
to handle the HTTP requests that arrive at the server. By itself,
it cannot respond to any actual HTTP requests; it must be subclassed
to handle each request method (e.g. GET or POST).
\class{BaseHTTPRequestHandler} provides a number of class and instance
variables, and methods for use by subclasses.

The handler will parse the request and the headers, then call a
method specific to the request type. The method name is constructed
from the request. For example, for the request \samp{SPAM}, the
\method{do_SPAM()} method will be called with no arguments. All of
the relevant information is stored into instance variables of the
handler.

\setindexsubitem{(BaseHTTPRequestHandler attribute)}

\class{BaseHTTPRequestHandler} has the following instance variables:

\begin{datadesc}{client_address}
Contains a tuple of the form \code{(\var{host}, \var{port})} referring
to the client's address.
\end{datadesc}

\begin{datadesc}{command}
Contains the command (request type). For example, \code{'GET'}.
\end{datadesc}

\begin{datadesc}{path}
Contains the request path.
\end{datadesc}

\begin{datadesc}{request_version}
Contains the version string from the request. For example,
\code{'HTTP/1.0'}.
\end{datadesc}

\begin{datadesc}{headers}
Holds an instance of the class specified by the \member{MessageClass}
class variable. This instance parses and manages the headers in
the HTTP request.
\end{datadesc}

\begin{datadesc}{rfile}
Contains an input stream, positioned at the start of the optional
input data.
\end{datadesc}

\begin{datadesc}{wfile}
Contains the output stream for writing a response back to the client.
Proper adherance to the HTTP protocol must be used when writing
to this stream.
\end{datadesc}

\setindexsubitem{(BaseHTTPRequestHandler attribute)}

\code{BaseHTTPRequestHandler} has the following class variables:

\begin{datadesc}{server_version}
Specifies the server software version.  You may want to override
this.
The format is multiple whitespace-separated strings,
where each string is of the form name[/version].
For example, \code{'BaseHTTP/0.2'}.
\end{datadesc}

\begin{datadesc}{sys_version}
Contains the Python system version, in a form usable by the
\member{version_string} method and the \member{server_version} class
variable. For example, \code{'Python/1.4'}.
\end{datadesc}

\begin{datadesc}{error_message_format}
Specifies a format string for building an error response to the
client. It uses parenthesized, keyed format specifiers, so the
format operand must be a dictionary. The \var{code} key should
be an integer, specifing the numeric HTTP error code value.
\var{message} should be a string containing a (detailed) error
message of what occurred, and \var{explain} should be an
explanation of the error code number. Default \var{message}
and \var{explain} values can found in the \var{responses}
class variable.
\end{datadesc}

\begin{datadesc}{protocol_version}
This specifies the HTTP protocol version used in responses.
Typically, this should not be overridden. Defaults to
\code{'HTTP/1.0'}.
\end{datadesc}

\begin{datadesc}{MessageClass}
Specifies a \class{rfc822.Message}-like class to parse HTTP
headers. Typically, this is not overridden, and it defaults to
\class{mimetools.Message}.
\withsubitem{(in module mimetools)}{\ttindex{Message}}
\end{datadesc}

\begin{datadesc}{responses}
This variable contains a mapping of error code integers to two-element
tuples containing a short and long message. For example,
\code{\{\var{code}: (\var{shortmessage}, \var{longmessage})\}}. The
\var{shortmessage} is usually used as the \var{message} key in an
error response, and \var{longmessage} as the \var{explain} key
(see the \member{error_message_format} class variable).
\end{datadesc}

\setindexsubitem{(BaseHTTPRequestHandler method)}

A \class{BaseHTTPRequestHandler} instance has the following methods:

\begin{funcdesc}{handle}{}
Overrides the superclass' \method{handle()} method to provide the
specific handler behavior. This method will parse and dispatch
the request to the appropriate \code{do_*()} method.
\end{funcdesc}

\begin{funcdesc}{send_error}{code\optional{, message}}
Sends and logs a complete error reply to the client. The numeric
\var{code} specifies the HTTP error code, with \var{message} as
optional, more specific text. A complete set of headers is sent,
followed by text composed using the \member{error_message_format}
class variable.
\end{funcdesc}

\begin{funcdesc}{send_response}{code\optional{, message}}
Sends a response header and logs the accepted request. The HTTP
response line is sent, followed by \emph{Server} and \emph{Date}
headers. The values for these two headers are picked up from the
\method{version_string()} and \method{date_time_string()} methods,
respectively.
\end{funcdesc}

\begin{funcdesc}{send_header}{keyword, value}
Writes a specific MIME header to the output stream. \var{keyword}
should specify the header keyword, with \var{value} specifying
its value.
\end{funcdesc}

\begin{funcdesc}{end_headers}{}
Sends a blank line, indicating the end of the MIME headers in
the response.
\end{funcdesc}

\begin{funcdesc}{log_request}{\optional{code\optional{, size}}}
Logs an accepted (successful) request. \var{code} should specify
the numeric HTTP code associated with the response. If a size of
the response is available, then it should be passed as the
\var{size} parameter.
\end{funcdesc}

\begin{funcdesc}{log_error}{...}
Logs an error when a request cannot be fulfilled. By default,
it passes the message to \method{log_message()}, so it takes the
same arguments (\var{format} and additional values).
\end{funcdesc}

\begin{funcdesc}{log_message}{format, ...}
Logs an arbitrary message to \code{sys.stderr}. This is typically
overridden to create custom error logging mechanisms. The
\var{format} argument is a standard printf-style format string,
where the additional arguments to \method{log_message()} are applied
as inputs to the formatting. The client address and current date
and time are prefixed to every message logged.
\end{funcdesc}

\begin{funcdesc}{version_string}{}
Returns the server software's version string. This is a combination
of the \member{server_version} and \member{sys_version} class variables.
\end{funcdesc}

\begin{funcdesc}{date_time_string}{}
Returns the current date and time, formatted for a message header.
\end{funcdesc}

\begin{funcdesc}{log_data_time_string}{}
Returns the current date and time, formatted for logging.
\end{funcdesc}

\begin{funcdesc}{address_string}{}
Returns the client address, formatted for logging. A name lookup
is performed on the client's IP address.
\end{funcdesc}


\chapter{Restricted Execution}

In general, executing Python programs have complete access to the
underlying operating system through the various functions and classes
contained in Python's modules.  For example, a Python program can open
any file\footnote{Provided the underlying OS gives you permission!}
for reading and writing by using the
\code{open()} built-in function.  This is exactly what you want for
most applications.

There is a class of applications for which this ``openness'' is
inappropriate.  Imagine a web browser that accepts ``applets'', snippets of
Python code, from anywhere on the Internet for execution on the local
system.  Since the originator of the code is unknown, it is obvious that it
cannot be trusted with the full resources of the local machine.

\emph{Restricted execution} is the basic Python framework that allows
for the segregation of trusted and untrusted code.  It is based on the
notion that trusted Python code (a \emph{supervisor}) can create a
``padded cell' (or environment) of limited permissions, and run the
untrusted code within this cell.  The untrusted code cannot break out
of its cell, and can only interact with sensitive system resources
through interfaces defined, and managed by the trusted code.  The term
``restricted execution'' is favored over the term ``safe-Python''
since true safety is hard to define, and is determined by the way the
restricted environment is created.  Note that the restricted
environments can be nested, with inner cells creating subcells of
lesser, but never greater, privledge.

An interesting aspect of Python's restricted execution model is that
the attributes presented to untrusted code usually have the same names
as those presented to trusted code.  Therefore no special interfaces
need to be learned to write code designed to run in a restricted
environment.  And because the exact nature of the padded cell is
determined by the supervisor, different restrictions can be imposed,
depending on the application.  For example, it might be deemed
``safe'' for untrusted code to read any file within a specified
directory, but never to write a file.  In this case, the supervisor
may redefine the built-in
\code{open()} function so that it raises an exception whenever the
\var{mode} parameter is \code{'w'}.  It might also perform a
\code{chroot()}-like operation on the \var{filename} parameter, such
that root is always relative to some safe ``sandbox'' area of the
filesystem.  In this case, the untrusted code would still see an
\code{open()} function in its \code{__builtin__} module, with the same
calling interface.  The semantics would be identical too, with
\code{IOError}s being raised when the supervisor determined that an
unallowable parameter is being used.

Two modules provide the framework for setting up restricted execution
environments:

\begin{description}

\item[rexec]
--- Basic restricted execution framework.

\item[Bastion]
--- Providing restricted access to objects.

\end{description}

\section{\module{rexec} ---
         Restricted execution framework}

\declaremodule{standard}{rexec}
\modulesynopsis{Basic restricted execution framework.}


This module contains the \class{RExec} class, which supports
\method{r_eval()}, \method{r_execfile()}, \method{r_exec()}, and
\method{r_import()} methods, which are restricted versions of the standard
Python functions \method{eval()}, \method{execfile()} and
the \keyword{exec} and \keyword{import} statements.
Code executed in this restricted environment will
only have access to modules and functions that are deemed safe; you
can subclass \class{RExec} to add or remove capabilities as desired.

\strong{Warning:}
While the \module{rexec} module is designed to perform as described
below, it does have a few known vulnerabilities which could be
exploited by carefully written code.  Thus it should not be relied
upon in situations requiring ``production ready'' security.  In such
situations, execution via sub-processes or very careful ``cleansing''
of both code and data to be processed may be necessary.
Alternatively, help in patching known \module{rexec} vulnerabilities
would be welcomed.

\emph{Note:} The \class{RExec} class can prevent code from performing
unsafe operations like reading or writing disk files, or using TCP/IP
sockets.  However, it does not protect against code using extremely
large amounts of memory or CPU time.  

\begin{classdesc}{RExec}{\optional{hooks\optional{, verbose}}}
Returns an instance of the \class{RExec} class.  

\var{hooks} is an instance of the \class{RHooks} class or a subclass of it.
If it is omitted or \code{None}, the default \class{RHooks} class is
instantiated.
Whenever the \module{rexec} module searches for a module (even a
built-in one) or reads a module's code, it doesn't actually go out to
the file system itself.  Rather, it calls methods of an \class{RHooks}
instance that was passed to or created by its constructor.  (Actually,
the \class{RExec} object doesn't make these calls --- they are made by
a module loader object that's part of the \class{RExec} object.  This
allows another level of flexibility, e.g. using packages.)

By providing an alternate \class{RHooks} object, we can control the
file system accesses made to import a module, without changing the
actual algorithm that controls the order in which those accesses are
made.  For instance, we could substitute an \class{RHooks} object that
passes all filesystem requests to a file server elsewhere, via some
RPC mechanism such as ILU.  Grail's applet loader uses this to support
importing applets from a URL for a directory.

If \var{verbose} is true, additional debugging output may be sent to
standard output.
\end{classdesc}

It is important to be aware that code running in a restricted
environment can still call the \function{sys.exit()} function.  To
disallow restricted code from exiting the interpreter, always protect
calls that cause restricted code to run with a
\keyword{try}/\keyword{except} statement that catches the
\exception{SystemExit} exception.  Removing the \function{sys.exit()}
function from the restricted environment is not sufficient --- the
restricted code could still use \code{raise SystemExit}.  Removing
\exception{SystemExit} is not a reasonable option; some library code
makes use of this and would break were it not available.


\begin{seealso}
  \seetitle[http://grail.sourceforge.net/]{Grail Home Page}{Grail is a
            Web browser written entirely in Python.  It uses the
            \module{rexec} module as a foundation for supporting
            Python applets, and can be used as an example usage of
            this module.}
\end{seealso}


\subsection{RExec Objects \label{rexec-objects}}

\class{RExec} instances support the following methods:

\begin{methoddesc}{r_eval}{code}
\var{code} must either be a string containing a Python expression, or
a compiled code object, which will be evaluated in the restricted
environment's \module{__main__} module.  The value of the expression or
code object will be returned.
\end{methoddesc}

\begin{methoddesc}{r_exec}{code}
\var{code} must either be a string containing one or more lines of
Python code, or a compiled code object, which will be executed in the
restricted environment's \module{__main__} module.
\end{methoddesc}

\begin{methoddesc}{r_execfile}{filename}
Execute the Python code contained in the file \var{filename} in the
restricted environment's \module{__main__} module.
\end{methoddesc}

Methods whose names begin with \samp{s_} are similar to the functions
beginning with \samp{r_}, but the code will be granted access to
restricted versions of the standard I/O streams \code{sys.stdin},
\code{sys.stderr}, and \code{sys.stdout}.

\begin{methoddesc}{s_eval}{code}
\var{code} must be a string containing a Python expression, which will
be evaluated in the restricted environment.  
\end{methoddesc}

\begin{methoddesc}{s_exec}{code}
\var{code} must be a string containing one or more lines of Python code,
which will be executed in the restricted environment.  
\end{methoddesc}

\begin{methoddesc}{s_execfile}{code}
Execute the Python code contained in the file \var{filename} in the
restricted environment.
\end{methoddesc}

\class{RExec} objects must also support various methods which will be
implicitly called by code executing in the restricted environment.
Overriding these methods in a subclass is used to change the policies
enforced by a restricted environment.

\begin{methoddesc}{r_import}{modulename\optional{, globals\optional{,
                             locals\optional{, fromlist}}}}
Import the module \var{modulename}, raising an \exception{ImportError}
exception if the module is considered unsafe.
\end{methoddesc}

\begin{methoddesc}{r_open}{filename\optional{, mode\optional{, bufsize}}}
Method called when \function{open()} is called in the restricted
environment.  The arguments are identical to those of \function{open()},
and a file object (or a class instance compatible with file objects)
should be returned.  \class{RExec}'s default behaviour is allow opening
any file for reading, but forbidding any attempt to write a file.  See
the example below for an implementation of a less restrictive
\method{r_open()}.
\end{methoddesc}

\begin{methoddesc}{r_reload}{module}
Reload the module object \var{module}, re-parsing and re-initializing it.  
\end{methoddesc}

\begin{methoddesc}{r_unload}{module}
Unload the module object \var{module} (i.e., remove it from the
restricted environment's \code{sys.modules} dictionary).
\end{methoddesc}

And their equivalents with access to restricted standard I/O streams:

\begin{methoddesc}{s_import}{modulename\optional{, globals\optional{,
                             locals\optional{, fromlist}}}}
Import the module \var{modulename}, raising an \exception{ImportError}
exception if the module is considered unsafe.
\end{methoddesc}

\begin{methoddesc}{s_reload}{module}
Reload the module object \var{module}, re-parsing and re-initializing it.  
\end{methoddesc}

\begin{methoddesc}{s_unload}{module}
Unload the module object \var{module}.   
% XXX what are the semantics of this?  
\end{methoddesc}


\subsection{Defining restricted environments \label{rexec-extension}}

The \class{RExec} class has the following class attributes, which are
used by the \method{__init__()} method.  Changing them on an existing
instance won't have any effect; instead, create a subclass of
\class{RExec} and assign them new values in the class definition.
Instances of the new class will then use those new values.  All these
attributes are tuples of strings.

\begin{memberdesc}{nok_builtin_names}
Contains the names of built-in functions which will \emph{not} be
available to programs running in the restricted environment.  The
value for \class{RExec} is \code{('open', 'reload', '__import__')}.
(This gives the exceptions, because by far the majority of built-in
functions are harmless.  A subclass that wants to override this
variable should probably start with the value from the base class and
concatenate additional forbidden functions --- when new dangerous
built-in functions are added to Python, they will also be added to
this module.)
\end{memberdesc}

\begin{memberdesc}{ok_builtin_modules}
Contains the names of built-in modules which can be safely imported.
The value for \class{RExec} is \code{('audioop', 'array', 'binascii',
'cmath', 'errno', 'imageop', 'marshal', 'math', 'md5', 'operator',
'parser', 'regex', 'rotor', 'select', 'strop', 'struct', 'time')}.  A
similar remark about overriding this variable applies --- use the
value from the base class as a starting point.
\end{memberdesc}

\begin{memberdesc}{ok_path}
Contains the directories which will be searched when an \keyword{import}
is performed in the restricted environment.  
The value for \class{RExec} is the same as \code{sys.path} (at the time
the module is loaded) for unrestricted code.
\end{memberdesc}

\begin{memberdesc}{ok_posix_names}
% Should this be called ok_os_names?
Contains the names of the functions in the \refmodule{os} module which will be
available to programs running in the restricted environment.  The
value for \class{RExec} is \code{('error', 'fstat', 'listdir',
'lstat', 'readlink', 'stat', 'times', 'uname', 'getpid', 'getppid',
'getcwd', 'getuid', 'getgid', 'geteuid', 'getegid')}.
\end{memberdesc}

\begin{memberdesc}{ok_sys_names}
Contains the names of the functions and variables in the \refmodule{sys}
module which will be available to programs running in the restricted
environment.  The value for \class{RExec} is \code{('ps1', 'ps2',
'copyright', 'version', 'platform', 'exit', 'maxint')}.
\end{memberdesc}

\begin{memberdesc}{ok_file_types}
Contains the file types from which modules are allowed to be loaded.
Each file type is an integer constant defined in the \refmodule{imp} module.
The meaningful values are \constant{PY_SOURCE}, \constant{PY_COMPILED}, and
\constant{C_EXTENSION}.  The value for \class{RExec} is \code{(C_EXTENSION,
PY_SOURCE)}.  Adding \constant{PY_COMPILED} in subclasses is not recommended;
an attacker could exit the restricted execution mode by putting a forged
byte-compiled file (\file{.pyc}) anywhere in your file system, for example
by writing it to \file{/tmp} or uploading it to the \file{/incoming}
directory of your public FTP server.
\end{memberdesc}


\subsection{An example}

Let us say that we want a slightly more relaxed policy than the
standard \class{RExec} class.  For example, if we're willing to allow
files in \file{/tmp} to be written, we can subclass the \class{RExec}
class:

\begin{verbatim}
class TmpWriterRExec(rexec.RExec):
    def r_open(self, file, mode='r', buf=-1):
        if mode in ('r', 'rb'):
            pass
        elif mode in ('w', 'wb', 'a', 'ab'):
            # check filename : must begin with /tmp/
            if file[:5]!='/tmp/': 
                raise IOError, "can't write outside /tmp"
            elif (string.find(file, '/../') >= 0 or
                 file[:3] == '../' or file[-3:] == '/..'):
                raise IOError, "'..' in filename forbidden"
        else: raise IOError, "Illegal open() mode"
        return open(file, mode, buf)
\end{verbatim}
%
Notice that the above code will occasionally forbid a perfectly valid
filename; for example, code in the restricted environment won't be
able to open a file called \file{/tmp/foo/../bar}.  To fix this, the
\method{r_open()} method would have to simplify the filename to
\file{/tmp/bar}, which would require splitting apart the filename and
performing various operations on it.  In cases where security is at
stake, it may be preferable to write simple code which is sometimes
overly restrictive, instead of more general code that is also more
complex and may harbor a subtle security hole.

\section{\module{Bastion} ---
         Restricting access to objects}

\declaremodule{standard}{Bastion}
\modulesynopsis{Providing restricted access to objects.}
\moduleauthor{Barry Warsaw}{bwarsaw@python.org}


% I'm concerned that the word 'bastion' won't be understood by people
% for whom English is a second language, making the module name
% somewhat mysterious.  Thus, the brief definition... --amk

According to the dictionary, a bastion is ``a fortified area or
position'', or ``something that is considered a stronghold.''  It's a
suitable name for this module, which provides a way to forbid access
to certain attributes of an object.  It must always be used with the
\refmodule{rexec} module, in order to allow restricted-mode programs
access to certain safe attributes of an object, while denying access
to other, unsafe attributes.

% I've punted on the issue of documenting keyword arguments for now.

\begin{funcdesc}{Bastion}{object\optional{, filter\optional{,
                          name\optional{, class}}}}
Protect the object \var{object}, returning a bastion for the
object.  Any attempt to access one of the object's attributes will
have to be approved by the \var{filter} function; if the access is
denied an \exception{AttributeError} exception will be raised.

If present, \var{filter} must be a function that accepts a string
containing an attribute name, and returns true if access to that
attribute will be permitted; if \var{filter} returns false, the access
is denied.  The default filter denies access to any function beginning
with an underscore (\character{_}).  The bastion's string representation
will be \samp{<Bastion for \var{name}>} if a value for
\var{name} is provided; otherwise, \samp{repr(\var{object})} will be
used.

\var{class}, if present, should be a subclass of \class{BastionClass}; 
see the code in \file{bastion.py} for the details.  Overriding the
default \class{BastionClass} will rarely be required.
\end{funcdesc}


\begin{classdesc}{BastionClass}{getfunc, name}
Class which actually implements bastion objects.  This is the default
class used by \function{Bastion()}.  The \var{getfunc} parameter is a
function which returns the value of an attribute which should be
exposed to the restricted execution environment when called with the
name of the attribute as the only parameter.  \var{name} is used to
construct the \function{repr()} of the \class{BastionClass} instance.
\end{classdesc}


\chapter{MULTIMEDIA EXTENSIONS}

The modules described in this chapter implement various algorithms
that are mainly useful for multimedia applications.  They are
available at the discretion of the installation.
			% Multimedia Services
\section{Built-in Module \sectcode{audioop}}
\bimodindex{audioop}

The \code{audioop} module contains some useful operations on sound fragments.
It operates on sound fragments consisting of signed integer samples
8, 16 or 32 bits wide, stored in Python strings.  This is the same
format as used by the \code{al} and \code{sunaudiodev} modules.  All
scalar items are integers, unless specified otherwise.

A few of the more complicated operations only take 16-bit samples,
otherwise the sample size (in bytes) is always a parameter of the operation.

The module defines the following variables and functions:

\renewcommand{\indexsubitem}{(in module audioop)}
\begin{excdesc}{error}
This exception is raised on all errors, such as unknown number of bytes
per sample, etc.
\end{excdesc}

\begin{funcdesc}{add}{fragment1\, fragment2\, width}
Return a fragment which is the addition of the two samples passed as
parameters.  \var{width} is the sample width in bytes, either
\code{1}, \code{2} or \code{4}.  Both fragments should have the same
length.
\end{funcdesc}

\begin{funcdesc}{adpcm2lin}{adpcmfragment\, width\, state}
Decode an Intel/DVI ADPCM coded fragment to a linear fragment.  See
the description of \code{lin2adpcm} for details on ADPCM coding.
Return a tuple \code{(\var{sample}, \var{newstate})} where the sample
has the width specified in \var{width}.
\end{funcdesc}

\begin{funcdesc}{adpcm32lin}{adpcmfragment\, width\, state}
Decode an alternative 3-bit ADPCM code.  See \code{lin2adpcm3} for
details.
\end{funcdesc}

\begin{funcdesc}{avg}{fragment\, width}
Return the average over all samples in the fragment.
\end{funcdesc}

\begin{funcdesc}{avgpp}{fragment\, width}
Return the average peak-peak value over all samples in the fragment.
No filtering is done, so the usefulness of this routine is
questionable.
\end{funcdesc}

\begin{funcdesc}{bias}{fragment\, width\, bias}
Return a fragment that is the original fragment with a bias added to
each sample.
\end{funcdesc}

\begin{funcdesc}{cross}{fragment\, width}
Return the number of zero crossings in the fragment passed as an
argument.
\end{funcdesc}

\begin{funcdesc}{findfactor}{fragment\, reference}
Return a factor \var{F} such that
\code{rms(add(fragment, mul(reference, -F)))} is minimal, i.e.,
return the factor with which you should multiply \var{reference} to
make it match as well as possible to \var{fragment}.  The fragments
should both contain 2-byte samples.

The time taken by this routine is proportional to \code{len(fragment)}. 
\end{funcdesc}

\begin{funcdesc}{findfit}{fragment\, reference}
This routine (which only accepts 2-byte sample fragments)

Try to match \var{reference} as well as possible to a portion of
\var{fragment} (which should be the longer fragment).  This is
(conceptually) done by taking slices out of \var{fragment}, using
\code{findfactor} to compute the best match, and minimizing the
result.  The fragments should both contain 2-byte samples.  Return a
tuple \code{(\var{offset}, \var{factor})} where \var{offset} is the
(integer) offset into \var{fragment} where the optimal match started
and \var{factor} is the (floating-point) factor as per
\code{findfactor}.
\end{funcdesc}

\begin{funcdesc}{findmax}{fragment\, length}
Search \var{fragment} for a slice of length \var{length} samples (not
bytes!)\ with maximum energy, i.e., return \var{i} for which
\code{rms(fragment[i*2:(i+length)*2])} is maximal.  The fragments
should both contain 2-byte samples.

The routine takes time proportional to \code{len(fragment)}.
\end{funcdesc}

\begin{funcdesc}{getsample}{fragment\, width\, index}
Return the value of sample \var{index} from the fragment.
\end{funcdesc}

\begin{funcdesc}{lin2lin}{fragment\, width\, newwidth}
Convert samples between 1-, 2- and 4-byte formats.
\end{funcdesc}

\begin{funcdesc}{lin2adpcm}{fragment\, width\, state}
Convert samples to 4 bit Intel/DVI ADPCM encoding.  ADPCM coding is an
adaptive coding scheme, whereby each 4 bit number is the difference
between one sample and the next, divided by a (varying) step.  The
Intel/DVI ADPCM algorithm has been selected for use by the IMA, so it
may well become a standard.

\code{State} is a tuple containing the state of the coder.  The coder
returns a tuple \code{(\var{adpcmfrag}, \var{newstate})}, and the
\var{newstate} should be passed to the next call of lin2adpcm.  In the
initial call \code{None} can be passed as the state.  \var{adpcmfrag}
is the ADPCM coded fragment packed 2 4-bit values per byte.
\end{funcdesc}

\begin{funcdesc}{lin2adpcm3}{fragment\, width\, state}
This is an alternative ADPCM coder that uses only 3 bits per sample.
It is not compatible with the Intel/DVI ADPCM coder and its output is
not packed (due to laziness on the side of the author).  Its use is
discouraged.
\end{funcdesc}

\begin{funcdesc}{lin2ulaw}{fragment\, width}
Convert samples in the audio fragment to U-LAW encoding and return
this as a Python string.  U-LAW is an audio encoding format whereby
you get a dynamic range of about 14 bits using only 8 bit samples.  It
is used by the Sun audio hardware, among others.
\end{funcdesc}

\begin{funcdesc}{minmax}{fragment\, width}
Return a tuple consisting of the minimum and maximum values of all
samples in the sound fragment.
\end{funcdesc}

\begin{funcdesc}{max}{fragment\, width}
Return the maximum of the {\em absolute value} of all samples in a
fragment.
\end{funcdesc}

\begin{funcdesc}{maxpp}{fragment\, width}
Return the maximum peak-peak value in the sound fragment.
\end{funcdesc}

\begin{funcdesc}{mul}{fragment\, width\, factor}
Return a fragment that has all samples in the original framgent
multiplied by the floating-point value \var{factor}.  Overflow is
silently ignored.
\end{funcdesc}

\begin{funcdesc}{reverse}{fragment\, width}
Reverse the samples in a fragment and returns the modified fragment.
\end{funcdesc}

\begin{funcdesc}{rms}{fragment\, width}
Return the root-mean-square of the fragment, i.e.
\iftexi
the square root of the quotient of the sum of all squared sample value,
divided by the sumber of samples.
\else
% in eqn: sqrt { sum S sub i sup 2  over n }
\begin{displaymath}
\catcode`_=8
\sqrt{\frac{\sum{{S_{i}}^{2}}}{n}}
\end{displaymath}
\fi
This is a measure of the power in an audio signal.
\end{funcdesc}

\begin{funcdesc}{tomono}{fragment\, width\, lfactor\, rfactor} 
Convert a stereo fragment to a mono fragment.  The left channel is
multiplied by \var{lfactor} and the right channel by \var{rfactor}
before adding the two channels to give a mono signal.
\end{funcdesc}

\begin{funcdesc}{tostereo}{fragment\, width\, lfactor\, rfactor}
Generate a stereo fragment from a mono fragment.  Each pair of samples
in the stereo fragment are computed from the mono sample, whereby left
channel samples are multiplied by \var{lfactor} and right channel
samples by \var{rfactor}.
\end{funcdesc}

\begin{funcdesc}{ulaw2lin}{fragment\, width}
Convert sound fragments in ULAW encoding to linearly encoded sound
fragments.  ULAW encoding always uses 8 bits samples, so \var{width}
refers only to the sample width of the output fragment here.
\end{funcdesc}

Note that operations such as \code{mul} or \code{max} make no
distinction between mono and stereo fragments, i.e.\ all samples are
treated equal.  If this is a problem the stereo fragment should be split
into two mono fragments first and recombined later.  Here is an example
of how to do that:
\bcode\begin{verbatim}
def mul_stereo(sample, width, lfactor, rfactor):
    lsample = audioop.tomono(sample, width, 1, 0)
    rsample = audioop.tomono(sample, width, 0, 1)
    lsample = audioop.mul(sample, width, lfactor)
    rsample = audioop.mul(sample, width, rfactor)
    lsample = audioop.tostereo(lsample, width, 1, 0)
    rsample = audioop.tostereo(rsample, width, 0, 1)
    return audioop.add(lsample, rsample, width)
\end{verbatim}\ecode

If you use the ADPCM coder to build network packets and you want your
protocol to be stateless (i.e.\ to be able to tolerate packet loss)
you should not only transmit the data but also the state.  Note that
you should send the \var{initial} state (the one you passed to
\code{lin2adpcm}) along to the decoder, not the final state (as returned by
the coder).  If you want to use \code{struct} to store the state in
binary you can code the first element (the predicted value) in 16 bits
and the second (the delta index) in 8.

The ADPCM coders have never been tried against other ADPCM coders,
only against themselves.  It could well be that I misinterpreted the
standards in which case they will not be interoperable with the
respective standards.

The \code{find...} routines might look a bit funny at first sight.
They are primarily meant to do echo cancellation.  A reasonably
fast way to do this is to pick the most energetic piece of the output
sample, locate that in the input sample and subtract the whole output
sample from the input sample:
\bcode\begin{verbatim}
def echocancel(outputdata, inputdata):
    pos = audioop.findmax(outputdata, 800)    # one tenth second
    out_test = outputdata[pos*2:]
    in_test = inputdata[pos*2:]
    ipos, factor = audioop.findfit(in_test, out_test)
    # Optional (for better cancellation):
    # factor = audioop.findfactor(in_test[ipos*2:ipos*2+len(out_test)], 
    #              out_test)
    prefill = '\0'*(pos+ipos)*2
    postfill = '\0'*(len(inputdata)-len(prefill)-len(outputdata))
    outputdata = prefill + audioop.mul(outputdata,2,-factor) + postfill
    return audioop.add(inputdata, outputdata, 2)
\end{verbatim}\ecode

\section{\module{imageop} ---
         Manipulate raw image data}

\declaremodule{builtin}{imageop}
\modulesynopsis{Manipulate raw image data.}


The \module{imageop} module contains some useful operations on images.
It operates on images consisting of 8 or 32 bit pixels stored in
Python strings.  This is the same format as used by
\function{gl.lrectwrite()} and the \refmodule{imgfile} module.

The module defines the following variables and functions:

\begin{excdesc}{error}
This exception is raised on all errors, such as unknown number of bits
per pixel, etc.
\end{excdesc}


\begin{funcdesc}{crop}{image, psize, width, height, x0, y0, x1, y1}
Return the selected part of \var{image}, which should be
\var{width} by \var{height} in size and consist of pixels of
\var{psize} bytes. \var{x0}, \var{y0}, \var{x1} and \var{y1} are like
the \function{gl.lrectread()} parameters, i.e.\ the boundary is
included in the new image.  The new boundaries need not be inside the
picture.  Pixels that fall outside the old image will have their value
set to zero.  If \var{x0} is bigger than \var{x1} the new image is
mirrored.  The same holds for the y coordinates.
\end{funcdesc}

\begin{funcdesc}{scale}{image, psize, width, height, newwidth, newheight}
Return \var{image} scaled to size \var{newwidth} by \var{newheight}.
No interpolation is done, scaling is done by simple-minded pixel
duplication or removal.  Therefore, computer-generated images or
dithered images will not look nice after scaling.
\end{funcdesc}

\begin{funcdesc}{tovideo}{image, psize, width, height}
Run a vertical low-pass filter over an image.  It does so by computing
each destination pixel as the average of two vertically-aligned source
pixels.  The main use of this routine is to forestall excessive
flicker if the image is displayed on a video device that uses
interlacing, hence the name.
\end{funcdesc}

\begin{funcdesc}{grey2mono}{image, width, height, threshold}
Convert a 8-bit deep greyscale image to a 1-bit deep image by
thresholding all the pixels.  The resulting image is tightly packed and
is probably only useful as an argument to \function{mono2grey()}.
\end{funcdesc}

\begin{funcdesc}{dither2mono}{image, width, height}
Convert an 8-bit greyscale image to a 1-bit monochrome image using a
(simple-minded) dithering algorithm.
\end{funcdesc}

\begin{funcdesc}{mono2grey}{image, width, height, p0, p1}
Convert a 1-bit monochrome image to an 8 bit greyscale or color image.
All pixels that are zero-valued on input get value \var{p0} on output
and all one-value input pixels get value \var{p1} on output.  To
convert a monochrome black-and-white image to greyscale pass the
values \code{0} and \code{255} respectively.
\end{funcdesc}

\begin{funcdesc}{grey2grey4}{image, width, height}
Convert an 8-bit greyscale image to a 4-bit greyscale image without
dithering.
\end{funcdesc}

\begin{funcdesc}{grey2grey2}{image, width, height}
Convert an 8-bit greyscale image to a 2-bit greyscale image without
dithering.
\end{funcdesc}

\begin{funcdesc}{dither2grey2}{image, width, height}
Convert an 8-bit greyscale image to a 2-bit greyscale image with
dithering.  As for \function{dither2mono()}, the dithering algorithm
is currently very simple.
\end{funcdesc}

\begin{funcdesc}{grey42grey}{image, width, height}
Convert a 4-bit greyscale image to an 8-bit greyscale image.
\end{funcdesc}

\begin{funcdesc}{grey22grey}{image, width, height}
Convert a 2-bit greyscale image to an 8-bit greyscale image.
\end{funcdesc}

\begin{datadesc}{backward_compatible}
If set to 0, the functions in this module use a non-backward
compatible way of representing multi-byte pixels on little-endian
systems.  The SGI for which this module was originally written is a
big-endian system, so setting this variable will have no effect.
However, the code wasn't originally intended to run on anything else,
so it made assumptions about byte order which are not universal.
Setting this variable to 0 will cause the byte order to be reversed on
little-endian systems, so that it then is the same as on big-endian
systems.
\end{datadesc}

\section{\module{aifc} ---
         Read and write AIFF and AIFC files}

\declaremodule{standard}{aifc}
\modulesynopsis{Read and write audio files in AIFF or AIFC format.}


This module provides support for reading and writing AIFF and AIFF-C
files.  AIFF is Audio Interchange File Format, a format for storing
digital audio samples in a file.  AIFF-C is a newer version of the
format that includes the ability to compress the audio data.
\index{Audio Interchange File Format}
\index{AIFF}
\index{AIFF-C}

\strong{Caveat:}  Some operations may only work under IRIX; these will
raise \exception{ImportError} when attempting to import the
\module{cl} module, which is only available on IRIX.

Audio files have a number of parameters that describe the audio data.
The sampling rate or frame rate is the number of times per second the
sound is sampled.  The number of channels indicate if the audio is
mono, stereo, or quadro.  Each frame consists of one sample per
channel.  The sample size is the size in bytes of each sample.  Thus a
frame consists of \var{nchannels}*\var{samplesize} bytes, and a
second's worth of audio consists of
\var{nchannels}*\var{samplesize}*\var{framerate} bytes.

For example, CD quality audio has a sample size of two bytes (16
bits), uses two channels (stereo) and has a frame rate of 44,100
frames/second.  This gives a frame size of 4 bytes (2*2), and a
second's worth occupies 2*2*44100 bytes, i.e.\ 176,400 bytes.

Module \module{aifc} defines the following function:

\begin{funcdesc}{open}{file, mode}
Open an AIFF or AIFF-C file and return an object instance with
methods that are described below.  The argument file is either a
string naming a file or a file object.  The mode is either the string
\code{'r'} when the file must be opened for reading, or \code{'w'}
when the file must be opened for writing.  When used for writing, the
file object should be seekable, unless you know ahead of time how many
samples you are going to write in total and use
\method{writeframesraw()} and \method{setnframes()}.
\end{funcdesc}

Objects returned by \function{open()} when a file is opened for
reading have the following methods:

\begin{methoddesc}[aifc]{getnchannels}{}
Return the number of audio channels (1 for mono, 2 for stereo).
\end{methoddesc}

\begin{methoddesc}[aifc]{getsampwidth}{}
Return the size in bytes of individual samples.
\end{methoddesc}

\begin{methoddesc}[aifc]{getframerate}{}
Return the sampling rate (number of audio frames per second).
\end{methoddesc}

\begin{methoddesc}[aifc]{getnframes}{}
Return the number of audio frames in the file.
\end{methoddesc}

\begin{methoddesc}[aifc]{getcomptype}{}
Return a four-character string describing the type of compression used
in the audio file.  For AIFF files, the returned value is
\code{'NONE'}.
\end{methoddesc}

\begin{methoddesc}[aifc]{getcompname}{}
Return a human-readable description of the type of compression used in
the audio file.  For AIFF files, the returned value is \code{'not
compressed'}.
\end{methoddesc}

\begin{methoddesc}[aifc]{getparams}{}
Return a tuple consisting of all of the above values in the above
order.
\end{methoddesc}

\begin{methoddesc}[aifc]{getmarkers}{}
Return a list of markers in the audio file.  A marker consists of a
tuple of three elements.  The first is the mark ID (an integer), the
second is the mark position in frames from the beginning of the data
(an integer), the third is the name of the mark (a string).
\end{methoddesc}

\begin{methoddesc}[aifc]{getmark}{id}
Return the tuple as described in \method{getmarkers()} for the mark
with the given \var{id}.
\end{methoddesc}

\begin{methoddesc}[aifc]{readframes}{nframes}
Read and return the next \var{nframes} frames from the audio file.  The
returned data is a string containing for each frame the uncompressed
samples of all channels.
\end{methoddesc}

\begin{methoddesc}[aifc]{rewind}{}
Rewind the read pointer.  The next \method{readframes()} will start from
the beginning.
\end{methoddesc}

\begin{methoddesc}[aifc]{setpos}{pos}
Seek to the specified frame number.
\end{methoddesc}

\begin{methoddesc}[aifc]{tell}{}
Return the current frame number.
\end{methoddesc}

\begin{methoddesc}[aifc]{close}{}
Close the AIFF file.  After calling this method, the object can no
longer be used.
\end{methoddesc}

Objects returned by \function{open()} when a file is opened for
writing have all the above methods, except for \method{readframes()} and
\method{setpos()}.  In addition the following methods exist.  The
\method{get*()} methods can only be called after the corresponding
\method{set*()} methods have been called.  Before the first
\method{writeframes()} or \method{writeframesraw()}, all parameters
except for the number of frames must be filled in.

\begin{methoddesc}[aifc]{aiff}{}
Create an AIFF file.  The default is that an AIFF-C file is created,
unless the name of the file ends in \code{'.aiff'} in which case the
default is an AIFF file.
\end{methoddesc}

\begin{methoddesc}[aifc]{aifc}{}
Create an AIFF-C file.  The default is that an AIFF-C file is created,
unless the name of the file ends in \code{'.aiff'} in which case the
default is an AIFF file.
\end{methoddesc}

\begin{methoddesc}[aifc]{setnchannels}{nchannels}
Specify the number of channels in the audio file.
\end{methoddesc}

\begin{methoddesc}[aifc]{setsampwidth}{width}
Specify the size in bytes of audio samples.
\end{methoddesc}

\begin{methoddesc}[aifc]{setframerate}{rate}
Specify the sampling frequency in frames per second.
\end{methoddesc}

\begin{methoddesc}[aifc]{setnframes}{nframes}
Specify the number of frames that are to be written to the audio file.
If this parameter is not set, or not set correctly, the file needs to
support seeking.
\end{methoddesc}

\begin{methoddesc}[aifc]{setcomptype}{type, name}
Specify the compression type.  If not specified, the audio data will
not be compressed.  In AIFF files, compression is not possible.  The
name parameter should be a human-readable description of the
compression type, the type parameter should be a four-character
string.  Currently the following compression types are supported:
NONE, ULAW, ALAW, G722.
\index{u-LAW}
\index{A-LAW}
\index{G.722}
\end{methoddesc}

\begin{methoddesc}[aifc]{setparams}{nchannels, sampwidth, framerate, comptype, compname}
Set all the above parameters at once.  The argument is a tuple
consisting of the various parameters.  This means that it is possible
to use the result of a \method{getparams()} call as argument to
\method{setparams()}.
\end{methoddesc}

\begin{methoddesc}[aifc]{setmark}{id, pos, name}
Add a mark with the given id (larger than 0), and the given name at
the given position.  This method can be called at any time before
\method{close()}.
\end{methoddesc}

\begin{methoddesc}[aifc]{tell}{}
Return the current write position in the output file.  Useful in
combination with \method{setmark()}.
\end{methoddesc}

\begin{methoddesc}[aifc]{writeframes}{data}
Write data to the output file.  This method can only be called after
the audio file parameters have been set.
\end{methoddesc}

\begin{methoddesc}[aifc]{writeframesraw}{data}
Like \method{writeframes()}, except that the header of the audio file
is not updated.
\end{methoddesc}

\begin{methoddesc}[aifc]{close}{}
Close the AIFF file.  The header of the file is updated to reflect the
actual size of the audio data. After calling this method, the object
can no longer be used.
\end{methoddesc}

\section{Built-in Module \module{jpeg}}
\label{module-jpeg}
\bimodindex{jpeg}

The module \module{jpeg} provides access to the jpeg compressor and
decompressor written by the Independent JPEG Group%
\index{Independent JPEG Group}%
. JPEG is a (draft?)
standard for compressing pictures.  For details on JPEG or the
Independent JPEG Group software refer to the JPEG standard or the
documentation provided with the software.

The \module{jpeg} module defines an exception and some functions.

\begin{excdesc}{error}
Exception raised by \function{compress()} and \function{decompress()}
in case of errors.
\end{excdesc}

\begin{funcdesc}{compress}{data, w, h, b}
Treat data as a pixmap of width \var{w} and height \var{h}, with
\var{b} bytes per pixel.  The data is in SGI GL order, so the first
pixel is in the lower-left corner. This means that \function{gl.lrectread()}
return data can immediately be passed to \function{compress()}.
Currently only 1 byte and 4 byte pixels are allowed, the former being
treated as greyscale and the latter as RGB color.
\function{compress()} returns a string that contains the compressed
picture, in JFIF\index{JFIF} format.
\end{funcdesc}

\begin{funcdesc}{decompress}{data}
Data is a string containing a picture in JFIF\index{JFIF} format. It
returns a tuple \code{(\var{data}, \var{width}, \var{height},
\var{bytesperpixel})}.  Again, the data is suitable to pass to
\function{gl.lrectwrite()}.
\end{funcdesc}

\begin{funcdesc}{setoption}{name, value}
Set various options.  Subsequent \function{compress()} and
\function{decompress()} calls will use these options.  The following
options are available:

\begin{tableii}{l|p{3in}}{code}{Option}{Effect}
  \lineii{'forcegray'}{%
    Force output to be grayscale, even if input is RGB.}
  \lineii{'quality'}{%
    Set the quality of the compressed image to a value between
    \code{0} and \code{100} (default is \code{75}).  This only affects
    compression.}
  \lineii{'optimize'}{%
    Perform Huffman table optimization.  Takes longer, but results in
    smaller compressed image.  This only affects compression.}
  \lineii{'smooth'}{%
    Perform inter-block smoothing on uncompressed image.  Only useful
    for low-quality images.  This only affects decompression.}
\end{tableii}
\end{funcdesc}

\section{Built-in Module \module{rgbimg}}
\label{module-rgbimg}
\bimodindex{rgbimg}

The \module{rgbimg} module allows Python programs to access SGI imglib image
files (also known as \file{.rgb} files).  The module is far from
complete, but is provided anyway since the functionality that there is
is enough in some cases.  Currently, colormap files are not supported.

The module defines the following variables and functions:

\begin{excdesc}{error}
This exception is raised on all errors, such as unsupported file type, etc.
\end{excdesc}

\begin{funcdesc}{sizeofimage}{file}
This function returns a tuple \code{(\var{x}, \var{y})} where
\var{x} and \var{y} are the size of the image in pixels.
Only 4 byte RGBA pixels, 3 byte RGB pixels, and 1 byte greyscale pixels
are currently supported.
\end{funcdesc}

\begin{funcdesc}{longimagedata}{file}
This function reads and decodes the image on the specified file, and
returns it as a Python string. The string has 4 byte RGBA pixels.
The bottom left pixel is the first in
the string. This format is suitable to pass to \code{gl.lrectwrite},
for instance.
\end{funcdesc}

\begin{funcdesc}{longstoimage}{data, x, y, z, file}
This function writes the RGBA data in \var{data} to image
file \var{file}. \var{x} and \var{y} give the size of the image.
\var{z} is 1 if the saved image should be 1 byte greyscale, 3 if the
saved image should be 3 byte RGB data, or 4 if the saved images should
be 4 byte RGBA data.  The input data always contains 4 bytes per pixel.
These are the formats returned by \code{gl.lrectread}.
\end{funcdesc}

\begin{funcdesc}{ttob}{flag}
This function sets a global flag which defines whether the scan lines
of the image are read or written from bottom to top (flag is zero,
compatible with SGI GL) or from top to bottom(flag is one,
compatible with X)\@.  The default is zero.
\end{funcdesc}

\section{\module{imghdr} ---
         Determine the type of an image}

\declaremodule{standard}{imghdr}
\modulesynopsis{Determine the type of image contained in a file or
                byte stream.}


The \module{imghdr} module determines the type of image contained in a
file or byte stream.

The \module{imghdr} module defines the following function:


\begin{funcdesc}{what}{filename\optional{, h}}
Tests the image data contained in the file named by \var{filename},
and returns a string describing the image type.  If optional \var{h}
is provided, the \var{filename} is ignored and \var{h} is assumed to
contain the byte stream to test.
\end{funcdesc}

The following image types are recognized, as listed below with the
return value from \function{what()}:

\begin{tableii}{l|l}{code}{Value}{Image format}
  \lineii{'rgb'}{SGI ImgLib Files}
  \lineii{'gif'}{GIF 87a and 89a Files}
  \lineii{'pbm'}{Portable Bitmap Files}
  \lineii{'pgm'}{Portable Graymap Files}
  \lineii{'ppm'}{Portable Pixmap Files}
  \lineii{'tiff'}{TIFF Files}
  \lineii{'rast'}{Sun Raster Files}
  \lineii{'xbm'}{X Bitmap Files}
  \lineii{'jpeg'}{JPEG data in JFIF format}
  \lineii{'bmp'}{BMP files}
  \lineii{'png'}{Portable Network Graphics}
\end{tableii}

You can extend the list of file types \module{imghdr} can recognize by
appending to this variable:

\begin{datadesc}{tests}
A list of functions performing the individual tests.  Each function
takes two arguments: the byte-stream and an open file-like object.
When \function{what()} is called with a byte-stream, the file-like
object will be \code{None}.

The test function should return a string describing the image type if
the test succeeded, or \code{None} if it failed.
\end{datadesc}

Example:

\begin{verbatim}
>>> import imghdr
>>> imghdr.what('/tmp/bass.gif')
'gif'
\end{verbatim}


\chapter{Cryptographic Services}
\index{cryptography}

The modules described in this chapter implement various algorithms of
a cryptographic nature.  They are available at the discretion of the
installation.  Here's an overview:

\begin{description}

\item[md5]
--- RSA's MD5 message digest algorithm.

\item[mpz]
--- Interface to the GNU MP library for arbitrary precision arithmetic.

\item[rotor]
--- Enigma-like encryption and decryption.

\end{description}

Hardcore cypherpunks will probably find the cryptographic modules
written by Andrew Kuchling of further interest; the package adds
built-in modules for DES and IDEA encryption, provides a Python module
for reading and decrypting PGP files, and then some.  These modules
are not distributed with Python but available separately.  See the URL
\url{http://www.magnet.com/\~amk/python/pct.html} or send email to
\email{amk@magnet.com} for more information.
\index{PGP}
\indexii{DES}{cipher}
\indexii{IDEA}{cipher}
\index{cryptography}
		% Cryptographic Services
\section{Built-in Module \module{md5}}
\declaremodule{builtin}{md5}

\modulesynopsis{RSA's MD5 message digest algorithm.}


This module implements the interface to RSA's MD5 message digest
\index{message digest, MD5}
algorithm (see also Internet \rfc{1321}).  Its use is quite
straightforward:\ use the \function{new()} to create an md5 object.
You can now feed this object with arbitrary strings using the
\method{update()} method, and at any point you can ask it for the
\dfn{digest} (a strong kind of 128-bit checksum,
a.k.a. ``fingerprint'') of the contatenation of the strings fed to it
so far using the \method{digest()} method.
\index{checksum!MD5}

For example, to obtain the digest of the string \code{'Nobody inspects
the spammish repetition'}:

\begin{verbatim}
>>> import md5
>>> m = md5.new()
>>> m.update("Nobody inspects")
>>> m.update(" the spammish repetition")
>>> m.digest()
'\273d\234\203\335\036\245\311\331\336\311\241\215\360\377\351'
\end{verbatim}

More condensed:

\begin{verbatim}
>>> md5.new("Nobody inspects the spammish repetition").digest()
'\273d\234\203\335\036\245\311\331\336\311\241\215\360\377\351'
\end{verbatim}

\begin{funcdesc}{new}{\optional{arg}}
Return a new md5 object.  If \var{arg} is present, the method call
\code{update(\var{arg})} is made.
\end{funcdesc}

\begin{funcdesc}{md5}{\optional{arg}}
For backward compatibility reasons, this is an alternative name for the
\function{new()} function.
\end{funcdesc}

An md5 object has the following methods:

\begin{methoddesc}[md5]{update}{arg}
Update the md5 object with the string \var{arg}.  Repeated calls are
equivalent to a single call with the concatenation of all the
arguments, i.e.\ \code{m.update(a); m.update(b)} is equivalent to
\code{m.update(a+b)}.
\end{methoddesc}

\begin{methoddesc}[md5]{digest}{}
Return the digest of the strings passed to the \method{update()}
method so far.  This is an 16-byte string which may contain
non-\ASCII{} characters, including null bytes.
\end{methoddesc}

\begin{methoddesc}[md5]{copy}{}
Return a copy (``clone'') of the md5 object.  This can be used to
efficiently compute the digests of strings that share a common initial
substring.
\end{methoddesc}

\section{\module{mpz} ---
         GNU arbitrary magnitude integers}

\declaremodule{builtin}{mpz}
\modulesynopsis{Interface to the GNU MP library for arbitrary
precision arithmetic.}


This is an optional module.  It is only available when Python is
configured to include it, which requires that the GNU MP software is
installed.
\index{MP, GNU library}
\index{arbitrary precision integers}
\index{integer!arbitrary precision}

This module implements the interface to part of the GNU MP library,
which defines arbitrary precision integer and rational number
arithmetic routines.  Only the interfaces to the \emph{integer}
(\function{mpz_*()}) routines are provided. If not stated
otherwise, the description in the GNU MP documentation can be applied.

Support for rational numbers\index{rational numbers} can be
implemented in Python.  For an example, see the
\module{Rat}\withsubitem{(demo module)}{\ttindex{Rat}} module, provided as
\file{Demos/classes/Rat.py} in the Python source distribution.

In general, \dfn{mpz}-numbers can be used just like other standard
Python numbers, e.g., you can use the built-in operators like \code{+},
\code{*}, etc., as well as the standard built-in functions like
\function{abs()}, \function{int()}, \ldots, \function{divmod()},
\function{pow()}.  \strong{Please note:} the \emph{bitwise-xor}
operation has been implemented as a bunch of \emph{and}s,
\emph{invert}s and \emph{or}s, because the library lacks an
\cfunction{mpz_xor()} function, and I didn't need one.

You create an mpz-number by calling the function \function{mpz()} (see
below for an exact description). An mpz-number is printed like this:
\code{mpz(\var{value})}.


\begin{funcdesc}{mpz}{value}
  Create a new mpz-number. \var{value} can be an integer, a long,
  another mpz-number, or even a string. If it is a string, it is
  interpreted as an array of radix-256 digits, least significant digit
  first, resulting in a positive number. See also the \method{binary()}
  method, described below.
\end{funcdesc}

\begin{datadesc}{MPZType}
  The type of the objects returned by \function{mpz()} and most other
  functions in this module.
\end{datadesc}


A number of \emph{extra} functions are defined in this module. Non
mpz-arguments are converted to mpz-values first, and the functions
return mpz-numbers.

\begin{funcdesc}{powm}{base, exponent, modulus}
  Return \code{pow(\var{base}, \var{exponent}) \%{} \var{modulus}}. If
  \code{\var{exponent} == 0}, return \code{mpz(1)}. In contrast to the
  \C{} library function, this version can handle negative exponents.
\end{funcdesc}

\begin{funcdesc}{gcd}{op1, op2}
  Return the greatest common divisor of \var{op1} and \var{op2}.
\end{funcdesc}

\begin{funcdesc}{gcdext}{a, b}
  Return a tuple \code{(\var{g}, \var{s}, \var{t})}, such that
  \code{\var{a}*\var{s} + \var{b}*\var{t} == \var{g} == gcd(\var{a}, \var{b})}.
\end{funcdesc}

\begin{funcdesc}{sqrt}{op}
  Return the square root of \var{op}. The result is rounded towards zero.
\end{funcdesc}

\begin{funcdesc}{sqrtrem}{op}
  Return a tuple \code{(\var{root}, \var{remainder})}, such that
  \code{\var{root}*\var{root} + \var{remainder} == \var{op}}.
\end{funcdesc}

\begin{funcdesc}{divm}{numerator, denominator, modulus}
  Returns a number \var{q} such that
  \code{\var{q} * \var{denominator} \%{} \var{modulus} ==
  \var{numerator}}.  One could also implement this function in Python,
  using \function{gcdext()}.
\end{funcdesc}

An mpz-number has one method:

\begin{methoddesc}[mpz]{binary}{}
  Convert this mpz-number to a binary string, where the number has been
  stored as an array of radix-256 digits, least significant digit first.

  The mpz-number must have a value greater than or equal to zero,
  otherwise \exception{ValueError} will be raised.
\end{methoddesc}

\section{Built-in Module \sectcode{rotor}}
\bimodindex{rotor}

This module implements a rotor-based encryption algorithm, contributed by
Lance Ellinghouse.  The design is derived from the Enigma device, a machine
used during World War II to encipher messages.  A rotor is simply a
permutation.  For example, if the character `A' is the origin of the rotor,
then a given rotor might map `A' to `L', `B' to `Z', `C' to `G', and so on.
To encrypt, we choose several different rotors, and set the origins of the
rotors to known positions; their initial position is the ciphering key.  To
encipher a character, we permute the original character by the first rotor,
and then apply the second rotor's permutation to the result. We continue
until we've applied all the rotors; the resulting character is our
ciphertext.  We then change the origin of the final rotor by one position,
from `A' to `B'; if the final rotor has made a complete revolution, then we
rotate the next-to-last rotor by one position, and apply the same procedure
recursively.  In other words, after enciphering one character, we advance
the rotors in the same fashion as a car's odometer. Decoding works in the
same way, except we reverse the permutations and apply them in the opposite
order.
\index{Ellinghouse, Lance}
\indexii{Enigma}{cipher}

The available functions in this module are:

\renewcommand{\indexsubitem}{(in module rotor)}
\begin{funcdesc}{newrotor}{key\optional{\, numrotors}}
Return a rotor object. \var{key} is a string containing the encryption key
for the object; it can contain arbitrary binary data. The key will be used
to randomly generate the rotor permutations and their initial positions.
\var{numrotors} is the number of rotor permutations in the returned object;
if it is omitted, a default value of 6 will be used.
\end{funcdesc}

Rotor objects have the following methods:

\renewcommand{\indexsubitem}{(rotor method)}
\begin{funcdesc}{setkey}{\optional{key}}
Sets the rotor's key to \var{key}.  If \var{key} is not given, this
function does nothing\footnote{This is for backwards compatibility.}.
\end{funcdesc}

\begin{funcdesc}{encrypt}{plaintext}
Reset the rotor object to its initial state and encrypt \var{plaintext},
returning a string containing the ciphertext.  The ciphertext is always the
same length as the original plaintext.
\end{funcdesc}

\begin{funcdesc}{encryptmore}{plaintext}
Encrypt \var{plaintext} without resetting the rotor object, and return a
string containing the ciphertext.
\end{funcdesc}

\begin{funcdesc}{decrypt}{ciphertext}
Reset the rotor object to its initial state and decrypt \var{ciphertext},
returning a string containing the ciphertext.  The plaintext string will
always be the same length as the ciphertext.
\end{funcdesc}

\begin{funcdesc}{decryptmore}{ciphertext}
Decrypt \var{ciphertext} without resetting the rotor object, and return a
string containing the ciphertext.
\end{funcdesc}

An example usage:
\bcode\begin{verbatim}
>>> import rotor
>>> rt = rotor.newrotor('key', 12)
>>> rt.encrypt('bar')
'\2534\363'
>>> rt.encryptmore('bar')
'\357\375$'
>>> rt.encrypt('bar')
'\2534\363'
>>> rt.decrypt('\2534\363')
'bar'
>>> rt.decryptmore('\357\375$')
'bar'
>>> rt.decrypt('\357\375$')
'l(\315'
>>> del rt
\end{verbatim}\ecode

The module's code is not an exact simulation of the original Enigma device;
it implements the rotor encryption scheme differently from the original. The
most important difference is that in the original Enigma, there were only 5
or 6 different rotors in existence, and they were applied twice to each
character; the cipher key was the order in which they were placed in the
machine.  The Python rotor module uses the supplied key to initialize a
random number generator; the rotor permutations and their initial positions
are then randomly generated.  The original device only enciphered the
letters of the alphabet, while this module can handle any 8-bit binary data;
it also produces binary output.  This module can also operate with an
arbitrary number of rotors.

The original Enigma cipher was broken in 1944. % XXX: Is this right?
The version implemented here is probably a good deal more difficult to crack
(especially if you use many rotors), but it won't be impossible for
a truly skilful and determined attacker to break the cipher.  So if you want
to keep the NSA out of your files, this rotor cipher may well be unsafe, but
for discouraging casual snooping through your files, it will probably be
just fine, and may be somewhat safer than using the Unix \file{crypt}
command.
\index{National Security Agency}\index{crypt(1)}
% XXX How were Unix commands represented in the docs?



%\chapter{Amoeba Specific Services}

\section{\module{amoeba} ---
         Amoeba system support}

\declaremodule{builtin}{amoeba}
  \platform{Amoeba}
\modulesynopsis{Functions for the Amoeba operating system.}


This module provides some object types and operations useful for
Amoeba applications.  It is only available on systems that support
Amoeba operations.  RPC errors and other Amoeba errors are reported as
the exception \code{amoeba.error = 'amoeba.error'}.

The module \module{amoeba} defines the following items:

\begin{funcdesc}{name_append}{path, cap}
Stores a capability in the Amoeba directory tree.
Arguments are the pathname (a string) and the capability (a capability
object as returned by
\function{name_lookup()}).
\end{funcdesc}

\begin{funcdesc}{name_delete}{path}
Deletes a capability from the Amoeba directory tree.
Argument is the pathname.
\end{funcdesc}

\begin{funcdesc}{name_lookup}{path}
Looks up a capability.
Argument is the pathname.
Returns a
\dfn{capability}
object, to which various interesting operations apply, described below.
\end{funcdesc}

\begin{funcdesc}{name_replace}{path, cap}
Replaces a capability in the Amoeba directory tree.
Arguments are the pathname and the new capability.
(This differs from
\function{name_append()}
in the behavior when the pathname already exists:
\function{name_append()}
finds this an error while
\function{name_replace()}
allows it, as its name suggests.)
\end{funcdesc}

\begin{datadesc}{capv}
A table representing the capability environment at the time the
interpreter was started.
(Alas, modifying this table does not affect the capability environment
of the interpreter.)
For example,
\code{amoeba.capv['ROOT']}
is the capability of your root directory, similar to
\code{getcap("ROOT")}
in C.
\end{datadesc}

\begin{excdesc}{error}
The exception raised when an Amoeba function returns an error.
The value accompanying this exception is a pair containing the numeric
error code and the corresponding string, as returned by the C function
\cfunction{err_why()}.
\end{excdesc}

\begin{funcdesc}{timeout}{msecs}
Sets the transaction timeout, in milliseconds.
Returns the previous timeout.
Initially, the timeout is set to 2 seconds by the Python interpreter.
\end{funcdesc}

\subsection{Capability Operations}

Capabilities are written in a convenient \ASCII{} format, also used by the
Amoeba utilities
\emph{c2a}(U)
and
\emph{a2c}(U).
For example:

\begin{verbatim}
>>> amoeba.name_lookup('/profile/cap')
aa:1c:95:52:6a:fa/14(ff)/8e:ba:5b:8:11:1a
>>> 
\end{verbatim}
%
The following methods are defined for capability objects.

\begin{methoddesc}[capability]{dir_list}{}
Returns a list of the names of the entries in an Amoeba directory.
\end{methoddesc}

\begin{methoddesc}[capability]{b_read}{offset, maxsize}
Reads (at most)
\var{maxsize}
bytes from a bullet file at offset
\var{offset.}
The data is returned as a string.
EOF is reported as an empty string.
\end{methoddesc}

\begin{methoddesc}[capability]{b_size}{}
Returns the size of a bullet file.
\end{methoddesc}

\begin{methoddesc}[capability]{dir_append}{}
\funcline{dir_delete}{}
\funcline{dir_lookup}{}
\funcline{dir_replace}{}
Like the corresponding
\samp{name_}*
functions, but with a path relative to the capability.
(For paths beginning with a slash the capability is ignored, since this
is the defined semantics for Amoeba.)
\end{methoddesc}

\begin{methoddesc}[capability]{std_info}{}
Returns the standard info string of the object.
\end{methoddesc}

\begin{methoddesc}[capability]{tod_gettime}{}
Returns the time (in seconds since the Epoch, in UCT, as for \POSIX) from
a time server.
\end{methoddesc}

\begin{methoddesc}[capability]{tod_settime}{t}
Sets the time kept by a time server.
\end{methoddesc}
		% AMOEBA ONLY

\chapter{MACINTOSH ONLY}

The modules in this chapter are available on the Apple Macintosh only.

\section{Built-in Module \sectcode{mac}}

\bimodindex{mac}
This module provides a subset of the operating system dependent
functionality provided by the optional built-in module \code{posix}.
It is best accessed through the more portable standard module
\code{os}.

The following functions are available in this module:
\code{chdir},
\code{getcwd},
\code{listdir},
\code{mkdir},
\code{rename},
\code{rmdir},
\code{stat},
\code{sync},
\code{unlink},
as well as the exception \code{error}.

\section{Standard Module \sectcode{macpath}}

\stmodindex{macpath}
This module provides a subset of the pathname manipulation functions
available from the optional standard module \code{posixpath}.  It is
best accessed through the more portable standard module \code{os}, as
\code{os.path}.

The following functions are available in this module:
\code{normcase},
\code{isabs},
\code{join},
\code{split},
\code{isdir},
\code{isfile},
\code{exists}.
			% MACINTOSH ONLY
\section{Built-in module \sectcode{ctb}}
\bimodindex{ctb}

This module provides a partial interface to the Macintosh
Communications Toolbox. Currently, only Connection Manager tools are
supported. 

\begin{datadesc}{error}
The exception raised on errors.
\end{datadesc}

\begin{datadesc}{cmData}
\dataline{cmCntl}
\dataline{cmAttn}
Flags for the \var{channel} argument of the \var{Read} and \var{Write}
methods.
\end{datadesc}

\begin{datadesc}{cmFlagsEOM}
End-of-message flag for \var{Read} and \var{Write}.
\end{datadesc}

\begin{datadesc}{choose*}
Values returned by \var{Choose}.
\end{datadesc}

\begin{datadesc}{cmStatus*}
Bits in the status as returned by \var{Status}.
\end{datadesc}

\begin{funcdesc}{available}{}
Returns 1 if the communication toolbox is available, zero otherwise.
\end{funcdesc}

\begin{funcdesc}{CMNew}{name\, sizes}
Create a connection object using the connection tool named
\var{name}. \var{sizes} is a 6-tuple given buffer sizes for data in,
data out, control in, control out, attention in and attention out.
Alternatively, passing \var{None} will result in default buffer sizes.
\end{funcdesc}

\subsection{connection object}
For all connection methods that take a \var{timeout} argument, a value
of \code{-1} is indefinite, meaning that the command runs to completion.

\renewcommand{\indexsubitem}{(connection object method)}

\begin{datadesc}{callback}
If this member is set to a value other than \var{None} it should point
to a function accepting a single argument (the connection
object). This will make all connection object methods work
asynchronously, with the callback routine being called upon
completion.

{\em Note:} for reasons beyond my understanding the callback routine
is never called currently. You are advised against using asynchronous
calls for the time being.
\end{datadesc}


\begin{funcdesc}{Open}{timeout}
Open an outgoing connection, waiting at most \var{timeout} seconds for
the connection to be established.
\end{funcdesc}

\begin{funcdesc}{Listen}{timeout}
Wait for an incoming connection. Stop waiting after \var{timeout}
seconds. This call is only meaningful to some tools.
\end{funcdesc}

\begin{funcdesc}{accept}{yesno}
Accept (when \var{yesno} is non-zero) or reject an incoming call after
\var{Listen} returned.
\end{funcdesc}

\begin{funcdesc}{Close}{timeout\, now}
Close a connection. When \var{now} is zero, the close is orderly
(i.e. outstanding output is flushed, etc) with a timeout of
\var{timeout} seconds. When \var{now} is non-zero the close is
immedeate, discarding output.
\end{funcdesc}

\begin{funcdesc}{Read}{len\, chan\, timeout}
Read \var{len} bytes or until \var{timeout} seconds have passed from
the channel \var{chan} (which is one of \var{cmData}, \var{cmCntl} or
\var{cmAttn}). Returns a 2-tuple: the data read and the end-of-message
flag.
\end{funcdesc}

\begin{funcdesc}{Write}{buf\, chan\, timeout\, eom}
Write \var{buf} to channel \var{chan}, aborting after \var{timeout}
seconds. When \var{eom} has the value \var{cmFlagsEOM} an
end-of-message indicator will be written after the data (if this
concept has a meaning for this communication tool). The method returns
the number of bytes written.
\end{funcdesc}

\begin{funcdesc}{Status}{}
Return connection status as the 2-tuple \code{(sizes,
flags)}. \var{Sizes} is a 6-tuple giving the actual buffer sizes used
(see \var{CMNew}), \var{flags} is a set of bits describing the state
of the connection.
\end{funcdesc}

\begin{funcdesc}{GetConfig}{}
Return the configuration string of the communication tool. These
configuration strings are tool-dependent, but usually easily parsed
and modified.
\end{funcdesc}

\begin{funcdesc}{SetConfig}{str}
Set the configuration string for the tool. The strings are parsed
left-to-right, with later values taking precedence. This means
individual configuration parameters can be modified by simply appending
something like \code{'baud 4800'} to the end of the string returned by
\var{GetConfig} and passing that to this method. The method returns
the number of characters actually parsed by the tool before it
encountered an error (or completed successfully).
\end{funcdesc}

\begin{funcdesc}{Choose}{}
Present the user with a dialog to choose a communication tool and
configure it. If there is an outstanding connection some choices (like
selecting a different tool) may cause the connection to be
aborted. The return value (one of the \var{choose*} constants) will
indicate this.
\end{funcdesc}

\begin{funcdesc}{Idle}{}
Give the tool a chance to use the processor. You should call this
method regularly.
\end{funcdesc}

\begin{funcdesc}{Abort}{}
Abort an outstanding asynchronous \var{Open} or \var{Listen}.
\end{funcdesc}

\begin{funcdesc}{Reset}{}
Reset a connection. Exact meaning depends on the tool.
\end{funcdesc}

\begin{funcdesc}{Break}{length}
Send a break. Whether this means anything, what it means and
interpretation of the \var{length} parameter depend on the tool in
use.
\end{funcdesc}

\section{Built-in Module \sectcode{macconsole}}
\label{module-macconsole}
\bimodindex{macconsole}

\setindexsubitem{(in module macconsole)}

This module is available on the Macintosh, provided Python has been
built using the Think \C{} compiler. It provides an interface to the
Think console package, with which basic text windows can be created.

\begin{datadesc}{options}
An object allowing you to set various options when creating windows,
see below.
\end{datadesc}

\begin{datadesc}{C_ECHO}
\dataline{C_NOECHO}
\dataline{C_CBREAK}
\dataline{C_RAW}
Options for the \code{setmode} method. \constant{C_ECHO} and
\constant{C_CBREAK} enable character echo, the other two disable it,
\constant{C_ECHO} and \constant{C_NOECHO} enable line-oriented input
(erase/kill processing, etc).
\end{datadesc}

\begin{funcdesc}{copen}{}
Open a new console window. Return a console window object.
\end{funcdesc}

\begin{funcdesc}{fopen}{fp}
Return the console window object corresponding with the given file
object. \var{fp} should be one of \code{sys.stdin}, \code{sys.stdout} or
\code{sys.stderr}.
\end{funcdesc}

\subsection{macconsole options object}
These options are examined when a window is created:

\setindexsubitem{(macconsole option)}
\begin{datadesc}{top}
\dataline{left}
The origin of the window.
\end{datadesc}

\begin{datadesc}{nrows}
\dataline{ncols}
The size of the window.
\end{datadesc}

\begin{datadesc}{txFont}
\dataline{txSize}
\dataline{txStyle}
The font, fontsize and fontstyle to be used in the window.
\end{datadesc}

\begin{datadesc}{title}
The title of the window.
\end{datadesc}

\begin{datadesc}{pause_atexit}
If set non-zero, the window will wait for user action before closing.
\end{datadesc}

\subsection{console window object}

\setindexsubitem{(console window attribute)}

\begin{datadesc}{file}
The file object corresponding to this console window. If the file is
buffered, you should call \code{\var{file}.flush()} between
\code{write()} and \code{read()} calls.
\end{datadesc}

\setindexsubitem{(console window method)}

\begin{funcdesc}{setmode}{mode}
Set the input mode of the console to \constant{C_ECHO}, etc.
\end{funcdesc}

\begin{funcdesc}{settabs}{n}
Set the tabsize to \var{n} spaces.
\end{funcdesc}

\begin{funcdesc}{cleos}{}
Clear to end-of-screen.
\end{funcdesc}

\begin{funcdesc}{cleol}{}
Clear to end-of-line.
\end{funcdesc}

\begin{funcdesc}{inverse}{onoff}
Enable inverse-video mode:\ characters with the high bit set are
displayed in inverse video (this disables the upper half of a
non-\ASCII{} character set).
\end{funcdesc}

\begin{funcdesc}{gotoxy}{x, y}
Set the cursor to position \code{(\var{x}, \var{y})}.
\end{funcdesc}

\begin{funcdesc}{hide}{}
Hide the window, remembering the contents.
\end{funcdesc}

\begin{funcdesc}{show}{}
Show the window again.
\end{funcdesc}

\begin{funcdesc}{echo2printer}{}
Copy everything written to the window to the printer as well.
\end{funcdesc}


\section{Built-in module \sectcode{macdnr}}
\bimodindex{macdnr}

This module provides an interface to the Macintosh Domain Name
Resolver. It is usually used in conjunction with the \var{mactcp} module, to
map hostnames to IP-addresses.

The \code{macdnr} module defines the following functions:

\renewcommand{\indexsubitem}{(in module macdnr)}

\begin{funcdesc}{Open}{\optional{filename}}
Open the domain name resolver extension. If \var{filename} is given it
should be the pathname of the extension, otherwise a default is
used. Normally, this call is not needed since the other calls will
open the extension automatically.
\end{funcdesc}

\begin{funcdesc}{Close}{}
Close the resolver extension. Again, not needed for normal use.
\end{funcdesc}

\begin{funcdesc}{StrToAddr}{hostname}
Look up the IP address for \var{hostname}. This call returns a dnr
result object of the ``address'' variation.
\end{funcdesc}

\begin{funcdesc}{AddrToName}{addr}
Do a reverse lookup on the 32-bit integer IP-address
\var{addr}. Returns a dnr result object of the ``address'' variation.
\end{funcdesc}

\begin{funcdesc}{AddrToStr}{addr}
Convert the 32-bit integer IP-address \var{addr} to a dotted-decimal
string. Returns the string.
\end{funcdesc}

\begin{funcdesc}{HInfo}{hostname}
Query the nameservers for a \code{HInfo} record for host
\var{hostname}. These records contain hardware and software
information about the machine in question (if they are available in
the first place). Returns a dnr result object of the ``hinfo''
variety.
\end{funcdesc}

\begin{funcdesc}{MXInfo}{domain}
Query the nameservers for a mail exchanger for \var{domain}. This is
the hostname of a host willing to accept SMTP mail for the given
domain. Returns a dnr result object of the ``mx'' variety.
\end{funcdesc}

\subsection{dnr result object}

Since the DNR calls all execute asynchronously you do not get the
results back immedeately. In stead, you get a dnr result object. You
can check this object to see whether the query is complete, and access
its attributes to obtain the information when it is.

Alternatively, you can also reference the result attributes directly,
this will result in an implicit wait for the query to complete.

The \var{rtnCode} and \var{cname} attributes are always available, the
others depend on the type of query (address, hinfo or mx).

\renewcommand{\indexsubitem}{(dnr result object method)}

% Add args, as in {arg1\, arg2 \optional{\, arg3}}
\begin{funcdesc}{wait}{}
Wait for the query to complete.
\end{funcdesc}

% Add args, as in {arg1\, arg2 \optional{\, arg3}}
\begin{funcdesc}{isdone}{}
Return 1 if the query is complete.
\end{funcdesc}

\begin{datadesc}{rtnCode}
The error code returned by the query.
\end{datadesc}

\begin{datadesc}{cname}
The canonical name of the host that was queried.
\end{datadesc}

\begin{datadesc}{ip0}
\dataline{ip1}
\dataline{ip2}
\dataline{ip3}
At most four integer IP addresses for this host. Unused entries are
zero. Valid only for address queries.
\end{datadesc}

\begin{datadesc}{cpuType}
\dataline{osType}
Textual strings giving the machine type an OS name. Valid for hinfo
queries.
\end{datadesc}

\begin{datadesc}{exchange}
The name of a mail-exchanger host. Valid for mx queries.
\end{datadesc}

\begin{datadesc}{preference}
The preference of this mx record. Not too useful, since the Macintosh
will only return a single mx record. Mx queries only.
\end{datadesc}

The simplest way to use the module to convert names to dotted-decimal
strings, without worrying about idle time, etc:
\begin{verbatim}
>>> def gethostname(name):
...     import macdnr
...     dnrr = macdnr.StrToAddr(name)
...     return macdnr.AddrToStr(dnrr.ip0)
\end{verbatim}

\section{Built-in Module \sectcode{macfs}}
\label{module-macfs}
\bimodindex{macfs}

\renewcommand{\indexsubitem}{(in module macfs)}

This module provides access to macintosh FSSpec handling, the Alias
Manager, finder aliases and the Standard File package.

Whenever a function or method expects a \var{file} argument, this
argument can be one of three things:\ (1) a full or partial Macintosh
pathname, (2) an FSSpec object or (3) a 3-tuple \code{(wdRefNum,
parID, name)} as described in Inside Mac VI\@. A description of aliases
and the standard file package can also be found there.

\begin{funcdesc}{FSSpec}{file}
Create an FSSpec object for the specified file.
\end{funcdesc}

\begin{funcdesc}{RawFSSpec}{data}
Create an FSSpec object given the raw data for the C structure for the
FSSpec as a string.  This is mainly useful if you have obtained an
FSSpec structure over a network.
\end{funcdesc}

\begin{funcdesc}{RawAlias}{data}
Create an Alias object given the raw data for the C structure for the
alias as a string.  This is mainly useful if you have obtained an
FSSpec structure over a network.
\end{funcdesc}

\begin{funcdesc}{FInfo}{}
Create a zero-filled FInfo object.
\end{funcdesc}

\begin{funcdesc}{ResolveAliasFile}{file}
Resolve an alias file. Returns a 3-tuple \code{(\var{fsspec}, \var{isfolder},
\var{aliased})} where \var{fsspec} is the resulting FSSpec object,
\var{isfolder} is true if \var{fsspec} points to a folder and
\var{aliased} is true if the file was an alias in the first place
(otherwise the FSSpec object for the file itself is returned).
\end{funcdesc}

\begin{funcdesc}{StandardGetFile}{\optional{type\, ...}}
Present the user with a standard ``open input file''
dialog. Optionally, you can pass up to four 4-char file types to limit
the files the user can choose from. The function returns an FSSpec
object and a flag indicating that the user completed the dialog
without cancelling.
\end{funcdesc}

\begin{funcdesc}{PromptGetFile}{prompt\optional{\, type\, ...}}
Similar to \var{StandardGetFile} but allows you to specify a prompt.
\end{funcdesc}

\begin{funcdesc}{StandardPutFile}{prompt\, \optional{default}}
Present the user with a standard ``open output file''
dialog. \var{prompt} is the prompt string, and the optional
\var{default} argument initializes the output file name. The function
returns an FSSpec object and a flag indicating that the user completed
the dialog without cancelling.
\end{funcdesc}

\begin{funcdesc}{GetDirectory}{\optional{prompt}}
Present the user with a non-standard ``select a directory''
dialog. \var{prompt} is the prompt string, and the optional.
Return an FSSpec object and a success-indicator.
\end{funcdesc}

\begin{funcdesc}{SetFolder}{\optional{fsspec}}
Set the folder that is initially presented to the user when one of
the file selection dialogs is presented. \var{Fsspec} should point to
a file in the folder, not the folder itself (the file need not exist,
though). If no argument is passed the folder will be set to the
current directory, i.e. what \code{os.getcwd()} returns.

Note that starting with system 7.5 the user can change Standard File
behaviour with the ``general controls'' controlpanel, thereby making
this call inoperative.
\end{funcdesc}

\begin{funcdesc}{FindFolder}{where\, which\, create}
Locates one of the ``special'' folders that MacOS knows about, such as
the trash or the Preferences folder. \var{Where} is the disk to
search, \var{which} is the 4-char string specifying which folder to
locate. Setting \var{create} causes the folder to be created if it
does not exist. Returns a \code{(vrefnum, dirid)} tuple.
\end{funcdesc}

\begin{funcdesc}{NewAliasMinimalFromFullPath}{pathname}
Return a minimal alias record object that points to the given file, which
must be specified as a full pathname. This is the only way to create an
alias record pointing to a non-existing file.

The constants for \var{where} and \var{which} can be obtained from the
standard module \var{MACFS}.
\end{funcdesc}

\begin{funcdesc}{FindApplication}{creator}
Locate the application with 4-char creator code \var{creator}. The
function returns an FSSpec object pointing to the application.
\end{funcdesc}

\subsection{FSSpec objects}

\renewcommand{\indexsubitem}{(FSSpec object attribute)}
\begin{datadesc}{data}
The raw data from the FSSpec object, suitable for passing
to other applications, for instance.
\end{datadesc}

\renewcommand{\indexsubitem}{(FSSpec object method)}
\begin{funcdesc}{as_pathname}{}
Return the full pathname of the file described by the FSSpec object.
\end{funcdesc}

\begin{funcdesc}{as_tuple}{}
Return the \code{(\var{wdRefNum}, \var{parID}, \var{name})} tuple of the file described
by the FSSpec object.
\end{funcdesc}

\begin{funcdesc}{NewAlias}{\optional{file}}
Create an Alias object pointing to the file described by this
FSSpec. If the optional \var{file} parameter is present the alias
will be relative to that file, otherwise it will be absolute.
\end{funcdesc}

\begin{funcdesc}{NewAliasMinimal}{}
Create a minimal alias pointing to this file.
\end{funcdesc}

\begin{funcdesc}{GetCreatorType}{}
Return the 4-char creator and type of the file.
\end{funcdesc}

\begin{funcdesc}{SetCreatorType}{creator\, type}
Set the 4-char creator and type of the file.
\end{funcdesc}

\begin{funcdesc}{GetFInfo}{}
Return a FInfo object describing the finder info for the file.
\end{funcdesc}

\begin{funcdesc}{SetFInfo}{finfo}
Set the finder info for the file to the values specified in the
\var{finfo} object.
\end{funcdesc}

\begin{funcdesc}{GetDates}{}
Return a tuple with three floating point values representing the
creation date, modification date and backup date of the file.
\end{funcdesc}

\begin{funcdesc}{SetDates}{crdate\, moddate\, backupdate}
Set the creation, modification and backup date of the file. The values
are in the standard floating point format used for times throughout
Python.
\end{funcdesc}

\subsection{alias objects}

\renewcommand{\indexsubitem}{(alias object attribute)}
\begin{datadesc}{data}
The raw data for the Alias record, suitable for storing in a resource
or transmitting to other programs.
\end{datadesc}

\renewcommand{\indexsubitem}{(alias object method)}
\begin{funcdesc}{Resolve}{\optional{file}}
Resolve the alias. If the alias was created as a relative alias you
should pass the file relative to which it is. Return the FSSpec for
the file pointed to and a flag indicating whether the alias object
itself was modified during the search process. If the file does
not exist but the path leading up to it does exist a valid fsspec
is returned.
\end{funcdesc}

\begin{funcdesc}{GetInfo}{num}
An interface to the C routine \code{GetAliasInfo()}.
\end{funcdesc}

\begin{funcdesc}{Update}{file\, \optional{file2}}
Update the alias to point to the \var{file} given. If \var{file2} is
present a relative alias will be created.
\end{funcdesc}

Note that it is currently not possible to directly manipulate a resource
as an alias object. Hence, after calling \var{Update} or after
\var{Resolve} indicates that the alias has changed the Python program
is responsible for getting the \var{data} from the alias object and
modifying the resource.


\subsection{FInfo objects}

See Inside Mac for a complete description of what the various fields
mean.

\renewcommand{\indexsubitem}{(FInfo object attribute)}
\begin{datadesc}{Creator}
The 4-char creator code of the file.
\end{datadesc}

\begin{datadesc}{Type}
The 4-char type code of the file.
\end{datadesc}

\begin{datadesc}{Flags}
The finder flags for the file as 16-bit integer. The bit values in
\var{Flags} are defined in standard module \var{MACFS}.
\end{datadesc}

\begin{datadesc}{Location}
A Point giving the position of the file's icon in its folder.
\end{datadesc}

\begin{datadesc}{Fldr}
The folder the file is in (as an integer).
\end{datadesc}

\section{\module{ic} ---
         Access to Internet Config}

\declaremodule{builtin}{ic}
  \platform{Mac}
\modulesynopsis{Access to Internet Config.}


This module provides access to Macintosh Internet
Config\index{Internet Config} package,
which stores preferences for Internet programs such as mail address,
default homepage, etc. Also, Internet Config contains an elaborate set
of mappings from Macintosh creator/type codes to foreign filename
extensions plus information on how to transfer files (binary, ascii,
etc.). Since MacOS 9, this module is a control panel named Internet.

There is a low-level companion module
\module{icglue}\refbimodindex{icglue} which provides the basic
Internet Config access functionality.  This low-level module is not
documented, but the docstrings of the routines document the parameters
and the routine names are the same as for the Pascal or \C{} API to
Internet Config, so the standard IC programmers' documentation can be
used if this module is needed.

The \module{ic} module defines the \exception{error} exception and
symbolic names for all error codes Internet Config can produce; see
the source for details.

\begin{excdesc}{error}
Exception raised on errors in the \module{ic} module.
\end{excdesc}


The \module{ic} module defines the following class and function:

\begin{classdesc}{IC}{\optional{signature\optional{, ic}}}
Create an Internet Config object. The signature is a 4-character creator
code of the current application (default \code{'Pyth'}) which may
influence some of ICs settings. The optional \var{ic} argument is a
low-level \code{icglue.icinstance} created beforehand, this may be
useful if you want to get preferences from a different config file,
etc.
\end{classdesc}

\begin{funcdesc}{launchurl}{url\optional{, hint}}
\funcline{parseurl}{data\optional{, start\optional{, end\optional{, hint}}}}
\funcline{mapfile}{file}
\funcline{maptypecreator}{type, creator\optional{, filename}}
\funcline{settypecreator}{file}
These functions are ``shortcuts'' to the methods of the same name,
described below.
\end{funcdesc}


\subsection{IC Objects}

\class{IC} objects have a mapping interface, hence to obtain the mail
address you simply get \code{\var{ic}['MailAddress']}. Assignment also
works, and changes the option in the configuration file.

The module knows about various datatypes, and converts the internal IC
representation to a ``logical'' Python data structure. Running the
\module{ic} module standalone will run a test program that lists all
keys and values in your IC database, this will have to serve as
documentation.

If the module does not know how to represent the data it returns an
instance of the \code{ICOpaqueData} type, with the raw data in its
\member{data} attribute. Objects of this type are also acceptable values
for assignment.

Besides the dictionary interface, \class{IC} objects have the
following methods:


\begin{methoddesc}{launchurl}{url\optional{, hint}}
Parse the given URL, launch the correct application and pass it the
URL. The optional \var{hint} can be a scheme name such as
\code{'mailto:'}, in which case incomplete URLs are completed with this
scheme.  If \var{hint} is not provided, incomplete URLs are invalid.
\end{methoddesc}

\begin{methoddesc}{parseurl}{data\optional{, start\optional{, end\optional{, hint}}}}
Find an URL somewhere in \var{data} and return start position, end
position and the URL. The optional \var{start} and \var{end} can be
used to limit the search, so for instance if a user clicks in a long
text field you can pass the whole text field and the click-position in
\var{start} and this routine will return the whole URL in which the
user clicked.  As above, \var{hint} is an optional scheme used to
complete incomplete URLs.
\end{methoddesc}

\begin{methoddesc}{mapfile}{file}
Return the mapping entry for the given \var{file}, which can be passed
as either a filename or an \function{macfs.FSSpec()} result, and which
need not exist.

The mapping entry is returned as a tuple \code{(\var{version},
\var{type}, \var{creator}, \var{postcreator}, \var{flags},
\var{extension}, \var{appname}, \var{postappname}, \var{mimetype},
\var{entryname})}, where \var{version} is the entry version
number, \var{type} is the 4-character filetype, \var{creator} is the
4-character creator type, \var{postcreator} is the 4-character creator
code of an
optional application to post-process the file after downloading,
\var{flags} are various bits specifying whether to transfer in binary
or ascii and such, \var{extension} is the filename extension for this
file type, \var{appname} is the printable name of the application to
which this file belongs, \var{postappname} is the name of the
postprocessing application, \var{mimetype} is the MIME type of this
file and \var{entryname} is the name of this entry.
\end{methoddesc}

\begin{methoddesc}{maptypecreator}{type, creator\optional{, filename}}
Return the mapping entry for files with given 4-character \var{type} and
\var{creator} codes. The optional \var{filename} may be specified to
further help finding the correct entry (if the creator code is
\code{'????'}, for instance).

The mapping entry is returned in the same format as for \var{mapfile}.
\end{methoddesc}

\begin{methoddesc}{settypecreator}{file}
Given an existing \var{file}, specified either as a filename or as an
\function{macfs.FSSpec()} result, set its creator and type correctly based
on its extension.  The finder is told about the change, so the finder
icon will be updated quickly.
\end{methoddesc}

\section{\module{MacOS} ---
         Access to MacOS specific interpreter features.}
\declaremodule{builtin}{MacOS}

\modulesynopsis{Access to MacOS specific interpreter features.}



This module provides access to MacOS specific functionality in the
Python interpreter, such as how the interpreter eventloop functions
and the like. Use with care.

Note the capitalisation of the module name, this is a historical
artifact.

\begin{excdesc}{Error}
This exception is raised on MacOS generated errors, either from
functions in this module or from other mac-specific modules like the
toolbox interfaces. The arguments are the integer error code (the
\cdata{OSErr} value) and a textual description of the error code.
Symbolic names for all known error codes are defined in the standard
module \module{macerrors}\refstmodindex{macerrors}.
\end{excdesc}

\begin{funcdesc}{SetEventHandler}{handler}
In the inner interpreter loop Python will occasionally check for events,
unless disabled with \function{ScheduleParams()}. With this function you
can pass a Python event-handler function that will be called if an event
is available. The event is passed as parameter and the function should return
non-zero if the event has been fully processed, otherwise event processing
continues (by passing the event to the console window package, for instance).

Call \function{SetEventHandler()} without a parameter to clear the
event handler. Setting an event handler while one is already set is an
error.
\end{funcdesc}

\begin{funcdesc}{SchedParams}{\optional{doint\optional{, evtmask\optional{,
                              besocial\optional{, interval\optional{,
                              bgyield}}}}}}
Influence the interpreter inner loop event handling. \var{Interval}
specifies how often (in seconds, floating point) the interpreter
should enter the event processing code. When true, \var{doint} causes
interrupt (command-dot) checking to be done. \var{evtmask} tells the
interpreter to do event processing for events in the mask (redraws,
mouseclicks to switch to other applications, etc). The \var{besocial}
flag gives other processes a chance to run. They are granted minimal
runtime when Python is in the foreground and \var{bgyield} seconds per
\var{interval} when Python runs in the background.

All parameters are optional, and default to the current value. The return
value of this function is a tuple with the old values of these options.
Initial defaults are that all processing is enabled, checking is done every
quarter second and the CPU is given up for a quarter second when in the
background.
\end{funcdesc}

\begin{funcdesc}{HandleEvent}{ev}
Pass the event record \var{ev} back to the Python event loop, or
possibly to the handler for the \code{sys.stdout} window (based on the
compiler used to build Python). This allows Python programs that do
their own event handling to still have some command-period and
window-switching capability.

If you attempt to call this function from an event handler set through
\function{SetEventHandler()} you will get an exception.
\end{funcdesc}

\begin{funcdesc}{GetErrorString}{errno}
Return the textual description of MacOS error code \var{errno}.
\end{funcdesc}

\begin{funcdesc}{splash}{resid}
This function will put a splash window
on-screen, with the contents of the DLOG resource specified by
\var{resid}. Calling with a zero argument will remove the splash
screen. This function is useful if you want an applet to post a splash screen
early in initialization without first having to load numerous
extension modules.
\end{funcdesc}

\begin{funcdesc}{DebugStr}{message \optional{, object}}
Drop to the low-level debugger with message \var{message}. The
optional \var{object} argument is not used, but can easily be
inspected from the debugger.

Note that you should use this function with extreme care: if no
low-level debugger like MacsBug is installed this call will crash your
system. It is intended mainly for developers of Python extension
modules.
\end{funcdesc}

\begin{funcdesc}{openrf}{name \optional{, mode}}
Open the resource fork of a file. Arguments are the same as for the
built-in function \function{open()}. The object returned has file-like
semantics, but it is not a Python file object, so there may be subtle
differences.
\end{funcdesc}


\section{Standard module \sectcode{macostools}}
\stmodindex{macostools}

This module contains some convenience routines for file-manipulation
on the Macintosh.

The \code{macostools} module defines the following functions:

\renewcommand{\indexsubitem}{(in module macostools)}

\begin{funcdesc}{copy}{src\, dst\optional{\, createpath, copytimes}}
Copy file \var{src} to \var{dst}. The files can be specified as
pathnames or \code{FSSpec} objects. If \var{createpath} is non-zero
\var{dst} must be a pathname and the folders leading to the
destination are created if necessary.  The method copies data and
resource fork and some finder information (creator, type, flags) and
optionally the creation, modification and backup times (default is to
copy them). Custom icons, comments and icon position are not copied.

If the source is an alias the original to which the alias points is
copied, not the aliasfile.
\end{funcdesc}

\begin{funcdesc}{copytree}{src\, dst}
Recursively copy a file tree from \var{src} to \var{dst}, creating
folders as needed. \var{Src} and \var{dst} should be specified as
pathnames.
\end{funcdesc}

\begin{funcdesc}{mkalias}{src\, dst}
Create a finder alias \var{dst} pointing to \var{src}. Both may be
specified as pathnames or \var{FSSpec} objects.
\end{funcdesc}

\begin{funcdesc}{touched}{dst}
Tell the finder that some bits of finder-information such as creator
or type for file \var{dst} has changed. The file can be specified by
pathname or fsspec. This call should prod the finder into redrawing the
files icon.
\end{funcdesc}

\begin{datadesc}{BUFSIZ}
The buffer size for \code{copy}, default 1 megabyte.
\end{datadesc}

Note that the process of creating finder aliases is not specified in
the Apple documentation. Hence, aliases created with \code{mkalias}
could conceivably have incompatible behaviour in some cases.

\section{Standard module \sectcode{findertools}}
\stmodindex{findertools}

This module contains routines that give Python programs access to some
functionality provided by the finder. They are implemented as wrappers
around the AppleEvent interface to the finder.

All file and folder parameters can be specified either as full
pathnames or as \code{FSSpec} objects.

The \code{findertools} module defines the following functions:

\renewcommand{\indexsubitem}{(in module macostools)}

\begin{funcdesc}{launch}{file}
Tell the finder to launch \var{file}. What launching means depends on the file:
applications are started, folders are opened and documents are opened
in the correct application.
\end{funcdesc}

\begin{funcdesc}{Print}{file}
Tell the finder to print a file (again specified by full pathname or
FSSpec). The behaviour is identical to selecting the file and using
the print command in the finder.
\end{funcdesc}

\begin{funcdesc}{copy}{file, destdir}
Tell the finder to copy a file or folder \var{file} to folder
\var{destdir}. The function returns an \code{Alias} object pointing to
the new file.
\end{funcdesc}

\begin{funcdesc}{move}{file, destdir}
Tell the finder to move a file or folder \var{file} to folder
\var{destdir}. The function returns an \code{Alias} object pointing to
the new file.
\end{funcdesc}

\begin{funcdesc}{sleep}{}
Tell the finder to put the mac to sleep, if your machine supports it.
\end{funcdesc}

\begin{funcdesc}{restart}{}
Tell the finder to perform an orderly restart of the machine.
\end{funcdesc}

\begin{funcdesc}{shutdown}{}
Tell the finder to perform an orderly shutdown of the machine.
\end{funcdesc}

\section{\module{mactcp} ---
         The MacTCP interfaces.}
\declaremodule{builtin}{mactcp}

\modulesynopsis{The MacTCP interfaces.}



This module provides an interface to the Macintosh TCP/IP driver%
\index{MacTCP} MacTCP. There is an accompanying module,
\module{macdnr}\refbimodindex{macdnr}, which provides an interface to
the name-server (allowing you to translate hostnames to IP addresses),
a module \module{MACTCPconst}\refstmodindex{MACTCPconst} which has
symbolic names for constants constants used by MacTCP. Since the
built-in module \module{socket} is also available on the Macintosh it
is usually easier to use sockets instead of the Macintosh-specific
MacTCP API.

A complete description of the MacTCP interface can be found in the
Apple MacTCP API documentation.

\begin{funcdesc}{MTU}{}
Return the Maximum Transmit Unit (the packet size) of the network
interface.\index{Maximum Transmit Unit}
\end{funcdesc}

\begin{funcdesc}{IPAddr}{}
Return the 32-bit integer IP address of the network interface.
\end{funcdesc}

\begin{funcdesc}{NetMask}{}
Return the 32-bit integer network mask of the interface.
\end{funcdesc}

\begin{funcdesc}{TCPCreate}{size}
Create a TCP Stream object. \var{size} is the size of the receive
buffer, \code{4096} is suggested by various sources.
\end{funcdesc}

\begin{funcdesc}{UDPCreate}{size, port}
Create a UDP Stream object. \var{size} is the size of the receive
buffer (and, hence, the size of the biggest datagram you can receive
on this port). \var{port} is the UDP port number you want to receive
datagrams on, a value of zero will make MacTCP select a free port.
\end{funcdesc}


\subsection{TCP Stream Objects}

\begin{memberdesc}[TCP Stream]{asr}
\index{asynchronous service routine}
\index{service routine, asynchronous}
When set to a value different than \code{None} this should refer to a
function with two integer parameters:\ an event code and a detail. This
function will be called upon network-generated events such as urgent
data arrival.  Macintosh documentation calls this the
\dfn{asynchronous service routine}.  In addition, it is called with
eventcode \code{MACTCP.PassiveOpenDone} when a \code{PassiveOpen}
completes. This is a Python addition to the MacTCP semantics.
It is safe to do further calls from \var{asr}.
\end{memberdesc}


\begin{methoddesc}[TCP Stream]{PassiveOpen}{port}
Wait for an incoming connection on TCP port \var{port} (zero makes the
system pick a free port). The call returns immediately, and you should
use \method{wait()} to wait for completion. You should not issue any method
calls other than \method{wait()}, \method{isdone()} or
\method{GetSockName()} before the call completes.
\end{methoddesc}

\begin{methoddesc}[TCP Stream]{wait}{}
Wait for \code{PassiveOpen} to complete.
\end{methoddesc}

\begin{methoddesc}[TCP Stream]{isdone}{}
Return \code{1} if a \code{PassiveOpen} has completed.
\end{methoddesc}

\begin{methoddesc}[TCP Stream]{GetSockName}{}
Return the TCP address of this side of a connection as a 2-tuple
\code{(\var{host}, \var{port})}, both integers.
\end{methoddesc}

\begin{methoddesc}[TCP Stream]{ActiveOpen}{lport, host, rport}
Open an outgoing connection to TCP address \code{(\var{host},
\var{rport})}. Use
local port \var{lport} (zero makes the system pick a free port). This
call blocks until the connection has been established.
\end{methoddesc}

\begin{methoddesc}[TCP Stream]{Send}{buf, push, urgent}
Send data \var{buf} over the connection. \var{push} and \var{urgent}
are flags as specified by the TCP standard.
\end{methoddesc}

\begin{methoddesc}[TCP Stream]{Rcv}{timeout}
Receive data. The call returns when \var{timeout} seconds have passed
or when (according to the MacTCP documentation) ``a reasonable amount
of data has been received''. The return value is a 3-tuple
\code{(\var{data}, \var{urgent}, \var{mark})}. If urgent data is
outstanding \code{Rcv} will always return that before looking at any
normal data. The first call returning urgent data will have the
\var{urgent} flag set, the last will have the \var{mark} flag set.
\end{methoddesc}

\begin{methoddesc}[TCP Stream]{Close}{}
Tell MacTCP that no more data will be transmitted on this
connection. The call returns when all data has been acknowledged by
the receiving side.
\end{methoddesc}

\begin{methoddesc}[TCP Stream]{Abort}{}
Forcibly close both sides of a connection, ignoring outstanding data.
\end{methoddesc}

\begin{methoddesc}[TCP Stream]{Status}{}
Return a TCP status object for this stream giving the current status
(see below).
\end{methoddesc}


\subsection{TCP Status Objects}

This object has no methods, only some members holding information on
the connection. A complete description of all fields in this objects
can be found in the Apple documentation. The most interesting ones are:

\begin{memberdesc}[TCP Status]{localHost}
\memberline{localPort}
\memberline{remoteHost}
\memberline{remotePort}
The integer IP-addresses and port numbers of both endpoints of the
connection. 
\end{memberdesc}

\begin{memberdesc}[TCP Status]{sendWindow}
The current window size.
\end{memberdesc}

\begin{memberdesc}[TCP Status]{amtUnackedData}
The number of bytes sent but not yet acknowledged. \code{sendWindow -
amtUnackedData} is what you can pass to \method{Send()} without
blocking.
\end{memberdesc}

\begin{memberdesc}[TCP Status]{amtUnreadData}
The number of bytes received but not yet read (what you can
\method{Recv()} without blocking).
\end{memberdesc}



\subsection{UDP Stream Objects}

Note that, unlike the name suggests, there is nothing stream-like
about UDP.


\begin{memberdesc}[UDP Stream]{asr}
\index{asynchronous service routine}
\index{service routine, asynchronous}
The asynchronous service routine to be called on events such as
datagram arrival without outstanding \code{Read} call. The \var{asr}
has a single argument, the event code.
\end{memberdesc}

\begin{memberdesc}[UDP Stream]{port}
A read-only member giving the port number of this UDP Stream.
\end{memberdesc}


\begin{methoddesc}[UDP Stream]{Read}{timeout}
Read a datagram, waiting at most \var{timeout} seconds (-1 is
infinite).  Return the data.
\end{methoddesc}

\begin{methoddesc}[UDP Stream]{Write}{host, port, buf}
Send \var{buf} as a datagram to IP-address \var{host}, port
\var{port}.
\end{methoddesc}

\section{\module{macspeech} ---
         Interface to the Macintosh Speech Manager}

\declaremodule{builtin}{macspeech}
  \platform{Mac}
\modulesynopsis{Interface to the Macintosh Speech Manager.}


This module provides an interface to the Macintosh Speech Manager,
\index{Macintosh Speech Manager}
\index{Speech Manager, Macintosh}
allowing you to let the Macintosh utter phrases. You need a version of
the Speech Manager extension (version 1 and 2 have been tested) in
your \file{Extensions} folder for this to work. The module does not
provide full access to all features of the Speech Manager.

This module is only available on MacOS9 and earlier in classic PPC
MacPython.

\begin{funcdesc}{Available}{}
Test availability of the Speech Manager extension (and, on the
PowerPC, the Speech Manager shared library). Return \code{0} or
\code{1}.
\end{funcdesc}

\begin{funcdesc}{Version}{}
Return the (integer) version number of the Speech Manager.
\end{funcdesc}

\begin{funcdesc}{SpeakString}{str}
Utter the string \var{str} using the default voice,
asynchronously. This aborts any speech that may still be active from
prior \function{SpeakString()} invocations.
\end{funcdesc}

\begin{funcdesc}{Busy}{}
Return the number of speech channels busy, system-wide.
\end{funcdesc}

\begin{funcdesc}{CountVoices}{}
Return the number of different voices available.
\end{funcdesc}

\begin{funcdesc}{GetIndVoice}{num}
Return a \pytype{Voice} object for voice number \var{num}.
\end{funcdesc}

\subsection{Voice Objects}
\label{voice-objects}

Voice objects contain the description of a voice. It is currently not
yet possible to access the parameters of a voice.

\setindexsubitem{(voice object method)}

\begin{methoddesc}[Voice]{GetGender}{}
Return the gender of the voice: \code{0} for male, \code{1} for female
and \code{-1} for neuter.
\end{methoddesc}

\begin{methoddesc}[Voice]{NewChannel}{}
Return a new Speech Channel object using this voice.
\end{methoddesc}

\subsection{Speech Channel Objects}
\label{speech-channel-objects}

A Speech Channel object allows you to speak strings with slightly more
control than \function{SpeakString()}, and allows you to use multiple
speakers at the same time. Please note that channel pitch and rate are
interrelated in some way, so that to make your Macintosh sing you will
have to adjust both.

\begin{methoddesc}[Speech Channel]{SpeakText}{str}
Start uttering the given string.
\end{methoddesc}

\begin{methoddesc}[Speech Channel]{Stop}{}
Stop babbling.
\end{methoddesc}

\begin{methoddesc}[Speech Channel]{GetPitch}{}
Return the current pitch of the channel, as a floating-point number.
\end{methoddesc}

\begin{methoddesc}[Speech Channel]{SetPitch}{pitch}
Set the pitch of the channel.
\end{methoddesc}

\begin{methoddesc}[Speech Channel]{GetRate}{}
Get the speech rate (utterances per minute) of the channel as a
floating point number.
\end{methoddesc}

\begin{methoddesc}[Speech Channel]{SetRate}{rate}
Set the speech rate of the channel.
\end{methoddesc}


\section{Standard Module \module{EasyDialogs}}
\label{module-EasyDialogs}
\stmodindex{EasyDialogs}

The \module{EasyDialogs} module contains some simple dialogs for
the Macintosh, modelled after the \module{stdwin} dialogs with similar
names. All routines have an optional parameter \var{id} with which you
can override the DLOG resource used for the dialog, as long as the
item numbers correspond. See the source for details.
 
The \module{EasyDialogs} module defines the following functions:


\begin{funcdesc}{Message}{str}
A modal dialog with the message text \var{str}, which should be at
most 255 characters long, is displayed. Control is returned when the
user clicks ``OK''.
\end{funcdesc}

\begin{funcdesc}{AskString}{prompt\optional{, default}}
Ask the user to input a string value, in a modal dialog. \var{prompt}
is the promt message, the optional \var{default} arg is the initial
value for the string. All strings can be at most 255 bytes
long. \function{AskString()} returns the string entered or \code{None}
in case the user cancelled.
\end{funcdesc}

\begin{funcdesc}{AskYesNoCancel}{question\optional{, default}}
Present a dialog with text \var{question} and three buttons labelled
``yes'', ``no'' and ``cancel''. Return \code{1} for yes, \code{0} for
no and \code{-1} for cancel. The default return value chosen by
hitting return is \code{0}. This can be changed with the optional
\var{default} argument.
\end{funcdesc}

\begin{funcdesc}{ProgressBar}{\optional{label\optional{, maxval}}}
Display a modeless progress dialog with a thermometer bar. \var{label}
is the text string displayed (default ``Working...''), \var{maxval} is
the value at which progress is complete (default \code{100}). The
returned object has one method, \code{set(\var{value})}, which sets
the value of the progress bar. The bar remains visible until the
object returned is discarded.

The progress bar has a ``cancel'' button, but it is currently
non-functional.
\end{funcdesc}

Note that \module{EasyDialogs} does not currently use the notification
manager. This means that displaying dialogs while the program is in
the background will lead to unexpected results and possibly
crashes. Also, all dialogs are modeless and hence expect to be at the
top of the stacking order. This is true when the dialogs are created,
but windows that pop-up later (like a console window) may also result
in crashes.

\section{\module{FrameWork} ---
         Interactive application framework}

\declaremodule{standard}{FrameWork}
  \platform{Mac}
\modulesynopsis{Interactive application framework.}


The \module{FrameWork} module contains classes that together provide a
framework for an interactive Macintosh application. The programmer
builds an application by creating subclasses that override various
methods of the bases classes, thereby implementing the functionality
wanted. Overriding functionality can often be done on various
different levels, i.e. to handle clicks in a single dialog window in a
non-standard way it is not necessary to override the complete event
handling.

Work on the \module{FrameWork} has pretty much stopped, now that
\module{PyObjC} is available for full Cocoa access from Python, and the
documentation describes only the most important functionality, and not
in the most logical manner at that. Examine the source or the examples
for more details.  The following are some comments posted on the
MacPython newsgroup about the strengths and limitations of
\module{FrameWork}:

\begin{quotation}
The strong point of \module{FrameWork} is that it allows you to break
into the control-flow at many different places. \refmodule{W}, for
instance, uses a different way to enable/disable menus and that plugs
right in leaving the rest intact.  The weak points of
\module{FrameWork} are that it has no abstract command interface (but
that shouldn't be difficult), that it's dialog support is minimal and
that it's control/toolbar support is non-existent.
\end{quotation}


The \module{FrameWork} module defines the following functions:


\begin{funcdesc}{Application}{}
An object representing the complete application. See below for a
description of the methods. The default \method{__init__()} routine
creates an empty window dictionary and a menu bar with an apple menu.
\end{funcdesc}

\begin{funcdesc}{MenuBar}{}
An object representing the menubar. This object is usually not created
by the user.
\end{funcdesc}

\begin{funcdesc}{Menu}{bar, title\optional{, after}}
An object representing a menu. Upon creation you pass the
\code{MenuBar} the menu appears in, the \var{title} string and a
position (1-based) \var{after} where the menu should appear (default:
at the end).
\end{funcdesc}

\begin{funcdesc}{MenuItem}{menu, title\optional{, shortcut, callback}}
Create a menu item object. The arguments are the menu to create, the
item title string and optionally the keyboard shortcut
and a callback routine. The callback is called with the arguments
menu-id, item number within menu (1-based), current front window and
the event record.

Instead of a callable object the callback can also be a string. In
this case menu selection causes the lookup of a method in the topmost
window and the application. The method name is the callback string
with \code{'domenu_'} prepended.

Calling the \code{MenuBar} \method{fixmenudimstate()} method sets the
correct dimming for all menu items based on the current front window.
\end{funcdesc}

\begin{funcdesc}{Separator}{menu}
Add a separator to the end of a menu.
\end{funcdesc}

\begin{funcdesc}{SubMenu}{menu, label}
Create a submenu named \var{label} under menu \var{menu}. The menu
object is returned.
\end{funcdesc}

\begin{funcdesc}{Window}{parent}
Creates a (modeless) window. \var{Parent} is the application object to
which the window belongs. The window is not displayed until later.
\end{funcdesc}

\begin{funcdesc}{DialogWindow}{parent}
Creates a modeless dialog window.
\end{funcdesc}

\begin{funcdesc}{windowbounds}{width, height}
Return a \code{(\var{left}, \var{top}, \var{right}, \var{bottom})}
tuple suitable for creation of a window of given width and height. The
window will be staggered with respect to previous windows, and an
attempt is made to keep the whole window on-screen. However, the window will
however always be the exact size given, so parts may be offscreen.
\end{funcdesc}

\begin{funcdesc}{setwatchcursor}{}
Set the mouse cursor to a watch.
\end{funcdesc}

\begin{funcdesc}{setarrowcursor}{}
Set the mouse cursor to an arrow.
\end{funcdesc}


\subsection{Application Objects \label{application-objects}}

Application objects have the following methods, among others:


\begin{methoddesc}[Application]{makeusermenus}{}
Override this method if you need menus in your application. Append the
menus to the attribute \member{menubar}.
\end{methoddesc}

\begin{methoddesc}[Application]{getabouttext}{}
Override this method to return a text string describing your
application.  Alternatively, override the \method{do_about()} method
for more elaborate ``about'' messages.
\end{methoddesc}

\begin{methoddesc}[Application]{mainloop}{\optional{mask\optional{, wait}}}
This routine is the main event loop, call it to set your application
rolling. \var{Mask} is the mask of events you want to handle,
\var{wait} is the number of ticks you want to leave to other
concurrent application (default 0, which is probably not a good
idea). While raising \var{self} to exit the mainloop is still
supported it is not recommended: call \code{self._quit()} instead.

The event loop is split into many small parts, each of which can be
overridden. The default methods take care of dispatching events to
windows and dialogs, handling drags and resizes, Apple Events, events
for non-FrameWork windows, etc.

In general, all event handlers should return \code{1} if the event is fully
handled and \code{0} otherwise (because the front window was not a FrameWork
window, for instance). This is needed so that update events and such
can be passed on to other windows like the Sioux console window.
Calling \function{MacOS.HandleEvent()} is not allowed within
\var{our_dispatch} or its callees, since this may result in an
infinite loop if the code is called through the Python inner-loop
event handler.
\end{methoddesc}

\begin{methoddesc}[Application]{asyncevents}{onoff}
Call this method with a nonzero parameter to enable
asynchronous event handling. This will tell the inner interpreter loop
to call the application event handler \var{async_dispatch} whenever events
are available. This will cause FrameWork window updates and the user
interface to remain working during long computations, but will slow the
interpreter down and may cause surprising results in non-reentrant code
(such as FrameWork itself). By default \var{async_dispatch} will immediately
call \var{our_dispatch} but you may override this to handle only certain
events asynchronously. Events you do not handle will be passed to Sioux
and such.

The old on/off value is returned.
\end{methoddesc}

\begin{methoddesc}[Application]{_quit}{}
Terminate the running \method{mainloop()} call at the next convenient
moment.
\end{methoddesc}

\begin{methoddesc}[Application]{do_char}{c, event}
The user typed character \var{c}. The complete details of the event
can be found in the \var{event} structure. This method can also be
provided in a \code{Window} object, which overrides the
application-wide handler if the window is frontmost.
\end{methoddesc}

\begin{methoddesc}[Application]{do_dialogevent}{event}
Called early in the event loop to handle modeless dialog events. The
default method simply dispatches the event to the relevant dialog (not
through the \code{DialogWindow} object involved). Override if you
need special handling of dialog events (keyboard shortcuts, etc).
\end{methoddesc}

\begin{methoddesc}[Application]{idle}{event}
Called by the main event loop when no events are available. The
null-event is passed (so you can look at mouse position, etc).
\end{methoddesc}


\subsection{Window Objects \label{window-objects}}

Window objects have the following methods, among others:

\setindexsubitem{(Window method)}

\begin{methoddesc}[Window]{open}{}
Override this method to open a window. Store the MacOS window-id in
\member{self.wid} and call the \method{do_postopen()} method to
register the window with the parent application.
\end{methoddesc}

\begin{methoddesc}[Window]{close}{}
Override this method to do any special processing on window
close. Call the \method{do_postclose()} method to cleanup the parent
state.
\end{methoddesc}

\begin{methoddesc}[Window]{do_postresize}{width, height, macoswindowid}
Called after the window is resized. Override if more needs to be done
than calling \code{InvalRect}.
\end{methoddesc}

\begin{methoddesc}[Window]{do_contentclick}{local, modifiers, event}
The user clicked in the content part of a window. The arguments are
the coordinates (window-relative), the key modifiers and the raw
event.
\end{methoddesc}

\begin{methoddesc}[Window]{do_update}{macoswindowid, event}
An update event for the window was received. Redraw the window.
\end{methoddesc}

\begin{methoddesc}{do_activate}{activate, event}
The window was activated (\code{\var{activate} == 1}) or deactivated
(\code{\var{activate} == 0}). Handle things like focus highlighting,
etc.
\end{methoddesc}


\subsection{ControlsWindow Object \label{controlswindow-object}}

ControlsWindow objects have the following methods besides those of
\code{Window} objects:


\begin{methoddesc}[ControlsWindow]{do_controlhit}{window, control,
                                                  pcode, event}
Part \var{pcode} of control \var{control} was hit by the
user. Tracking and such has already been taken care of.
\end{methoddesc}


\subsection{ScrolledWindow Object \label{scrolledwindow-object}}

ScrolledWindow objects are ControlsWindow objects with the following
extra methods:


\begin{methoddesc}[ScrolledWindow]{scrollbars}{\optional{wantx\optional{,
                                               wanty}}}
Create (or destroy) horizontal and vertical scrollbars. The arguments
specify which you want (default: both). The scrollbars always have
minimum \code{0} and maximum \code{32767}.
\end{methoddesc}

\begin{methoddesc}[ScrolledWindow]{getscrollbarvalues}{}
You must supply this method. It should return a tuple \code{(\var{x},
\var{y})} giving the current position of the scrollbars (between
\code{0} and \code{32767}). You can return \code{None} for either to
indicate the whole document is visible in that direction.
\end{methoddesc}

\begin{methoddesc}[ScrolledWindow]{updatescrollbars}{}
Call this method when the document has changed. It will call
\method{getscrollbarvalues()} and update the scrollbars.
\end{methoddesc}

\begin{methoddesc}[ScrolledWindow]{scrollbar_callback}{which, what, value}
Supplied by you and called after user interaction. \var{which} will
be \code{'x'} or \code{'y'}, \var{what} will be \code{'-'},
\code{'--'}, \code{'set'}, \code{'++'} or \code{'+'}. For
\code{'set'}, \var{value} will contain the new scrollbar position.
\end{methoddesc}

\begin{methoddesc}[ScrolledWindow]{scalebarvalues}{absmin, absmax,
                                                   curmin, curmax}
Auxiliary method to help you calculate values to return from
\method{getscrollbarvalues()}. You pass document minimum and maximum value
and topmost (leftmost) and bottommost (rightmost) visible values and
it returns the correct number or \code{None}.
\end{methoddesc}

\begin{methoddesc}[ScrolledWindow]{do_activate}{onoff, event}
Takes care of dimming/highlighting scrollbars when a window becomes
frontmost. If you override this method, call this one at the end of
your method.
\end{methoddesc}

\begin{methoddesc}[ScrolledWindow]{do_postresize}{width, height, window}
Moves scrollbars to the correct position. Call this method initially
if you override it.
\end{methoddesc}

\begin{methoddesc}[ScrolledWindow]{do_controlhit}{window, control,
                                                  pcode, event}
Handles scrollbar interaction. If you override it call this method
first, a nonzero return value indicates the hit was in the scrollbars
and has been handled.
\end{methoddesc}


\subsection{DialogWindow Objects \label{dialogwindow-objects}}

DialogWindow objects have the following methods besides those of
\code{Window} objects:


\begin{methoddesc}[DialogWindow]{open}{resid}
Create the dialog window, from the DLOG resource with id
\var{resid}. The dialog object is stored in \member{self.wid}.
\end{methoddesc}

\begin{methoddesc}[DialogWindow]{do_itemhit}{item, event}
Item number \var{item} was hit. You are responsible for redrawing
toggle buttons, etc.
\end{methoddesc}

\section{Standard Module \sectcode{MiniAEFrame}}
\stmodindex{MiniAEFrame}
\label{module-MiniAEFrame}

The module \var{MiniAEFrame} provides a framework for an application
that can function as an OSA server, i.e. receive and process
AppleEvents. It can be used in conjunction with \var{FrameWork} or
standalone.

This module is temporary, it will eventually be replaced by a module
that handles argument names better and possibly automates making your
application scriptable.

The \var{MiniAEFrame} module defines the following classes:

\setindexsubitem{(in module MiniAEFrame)}

\begin{funcdesc}{AEServer}{}
A class that handles AppleEvent dispatch. Your application should
subclass this class together with either
\code{MiniAEFrame.MiniApplication} or
\code{FrameWork.Application}. Your \code{__init__} method should call
the \code{__init__} method for both classes.
\end{funcdesc}

\begin{funcdesc}{MiniApplication}{}
A class that is more or less compatible with
\code{FrameWork.Application} but with less functionality. Its
eventloop supports the apple menu, command-dot and AppleEvents, other
events are passed on to the Python interpreter and/or Sioux.
Useful if your application wants to use \code{AEServer} but does not
provide its own windows, etc.
\end{funcdesc}

\subsection{AEServer Objects}

\setindexsubitem{(AEServer method)}

\begin{funcdesc}{installaehandler}{classe, type, callback}
Installs an AppleEvent handler. \code{Classe} and \code{type} are the
four-char OSA Class and Type designators, \code{'****'} wildcards are
allowed. When a matching AppleEvent is received the parameters are
decoded and your callback is invoked.
\end{funcdesc}

\begin{funcdesc}{callback}{_object, **kwargs}
Your callback is called with the OSA Direct Object as first positional
parameter. The other parameters are passed as keyword arguments, with
the 4-char designator as name. Three extra keyword parameters are
passed: \code{_class} and \code{_type} are the Class and Type
designators and \code{_attributes} is a dictionary with the AppleEvent
attributes.

The return value of your method is packed with
\code{aetools.packevent} and sent as reply.
\end{funcdesc}

Note that there are some serious problems with the current
design. AppleEvents which have non-identifier 4-char designators for
arguments are not implementable, and it is not possible to return an
error to the originator. This will be addressed in a future release.


%\chapter{Standard Windowing Interface}

The modules in this chapter are available only on those systems where
the STDWIN library is available.  STDWIN runs on \UNIX{} under X11 and
on the Macintosh.  See CWI report CS-R8817.

\strong{Warning:} Using STDWIN is not recommended for new
applications.  It has never been ported to Microsoft Windows or
Windows NT, and for X11 or the Macintosh it lacks important
functionality --- in particular, it has no tools for the construction
of dialogs.  For most platforms, alternative, native solutions exist
(though none are currently documented in this manual): Tkinter for
\UNIX{} under X11, native Xt with Motif or Athena widgets for \UNIX{}
under X11, Win32 for Windows and Windows NT, and a collection of
native toolkit interfaces for the Macintosh.

\section{Built-in Module \sectcode{stdwin}}
\bimodindex{stdwin}

This module defines several new object types and functions that
provide access to the functionality of STDWIN.

On Unix running X11, it can only be used if the \code{DISPLAY}
environment variable is set or an explicit \samp{-display
\var{displayname}} argument is passed to the Python interpreter.

Functions have names that usually resemble their C STDWIN counterparts
with the initial `w' dropped.
Points are represented by pairs of integers; rectangles
by pairs of points.
For a complete description of STDWIN please refer to the documentation
of STDWIN for C programmers (aforementioned CWI report).

\subsection{Functions Defined in Module \sectcode{stdwin}}
\nodename{STDWIN Functions}

The following functions are defined in the \code{stdwin} module:

\renewcommand{\indexsubitem}{(in module stdwin)}
\begin{funcdesc}{open}{title}
Open a new window whose initial title is given by the string argument.
Return a window object; window object methods are described below.%
\footnote{The Python version of STDWIN does not support draw procedures; all
	drawing requests are reported as draw events.}
\end{funcdesc}

\begin{funcdesc}{getevent}{}
Wait for and return the next event.
An event is returned as a triple: the first element is the event
type, a small integer; the second element is the window object to which
the event applies, or
\code{None}
if it applies to no window in particular;
the third element is type-dependent.
Names for event types and command codes are defined in the standard
module
\code{stdwinevent}.
\end{funcdesc}

\begin{funcdesc}{pollevent}{}
Return the next event, if one is immediately available.
If no event is available, return \code{()}.
\end{funcdesc}

\begin{funcdesc}{getactive}{}
Return the window that is currently active, or \code{None} if no
window is currently active.  (This can be emulated by monitoring
WE_ACTIVATE and WE_DEACTIVATE events.)
\end{funcdesc}

\begin{funcdesc}{listfontnames}{pattern}
Return the list of font names in the system that match the pattern (a
string).  The pattern should normally be \code{'*'}; returns all
available fonts.  If the underlying window system is X11, other
patterns follow the standard X11 font selection syntax (as used e.g.
in resource definitions), i.e. the wildcard character \code{'*'}
matches any sequence of characters (including none) and \code{'?'}
matches any single character.
On the Macintosh this function currently returns an empty list.
\end{funcdesc}

\begin{funcdesc}{setdefscrollbars}{hflag\, vflag}
Set the flags controlling whether subsequently opened windows will
have horizontal and/or vertical scroll bars.
\end{funcdesc}

\begin{funcdesc}{setdefwinpos}{h\, v}
Set the default window position for windows opened subsequently.
\end{funcdesc}

\begin{funcdesc}{setdefwinsize}{width\, height}
Set the default window size for windows opened subsequently.
\end{funcdesc}

\begin{funcdesc}{getdefscrollbars}{}
Return the flags controlling whether subsequently opened windows will
have horizontal and/or vertical scroll bars.
\end{funcdesc}

\begin{funcdesc}{getdefwinpos}{}
Return the default window position for windows opened subsequently.
\end{funcdesc}

\begin{funcdesc}{getdefwinsize}{}
Return the default window size for windows opened subsequently.
\end{funcdesc}

\begin{funcdesc}{getscrsize}{}
Return the screen size in pixels.
\end{funcdesc}

\begin{funcdesc}{getscrmm}{}
Return the screen size in millimeters.
\end{funcdesc}

\begin{funcdesc}{fetchcolor}{colorname}
Return the pixel value corresponding to the given color name.
Return the default foreground color for unknown color names.
Hint: the following code tests whether you are on a machine that
supports more than two colors:
\bcode\begin{verbatim}
if stdwin.fetchcolor('black') <> \
          stdwin.fetchcolor('red') <> \
          stdwin.fetchcolor('white'):
    print 'color machine'
else:
    print 'monochrome machine'
\end{verbatim}\ecode
\end{funcdesc}

\begin{funcdesc}{setfgcolor}{pixel}
Set the default foreground color.
This will become the default foreground color of windows opened
subsequently, including dialogs.
\end{funcdesc}

\begin{funcdesc}{setbgcolor}{pixel}
Set the default background color.
This will become the default background color of windows opened
subsequently, including dialogs.
\end{funcdesc}

\begin{funcdesc}{getfgcolor}{}
Return the pixel value of the current default foreground color.
\end{funcdesc}

\begin{funcdesc}{getbgcolor}{}
Return the pixel value of the current default background color.
\end{funcdesc}

\begin{funcdesc}{setfont}{fontname}
Set the current default font.
This will become the default font for windows opened subsequently,
and is also used by the text measuring functions \code{textwidth},
\code{textbreak}, \code{lineheight} and \code{baseline} below.
This accepts two more optional parameters, size and style:
Size is the font size (in `points').
Style is a single character specifying the style, as follows:
\code{'b'} = bold,
\code{'i'} = italic,
\code{'o'} = bold + italic,
\code{'u'} = underline;
default style is roman.
Size and style are ignored under X11 but used on the Macintosh.
(Sorry for all this complexity --- a more uniform interface is being designed.)
\end{funcdesc}

\begin{funcdesc}{menucreate}{title}
Create a menu object referring to a global menu (a menu that appears in
all windows).
Methods of menu objects are described below.
Note: normally, menus are created locally; see the window method
\code{menucreate} below.
\strong{Warning:} the menu only appears in a window as long as the object
returned by this call exists.
\end{funcdesc}

\begin{funcdesc}{newbitmap}{width\, height}
Create a new bitmap object of the given dimensions.
Methods of bitmap objects are described below.
Not available on the Macintosh.
\end{funcdesc}

\begin{funcdesc}{fleep}{}
Cause a beep or bell (or perhaps a `visual bell' or flash, hence the
name).
\end{funcdesc}

\begin{funcdesc}{message}{string}
Display a dialog box containing the string.
The user must click OK before the function returns.
\end{funcdesc}

\begin{funcdesc}{askync}{prompt\, default}
Display a dialog that prompts the user to answer a question with yes or
no.
Return 0 for no, 1 for yes.
If the user hits the Return key, the default (which must be 0 or 1) is
returned.
If the user cancels the dialog, the
\code{KeyboardInterrupt}
exception is raised.
\end{funcdesc}

\begin{funcdesc}{askstr}{prompt\, default}
Display a dialog that prompts the user for a string.
If the user hits the Return key, the default string is returned.
If the user cancels the dialog, the
\code{KeyboardInterrupt}
exception is raised.
\end{funcdesc}

\begin{funcdesc}{askfile}{prompt\, default\, new}
Ask the user to specify a filename.
If
\var{new}
is zero it must be an existing file; otherwise, it must be a new file.
If the user cancels the dialog, the
\code{KeyboardInterrupt}
exception is raised.
\end{funcdesc}

\begin{funcdesc}{setcutbuffer}{i\, string}
Store the string in the system's cut buffer number
\var{i},
where it can be found (for pasting) by other applications.
On X11, there are 8 cut buffers (numbered 0..7).
Cut buffer number 0 is the `clipboard' on the Macintosh.
\end{funcdesc}

\begin{funcdesc}{getcutbuffer}{i}
Return the contents of the system's cut buffer number
\var{i}.
\end{funcdesc}

\begin{funcdesc}{rotatecutbuffers}{n}
On X11, rotate the 8 cut buffers by
\var{n}.
Ignored on the Macintosh.
\end{funcdesc}

\begin{funcdesc}{getselection}{i}
Return X11 selection number
\var{i.}
Selections are not cut buffers.
Selection numbers are defined in module
\code{stdwinevents}.
Selection \code{WS_PRIMARY} is the
\dfn{primary}
selection (used by
xterm,
for instance);
selection \code{WS_SECONDARY} is the
\dfn{secondary}
selection; selection \code{WS_CLIPBOARD} is the
\dfn{clipboard}
selection (used by
xclipboard).
On the Macintosh, this always returns an empty string.
\end{funcdesc}

\begin{funcdesc}{resetselection}{i}
Reset selection number
\var{i},
if this process owns it.
(See window method
\code{setselection()}).
\end{funcdesc}

\begin{funcdesc}{baseline}{}
Return the baseline of the current font (defined by STDWIN as the
vertical distance between the baseline and the top of the
characters).
\end{funcdesc}

\begin{funcdesc}{lineheight}{}
Return the total line height of the current font.
\end{funcdesc}

\begin{funcdesc}{textbreak}{str\, width}
Return the number of characters of the string that fit into a space of
\var{width}
bits wide when drawn in the curent font.
\end{funcdesc}

\begin{funcdesc}{textwidth}{str}
Return the width in bits of the string when drawn in the current font.
\end{funcdesc}

\begin{funcdesc}{connectionnumber}{}
\funcline{fileno}{}
(X11 under \UNIX{} only) Return the ``connection number'' used by the
underlying X11 implementation.  (This is normally the file number of
the socket.)  Both functions return the same value;
\code{connectionnumber()} is named after the corresponding function in
X11 and STDWIN, while \code{fileno()} makes it possible to use the
\code{stdwin} module as a ``file'' object parameter to
\code{select.select()}.  Note that if \code{select()} implies that
input is possible on \code{stdwin}, this does not guarantee that an
event is ready --- it may be some internal communication going on
between the X server and the client library.  Thus, you should call
\code{stdwin.pollevent()} until it returns \code{None} to check for
events if you don't want your program to block.  Because of internal
buffering in X11, it is also possible that \code{stdwin.pollevent()}
returns an event while \code{select()} does not find \code{stdwin} to
be ready, so you should read any pending events with
\code{stdwin.pollevent()} until it returns \code{None} before entering
a blocking \code{select()} call.
\ttindex{select}
\end{funcdesc}

\subsection{Window Objects}
\nodename{STDWIN Window Objects}

Window objects are created by \code{stdwin.open()}.  They are closed
by their \code{close()} method or when they are garbage-collected.
Window objects have the following methods:

\renewcommand{\indexsubitem}{(window method)}

\begin{funcdesc}{begindrawing}{}
Return a drawing object, whose methods (described below) allow drawing
in the window.
\end{funcdesc}

\begin{funcdesc}{change}{rect}
Invalidate the given rectangle; this may cause a draw event.
\end{funcdesc}

\begin{funcdesc}{gettitle}{}
Returns the window's title string.
\end{funcdesc}

\begin{funcdesc}{getdocsize}{}
\begin{sloppypar}
Return a pair of integers giving the size of the document as set by
\code{setdocsize()}.
\end{sloppypar}
\end{funcdesc}

\begin{funcdesc}{getorigin}{}
Return a pair of integers giving the origin of the window with respect
to the document.
\end{funcdesc}

\begin{funcdesc}{gettitle}{}
Return the window's title string.
\end{funcdesc}

\begin{funcdesc}{getwinsize}{}
Return a pair of integers giving the size of the window.
\end{funcdesc}

\begin{funcdesc}{getwinpos}{}
Return a pair of integers giving the position of the window's upper
left corner (relative to the upper left corner of the screen).
\end{funcdesc}

\begin{funcdesc}{menucreate}{title}
Create a menu object referring to a local menu (a menu that appears
only in this window).
Methods of menu objects are described below.
{\bf Warning:} the menu only appears as long as the object
returned by this call exists.
\end{funcdesc}

\begin{funcdesc}{scroll}{rect\, point}
Scroll the given rectangle by the vector given by the point.
\end{funcdesc}

\begin{funcdesc}{setdocsize}{point}
Set the size of the drawing document.
\end{funcdesc}

\begin{funcdesc}{setorigin}{point}
Move the origin of the window (its upper left corner)
to the given point in the document.
\end{funcdesc}

\begin{funcdesc}{setselection}{i\, str}
Attempt to set X11 selection number
\var{i}
to the string
\var{str}.
(See stdwin method
\code{getselection()}
for the meaning of
\var{i}.)
Return true if it succeeds.
If  succeeds, the window ``owns'' the selection until
(a) another application takes ownership of the selection; or
(b) the window is deleted; or
(c) the application clears ownership by calling
\code{stdwin.resetselection(\var{i})}.
When another application takes ownership of the selection, a
\code{WE_LOST_SEL}
event is received for no particular window and with the selection number
as detail.
Ignored on the Macintosh.
\end{funcdesc}

\begin{funcdesc}{settimer}{dsecs}
Schedule a timer event for the window in
\code{\var{dsecs}/10}
seconds.
\end{funcdesc}

\begin{funcdesc}{settitle}{title}
Set the window's title string.
\end{funcdesc}

\begin{funcdesc}{setwincursor}{name}
\begin{sloppypar}
Set the window cursor to a cursor of the given name.
It raises the
\code{RuntimeError}
exception if no cursor of the given name exists.
Suitable names include
\code{'ibeam'},
\code{'arrow'},
\code{'cross'},
\code{'watch'}
and
\code{'plus'}.
On X11, there are many more (see
\file{<X11/cursorfont.h>}).
\end{sloppypar}
\end{funcdesc}

\begin{funcdesc}{setwinpos}{h\, v}
Set the the position of the window's upper left corner (relative to
the upper left corner of the screen).
\end{funcdesc}

\begin{funcdesc}{setwinsize}{width\, height}
Set the window's size.
\end{funcdesc}

\begin{funcdesc}{show}{rect}
Try to ensure that the given rectangle of the document is visible in
the window.
\end{funcdesc}

\begin{funcdesc}{textcreate}{rect}
Create a text-edit object in the document at the given rectangle.
Methods of text-edit objects are described below.
\end{funcdesc}

\begin{funcdesc}{setactive}{}
Attempt to make this window the active window.  If successful, this
will generate a WE_ACTIVATE event (and a WE_DEACTIVATE event in case
another window in this application became inactive).
\end{funcdesc}

\begin{funcdesc}{close}{}
Discard the window object.  It should not be used again.
\end{funcdesc}

\subsection{Drawing Objects}

Drawing objects are created exclusively by the window method
\code{begindrawing()}.
Only one drawing object can exist at any given time; the drawing object
must be deleted to finish drawing.
No drawing object may exist when
\code{stdwin.getevent()}
is called.
Drawing objects have the following methods:

\renewcommand{\indexsubitem}{(drawing method)}

\begin{funcdesc}{box}{rect}
Draw a box just inside a rectangle.
\end{funcdesc}

\begin{funcdesc}{circle}{center\, radius}
Draw a circle with given center point and radius.
\end{funcdesc}

\begin{funcdesc}{elarc}{center\, \(rh\, rv\)\, \(a1\, a2\)}
Draw an elliptical arc with given center point.
\code{(\var{rh}, \var{rv})}
gives the half sizes of the horizontal and vertical radii.
\code{(\var{a1}, \var{a2})}
gives the angles (in degrees) of the begin and end points.
0 degrees is at 3 o'clock, 90 degrees is at 12 o'clock.
\end{funcdesc}

\begin{funcdesc}{erase}{rect}
Erase a rectangle.
\end{funcdesc}

\begin{funcdesc}{fillcircle}{center\, radius}
Draw a filled circle with given center point and radius.
\end{funcdesc}

\begin{funcdesc}{fillelarc}{center\, \(rh\, rv\)\, \(a1\, a2\)}
Draw a filled elliptical arc; arguments as for \code{elarc}.
\end{funcdesc}

\begin{funcdesc}{fillpoly}{points}
Draw a filled polygon given by a list (or tuple) of points.
\end{funcdesc}

\begin{funcdesc}{invert}{rect}
Invert a rectangle.
\end{funcdesc}

\begin{funcdesc}{line}{p1\, p2}
Draw a line from point
\var{p1}
to
\var{p2}.
\end{funcdesc}

\begin{funcdesc}{paint}{rect}
Fill a rectangle.
\end{funcdesc}

\begin{funcdesc}{poly}{points}
Draw the lines connecting the given list (or tuple) of points.
\end{funcdesc}

\begin{funcdesc}{shade}{rect\, percent}
Fill a rectangle with a shading pattern that is about
\var{percent}
percent filled.
\end{funcdesc}

\begin{funcdesc}{text}{p\, str}
Draw a string starting at point p (the point specifies the
top left coordinate of the string).
\end{funcdesc}

\begin{funcdesc}{xorcircle}{center\, radius}
\funcline{xorelarc}{center\, \(rh\, rv\)\, \(a1\, a2\)}
\funcline{xorline}{p1\, p2}
\funcline{xorpoly}{points}
Draw a circle, an elliptical arc, a line or a polygon, respectively,
in XOR mode.
\end{funcdesc}

\begin{funcdesc}{setfgcolor}{}
\funcline{setbgcolor}{}
\funcline{getfgcolor}{}
\funcline{getbgcolor}{}
These functions are similar to the corresponding functions described
above for the
\code{stdwin}
module, but affect or return the colors currently used for drawing
instead of the global default colors.
When a drawing object is created, its colors are set to the window's
default colors, which are in turn initialized from the global default
colors when the window is created.
\end{funcdesc}

\begin{funcdesc}{setfont}{}
\funcline{baseline}{}
\funcline{lineheight}{}
\funcline{textbreak}{}
\funcline{textwidth}{}
These functions are similar to the corresponding functions described
above for the
\code{stdwin}
module, but affect or use the current drawing font instead of
the global default font.
When a drawing object is created, its font is set to the window's
default font, which is in turn initialized from the global default
font when the window is created.
\end{funcdesc}

\begin{funcdesc}{bitmap}{point\, bitmap\, mask}
Draw the \var{bitmap} with its top left corner at \var{point}.
If the optional \var{mask} argument is present, it should be either
the same object as \var{bitmap}, to draw only those bits that are set
in the bitmap, in the foreground color, or \code{None}, to draw all
bits (ones are drawn in the foreground color, zeros in the background
color).
Not available on the Macintosh.
\end{funcdesc}

\begin{funcdesc}{cliprect}{rect}
Set the ``clipping region'' to a rectangle.
The clipping region limits the effect of all drawing operations, until
it is changed again or until the drawing object is closed.  When a
drawing object is created the clipping region is set to the entire
window.  When an object to be drawn falls partly outside the clipping
region, the set of pixels drawn is the intersection of the clipping
region and the set of pixels that would be drawn by the same operation
in the absence of a clipping region.
\end{funcdesc}

\begin{funcdesc}{noclip}{}
Reset the clipping region to the entire window.
\end{funcdesc}

\begin{funcdesc}{close}{}
\funcline{enddrawing}{}
Discard the drawing object.  It should not be used again.
\end{funcdesc}

\subsection{Menu Objects}

A menu object represents a menu.
The menu is destroyed when the menu object is deleted.
The following methods are defined:

\renewcommand{\indexsubitem}{(menu method)}

\begin{funcdesc}{additem}{text\, shortcut}
Add a menu item with given text.
The shortcut must be a string of length 1, or omitted (to specify no
shortcut).
\end{funcdesc}

\begin{funcdesc}{setitem}{i\, text}
Set the text of item number
\var{i}.
\end{funcdesc}

\begin{funcdesc}{enable}{i\, flag}
Enable or disables item
\var{i}.
\end{funcdesc}

\begin{funcdesc}{check}{i\, flag}
Set or clear the
\dfn{check mark}
for item
\var{i}.
\end{funcdesc}

\begin{funcdesc}{close}{}
Discard the menu object.  It should not be used again.
\end{funcdesc}

\subsection{Bitmap Objects}

A bitmap represents a rectangular array of bits.
The top left bit has coordinate (0, 0).
A bitmap can be drawn with the \code{bitmap} method of a drawing object.
Bitmaps are currently not available on the Macintosh.

The following methods are defined:

\renewcommand{\indexsubitem}{(bitmap method)}

\begin{funcdesc}{getsize}{}
Return a tuple representing the width and height of the bitmap.
(This returns the values that have been passed to the \code{newbitmap}
function.)
\end{funcdesc}

\begin{funcdesc}{setbit}{point\, bit}
Set the value of the bit indicated by \var{point} to \var{bit}.
\end{funcdesc}

\begin{funcdesc}{getbit}{point}
Return the value of the bit indicated by \var{point}.
\end{funcdesc}

\begin{funcdesc}{close}{}
Discard the bitmap object.  It should not be used again.
\end{funcdesc}

\subsection{Text-edit Objects}

A text-edit object represents a text-edit block.
For semantics, see the STDWIN documentation for C programmers.
The following methods exist:

\renewcommand{\indexsubitem}{(text-edit method)}

\begin{funcdesc}{arrow}{code}
Pass an arrow event to the text-edit block.
The
\var{code}
must be one of
\code{WC_LEFT},
\code{WC_RIGHT},
\code{WC_UP}
or
\code{WC_DOWN}
(see module
\code{stdwinevents}).
\end{funcdesc}

\begin{funcdesc}{draw}{rect}
Pass a draw event to the text-edit block.
The rectangle specifies the redraw area.
\end{funcdesc}

\begin{funcdesc}{event}{type\, window\, detail}
Pass an event gotten from
\code{stdwin.getevent()}
to the text-edit block.
Return true if the event was handled.
\end{funcdesc}

\begin{funcdesc}{getfocus}{}
Return 2 integers representing the start and end positions of the
focus, usable as slice indices on the string returned by
\code{gettext()}.
\end{funcdesc}

\begin{funcdesc}{getfocustext}{}
Return the text in the focus.
\end{funcdesc}

\begin{funcdesc}{getrect}{}
Return a rectangle giving the actual position of the text-edit block.
(The bottom coordinate may differ from the initial position because
the block automatically shrinks or grows to fit.)
\end{funcdesc}

\begin{funcdesc}{gettext}{}
Return the entire text buffer.
\end{funcdesc}

\begin{funcdesc}{move}{rect}
Specify a new position for the text-edit block in the document.
\end{funcdesc}

\begin{funcdesc}{replace}{str}
Replace the text in the focus by the given string.
The new focus is an insert point at the end of the string.
\end{funcdesc}

\begin{funcdesc}{setfocus}{i\, j}
Specify the new focus.
Out-of-bounds values are silently clipped.
\end{funcdesc}

\begin{funcdesc}{settext}{str}
Replace the entire text buffer by the given string and set the focus
to \code{(0, 0)}.
\end{funcdesc}

\begin{funcdesc}{setview}{rect}
Set the view rectangle to \var{rect}.  If \var{rect} is \code{None},
viewing mode is reset.  In viewing mode, all output from the text-edit
object is clipped to the viewing rectangle.  This may be useful to
implement your own scrolling text subwindow.
\end{funcdesc}

\begin{funcdesc}{close}{}
Discard the text-edit object.  It should not be used again.
\end{funcdesc}

\subsection{Example}
\nodename{STDWIN Example}

Here is a minimal example of using STDWIN in Python.
It creates a window and draws the string ``Hello world'' in the top
left corner of the window.
The window will be correctly redrawn when covered and re-exposed.
The program quits when the close icon or menu item is requested.

\bcode\begin{verbatim}
import stdwin
from stdwinevents import *

def main():
    mywin = stdwin.open('Hello')
    #
    while 1:
        (type, win, detail) = stdwin.getevent()
        if type == WE_DRAW:
            draw = win.begindrawing()
            draw.text((0, 0), 'Hello, world')
            del draw
        elif type == WE_CLOSE:
            break

main()
\end{verbatim}\ecode

\section{Standard Module \sectcode{stdwinevents}}
\stmodindex{stdwinevents}

This module defines constants used by STDWIN for event types
(\code{WE_ACTIVATE} etc.), command codes (\code{WC_LEFT} etc.)
and selection types (\code{WS_PRIMARY} etc.).
Read the file for details.
Suggested usage is

\bcode\begin{verbatim}
>>> from stdwinevents import *
>>> 
\end{verbatim}\ecode

\section{Standard Module \sectcode{rect}}
\stmodindex{rect}

This module contains useful operations on rectangles.
A rectangle is defined as in module
\code{stdwin}:
a pair of points, where a point is a pair of integers.
For example, the rectangle

\bcode\begin{verbatim}
(10, 20), (90, 80)
\end{verbatim}\ecode

is a rectangle whose left, top, right and bottom edges are 10, 20, 90
and 80, respectively.
Note that the positive vertical axis points down (as in
\code{stdwin}).

The module defines the following objects:

\renewcommand{\indexsubitem}{(in module rect)}
\begin{excdesc}{error}
The exception raised by functions in this module when they detect an
error.
The exception argument is a string describing the problem in more
detail.
\end{excdesc}

\begin{datadesc}{empty}
The rectangle returned when some operations return an empty result.
This makes it possible to quickly check whether a result is empty:

\bcode\begin{verbatim}
>>> import rect
>>> r1 = (10, 20), (90, 80)
>>> r2 = (0, 0), (10, 20)
>>> r3 = rect.intersect([r1, r2])
>>> if r3 is rect.empty: print 'Empty intersection'
Empty intersection
>>> 
\end{verbatim}\ecode
\end{datadesc}

\begin{funcdesc}{is_empty}{r}
Returns true if the given rectangle is empty.
A rectangle
\code{(\var{left}, \var{top}), (\var{right}, \var{bottom})}
is empty if
\iftexi
\code{\var{left} >= \var{right}} or \code{\var{top} => \var{bottom}}.
\else
$\var{left} \geq \var{right}$ or $\var{top} \geq \var{bottom}$.
%%JHXXX{\em left~$\geq$~right} or {\em top~$\leq$~bottom}.
\fi
\end{funcdesc}

\begin{funcdesc}{intersect}{list}
Returns the intersection of all rectangles in the list argument.
It may also be called with a tuple argument.
Raises
\code{rect.error}
if the list is empty.
Returns
\code{rect.empty}
if the intersection of the rectangles is empty.
\end{funcdesc}

\begin{funcdesc}{union}{list}
Returns the smallest rectangle that contains all non-empty rectangles in
the list argument.
It may also be called with a tuple argument or with two or more
rectangles as arguments.
Returns
\code{rect.empty}
if the list is empty or all its rectangles are empty.
\end{funcdesc}

\begin{funcdesc}{pointinrect}{point\, rect}
Returns true if the point is inside the rectangle.
By definition, a point
\code{(\var{h}, \var{v})}
is inside a rectangle
\code{(\var{left}, \var{top}), (\var{right}, \var{bottom})} if
\iftexi
\code{\var{left} <= \var{h} < \var{right}} and
\code{\var{top} <= \var{v} < \var{bottom}}.
\else
$\var{left} \leq \var{h} < \var{right}$ and
$\var{top} \leq \var{v} < \var{bottom}$.
\fi
\end{funcdesc}

\begin{funcdesc}{inset}{rect\, \(dh\, dv\)}
Returns a rectangle that lies inside the
\code{rect}
argument by
\var{dh}
pixels horizontally
and
\var{dv}
pixels
vertically.
If
\var{dh}
or
\var{dv}
is negative, the result lies outside
\var{rect}.
\end{funcdesc}

\begin{funcdesc}{rect2geom}{rect}
Converts a rectangle to geometry representation:
\code{(\var{left}, \var{top}), (\var{width}, \var{height})}.
\end{funcdesc}

\begin{funcdesc}{geom2rect}{geom}
Converts a rectangle given in geometry representation back to the
standard rectangle representation
\code{(\var{left}, \var{top}), (\var{right}, \var{bottom})}.
\end{funcdesc}
		% STDWIN ONLY

\chapter{SGI IRIX Specific Services}

The modules described in this chapter provide interfaces to features
that are unique to SGI's IRIX operating system (versions 4 and 5).
			% SGI IRIX ONLY
\section{\module{al} ---
         Audio functions on the SGI}

\declaremodule{builtin}{al}
  \platform{IRIX}
\modulesynopsis{Audio functions on the SGI.}


This module provides access to the audio facilities of the SGI Indy
and Indigo workstations.  See section 3A of the IRIX man pages for
details.  You'll need to read those man pages to understand what these
functions do!  Some of the functions are not available in IRIX
releases before 4.0.5.  Again, see the manual to check whether a
specific function is available on your platform.

All functions and methods defined in this module are equivalent to
the C functions with \samp{AL} prefixed to their name.

Symbolic constants from the C header file \code{<audio.h>} are
defined in the standard module \module{AL}\refstmodindex{AL}, see
below.

\strong{Warning:} the current version of the audio library may dump core
when bad argument values are passed rather than returning an error
status.  Unfortunately, since the precise circumstances under which
this may happen are undocumented and hard to check, the Python
interface can provide no protection against this kind of problems.
(One example is specifying an excessive queue size --- there is no
documented upper limit.)

The module defines the following functions:


\begin{funcdesc}{openport}{name, direction\optional{, config}}
The name and direction arguments are strings.  The optional
\var{config} argument is a configuration object as returned by
\function{newconfig()}.  The return value is an \dfn{audio port
object}; methods of audio port objects are described below.
\end{funcdesc}

\begin{funcdesc}{newconfig}{}
The return value is a new \dfn{audio configuration object}; methods of
audio configuration objects are described below.
\end{funcdesc}

\begin{funcdesc}{queryparams}{device}
The device argument is an integer.  The return value is a list of
integers containing the data returned by \cfunction{ALqueryparams()}.
\end{funcdesc}

\begin{funcdesc}{getparams}{device, list}
The \var{device} argument is an integer.  The list argument is a list
such as returned by \function{queryparams()}; it is modified in place
(!).
\end{funcdesc}

\begin{funcdesc}{setparams}{device, list}
The \var{device} argument is an integer.  The \var{list} argument is a
list such as returned by \function{queryparams()}.
\end{funcdesc}


\subsection{Configuration Objects}
\label{al-config-objects}

Configuration objects (returned by \function{newconfig()} have the
following methods:

\begin{methoddesc}[audio configuration]{getqueuesize}{}
Return the queue size.
\end{methoddesc}

\begin{methoddesc}[audio configuration]{setqueuesize}{size}
Set the queue size.
\end{methoddesc}

\begin{methoddesc}[audio configuration]{getwidth}{}
Get the sample width.
\end{methoddesc}

\begin{methoddesc}[audio configuration]{setwidth}{width}
Set the sample width.
\end{methoddesc}

\begin{methoddesc}[audio configuration]{getchannels}{}
Get the channel count.
\end{methoddesc}

\begin{methoddesc}[audio configuration]{setchannels}{nchannels}
Set the channel count.
\end{methoddesc}

\begin{methoddesc}[audio configuration]{getsampfmt}{}
Get the sample format.
\end{methoddesc}

\begin{methoddesc}[audio configuration]{setsampfmt}{sampfmt}
Set the sample format.
\end{methoddesc}

\begin{methoddesc}[audio configuration]{getfloatmax}{}
Get the maximum value for floating sample formats.
\end{methoddesc}

\begin{methoddesc}[audio configuration]{setfloatmax}{floatmax}
Set the maximum value for floating sample formats.
\end{methoddesc}


\subsection{Port Objects}
\label{al-port-objects}

Port objects, as returned by \function{openport()}, have the following
methods:

\begin{methoddesc}[audio port]{closeport}{}
Close the port.
\end{methoddesc}

\begin{methoddesc}[audio port]{getfd}{}
Return the file descriptor as an int.
\end{methoddesc}

\begin{methoddesc}[audio port]{getfilled}{}
Return the number of filled samples.
\end{methoddesc}

\begin{methoddesc}[audio port]{getfillable}{}
Return the number of fillable samples.
\end{methoddesc}

\begin{methoddesc}[audio port]{readsamps}{nsamples}
Read a number of samples from the queue, blocking if necessary.
Return the data as a string containing the raw data, (e.g., 2 bytes per
sample in big-endian byte order (high byte, low byte) if you have set
the sample width to 2 bytes).
\end{methoddesc}

\begin{methoddesc}[audio port]{writesamps}{samples}
Write samples into the queue, blocking if necessary.  The samples are
encoded as described for the \method{readsamps()} return value.
\end{methoddesc}

\begin{methoddesc}[audio port]{getfillpoint}{}
Return the `fill point'.
\end{methoddesc}

\begin{methoddesc}[audio port]{setfillpoint}{fillpoint}
Set the `fill point'.
\end{methoddesc}

\begin{methoddesc}[audio port]{getconfig}{}
Return a configuration object containing the current configuration of
the port.
\end{methoddesc}

\begin{methoddesc}[audio port]{setconfig}{config}
Set the configuration from the argument, a configuration object.
\end{methoddesc}

\begin{methoddesc}[audio port]{getstatus}{list}
Get status information on last error.
\end{methoddesc}


\section{\module{AL} ---
         Constants used with the \module{al} module}

\declaremodule{standard}{AL}
  \platform{IRIX}
\modulesynopsis{Constants used with the \module{al} module.}


This module defines symbolic constants needed to use the built-in
module \module{al} (see above); they are equivalent to those defined
in the C header file \code{<audio.h>} except that the name prefix
\samp{AL_} is omitted.  Read the module source for a complete list of
the defined names.  Suggested use:

\begin{verbatim}
import al
from AL import *
\end{verbatim}

\section{\module{cd} ---
         CD-ROM access on SGI systems}

\declaremodule{builtin}{cd}
  \platform{IRIX}
\modulesynopsis{Interface to the CD-ROM on Silicon Graphics systems.}


This module provides an interface to the Silicon Graphics CD library.
It is available only on Silicon Graphics systems.

The way the library works is as follows.  A program opens the CD-ROM
device with \function{open()} and creates a parser to parse the data
from the CD with \function{createparser()}.  The object returned by
\function{open()} can be used to read data from the CD, but also to get
status information for the CD-ROM device, and to get information about
the CD, such as the table of contents.  Data from the CD is passed to
the parser, which parses the frames, and calls any callback
functions that have previously been added.

An audio CD is divided into \dfn{tracks} or \dfn{programs} (the terms
are used interchangeably).  Tracks can be subdivided into
\dfn{indices}.  An audio CD contains a \dfn{table of contents} which
gives the starts of the tracks on the CD.  Index 0 is usually the
pause before the start of a track.  The start of the track as given by
the table of contents is normally the start of index 1.

Positions on a CD can be represented in two ways.  Either a frame
number or a tuple of three values, minutes, seconds and frames.  Most
functions use the latter representation.  Positions can be both
relative to the beginning of the CD, and to the beginning of the
track.

Module \module{cd} defines the following functions and constants:


\begin{funcdesc}{createparser}{}
Create and return an opaque parser object.  The methods of the parser
object are described below.
\end{funcdesc}

\begin{funcdesc}{msftoframe}{minutes, seconds, frames}
Converts a \code{(\var{minutes}, \var{seconds}, \var{frames})} triple
representing time in absolute time code into the corresponding CD
frame number.
\end{funcdesc}

\begin{funcdesc}{open}{\optional{device\optional{, mode}}}
Open the CD-ROM device.  The return value is an opaque player object;
methods of the player object are described below.  The device is the
name of the SCSI device file, e.g. \code{'/dev/scsi/sc0d4l0'}, or
\code{None}.  If omitted or \code{None}, the hardware inventory is
consulted to locate a CD-ROM drive.  The \var{mode}, if not omited,
should be the string \code{'r'}.
\end{funcdesc}

The module defines the following variables:

\begin{excdesc}{error}
Exception raised on various errors.
\end{excdesc}

\begin{datadesc}{DATASIZE}
The size of one frame's worth of audio data.  This is the size of the
audio data as passed to the callback of type \code{audio}.
\end{datadesc}

\begin{datadesc}{BLOCKSIZE}
The size of one uninterpreted frame of audio data.
\end{datadesc}

The following variables are states as returned by
\function{getstatus()}:

\begin{datadesc}{READY}
The drive is ready for operation loaded with an audio CD.
\end{datadesc}

\begin{datadesc}{NODISC}
The drive does not have a CD loaded.
\end{datadesc}

\begin{datadesc}{CDROM}
The drive is loaded with a CD-ROM.  Subsequent play or read operations
will return I/O errors.
\end{datadesc}

\begin{datadesc}{ERROR}
An error occurred while trying to read the disc or its table of
contents.
\end{datadesc}

\begin{datadesc}{PLAYING}
The drive is in CD player mode playing an audio CD through its audio
jacks.
\end{datadesc}

\begin{datadesc}{PAUSED}
The drive is in CD layer mode with play paused.
\end{datadesc}

\begin{datadesc}{STILL}
The equivalent of \constant{PAUSED} on older (non 3301) model Toshiba
CD-ROM drives.  Such drives have never been shipped by SGI.
\end{datadesc}

\begin{datadesc}{audio}
\dataline{pnum}
\dataline{index}
\dataline{ptime}
\dataline{atime}
\dataline{catalog}
\dataline{ident}
\dataline{control}
Integer constants describing the various types of parser callbacks
that can be set by the \method{addcallback()} method of CD parser
objects (see below).
\end{datadesc}


\subsection{Player Objects}
\label{player-objects}

Player objects (returned by \function{open()}) have the following
methods:

\begin{methoddesc}[CD player]{allowremoval}{}
Unlocks the eject button on the CD-ROM drive permitting the user to
eject the caddy if desired.
\end{methoddesc}

\begin{methoddesc}[CD player]{bestreadsize}{}
Returns the best value to use for the \var{num_frames} parameter of
the \method{readda()} method.  Best is defined as the value that
permits a continuous flow of data from the CD-ROM drive.
\end{methoddesc}

\begin{methoddesc}[CD player]{close}{}
Frees the resources associated with the player object.  After calling
\method{close()}, the methods of the object should no longer be used.
\end{methoddesc}

\begin{methoddesc}[CD player]{eject}{}
Ejects the caddy from the CD-ROM drive.
\end{methoddesc}

\begin{methoddesc}[CD player]{getstatus}{}
Returns information pertaining to the current state of the CD-ROM
drive.  The returned information is a tuple with the following values:
\var{state}, \var{track}, \var{rtime}, \var{atime}, \var{ttime},
\var{first}, \var{last}, \var{scsi_audio}, \var{cur_block}.
\var{rtime} is the time relative to the start of the current track;
\var{atime} is the time relative to the beginning of the disc;
\var{ttime} is the total time on the disc.  For more information on
the meaning of the values, see the man page \manpage{CDgetstatus}{3dm}.
The value of \var{state} is one of the following: \constant{ERROR},
\constant{NODISC}, \constant{READY}, \constant{PLAYING},
\constant{PAUSED}, \constant{STILL}, or \constant{CDROM}.
\end{methoddesc}

\begin{methoddesc}[CD player]{gettrackinfo}{track}
Returns information about the specified track.  The returned
information is a tuple consisting of two elements, the start time of
the track and the duration of the track.
\end{methoddesc}

\begin{methoddesc}[CD player]{msftoblock}{min, sec, frame}
Converts a minutes, seconds, frames triple representing a time in
absolute time code into the corresponding logical block number for the
given CD-ROM drive.  You should use \function{msftoframe()} rather than
\method{msftoblock()} for comparing times.  The logical block number
differs from the frame number by an offset required by certain CD-ROM
drives.
\end{methoddesc}

\begin{methoddesc}[CD player]{play}{start, play}
Starts playback of an audio CD in the CD-ROM drive at the specified
track.  The audio output appears on the CD-ROM drive's headphone and
audio jacks (if fitted).  Play stops at the end of the disc.
\var{start} is the number of the track at which to start playing the
CD; if \var{play} is 0, the CD will be set to an initial paused
state.  The method \method{togglepause()} can then be used to commence
play.
\end{methoddesc}

\begin{methoddesc}[CD player]{playabs}{minutes, seconds, frames, play}
Like \method{play()}, except that the start is given in minutes,
seconds, and frames instead of a track number.
\end{methoddesc}

\begin{methoddesc}[CD player]{playtrack}{start, play}
Like \method{play()}, except that playing stops at the end of the
track.
\end{methoddesc}

\begin{methoddesc}[CD player]{playtrackabs}{track, minutes, seconds, frames, play}
Like \method{play()}, except that playing begins at the specified
absolute time and ends at the end of the specified track.
\end{methoddesc}

\begin{methoddesc}[CD player]{preventremoval}{}
Locks the eject button on the CD-ROM drive thus preventing the user
from arbitrarily ejecting the caddy.
\end{methoddesc}

\begin{methoddesc}[CD player]{readda}{num_frames}
Reads the specified number of frames from an audio CD mounted in the
CD-ROM drive.  The return value is a string representing the audio
frames.  This string can be passed unaltered to the
\method{parseframe()} method of the parser object.
\end{methoddesc}

\begin{methoddesc}[CD player]{seek}{minutes, seconds, frames}
Sets the pointer that indicates the starting point of the next read of
digital audio data from a CD-ROM.  The pointer is set to an absolute
time code location specified in \var{minutes}, \var{seconds}, and
\var{frames}.  The return value is the logical block number to which
the pointer has been set.
\end{methoddesc}

\begin{methoddesc}[CD player]{seekblock}{block}
Sets the pointer that indicates the starting point of the next read of
digital audio data from a CD-ROM.  The pointer is set to the specified
logical block number.  The return value is the logical block number to
which the pointer has been set.
\end{methoddesc}

\begin{methoddesc}[CD player]{seektrack}{track}
Sets the pointer that indicates the starting point of the next read of
digital audio data from a CD-ROM.  The pointer is set to the specified
track.  The return value is the logical block number to which the
pointer has been set.
\end{methoddesc}

\begin{methoddesc}[CD player]{stop}{}
Stops the current playing operation.
\end{methoddesc}

\begin{methoddesc}[CD player]{togglepause}{}
Pauses the CD if it is playing, and makes it play if it is paused.
\end{methoddesc}


\subsection{Parser Objects}
\label{cd-parser-objects}

Parser objects (returned by \function{createparser()}) have the
following methods:

\begin{methoddesc}[CD parser]{addcallback}{type, func, arg}
Adds a callback for the parser.  The parser has callbacks for eight
different types of data in the digital audio data stream.  Constants
for these types are defined at the \module{cd} module level (see above).
The callback is called as follows: \code{\var{func}(\var{arg}, type,
data)}, where \var{arg} is the user supplied argument, \var{type} is
the particular type of callback, and \var{data} is the data returned
for this \var{type} of callback.  The type of the data depends on the
\var{type} of callback as follows:

\begin{tableii}{l|p{4in}}{code}{Type}{Value}
  \lineii{audio}{String which can be passed unmodified to
\function{al.writesamps()}.}
  \lineii{pnum}{Integer giving the program (track) number.}
  \lineii{index}{Integer giving the index number.}
  \lineii{ptime}{Tuple consisting of the program time in minutes,
seconds, and frames.}
  \lineii{atime}{Tuple consisting of the absolute time in minutes,
seconds, and frames.}
  \lineii{catalog}{String of 13 characters, giving the catalog number
of the CD.}
  \lineii{ident}{String of 12 characters, giving the ISRC
identification number of the recording.  The string consists of two
characters country code, three characters owner code, two characters
giving the year, and five characters giving a serial number.}
  \lineii{control}{Integer giving the control bits from the CD
subcode data}
\end{tableii}
\end{methoddesc}

\begin{methoddesc}[CD parser]{deleteparser}{}
Deletes the parser and frees the memory it was using.  The object
should not be used after this call.  This call is done automatically
when the last reference to the object is removed.
\end{methoddesc}

\begin{methoddesc}[CD parser]{parseframe}{frame}
Parses one or more frames of digital audio data from a CD such as
returned by \method{readda()}.  It determines which subcodes are
present in the data.  If these subcodes have changed since the last
frame, then \method{parseframe()} executes a callback of the
appropriate type passing to it the subcode data found in the frame.
Unlike the \C{} function, more than one frame of digital audio data
can be passed to this method.
\end{methoddesc}

\begin{methoddesc}[CD parser]{removecallback}{type}
Removes the callback for the given \var{type}.
\end{methoddesc}

\begin{methoddesc}[CD parser]{resetparser}{}
Resets the fields of the parser used for tracking subcodes to an
initial state.  \method{resetparser()} should be called after the disc
has been changed.
\end{methoddesc}

\section{Built-in Module \sectcode{fl}}
\bimodindex{fl}

This module provides an interface to the FORMS Library by Mark
Overmars.  The source for the library can be retrieved by anonymous
ftp from host \samp{ftp.cs.ruu.nl}, directory \file{SGI/FORMS}.  It
was last tested with version 2.0b.

Most functions are literal translations of their C equivalents,
dropping the initial \samp{fl_} from their name.  Constants used by
the library are defined in module \code{FL} described below.

The creation of objects is a little different in Python than in C:
instead of the `current form' maintained by the library to which new
FORMS objects are added, all functions that add a FORMS object to a
form are methods of the Python object representing the form.
Consequently, there are no Python equivalents for the C functions
\code{fl_addto_form} and \code{fl_end_form}, and the equivalent of
\code{fl_bgn_form} is called \code{fl.make_form}.

Watch out for the somewhat confusing terminology: FORMS uses the word
\dfn{object} for the buttons, sliders etc. that you can place in a form.
In Python, `object' means any value.  The Python interface to FORMS
introduces two new Python object types: form objects (representing an
entire form) and FORMS objects (representing one button, slider etc.).
Hopefully this isn't too confusing...

There are no `free objects' in the Python interface to FORMS, nor is
there an easy way to add object classes written in Python.  The FORMS
interface to GL event handling is available, though, so you can mix
FORMS with pure GL windows.

\strong{Please note:} importing \code{fl} implies a call to the GL function
\code{foreground()} and to the FORMS routine \code{fl_init()}.

\subsection{Functions Defined in Module \sectcode{fl}}

Module \code{fl} defines the following functions.  For more information
about what they do, see the description of the equivalent C function
in the FORMS documentation:

\renewcommand{\indexsubitem}{(in module fl)}
\begin{funcdesc}{make_form}{type\, width\, height}
Create a form with given type, width and height.  This returns a
\dfn{form} object, whose methods are described below.
\end{funcdesc}

\begin{funcdesc}{do_forms}{}
The standard FORMS main loop.  Returns a Python object representing
the FORMS object needing interaction, or the special value
\code{FL.EVENT}.
\end{funcdesc}

\begin{funcdesc}{check_forms}{}
Check for FORMS events.  Returns what \code{do_forms} above returns,
or \code{None} if there is no event that immediately needs
interaction.
\end{funcdesc}

\begin{funcdesc}{set_event_call_back}{function}
Set the event callback function.
\end{funcdesc}

\begin{funcdesc}{set_graphics_mode}{rgbmode\, doublebuffering}
Set the graphics modes.
\end{funcdesc}

\begin{funcdesc}{get_rgbmode}{}
Return the current rgb mode.  This is the value of the C global
variable \code{fl_rgbmode}.
\end{funcdesc}

\begin{funcdesc}{show_message}{str1\, str2\, str3}
Show a dialog box with a three-line message and an OK button.
\end{funcdesc}

\begin{funcdesc}{show_question}{str1\, str2\, str3}
Show a dialog box with a three-line message and YES and NO buttons.
It returns \code{1} if the user pressed YES, \code{0} if NO.
\end{funcdesc}

\begin{funcdesc}{show_choice}{str1\, str2\, str3\, but1\optional{\, but2\,
but3}}
Show a dialog box with a three-line message and up to three buttons.
It returns the number of the button clicked by the user
(\code{1}, \code{2} or \code{3}).
\end{funcdesc}

\begin{funcdesc}{show_input}{prompt\, default}
Show a dialog box with a one-line prompt message and text field in
which the user can enter a string.  The second argument is the default
input string.  It returns the string value as edited by the user.
\end{funcdesc}

\begin{funcdesc}{show_file_selector}{message\, directory\, pattern\, default}
Show a dialog box in which the user can select a file.  It returns
the absolute filename selected by the user, or \code{None} if the user
presses Cancel.
\end{funcdesc}

\begin{funcdesc}{get_directory}{}
\funcline{get_pattern}{}
\funcline{get_filename}{}
These functions return the directory, pattern and filename (the tail
part only) selected by the user in the last \code{show_file_selector}
call.
\end{funcdesc}

\begin{funcdesc}{qdevice}{dev}
\funcline{unqdevice}{dev}
\funcline{isqueued}{dev}
\funcline{qtest}{}
\funcline{qread}{}
%\funcline{blkqread}{?}
\funcline{qreset}{}
\funcline{qenter}{dev\, val}
\funcline{get_mouse}{}
\funcline{tie}{button\, valuator1\, valuator2}
These functions are the FORMS interfaces to the corresponding GL
functions.  Use these if you want to handle some GL events yourself
when using \code{fl.do_events}.  When a GL event is detected that
FORMS cannot handle, \code{fl.do_forms()} returns the special value
\code{FL.EVENT} and you should call \code{fl.qread()} to read the
event from the queue.  Don't use the equivalent GL functions!
\end{funcdesc}

\begin{funcdesc}{color}{}
\funcline{mapcolor}{}
\funcline{getmcolor}{}
See the description in the FORMS documentation of \code{fl_color},
\code{fl_mapcolor} and \code{fl_getmcolor}.
\end{funcdesc}

\subsection{Form Objects}

Form objects (returned by \code{fl.make_form()} above) have the
following methods.  Each method corresponds to a C function whose name
is prefixed with \samp{fl_}; and whose first argument is a form
pointer; please refer to the official FORMS documentation for
descriptions.

All the \samp{add_{\rm \ldots}} functions return a Python object representing
the FORMS object.  Methods of FORMS objects are described below.  Most
kinds of FORMS object also have some methods specific to that kind;
these methods are listed here.

\begin{flushleft}
\renewcommand{\indexsubitem}{(form object method)}
\begin{funcdesc}{show_form}{placement\, bordertype\, name}
  Show the form.
\end{funcdesc}

\begin{funcdesc}{hide_form}{}
  Hide the form.
\end{funcdesc}

\begin{funcdesc}{redraw_form}{}
  Redraw the form.
\end{funcdesc}

\begin{funcdesc}{set_form_position}{x\, y}
Set the form's position.
\end{funcdesc}

\begin{funcdesc}{freeze_form}{}
Freeze the form.
\end{funcdesc}

\begin{funcdesc}{unfreeze_form}{}
  Unfreeze the form.
\end{funcdesc}

\begin{funcdesc}{activate_form}{}
  Activate the form.
\end{funcdesc}

\begin{funcdesc}{deactivate_form}{}
  Deactivate the form.
\end{funcdesc}

\begin{funcdesc}{bgn_group}{}
  Begin a new group of objects; return a group object.
\end{funcdesc}

\begin{funcdesc}{end_group}{}
  End the current group of objects.
\end{funcdesc}

\begin{funcdesc}{find_first}{}
  Find the first object in the form.
\end{funcdesc}

\begin{funcdesc}{find_last}{}
  Find the last object in the form.
\end{funcdesc}

%---

\begin{funcdesc}{add_box}{type\, x\, y\, w\, h\, name}
Add a box object to the form.
No extra methods.
\end{funcdesc}

\begin{funcdesc}{add_text}{type\, x\, y\, w\, h\, name}
Add a text object to the form.
No extra methods.
\end{funcdesc}

%\begin{funcdesc}{add_bitmap}{type\, x\, y\, w\, h\, name}
%Add a bitmap object to the form.
%\end{funcdesc}

\begin{funcdesc}{add_clock}{type\, x\, y\, w\, h\, name}
Add a clock object to the form. \\
Method:
\code{get_clock}.
\end{funcdesc}

%---

\begin{funcdesc}{add_button}{type\, x\, y\, w\, h\,  name}
Add a button object to the form. \\
Methods:
\code{get_button},
\code{set_button}.
\end{funcdesc}

\begin{funcdesc}{add_lightbutton}{type\, x\, y\, w\, h\, name}
Add a lightbutton object to the form. \\
Methods:
\code{get_button},
\code{set_button}.
\end{funcdesc}

\begin{funcdesc}{add_roundbutton}{type\, x\, y\, w\, h\, name}
Add a roundbutton object to the form. \\
Methods:
\code{get_button},
\code{set_button}.
\end{funcdesc}

%---

\begin{funcdesc}{add_slider}{type\, x\, y\, w\, h\, name}
Add a slider object to the form. \\
Methods:
\code{set_slider_value},
\code{get_slider_value},
\code{set_slider_bounds},
\code{get_slider_bounds},
\code{set_slider_return},
\code{set_slider_size},
\code{set_slider_precision},
\code{set_slider_step}.
\end{funcdesc}

\begin{funcdesc}{add_valslider}{type\, x\, y\, w\, h\, name}
Add a valslider object to the form. \\
Methods:
\code{set_slider_value},
\code{get_slider_value},
\code{set_slider_bounds},
\code{get_slider_bounds},
\code{set_slider_return},
\code{set_slider_size},
\code{set_slider_precision},
\code{set_slider_step}.
\end{funcdesc}

\begin{funcdesc}{add_dial}{type\, x\, y\, w\, h\, name}
Add a dial object to the form. \\
Methods:
\code{set_dial_value},
\code{get_dial_value},
\code{set_dial_bounds},
\code{get_dial_bounds}.
\end{funcdesc}

\begin{funcdesc}{add_positioner}{type\, x\, y\, w\, h\, name}
Add a positioner object to the form. \\
Methods:
\code{set_positioner_xvalue},
\code{set_positioner_yvalue},
\code{set_positioner_xbounds},
\code{set_positioner_ybounds},
\code{get_positioner_xvalue},
\code{get_positioner_yvalue},
\code{get_positioner_xbounds},
\code{get_positioner_ybounds}.
\end{funcdesc}

\begin{funcdesc}{add_counter}{type\, x\, y\, w\, h\, name}
Add a counter object to the form. \\
Methods:
\code{set_counter_value},
\code{get_counter_value},
\code{set_counter_bounds},
\code{set_counter_step},
\code{set_counter_precision},
\code{set_counter_return}.
\end{funcdesc}

%---

\begin{funcdesc}{add_input}{type\, x\, y\, w\, h\, name}
Add a input object to the form. \\
Methods:
\code{set_input},
\code{get_input},
\code{set_input_color},
\code{set_input_return}.
\end{funcdesc}

%---

\begin{funcdesc}{add_menu}{type\, x\, y\, w\, h\, name}
Add a menu object to the form. \\
Methods:
\code{set_menu},
\code{get_menu},
\code{addto_menu}.
\end{funcdesc}

\begin{funcdesc}{add_choice}{type\, x\, y\, w\, h\, name}
Add a choice object to the form. \\
Methods:
\code{set_choice},
\code{get_choice},
\code{clear_choice},
\code{addto_choice},
\code{replace_choice},
\code{delete_choice},
\code{get_choice_text},
\code{set_choice_fontsize},
\code{set_choice_fontstyle}.
\end{funcdesc}

\begin{funcdesc}{add_browser}{type\, x\, y\, w\, h\, name}
Add a browser object to the form. \\
Methods:
\code{set_browser_topline},
\code{clear_browser},
\code{add_browser_line},
\code{addto_browser},
\code{insert_browser_line},
\code{delete_browser_line},
\code{replace_browser_line},
\code{get_browser_line},
\code{load_browser},
\code{get_browser_maxline},
\code{select_browser_line},
\code{deselect_browser_line},
\code{deselect_browser},
\code{isselected_browser_line},
\code{get_browser},
\code{set_browser_fontsize},
\code{set_browser_fontstyle},
\code{set_browser_specialkey}.
\end{funcdesc}

%---

\begin{funcdesc}{add_timer}{type\, x\, y\, w\, h\, name}
Add a timer object to the form. \\
Methods:
\code{set_timer},
\code{get_timer}.
\end{funcdesc}
\end{flushleft}

Form objects have the following data attributes; see the FORMS
documentation:

\begin{tableiii}{|l|c|l|}{code}{Name}{Type}{Meaning}
  \lineiii{window}{int (read-only)}{GL window id}
  \lineiii{w}{float}{form width}
  \lineiii{h}{float}{form height}
  \lineiii{x}{float}{form x origin}
  \lineiii{y}{float}{form y origin}
  \lineiii{deactivated}{int}{nonzero if form is deactivated}
  \lineiii{visible}{int}{nonzero if form is visible}
  \lineiii{frozen}{int}{nonzero if form is frozen}
  \lineiii{doublebuf}{int}{nonzero if double buffering on}
\end{tableiii}

\subsection{FORMS Objects}

Besides methods specific to particular kinds of FORMS objects, all
FORMS objects also have the following methods:

\renewcommand{\indexsubitem}{(FORMS object method)}
\begin{funcdesc}{set_call_back}{function\, argument}
Set the object's callback function and argument.  When the object
needs interaction, the callback function will be called with two
arguments: the object, and the callback argument.  (FORMS objects
without a callback function are returned by \code{fl.do_forms()} or
\code{fl.check_forms()} when they need interaction.)  Call this method
without arguments to remove the callback function.
\end{funcdesc}

\begin{funcdesc}{delete_object}{}
  Delete the object.
\end{funcdesc}

\begin{funcdesc}{show_object}{}
  Show the object.
\end{funcdesc}

\begin{funcdesc}{hide_object}{}
  Hide the object.
\end{funcdesc}

\begin{funcdesc}{redraw_object}{}
  Redraw the object.
\end{funcdesc}

\begin{funcdesc}{freeze_object}{}
  Freeze the object.
\end{funcdesc}

\begin{funcdesc}{unfreeze_object}{}
  Unfreeze the object.
\end{funcdesc}

%\begin{funcdesc}{handle_object}{} XXX
%\end{funcdesc}

%\begin{funcdesc}{handle_object_direct}{} XXX
%\end{funcdesc}

FORMS objects have these data attributes; see the FORMS documentation:

\begin{tableiii}{|l|c|l|}{code}{Name}{Type}{Meaning}
  \lineiii{objclass}{int (read-only)}{object class}
  \lineiii{type}{int (read-only)}{object type}
  \lineiii{boxtype}{int}{box type}
  \lineiii{x}{float}{x origin}
  \lineiii{y}{float}{y origin}
  \lineiii{w}{float}{width}
  \lineiii{h}{float}{height}
  \lineiii{col1}{int}{primary color}
  \lineiii{col2}{int}{secondary color}
  \lineiii{align}{int}{alignment}
  \lineiii{lcol}{int}{label color}
  \lineiii{lsize}{float}{label font size}
  \lineiii{label}{string}{label string}
  \lineiii{lstyle}{int}{label style}
  \lineiii{pushed}{int (read-only)}{(see FORMS docs)}
  \lineiii{focus}{int (read-only)}{(see FORMS docs)}
  \lineiii{belowmouse}{int (read-only)}{(see FORMS docs)}
  \lineiii{frozen}{int (read-only)}{(see FORMS docs)}
  \lineiii{active}{int (read-only)}{(see FORMS docs)}
  \lineiii{input}{int (read-only)}{(see FORMS docs)}
  \lineiii{visible}{int (read-only)}{(see FORMS docs)}
  \lineiii{radio}{int (read-only)}{(see FORMS docs)}
  \lineiii{automatic}{int (read-only)}{(see FORMS docs)}
\end{tableiii}

\section{Standard Module \sectcode{FL}}
\nodename{FL (uppercase)}
\stmodindex{FL}

This module defines symbolic constants needed to use the built-in
module \code{fl} (see above); they are equivalent to those defined in
the C header file \file{<forms.h>} except that the name prefix
\samp{FL_} is omitted.  Read the module source for a complete list of
the defined names.  Suggested use:

\bcode\begin{verbatim}
import fl
from FL import *
\end{verbatim}\ecode

\section{Standard Module \sectcode{flp}}
\stmodindex{flp}

This module defines functions that can read form definitions created
by the `form designer' (\code{fdesign}) program that comes with the
FORMS library (see module \code{fl} above).

For now, see the file \file{flp.doc} in the Python library source
directory for a description.

XXX A complete description should be inserted here!

\section{Built-in Module \sectcode{fm}}
\label{module-fm}
\bimodindex{fm}

This module provides access to the IRIS {\em Font Manager} library.
It is available only on Silicon Graphics machines.
See also: 4Sight User's Guide, Section 1, Chapter 5: Using the IRIS
Font Manager.

This is not yet a full interface to the IRIS Font Manager.
Among the unsupported features are: matrix operations; cache
operations; character operations (use string operations instead); some
details of font info; individual glyph metrics; and printer matching.

It supports the following operations:

\renewcommand{\indexsubitem}{(in module fm)}
\begin{funcdesc}{init}{}
Initialization function.
Calls \code{fminit()}.
It is normally not necessary to call this function, since it is called
automatically the first time the \code{fm} module is imported.
\end{funcdesc}

\begin{funcdesc}{findfont}{fontname}
Return a font handle object.
Calls \code{fmfindfont(\var{fontname})}.
\end{funcdesc}

\begin{funcdesc}{enumerate}{}
Returns a list of available font names.
This is an interface to \code{fmenumerate()}.
\end{funcdesc}

\begin{funcdesc}{prstr}{string}
Render a string using the current font (see the \code{setfont()} font
handle method below).
Calls \code{fmprstr(\var{string})}.
\end{funcdesc}

\begin{funcdesc}{setpath}{string}
Sets the font search path.
Calls \code{fmsetpath(string)}.
(XXX Does not work!?!)
\end{funcdesc}

\begin{funcdesc}{fontpath}{}
Returns the current font search path.
\end{funcdesc}

Font handle objects support the following operations:

\renewcommand{\indexsubitem}{(font handle method)}
\begin{funcdesc}{scalefont}{factor}
Returns a handle for a scaled version of this font.
Calls \code{fmscalefont(\var{fh}, \var{factor})}.
\end{funcdesc}

\begin{funcdesc}{setfont}{}
Makes this font the current font.
Note: the effect is undone silently when the font handle object is
deleted.
Calls \code{fmsetfont(\var{fh})}.
\end{funcdesc}

\begin{funcdesc}{getfontname}{}
Returns this font's name.
Calls \code{fmgetfontname(\var{fh})}.
\end{funcdesc}

\begin{funcdesc}{getcomment}{}
Returns the comment string associated with this font.
Raises an exception if there is none.
Calls \code{fmgetcomment(\var{fh})}.
\end{funcdesc}

\begin{funcdesc}{getfontinfo}{}
Returns a tuple giving some pertinent data about this font.
This is an interface to \code{fmgetfontinfo()}.
The returned tuple contains the following numbers:
{\tt(\var{printermatched}, \var{fixed_width}, \var{xorig}, \var{yorig},
\var{xsize}, \var{ysize}, \var{height}, \var{nglyphs})}.
\end{funcdesc}

\begin{funcdesc}{getstrwidth}{string}
Returns the width, in pixels, of the string when drawn in this font.
Calls \code{fmgetstrwidth(\var{fh}, \var{string})}.
\end{funcdesc}

\section{\module{gl} ---
         \emph{Graphics Library} interface}

\declaremodule{builtin}{gl}
  \platform{IRIX}
\modulesynopsis{Functions from the Silicon Graphics \emph{Graphics Library}.}


This module provides access to the Silicon Graphics
\emph{Graphics Library}.
It is available only on Silicon Graphics machines.

\warning{Some illegal calls to the GL library cause the Python
interpreter to dump core.
In particular, the use of most GL calls is unsafe before the first
window is opened.}

The module is too large to document here in its entirety, but the
following should help you to get started.
The parameter conventions for the C functions are translated to Python as
follows:

\begin{itemize}
\item
All (short, long, unsigned) int values are represented by Python
integers.
\item
All float and double values are represented by Python floating point
numbers.
In most cases, Python integers are also allowed.
\item
All arrays are represented by one-dimensional Python lists.
In most cases, tuples are also allowed.
\item
\begin{sloppypar}
All string and character arguments are represented by Python strings,
for instance,
\code{winopen('Hi There!')}
and
\code{rotate(900, 'z')}.
\end{sloppypar}
\item
All (short, long, unsigned) integer arguments or return values that are
only used to specify the length of an array argument are omitted.
For example, the C call

\begin{verbatim}
lmdef(deftype, index, np, props)
\end{verbatim}

is translated to Python as

\begin{verbatim}
lmdef(deftype, index, props)
\end{verbatim}

\item
Output arguments are omitted from the argument list; they are
transmitted as function return values instead.
If more than one value must be returned, the return value is a tuple.
If the C function has both a regular return value (that is not omitted
because of the previous rule) and an output argument, the return value
comes first in the tuple.
Examples: the C call

\begin{verbatim}
getmcolor(i, &red, &green, &blue)
\end{verbatim}

is translated to Python as

\begin{verbatim}
red, green, blue = getmcolor(i)
\end{verbatim}

\end{itemize}

The following functions are non-standard or have special argument
conventions:

\begin{funcdesc}{varray}{argument}
%JHXXX the argument-argument added
Equivalent to but faster than a number of
\code{v3d()}
calls.
The \var{argument} is a list (or tuple) of points.
Each point must be a tuple of coordinates
\code{(\var{x}, \var{y}, \var{z})} or \code{(\var{x}, \var{y})}.
The points may be 2- or 3-dimensional but must all have the
same dimension.
Float and int values may be mixed however.
The points are always converted to 3D double precision points
by assuming \code{\var{z} = 0.0} if necessary (as indicated in the man page),
and for each point
\code{v3d()}
is called.
\end{funcdesc}

\begin{funcdesc}{nvarray}{}
Equivalent to but faster than a number of
\code{n3f}
and
\code{v3f}
calls.
The argument is an array (list or tuple) of pairs of normals and points.
Each pair is a tuple of a point and a normal for that point.
Each point or normal must be a tuple of coordinates
\code{(\var{x}, \var{y}, \var{z})}.
Three coordinates must be given.
Float and int values may be mixed.
For each pair,
\code{n3f()}
is called for the normal, and then
\code{v3f()}
is called for the point.
\end{funcdesc}

\begin{funcdesc}{vnarray}{}
Similar to 
\code{nvarray()}
but the pairs have the point first and the normal second.
\end{funcdesc}

\begin{funcdesc}{nurbssurface}{s_k, t_k, ctl, s_ord, t_ord, type}
% XXX s_k[], t_k[], ctl[][]
Defines a nurbs surface.
The dimensions of
\code{\var{ctl}[][]}
are computed as follows:
\code{[len(\var{s_k}) - \var{s_ord}]},
\code{[len(\var{t_k}) - \var{t_ord}]}.
\end{funcdesc}

\begin{funcdesc}{nurbscurve}{knots, ctlpoints, order, type}
Defines a nurbs curve.
The length of ctlpoints is
\code{len(\var{knots}) - \var{order}}.
\end{funcdesc}

\begin{funcdesc}{pwlcurve}{points, type}
Defines a piecewise-linear curve.
\var{points}
is a list of points.
\var{type}
must be
\code{N_ST}.
\end{funcdesc}

\begin{funcdesc}{pick}{n}
\funcline{select}{n}
The only argument to these functions specifies the desired size of the
pick or select buffer.
\end{funcdesc}

\begin{funcdesc}{endpick}{}
\funcline{endselect}{}
These functions have no arguments.
They return a list of integers representing the used part of the
pick/select buffer.
No method is provided to detect buffer overrun.
\end{funcdesc}

Here is a tiny but complete example GL program in Python:

\begin{verbatim}
import gl, GL, time

def main():
    gl.foreground()
    gl.prefposition(500, 900, 500, 900)
    w = gl.winopen('CrissCross')
    gl.ortho2(0.0, 400.0, 0.0, 400.0)
    gl.color(GL.WHITE)
    gl.clear()
    gl.color(GL.RED)
    gl.bgnline()
    gl.v2f(0.0, 0.0)
    gl.v2f(400.0, 400.0)
    gl.endline()
    gl.bgnline()
    gl.v2f(400.0, 0.0)
    gl.v2f(0.0, 400.0)
    gl.endline()
    time.sleep(5)

main()
\end{verbatim}


\begin{seealso}
  \seetitle[http://pyopengl.sourceforge.net/]
           {PyOpenGL: The Python OpenGL Binding}
           {An interface to OpenGL\index{OpenGL} is also available;
            see information about the
            \strong{PyOpenGL}\index{PyOpenGL} project online at
            \url{http://pyopengl.sourceforge.net/}.  This may be a
            better option if support for SGI hardware from before
            about 1996 is not required.}
\end{seealso}


\section{\module{DEVICE} ---
         Constants used with the \module{gl} module}

\declaremodule{standard}{DEVICE}
  \platform{IRIX}
\modulesynopsis{Constants used with the \module{gl} module.}

This modules defines the constants used by the Silicon Graphics
\emph{Graphics Library} that C programmers find in the header file
\code{<gl/device.h>}.
Read the module source file for details.


\section{\module{GL} ---
         Constants used with the \module{gl} module}

\declaremodule[gl-constants]{standard}{GL}
  \platform{IRIX}
\modulesynopsis{Constants used with the \module{gl} module.}

This module contains constants used by the Silicon Graphics
\emph{Graphics Library} from the C header file \code{<gl/gl.h>}.
Read the module source file for details.

\section{Built-in Module \module{imgfile}}
\label{module-imgfile}
\bimodindex{imgfile}

The \module{imgfile} module allows Python programs to access SGI imglib image
files (also known as \file{.rgb} files).  The module is far from
complete, but is provided anyway since the functionality that there is
is enough in some cases.  Currently, colormap files are not supported.

The module defines the following variables and functions:

\begin{excdesc}{error}
This exception is raised on all errors, such as unsupported file type, etc.
\end{excdesc}

\begin{funcdesc}{getsizes}{file}
This function returns a tuple \code{(\var{x}, \var{y}, \var{z})} where
\var{x} and \var{y} are the size of the image in pixels and
\var{z} is the number of
bytes per pixel. Only 3 byte RGB pixels and 1 byte greyscale pixels
are currently supported.
\end{funcdesc}

\begin{funcdesc}{read}{file}
This function reads and decodes the image on the specified file, and
returns it as a Python string. The string has either 1 byte greyscale
pixels or 4 byte RGBA pixels. The bottom left pixel is the first in
the string. This format is suitable to pass to \function{gl.lrectwrite()},
for instance.
\end{funcdesc}

\begin{funcdesc}{readscaled}{file, x, y, filter\optional{, blur}}
This function is identical to read but it returns an image that is
scaled to the given \var{x} and \var{y} sizes. If the \var{filter} and
\var{blur} parameters are omitted scaling is done by
simply dropping or duplicating pixels, so the result will be less than
perfect, especially for computer-generated images.

Alternatively, you can specify a filter to use to smoothen the image
after scaling. The filter forms supported are \code{'impulse'},
\code{'box'}, \code{'triangle'}, \code{'quadratic'} and
\code{'gaussian'}. If a filter is specified \var{blur} is an optional
parameter specifying the blurriness of the filter. It defaults to \code{1.0}.

\function{readscaled()} makes no attempt to keep the aspect ratio
correct, so that is the users' responsibility.
\end{funcdesc}

\begin{funcdesc}{ttob}{flag}
This function sets a global flag which defines whether the scan lines
of the image are read or written from bottom to top (flag is zero,
compatible with SGI GL) or from top to bottom(flag is one,
compatible with X).  The default is zero.
\end{funcdesc}

\begin{funcdesc}{write}{file, data, x, y, z}
This function writes the RGB or greyscale data in \var{data} to image
file \var{file}. \var{x} and \var{y} give the size of the image,
\var{z} is 1 for 1 byte greyscale images or 3 for RGB images (which are
stored as 4 byte values of which only the lower three bytes are used).
These are the formats returned by \function{gl.lrectread()}.
\end{funcdesc}

%\section{Standard Module \module{panel}}
\declaremodule{standard}{panel}

\modulesynopsis{None}


\strong{Please note:} The FORMS library, to which the
\code{fl}\refbimodindex{fl} module described above interfaces, is a
simpler and more accessible user interface library for use with GL
than the \code{panel} module (besides also being by a Dutch author).

This module should be used instead of the built-in module
\code{pnl}\refbimodindex{pnl}
to interface with the
\emph{Panel Library}.

The module is too large to document here in its entirety.
One interesting function:

\begin{funcdesc}{defpanellist}{filename}
Parses a panel description file containing S-expressions written by the
\emph{Panel Editor}
that accompanies the Panel Library and creates the described panels.
It returns a list of panel objects.
\end{funcdesc}

\strong{Warning:}
the Python interpreter will dump core if you don't create a GL window
before calling
\code{panel.mkpanel()}
or
\code{panel.defpanellist()}.

\section{Standard Module \module{panelparser}}
\declaremodule{standard}{panelparser}

\modulesynopsis{None}


This module defines a self-contained parser for S-expressions as output
by the Panel Editor (which is written in Scheme so it can't help writing
S-expressions).
The relevant function is
\code{panelparser.parse_file(\var{file})}
which has a file object (not a filename!) as argument and returns a list
of parsed S-expressions.
Each S-expression is converted into a Python list, with atoms converted
to Python strings and sub-expressions (recursively) to Python lists.
For more details, read the module file.
% XXXXJH should be funcdesc, I think

\section{Built-in Module \module{pnl}}
\declaremodule{builtin}{pnl}

\modulesynopsis{None}


This module provides access to the
\emph{Panel Library}
built by NASA Ames\index{NASA} (to get it, send e-mail to
\code{panel-request@nas.nasa.gov}).
All access to it should be done through the standard module
\code{panel}\refstmodindex{panel},
which transparantly exports most functions from
\code{pnl}
but redefines
\code{pnl.dopanel()}.

\strong{Warning:}
the Python interpreter will dump core if you don't create a GL window
before calling
\code{pnl.mkpanel()}.

The module is too large to document here in its entirety.


\chapter{SunOS Specific Services}
\label{sunos}

The modules described in this chapter provide interfaces to features
that are unique to SunOS 5 (also known as Solaris version 2).
			% SUNOS ONLY
\section{\module{sunaudiodev} ---
         Access to Sun audio hardware.}
\declaremodule{builtin}{sunaudiodev}

\modulesynopsis{Access to Sun audio hardware.}


This module allows you to access the Sun audio interface. The Sun
audio hardware is capable of recording and playing back audio data
in u-LAW\index{u-LAW} format with a sample rate of 8K per second. A
full description can be found in the \manpage{audio}{7I} manual page.

The module defines the following variables and functions:

\begin{excdesc}{error}
This exception is raised on all errors. The argument is a string
describing what went wrong.
\end{excdesc}

\begin{funcdesc}{open}{mode}
This function opens the audio device and returns a Sun audio device
object. This object can then be used to do I/O on. The \var{mode} parameter
is one of \code{'r'} for record-only access, \code{'w'} for play-only
access, \code{'rw'} for both and \code{'control'} for access to the
control device. Since only one process is allowed to have the recorder
or player open at the same time it is a good idea to open the device
only for the activity needed. See \manpage{audio}{7I} for details.

As per the manpage, this module first looks in the environment
variable \code{AUDIODEV} for the base audio device filename.  If not
found, it falls back to \file{/dev/audio}.  The control device is
calculated by appending ``ctl'' to the base audio device.
\end{funcdesc}


\subsection{Audio Device Objects}
\label{audio-device-objects}

The audio device objects are returned by \function{open()} define the
following methods (except \code{control} objects which only provide
\method{getinfo()}, \method{setinfo()}, \method{fileno()}, and
\method{drain()}):

\begin{methoddesc}[audio device]{close}{}
This method explicitly closes the device. It is useful in situations
where deleting the object does not immediately close it since there
are other references to it. A closed device should not be used again.
\end{methoddesc}

\begin{methoddesc}[audio device]{fileno}{}
Returns the file descriptor associated with the device.  This can be
used to set up \code{SIGPOLL} notification, as described below.
\end{methoddocs}

\begin{methoddesc}[audio device]{drain}{}
This method waits until all pending output is processed and then returns.
Calling this method is often not necessary: destroying the object will
automatically close the audio device and this will do an implicit drain.
\end{methoddesc}

\begin{methoddesc}[audio device]{flush}{}
This method discards all pending output. It can be used avoid the
slow response to a user's stop request (due to buffering of up to one
second of sound).
\end{methoddesc}

\begin{methoddesc}[audio device]{getinfo}{}
This method retrieves status information like input and output volume,
etc. and returns it in the form of
an audio status object. This object has no methods but it contains a
number of attributes describing the current device status. The names
and meanings of the attributes are described in
\file{/usr/include/sun/audioio.h} and in the \manpage{audio}{7I}
manual page.  Member names
are slightly different from their \C{} counterparts: a status object is
only a single structure. Members of the \cdata{play} substructure have
\samp{o_} prepended to their name and members of the \cdata{record}
structure have \samp{i_}. So, the \C{} member \cdata{play.sample_rate} is
accessed as \member{o_sample_rate}, \cdata{record.gain} as \member{i_gain}
and \cdata{monitor_gain} plainly as \member{monitor_gain}.
\end{methoddesc}

\begin{methoddesc}[audio device]{ibufcount}{}
This method returns the number of samples that are buffered on the
recording side, i.e.\ the program will not block on a
\function{read()} call of so many samples.
\end{methoddesc}

\begin{methoddesc}[audio device]{obufcount}{}
This method returns the number of samples buffered on the playback
side. Unfortunately, this number cannot be used to determine a number
of samples that can be written without blocking since the kernel
output queue length seems to be variable.
\end{methoddesc}

\begin{methoddesc}[audio device]{read}{size}
This method reads \var{size} samples from the audio input and returns
them as a Python string. The function blocks until enough data is available.
\end{methoddesc}

\begin{methoddesc}[audio device]{setinfo}{status}
This method sets the audio device status parameters. The \var{status}
parameter is an device status object as returned by \function{getinfo()} and
possibly modified by the program.
\end{methoddesc}

\begin{methoddesc}[audio device]{write}{samples}
Write is passed a Python string containing audio samples to be played.
If there is enough buffer space free it will immediately return,
otherwise it will block.
\end{methoddesc}

There is a companion module,
\module{SUNAUDIODEV}\refstmodindex{SUNAUDIODEV}, which defines useful
symbolic constants like \constant{MIN_GAIN}, \constant{MAX_GAIN},
\constant{SPEAKER}, etc. The names of the constants are the same names
as used in the \C{} include file \code{<sun/audioio.h>}, with the
leading string \samp{AUDIO_} stripped.

The audio device supports asynchronous notification of various events,
through the SIGPOLL signal.  Here's an example of how you might enable 
this in Python:

\begin{verbatim}
def handle_sigpoll(signum, frame):
    print 'I got a SIGPOLL update'
pp
import fcntl, signal, STROPTS

signal.signal(signal.SIGPOLL, handle_sigpoll)
fcntl.ioctl(audio_obj.fileno(), STROPTS.I_SETSIG, STROPTS.S_MSG)
\end{verbatim}


\chapter{Undocumented Modules}

Here's a quick listing of modules that are currently undocumented, but
that should be documented.  Feel free to contribute documentation for
them!  (The idea and most contents for this chapter were taken from a
posting by Fredrik Lundh; I have revised some modules' status.)


\section{Fundamental, and pretty straightforward to document}

cPickle.c -- mostly the same as pickle but no subclassing

cStringIO.c -- mostly the same as StringIO but no subclassing


\section{Frameworks; somewhat harder to document, but well worth the effort}

Tkinter.py -- Interface to Tcl/Tk for graphical user interfaces;
Fredrik Lundh is working on this one!

CGIHTTPServer.py -- CGI-savvy HTTP Server

SimpleHTTPServer.py -- Simple HTTP Server


\section{Stuff useful to a lot of people, including the CGI crowd}

MimeWriter.py -- Generic MIME writer

multifile.py -- make each part of a multipart message ``feel'' like

fileinput.py -- convenient loop over the lines in a list of input files.


\section{Miscellaneous useful utilities}

Some of these are very old and/or not very robust; marked with ``hmm''.

calendar.py -- Calendar printing functions

cmp.py -- Efficiently compare files

cmpcache.py -- Efficiently compare files (uses statcache)

dircache.py -- like os.listdir, but caches results

dircmp.py -- class to build directory diff tools on

linecache.py -- Cache lines from files (used by pdb)

pipes.py -- Conversion pipeline templates (hmm)

popen2.py -- improved popen, can read AND write simultaneously

statcache.py -- Maintain a cache of file stats

colorsys.py -- Conversion between RGB and other color systems

dbhash.py -- (g)dbm-like wrapper for bsdhash.hashopen

mhlib.py -- MH interface

pty.py -- Pseudo terminal utilities

tty.py -- Terminal utilities

cmd.py -- build line-oriented command interpreters (used by pdb)

bdb.py -- A generic Python debugger base class (used by pdb)

ihooks.py -- Import hook support (for ni and rexec)


\section{Parsing Python}

(One could argue that these should all be documented together with the
parser module.)

tokenize.py -- regular expression that recognizes Python tokens; also
contains helper code for colorizing Python source code.

pyclbr.py -- Parse a Python file and retrieve classes and methods


\section{Platform specific modules}

ntpath.py -- equivalent of posixpath on 32-bit Windows

dospath.py -- equivalent of posixpath on MS-DOS


\section{Code objects and files, debugger etc.}

compileall.py -- force "compilation" of all .py files in a directory

py_compile.py -- "compile" a .py file to a .pyc file

repr.py -- Redo the `...` (representation) but with limits on most
sizes (used by pdb)

copy_reg.py -- helper to provide extensibility for pickle/cPickle


\section{Multimedia}

audiodev.py -- Plays audio files

sunau.py -- parse Sun and NeXT audio files

sunaudio.py -- interpret sun audio headers

toaiff.py -- Convert "arbitrary" sound files to AIFF files

sndhdr.py -- recognizing sound files

wave.py -- parse WAVE files

whatsound.py -- recognizing sound files


\section{Oddities}

These modules are probably also obsolete, or just not very useful.

bisect.py -- Bisection algorithms (this is actually useful at times)

dump.py -- Print python code that reconstructs a variable

find.py -- find files matching pattern in directory tree

fpformat.py -- General floating point formatting functions -- obsolete

grep.py -- grep

mutex.py -- Mutual exclusion -- for use with module sched

packmail.py -- create a self-unpacking \UNIX{} shell archive

poly.py -- Polynomials

sched.py -- event scheduler class

shutil.py -- utility functions usable in a shell-like program

util.py -- useful functions that don't fit elsewhere

zmod.py -- Compute properties of mathematical "fields"

tzparse.py -- Parse a timezone specification (unfinished)


\section{Obsolete}

newdir.py -- New dir() function (the standard dir() is now just as good)

addpack.py -- standard support for "packages" (use ni instead)

fmt.py -- text formatting abstractions (too slow)

Para.py -- helper for fmt.py

lockfile.py -- wrapper around FCNTL file locking (use
fcntl.lockf/flock intead)

tb.py -- Print tracebacks, with a dump of local variables (use
pdb.pm() or traceback.py instead)

codehack.py -- extract function name or line number from a function
code object (these are now accessible as attributes: co.co_name,
func.func_name, co.co_firstlineno)


\section{Extension modules}

bsddbmodule.c -- Interface to the Berkeley DB interface (yet another
dbm clone).

cursesmodule.c -- Curses interface.

dbhashmodule.c -- Obsolete; this functionality is now provided by
bsddbmodule.c.

dlmodule.c --  A highly experimental and dangerous device for calling
arbitrary C functions in arbitrary shared libraries.

newmodule.c -- Tommy Burnette's `new' module (creates new empty
objects of certain kinds) -- dangerous.

nismodule.c -- NIS (a.k.a. Sun's Yellow Pages) interface.

timingmodule.c -- Measure time intervals to high resolution (obsolete
-- use time.clock() instead).

resource.c -- Interface to getrusage() and friends.

stdwinmodule.c -- Interface to STDWIN (an old, unsupported
platform-independent GUI package).  Obsolete; use Tkinter for a
platform-independent GUI instead.

The following are SGI specific:

clmodule.c -- Interface to the SGI compression library.

svmodule.c -- Interface to the ``simple video'' board on SGI Indigo
(obsolete hardware).


\renewcommand{\indexname}{Module Index}
\renewcommand{\indexlabel}{modindex}
\inputindex{modlib.ind}		% Module Index

\renewcommand{\indexname}{Index}
\renewcommand{\indexlabel}{genindex}
\inputindex{lib.ind}			% Index

\end{document}

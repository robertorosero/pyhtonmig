\section{\module{ossaudiodev} ---
         Access to Open Sound System-compatible audio hardware}

\declaremodule{builtin}{ossaudiodev}
  \platform{OSS}
\modulesynopsis{Access to OSS-compatible audio hardware.}

% I know FreeBSD uses OSS -- what about Net- and Open-?
This module allows you to access the Open Sound System audio interface.
The Open Sound System interface is present on Linux and FreeBSD.

This module provides a very "bare bones" wrapper over the IOCTLs used to
access the audio hardware.  The best---albeit rather daunting---way to
get a feel for the interface is from the Open Sound System official
documentation:

\url{http://www.opensound.com/pguide/oss.pdf}

The module defines a number of constants which may be used to program
the device.  These constants are the same as those defined in the C
include file \code{<sys/soundcard.h>}.

\code{ossaudiodev} defines the following variables and functions:

\begin{excdesc}{error}
This exception is raised on errors.  The argument is a string describing
what went wrong.
\end{excdesc}

\begin{funcdesc}{open}{\optional{device, }mode}
This function opens the audio device and returns an OSS audio device
object.  This object can then be used to do I/O on.  The \var{device}
parameter is the audio device filename to use.  If it is not specified,
this module first looks in the environment variable \code{AUDIODEV} for
a device to use.  If not found, it falls back to \file{/dev/dsp}.

The \var{mode} parameter is one of \code{'r'} for record-only access,
\code{'w'} for play-only access and \code{'rw'} for both.  Since many
soundcards only allow one process to have the recorder or player open at
a time it is a good idea to open the device only for the activity
needed.  Further, some soundcards are half-duplex: they can be opened
for reading or writing, but not both at once.
\end{funcdesc}

\begin{funcdesc}{openmixer}{\optional{device\optional{, mode}}} This function
opens the mixer device and returns an OSS mixer device object.  The
\var{device} parameter is the mixer device filename to use.  If it is
not specified, this module first looks in the environment variable
\code{MIXERDEV} for a device to use.  If not found, it falls back to
\file{/dev/mixer}.  You may specify \code{'r'}, \code{'rw'} or
\code{'w'} for \var{mode}; the default is \code{'r'}.

\end{funcdesc}

\subsection{Audio Device Objects \label{ossaudio-device-objects}}

Setting up the device

To set up the device, three functions must be called in the correct
sequence:
\begin{enumerate}
\item \code{setfmt()} to set the output format,
\item \code{channels()} to set the number of channels, and
\item \code{speed()} to set the sample rate.
\end{enumerate}

The audio device objects are returned by \function{open()} define the
following methods:

\begin{methoddesc}[audio device]{close}{}
This method explicitly closes the device.  It is useful in situations
where deleting the object does not immediately close it since there are
other references to it.  A closed device should not be used again.
\end{methoddesc}

\begin{methoddesc}[audio device]{fileno}{}
Returns the file descriptor associated with the device.
\end{methoddesc}

\begin{methoddesc}[audio device]{read}{size}
Reads \var{size} samples from the audio input and returns them as a
Python string.  The function blocks until enough data is available.
\end{methoddesc}

\begin{methoddesc}[audio device]{write}{data}
Writes Python string \var{data} to the audio device and returns the
number of bytes written.  If the audio device is opened in blocking
mode, the entire string is always written.  If the device is opened in
nonblocking mode, some data may not be written---see \code{writeall}.
\end{methoddesc}

\begin{methoddesc}[audio device]{writeall}{data}
Writes the entire Python string \var{data} to the audio device.  If the
device is opened in blocking mode, behaves identially to \code{write};
in nonblocking mode, waits until the device becomes available before
feeding it more data.  Returns None, since the amount of data written is
always equal to the amount of data supplied.
\end{methoddesc}

Simple IOCTLs:

\begin{methoddesc}[audio device]{nonblock}{}
Attempts to put the device into nonblocking mode.  Once in nonblocking
mode there is no way to return to blocking mode.

Raises \exception{IOError} if the IOCTL failed.
\end{methoddesc}

\begin{methoddesc}[audio device]{getfmts}{}
Returns a bitmask of the audio output formats supported by the
soundcard.  On a typical Linux system, these formats are:

\begin{tableii}{l|l}{constant}{Format}{Description}
\lineii{AFMT_MU_LAW}
       {a logarithmic encoding.  This is the default format on
        /dev/audio and is the format used by Sun .au files.}
\lineii{AFMT_A_LAW}
       {a logarithmic encoding}
\lineii{AFMT_IMA_ADPCM}
       {a 4:1 compressed format defined by the Interactive Multimedia
        Association.} 
\lineii{AFMT_U8}
       {Unsigned, 8-bit audio.}
\lineii{AFMT_S16_LE}
       {Unsigned, 16-bit audio, little-endian byte order (as used by
        Intel processors)}
\lineii{AFMT_S16_BE}
       {Unsigned, 16-bit audio, big-endian byte order (as used by 68k,
        PowerPC, Sparc)}
\lineii{AFMT_S8}
       {Signed, 8 bit audio.}
\lineii{AFMT_U16_LE}
       {Signed, 16-bit little-endian audio}
\lineii{AFMT_U16_BE}
       {Signed, 16-bit big-endian audio}
\end{tableii}
Most systems support only a subset of these formats.  Many devices only
support \code{AFMT_U8}; the most common format used today is
\code{AFMT_S16_LE}.
\end{methoddesc}

\begin{methoddesc}[audio device]{setfmt}{format}
Used to set the current audio format to \var{format}---see
\code{getfmts} for a list.  May also be used to return the current audio
format---do this by passing an ``audio format'' of \code{AFMT_QUERY}.
Returns the audio format that the device was set to, which may not be
the requested format.
\end{methoddesc}

\begin{methoddesc}[audio device]{channels}{num_channels}
Sets the number of output channels to \var{num_channels}.  A value of 1
indicates monophonic sound, 2 stereophonic.  Some devices may have more
than 2 channels, and some high-end devices may not support mono.
Returns the number of channels the device was set to.
\end{methoddesc}

\begin{methoddesc}[audio device]{speed}{samplerate}
Sets the samplerate to \var{samplerate} samples per second and returns
the rate actually set.  Most sound devices don't support arbitrary
sample rates.  Common rates are:

8000---default rate
11025---speech recording
22050
44100---Audio CD-quality (at 16 bits/sample and 2 channels)
96000---DVD-quality
\end{methoddesc}

\begin{methoddesc}[audio device]{sync}
Waits until the sound device has played every byte in its buffer and
returns.  This also occurs when the sound device is closed.  The OSS
documentation recommends simply closing and re-opening the device rather
than using \code{sync}.
\end{methoddesc}

\begin{methoddesc}[audio device]{reset}
Immediately stops and playing or recording and returns the device to a
state where it can accept commands.  The OSS documentation recommends
closing and re-opening the device after calling \code{reset}.
\end{methoddesc}

\begin{methoddesc}[audio device]{post}
To be used like a lightweight \code{sync}, the \code{post} IOCTL informs
the audio device that there is a likely to be a pause in the audio
output---i.e., after playing a spot sound effect, before waiting for
user input, or before doing disk IO.
\end{methoddesc}

Convenience methods

\begin{methoddesc}[audio device]{setparameters}{samplerate,num_channels,format,emulate}
Initialise the sound device in one method.  \var{samplerate},
\var{channels} and \var{format} should be as specified in the
\code{speed}, \code{channels} and \code{setfmt} methods.  If
\var{emulate} is true, attempt to find the closest matching format
instead, otherwise raise ValueError if the device does not support the
format.  The default is to raise ValueError on unsupported formats.
\end{methoddesc}

\begin{methoddesc}[audio device]{bufsize}{}
Returns the size of the hardware buffer, in samples.
\end{methoddesc}

\begin{methoddesc}[audio device]{obufcount}{}
Returns the number of samples that are in the hardware buffer yet to be
played.
\end{methoddesc}

\begin{methoddesc}[audio device]{obuffree}{}
Returns the number of samples that could be queued into the hardware
buffer to be played without blocking.
\end{methoddesc}

\subsection{Mixer Device Objects \label{mixer-device-objects}}

File-like interface

\begin{methoddesc}[mixer device]{close}{}
This method closes the open mixer device file.  Any further attempts to
use the mixer after this file is closed will raise an IOError.
\end{methoddesc}

\begin{methoddesc}[mixer device]{fileno}{}
Returns the file handle number of the open mixer device file.
\end{methoddesc}

Mixer interface

\begin{methoddesc}[mixer device]{controls}{}
This method returns a bitmask specifying the available mixer controls
(``Control'' being a specific mixable ``channel'', such as
\code{SOUND_MIXER_PCM} or \code{SOUND_MIXER_SYNTH}).  This
bitmask indicates a subset of all available mixer channels---the
\code{SOUND_MIXER_*} constants defined at module level.  To determine if,
for example, the current mixer object supports a PCM mixer, use the
following Python code:

\begin{verbatim}
mixer=ossaudiodev.openmixer()
if mixer.channels() & (1 << ossaudiodev.SOUND_MIXER_PCM):
	# PCM is supported
	<code>
\end{verbatim}

For most purposes, the \code{SOUND_MIXER_VOLUME} (Master volume) and
\code{SOUND_MIXER_PCM} channels should suffice---but code that uses the
mixer should be flexible when it comes to choosing sound channels.  On
the Gravis Ultrasound, for example, \code{SOUND_MIXER_VOLUME} does not
exist.
\end{methoddesc}

\begin{methoddesc}[mixer device]{stereocontrols}{}
Returns a bitmask indicating stereo mixer channels.  If a bit is set,
the corresponding channel is stereo; if it is unset, the channel is
either monophonic or not supported by the mixer (use in combination with
\function{channels} to determine which).

See the code example for the \function{channels} function for an example
of getting data from a bitmask.
\end{methoddesc}

\begin{methoddesc}[mixer device]{reccontrols}{}
Returns a bitmask specifying the mixer controls that may be used to
record.  See the code example for \function{controls} for an example of
reading from a bitmask.
\end{methoddesc}

\begin{methoddesc}[mixer device]{get}{channel}
Returns the volume of a given mixer channel.  The returned volume is a
2-tuple of \code{left volume, right volume}.  Volumes are specified as
numbers from 0 (silent) to 100 (full volume).  If the channel is
monophonic, a 2-tuple is still returned, but both channel volumes are
the same.

If an unknown channel is specified, \exception{error} is raised.
\end{methoddesc}

\begin{methoddesc}[mixer device]{set}{channel, (left, right)}
Sets the volume for a given mixer channel to \code{(left, right)}.
\code{left} and \code{right} must be ints and between 0 (silent) and 100
(full volume).  On success, the new volume is returned as a 2-tuple.
Note that this may not be exactly the same as the volume specified,
because of the limited resolution of some soundcard's mixers.

Raises \exception{IOError} if an invalid mixer channel was specified;
\exception{TypeError} if the argument format was incorrect, and
\exception{error} if the specified volumes were out-of-range.
\end{methoddesc}

\begin{methoddesc}[mixer device]{get_recsrc}{}
This method returns a bitmask indicating which channel or channels are
currently being used as a recording source.
\end{methoddesc}

\begin{methoddesc}[mixer device]{set_recsrc}{bitmask}
Call this function to specify a recording source.  Returns a bitmask
indicating the new recording source (or sources) if successful; raises
\exception{IOError} if an invalid source was specified.  To set the current
recording source to the microphone input:

\begin{verbatim}
mixer.setrecsrc (1 << ossaudiodev.SOUND_MIXER_MIC)
\end{verbatim}
\end{methoddesc}




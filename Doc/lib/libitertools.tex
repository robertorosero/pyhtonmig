\section{\module{itertools} ---
         Functions creating iterators for efficient looping}

\declaremodule{standard}{itertools}
\modulesynopsis{Functions creating iterators for efficient looping.}
\moduleauthor{Raymond Hettinger}{python@rcn.com}
\sectionauthor{Raymond Hettinger}{python@rcn.com}
\versionadded{2.3}


This module implements a number of iterator building blocks inspired
by constructs from the Haskell and SML programming languages.  Each
has been recast in a form suitable for Python.

The module standardizes a core set of fast, memory efficient tools
that are useful by themselves or in combination.  Standardization helps
avoid the readability and reliability problems which arise when many
different individuals create their own slightly varying implementations,
each with their own quirks and naming conventions.

The tools are designed to combine readily with one another.  This makes
it easy to construct more specialized tools succinctly and efficiently
in pure Python.

For instance, SML provides a tabulation tool: \code{tabulate(f)}
which produces a sequence \code{f(0), f(1), ...}.  This toolbox
provides \function{imap()} and \function{count()} which can be combined
to form \code{imap(f, count())} and produce an equivalent result.

Likewise, the functional tools are designed to work well with the
high-speed functions provided by the \refmodule{operator} module.

The module author welcomes suggestions for other basic building blocks
to be added to future versions of the module.

Whether cast in pure python form or C code, tools that use iterators
are more memory efficient (and faster) than their list based counterparts.
Adopting the principles of just-in-time manufacturing, they create
data when and where needed instead of consuming memory with the
computer equivalent of ``inventory''.

The performance advantage of iterators becomes more acute as the number
of elements increases -- at some point, lists grow large enough to
to severely impact memory cache performance and start running slowly.

\begin{seealso}
  \seetext{The Standard ML Basis Library,
           \citetitle[http://www.standardml.org/Basis/]
           {The Standard ML Basis Library}.}

  \seetext{Haskell, A Purely Functional Language,
           \citetitle[http://www.haskell.org/definition/]
           {Definition of Haskell and the Standard Libraries}.}
\end{seealso}


\subsection{Itertool functions \label{itertools-functions}}

The following module functions all construct and return iterators.
Some provide streams of infinite length, so they should only be accessed
by functions or loops that truncate the stream.

\begin{funcdesc}{chain}{*iterables}
  Make an iterator that returns elements from the first iterable until
  it is exhausted, then proceeds to the next iterable, until all of the
  iterables are exhausted.  Used for treating consecutive sequences as
  a single sequence.  Equivalent to:

  \begin{verbatim}
     def chain(*iterables):
         for it in iterables:
             for element in it:
                 yield element
  \end{verbatim}
\end{funcdesc}

\begin{funcdesc}{count}{\optional{n}}
  Make an iterator that returns consecutive integers starting with \var{n}.
  Does not currently support python long integers.  Often used as an
  argument to \function{imap()} to generate consecutive data points.
  Also, used with \function{izip()} to add sequence numbers.  Equivalent to:

  \begin{verbatim}
     def count(n=0):
         while True:
             yield n
             n += 1
  \end{verbatim}

  Note, \function{count()} does not check for overflow and will return
  negative numbers after exceeding \code{sys.maxint}.  This behavior
  may change in the future.
\end{funcdesc}

\begin{funcdesc}{cycle}{iterable}
  Make an iterator returning elements from the iterable and saving a
  copy of each.  When the iterable is exhausted, return elements from
  the saved copy.  Repeats indefinitely.  Equivalent to:

  \begin{verbatim}
     def cycle(iterable):
         saved = []
         for element in iterable:
             yield element
             saved.append(element)
         while saved:
             for element in saved:
                   yield element
  \end{verbatim}

  Note, this is the only member of the toolkit that may require
  significant auxiliary storage (depending on the length of the
  iterable).
\end{funcdesc}

\begin{funcdesc}{dropwhile}{predicate, iterable}
  Make an iterator that drops elements from the iterable as long as
  the predicate is true; afterwards, returns every element.  Note,
  the iterator does not produce \emph{any} output until the predicate
  is true, so it may have a lengthy start-up time.  Equivalent to:

  \begin{verbatim}
     def dropwhile(predicate, iterable):
         iterable = iter(iterable)
         for x in iterable:
             if not predicate(x):
                 yield x
                 break
         for x in iterable:
             yield x
  \end{verbatim}
\end{funcdesc}

\begin{funcdesc}{ifilter}{predicate, iterable}
  Make an iterator that filters elements from iterable returning only
  those for which the predicate is \code{True}.
  If \var{predicate} is \code{None}, return the items that are true.
  Equivalent to:

  \begin{verbatim}
     def ifilter(predicate, iterable):
         if predicate is None:
             def predicate(x):
                 return x
         for x in iterable:
             if predicate(x):
                 yield x
  \end{verbatim}
\end{funcdesc}

\begin{funcdesc}{ifilterfalse}{predicate, iterable}
  Make an iterator that filters elements from iterable returning only
  those for which the predicate is \code{False}.
  If \var{predicate} is \code{None}, return the items that are false.
  Equivalent to:

  \begin{verbatim}
     def ifilterfalse(predicate, iterable):
         if predicate is None:
             def predicate(x):
                 return x
         for x in iterable:
             if not predicate(x):
                 yield x
  \end{verbatim}
\end{funcdesc}

\begin{funcdesc}{imap}{function, *iterables}
  Make an iterator that computes the function using arguments from
  each of the iterables.  If \var{function} is set to \code{None}, then
  \function{imap()} returns the arguments as a tuple.  Like
  \function{map()} but stops when the shortest iterable is exhausted
  instead of filling in \code{None} for shorter iterables.  The reason
  for the difference is that infinite iterator arguments are typically
  an error for \function{map()} (because the output is fully evaluated)
  but represent a common and useful way of supplying arguments to
  \function{imap()}.
  Equivalent to:

  \begin{verbatim}
     def imap(function, *iterables):
         iterables = map(iter, iterables)
         while True:
             args = [i.next() for i in iterables]
             if function is None:
                 yield tuple(args)
             else:
                 yield function(*args)
  \end{verbatim}
\end{funcdesc}

\begin{funcdesc}{islice}{iterable, \optional{start,} stop \optional{, step}}
  Make an iterator that returns selected elements from the iterable.
  If \var{start} is non-zero, then elements from the iterable are skipped
  until start is reached.  Afterward, elements are returned consecutively
  unless \var{step} is set higher than one which results in items being
  skipped.  If \var{stop} is \code{None}, then iteration continues until
  the iterator is exhausted, if at all; otherwise, it stops at the specified
  position.  Unlike regular slicing,
  \function{islice()} does not support negative values for \var{start},
  \var{stop}, or \var{step}.  Can be used to extract related fields
  from data where the internal structure has been flattened (for
  example, a multi-line report may list a name field on every
  third line).  Equivalent to:

  \begin{verbatim}
     def islice(iterable, *args):
         s = slice(*args)
         next, stop, step = s.start or 0, s.stop, s.step or 1
         for cnt, element in enumerate(iterable):
             if cnt < next:
                 continue
             if stop is not None and cnt >= stop:
                 break
             yield element
             next += step             
  \end{verbatim}
\end{funcdesc}

\begin{funcdesc}{izip}{*iterables}
  Make an iterator that aggregates elements from each of the iterables.
  Like \function{zip()} except that it returns an iterator instead of
  a list.  Used for lock-step iteration over several iterables at a
  time.  Equivalent to:

  \begin{verbatim}
     def izip(*iterables):
         iterables = map(iter, iterables)
         while iterables:
             result = [i.next() for i in iterables]
             yield tuple(result)
  \end{verbatim}

  \versionchanged[When no iterables are specified, returns a zero length
                  iterator instead of raising a TypeError exception]{2.4}  
\end{funcdesc}

\begin{funcdesc}{repeat}{object\optional{, times}}
  Make an iterator that returns \var{object} over and over again.
  Runs indefinitely unless the \var{times} argument is specified.
  Used as argument to \function{imap()} for invariant parameters
  to the called function.  Also used with \function{izip()} to create
  an invariant part of a tuple record.  Equivalent to:

  \begin{verbatim}
     def repeat(object, times=None):
         if times is None:
             while True:
                 yield object
         else:
             for i in xrange(times):
                 yield object
  \end{verbatim}
\end{funcdesc}

\begin{funcdesc}{starmap}{function, iterable}
  Make an iterator that computes the function using arguments tuples
  obtained from the iterable.  Used instead of \function{imap()} when
  argument parameters are already grouped in tuples from a single iterable
  (the data has been ``pre-zipped'').  The difference between
  \function{imap()} and \function{starmap()} parallels the distinction
  between \code{function(a,b)} and \code{function(*c)}.
  Equivalent to:

  \begin{verbatim}
     def starmap(function, iterable):
         iterable = iter(iterable)
         while True:
             yield function(*iterable.next())
  \end{verbatim}
\end{funcdesc}

\begin{funcdesc}{takewhile}{predicate, iterable}
  Make an iterator that returns elements from the iterable as long as
  the predicate is true.  Equivalent to:

  \begin{verbatim}
     def takewhile(predicate, iterable):
         for x in iterable:
             if predicate(x):
                 yield x
             else:
                 break
  \end{verbatim}
\end{funcdesc}


\subsection{Examples \label{itertools-example}}

The following examples show common uses for each tool and
demonstrate ways they can be combined.

\begin{verbatim}

>>> amounts = [120.15, 764.05, 823.14]
>>> for checknum, amount in izip(count(1200), amounts):
...     print 'Check %d is for $%.2f' % (checknum, amount)
...
Check 1200 is for $120.15
Check 1201 is for $764.05
Check 1202 is for $823.14

>>> import operator
>>> for cube in imap(operator.pow, xrange(1,4), repeat(3)):
...    print cube
...
1
8
27

>>> reportlines = ['EuroPython', 'Roster', '', 'alex', '', 'laura',
                  '', 'martin', '', 'walter', '', 'samuele']
>>> for name in islice(reportlines, 3, None, 2):
...    print name.title()
...
Alex
Laura
Martin
Walter
Samuele

\end{verbatim}

This section shows how itertools can be combined to create other more
powerful itertools.  Note that \function{enumerate()} and \method{iteritems()}
already have efficient implementations in Python.  They are only included here
to illustrate how higher level tools can be created from building blocks.

\begin{verbatim}
def take(n, seq):
    return list(islice(seq, n))

def enumerate(iterable):
    return izip(count(), iterable)

def tabulate(function):
    "Return function(0), function(1), ..."
    return imap(function, count())

def iteritems(mapping):
    return izip(mapping.iterkeys(), mapping.itervalues())

def nth(iterable, n):
    "Returns the nth item"
    return list(islice(iterable, n, n+1))

def all(pred, seq):
    "Returns True if pred(x) is True for every element in the iterable"
    return False not in imap(pred, seq)

def some(pred, seq):
    "Returns True if pred(x) is True at least one element in the iterable"
    return True in imap(pred, seq)

def no(pred, seq):
    "Returns True if pred(x) is False for every element in the iterable"
    return True not in imap(pred, seq)

def quantify(pred, seq):
    "Count how many times the predicate is True in the sequence"
    return sum(imap(pred, seq))

def padnone(seq):
    "Returns the sequence elements and then returns None indefinitely"
    return chain(seq, repeat(None))

def ncycles(seq, n):
    "Returns the sequence elements n times"
    return chain(*repeat(seq, n))

def dotproduct(vec1, vec2):
    return sum(imap(operator.mul, vec1, vec2))

def window(seq, n=2):
    "Returns a sliding window (of width n) over data from the iterable"
    "   s -> (s0,s1,...s[n-1]), (s1,s2,...,sn), ...                   "
    it = iter(seq)
    result = tuple(islice(it, n))
    if len(result) == n:
        yield result    
    for elem in it:
        result = result[1:] + (elem,)
        yield result

def tee(iterable):
    "Return two independent iterators from a single iterable"
    def gen(next, data={}, cnt=[0]):
        dpop = data.pop
        for i in count():
            if i == cnt[0]:
                item = data[i] = next()
                cnt[0] += 1
            else:
                item = dpop(i)
            yield item
    next = iter(iterable).next
    return (gen(next), gen(next))

\end{verbatim}

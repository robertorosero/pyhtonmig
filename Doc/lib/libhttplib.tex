\section{Standard Module \module{httplib}}
\declaremodule{standard}{httplib}

\modulesynopsis{HTTP protocol client (requires sockets).}

\indexii{HTTP}{protocol}


This module defines a class which implements the client side of the
HTTP protocol.  It is normally not used directly --- the module
\module{urllib}\refstmodindex{urllib} uses it to handle URLs that use
HTTP.

The module defines one class, \class{HTTP}:

\begin{classdesc}{HTTP}{\optional{host\optional{, port}}}
An \class{HTTP} instance
represents one transaction with an HTTP server.  It should be
instantiated passing it a host and optional port number.  If no port
number is passed, the port is extracted from the host string if it has
the form \code{\var{host}:\var{port}}, else the default HTTP port (80)
is used.  If no host is passed, no connection is made, and the
\method{connect()} method should be used to connect to a server.  For
example, the following calls all create instances that connect to the
server at the same host and port:

\begin{verbatim}
>>> h1 = httplib.HTTP('www.cwi.nl')
>>> h2 = httplib.HTTP('www.cwi.nl:80')
>>> h3 = httplib.HTTP('www.cwi.nl', 80)
\end{verbatim}

Once an \class{HTTP} instance has been connected to an HTTP server, it
should be used as follows:

\begin{enumerate}

\item[1.] Make exactly one call to the \method{putrequest()} method.

\item[2.] Make zero or more calls to the \method{putheader()} method.

\item[3.] Call the \method{endheaders()} method (this can be omitted if
step 4 makes no calls).

\item[4.] Optional calls to the \method{send()} method.

\item[5.] Call the \method{getreply()} method.

\item[6.] Call the \method{getfile()} method and read the data off the
file object that it returns.

\end{enumerate}
\end{classdesc}

\subsection{HTTP Objects}

\class{HTTP} instances have the following methods:


\begin{methoddesc}{set_debuglevel}{level}
Set the debugging level (the amount of debugging output printed).
The default debug level is \code{0}, meaning no debugging output is
printed.
\end{methoddesc}

\begin{methoddesc}{connect}{host\optional{, port}}
Connect to the server given by \var{host} and \var{port}.  See the
intro for the default port.  This should be called directly only if
the instance was instantiated without passing a host.
\end{methoddesc}

\begin{methoddesc}{send}{data}
Send data to the server.  This should be used directly only after the
\method{endheaders()} method has been called and before
\method{getreply()} has been called.
\end{methoddesc}

\begin{methoddesc}{putrequest}{request, selector}
This should be the first call after the connection to the server has
been made.  It sends a line to the server consisting of the
\var{request} string, the \var{selector} string, and the HTTP version
(\code{HTTP/1.0}).
\end{methoddesc}

\begin{methoddesc}{putheader}{header, argument\optional{, ...}}
Send an \rfc{822} style header to the server.  It sends a line to the
server consisting of the header, a colon and a space, and the first
argument.  If more arguments are given, continuation lines are sent,
each consisting of a tab and an argument.
\end{methoddesc}

\begin{methoddesc}{endheaders}{}
Send a blank line to the server, signalling the end of the headers.
\end{methoddesc}

\begin{methoddesc}{getreply}{}
Complete the request by shutting down the sending end of the socket,
read the reply from the server, and return a triple
\code{(\var{replycode}, \var{message}, \var{headers})}.  Here,
\var{replycode} is the integer reply code from the request (e.g.\
\code{200} if the request was handled properly); \var{message} is the
message string corresponding to the reply code; and \var{headers} is
an instance of the class \class{mimetools.Message} containing the
headers received from the server.  See the description of the
\module{mimetools}\refstmodindex{mimetools} module.
\end{methoddesc}

\begin{methoddesc}{getfile}{}
Return a file object from which the data returned by the server can be
read, using the \method{read()}, \method{readline()} or
\method{readlines()} methods.
\end{methoddesc}

\subsection{Example}
\nodename{HTTP Example}

Here is an example session:

\begin{verbatim}
>>> import httplib
>>> h = httplib.HTTP('www.cwi.nl')
>>> h.putrequest('GET', '/index.html')
>>> h.putheader('Accept', 'text/html')
>>> h.putheader('Accept', 'text/plain')
>>> h.endheaders()
>>> errcode, errmsg, headers = h.getreply()
>>> print errcode # Should be 200
>>> f = h.getfile()
>>> data = f.read() # Get the raw HTML
>>> f.close()
\end{verbatim}

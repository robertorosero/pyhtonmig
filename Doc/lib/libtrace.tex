\section{\module{trace} ---
         Trace or track Python statement execution}

\declaremodule{standard}{trace}
\modulesynopsis{Trace or track Python statement execution.}

The \module{trace} module allows you to trace program execution, generate
annotated statement coverage listings, print caller/callee relationships and
list functions executed during a program run.  It can be used in another
program or from the command line.

\subsection{Command Line Usage\label{trace-cli}}

The \module{trace} module can be invoked from the command line.  It can be
as simple as

\begin{verbatim}
python -m trace --count somefile.py ...
\end{verbatim}

The above will generate annotated listings of all Python modules imported
during the execution of \file{somefile.py}.

The following command-line arguments are supported:

\begin{description}
\item[\longprogramopt{trace}, \programopt{-t}]
Display lines as they are executed.

\item[\longprogramopt{count}, \programopt{-c}]
Produce a set of  annotated listing files upon program
completion that shows how many times each statement was executed.

\item[\longprogramopt{report}, \programopt{-r}]
Produce an annotated list from an earlier program run that
used the \longprogramopt{count} and \longprogramopt{file} arguments.

\item[\longprogramopt{no-report}, \programopt{-R}]
Do not generate annotated listings.  This is useful if you intend to make
several runs with \longprogramopt{count} then produce a single set
of annotated listings at the end.

\item[\longprogramopt{listfuncs}, \programopt{-l}]
List the functions executed by running the program.

\item[\longprogramopt{trackcalls}, \programopt{-T}]
Generate calling relationships exposed by running the program.

\item[\longprogramopt{file}, \programopt{-f}]
Name a file containing (or to contain) counts.

\item[\longprogramopt{coverdir}, \programopt{-C}]
Name a directory in which to save annotated listing files.

\item[\longprogramopt{missing}, \programopt{-m}]
When generating annotated listings, mark lines which
were not executed with \code{>}\code{>}\code{>}\code{>}\code{>}\code{>}.

\item[\longprogramopt{summary}, \programopt{-s}]
When using \longprogramopt{count} or \longprogramopt{report}, write a
brief summary to stdout for each file processed.

\item[\longprogramopt{ignore-module}]
Ignore the named module and its submodules (if it is
a package).  May be given multiple times.

\item[\longprogramopt{ignore-dir}]
Ignore all modules and packages in the named directory
and subdirectories.  May be given multiple times.
\end{description}

\subsection{Programming Interface\label{trace-api}}

\begin{classdesc}{Trace}{\optional{count=1\optional{, trace=1\optional{,
                         countfuncs=0\optional{, countcallers=0\optional{,
                         ignoremods=()\optional{, ignoredirs=()\optional{,
                         infile=None\optional{, outfile=None}}}}}}}}}
Create an object to trace execution of a single statement or expression.
All parameters are optional.  \var{count} enables counting of line numbers.
\var{trace} enables line execution tracing.  \var{countfuncs} enables
listing of the functions called during the run.  \var{countcallers} enables
call relationship tracking.  \var{ignoremods} is a list of modules or
packages to ignore.  \var{ignoredirs} is a list of directories whose modules
or packages should be ignored.  \var{infile} is the file from which to read
stored count information.  \var{outfile} is a file in which to write updated
count information.
\end{classdesc}

\begin{methoddesc}[Trace]{run}{cmd}
Run \var{cmd} under control of the Trace object with the current tracing
parameters.
\end{methoddesc}

\begin{methoddesc}[Trace]{runctx}{cmd\optional{, globals=None\optional{,
                                  locals=None}}}
Run \var{cmd} under control of the Trace object with the current tracing
parameters in the defined global and local environments.  If not defined,
\var{globals} and \var{locals} default to empty dictionaries.
\end{methoddesc}

\begin{methoddesc}[Trace]{runfunc}{func, *args, **kwds}
Call \var{func} with the given arguments under control of the
\class{Trace} object with the current tracing parameters.
\end{methoddesc}

This is a simple example showing the use of this module:

\begin{verbatim}
import sys
import trace

# create a Trace object, telling it what to ignore, and whether to
# do tracing or line-counting or both.
tracer = trace.Trace(
    ignoredirs=[sys.prefix, sys.exec_prefix],
    trace=0,
    count=1)

# run the new command using the given tracer
tracer.run('main()')

# make a report, placing output in /tmp
r = tracer.results()
r.write_results(show_missing=True, coverdir="/tmp")
\end{verbatim}

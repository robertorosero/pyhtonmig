\section{\module{socket} ---
         Low-level networking interface}

\declaremodule{builtin}{socket}
\modulesynopsis{Low-level networking interface.}


This module provides access to the BSD \emph{socket} interface.
It is available on all modern \UNIX{} systems, Windows, MacOS, BeOS,
OS/2, and probably additional platforms.

For an introduction to socket programming (in C), see the following
papers: \citetitle{An Introductory 4.3BSD Interprocess Communication
Tutorial}, by Stuart Sechrest and \citetitle{An Advanced 4.3BSD
Interprocess Communication Tutorial}, by Samuel J.  Leffler et al,
both in the \citetitle{\UNIX{} Programmer's Manual, Supplementary Documents 1}
(sections PS1:7 and PS1:8).  The platform-specific reference material
for the various socket-related system calls are also a valuable source
of information on the details of socket semantics.  For \UNIX, refer
to the manual pages; for Windows, see the WinSock (or Winsock 2)
specification.
For IPv6-ready APIs, readers may want to refer to \rfc{2553} titled
\citetitle{Basic Socket Interface Extensions for IPv6}.

The Python interface is a straightforward transliteration of the
\UNIX{} system call and library interface for sockets to Python's
object-oriented style: the \function{socket()} function returns a
\dfn{socket object}\obindex{socket} whose methods implement the
various socket system calls.  Parameter types are somewhat
higher-level than in the C interface: as with \method{read()} and
\method{write()} operations on Python files, buffer allocation on
receive operations is automatic, and buffer length is implicit on send
operations.

Socket addresses are represented as follows:
A single string is used for the \constant{AF_UNIX} address family.
A pair \code{(\var{host}, \var{port})} is used for the
\constant{AF_INET} address family, where \var{host} is a string
representing either a hostname in Internet domain notation like
\code{'daring.cwi.nl'} or an IPv4 address like \code{'100.50.200.5'},
and \var{port} is an integral port number.
For \constant{AF_INET6} address family, a four-tuple
\code{(\var{host}, \var{port}, \var{flowinfo}, \var{scopeid})} is
used, where \var{flowinfo} and \var{scopeid} represents
\code{sin6_flowinfo} and \code{sin6_scope_id} member in
\constant{struct sockaddr_in6} in C.
For \module{socket} module methods, \var{flowinfo} and \var{scopeid}
can be omitted just for backward compatibility. Note, however,
omission of \var{scopeid} can cause problems in manipulating scoped
IPv6 addresses. Other address families are currently not supported.
The address format required by a particular socket object is
automatically selected based on the address family specified when the
socket object was created.

For IPv4 addresses, two special forms are accepted instead of a host
address: the empty string represents \constant{INADDR_ANY}, and the string
\code{'<broadcast>'} represents \constant{INADDR_BROADCAST}.
The behavior is not available for IPv6 for backward compatibility,
therefore, you may want to avoid these if you intend to support IPv6 with
your Python programs.

If you use a hostname in the \var{host} portion of IPv4/v6 socket
address, the program may show a nondeterministic behavior, as Python
uses the first address returned from the DNS resolution.  The socket
address will be resolved differently into an actual IPv4/v6 address,
depending on the results from DNS resolution and/or the host
configuration.  For deterministic behavior use a numeric address in
\var{host} portion.

All errors raise exceptions.  The normal exceptions for invalid
argument types and out-of-memory conditions can be raised; errors
related to socket or address semantics raise the error
\exception{socket.error}.

Non-blocking mode is supported through
\method{setblocking()}.  A generalization of this based on timeouts
is supported through \method{settimeout()}.

The module \module{socket} exports the following constants and functions:


\begin{excdesc}{error}
This exception is raised for socket-related errors.
The accompanying value is either a string telling what went wrong or a
pair \code{(\var{errno}, \var{string})}
representing an error returned by a system
call, similar to the value accompanying \exception{os.error}.
See the module \refmodule{errno}\refbimodindex{errno}, which contains
names for the error codes defined by the underlying operating system.
\end{excdesc}

\begin{excdesc}{herror}
This exception is raised for address-related errors, i.e. for
functions that use \var{h_errno} in the C API, including
\function{gethostbyname_ex()} and \function{gethostbyaddr()}.

The accompanying value is a pair \code{(\var{h_errno}, \var{string})}
representing an error returned by a library call. \var{string}
represents the description of \var{h_errno}, as returned by
the \cfunction{hstrerror()} C function.
\end{excdesc}

\begin{excdesc}{gaierror}
This exception is raised for address-related errors, for
\function{getaddrinfo()} and \function{getnameinfo()}.
The accompanying value is a pair \code{(\var{error}, \var{string})}
representing an error returned by a library call.
\var{string} represents the description of \var{error}, as returned
by the \cfunction{gai_strerror()} C function.
\end{excdesc}

\begin{excdesc}{timeout}
This exception is raised when a timeout occurs on a socket which has
had timeouts enabled via a prior call to \method{settimeout()}.  The
accompanying value is a string whose value is currently always ``timed
out''.
\versionadded{2.3}
\end{excdesc}

\begin{datadesc}{AF_UNIX}
\dataline{AF_INET}
\dataline{AF_INET6}
These constants represent the address (and protocol) families,
used for the first argument to \function{socket()}.  If the
\constant{AF_UNIX} constant is not defined then this protocol is
unsupported.
\end{datadesc}

\begin{datadesc}{SOCK_STREAM}
\dataline{SOCK_DGRAM}
\dataline{SOCK_RAW}
\dataline{SOCK_RDM}
\dataline{SOCK_SEQPACKET}
These constants represent the socket types,
used for the second argument to \function{socket()}.
(Only \constant{SOCK_STREAM} and
\constant{SOCK_DGRAM} appear to be generally useful.)
\end{datadesc}

\begin{datadesc}{SO_*}
\dataline{SOMAXCONN}
\dataline{MSG_*}
\dataline{SOL_*}
\dataline{IPPROTO_*}
\dataline{IPPORT_*}
\dataline{INADDR_*}
\dataline{IP_*}
\dataline{IPV6_*}
\dataline{EAI_*}
\dataline{AI_*}
\dataline{NI_*}
\dataline{TCP_*}
Many constants of these forms, documented in the \UNIX{} documentation on
sockets and/or the IP protocol, are also defined in the socket module.
They are generally used in arguments to the \method{setsockopt()} and
\method{getsockopt()} methods of socket objects.  In most cases, only
those symbols that are defined in the \UNIX{} header files are defined;
for a few symbols, default values are provided.
\end{datadesc}

\begin{datadesc}{has_ipv6}
This constant contains a boolean value which indicates if IPv6 is
supported on this platform.
\versionadded{2.3}
\end{datadesc}

\begin{funcdesc}{getaddrinfo}{host, port\optional{, family\optional{,
                              socktype\optional{, proto\optional{,
                              flags}}}}}
Resolves the \var{host}/\var{port} argument, into a sequence of
5-tuples that contain all the necessary argument for the sockets
manipulation. \var{host} is a domain name, a string representation of
IPv4/v6 address or \code{None}.
\var{port} is a string service name (like \code{'http'}), a numeric
port number or \code{None}.

The rest of the arguments are optional and must be numeric if
specified.  For \var{host} and \var{port}, by passing either an empty
string or \code{None}, you can pass \code{NULL} to the C API.  The
\function{getaddrinfo()} function returns a list of 5-tuples with
the following structure:

\code{(\var{family}, \var{socktype}, \var{proto}, \var{canonname},
      \var{sockaddr})}

\var{family}, \var{socktype}, \var{proto} are all integer and are meant to
be passed to the \function{socket()} function.
\var{canonname} is a string representing the canonical name of the \var{host}.
It can be a numeric IPv4/v6 address when \constant{AI_CANONNAME} is specified
for a numeric \var{host}.
\var{sockaddr} is a tuple describing a socket address, as described above.
See the source for the \refmodule{httplib} and other library modules
for a typical usage of the function.
\versionadded{2.2}
\end{funcdesc}

\begin{funcdesc}{getfqdn}{\optional{name}}
Return a fully qualified domain name for \var{name}.
If \var{name} is omitted or empty, it is interpreted as the local
host.  To find the fully qualified name, the hostname returned by
\function{gethostbyaddr()} is checked, then aliases for the host, if
available.  The first name which includes a period is selected.  In
case no fully qualified domain name is available, the hostname is
returned.
\versionadded{2.0}
\end{funcdesc}

\begin{funcdesc}{gethostbyname}{hostname}
Translate a host name to IPv4 address format.  The IPv4 address is
returned as a string, such as  \code{'100.50.200.5'}.  If the host name
is an IPv4 address itself it is returned unchanged.  See
\function{gethostbyname_ex()} for a more complete interface.
\function{gethostbyname()} does not support IPv6 name resolution, and
\function{getaddrinfo()} should be used instead for IPv4/v6 dual stack support.
\end{funcdesc}

\begin{funcdesc}{gethostbyname_ex}{hostname}
Translate a host name to IPv4 address format, extended interface.
Return a triple \code{(\var{hostname}, \var{aliaslist},
\var{ipaddrlist})} where
\var{hostname} is the primary host name responding to the given
\var{ip_address}, \var{aliaslist} is a (possibly empty) list of
alternative host names for the same address, and \var{ipaddrlist} is
a list of IPv4 addresses for the same interface on the same
host (often but not always a single address).
\function{gethostbyname_ex()} does not support IPv6 name resolution, and
\function{getaddrinfo()} should be used instead for IPv4/v6 dual stack support.
\end{funcdesc}

\begin{funcdesc}{gethostname}{}
Return a string containing the hostname of the machine where 
the Python interpreter is currently executing.
If you want to know the current machine's IP address, you may want to use
\code{gethostbyname(gethostname())}.
This operation assumes that there is a valid address-to-host mapping for
the host, and the assumption does not always hold.
Note: \function{gethostname()} doesn't always return the fully qualified
domain name; use \code{gethostbyaddr(gethostname())}
(see below).
\end{funcdesc}

\begin{funcdesc}{gethostbyaddr}{ip_address}
Return a triple \code{(\var{hostname}, \var{aliaslist},
\var{ipaddrlist})} where \var{hostname} is the primary host name
responding to the given \var{ip_address}, \var{aliaslist} is a
(possibly empty) list of alternative host names for the same address,
and \var{ipaddrlist} is a list of IPv4/v6 addresses for the same interface
on the same host (most likely containing only a single address).
To find the fully qualified domain name, use the function
\function{getfqdn()}.
\function{gethostbyaddr} supports both IPv4 and IPv6.
\end{funcdesc}

\begin{funcdesc}{getnameinfo}{sockaddr, flags}
Translate a socket address \var{sockaddr} into a 2-tuple
\code{(\var{host}, \var{port})}.
Depending on the settings of \var{flags}, the result can contain a
fully-qualified domain name or numeric address representation in
\var{host}.  Similarly, \var{port} can contain a string port name or a
numeric port number.
\versionadded{2.2}
\end{funcdesc}

\begin{funcdesc}{getprotobyname}{protocolname}
Translate an Internet protocol name (for example, \code{'icmp'}) to a constant
suitable for passing as the (optional) third argument to the
\function{socket()} function.  This is usually only needed for sockets
opened in ``raw'' mode (\constant{SOCK_RAW}); for the normal socket
modes, the correct protocol is chosen automatically if the protocol is
omitted or zero.
\end{funcdesc}

\begin{funcdesc}{getservbyname}{servicename, protocolname}
Translate an Internet service name and protocol name to a port number
for that service.  The protocol name should be \code{'tcp'} or
\code{'udp'}.
\end{funcdesc}

\begin{funcdesc}{socket}{family, type\optional{, proto}}
Create a new socket using the given address family, socket type and
protocol number.  The address family should be \constant{AF_INET}, \constant{AF_INET6} or
\constant{AF_UNIX}.  The socket type should be \constant{SOCK_STREAM},
\constant{SOCK_DGRAM} or perhaps one of the other \samp{SOCK_} constants.
The protocol number is usually zero and may be omitted in that case.
\end{funcdesc}

\begin{funcdesc}{ssl}{sock\optional{, keyfile, certfile}}
Initiate a SSL connection over the socket \var{sock}. \var{keyfile} is
the name of a PEM formatted file that contains your private
key. \var{certfile} is a PEM formatted certificate chain file. On
success, a new \class{SSLObject} is returned.

\warning{This does not do any certificate verification!}
\end{funcdesc}

\begin{funcdesc}{fromfd}{fd, family, type\optional{, proto}}
Build a socket object from an existing file descriptor (an integer as
returned by a file object's \method{fileno()} method).  Address family,
socket type and protocol number are as for the \function{socket()} function
above.  The file descriptor should refer to a socket, but this is not
checked --- subsequent operations on the object may fail if the file
descriptor is invalid.  This function is rarely needed, but can be
used to get or set socket options on a socket passed to a program as
standard input or output (such as a server started by the \UNIX{} inet
daemon).  The socket is assumed to be in blocking mode.
Availability: \UNIX.
\end{funcdesc}

\begin{funcdesc}{ntohl}{x}
Convert 32-bit integers from network to host byte order.  On machines
where the host byte order is the same as network byte order, this is a
no-op; otherwise, it performs a 4-byte swap operation.
\end{funcdesc}

\begin{funcdesc}{ntohs}{x}
Convert 16-bit integers from network to host byte order.  On machines
where the host byte order is the same as network byte order, this is a
no-op; otherwise, it performs a 2-byte swap operation.
\end{funcdesc}

\begin{funcdesc}{htonl}{x}
Convert 32-bit integers from host to network byte order.  On machines
where the host byte order is the same as network byte order, this is a
no-op; otherwise, it performs a 4-byte swap operation.
\end{funcdesc}

\begin{funcdesc}{htons}{x}
Convert 16-bit integers from host to network byte order.  On machines
where the host byte order is the same as network byte order, this is a
no-op; otherwise, it performs a 2-byte swap operation.
\end{funcdesc}

\begin{funcdesc}{inet_aton}{ip_string}
Convert an IPv4 address from dotted-quad string format (for example,
'123.45.67.89') to 32-bit packed binary format, as a string four
characters in length.  This is useful when conversing with a program
that uses the standard C library and needs objects of type
\ctype{struct in_addr}, which is the C type for the 32-bit packed
binary this function returns.

If the IPv4 address string passed to this function is invalid,
\exception{socket.error} will be raised. Note that exactly what is
valid depends on the underlying C implementation of
\cfunction{inet_aton()}.

\function{inet_aton()} does not support IPv6, and
\function{getnameinfo()} should be used instead for IPv4/v6 dual stack
support.
\end{funcdesc}

\begin{funcdesc}{inet_ntoa}{packed_ip}
Convert a 32-bit packed IPv4 address (a string four characters in
length) to its standard dotted-quad string representation (for
example, '123.45.67.89').  This is useful when conversing with a
program that uses the standard C library and needs objects of type
\ctype{struct in_addr}, which is the C type for the 32-bit packed
binary data this function takes as an argument.

If the string passed to this function is not exactly 4 bytes in
length, \exception{socket.error} will be raised.
\function{inet_ntoa()} does not support IPv6, and
\function{getnameinfo()} should be used instead for IPv4/v6 dual stack
support.
\end{funcdesc}

\begin{funcdesc}{inet_pton}{address_family, ip_string}
Convert an IP address from its family-specific string format to a packed,
binary format.
\function{inet_pton()} is useful when a library or network protocol calls for
an object of type \ctype{struct in_addr} (similar to \function{inet_aton()})
or \ctype{struct in6_addr}.

Supported values for \var{address_family} are currently
\constant{AF_INET} and \constant{AF_INET6}.
If the IP address string \var{ip_string} is invalid,
\exception{socket.error} will be raised. Note that exactly what is valid
depends on both the value of \var{address_family} and the underlying
implementation of \cfunction{inet_pton()}.

Availability: \UNIX{} (maybe not all platforms).
\versionadded{2.3}
\end{funcdesc}

\begin{funcdesc}{inet_ntop}{address_family, packed_ip}
Convert a packed IP address (a string of some number of characters) to
its standard, family-specific string representation (for example,
\code{'7.10.0.5'} or \code{'5aef:2b::8'})
\function{inet_ntop()} is useful when a library or network protocol returns
an object of type \ctype{struct in_addr} (similar to \function{inet_ntoa()})
or \ctype{struct in6_addr}.

Supported values for \var{address_family} are currently
\constant{AF_INET} and \constant{AF_INET6}.
If the string \var{packed_ip} is not the correct length for the
specified address family, \exception{ValueError} will be raised.  A
\exception{socket.error} is raised for errors from the call to
\function{inet_ntop()}.

Availability: \UNIX{} (maybe not all platforms).
\versionadded{2.3}
\end{funcdesc}

\begin{funcdesc}{getdefaulttimeout}{}
Return the default timeout in floating seconds for new socket objects.
A value of \code{None} indicates that new socket objects have no timeout.
When the socket module is first imported, the default is \code{None}.
\versionadded{2.3}
\end{funcdesc}

\begin{funcdesc}{setdefaulttimeout}{timeout}
Set the default timeout in floating seconds for new socket objects.
A value of \code{None} indicates that new socket objects have no timeout.
When the socket module is first imported, the default is \code{None}.
\versionadded{2.3}
\end{funcdesc}

\begin{datadesc}{SocketType}
This is a Python type object that represents the socket object type.
It is the same as \code{type(socket(...))}.
\end{datadesc}


\begin{seealso}
  \seemodule{SocketServer}{Classes that simplify writing network servers.}
\end{seealso}


\subsection{Socket Objects \label{socket-objects}}

Socket objects have the following methods.  Except for
\method{makefile()} these correspond to \UNIX{} system calls
applicable to sockets.

\begin{methoddesc}[socket]{accept}{}
Accept a connection.
The socket must be bound to an address and listening for connections.
The return value is a pair \code{(\var{conn}, \var{address})}
where \var{conn} is a \emph{new} socket object usable to send and
receive data on the connection, and \var{address} is the address bound
to the socket on the other end of the connection.
\end{methoddesc}

\begin{methoddesc}[socket]{bind}{address}
Bind the socket to \var{address}.  The socket must not already be bound.
(The format of \var{address} depends on the address family --- see
above.)  \note{This method has historically accepted a pair
of parameters for \constant{AF_INET} addresses instead of only a
tuple.  This was never intentional and is no longer available in
Python 2.0 and later.}
\end{methoddesc}

\begin{methoddesc}[socket]{close}{}
Close the socket.  All future operations on the socket object will fail.
The remote end will receive no more data (after queued data is flushed).
Sockets are automatically closed when they are garbage-collected.
\end{methoddesc}

\begin{methoddesc}[socket]{connect}{address}
Connect to a remote socket at \var{address}.
(The format of \var{address} depends on the address family --- see
above.)  \note{This method has historically accepted a pair
of parameters for \constant{AF_INET} addresses instead of only a
tuple.  This was never intentional and is no longer available in
Python 2.0 and later.}
\end{methoddesc}

\begin{methoddesc}[socket]{connect_ex}{address}
Like \code{connect(\var{address})}, but return an error indicator
instead of raising an exception for errors returned by the C-level
\cfunction{connect()} call (other problems, such as ``host not found,''
can still raise exceptions).  The error indicator is \code{0} if the
operation succeeded, otherwise the value of the \cdata{errno}
variable.  This is useful to support, for example, asynchronous connects.
\note{This method has historically accepted a pair of
parameters for \constant{AF_INET} addresses instead of only a tuple.
This was never intentional and is no longer available in Python
2.0 and later.}
\end{methoddesc}

\begin{methoddesc}[socket]{fileno}{}
Return the socket's file descriptor (a small integer).  This is useful
with \function{select.select()}.
\end{methoddesc}

\begin{methoddesc}[socket]{getpeername}{}
Return the remote address to which the socket is connected.  This is
useful to find out the port number of a remote IPv4/v6 socket, for instance.
(The format of the address returned depends on the address family ---
see above.)  On some systems this function is not supported.
\end{methoddesc}

\begin{methoddesc}[socket]{getsockname}{}
Return the socket's own address.  This is useful to find out the port
number of an IPv4/v6 socket, for instance.
(The format of the address returned depends on the address family ---
see above.)
\end{methoddesc}

\begin{methoddesc}[socket]{getsockopt}{level, optname\optional{, buflen}}
Return the value of the given socket option (see the \UNIX{} man page
\manpage{getsockopt}{2}).  The needed symbolic constants
(\constant{SO_*} etc.) are defined in this module.  If \var{buflen}
is absent, an integer option is assumed and its integer value
is returned by the function.  If \var{buflen} is present, it specifies
the maximum length of the buffer used to receive the option in, and
this buffer is returned as a string.  It is up to the caller to decode
the contents of the buffer (see the optional built-in module
\refmodule{struct} for a way to decode C structures encoded as strings).
\end{methoddesc}

\begin{methoddesc}[socket]{listen}{backlog}
Listen for connections made to the socket.  The \var{backlog} argument
specifies the maximum number of queued connections and should be at
least 1; the maximum value is system-dependent (usually 5).
\end{methoddesc}

\begin{methoddesc}[socket]{makefile}{\optional{mode\optional{, bufsize}}}
Return a \dfn{file object} associated with the socket.  (File objects
are described in \ref{bltin-file-objects}, ``File Objects.'')
The file object references a \cfunction{dup()}ped version of the
socket file descriptor, so the file object and socket object may be
closed or garbage-collected independently.
The socket should be in blocking mode.
\index{I/O control!buffering}The optional \var{mode}
and \var{bufsize} arguments are interpreted the same way as by the
built-in \function{file()} function; see ``Built-in Functions''
(section \ref{built-in-funcs}) for more information.
\end{methoddesc}

\begin{methoddesc}[socket]{recv}{bufsize\optional{, flags}}
Receive data from the socket.  The return value is a string representing
the data received.  The maximum amount of data to be received
at once is specified by \var{bufsize}.  See the \UNIX{} manual page
\manpage{recv}{2} for the meaning of the optional argument
\var{flags}; it defaults to zero.
\end{methoddesc}

\begin{methoddesc}[socket]{recvfrom}{bufsize\optional{, flags}}
Receive data from the socket.  The return value is a pair
\code{(\var{string}, \var{address})} where \var{string} is a string
representing the data received and \var{address} is the address of the
socket sending the data.  The optional \var{flags} argument has the
same meaning as for \method{recv()} above.
(The format of \var{address} depends on the address family --- see above.)
\end{methoddesc}

\begin{methoddesc}[socket]{send}{string\optional{, flags}}
Send data to the socket.  The socket must be connected to a remote
socket.  The optional \var{flags} argument has the same meaning as for
\method{recv()} above.  Returns the number of bytes sent.
Applications are responsible for checking that all data has been sent;
if only some of the data was transmitted, the application needs to
attempt delivery of the remaining data.
\end{methoddesc}

\begin{methoddesc}[socket]{sendall}{string\optional{, flags}}
Send data to the socket.  The socket must be connected to a remote
socket.  The optional \var{flags} argument has the same meaning as for
\method{recv()} above.  Unlike \method{send()}, this method continues
to send data from \var{string} until either all data has been sent or
an error occurs.  \code{None} is returned on success.  On error, an
exception is raised, and there is no way to determine how much data,
if any, was successfully sent.
\end{methoddesc}

\begin{methoddesc}[socket]{sendto}{string\optional{, flags}, address}
Send data to the socket.  The socket should not be connected to a
remote socket, since the destination socket is specified by
\var{address}.  The optional \var{flags} argument has the same
meaning as for \method{recv()} above.  Return the number of bytes sent.
(The format of \var{address} depends on the address family --- see above.)
\end{methoddesc}

\begin{methoddesc}[socket]{setblocking}{flag}
Set blocking or non-blocking mode of the socket: if \var{flag} is 0,
the socket is set to non-blocking, else to blocking mode.  Initially
all sockets are in blocking mode.  In non-blocking mode, if a
\method{recv()} call doesn't find any data, or if a
\method{send()} call can't immediately dispose of the data, a
\exception{error} exception is raised; in blocking mode, the calls
block until they can proceed.
\code{s.setblocking(0)} is equivalent to \code{s.settimeout(0)};
\code{s.setblocking(1)} is equivalent to \code{s.settimeout(None)}.
\end{methoddesc}

\begin{methoddesc}[socket]{settimeout}{value}
Set a timeout on blocking socket operations.  The \var{value} argument
can be a nonnegative float expressing seconds, or \code{None}.
If a float is
given, subsequent socket operations will raise an \exception{timeout}
exception if the timeout period \var{value} has elapsed before the
operation has completed.  Setting a timeout of \code{None} disables
timeouts on socket operations.
\code{s.settimeout(0.0)} is equivalent to \code{s.setblocking(0)};
\code{s.settimeout(None)} is equivalent to \code{s.setblocking(1)}.
\versionadded{2.3}
\end{methoddesc}

\begin{methoddesc}[socket]{gettimeout}{}
Returns the timeout in floating seconds associated with socket
operations, or \code{None} if no timeout is set.  This reflects
the last call to \method{setblocking()} or \method{settimeout()}.
\versionadded{2.3}
\end{methoddesc}

Some notes on socket blocking and timeouts: A socket object can be in
one of three modes: blocking, non-blocking, or timeout.  Sockets are
always created in blocking mode.  In blocking mode, operations block
until complete.  In non-blocking mode, operations fail (with an error
that is unfortunately system-dependent) if they cannot be completed
immediately.  In timeout mode, operations fail if they cannot be
completed within the timeout specified for the socket.  The
\method{setblocking()} method is simply a shorthand for certain
\method{settimeout()} calls.

Timeout mode internally sets the socket in non-blocking mode.  The
blocking and timeout modes are shared between file descriptors and
socket objects that refer to the same network endpoint.  A consequence
of this is that file objects returned by the \method{makefile()}
method should only be used when the socket is in blocking mode; in
timeout or non-blocking mode file operations that cannot be completed
immediately will fail.

\begin{methoddesc}[socket]{setsockopt}{level, optname, value}
Set the value of the given socket option (see the \UNIX{} manual page
\manpage{setsockopt}{2}).  The needed symbolic constants are defined in
the \module{socket} module (\constant{SO_*} etc.).  The value can be an
integer or a string representing a buffer.  In the latter case it is
up to the caller to ensure that the string contains the proper bits
(see the optional built-in module
\refmodule{struct}\refbimodindex{struct} for a way to encode C
structures as strings). 
\end{methoddesc}

\begin{methoddesc}[socket]{shutdown}{how}
Shut down one or both halves of the connection.  If \var{how} is
\code{0}, further receives are disallowed.  If \var{how} is \code{1},
further sends are disallowed.  If \var{how} is \code{2}, further sends
and receives are disallowed.
\end{methoddesc}

Note that there are no methods \method{read()} or \method{write()};
use \method{recv()} and \method{send()} without \var{flags} argument
instead.


\subsection{SSL Objects \label{ssl-objects}}

SSL objects have the following methods.

\begin{methoddesc}{write}{s}
Writes the string \var{s} to the on the object's SSL connection.
The return value is the number of bytes written.
\end{methoddesc}

\begin{methoddesc}{read}{\optional{n}}
If \var{n} is provided, read \var{n} bytes from the SSL connection, otherwise
read until EOF. The return value is a string of the bytes read.
\end{methoddesc}

\subsection{Example \label{socket-example}}

Here are four minimal example programs using the TCP/IP protocol:\ a
server that echoes all data that it receives back (servicing only one
client), and a client using it.  Note that a server must perform the
sequence \function{socket()}, \method{bind()}, \method{listen()},
\method{accept()} (possibly repeating the \method{accept()} to service
more than one client), while a client only needs the sequence
\function{socket()}, \method{connect()}.  Also note that the server
does not \method{send()}/\method{recv()} on the 
socket it is listening on but on the new socket returned by
\method{accept()}.

The first two examples support IPv4 only.

\begin{verbatim}
# Echo server program
import socket

HOST = ''                 # Symbolic name meaning the local host
PORT = 50007              # Arbitrary non-privileged port
s = socket.socket(socket.AF_INET, socket.SOCK_STREAM)
s.bind((HOST, PORT))
s.listen(1)
conn, addr = s.accept()
print 'Connected by', addr
while 1:
    data = conn.recv(1024)
    if not data: break
    conn.send(data)
conn.close()
\end{verbatim}

\begin{verbatim}
# Echo client program
import socket

HOST = 'daring.cwi.nl'    # The remote host
PORT = 50007              # The same port as used by the server
s = socket.socket(socket.AF_INET, socket.SOCK_STREAM)
s.connect((HOST, PORT))
s.send('Hello, world')
data = s.recv(1024)
s.close()
print 'Received', `data`
\end{verbatim}

The next two examples are identical to the above two, but support both
IPv4 and IPv6.
The server side will listen to the first address family available
(it should listen to both instead).
On most of IPv6-ready systems, IPv6 will take precedence
and the server may not accept IPv4 traffic.
The client side will try to connect to the all addresses returned as a result
of the name resolution, and sends traffic to the first one connected
successfully.

\begin{verbatim}
# Echo server program
import socket
import sys

HOST = ''                 # Symbolic name meaning the local host
PORT = 50007              # Arbitrary non-privileged port
s = None
for res in socket.getaddrinfo(HOST, PORT, socket.AF_UNSPEC, socket.SOCK_STREAM, 0, socket.AI_PASSIVE):
    af, socktype, proto, canonname, sa = res
    try:
	s = socket.socket(af, socktype, proto)
    except socket.error, msg:
	s = None
	continue
    try:
	s.bind(sa)
	s.listen(1)
    except socket.error, msg:
	s.close()
	s = None
	continue
    break
if s is None:
    print 'could not open socket'
    sys.exit(1)
conn, addr = s.accept()
print 'Connected by', addr
while 1:
    data = conn.recv(1024)
    if not data: break
    conn.send(data)
conn.close()
\end{verbatim}

\begin{verbatim}
# Echo client program
import socket
import sys

HOST = 'daring.cwi.nl'    # The remote host
PORT = 50007              # The same port as used by the server
s = None
for res in socket.getaddrinfo(HOST, PORT, socket.AF_UNSPEC, socket.SOCK_STREAM):
    af, socktype, proto, canonname, sa = res
    try:
	s = socket.socket(af, socktype, proto)
    except socket.error, msg:
	s = None
	continue
    try:
	s.connect(sa)
    except socket.error, msg:
	s.close()
	s = None
	continue
    break
if s is None:
    print 'could not open socket'
    sys.exit(1)
s.send('Hello, world')
data = s.recv(1024)
s.close()
print 'Received', `data`
\end{verbatim}

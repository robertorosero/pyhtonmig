\section{Built-in module \sectcode{htmllib}}
\stmodindex{htmllib}
\index{HTML}
\index{hypertext}

\renewcommand{\indexsubitem}{(in module htmllib)}

This module defines a number of classes which can serve as a basis for
parsing text files formatted in HTML (HyperText Mark-up Language).
The classes are not directly concerned with I/O --- the have to be fed
their input in string form, and will make calls to methods of a
``formatter'' object in order to produce output.  The classes are
designed to be used as base classes for other classes in order to add
functionality, and allow most of their methods to be extended or
overridden.  In turn, the classes are derived from and extend the
class \code{SGMLParser} defined in module \code{sgmllib}.
\index{SGML}
\stmodindex{sgmllib}
\ttindex{SGMLParser}
\index{formatter}

The following is a summary of the interface defined by
\code{sgmllib.SGMLParser}:

\begin{itemize}

\item
The interface to feed data to an instance is through the \code{feed()}
method, which takes a string argument.  This can be called with as
little or as much text at a time.  When the data contains complete
HTML elements, these are processed immediately; incomplete elements
are saved in a buffer.  To force processing of all unprocessed data,
call the \code{close()} method.

Example: to parse the entire contents of a file, do
\code{parser.feed(open(file).read()); parser.close()}.

\item
The interface to define semantics for HTML tags is very simple: derive
a class and define methods called \code{start_\var{tag}()},
\code{end_\var{tag}()}, or \code{do_\var{tag}()}.  The parser will
call these at appropriate moments: \code{start_\var{tag}} or
\code{do_\var{tag}} is called when an opening tag of the form
\code{<\var{tag} ...>} is encountered; \code{end_\var{tag}} is called
when a closing tag of the form \code{<\var{tag}>} is encountered.  If
an opening tag requires a corresponding closing tag, like \code{<H1>}
... \code{</H1>}, the class should define the \code{start_\var{tag}}
method; if a tag requires no closing tag, like \code{<P>}, the class
should define the \code{do_\var{tag}} method.

\end{itemize}

The module defines the following classes:

\begin{funcdesc}{HTMLParser}{}
This is the most basic HTML parser class.  It defines one additional
entity name over the names defined by the \code{SGMLParser} base
class, \code{\&bullet;}.  It also defines handlers for the following
tags: \code{<LISTING>...</LISTING>}, \code{<XMP>...</XMP>}, and
\code{<PLAINTEXT>} (the latter is terminated only by end of file).
\end{funcdesc}

\begin{funcdesc}{CollectingParser}{}
This class, derived from \code{HTMLParser}, collects various useful
bits of information from the HTML text.  To this end it defines
additional handlers for the following tags: \code{<A>...</A>},
\code{<HEAD>...</HEAD>}, \code{<BODY>...</BODY>},
\code{<TITLE>...</TITLE>}, \code{<NEXTID>}, and \code{<ISINDEX>}.
\end{funcdesc}

\begin{funcdesc}{FormattingParser}{formatter\, stylesheet}
This class, derived from \code{CollectingParser}, interprets a wide
selection of HTML tags so it can produce formatted output from the
parsed data.  It is initialized with two objects, a \var{formatter}
which should define a number of methods to format text into
paragraphs, and a \var{stylesheet} which defines a number of static
parameters for the formatting process.  Formatters and style sheets
are documented later in this section.
\index{formatter}
\index{style sheet}
\end{funcdesc}

\begin{funcdesc}{AnchoringParser}{formatter\, stylesheet}
This class, derived from \code{FormattingParser}, extends the handling
of the \code{<A>...</A>} tag pair to call the formatter's
\code{bgn_anchor()} and \code{end_anchor()} methods.  This allows the
formatter to display the anchor in a different font or color, etc.
\end{funcdesc}

Instances of \code{CollectingParser} (and thus also instances of
\code{FormattingParser} and \code{AnchoringParser}) have the following
instance variables:

\begin{datadesc}{anchornames}
A list of the values if the \code{NAME} attributes of the \code{<A>}
tags encountered.
\end{datadesc}

\begin{datadesc}{anchors}
A list of the values of \code{HREF} attributes of the \code{<A>} tags
encountered.
\end{datadesc}

\begin{datadesc}{anchortypes}
A list of the values if the \code{TYPE} attributes of the \code{<A>}
tags encountered.
\end{datadesc}

\begin{datadesc}{inanchor}
Outside an \code{<A>...</A>} tag pair, this is zero.  inside such a
pair, it is a unique integer, which is positive if the anchor has a
\code{HREF} attribute, negative if it hasn't.  Its absolute value is
one more than the index of the anchor in the \code{anchors},
\code{anchornames} and \code{anchortypes} lists.
\end{datadesc}

\begin{datadesc}{isindex}
True if the \code{<ISINDEX>} tag has been encountered.
\end{datadesc}

\begin{datadesc}{nextid}
The attribute list of the last \code{<NEXTID>} tag encountered, or
an empty list if none.
\end{datadesc}

\begin{datadesc}{title}
The text inside the last \code{<TITLE>...</TITLE>} tag pair, or
\code{''} if no title has been encountered yet.
\end{datadesc}

The \code{anchors}, \code{anchornames} and \code{anchortypes} lists
are ``parallel arrays'': items in these lists with the same index
pertain to the same anchor.  Missing attributes default to the empty
string.  Anchors with neither a \code{HREF} not a \code{NAME}
attribute are not entered in these lists at all.

The module also defines a number of style sheet classes.  These should
never be instantiated --- their class variables are the only behavior
required.  Note that style sheets are specifically designed for a
particular formatter implementation.  The currently defined style
sheets are:
\index{style sheet}

\begin{datadesc}{NullStylesheet}
A style sheet for use on a dumb output device such as an ASCII
terminal.
\end{datadesc}

\begin{datadesc}{X11Stylesheet}
A style sheet for use with an X11 server.
\end{datadesc}

\begin{datadesc}{MacStylesheet}
A style sheet for use on Apple Macintosh computers.
\end{datadesc}

\begin{datadesc}{StdwinStylesheet}
A style sheet for use with the \code{stdwin} module; it is an alias
for either \code{X11Stylesheet} or \code{MacStylesheet}.
\bimodindex{stdwin}
\end{datadesc}

\begin{datadesc}{GLStylesheet}
A style sheet for use with the SGI Graphics Library and its font
manager (the SGI-specific built-in modules \code{gl} and \code{fm}).
\bimodindex{gl}
\bimodindex{fm}
\end{datadesc}

Style sheets have the following class variables:

\begin{datadesc}{stdfontset}
A list of up to four font definititions, respectively for the roman,
italic, bold and constant-width variant of a font for normal text.  If
the list contains less than four font definitions, the last item is
used as the default for missing items.  The type of a font definition
depends on the formatter in use; its only use is as a parameter to the
formatter's \code{setfont()} method.
\end{datadesc}

\begin{datadesc}{h1fontset}
\dataline{h2fontset}
\dataline{h3fontset}
The font set used for various headers (text inside \code{<H1>...</H1>}
tag pairs etc.).
\end{datadesc}

\begin{datadesc}{stdindent}
The indentation of normal text.  This is measured in the ``native''
units of the formatter in use; for some formatters these are
characters, for others (especially those that actually support
variable-spacing fonts) in pixels or printer points.
\end{datadesc}

\begin{datadesc}{ddindent}
The indentation used for the first level of \code{<DD>} tags.
\end{datadesc}

\begin{datadesc}{ulindent}
The indentation used for the first level of \code{<UL>} tags.
\end{datadesc}

\begin{datadesc}{h1indent}
The indentation used for level 1 headers.
\end{datadesc}

\begin{datadesc}{h2indent}
The indentation used for level 2 headers.
\end{datadesc}

\begin{datadesc}{literalindent}
The indentation used for literal text (text inside
\code{<PRE>...</PRE>} and similar tag pairs).
\end{datadesc}

Although no documented implementation of a formatter exists, the
\code{FormattingParser} class assumes that formatters have a
certain interface.  This interface requires the following methods:
\index{formatter}

\begin{funcdesc}{setfont}{fontspec}
Set the font to be used subsequently.  The \var{fontspec} argument is
an item in a style sheet's font set.
\end{funcdesc}

\begin{funcdesc}{flush}{}
Finish the current line, if not empty, and begin a new one.
\end{funcdesc}

\begin{funcdesc}{setleftindent}{n}
Set the left indentation of the following lines to \var{n} units.
\end{funcdesc}

\begin{funcdesc}{needvspace}{n}
Require at least \var{n} blank lines before the next line.  Implies
\code{flush()}.
\end{funcdesc}

\begin{funcdesc}{addword}{word\, space}
Add a var{word} to the current paragraph, followed by \var{space}
spaces.
\end{funcdesc}

\begin{datadesc}{nospace}
If this instance variable is true, empty words are ignored by
\code{addword}.  It is set to false after a non-empty word has been
added.
\end{datadesc}

\begin{funcdesc}{setjust}{justification}
Set the justification of the current paragraph.  The
\var{justification} can be \code{'c'} (center), \code{'l'} (left
justified), \code{'r'} (right justified) or \code{'lr'} (left and
right justified).
\end{funcdesc}

\begin{funcdesc}{bgn_anchor}{id}
Begin an anchor.  The \var{id} parameter is the value of the parser's
\code{inanchor} attribute.
\end{funcdesc}

\begin{funcdesc}{end_anchor}{id}
End an anchor.  The \var{id} parameter is the value of the parser's
\code{inanchor} attribute.
\end{funcdesc}

A sample formatters implementation can be found in the module
\code{fmt}, which in turn uses the module \code{Para}.  These are
currently not intended as a 
\ttindex{fmt}
\ttindex{Para}

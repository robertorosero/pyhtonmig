\section{\module{htmllib} ---
         A parser for HTML documents}

\declaremodule{standard}{htmllib}
\modulesynopsis{A parser for HTML documents.}

\index{HTML}
\index{hypertext}


This module defines a class which can serve as a base for parsing text
files formatted in the HyperText Mark-up Language (HTML).  The class
is not directly concerned with I/O --- it must be provided with input
in string form via a method, and makes calls to methods of a
``formatter'' object in order to produce output.  The
\class{HTMLParser} class is designed to be used as a base class for
other classes in order to add functionality, and allows most of its
methods to be extended or overridden.  In turn, this class is derived
from and extends the \class{SGMLParser} class defined in module
\refmodule{sgmllib}\refstmodindex{sgmllib}.  The \class{HTMLParser}
implementation supports the HTML 2.0 language as described in
\rfc{1866}.  Two implementations of formatter objects are provided in
the \refmodule{formatter}\refstmodindex{formatter}\ module; refer to the
documentation for that module for information on the formatter
interface.
\withsubitem{(in module sgmllib)}{\ttindex{SGMLParser}}

The following is a summary of the interface defined by
\class{sgmllib.SGMLParser}:

\begin{itemize}

\item
The interface to feed data to an instance is through the \method{feed()}
method, which takes a string argument.  This can be called with as
little or as much text at a time as desired; \samp{p.feed(a);
p.feed(b)} has the same effect as \samp{p.feed(a+b)}.  When the data
contains complete HTML tags, these are processed immediately;
incomplete elements are saved in a buffer.  To force processing of all
unprocessed data, call the \method{close()} method.

For example, to parse the entire contents of a file, use:
\begin{verbatim}
parser.feed(open('myfile.html').read())
parser.close()
\end{verbatim}

\item
The interface to define semantics for HTML tags is very simple: derive
a class and define methods called \method{start_\var{tag}()},
\method{end_\var{tag}()}, or \method{do_\var{tag}()}.  The parser will
call these at appropriate moments: \method{start_\var{tag}} or
\method{do_\var{tag}()} is called when an opening tag of the form
\code{<\var{tag} ...>} is encountered; \method{end_\var{tag}()} is called
when a closing tag of the form \code{<\var{tag}>} is encountered.  If
an opening tag requires a corresponding closing tag, like \code{<H1>}
... \code{</H1>}, the class should define the \method{start_\var{tag}()}
method; if a tag requires no closing tag, like \code{<P>}, the class
should define the \method{do_\var{tag}()} method.

\end{itemize}

The module defines a single class:

\begin{classdesc}{HTMLParser}{formatter}
This is the basic HTML parser class.  It supports all entity names
required by the HTML 2.0 specification (\rfc{1866}).  It also defines
handlers for all HTML 2.0 and many HTML 3.0 and 3.2 elements.
\end{classdesc}


\begin{seealso}
  \seemodule{formatter}{Interface definition for transforming an
                        abstract flow of formatting events into
                        specific output events on writer objects.}
  \seemodule{HTMLParser}{Alternate HTML parser that offers a slightly
                         lower-level view of the input, but is
                         designed to work with XHTML, and does not
                         implement some of the SGML syntax not used in
                         ``HTML as deployed'' and which isn't legal
                         for XHTML.}
  \seemodule{htmlentitydefs}{Definition of replacement text for HTML
                             2.0 entities.}
  \seemodule{sgmllib}{Base class for \class{HTMLParser}.}
\end{seealso}


\subsection{HTMLParser Objects \label{html-parser-objects}}

In addition to tag methods, the \class{HTMLParser} class provides some
additional methods and instance variables for use within tag methods.

\begin{memberdesc}{formatter}
This is the formatter instance associated with the parser.
\end{memberdesc}

\begin{memberdesc}{nofill}
Boolean flag which should be true when whitespace should not be
collapsed, or false when it should be.  In general, this should only
be true when character data is to be treated as ``preformatted'' text,
as within a \code{<PRE>} element.  The default value is false.  This
affects the operation of \method{handle_data()} and \method{save_end()}.
\end{memberdesc}


\begin{methoddesc}{anchor_bgn}{href, name, type}
This method is called at the start of an anchor region.  The arguments
correspond to the attributes of the \code{<A>} tag with the same
names.  The default implementation maintains a list of hyperlinks
(defined by the \code{HREF} attribute for \code{<A>} tags) within the
document.  The list of hyperlinks is available as the data attribute
\member{anchorlist}.
\end{methoddesc}

\begin{methoddesc}{anchor_end}{}
This method is called at the end of an anchor region.  The default
implementation adds a textual footnote marker using an index into the
list of hyperlinks created by \method{anchor_bgn()}.
\end{methoddesc}

\begin{methoddesc}{handle_image}{source, alt\optional{, ismap\optional{, align\optional{, width\optional{, height}}}}}
This method is called to handle images.  The default implementation
simply passes the \var{alt} value to the \method{handle_data()}
method.
\end{methoddesc}

\begin{methoddesc}{save_bgn}{}
Begins saving character data in a buffer instead of sending it to the
formatter object.  Retrieve the stored data via \method{save_end()}.
Use of the \method{save_bgn()} / \method{save_end()} pair may not be
nested.
\end{methoddesc}

\begin{methoddesc}{save_end}{}
Ends buffering character data and returns all data saved since the
preceding call to \method{save_bgn()}.  If the \member{nofill} flag is
false, whitespace is collapsed to single spaces.  A call to this
method without a preceding call to \method{save_bgn()} will raise a
\exception{TypeError} exception.
\end{methoddesc}



\section{\module{htmlentitydefs} ---
         Definitions of HTML general entities}

\declaremodule{standard}{htmlentitydefs}
\modulesynopsis{Definitions of HTML general entities.}
\sectionauthor{Fred L. Drake, Jr.}{fdrake@acm.org}

This module defines three dictionaries, \code{name2codepoint},
\code{codepoint2name}, and \code{entitydefs}. \code{entitydefs} is
used by the \refmodule{htmllib} module to provide the
\member{entitydefs} member of the \class{HTMLParser} class.  The
definition provided here contains all the entities defined by XHTML 1.0 
that can be handled using simple textual substitution in the Latin-1
character set (ISO-8859-1).


\begin{datadesc}{entitydefs}
  A dictionary mapping XHTML 1.0 entity definitions to their
  replacement text in ISO Latin-1.

\end{datadesc}

\begin{datadesc}{name2codepoint}
  A dictionary that maps HTML entity names to the Unicode codepoints.
  \versionadded{2.3}
\end{datadesc}

\begin{datadesc}{codepoint2name}
  A dictionary that maps Unicode codepoints to HTML entity names.
  \versionadded{2.3}
\end{datadesc}

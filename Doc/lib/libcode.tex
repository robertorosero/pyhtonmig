\section{\module{code} ---
         Code object services.}
\declaremodule{standard}{code}

\modulesynopsis{Code object services.}


The \code{code} module defines operations pertaining to Python code
objects.

The \code{code} module defines the following functions:


\begin{funcdesc}{compile_command}{source, \optional{filename\optional{, symbol}}}
This function is useful for programs that want to emulate Python's
interpreter main loop (a.k.a. the read-eval-print loop).  The tricky
part is to determine when the user has entered an incomplete command
that can be completed by entering more text (as opposed to a complete
command or a syntax error).  This function \emph{almost} always makes
the same decision as the real interpreter main loop.

Arguments: \var{source} is the source string; \var{filename} is the
optional filename from which source was read, defaulting to
\code{"<input>"}; and \var{symbol} is the optional grammar start
symbol, which should be either \code{"single"} (the default) or
\code{"eval"}.

Return a code object (the same as \code{compile(\var{source},
\var{filename}, \var{symbol})}) if the command is complete and valid;
return \code{None} if the command is incomplete; raise
\code{SyntaxError} if the command is a syntax error.


\end{funcdesc}

\section{\module{codecs} ---
         Codec registry and base classes}

\declaremodule{standard}{codecs}
\modulesynopsis{Encode and decode data and streams.}
\moduleauthor{Marc-Andre Lemburg}{mal@lemburg.com}
\sectionauthor{Marc-Andre Lemburg}{mal@lemburg.com}
\sectionauthor{Martin v. L\"owis}{martin@v.loewis.de}

\index{Unicode}
\index{Codecs}
\indexii{Codecs}{encode}
\indexii{Codecs}{decode}
\index{streams}
\indexii{stackable}{streams}


This module defines base classes for standard Python codecs (encoders
and decoders) and provides access to the internal Python codec
registry which manages the codec and error handling lookup process.

It defines the following functions:

\begin{funcdesc}{register}{search_function}
Register a codec search function. Search functions are expected to
take one argument, the encoding name in all lower case letters, and
return a \class{CodecInfo} object having the following attributes:

\begin{itemize}
  \item \code{name} The name of the encoding;
  \item \code{encoder} The stateless encoding function;
  \item \code{decoder} The stateless decoding function;
  \item \code{incrementalencoder} An incremental encoder class or factory function;
  \item \code{incrementaldecoder} An incremental decoder class or factory function;
  \item \code{streamwriter} A stream writer class or factory function;
  \item \code{streamreader} A stream reader class or factory function.
\end{itemize}

The various functions or classes take the following arguments:

  \var{encoder} and \var{decoder}: These must be functions or methods
  which have the same interface as the
  \method{encode()}/\method{decode()} methods of Codec instances (see
  Codec Interface). The functions/methods are expected to work in a
  stateless mode.

  \var{incrementalencoder} and \var{incrementalencoder}: These have to be
  factory functions providing the following interface:

        \code{factory(\var{errors}='strict')}

  The factory functions must return objects providing the interfaces
  defined by the base classes \class{IncrementalEncoder} and
  \class{IncrementalEncoder}, respectively. Incremental codecs can maintain
  state.

  \var{streamreader} and \var{streamwriter}: These have to be
  factory functions providing the following interface:

        \code{factory(\var{stream}, \var{errors}='strict')}

  The factory functions must return objects providing the interfaces
  defined by the base classes \class{StreamWriter} and
  \class{StreamReader}, respectively. Stream codecs can maintain
  state.

  Possible values for errors are \code{'strict'} (raise an exception
  in case of an encoding error), \code{'replace'} (replace malformed
  data with a suitable replacement marker, such as \character{?}),
  \code{'ignore'} (ignore malformed data and continue without further
  notice), \code{'xmlcharrefreplace'} (replace with the appropriate XML
  character reference (for encoding only)) and \code{'backslashreplace'}
  (replace with backslashed escape sequences (for encoding only)) as
  well as any other error handling name defined via
  \function{register_error()}.

In case a search function cannot find a given encoding, it should
return \code{None}.
\end{funcdesc}

\begin{funcdesc}{lookup}{encoding}
Looks up the codec info in the Python codec registry and returns a
\class{CodecInfo} object as defined above.

Encodings are first looked up in the registry's cache. If not found,
the list of registered search functions is scanned. If no \class{CodecInfo}
object is found, a \exception{LookupError} is raised. Otherwise, the
\class{CodecInfo} object is stored in the cache and returned to the caller.
\end{funcdesc}

To simplify access to the various codecs, the module provides these
additional functions which use \function{lookup()} for the codec
lookup:

\begin{funcdesc}{getencoder}{encoding}
Look up the codec for the given encoding and return its encoder
function.

Raises a \exception{LookupError} in case the encoding cannot be found.
\end{funcdesc}

\begin{funcdesc}{getdecoder}{encoding}
Look up the codec for the given encoding and return its decoder
function.

Raises a \exception{LookupError} in case the encoding cannot be found.
\end{funcdesc}

\begin{funcdesc}{getincrementalencoder}{encoding}
Look up the codec for the given encoding and return its incremental encoder
class or factory function.

Raises a \exception{LookupError} in case the encoding cannot be found or the
codec doesn't support an incremental encoder.
\versionadded{2.5}
\end{funcdesc}

\begin{funcdesc}{getincrementaldecoder}{encoding}
Look up the codec for the given encoding and return its incremental decoder
class or factory function.

Raises a \exception{LookupError} in case the encoding cannot be found or the
codec doesn't support an incremental decoder.
\versionadded{2.5}
\end{funcdesc}

\begin{funcdesc}{getreader}{encoding}
Look up the codec for the given encoding and return its StreamReader
class or factory function.

Raises a \exception{LookupError} in case the encoding cannot be found.
\end{funcdesc}

\begin{funcdesc}{getwriter}{encoding}
Look up the codec for the given encoding and return its StreamWriter
class or factory function.

Raises a \exception{LookupError} in case the encoding cannot be found.
\end{funcdesc}

\begin{funcdesc}{register_error}{name, error_handler}
Register the error handling function \var{error_handler} under the
name \var{name}. \var{error_handler} will be called during encoding
and decoding in case of an error, when \var{name} is specified as the
errors parameter.

For encoding \var{error_handler} will be called with a
\exception{UnicodeEncodeError} instance, which contains information about
the location of the error. The error handler must either raise this or
a different exception or return a tuple with a replacement for the
unencodable part of the input and a position where encoding should
continue. The encoder will encode the replacement and continue encoding
the original input at the specified position. Negative position values
will be treated as being relative to the end of the input string. If the
resulting position is out of bound an \exception{IndexError} will be raised.

Decoding and translating works similar, except \exception{UnicodeDecodeError}
or \exception{UnicodeTranslateError} will be passed to the handler and
that the replacement from the error handler will be put into the output
directly.
\end{funcdesc}

\begin{funcdesc}{lookup_error}{name}
Return the error handler previously registered under the name \var{name}.

Raises a \exception{LookupError} in case the handler cannot be found.
\end{funcdesc}

\begin{funcdesc}{strict_errors}{exception}
Implements the \code{strict} error handling.
\end{funcdesc}

\begin{funcdesc}{replace_errors}{exception}
Implements the \code{replace} error handling.
\end{funcdesc}

\begin{funcdesc}{ignore_errors}{exception}
Implements the \code{ignore} error handling.
\end{funcdesc}

\begin{funcdesc}{xmlcharrefreplace_errors_errors}{exception}
Implements the \code{xmlcharrefreplace} error handling.
\end{funcdesc}

\begin{funcdesc}{backslashreplace_errors_errors}{exception}
Implements the \code{backslashreplace} error handling.
\end{funcdesc}

To simplify working with encoded files or stream, the module
also defines these utility functions:

\begin{funcdesc}{open}{filename, mode\optional{, encoding\optional{,
                       errors\optional{, buffering}}}}
Open an encoded file using the given \var{mode} and return
a wrapped version providing transparent encoding/decoding.

\note{The wrapped version will only accept the object format
defined by the codecs, i.e.\ Unicode objects for most built-in
codecs.  Output is also codec-dependent and will usually be Unicode as
well.}

\var{encoding} specifies the encoding which is to be used for the
file.

\var{errors} may be given to define the error handling. It defaults
to \code{'strict'} which causes a \exception{ValueError} to be raised
in case an encoding error occurs.

\var{buffering} has the same meaning as for the built-in
\function{open()} function.  It defaults to line buffered.
\end{funcdesc}

\begin{funcdesc}{EncodedFile}{file, input\optional{,
                              output\optional{, errors}}}
Return a wrapped version of file which provides transparent
encoding translation.

Strings written to the wrapped file are interpreted according to the
given \var{input} encoding and then written to the original file as
strings using the \var{output} encoding. The intermediate encoding will
usually be Unicode but depends on the specified codecs.

If \var{output} is not given, it defaults to \var{input}.

\var{errors} may be given to define the error handling. It defaults to
\code{'strict'}, which causes \exception{ValueError} to be raised in case
an encoding error occurs.
\end{funcdesc}

\begin{funcdesc}{iterencode}{iterable, encoding\optional{, errors}}
Uses an incremental encoder to iteratively encode the input provided by
\var{iterable}. This function is a generator. \var{errors} (as well as
any other keyword argument) is passed through to the incremental encoder.
\versionadded{2.5}
\end{funcdesc}

\begin{funcdesc}{iterdecode}{iterable, encoding\optional{, errors}}
Uses an incremental decoder to iteratively decode the input provided by
\var{iterable}. This function is a generator. \var{errors} (as well as
any other keyword argument) is passed through to the incremental decoder.
\versionadded{2.5}
\end{funcdesc}

The module also provides the following constants which are useful
for reading and writing to platform dependent files:

\begin{datadesc}{BOM}
\dataline{BOM_BE}
\dataline{BOM_LE}
\dataline{BOM_UTF8}
\dataline{BOM_UTF16}
\dataline{BOM_UTF16_BE}
\dataline{BOM_UTF16_LE}
\dataline{BOM_UTF32}
\dataline{BOM_UTF32_BE}
\dataline{BOM_UTF32_LE}
These constants define various encodings of the Unicode byte order mark
(BOM) used in UTF-16 and UTF-32 data streams to indicate the byte order
used in the stream or file and in UTF-8 as a Unicode signature.
\constant{BOM_UTF16} is either \constant{BOM_UTF16_BE} or
\constant{BOM_UTF16_LE} depending on the platform's native byte order,
\constant{BOM} is an alias for \constant{BOM_UTF16}, \constant{BOM_LE}
for \constant{BOM_UTF16_LE} and \constant{BOM_BE} for \constant{BOM_UTF16_BE}.
The others represent the BOM in UTF-8 and UTF-32 encodings.
\end{datadesc}


\subsection{Codec Base Classes \label{codec-base-classes}}

The \module{codecs} module defines a set of base classes which define the
interface and can also be used to easily write you own codecs for use
in Python.

Each codec has to define four interfaces to make it usable as codec in
Python: stateless encoder, stateless decoder, stream reader and stream
writer. The stream reader and writers typically reuse the stateless
encoder/decoder to implement the file protocols.

The \class{Codec} class defines the interface for stateless
encoders/decoders.

To simplify and standardize error handling, the \method{encode()} and
\method{decode()} methods may implement different error handling
schemes by providing the \var{errors} string argument.  The following
string values are defined and implemented by all standard Python
codecs:

\begin{tableii}{l|l}{code}{Value}{Meaning}
  \lineii{'strict'}{Raise \exception{UnicodeError} (or a subclass);
                    this is the default.}
  \lineii{'ignore'}{Ignore the character and continue with the next.}
  \lineii{'replace'}{Replace with a suitable replacement character;
                     Python will use the official U+FFFD REPLACEMENT
                     CHARACTER for the built-in Unicode codecs on
                     decoding and '?' on encoding.}
  \lineii{'xmlcharrefreplace'}{Replace with the appropriate XML
                     character reference (only for encoding).}
  \lineii{'backslashreplace'}{Replace with backslashed escape sequences
                     (only for encoding).}
\end{tableii}

The set of allowed values can be extended via \method{register_error}.


\subsubsection{Codec Objects \label{codec-objects}}

The \class{Codec} class defines these methods which also define the
function interfaces of the stateless encoder and decoder:

\begin{methoddesc}[Codec]{encode}{input\optional{, errors}}
  Encodes the object \var{input} and returns a tuple (output object,
  length consumed).  While codecs are not restricted to use with Unicode, in
  a Unicode context, encoding converts a Unicode object to a plain string
  using a particular character set encoding (e.g., \code{cp1252} or
  \code{iso-8859-1}).

  \var{errors} defines the error handling to apply. It defaults to
  \code{'strict'} handling.

  The method may not store state in the \class{Codec} instance. Use
  \class{StreamCodec} for codecs which have to keep state in order to
  make encoding/decoding efficient.

  The encoder must be able to handle zero length input and return an
  empty object of the output object type in this situation.
\end{methoddesc}

\begin{methoddesc}[Codec]{decode}{input\optional{, errors}}
  Decodes the object \var{input} and returns a tuple (output object,
  length consumed).  In a Unicode context, decoding converts a plain string
  encoded using a particular character set encoding to a Unicode object.

  \var{input} must be an object which provides the \code{bf_getreadbuf}
  buffer slot.  Python strings, buffer objects and memory mapped files
  are examples of objects providing this slot.

  \var{errors} defines the error handling to apply. It defaults to
  \code{'strict'} handling.

  The method may not store state in the \class{Codec} instance. Use
  \class{StreamCodec} for codecs which have to keep state in order to
  make encoding/decoding efficient.

  The decoder must be able to handle zero length input and return an
  empty object of the output object type in this situation.
\end{methoddesc}

The \class{IncrementalEncoder} and \class{IncrementalDecoder} classes provide
the basic interface for incremental encoding and decoding. Encoding/decoding the
input isn't done with one call to the stateless encoder/decoder function,
but with multiple calls to the \method{encode}/\method{decode} method of the
incremental encoder/decoder. The incremental encoder/decoder keeps track of
the encoding/decoding process during method calls.

The joined output of calls to the \method{encode}/\method{decode} method is the
same as if all the single inputs were joined into one, and this input was
encoded/decoded with the stateless encoder/decoder.


\subsubsection{IncrementalEncoder Objects \label{incremental-encoder-objects}}

\versionadded{2.5}

The \class{IncrementalEncoder} class is used for encoding an input in multiple
steps. It defines the following methods which every incremental encoder must
define in order to be compatible with the Python codec registry.

\begin{classdesc}{IncrementalEncoder}{\optional{errors}}
  Constructor for an \class{IncrementalEncoder} instance.

  All incremental encoders must provide this constructor interface. They are
  free to add additional keyword arguments, but only the ones defined
  here are used by the Python codec registry.

  The \class{IncrementalEncoder} may implement different error handling
  schemes by providing the \var{errors} keyword argument. These
  parameters are predefined:

  \begin{itemize}
    \item \code{'strict'} Raise \exception{ValueError} (or a subclass);
                          this is the default.
    \item \code{'ignore'} Ignore the character and continue with the next.
    \item \code{'replace'} Replace with a suitable replacement character
    \item \code{'xmlcharrefreplace'} Replace with the appropriate XML
                     character reference
    \item \code{'backslashreplace'} Replace with backslashed escape sequences.
  \end{itemize}

  The \var{errors} argument will be assigned to an attribute of the
  same name. Assigning to this attribute makes it possible to switch
  between different error handling strategies during the lifetime
  of the \class{IncrementalEncoder} object.

  The set of allowed values for the \var{errors} argument can
  be extended with \function{register_error()}.
\end{classdesc}

\begin{methoddesc}{encode}{object\optional{, final}}
  Encodes \var{object} (taking the current state of the encoder into account)
  and returns the resulting encoded object. If this is the last call to
  \method{encode} \var{final} must be true (the default is false).
\end{methoddesc}

\begin{methoddesc}{reset}{}
  Reset the encoder to the initial state.
\end{methoddesc}

\begin{methoddesc}{getstate}{}
  Return the current state of the encoder which must be an integer.
  The implementation should make sure that \code{0} is the most common state.
  (States that are more complicated than integers can be converted into an
  integer by marshaling/pickling the state and encoding the bytes of the
  resulting string into an integer).
  \versionadded{3.0}
\end{methoddesc}

\begin{methoddesc}{setstate}{state}
  Set the state of the encoder to \var{state}. \var{state} must be an
  encoder state returned by \method{getstate}.
  \versionadded{3.0}
\end{methoddesc}


\subsubsection{IncrementalDecoder Objects \label{incremental-decoder-objects}}

The \class{IncrementalDecoder} class is used for decoding an input in multiple
steps. It defines the following methods which every incremental decoder must
define in order to be compatible with the Python codec registry.

\begin{classdesc}{IncrementalDecoder}{\optional{errors}}
  Constructor for an \class{IncrementalDecoder} instance.

  All incremental decoders must provide this constructor interface. They are
  free to add additional keyword arguments, but only the ones defined
  here are used by the Python codec registry.

  The \class{IncrementalDecoder} may implement different error handling
  schemes by providing the \var{errors} keyword argument. These
  parameters are predefined:

  \begin{itemize}
    \item \code{'strict'} Raise \exception{ValueError} (or a subclass);
                          this is the default.
    \item \code{'ignore'} Ignore the character and continue with the next.
    \item \code{'replace'} Replace with a suitable replacement character.
  \end{itemize}

  The \var{errors} argument will be assigned to an attribute of the
  same name. Assigning to this attribute makes it possible to switch
  between different error handling strategies during the lifetime
  of the \class{IncrementalEncoder} object.

  The set of allowed values for the \var{errors} argument can
  be extended with \function{register_error()}.
\end{classdesc}

\begin{methoddesc}{decode}{object\optional{, final}}
  Decodes \var{object} (taking the current state of the decoder into account)
  and returns the resulting decoded object. If this is the last call to
  \method{decode} \var{final} must be true (the default is false).
  If \var{final} is true the decoder must decode the input completely and must
  flush all buffers. If this isn't possible (e.g. because of incomplete byte
  sequences at the end of the input) it must initiate error handling just like
  in the stateless case (which might raise an exception).
\end{methoddesc}

\begin{methoddesc}{reset}{}
  Reset the decoder to the initial state.
\end{methoddesc}

\begin{methoddesc}{getstate}{}
  Return the current state of the decoder. This must be a tuple with two
  items, the first must be the buffer containing the still undecoded input.
  The second must be an integer and can be additional state info.
  (The implementation should make sure that \code{0} is the most common
  additional state info.) If this additional state info is \code{0} it must
  be possible to set the decoder to the state which has no input buffered
  and \code{0} as the additional state info, so that feeding the previously
  buffered input to the decoder returns it to the previous state without
  producing any output. (Additional state info that is more complicated
  than integers can be converted into an integer by marshaling/pickling
  the info and encoding the bytes of the resulting string into an integer.)
  \versionadded{3.0}
\end{methoddesc}

\begin{methoddesc}{setstate}{state}
  Set the state of the encoder to \var{state}. \var{state} must be a
  decoder state returned by \method{getstate}.
  \versionadded{3.0}
\end{methoddesc}


The \class{StreamWriter} and \class{StreamReader} classes provide
generic working interfaces which can be used to implement new
encoding submodules very easily. See \module{encodings.utf_8} for an
example of how this is done.


\subsubsection{StreamWriter Objects \label{stream-writer-objects}}

The \class{StreamWriter} class is a subclass of \class{Codec} and
defines the following methods which every stream writer must define in
order to be compatible with the Python codec registry.

\begin{classdesc}{StreamWriter}{stream\optional{, errors}}
  Constructor for a \class{StreamWriter} instance. 

  All stream writers must provide this constructor interface. They are
  free to add additional keyword arguments, but only the ones defined
  here are used by the Python codec registry.

  \var{stream} must be a file-like object open for writing binary
  data.

  The \class{StreamWriter} may implement different error handling
  schemes by providing the \var{errors} keyword argument. These
  parameters are predefined:

  \begin{itemize}
    \item \code{'strict'} Raise \exception{ValueError} (or a subclass);
                          this is the default.
    \item \code{'ignore'} Ignore the character and continue with the next.
    \item \code{'replace'} Replace with a suitable replacement character
    \item \code{'xmlcharrefreplace'} Replace with the appropriate XML
                     character reference
    \item \code{'backslashreplace'} Replace with backslashed escape sequences.
  \end{itemize}

  The \var{errors} argument will be assigned to an attribute of the
  same name. Assigning to this attribute makes it possible to switch
  between different error handling strategies during the lifetime
  of the \class{StreamWriter} object.

  The set of allowed values for the \var{errors} argument can
  be extended with \function{register_error()}.
\end{classdesc}

\begin{methoddesc}{write}{object}
  Writes the object's contents encoded to the stream.
\end{methoddesc}

\begin{methoddesc}{writelines}{list}
  Writes the concatenated list of strings to the stream (possibly by
  reusing the \method{write()} method).
\end{methoddesc}

\begin{methoddesc}{reset}{}
  Flushes and resets the codec buffers used for keeping state.

  Calling this method should ensure that the data on the output is put
  into a clean state that allows appending of new fresh data without
  having to rescan the whole stream to recover state.
\end{methoddesc}

In addition to the above methods, the \class{StreamWriter} must also
inherit all other methods and attributes from the underlying stream.


\subsubsection{StreamReader Objects \label{stream-reader-objects}}

The \class{StreamReader} class is a subclass of \class{Codec} and
defines the following methods which every stream reader must define in
order to be compatible with the Python codec registry.

\begin{classdesc}{StreamReader}{stream\optional{, errors}}
  Constructor for a \class{StreamReader} instance. 

  All stream readers must provide this constructor interface. They are
  free to add additional keyword arguments, but only the ones defined
  here are used by the Python codec registry.

  \var{stream} must be a file-like object open for reading (binary)
  data.

  The \class{StreamReader} may implement different error handling
  schemes by providing the \var{errors} keyword argument. These
  parameters are defined:

  \begin{itemize}
    \item \code{'strict'} Raise \exception{ValueError} (or a subclass);
                          this is the default.
    \item \code{'ignore'} Ignore the character and continue with the next.
    \item \code{'replace'} Replace with a suitable replacement character.
  \end{itemize}

  The \var{errors} argument will be assigned to an attribute of the
  same name. Assigning to this attribute makes it possible to switch
  between different error handling strategies during the lifetime
  of the \class{StreamReader} object.

  The set of allowed values for the \var{errors} argument can
  be extended with \function{register_error()}.
\end{classdesc}

\begin{methoddesc}{read}{\optional{size\optional{, chars, \optional{firstline}}}}
  Decodes data from the stream and returns the resulting object.

  \var{chars} indicates the number of characters to read from the
  stream. \function{read()} will never return more than \var{chars}
  characters, but it might return less, if there are not enough
  characters available.

  \var{size} indicates the approximate maximum number of bytes to read
  from the stream for decoding purposes. The decoder can modify this
  setting as appropriate. The default value -1 indicates to read and
  decode as much as possible.  \var{size} is intended to prevent having
  to decode huge files in one step.

  \var{firstline} indicates that it would be sufficient to only return
  the first line, if there are decoding errors on later lines.

  The method should use a greedy read strategy meaning that it should
  read as much data as is allowed within the definition of the encoding
  and the given size, e.g.  if optional encoding endings or state
  markers are available on the stream, these should be read too.

  \versionchanged[\var{chars} argument added]{2.4}
  \versionchanged[\var{firstline} argument added]{2.4.2}
\end{methoddesc}

\begin{methoddesc}{readline}{\optional{size\optional{, keepends}}}
  Read one line from the input stream and return the
  decoded data.

  \var{size}, if given, is passed as size argument to the stream's
  \method{readline()} method.

  If \var{keepends} is false line-endings will be stripped from the
  lines returned.

  \versionchanged[\var{keepends} argument added]{2.4}
\end{methoddesc}

\begin{methoddesc}{readlines}{\optional{sizehint\optional{, keepends}}}
  Read all lines available on the input stream and return them as a list
  of lines.

  Line-endings are implemented using the codec's decoder method and are
  included in the list entries if \var{keepends} is true.

  \var{sizehint}, if given, is passed as the \var{size} argument to the
  stream's \method{read()} method.
\end{methoddesc}

\begin{methoddesc}{reset}{}
  Resets the codec buffers used for keeping state.

  Note that no stream repositioning should take place.  This method is
  primarily intended to be able to recover from decoding errors.
\end{methoddesc}

In addition to the above methods, the \class{StreamReader} must also
inherit all other methods and attributes from the underlying stream.

The next two base classes are included for convenience. They are not
needed by the codec registry, but may provide useful in practice.


\subsubsection{StreamReaderWriter Objects \label{stream-reader-writer}}

The \class{StreamReaderWriter} allows wrapping streams which work in
both read and write modes.

The design is such that one can use the factory functions returned by
the \function{lookup()} function to construct the instance.

\begin{classdesc}{StreamReaderWriter}{stream, Reader, Writer, errors}
  Creates a \class{StreamReaderWriter} instance.
  \var{stream} must be a file-like object.
  \var{Reader} and \var{Writer} must be factory functions or classes
  providing the \class{StreamReader} and \class{StreamWriter} interface
  resp.
  Error handling is done in the same way as defined for the
  stream readers and writers.
\end{classdesc}

\class{StreamReaderWriter} instances define the combined interfaces of
\class{StreamReader} and \class{StreamWriter} classes. They inherit
all other methods and attributes from the underlying stream.


\subsubsection{StreamRecoder Objects \label{stream-recoder-objects}}

The \class{StreamRecoder} provide a frontend - backend view of
encoding data which is sometimes useful when dealing with different
encoding environments.

The design is such that one can use the factory functions returned by
the \function{lookup()} function to construct the instance.

\begin{classdesc}{StreamRecoder}{stream, encode, decode,
                                 Reader, Writer, errors}
  Creates a \class{StreamRecoder} instance which implements a two-way
  conversion: \var{encode} and \var{decode} work on the frontend (the
  input to \method{read()} and output of \method{write()}) while
  \var{Reader} and \var{Writer} work on the backend (reading and
  writing to the stream).

  You can use these objects to do transparent direct recodings from
  e.g.\ Latin-1 to UTF-8 and back.

  \var{stream} must be a file-like object.

  \var{encode}, \var{decode} must adhere to the \class{Codec}
  interface. \var{Reader}, \var{Writer} must be factory functions or
  classes providing objects of the \class{StreamReader} and
  \class{StreamWriter} interface respectively.

  \var{encode} and \var{decode} are needed for the frontend
  translation, \var{Reader} and \var{Writer} for the backend
  translation.  The intermediate format used is determined by the two
  sets of codecs, e.g. the Unicode codecs will use Unicode as the
  intermediate encoding.

  Error handling is done in the same way as defined for the
  stream readers and writers.
\end{classdesc}

\class{StreamRecoder} instances define the combined interfaces of
\class{StreamReader} and \class{StreamWriter} classes. They inherit
all other methods and attributes from the underlying stream.

\subsection{Encodings and Unicode\label{encodings-overview}}

Unicode strings are stored internally as sequences of codepoints (to
be precise as \ctype{Py_UNICODE} arrays). Depending on the way Python is
compiled (either via \longprogramopt{enable-unicode=ucs2} or 
\longprogramopt{enable-unicode=ucs4}, with the former being the default)
\ctype{Py_UNICODE} is either a 16-bit or
32-bit data type. Once a Unicode object is used outside of CPU and
memory, CPU endianness and how these arrays are stored as bytes become
an issue. Transforming a unicode object into a sequence of bytes is
called encoding and recreating the unicode object from the sequence of
bytes is known as decoding. There are many different methods for how this
transformation can be done (these methods are also called encodings).
The simplest method is to map the codepoints 0-255 to the bytes
\code{0x0}-\code{0xff}. This means that a unicode object that contains 
codepoints above \code{U+00FF} can't be encoded with this method (which 
is called \code{'latin-1'} or \code{'iso-8859-1'}).
\function{unicode.encode()} will raise a \exception{UnicodeEncodeError}
that looks like this: \samp{UnicodeEncodeError: 'latin-1' codec can't
encode character u'\e u1234' in position 3: ordinal not in range(256)}.

There's another group of encodings (the so called charmap encodings)
that choose a different subset of all unicode code points and how
these codepoints are mapped to the bytes \code{0x0}-\code{0xff.}
To see how this is done simply open e.g. \file{encodings/cp1252.py}
(which is an encoding that is used primarily on Windows).
There's a string constant with 256 characters that shows you which 
character is mapped to which byte value.

All of these encodings can only encode 256 of the 65536 (or 1114111)
codepoints defined in unicode. A simple and straightforward way that
can store each Unicode code point, is to store each codepoint as two
consecutive bytes. There are two possibilities: Store the bytes in big
endian or in little endian order. These two encodings are called
UTF-16-BE and UTF-16-LE respectively. Their disadvantage is that if
e.g. you use UTF-16-BE on a little endian machine you will always have
to swap bytes on encoding and decoding. UTF-16 avoids this problem:
Bytes will always be in natural endianness. When these bytes are read
by a CPU with a different endianness, then bytes have to be swapped
though. To be able to detect the endianness of a UTF-16 byte sequence,
there's the so called BOM (the "Byte Order Mark"). This is the Unicode
character \code{U+FEFF}. This character will be prepended to every UTF-16
byte sequence. The byte swapped version of this character (\code{0xFFFE}) is
an illegal character that may not appear in a Unicode text. So when
the first character in an UTF-16 byte sequence appears to be a \code{U+FFFE}
the bytes have to be swapped on decoding. Unfortunately upto Unicode
4.0 the character \code{U+FEFF} had a second purpose as a \samp{ZERO WIDTH
NO-BREAK SPACE}: A character that has no width and doesn't allow a
word to be split. It can e.g. be used to give hints to a ligature
algorithm. With Unicode 4.0 using \code{U+FEFF} as a \samp{ZERO WIDTH NO-BREAK
SPACE} has been deprecated (with \code{U+2060} (\samp{WORD JOINER}) assuming
this role). Nevertheless Unicode software still must be able to handle
\code{U+FEFF} in both roles: As a BOM it's a device to determine the storage
layout of the encoded bytes, and vanishes once the byte sequence has
been decoded into a Unicode string; as a \samp{ZERO WIDTH NO-BREAK SPACE}
it's a normal character that will be decoded like any other.

There's another encoding that is able to encoding the full range of
Unicode characters: UTF-8. UTF-8 is an 8-bit encoding, which means
there are no issues with byte order in UTF-8. Each byte in a UTF-8
byte sequence consists of two parts: Marker bits (the most significant
bits) and payload bits. The marker bits are a sequence of zero to six
1 bits followed by a 0 bit. Unicode characters are encoded like this
(with x being payload bits, which when concatenated give the Unicode
character):

\begin{tableii}{l|l}{textrm}{Range}{Encoding}
\lineii{\code{U-00000000} ... \code{U-0000007F}}{0xxxxxxx}
\lineii{\code{U-00000080} ... \code{U-000007FF}}{110xxxxx 10xxxxxx}
\lineii{\code{U-00000800} ... \code{U-0000FFFF}}{1110xxxx 10xxxxxx 10xxxxxx}
\lineii{\code{U-00010000} ... \code{U-001FFFFF}}{11110xxx 10xxxxxx 10xxxxxx 10xxxxxx}
\lineii{\code{U-00200000} ... \code{U-03FFFFFF}}{111110xx 10xxxxxx 10xxxxxx 10xxxxxx 10xxxxxx}
\lineii{\code{U-04000000} ... \code{U-7FFFFFFF}}{1111110x 10xxxxxx 10xxxxxx 10xxxxxx 10xxxxxx 10xxxxxx}
\end{tableii}

The least significant bit of the Unicode character is the rightmost x
bit.

As UTF-8 is an 8-bit encoding no BOM is required and any \code{U+FEFF}
character in the decoded Unicode string (even if it's the first
character) is treated as a \samp{ZERO WIDTH NO-BREAK SPACE}.

Without external information it's impossible to reliably determine
which encoding was used for encoding a Unicode string. Each charmap
encoding can decode any random byte sequence. However that's not
possible with UTF-8, as UTF-8 byte sequences have a structure that
doesn't allow arbitrary byte sequence. To increase the reliability
with which a UTF-8 encoding can be detected, Microsoft invented a
variant of UTF-8 (that Python 2.5 calls \code{"utf-8-sig"}) for its Notepad
program: Before any of the Unicode characters is written to the file,
a UTF-8 encoded BOM (which looks like this as a byte sequence: \code{0xef},
\code{0xbb}, \code{0xbf}) is written. As it's rather improbable that any
charmap encoded file starts with these byte values (which would e.g. map to

   LATIN SMALL LETTER I WITH DIAERESIS \\
   RIGHT-POINTING DOUBLE ANGLE QUOTATION MARK \\
   INVERTED QUESTION MARK

in iso-8859-1), this increases the probability that a utf-8-sig
encoding can be correctly guessed from the byte sequence. So here the
BOM is not used to be able to determine the byte order used for
generating the byte sequence, but as a signature that helps in
guessing the encoding. On encoding the utf-8-sig codec will write
\code{0xef}, \code{0xbb}, \code{0xbf} as the first three bytes to the file.
On decoding utf-8-sig will skip those three bytes if they appear as the
first three bytes in the file.


\subsection{Standard Encodings\label{standard-encodings}}

Python comes with a number of codecs built-in, either implemented as C
functions or with dictionaries as mapping tables. The following table
lists the codecs by name, together with a few common aliases, and the
languages for which the encoding is likely used. Neither the list of
aliases nor the list of languages is meant to be exhaustive. Notice
that spelling alternatives that only differ in case or use a hyphen
instead of an underscore are also valid aliases.

Many of the character sets support the same languages. They vary in
individual characters (e.g. whether the EURO SIGN is supported or
not), and in the assignment of characters to code positions. For the
European languages in particular, the following variants typically
exist:

\begin{itemize}
\item an ISO 8859 codeset
\item a Microsoft Windows code page, which is typically derived from
      a 8859 codeset, but replaces control characters with additional
      graphic characters
\item an IBM EBCDIC code page
\item an IBM PC code page, which is \ASCII{} compatible
\end{itemize}

\begin{longtableiii}{l|l|l}{textrm}{Codec}{Aliases}{Languages}

\lineiii{ascii}
        {646, us-ascii}
        {English}

\lineiii{big5}
        {big5-tw, csbig5}
        {Traditional Chinese}

\lineiii{big5hkscs}
        {big5-hkscs, hkscs}
        {Traditional Chinese}

\lineiii{cp037}
        {IBM037, IBM039}
        {English}

\lineiii{cp424}
        {EBCDIC-CP-HE, IBM424}
        {Hebrew}

\lineiii{cp437}
        {437, IBM437}
        {English}

\lineiii{cp500}
        {EBCDIC-CP-BE, EBCDIC-CP-CH, IBM500}
        {Western Europe}

\lineiii{cp737}
        {}
        {Greek}

\lineiii{cp775}
        {IBM775}
        {Baltic languages}

\lineiii{cp850}
        {850, IBM850}
        {Western Europe}

\lineiii{cp852}
        {852, IBM852}
        {Central and Eastern Europe}

\lineiii{cp855}
        {855, IBM855}
        {Bulgarian, Byelorussian, Macedonian, Russian, Serbian}

\lineiii{cp856}
        {}
        {Hebrew}

\lineiii{cp857}
        {857, IBM857}
        {Turkish}

\lineiii{cp860}
        {860, IBM860}
        {Portuguese}

\lineiii{cp861}
        {861, CP-IS, IBM861}
        {Icelandic}

\lineiii{cp862}
        {862, IBM862}
        {Hebrew}

\lineiii{cp863}
        {863, IBM863}
        {Canadian}

\lineiii{cp864}
        {IBM864}
        {Arabic}

\lineiii{cp865}
        {865, IBM865}
        {Danish, Norwegian}

\lineiii{cp866}
        {866, IBM866}
        {Russian}

\lineiii{cp869}
        {869, CP-GR, IBM869}
        {Greek}

\lineiii{cp874}
        {}
        {Thai}

\lineiii{cp875}
        {}
        {Greek}

\lineiii{cp932}
        {932, ms932, mskanji, ms-kanji}
        {Japanese}

\lineiii{cp949}
        {949, ms949, uhc}
        {Korean}

\lineiii{cp950}
        {950, ms950}
        {Traditional Chinese}

\lineiii{cp1006}
        {}
        {Urdu}

\lineiii{cp1026}
        {ibm1026}
        {Turkish}

\lineiii{cp1140}
        {ibm1140}
        {Western Europe}

\lineiii{cp1250}
        {windows-1250}
        {Central and Eastern Europe}

\lineiii{cp1251}
        {windows-1251}
        {Bulgarian, Byelorussian, Macedonian, Russian, Serbian}

\lineiii{cp1252}
        {windows-1252}
        {Western Europe}

\lineiii{cp1253}
        {windows-1253}
        {Greek}

\lineiii{cp1254}
        {windows-1254}
        {Turkish}

\lineiii{cp1255}
        {windows-1255}
        {Hebrew}

\lineiii{cp1256}
        {windows1256}
        {Arabic}

\lineiii{cp1257}
        {windows-1257}
        {Baltic languages}

\lineiii{cp1258}
        {windows-1258}
        {Vietnamese}

\lineiii{euc_jp}
        {eucjp, ujis, u-jis}
        {Japanese}

\lineiii{euc_jis_2004}
        {jisx0213, eucjis2004}
        {Japanese}

\lineiii{euc_jisx0213}
        {eucjisx0213}
        {Japanese}

\lineiii{euc_kr}
        {euckr, korean, ksc5601, ks_c-5601, ks_c-5601-1987, ksx1001, ks_x-1001}
        {Korean}

\lineiii{gb2312}
        {chinese, csiso58gb231280, euc-cn, euccn, eucgb2312-cn, gb2312-1980,
         gb2312-80, iso-ir-58}
        {Simplified Chinese}

\lineiii{gbk}
        {936, cp936, ms936}
        {Unified Chinese}

\lineiii{gb18030}
        {gb18030-2000}
        {Unified Chinese}

\lineiii{hz}
        {hzgb, hz-gb, hz-gb-2312}
        {Simplified Chinese}

\lineiii{iso2022_jp}
        {csiso2022jp, iso2022jp, iso-2022-jp}
        {Japanese}

\lineiii{iso2022_jp_1}
        {iso2022jp-1, iso-2022-jp-1}
        {Japanese}

\lineiii{iso2022_jp_2}
        {iso2022jp-2, iso-2022-jp-2}
        {Japanese, Korean, Simplified Chinese, Western Europe, Greek}

\lineiii{iso2022_jp_2004}
        {iso2022jp-2004, iso-2022-jp-2004}
        {Japanese}

\lineiii{iso2022_jp_3}
        {iso2022jp-3, iso-2022-jp-3}
        {Japanese}

\lineiii{iso2022_jp_ext}
        {iso2022jp-ext, iso-2022-jp-ext}
        {Japanese}

\lineiii{iso2022_kr}
        {csiso2022kr, iso2022kr, iso-2022-kr}
        {Korean}

\lineiii{latin_1}
        {iso-8859-1, iso8859-1, 8859, cp819, latin, latin1, L1}
        {West Europe}

\lineiii{iso8859_2}
        {iso-8859-2, latin2, L2}
        {Central and Eastern Europe}

\lineiii{iso8859_3}
        {iso-8859-3, latin3, L3}
        {Esperanto, Maltese}

\lineiii{iso8859_4}
        {iso-8859-4, latin4, L4}
        {Baltic languagues}

\lineiii{iso8859_5}
        {iso-8859-5, cyrillic}
        {Bulgarian, Byelorussian, Macedonian, Russian, Serbian}

\lineiii{iso8859_6}
        {iso-8859-6, arabic}
        {Arabic}

\lineiii{iso8859_7}
        {iso-8859-7, greek, greek8}
        {Greek}

\lineiii{iso8859_8}
        {iso-8859-8, hebrew}
        {Hebrew}

\lineiii{iso8859_9}
        {iso-8859-9, latin5, L5}
        {Turkish}

\lineiii{iso8859_10}
        {iso-8859-10, latin6, L6}
        {Nordic languages}

\lineiii{iso8859_13}
        {iso-8859-13}
        {Baltic languages}

\lineiii{iso8859_14}
        {iso-8859-14, latin8, L8}
        {Celtic languages}

\lineiii{iso8859_15}
        {iso-8859-15}
        {Western Europe}

\lineiii{johab}
        {cp1361, ms1361}
        {Korean}

\lineiii{koi8_r}
        {}
        {Russian}

\lineiii{koi8_u}
        {}
        {Ukrainian}

\lineiii{mac_cyrillic}
        {maccyrillic}
        {Bulgarian, Byelorussian, Macedonian, Russian, Serbian}

\lineiii{mac_greek}
        {macgreek}
        {Greek}

\lineiii{mac_iceland}
        {maciceland}
        {Icelandic}

\lineiii{mac_latin2}
        {maclatin2, maccentraleurope}
        {Central and Eastern Europe}

\lineiii{mac_roman}
        {macroman}
        {Western Europe}

\lineiii{mac_turkish}
        {macturkish}
        {Turkish}

\lineiii{ptcp154}
        {csptcp154, pt154, cp154, cyrillic-asian}
        {Kazakh}

\lineiii{shift_jis}
        {csshiftjis, shiftjis, sjis, s_jis}
        {Japanese}

\lineiii{shift_jis_2004}
        {shiftjis2004, sjis_2004, sjis2004}
        {Japanese}

\lineiii{shift_jisx0213}
        {shiftjisx0213, sjisx0213, s_jisx0213}
        {Japanese}

\lineiii{utf_16}
        {U16, utf16}
        {all languages}

\lineiii{utf_16_be}
        {UTF-16BE}
        {all languages (BMP only)}

\lineiii{utf_16_le}
        {UTF-16LE}
        {all languages (BMP only)}

\lineiii{utf_7}
        {U7, unicode-1-1-utf-7}
        {all languages}

\lineiii{utf_8}
        {U8, UTF, utf8}
        {all languages}

\lineiii{utf_8_sig}
        {}
        {all languages}

\end{longtableiii}

A number of codecs are specific to Python, so their codec names have
no meaning outside Python. Some of them don't convert from Unicode
strings to byte strings, but instead use the property of the Python
codecs machinery that any bijective function with one argument can be
considered as an encoding.

For the codecs listed below, the result in the ``encoding'' direction
is always a byte string. The result of the ``decoding'' direction is
listed as operand type in the table.

\begin{tableiv}{l|l|l|l}{textrm}{Codec}{Aliases}{Operand type}{Purpose}

\lineiv{base64_codec}
         {base64, base-64}
         {byte string}
         {Convert operand to MIME base64}

\lineiv{bz2_codec}
         {bz2}
         {byte string}
         {Compress the operand using bz2}

\lineiv{hex_codec}
         {hex}
         {byte string}
         {Convert operand to hexadecimal representation, with two
          digits per byte}

\lineiv{idna}
         {}
         {Unicode string}
         {Implements \rfc{3490},
          see also \refmodule{encodings.idna}}

\lineiv{mbcs}
         {dbcs}
         {Unicode string}
         {Windows only: Encode operand according to the ANSI codepage (CP_ACP)}

\lineiv{palmos}
         {}
         {Unicode string}
         {Encoding of PalmOS 3.5}

\lineiv{punycode}
         {}
         {Unicode string}
         {Implements \rfc{3492}}

\lineiv{quopri_codec}
         {quopri, quoted-printable, quotedprintable}
         {byte string}
         {Convert operand to MIME quoted printable}

\lineiv{raw_unicode_escape}
         {}
         {Unicode string}
         {Produce a string that is suitable as raw Unicode literal in
          Python source code}

\lineiv{rot_13}
         {rot13}
         {Unicode string}
         {Returns the Caesar-cypher encryption of the operand}

\lineiv{string_escape}
         {}
         {byte string}
         {Produce a string that is suitable as string literal in
          Python source code}

\lineiv{undefined}
         {}
         {any}
         {Raise an exception for all conversions. Can be used as the
          system encoding if no automatic coercion between byte and
          Unicode strings is desired.} 

\lineiv{unicode_escape}
         {}
         {Unicode string}
         {Produce a string that is suitable as Unicode literal in
          Python source code}

\lineiv{unicode_internal}
         {}
         {Unicode string}
         {Return the internal representation of the operand}

\lineiv{uu_codec}
         {uu}
         {byte string}
         {Convert the operand using uuencode}

\lineiv{zlib_codec}
         {zip, zlib}
         {byte string}
         {Compress the operand using gzip}

\end{tableiv}

\versionadded[The \code{idna} and \code{punycode} encodings]{2.3}

\subsection{\module{encodings.idna} ---
            Internationalized Domain Names in Applications}

\declaremodule{standard}{encodings.idna}
\modulesynopsis{Internationalized Domain Names implementation}
% XXX The next line triggers a formatting bug, so it's commented out
% until that can be fixed.
%\moduleauthor{Martin v. L\"owis}

\versionadded{2.3}

This module implements \rfc{3490} (Internationalized Domain Names in
Applications) and \rfc{3492} (Nameprep: A Stringprep Profile for
Internationalized Domain Names (IDN)). It builds upon the
\code{punycode} encoding and \refmodule{stringprep}.

These RFCs together define a protocol to support non-\ASCII{} characters
in domain names. A domain name containing non-\ASCII{} characters (such
as ``www.Alliancefran\c caise.nu'') is converted into an
\ASCII-compatible encoding (ACE, such as
``www.xn--alliancefranaise-npb.nu''). The ACE form of the domain name
is then used in all places where arbitrary characters are not allowed
by the protocol, such as DNS queries, HTTP \mailheader{Host} fields, and so
on. This conversion is carried out in the application; if possible
invisible to the user: The application should transparently convert
Unicode domain labels to IDNA on the wire, and convert back ACE labels
to Unicode before presenting them to the user.

Python supports this conversion in several ways: The \code{idna} codec
allows to convert between Unicode and the ACE. Furthermore, the
\refmodule{socket} module transparently converts Unicode host names to
ACE, so that applications need not be concerned about converting host
names themselves when they pass them to the socket module. On top of
that, modules that have host names as function parameters, such as
\refmodule{httplib} and \refmodule{ftplib}, accept Unicode host names
(\refmodule{httplib} then also transparently sends an IDNA hostname in
the \mailheader{Host} field if it sends that field at all). 

When receiving host names from the wire (such as in reverse name
lookup), no automatic conversion to Unicode is performed: Applications
wishing to present such host names to the user should decode them to
Unicode.

The module \module{encodings.idna} also implements the nameprep
procedure, which performs certain normalizations on host names, to
achieve case-insensitivity of international domain names, and to unify
similar characters. The nameprep functions can be used directly if
desired.

\begin{funcdesc}{nameprep}{label}
Return the nameprepped version of \var{label}. The implementation
currently assumes query strings, so \code{AllowUnassigned} is
true.
\end{funcdesc}

\begin{funcdesc}{ToASCII}{label}
Convert a label to \ASCII, as specified in \rfc{3490}.
\code{UseSTD3ASCIIRules} is assumed to be false.
\end{funcdesc}

\begin{funcdesc}{ToUnicode}{label}
Convert a label to Unicode, as specified in \rfc{3490}.
\end{funcdesc}

 \subsection{\module{encodings.utf_8_sig} ---
             UTF-8 codec with BOM signature}
\declaremodule{standard}{encodings.utf-8-sig}   % XXX utf_8_sig gives TeX errors
\modulesynopsis{UTF-8 codec with BOM signature}
\moduleauthor{Walter D\"orwald}{}

\versionadded{2.5}

This module implements a variant of the UTF-8 codec: On encoding a
UTF-8 encoded BOM will be prepended to the UTF-8 encoded bytes. For
the stateful encoder this is only done once (on the first write to the
byte stream).  For decoding an optional UTF-8 encoded BOM at the start
of the data will be skipped.

\section{\module{gdbm} ---
         GNU's reinterpretation of dbm}

\declaremodule{builtin}{gdbm}
  \platform{Unix}
\modulesynopsis{GNU's reinterpretation of dbm.}


This module is quite similar to the \refmodule{dbm}\refbimodindex{dbm}
module, but uses \code{gdbm} instead to provide some additional
functionality.  Please note that the file formats created by
\code{gdbm} and \code{dbm} are incompatible.

The \module{gdbm} module provides an interface to the GNU DBM
library.  \code{gdbm} objects behave like mappings
(dictionaries), except that keys and values are always strings.
Printing a \code{gdbm} object doesn't print the keys and values, and
the \method{items()} and \method{values()} methods are not supported.

The module defines the following constant and functions:

\begin{excdesc}{error}
Raised on \code{gdbm}-specific errors, such as I/O errors.
\exception{KeyError} is raised for general mapping errors like
specifying an incorrect key.
\end{excdesc}

\begin{funcdesc}{open}{filename, \optional{flag, \optional{mode}}}
Open a \code{gdbm} database and return a \code{gdbm} object.  The
\var{filename} argument is the name of the database file.

The optional \var{flag} argument can be
\code{'r'} (to open an existing database for reading only --- default),
\code{'w'} (to open an existing database for reading and writing),
\code{'c'} (which creates the database if it doesn't exist), or
\code{'n'} (which always creates a new empty database).

Appending \character{f} to the flag opens the database in fast mode;
altered data will not automatically be written to the disk after every
change.  This results in faster writes to the database, but may result
in an inconsistent database if the program crashes while the database
is still open.  Use the \method{sync()} method to force any unwritten
data to be written to the disk.

The optional \var{mode} argument is the \UNIX{} mode of the file, used
only when the database has to be created.  It defaults to octal
\code{0666}.
\end{funcdesc}

In addition to the dictionary-like methods, \code{gdbm} objects have the
following methods:

\begin{funcdesc}{firstkey}{}
It's possible to loop over every key in the database using this method 
and the \method{nextkey()} method.  The traversal is ordered by
\code{gdbm}'s internal hash values, and won't be sorted by the key
values.  This method returns the starting key.
\end{funcdesc}

\begin{funcdesc}{nextkey}{key}
Returns the key that follows \var{key} in the traversal.  The
following code prints every key in the database \code{db}, without
having to create a list in memory that contains them all:

\begin{verbatim}
k = db.firstkey()
while k != None:
    print k
    k = db.nextkey(k)
\end{verbatim}
\end{funcdesc}

\begin{funcdesc}{reorganize}{}
If you have carried out a lot of deletions and would like to shrink
the space used by the \code{gdbm} file, this routine will reorganize
the database.  \code{gdbm} will not shorten the length of a database
file except by using this reorganization; otherwise, deleted file
space will be kept and reused as new (key, value) pairs are added.
\end{funcdesc}

\begin{funcdesc}{sync}{}
When the database has been opened in fast mode, this method forces any 
unwritten data to be written to the disk.
\end{funcdesc}


\begin{seealso}
  \seemodule{anydbm}{Generic interface to \code{dbm}-style databases.}
  \seemodule{whichdb}{Utility module used to determine the type of an
                      existing database.}
\end{seealso}

\section{\module{shlex} ---
         Simple lexical analysis}

\declaremodule{standard}{shlex}
\modulesynopsis{Simple lexical analysis for \UNIX{} shell-like languages.}
\moduleauthor{Eric S. Raymond}{esr@snark.thyrsus.com}
\sectionauthor{Eric S. Raymond}{esr@snark.thyrsus.com}

\versionadded{1.5.2}

The \class{shlex} class makes it easy to write lexical analyzers for
simple syntaxes resembling that of the \UNIX{} shell.  This will often
be useful for writing minilanguages, e.g.\ in run control files for
Python applications.

\begin{classdesc}{shlex}{\optional{stream}}
A \class{shlex} instance or subclass instance is a lexical analyzer
object.  The initialization argument, if present, specifies where to
read characters from. It must be a file- or stream-like object with
\method{read()} and \method{readline()} methods.  If no argument is given,
input will be taken from \code{sys.stdin}.
\end{classdesc}


\begin{seealso}
  \seemodule{ConfigParser}{Parser for configuration files similar to the
                           Windows \file{.ini} files.}
\end{seealso}


\subsection{shlex Objects \label{shlex-objects}}

A \class{shlex} instance has the following methods:

\begin{methoddesc}{get_token}{}
Return a token.  If tokens have been stacked using
\method{push_token()}, pop a token off the stack.  Otherwise, read one
from the input stream.  If reading encounters an immediate
end-of-file, an empty string is returned. 
\end{methoddesc}

\begin{methoddesc}{push_token}{str}
Push the argument onto the token stack.
\end{methoddesc}

Instances of \class{shlex} subclasses have some public instance
variables which either control lexical analysis or can be used
for debugging:

\begin{memberdesc}{commenters}
The string of characters that are recognized as comment beginners.
All characters from the comment beginner to end of line are ignored.
Includes just \character{\#} by default.   
\end{memberdesc}

\begin{memberdesc}{wordchars}
The string of characters that will accumulate into multi-character
tokens. By default, includes all \ASCII{} alphanumerics and
underscore.
\end{memberdesc}

\begin{memberdesc}{whitespace}
Characters that will be considered whitespace and skipped.  Whitespace
bounds tokens.  By default, includes space, tab, linefeed and
carriage-return.
\end{memberdesc}

\begin{memberdesc}{quotes}
Characters that will be considered string quotes.  The token
accumulates until the same quote is encountered again (thus, different
quote types protect each other as in the shell.)  By default, includes
\ASCII{} single and double quotes.
\end{memberdesc}

Note that any character not declared to be a word character,
whitespace, or a quote will be returned as a single-character token.

Quote and comment characters are not recognized within words.  Thus,
the bare words \samp{ain't} and \samp{ain\#t} would be returned as single
tokens by the default parser.

\begin{memberdesc}{lineno}
Source line number (count of newlines seen so far plus one).
\end{memberdesc}

\begin{memberdesc}{token}
The token buffer.  It may be useful to examine this when catching
exceptions.
\end{memberdesc}

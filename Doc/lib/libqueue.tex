\section{Standard Module \sectcode{Queue}}
\stmodindex{Queue}

\label{module-Queue}

% ==== 2. ====
% Give a short overview of what the module does.
% If it is platform specific, mention this.
% Mention other important restrictions or general operating principles.
% For example:

The \code{Queue} module implements a multi-producer, multi-consumer
FIFO queue.  It is especially useful in threads programming when
information must be exchanged safely between multiple threads.  The
\code{Queue} class in this module implements all the required locking
semantics.  It depends on the availability of thread support in
Python.

The \code{Queue} module defines the following exception:

\setindexsubitem{(in module Queue)}

\begin{excdesc}{Empty}
Exception raised when non-blocking get (e.g. \code{get_nowait()}) is
called on a Queue object which is empty, or for which the emptyiness
cannot be determined (i.e. because the appropriate locks cannot be
acquired).
\end{excdesc}

\subsection{Queue Objects}

Class \code{Queue} implements queue objects and has the methods
described below.  This class can be derived from in order to implement
other queue organizations (e.g. stack) but the inheritable interface
is not described here.  See the source code for details.  The public
interface methods are:

\setindexsubitem{(__init__ method)}

\begin{funcdesc}{__init__}{maxsize}
Constructor for the class.  \var{maxsize} is an integer that sets the
upperbound limit on the number of items that can be placed in the
queue.  Insertion will block once this size has been reached, until
queue items are consumed.  If \var{maxsize} is less than or equal to
zero, the queue size is infinite.
\end{funcdesc}

\setindexsubitem{(qsize method)}

\begin{funcdesc}{qsize}{}
Returns the approximate size of the queue.  Because of multithreading
semantics, this number is not reliable.
\end{funcdesc}

\setindexsubitem{(empty method)}

\begin{funcdesc}{empty}{}
Returns 1 if the queue is empty, 0 otherwise.  Because of
multithreading semantics, this is not reliable.
\end{funcdesc}

\setindexsubitem{(full method)}

\begin{funcdesc}{full}{}
Returns 1 if the queue is full, 0 otherwise.  Because of
multithreading semantics, this is not reliable.
\end{funcdesc}

\setindexsubitem{(put method)}

\begin{funcdesc}{put}{item}
Puts \var{item} into the queue.
\end{funcdesc}

\setindexsubitem{(get method)}

\begin{funcdesc}{get}{}
Gets and returns an item from the queue, blocking if necessary until
one is available.
\end{funcdesc}

\setindexsubitem{(get_nowait method)}

\begin{funcdesc}{get_nowait}{}
Gets and returns an item from the queue if one is immediately
available.  Raises an \code{Empty} exception if the queue is empty or
if the queue's emptiness cannot be determined.
\end{funcdesc}

\section{\module{SimpleXMLRPCServer} ---
         Basic XML-RPC server}

\declaremodule{standard}{SimpleXMLRPCServer}
\modulesynopsis{Basic XML-RPC server implementation.}
\moduleauthor{Brian Quinlan}{brianq@activestate.com}
\sectionauthor{Fred L. Drake, Jr.}{fdrake@acm.org}

\versionadded{2.2}

The \module{SimpleXMLRPCServer} module provides a basic server
framework for XML-RPC servers written in Python.  Servers can either
be free standing, using \class{SimpleXMLRPCServer}, or embedded in a
CGI environment, using \class{CGIXMLRPCRequestHandler}.

\begin{classdesc}{SimpleXMLRPCServer}{addr\optional{,
                                      requestHandler\optional{,
					                  logRequests\optional{,
                                      allow_none\optional{,
                                      encoding}}}}}

  Create a new server instance.  This class
  provides methods for registration of functions that can be called by
  the XML-RPC protocol.  The \var{requestHandler} parameter
  should be a factory for request handler instances; it defaults to
  \class{SimpleXMLRPCRequestHandler}.  The \var{addr} and
  \var{requestHandler} parameters are passed to the
  \class{\refmodule{SocketServer}.TCPServer} constructor.  If
  \var{logRequests} is true (the default), requests will be logged;
  setting this parameter to false will turn off logging.  
  The \var{allow_none} and \var{encoding} parameters are passed on to 
  \module{xmlrpclib} and control the XML-RPC responses that will be returned 
  from the server. The \var{bind_and_activate} parameter controls whether
  \method{server_bind()} and \method{server_activate()} are called immediately
  by the constructor; it defaults to true. Setting it to false allows code to
  manipulate the \var{allow_reuse_address} class variable before the address
  is bound.
  \versionchanged[The \var{allow_none} and \var{encoding} parameters were added]{2.5}
  \versionchanged[The \var{bind_and_activate} parameter was added]{2.6}
\end{classdesc}

\begin{classdesc}{CGIXMLRPCRequestHandler}{\optional{allow_none\optional{, encoding}}}
  Create a new instance to handle XML-RPC requests in a CGI
  environment. 
  The \var{allow_none} and \var{encoding} parameters are passed on to 
  \module{xmlrpclib} and control the XML-RPC responses that will be returned 
  from the server.
  \versionadded{2.3}
  \versionchanged[The \var{allow_none} and \var{encoding} parameters were added]{2.5}
\end{classdesc}

\begin{classdesc}{SimpleXMLRPCRequestHandler}{}
  Create a new request handler instance.  This request handler
  supports \code{POST} requests and modifies logging so that the
  \var{logRequests} parameter to the \class{SimpleXMLRPCServer}
  constructor parameter is honored.
\end{classdesc}


\subsection{SimpleXMLRPCServer Objects \label{simple-xmlrpc-servers}}

The \class{SimpleXMLRPCServer} class is based on
\class{SocketServer.TCPServer} and provides a means of creating
simple, stand alone XML-RPC servers.

\begin{methoddesc}[SimpleXMLRPCServer]{register_function}{function\optional{,
                                                          name}}
  Register a function that can respond to XML-RPC requests.  If
  \var{name} is given, it will be the method name associated with
  \var{function}, otherwise \code{\var{function}.__name__} will be
  used.  \var{name} can be either a normal or Unicode string, and may
  contain characters not legal in Python identifiers, including the
  period character.
\end{methoddesc}

\begin{methoddesc}[SimpleXMLRPCServer]{register_instance}{instance\optional{,
                                       allow_dotted_names}}
  Register an object which is used to expose method names which have
  not been registered using \method{register_function()}.  If
  \var{instance} contains a \method{_dispatch()} method, it is called
  with the requested method name and the parameters from the request.  Its
  API is \code{def \method{_dispatch}(self, method, params)} (note that
  \var{params} does not represent a variable argument list).  If it calls an
  underlying function to perform its task, that function is called as
  \code{func(*params)}, expanding the parameter list.
  The return value from \method{_dispatch()} is returned to the client as
  the result.  If
  \var{instance} does not have a \method{_dispatch()} method, it is
  searched for an attribute matching the name of the requested method.

  If the optional \var{allow_dotted_names} argument is true and the
  instance does not have a \method{_dispatch()} method, then
  if the requested method name contains periods, each component of the
  method name is searched for individually, with the effect that a
  simple hierarchical search is performed.  The value found from this
  search is then called with the parameters from the request, and the
  return value is passed back to the client.

  \begin{notice}[warning]
  Enabling the \var{allow_dotted_names} option allows intruders to access
  your module's global variables and may allow intruders to execute
  arbitrary code on your machine.  Only use this option on a secure,
  closed network.
  \end{notice}

  \versionchanged[\var{allow_dotted_names} was added to plug a security hole;
  prior versions are insecure]{2.3.5, 2.4.1}

\end{methoddesc}

\begin{methoddesc}[SimpleXMLRPCServer]{register_introspection_functions}{}
  Registers the XML-RPC introspection functions \code{system.listMethods},
  \code{system.methodHelp} and \code{system.methodSignature}. 
  \versionadded{2.3}
\end{methoddesc}

\begin{methoddesc}[SimpleXMLRPCServer]{register_multicall_functions}{}
  Registers the XML-RPC multicall function system.multicall.
\end{methoddesc}

\begin{memberdesc}[SimpleXMLRPCServer]{rpc_paths}
An attribute value that must be a tuple listing valid path portions of
the URL for receiving XML-RPC requests.  Requests posted to other
paths will result in a 404 ``no such page'' HTTP error.  If this
tuple is empty, all paths will be considered valid.
The default value is \code{('/', '/RPC2')}.
  \versionadded{2.5}
\end{memberdesc}

Example:

\begin{verbatim}
from SimpleXMLRPCServer import SimpleXMLRPCServer

# Create server
server = SimpleXMLRPCServer(("localhost", 8000))
server.register_introspection_functions()

# Register pow() function; this will use the value of 
# pow.__name__ as the name, which is just 'pow'.
server.register_function(pow)

# Register a function under a different name
def adder_function(x,y):
    return x + y
server.register_function(adder_function, 'add')

# Register an instance; all the methods of the instance are 
# published as XML-RPC methods (in this case, just 'div').
class MyFuncs:
    def div(self, x, y): 
        return x // y
    
server.register_instance(MyFuncs())

# Run the server's main loop
server.serve_forever()
\end{verbatim}

The following client code will call the methods made available by 
the preceding server:

\begin{verbatim}
import xmlrpclib

s = xmlrpclib.Server('http://localhost:8000')
print s.pow(2,3)  # Returns 2**3 = 8
print s.add(2,3)  # Returns 5
print s.div(5,2)  # Returns 5//2 = 2

# Print list of available methods
print s.system.listMethods()
\end{verbatim}


\subsection{CGIXMLRPCRequestHandler}

The \class{CGIXMLRPCRequestHandler} class can be used to 
handle XML-RPC requests sent to Python CGI scripts.

\begin{methoddesc}[CGIXMLRPCRequestHandler]{register_function}{function\optional{, name}}
Register a function that can respond to XML-RPC requests. If 
\var{name} is given, it will be the method name associated with 
function, otherwise \var{function.__name__} will be used. \var{name}
can be either a normal or Unicode string, and may contain 
characters not legal in Python identifiers, including the period
character. 
\end{methoddesc}

\begin{methoddesc}[CGIXMLRPCRequestHandler]{register_instance}{instance}
Register an object which is used to expose method names 
which have not been registered using \method{register_function()}. If 
instance contains a \method{_dispatch()} method, it is called with the 
requested method name and the parameters from the 
request; the return value is returned to the client as the result.
If instance does not have a \method{_dispatch()} method, it is searched 
for an attribute matching the name of the requested method; if 
the requested method name contains periods, each 
component of the method name is searched for individually, 
with the effect that a simple hierarchical search is performed. 
The value found from this search is then called with the 
parameters from the request, and the return value is passed 
back to the client. 
\end{methoddesc}

\begin{methoddesc}[CGIXMLRPCRequestHandler]{register_introspection_functions}{}
Register the XML-RPC introspection functions 
\code{system.listMethods}, \code{system.methodHelp} and 
\code{system.methodSignature}.
\end{methoddesc}

\begin{methoddesc}[CGIXMLRPCRequestHandler]{register_multicall_functions}{}
Register the XML-RPC multicall function \code{system.multicall}.
\end{methoddesc}

\begin{methoddesc}[CGIXMLRPCRequestHandler]{handle_request}{\optional{request_text = None}}
Handle a XML-RPC request. If \var{request_text} is given, it 
should be the POST data provided by the HTTP server, 
otherwise the contents of stdin will be used.
\end{methoddesc}

Example:

\begin{verbatim}
class MyFuncs:
    def div(self, x, y) : return x // y


handler = CGIXMLRPCRequestHandler()
handler.register_function(pow)
handler.register_function(lambda x,y: x+y, 'add')
handler.register_introspection_functions()
handler.register_instance(MyFuncs())
handler.handle_request()
\end{verbatim}

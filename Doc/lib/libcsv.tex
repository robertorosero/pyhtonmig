\section{\module{csv} --- CSV File Reading and Writing}

\declaremodule{standard}{csv}
\modulesynopsis{Write and read tabular data to and from delimited files.}
\sectionauthor{Skip Montanaro}{skip@pobox.com}

\versionadded{2.3}
\index{csv}
\indexii{data}{tabular}

The so-called CSV (Comma Separated Values) format is the most common import
and export format for spreadsheets and databases.  There is no ``CSV
standard'', so the format is operationally defined by the many applications
which read and write it.  The lack of a standard means that subtle
differences often exist in the data produced and consumed by different
applications.  These differences can make it annoying to process CSV files
from multiple sources.  Still, while the delimiters and quoting characters
vary, the overall format is similar enough that it is possible to write a
single module which can efficiently manipulate such data, hiding the details
of reading and writing the data from the programmer.

The \module{csv} module implements classes to read and write tabular data in
CSV format.  It allows programmers to say, ``write this data in the format
preferred by Excel,'' or ``read data from this file which was generated by
Excel,'' without knowing the precise details of the CSV format used by
Excel.  Programmers can also describe the CSV formats understood by other
applications or define their own special-purpose CSV formats.

The \module{csv} module's \class{reader} and \class{writer} objects read and
write sequences.  Programmers can also read and write data in dictionary
form using the \class{DictReader} and \class{DictWriter} classes.

\begin{notice}
  This version of the \module{csv} module doesn't support Unicode
  input.  Also, there are currently some issues regarding \ASCII{} NUL
  characters.  Accordingly, all input should generally be printable
  \ASCII{} to be safe.  These restrictions will be removed in the future.
\end{notice}

\begin{seealso}
%  \seemodule{array}{Arrays of uniformly types numeric values.}
  \seepep{305}{CSV File API}
         {The Python Enhancement Proposal which proposed this addition
          to Python.}
\end{seealso}


\subsection{Module Contents \label{csv-contents}}

The \module{csv} module defines the following functions:

\begin{funcdesc}{reader}{csvfile\optional{,
                         dialect=\code{'excel'}}\optional{, fmtparam}}
Return a reader object which will iterate over lines in the given
{}\var{csvfile}.  \var{csvfile} can be any object which supports the
iterator protocol and returns a string each time its \method{next}
method is called - file objects and list objects are both suitable.  
If \var{csvfile} is a file object, it must be opened with
the 'b' flag on platforms where that makes a difference.  An optional
{}\var{dialect} parameter can be given
which is used to define a set of parameters specific to a particular CSV
dialect.  It may be an instance of a subclass of the \class{Dialect}
class or one of the strings returned by the \function{list_dialects}
function.  The other optional {}\var{fmtparam} keyword arguments can be
given to override individual formatting parameters in the current
dialect.  For more information about the dialect and formatting
parameters, see section~\ref{csv-fmt-params}, ``Dialects and Formatting
Parameters'' for details of these parameters.

All data read are returned as strings.  No automatic data type
conversion is performed.
\end{funcdesc}

\begin{funcdesc}{writer}{csvfile\optional{,
                         dialect=\code{'excel'}}\optional{, fmtparam}}
Return a writer object responsible for converting the user's data into
delimited strings on the given file-like object.  \var{csvfile} can be any
object with a \function{write} method.  If \var{csvfile} is a file object,
it must be opened with the 'b' flag on platforms where that makes a
difference.  An optional
{}\var{dialect} parameter can be given which is used to define a set of
parameters specific to a particular CSV dialect.  It may be an instance
of a subclass of the \class{Dialect} class or one of the strings
returned by the \function{list_dialects} function.  The other optional
{}\var{fmtparam} keyword arguments can be given to override individual
formatting parameters in the current dialect.  For more information
about the dialect and formatting parameters, see
section~\ref{csv-fmt-params}, ``Dialects and Formatting Parameters'' for
details of these parameters.  To make it as easy as possible to
interface with modules which implement the DB API, the value
\constant{None} is written as the empty string.  While this isn't a
reversible transformation, it makes it easier to dump SQL NULL data values
to CSV files without preprocessing the data returned from a
\code{cursor.fetch*()} call.  All other non-string data are stringified
with \function{str()} before being written.
\end{funcdesc}

\begin{funcdesc}{register_dialect}{name\optional{, dialect}\optional{, fmtparam}}
Associate \var{dialect} with \var{name}.  \var{name} must be a string
or Unicode object. The dialect can be specified either by passing a
sub-class of \class{Dialect}, or by \var{fmtparam} keyword arguments,
or both, with keyword arguments overriding parameters of the dialect.
For more information about the dialect and formatting parameters, see
section~\ref{csv-fmt-params}, ``Dialects and Formatting Parameters''
for details of these parameters.
\end{funcdesc}

\begin{funcdesc}{unregister_dialect}{name}
Delete the dialect associated with \var{name} from the dialect registry.  An
\exception{Error} is raised if \var{name} is not a registered dialect
name.
\end{funcdesc}

\begin{funcdesc}{get_dialect}{name}
Return the dialect associated with \var{name}.  An \exception{Error} is
raised if \var{name} is not a registered dialect name.
\end{funcdesc}

\begin{funcdesc}{list_dialects}{}
Return the names of all registered dialects.
\end{funcdesc}

\begin{funcdesc}{field_size_limit}{\optional{new_limit}}
  Returns the current maximum field size allowed by the parser. If
  \var{new_limit} is given, this becomes the new limit.
  \versionadded{2.5}
\end{funcdesc}


The \module{csv} module defines the following classes:

\begin{classdesc}{DictReader}{csvfile\optional{,
			      fieldnames=\constant{None},\optional{,
                              restkey=\constant{None}\optional{,
			      restval=\constant{None}\optional{,
                              dialect=\code{'excel'}\optional{,
			      *args, **kwds}}}}}}
Create an object which operates like a regular reader but maps the
information read into a dict whose keys are given by the optional
{} \var{fieldnames}
parameter.  If the \var{fieldnames} parameter is omitted, the values in
the first row of the \var{csvfile} will be used as the fieldnames.
If the row read has fewer fields than the fieldnames sequence,
the value of \var{restval} will be used as the default value.  If the row
read has more fields than the fieldnames sequence, the remaining data is
added as a sequence keyed by the value of \var{restkey}.  If the row read
has fewer fields than the fieldnames sequence, the remaining keys take the
value of the optional \var{restval} parameter.  Any other optional or
keyword arguments are passed to the underlying \class{reader} instance.
\end{classdesc}


\begin{classdesc}{DictWriter}{csvfile, fieldnames\optional{,
                              restval=""\optional{,
                              extrasaction=\code{'raise'}\optional{,
                              dialect=\code{'excel'}\optional{,
			      *args, **kwds}}}}}
Create an object which operates like a regular writer but maps dictionaries
onto output rows.  The \var{fieldnames} parameter identifies the order in
which values in the dictionary passed to the \method{writerow()} method are
written to the \var{csvfile}.  The optional \var{restval} parameter
specifies the value to be written if the dictionary is missing a key in
\var{fieldnames}.  If the dictionary passed to the \method{writerow()}
method contains a key not found in \var{fieldnames}, the optional
\var{extrasaction} parameter indicates what action to take.  If it is set
to \code{'raise'} a \exception{ValueError} is raised.  If it is set to
\code{'ignore'}, extra values in the dictionary are ignored.  Any other
optional or keyword arguments are passed to the underlying \class{writer}
instance.

Note that unlike the \class{DictReader} class, the \var{fieldnames}
parameter of the \class{DictWriter} is not optional.  Since Python's
\class{dict} objects are not ordered, there is not enough information
available to deduce the order in which the row should be written to the
\var{csvfile}.

\end{classdesc}

\begin{classdesc*}{Dialect}{}
The \class{Dialect} class is a container class relied on primarily for its
attributes, which are used to define the parameters for a specific
\class{reader} or \class{writer} instance.
\end{classdesc*}

\begin{classdesc}{excel}{}
The \class{excel} class defines the usual properties of an Excel-generated
CSV file.
\end{classdesc}

\begin{classdesc}{excel_tab}{}
The \class{excel_tab} class defines the usual properties of an
Excel-generated TAB-delimited file.
\end{classdesc}

\begin{classdesc}{Sniffer}{}
The \class{Sniffer} class is used to deduce the format of a CSV file.
\end{classdesc}

The \class{Sniffer} class provides two methods:

\begin{methoddesc}{sniff}{sample\optional{,delimiters=None}}
Analyze the given \var{sample} and return a \class{Dialect} subclass
reflecting the parameters found.  If the optional \var{delimiters} parameter
is given, it is interpreted as a string containing possible valid delimiter
characters.
\end{methoddesc}

\begin{methoddesc}{has_header}{sample}
Analyze the sample text (presumed to be in CSV format) and return
\constant{True} if the first row appears to be a series of column
headers.
\end{methoddesc}


The \module{csv} module defines the following constants:

\begin{datadesc}{QUOTE_ALL}
Instructs \class{writer} objects to quote all fields.
\end{datadesc}

\begin{datadesc}{QUOTE_MINIMAL}
Instructs \class{writer} objects to only quote those fields which contain
special characters such as \var{delimiter}, \var{quotechar} or any of the
characters in \var{lineterminator}.
\end{datadesc}

\begin{datadesc}{QUOTE_NONNUMERIC}
Instructs \class{writer} objects to quote all non-numeric
fields. 

Instructs the reader to convert all non-quoted fields to type \var{float}.
\end{datadesc}

\begin{datadesc}{QUOTE_NONE}
Instructs \class{writer} objects to never quote fields.  When the current
\var{delimiter} occurs in output data it is preceded by the current
\var{escapechar} character.  If \var{escapechar} is not set, the writer
will raise \exception{Error} if any characters that require escaping
are encountered.

Instructs \class{reader} to perform no special processing of quote characters.
\end{datadesc}


The \module{csv} module defines the following exception:

\begin{excdesc}{Error}
Raised by any of the functions when an error is detected.
\end{excdesc}


\subsection{Dialects and Formatting Parameters\label{csv-fmt-params}}

To make it easier to specify the format of input and output records,
specific formatting parameters are grouped together into dialects.  A
dialect is a subclass of the \class{Dialect} class having a set of specific
methods and a single \method{validate()} method.  When creating \class{reader}
or \class{writer} objects, the programmer can specify a string or a subclass
of the \class{Dialect} class as the dialect parameter.  In addition to, or
instead of, the \var{dialect} parameter, the programmer can also specify
individual formatting parameters, which have the same names as the
attributes defined below for the \class{Dialect} class.

Dialects support the following attributes:

\begin{memberdesc}[Dialect]{delimiter}
A one-character string used to separate fields.  It defaults to \code{','}.
\end{memberdesc}

\begin{memberdesc}[Dialect]{doublequote}
Controls how instances of \var{quotechar} appearing inside a field should
be themselves be quoted.  When \constant{True}, the character is doubled.
When \constant{False}, the \var{escapechar} is used as a prefix to the
\var{quotechar}.  It defaults to \constant{True}.

On output, if \var{doublequote} is \constant{False} and no
\var{escapechar} is set, \exception{Error} is raised if a \var{quotechar}
is found in a field.
\end{memberdesc}

\begin{memberdesc}[Dialect]{escapechar}
A one-character string used by the writer to escape the \var{delimiter} if
\var{quoting} is set to \constant{QUOTE_NONE} and the \var{quotechar}
if \var{doublequote} is \constant{False}. On reading, the \var{escapechar}
removes any special meaning from the following character. It defaults
to \constant{None}, which disables escaping.
\end{memberdesc}

\begin{memberdesc}[Dialect]{lineterminator}
The string used to terminate lines produced by the \class{writer}.
It defaults to \code{'\e r\e n'}. 

\note{The \class{reader} is hard-coded to recognise either \code{'\e r'}
or \code{'\e n'} as end-of-line, and ignores \var{lineterminator}. This
behavior may change in the future.}
\end{memberdesc}

\begin{memberdesc}[Dialect]{quotechar}
A one-character string used to quote fields containing special characters,
such as the \var{delimiter} or \var{quotechar}, or which contain new-line
characters.  It defaults to \code{'"'}.
\end{memberdesc}

\begin{memberdesc}[Dialect]{quoting}
Controls when quotes should be generated by the writer and recognised
by the reader.  It can take on any of the \constant{QUOTE_*} constants
(see section~\ref{csv-contents}) and defaults to \constant{QUOTE_MINIMAL}.
\end{memberdesc}

\begin{memberdesc}[Dialect]{skipinitialspace}
When \constant{True}, whitespace immediately following the \var{delimiter}
is ignored.  The default is \constant{False}.
\end{memberdesc}


\subsection{Reader Objects}

Reader objects (\class{DictReader} instances and objects returned by
the \function{reader()} function) have the following public methods:

\begin{methoddesc}[csv reader]{next}{}
Return the next row of the reader's iterable object as a list, parsed
according to the current dialect.
\end{methoddesc}

Reader objects have the following public attributes:

\begin{memberdesc}[csv reader]{dialect}
A read-only description of the dialect in use by the parser.
\end{memberdesc}

\begin{memberdesc}[csv reader]{line_num}
 The number of lines read from the source iterator. This is not the same
 as the number of records returned, as records can span multiple lines.
\end{memberdesc}


\subsection{Writer Objects}

\class{Writer} objects (\class{DictWriter} instances and objects returned by
the \function{writer()} function) have the following public methods.  A
{}\var{row} must be a sequence of strings or numbers for \class{Writer}
objects and a dictionary mapping fieldnames to strings or numbers (by
passing them through \function{str()} first) for {}\class{DictWriter}
objects.  Note that complex numbers are written out surrounded by parens.
This may cause some problems for other programs which read CSV files
(assuming they support complex numbers at all).

\begin{methoddesc}[csv writer]{writerow}{row}
Write the \var{row} parameter to the writer's file object, formatted
according to the current dialect.
\end{methoddesc}

\begin{methoddesc}[csv writer]{writerows}{rows}
Write all the \var{rows} parameters (a list of \var{row} objects as
described above) to the writer's file object, formatted
according to the current dialect.
\end{methoddesc}

Writer objects have the following public attribute:

\begin{memberdesc}[csv writer]{dialect}
A read-only description of the dialect in use by the writer.
\end{memberdesc}



\subsection{Examples}

The simplest example of reading a CSV file:

\begin{verbatim}
import csv
reader = csv.reader(open("some.csv", "rb"))
for row in reader:
    print row
\end{verbatim}

Reading a file with an alternate format:

\begin{verbatim}
import csv
reader = csv.reader(open("passwd", "rb"), delimiter=':', quoting=csv.QUOTE_NONE)
for row in reader:
    print row
\end{verbatim}

The corresponding simplest possible writing example is:

\begin{verbatim}
import csv
writer = csv.writer(open("some.csv", "wb"))
writer.writerows(someiterable)
\end{verbatim}

Registering a new dialect:

\begin{verbatim}
import csv

csv.register_dialect('unixpwd', delimiter=':', quoting=csv.QUOTE_NONE)

reader = csv.reader(open("passwd", "rb"), 'unixpwd')
\end{verbatim}

A slightly more advanced use of the reader - catching and reporting errors:

\begin{verbatim}
import csv, sys
filename = "some.csv"
reader = csv.reader(open(filename, "rb"))
try:
    for row in reader:
        print row
except csv.Error, e:
    sys.exit('file %s, line %d: %s' % (filename, reader.line_num, e))
\end{verbatim}

And while the module doesn't directly support parsing strings, it can
easily be done:

\begin{verbatim}
import csv
for row in csv.reader(['one,two,three']):
    print row
\end{verbatim}

The \module{csv} module doesn't directly support reading and writing
Unicode, but it is 8-bit clean save for some problems with \ASCII{} NUL
characters, so you can write classes that handle the encoding and decoding
for you as long as you avoid encodings like utf-16 that use NULs:

\begin{verbatim}
import csv

class UnicodeReader:
    def __init__(self, f, dialect=csv.excel, encoding="utf-8", **kwds):
        self.reader = csv.reader(f, dialect=dialect, **kwds)
        self.encoding = encoding

    def next(self):
        row = self.reader.next()
        return [unicode(s, self.encoding) for s in row]

    def __iter__(self):
        return self

class UnicodeWriter:
    def __init__(self, f, dialect=csv.excel, encoding="utf-8", **kwds):
        self.writer = csv.writer(f, dialect=dialect, **kwds)
        self.encoding = encoding

    def writerow(self, row):
        self.writer.writerow([s.encode(self.encoding) for s in row])

    def writerows(self, rows):
        for row in rows:
            self.writerow(row)
\end{verbatim}

They should work just like the \class{csv.reader} and \class{csv.writer}
classes but add an \var{encoding} parameter.

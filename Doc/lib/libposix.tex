\section{Built-in Module \sectcode{posix}}
\label{module-posix}
\bimodindex{posix}

This module provides access to operating system functionality that is
standardized by the C Standard and the POSIX standard (a thinly disguised
\UNIX{} interface).

\strong{Do not import this module directly.}  Instead, import the
module \code{os}, which provides a \emph{portable} version of this
interface.  On \UNIX{}, the \code{os} module provides a superset of
the \code{posix} interface.  On non-\UNIX{} operating systems the
\code{posix} module is not available, but a subset is always available
through the \code{os} interface.  Once \code{os} is imported, there is
\emph{no} performance penalty in using it instead of
\code{posix}.  In addition, \code{os} provides some additional
functionality, such as automatically calling \code{putenv()}
when an entry is \code{os.environ} is changed.
\stmodindex{os}

The descriptions below are very terse; refer to the corresponding
\UNIX{} manual (or POSIX documentation) entry for more information.
Arguments called \var{path} refer to a pathname given as a string.

Errors are reported as exceptions; the usual exceptions are given
for type errors, while errors reported by the system calls raise
\code{posix.error}, described below.

Module \code{posix} defines the following data items:

\renewcommand{\indexsubitem}{(data in module posix)}
\begin{datadesc}{environ}
A dictionary representing the string environment at the time
the interpreter was started.
For example,
\code{posix.environ['HOME']}
is the pathname of your home directory, equivalent to
\code{getenv("HOME")}
in C.

Modifying this dictionary does not affect the string environment
passed on by \code{execv()}, \code{popen()} or \code{system()}; if you
need to change the environment, pass \code{environ} to \code{execve()}
or add variable assignments and export statements to the command
string for \code{system()} or \code{popen()}.

\emph{However:} If you are using this module via the \code{os} module
(as you should -- see the introduction above), \code{environ} is a
a mapping object that behaves almost like a dictionary but invokes
\code{putenv()} automatically called whenever an item is changed.
\end{datadesc}

\renewcommand{\indexsubitem}{(exception in module posix)}
\begin{excdesc}{error}
This exception is raised when a POSIX function returns a
POSIX-related error (e.g., not for illegal argument types).  Its
string value is \code{'posix.error'}.  The accompanying value is a
pair containing the numeric error code from \code{errno} and the
corresponding string, as would be printed by the C function
\code{perror()}.
\end{excdesc}

It defines the following functions and constants:

\renewcommand{\indexsubitem}{(in module posix)}
\begin{funcdesc}{chdir}{path}
Change the current working directory to \var{path}.
\end{funcdesc}

\begin{funcdesc}{chmod}{path\, mode}
Change the mode of \var{path} to the numeric \var{mode}.
\end{funcdesc}

\begin{funcdesc}{chown}{path\, uid, gid}
Change the owner and group id of \var{path} to the numeric \var{uid}
and \var{gid}.
(Not on MS-DOS.)
\end{funcdesc}

\begin{funcdesc}{close}{fd}
Close file descriptor \var{fd}.

Note: this function is intended for low-level I/O and must be applied
to a file descriptor as returned by \code{posix.open()} or
\code{posix.pipe()}.  To close a ``file object'' returned by the
built-in function \code{open} or by \code{posix.popen} or
\code{posix.fdopen}, use its \code{close()} method.
\end{funcdesc}

\begin{funcdesc}{dup}{fd}
Return a duplicate of file descriptor \var{fd}.
\end{funcdesc}

\begin{funcdesc}{dup2}{fd\, fd2}
Duplicate file descriptor \var{fd} to \var{fd2}, closing the latter
first if necessary.  Return \code{None}.
\end{funcdesc}

\begin{funcdesc}{execv}{path\, args}
Execute the executable \var{path} with argument list \var{args},
replacing the current process (i.e., the Python interpreter).
The argument list may be a tuple or list of strings.
(Not on MS-DOS.)
\end{funcdesc}

\begin{funcdesc}{execve}{path\, args\, env}
Execute the executable \var{path} with argument list \var{args},
and environment \var{env},
replacing the current process (i.e., the Python interpreter).
The argument list may be a tuple or list of strings.
The environment must be a dictionary mapping strings to strings.
(Not on MS-DOS.)
\end{funcdesc}

\begin{funcdesc}{_exit}{n}
Exit to the system with status \var{n}, without calling cleanup
handlers, flushing stdio buffers, etc.
(Not on MS-DOS.)

Note: the standard way to exit is \code{sys.exit(\var{n})}.
\code{posix._exit()} should normally only be used in the child process
after a \code{fork()}.
\end{funcdesc}

\begin{funcdesc}{fdopen}{fd\optional{\, mode\optional{\, bufsize}}}
Return an open file object connected to the file descriptor \var{fd}.
The \var{mode} and \var{bufsize} arguments have the same meaning as
the corresponding arguments to the built-in \code{open()} function.
\end{funcdesc}

\begin{funcdesc}{fork}{}
Fork a child process.  Return 0 in the child, the child's process id
in the parent.
(Not on MS-DOS.)
\end{funcdesc}

\begin{funcdesc}{fstat}{fd}
Return status for file descriptor \var{fd}, like \code{stat()}.
\end{funcdesc}

\begin{funcdesc}{ftruncate}{fd\, length}
Truncate the file corresponding to file descriptor \var{fd}, 
so that it is at most \var{length} bytes in size.
\end{funcdesc}

\begin{funcdesc}{getcwd}{}
Return a string representing the current working directory.
\end{funcdesc}

\begin{funcdesc}{getegid}{}
Return the current process's effective group id.
(Not on MS-DOS.)
\end{funcdesc}

\begin{funcdesc}{geteuid}{}
Return the current process's effective user id.
(Not on MS-DOS.)
\end{funcdesc}

\begin{funcdesc}{getgid}{}
Return the current process's group id.
(Not on MS-DOS.)
\end{funcdesc}

\begin{funcdesc}{getpgrp}{}
Return the current process group id.
(Not on MS-DOS.)
\end{funcdesc}

\begin{funcdesc}{getpid}{}
Return the current process id.
(Not on MS-DOS.)
\end{funcdesc}

\begin{funcdesc}{getppid}{}
Return the parent's process id.
(Not on MS-DOS.)
\end{funcdesc}

\begin{funcdesc}{getuid}{}
Return the current process's user id.
(Not on MS-DOS.)
\end{funcdesc}

\begin{funcdesc}{kill}{pid\, sig}
Kill the process \var{pid} with signal \var{sig}.
(Not on MS-DOS.)
\end{funcdesc}

\begin{funcdesc}{link}{src\, dst}
Create a hard link pointing to \var{src} named \var{dst}.
(Not on MS-DOS.)
\end{funcdesc}

\begin{funcdesc}{listdir}{path}
Return a list containing the names of the entries in the directory.
The list is in arbitrary order.  It does not include the special
entries \code{'.'} and \code{'..'} even if they are present in the
directory.
\end{funcdesc}

\begin{funcdesc}{lseek}{fd\, pos\, how}
Set the current position of file descriptor \var{fd} to position
\var{pos}, modified by \var{how}: 0 to set the position relative to
the beginning of the file; 1 to set it relative to the current
position; 2 to set it relative to the end of the file.
\end{funcdesc}

\begin{funcdesc}{lstat}{path}
Like \code{stat()}, but do not follow symbolic links.  (On systems
without symbolic links, this is identical to \code{posix.stat}.)
\end{funcdesc}

\begin{funcdesc}{mkfifo}{path\optional{\, mode}}
Create a FIFO (a POSIX named pipe) named \var{path} with numeric mode
\var{mode}.  The default \var{mode} is 0666 (octal).  The current
umask value is first masked out from the mode.
(Not on MS-DOS.)

FIFOs are pipes that can be accessed like regular files.  FIFOs exist
until they are deleted (for example with \code{os.unlink}).
Generally, FIFOs are used as rendez-vous between ``client'' and
``server'' type processes: the server opens the FIFO for reading, and
the client opens it for writing.  Note that \code{mkfifo()} doesn't
open the FIFO -- it just creates the rendez-vous point.
\end{funcdesc}

\begin{funcdesc}{mkdir}{path\optional{\, mode}}
Create a directory named \var{path} with numeric mode \var{mode}.
The default \var{mode} is 0777 (octal).  On some systems, \var{mode}
is ignored.  Where it is used, the current umask value is first
masked out.
\end{funcdesc}

\begin{funcdesc}{nice}{increment}
Add \var{incr} to the process' ``niceness''.  Return the new niceness.
(Not on MS-DOS.)
\end{funcdesc}

\begin{funcdesc}{open}{file\, flags\optional{\, mode}}
Open the file \var{file} and set various flags according to
\var{flags} and possibly its mode according to \var{mode}.
The default \var{mode} is 0777 (octal), and the current umask value is
first masked out.  Return the file descriptor for the newly opened
file.

For a description of the flag and mode values, see the \UNIX{} or C
run-time documentation; flag constants (like \code{O_RDONLY} and
\code{O_WRONLY}) are defined in this module too (see below).

Note: this function is intended for low-level I/O.  For normal usage,
use the built-in function \code{open}, which returns a ``file object''
with \code{read()} and  \code{write()} methods (and many more).
\end{funcdesc}

\begin{funcdesc}{pipe}{}
Create a pipe.  Return a pair of file descriptors \code{(r, w)}
usable for reading and writing, respectively.
(Not on MS-DOS.)
\end{funcdesc}

\begin{funcdesc}{plock}{op}
Lock program segments into memory.  The value of \var{op}
(defined in \code{<sys/lock.h>}) determines which segments are locked.
(Not on MS-DOS.)
\end{funcdesc}

\begin{funcdesc}{popen}{command\optional{\, mode\optional{\, bufsize}}}
Open a pipe to or from \var{command}.  The return value is an open
file object connected to the pipe, which can be read or written
depending on whether \var{mode} is \code{'r'} (default) or \code{'w'}.
The \var{bufsize} argument has the same meaning as the corresponding
argument to the built-in \code{open()} function.  The exit status of
the command (encoded in the format specified for \code{wait()}) is
available as the return value of the \code{close()} method of the file
object.
(Not on MS-DOS.)
\end{funcdesc}

\begin{funcdesc}{putenv}{varname\, value}
Set the environment variable named \var{varname} to the string \var{value}.  
Such changes to the environment affect
subprocesses started with \code{os.system()}, \code{os.popen()} or
\code{os.fork()} and \code{os.execv()}.  (Not on all systems.)

When \code{putenv()} is
supported, assignments to items in \code{os.environ} are automatically
translated into corresponding calls to \code{os.putenv()}; however,
calls to \code{os.putenv()} don't update \code{os.environ}, so it is
actually preferable to assign to items of \code{os.environ}.  
\end{funcdesc}

\begin{funcdesc}{read}{fd\, n}
Read at most \var{n} bytes from file descriptor \var{fd}.
Return a string containing the bytes read.

Note: this function is intended for low-level I/O and must be applied
to a file descriptor as returned by \code{posix.open()} or
\code{posix.pipe()}.  To read a ``file object'' returned by the
built-in function \code{open} or by \code{posix.popen} or
\code{posix.fdopen}, or \code{sys.stdin}, use its
\code{read()} or \code{readline()} methods.
\end{funcdesc}

\begin{funcdesc}{readlink}{path}
Return a string representing the path to which the symbolic link
points.  (On systems without symbolic links, this always raises
\code{posix.error}.)
\end{funcdesc}

\begin{funcdesc}{remove}{path}
Remove the file \var{path}.  See \code{rmdir} below to remove a directory.
This is identical to the \code{unlink} function documented below.
\end{funcdesc}

\begin{funcdesc}{rename}{src\, dst}
Rename the file or directory \var{src} to \var{dst}.
\end{funcdesc}

\begin{funcdesc}{rmdir}{path}
Remove the directory \var{path}.
\end{funcdesc}

\begin{funcdesc}{setgid}{gid}
Set the current process's group id.
(Not on MS-DOS.)
\end{funcdesc}

\begin{funcdesc}{setpgrp}{}
Calls the system call \code{setpgrp()} or \code{setpgrp(0, 0)}
depending on which version is implemented (if any).  See the \UNIX{}
manual for the semantics.
(Not on MS-DOS.)
\end{funcdesc}

\begin{funcdesc}{setpgid}{pid\, pgrp}
Calls the system call \code{setpgid()}.  See the \UNIX{} manual for
the semantics.
(Not on MS-DOS.)
\end{funcdesc}

\begin{funcdesc}{setsid}{}
Calls the system call \code{setsid()}.  See the \UNIX{} manual for the
semantics.
(Not on MS-DOS.)
\end{funcdesc}

\begin{funcdesc}{setuid}{uid}
Set the current process's user id.
(Not on MS-DOS.)
\end{funcdesc}

\begin{funcdesc}{stat}{path}
Perform a {\em stat} system call on the given path.  The return value
is a tuple of at least 10 integers giving the most important (and
portable) members of the {\em stat} structure, in the order
\code{st_mode},
\code{st_ino},
\code{st_dev},
\code{st_nlink},
\code{st_uid},
\code{st_gid},
\code{st_size},
\code{st_atime},
\code{st_mtime},
\code{st_ctime}.
More items may be added at the end by some implementations.
(On MS-DOS, some items are filled with dummy values.)

Note: The standard module \code{stat} defines functions and constants
that are useful for extracting information from a stat structure.
\end{funcdesc}

\begin{funcdesc}{symlink}{src\, dst}
Create a symbolic link pointing to \var{src} named \var{dst}.  (On
systems without symbolic links, this always raises
\code{posix.error}.)
\end{funcdesc}

\begin{funcdesc}{system}{command}
Execute the command (a string) in a subshell.  This is implemented by
calling the Standard C function \code{system()}, and has the same
limitations.  Changes to \code{posix.environ}, \code{sys.stdin} etc.\ are
not reflected in the environment of the executed command.  The return
value is the exit status of the process encoded in the format
specified for \code{wait()}.
\end{funcdesc}

\begin{funcdesc}{tcgetpgrp}{fd}
Return the process group associated with the terminal given by
\var{fd} (an open file descriptor as returned by \code{posix.open()}).
(Not on MS-DOS.)
\end{funcdesc}

\begin{funcdesc}{tcsetpgrp}{fd\, pg}
Set the process group associated with the terminal given by
\var{fd} (an open file descriptor as returned by \code{posix.open()})
to \var{pg}.
(Not on MS-DOS.)
\end{funcdesc}

\begin{funcdesc}{times}{}
Return a 5-tuple of floating point numbers indicating accumulated (CPU
or other)
times, in seconds.  The items are: user time, system time, children's
user time, children's system time, and elapsed real time since a fixed
point in the past, in that order.  See the \UNIX{}
manual page {\it times}(2).  (Not on MS-DOS.)
\end{funcdesc}

\begin{funcdesc}{umask}{mask}
Set the current numeric umask and returns the previous umask.
(Not on MS-DOS.)
\end{funcdesc}

\begin{funcdesc}{uname}{}
Return a 5-tuple containing information identifying the current
operating system.  The tuple contains 5 strings:
\code{(\var{sysname}, \var{nodename}, \var{release}, \var{version}, \var{machine})}.
Some systems truncate the nodename to 8
characters or to the leading component; a better way to get the
hostname is \code{socket.gethostname()}.  (Not on MS-DOS, nor on older
\UNIX{} systems.)
\end{funcdesc}

\begin{funcdesc}{unlink}{path}
Remove the file \var{path}.  This is the same function as \code{remove};
the \code{unlink} name is its traditional \UNIX{} name.
\end{funcdesc}

\begin{funcdesc}{utime}{path\, {\rm (}atime, mtime{\rm )}}
Set the access and modified time of the file to the given values.
(The second argument is a tuple of two items.)
\end{funcdesc}

\begin{funcdesc}{wait}{}
Wait for completion of a child process, and return a tuple containing
its pid and exit status indication: a 16-bit number, whose low byte is
the signal number that killed the process, and whose high byte is the
exit status (if the signal number is zero); the high bit of the low
byte is set if a core file was produced.  (Not on MS-DOS.)
\end{funcdesc}

\begin{funcdesc}{waitpid}{pid\, options}
Wait for completion of a child process given by proces id, and return
a tuple containing its pid and exit status indication (encoded as for
\code{wait()}).  The semantics of the call are affected by the value of
the integer options, which should be 0 for normal operation.  (If the
system does not support \code{waitpid()}, this always raises
\code{posix.error}.  Not on MS-DOS.)
\end{funcdesc}

\begin{funcdesc}{write}{fd\, str}
Write the string \var{str} to file descriptor \var{fd}.
Return the number of bytes actually written.

Note: this function is intended for low-level I/O and must be applied
to a file descriptor as returned by \code{posix.open()} or
\code{posix.pipe()}.  To write a ``file object'' returned by the
built-in function \code{open} or by \code{posix.popen} or
\code{posix.fdopen}, or \code{sys.stdout} or \code{sys.stderr}, use
its \code{write()} method.
\end{funcdesc}

\begin{datadesc}{WNOHANG}
The option for \code{waitpid()} to avoid hanging if no child process
status is available immediately.
\end{datadesc}


\begin{datadesc}{O_RDONLY}
\end{datadesc}
\begin{datadesc}{O_WRONLY}
\end{datadesc}
\begin{datadesc}{O_RDWR}
\end{datadesc}
\begin{datadesc}{O_NDELAY}
\end{datadesc}
\begin{datadesc}{O_NONBLOCK}
\end{datadesc}
\begin{datadesc}{O_APPEND}
\end{datadesc}
\begin{datadesc}{O_DSYNC}
\end{datadesc}
\begin{datadesc}{O_RSYNC}
\end{datadesc}
\begin{datadesc}{O_SYNC}
\end{datadesc}
\begin{datadesc}{O_NOCTTY}
\end{datadesc}
\begin{datadesc}{O_CREAT}
\end{datadesc}
\begin{datadesc}{O_EXCL}
\end{datadesc}
\begin{datadesc}{O_TRUNC}
Options for the \code{flag} argument to the \code{open()} function.
These can be bit-wise OR'd together.
\end{datadesc}

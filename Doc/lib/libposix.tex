\section{Built-in Module \module{posix}}
\declaremodule{builtin}{posix}

\modulesynopsis{The most common \POSIX{} system calls (normally used via module
\module{os}).}


This module provides access to operating system functionality that is
standardized by the \C{} Standard and the \POSIX{} standard (a thinly
disguised \UNIX{} interface).

\strong{Do not import this module directly.}  Instead, import the
module \module{os}, which provides a \emph{portable} version of this
interface.  On \UNIX{}, the \module{os} module provides a superset of
the \module{posix} interface.  On non-\UNIX{} operating systems the
\module{posix} module is not available, but a subset is always
available through the \module{os} interface.  Once \module{os} is
imported, there is \emph{no} performance penalty in using it instead
of \module{posix}.  In addition, \module{os} provides some additional
functionality, such as automatically calling \function{putenv()}
when an entry in \code{os.environ} is changed.
\refstmodindex{os}

The descriptions below are very terse; refer to the corresponding
\UNIX{} manual (or \POSIX{} documentation) entry for more information.
Arguments called \var{path} refer to a pathname given as a string.

Errors are reported as exceptions; the usual exceptions are given
for type errors, while errors reported by the system calls raise
\exception{error}, described below.

Module \module{posix} defines the following data items:

\begin{datadesc}{environ}
A dictionary representing the string environment at the time
the interpreter was started.
For example,
\code{posix.environ['HOME']}
is the pathname of your home directory, equivalent to
\code{getenv("HOME")}
in \C{}.

Modifying this dictionary does not affect the string environment
passed on by \function{execv()}, \function{popen()} or
\function{system()}; if you need to change the environment, pass
\code{environ} to \function{execve()} or add variable assignments and
export statements to the command string for \function{system()} or
\function{popen()}.

\emph{However:} If you are using this module via the \module{os}
module (as you should -- see the introduction above), \code{environ}
is a a mapping object that behaves almost like a dictionary but
invokes \function{putenv()} automatically called whenever an item is
changed.
\end{datadesc}

\begin{excdesc}{error}
This exception is raised when a \POSIX{} function returns a
\POSIX{}-related error (e.g., not for illegal argument types).  The
accompanying value is a pair containing the numeric error code from
\cdata{errno} and the corresponding string, as would be printed by the
\C{} function \cfunction{perror()}.  See the module
\module{errno}\refbimodindex{errno}, which contains names for the
error codes defined by the underlying operating system.

When exceptions are classes, this exception carries two attributes,
\member{errno} and \member{strerror}.  The first holds the value of
the \C{} \cdata{errno} variable, and the latter holds the
corresponding error message from \cfunction{strerror()}.

When exceptions are strings, the string for the exception is
\code{'os.error'}; this reflects the more portable access to the
exception through the \module{os} module.
\end{excdesc}

It defines the following functions and constants:

\begin{funcdesc}{chdir}{path}
Change the current working directory to \var{path}.
\end{funcdesc}

\begin{funcdesc}{chmod}{path, mode}
Change the mode of \var{path} to the numeric \var{mode}.
\end{funcdesc}

\begin{funcdesc}{chown}{path, uid, gid}
Change the owner and group id of \var{path} to the numeric \var{uid}
and \var{gid}.
(Not on MS-DOS.)
\end{funcdesc}

\begin{funcdesc}{close}{fd}
Close file descriptor \var{fd}.

Note: this function is intended for low-level I/O and must be applied
to a file descriptor as returned by \function{open()} or
\function{pipe()}.  To close a ``file object'' returned by the
built-in function \function{open()} or by \function{popen()} or
\function{fdopen()}, use its \method{close()} method.
\end{funcdesc}

\begin{funcdesc}{dup}{fd}
Return a duplicate of file descriptor \var{fd}.
\end{funcdesc}

\begin{funcdesc}{dup2}{fd, fd2}
Duplicate file descriptor \var{fd} to \var{fd2}, closing the latter
first if necessary.
\end{funcdesc}

\begin{funcdesc}{execv}{path, args}
Execute the executable \var{path} with argument list \var{args},
replacing the current process (i.e., the Python interpreter).
The argument list may be a tuple or list of strings.
(Not on MS-DOS.)
\end{funcdesc}

\begin{funcdesc}{execve}{path, args, env}
Execute the executable \var{path} with argument list \var{args},
and environment \var{env},
replacing the current process (i.e., the Python interpreter).
The argument list may be a tuple or list of strings.
The environment must be a dictionary mapping strings to strings.
(Not on MS-DOS.)
\end{funcdesc}

\begin{funcdesc}{_exit}{n}
Exit to the system with status \var{n}, without calling cleanup
handlers, flushing stdio buffers, etc.
(Not on MS-DOS.)

Note: the standard way to exit is \code{sys.exit(\var{n})}.
\function{_exit()} should normally only be used in the child process
after a \function{fork()}.
\end{funcdesc}

\begin{funcdesc}{fdopen}{fd\optional{, mode\optional{, bufsize}}}
Return an open file object connected to the file descriptor \var{fd}.
The \var{mode} and \var{bufsize} arguments have the same meaning as
the corresponding arguments to the built-in \function{open()} function.
\end{funcdesc}

\begin{funcdesc}{fork}{}
Fork a child process.  Return \code{0} in the child, the child's
process id in the parent.
(Not on MS-DOS.)
\end{funcdesc}

\begin{funcdesc}{fstat}{fd}
Return status for file descriptor \var{fd}, like \function{stat()}.
\end{funcdesc}

\begin{funcdesc}{ftruncate}{fd, length}
Truncate the file corresponding to file descriptor \var{fd}, 
so that it is at most \var{length} bytes in size.
\end{funcdesc}

\begin{funcdesc}{getcwd}{}
Return a string representing the current working directory.
\end{funcdesc}

\begin{funcdesc}{getegid}{}
Return the current process' effective group id.
(Not on MS-DOS.)
\end{funcdesc}

\begin{funcdesc}{geteuid}{}
Return the current process' effective user id.
(Not on MS-DOS.)
\end{funcdesc}

\begin{funcdesc}{getgid}{}
Return the current process' group id.
(Not on MS-DOS.)
\end{funcdesc}

\begin{funcdesc}{getpgrp}{}
Return the current process group id.
(Not on MS-DOS.)
\end{funcdesc}

\begin{funcdesc}{getpid}{}
Return the current process id.
(Not on MS-DOS.)
\end{funcdesc}

\begin{funcdesc}{getppid}{}
Return the parent's process id.
(Not on MS-DOS.)
\end{funcdesc}

\begin{funcdesc}{getuid}{}
Return the current process' user id.
(Not on MS-DOS.)
\end{funcdesc}

\begin{funcdesc}{kill}{pid, sig}
Kill the process \var{pid} with signal \var{sig}.
(Not on MS-DOS.)
\end{funcdesc}

\begin{funcdesc}{link}{src, dst}
Create a hard link pointing to \var{src} named \var{dst}.
(Not on MS-DOS.)
\end{funcdesc}

\begin{funcdesc}{listdir}{path}
Return a list containing the names of the entries in the directory.
The list is in arbitrary order.  It does not include the special
entries \code{'.'} and \code{'..'} even if they are present in the
directory.
\end{funcdesc}

\begin{funcdesc}{lseek}{fd, pos, how}
Set the current position of file descriptor \var{fd} to position
\var{pos}, modified by \var{how}: \code{0} to set the position
relative to the beginning of the file; \code{1} to set it relative to
the current position; \code{2} to set it relative to the end of the
file.
\end{funcdesc}

\begin{funcdesc}{lstat}{path}
Like \function{stat()}, but do not follow symbolic links.  (On systems
without symbolic links, this is identical to \function{stat()}.)
\end{funcdesc}

\begin{funcdesc}{mkfifo}{path\optional{, mode}}
Create a FIFO (a \POSIX{} named pipe) named \var{path} with numeric mode
\var{mode}.  The default \var{mode} is \code{0666} (octal).  The current
umask value is first masked out from the mode.
(Not on MS-DOS.)

FIFOs are pipes that can be accessed like regular files.  FIFOs exist
until they are deleted (for example with \function{os.unlink()}).
Generally, FIFOs are used as rendezvous between ``client'' and
``server'' type processes: the server opens the FIFO for reading, and
the client opens it for writing.  Note that \function{mkfifo()}
doesn't open the FIFO --- it just creates the rendezvous point.
\end{funcdesc}

\begin{funcdesc}{mkdir}{path\optional{, mode}}
Create a directory named \var{path} with numeric mode \var{mode}.
The default \var{mode} is \code{0777} (octal).  On some systems,
\var{mode} is ignored.  Where it is used, the current umask value is
first masked out.
\end{funcdesc}

\begin{funcdesc}{nice}{increment}
Add \var{increment} to the process' ``niceness''.  Return the new
niceness.  (Not on MS-DOS.)
\end{funcdesc}

\begin{funcdesc}{open}{file, flags\optional{, mode}}
Open the file \var{file} and set various flags according to
\var{flags} and possibly its mode according to \var{mode}.
The default \var{mode} is \code{0777} (octal), and the current umask
value is first masked out.  Return the file descriptor for the newly
opened file.

For a description of the flag and mode values, see the \UNIX{} or \C{}
run-time documentation; flag constants (like \constant{O_RDONLY} and
\constant{O_WRONLY}) are defined in this module too (see below).

Note: this function is intended for low-level I/O.  For normal usage,
use the built-in function \function{open()}, which returns a ``file
object'' with \method{read()} and \method{write()} methods (and many
more).
\end{funcdesc}

\begin{funcdesc}{pipe}{}
Create a pipe.  Return a pair of file descriptors \code{(\var{r},
\var{w})} usable for reading and writing, respectively.
(Not on MS-DOS.)
\end{funcdesc}

\begin{funcdesc}{plock}{op}
Lock program segments into memory.  The value of \var{op}
(defined in \code{<sys/lock.h>}) determines which segments are locked.
(Not on MS-DOS.)
\end{funcdesc}

\begin{funcdesc}{popen}{command\optional{, mode\optional{, bufsize}}}
Open a pipe to or from \var{command}.  The return value is an open
file object connected to the pipe, which can be read or written
depending on whether \var{mode} is \code{'r'} (default) or \code{'w'}.
The \var{bufsize} argument has the same meaning as the corresponding
argument to the built-in \function{open()} function.  The exit status of
the command (encoded in the format specified for \function{wait()}) is
available as the return value of the \method{close()} method of the file
object.
(Not on MS-DOS.)
\end{funcdesc}

\begin{funcdesc}{putenv}{varname, value}
\index{environment variables!setting}
Set the environment variable named \var{varname} to the string
\var{value}.  Such changes to the environment affect subprocesses
started with \function{os.system()}, \function{os.popen()} or
\function{os.fork()} and \function{os.execv()}.  (Not on all systems.)

When \function{putenv()} is
supported, assignments to items in \code{os.environ} are automatically
translated into corresponding calls to \function{putenv()}; however,
calls to \function{putenv()} don't update \code{os.environ}, so it is
actually preferable to assign to items of \code{os.environ}.  
\end{funcdesc}

\begin{funcdesc}{strerror}{code}
Return the error message corresponding to the error code in \var{code}.
\end{funcdesc}

\begin{funcdesc}{read}{fd, n}
Read at most \var{n} bytes from file descriptor \var{fd}.
Return a string containing the bytes read.

Note: this function is intended for low-level I/O and must be applied
to a file descriptor as returned by \function{open()} or
\function{pipe()}.  To read a ``file object'' returned by the
built-in function \function{open()} or by \function{popen()} or
\function{fdopen()}, or \code{sys.stdin}, use its
\method{read()} or \method{readline()} methods.
\end{funcdesc}

\begin{funcdesc}{readlink}{path}
Return a string representing the path to which the symbolic link
points.  (On systems without symbolic links, this always raises
\exception{error}.)
\end{funcdesc}

\begin{funcdesc}{remove}{path}
Remove the file \var{path}.  See \function{rmdir()} below to remove a
directory.  This is identical to the \function{unlink()} function
documented below.
\end{funcdesc}

\begin{funcdesc}{rename}{src, dst}
Rename the file or directory \var{src} to \var{dst}.
\end{funcdesc}

\begin{funcdesc}{rmdir}{path}
Remove the directory \var{path}.
\end{funcdesc}

\begin{funcdesc}{setgid}{gid}
Set the current process' group id.
(Not on MS-DOS.)
\end{funcdesc}

\begin{funcdesc}{setpgrp}{}
Calls the system call \cfunction{setpgrp()} or \cfunction{setpgrp(0,
0)} depending on which version is implemented (if any).  See the
\UNIX{} manual for the semantics.
(Not on MS-DOS.)
\end{funcdesc}

\begin{funcdesc}{setpgid}{pid, pgrp}
Calls the system call \cfunction{setpgid()}.  See the \UNIX{} manual
for the semantics.
(Not on MS-DOS.)
\end{funcdesc}

\begin{funcdesc}{setsid}{}
Calls the system call \cfunction{setsid()}.  See the \UNIX{} manual
for the semantics.
(Not on MS-DOS.)
\end{funcdesc}

\begin{funcdesc}{setuid}{uid}
Set the current process' user id.
(Not on MS-DOS.)
\end{funcdesc}

\begin{funcdesc}{stat}{path}
Perform a \cfunction{stat()} system call on the given path.  The
return value is a tuple of at least 10 integers giving the most
important (and portable) members of the \emph{stat} structure, in the
order
\code{st_mode},
\code{st_ino},
\code{st_dev},
\code{st_nlink},
\code{st_uid},
\code{st_gid},
\code{st_size},
\code{st_atime},
\code{st_mtime},
\code{st_ctime}.
More items may be added at the end by some implementations.
(On MS-DOS, some items are filled with dummy values.)

Note: The standard module \module{stat}\refstmodindex{stat} defines
functions and constants that are useful for extracting information
from a stat structure.
\end{funcdesc}

\begin{funcdesc}{symlink}{src, dst}
Create a symbolic link pointing to \var{src} named \var{dst}.  (On
systems without symbolic links, this always raises \exception{error}.)
\end{funcdesc}

\begin{funcdesc}{system}{command}
Execute the command (a string) in a subshell.  This is implemented by
calling the Standard \C{} function \cfunction{system()}, and has the
same limitations.  Changes to \code{posix.environ}, \code{sys.stdin}
etc.\ are not reflected in the environment of the executed command.
The return value is the exit status of the process encoded in the
format specified for \function{wait()}.
\end{funcdesc}

\begin{funcdesc}{tcgetpgrp}{fd}
Return the process group associated with the terminal given by
\var{fd} (an open file descriptor as returned by \function{open()}).
(Not on MS-DOS.)
\end{funcdesc}

\begin{funcdesc}{tcsetpgrp}{fd, pg}
Set the process group associated with the terminal given by
\var{fd} (an open file descriptor as returned by \function{open()})
to \var{pg}.
(Not on MS-DOS.)
\end{funcdesc}

\begin{funcdesc}{times}{}
Return a 5-tuple of floating point numbers indicating accumulated (CPU
or other)
times, in seconds.  The items are: user time, system time, children's
user time, children's system time, and elapsed real time since a fixed
point in the past, in that order.  See the \UNIX{}
manual page \manpage{times}{2}.  (Not on MS-DOS.)
\end{funcdesc}

\begin{funcdesc}{umask}{mask}
Set the current numeric umask and returns the previous umask.
(Not on MS-DOS.)
\end{funcdesc}

\begin{funcdesc}{uname}{}
Return a 5-tuple containing information identifying the current
operating system.  The tuple contains 5 strings:
\code{(\var{sysname}, \var{nodename}, \var{release}, \var{version},
\var{machine})}.  Some systems truncate the nodename to 8
characters or to the leading component; a better way to get the
hostname is \function{socket.gethostname()}%
\withsubitem{(in module socket)}{\ttindex{gethostname()}}
or even
\code{socket.gethostbyaddr(socket.gethostname())}%
\withsubitem{(in module socket)}{\ttindex{gethostbyaddr()}}.
(Not on MS-DOS, nor on older \UNIX{} systems.)
\end{funcdesc}

\begin{funcdesc}{unlink}{path}
Remove the file \var{path}.  This is the same function as \code{remove};
the \code{unlink} name is its traditional \UNIX{} name.
\end{funcdesc}

\begin{funcdesc}{utime}{path, {\rm (}atime, mtime{\rm )}}
Set the access and modified time of the file to the given values.
(The second argument is a tuple of two items.)
\end{funcdesc}

\begin{funcdesc}{wait}{}
Wait for completion of a child process, and return a tuple containing
its pid and exit status indication: a 16-bit number, whose low byte is
the signal number that killed the process, and whose high byte is the
exit status (if the signal number is zero); the high bit of the low
byte is set if a core file was produced.  (Not on MS-DOS.)
\end{funcdesc}

\begin{funcdesc}{waitpid}{pid, options}
Wait for completion of a child process given by proces id, and return
a tuple containing its pid and exit status indication (encoded as for
\function{wait()}).  The semantics of the call are affected by the
value of the integer \var{options}, which should be \code{0} for
normal operation.  (If the system does not support
\function{waitpid()}, this always raises \exception{error}.  Not on
MS-DOS.)
\end{funcdesc}

\begin{funcdesc}{write}{fd, str}
Write the string \var{str} to file descriptor \var{fd}.
Return the number of bytes actually written.

Note: this function is intended for low-level I/O and must be applied
to a file descriptor as returned by \function{open()} or
\function{pipe()}.  To write a ``file object'' returned by the
built-in function \function{open()} or by \function{popen()} or
\function{fdopen()}, or \code{sys.stdout} or \code{sys.stderr}, use
its \method{write()} method.
\end{funcdesc}

\begin{datadesc}{WNOHANG}
The option for \function{waitpid()} to avoid hanging if no child
process status is available immediately.
\end{datadesc}


\begin{datadesc}{O_RDONLY}
\dataline{O_WRONLY}
\dataline{O_RDWR}
\dataline{O_NDELAY}
\dataline{O_NONBLOCK}
\dataline{O_APPEND}
\dataline{O_DSYNC}
\dataline{O_RSYNC}
\dataline{O_SYNC}
\dataline{O_NOCTTY}
\dataline{O_CREAT}
\dataline{O_EXCL}
\dataline{O_TRUNC}
Options for the \code{flag} argument to the \function{open()} function.
These can be bit-wise OR'd together.
\end{datadesc}

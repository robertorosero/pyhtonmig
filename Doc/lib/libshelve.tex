\section{Built-in module \sectcode{shelve}}
\stmodindex{shelve}
\stmodindex{pickle}
\bimodindex{dbm}

A ``shelf'' is a persistent, dictionary-like object.  The difference
with ``dbm'' databases is that the values (not the keys!) in a shelf
can be essentially arbitrary Python objects --- anything that the
\code{pickle} module can handle.  This includes most class instances,
recursive data types, and objects containing lots of shared
sub-objects.  The keys are ordinary strings.

To summarize the interface (\code{key} is a string, \code{data} is an
arbitrary object):

\begin{verbatim}
import shelve

d = shelve.open(filename) # open, with (g)dbm filename -- no suffix

d[key] = data   # store data at key (overwrites old data if
                # using an existing key)
data = d[key]   # retrieve data at key (raise KeyError if no
                # such key)
del d[key]      # delete data stored at key (raises KeyError
                # if no such key)
flag = d.has_key(key)   # true if the key exists
list = d.keys() # a list of all existing keys (slow!)

d.close()       # close it
\end{verbatim}

Restrictions:

\begin{itemize}

\item
The choice of which database package will be used (e.g. dbm or gdbm)
depends on which interface is available.  Therefore it isn't safe to
open the database directly using dbm.  The database is also
(unfortunately) subject to the limitations of dbm, if it is used ---
this means that (the pickled representation of) the objects stored in
the database should be fairly small, and in rare cases key collisions
may cause the database to refuse updates.

\item
Dependent on the implementation, closing a persistent dictionary may
or may not be necessary to flush changes to disk.

\item
The \code{shelve} module does not support {\em concurrent} access to
shelved objects.  Two programs should not try to simultaneously access
the same shelf.

\end{itemize}

\section{\module{runpy} ---
         Locating and executing Python modules.}

\declaremodule{standard}{runpy}		% standard library, in Python

\moduleauthor{Nick Coghlan}{ncoghlan@gmail.com}

\modulesynopsis{Locate and execute Python modules as scripts}

\versionadded{2.5}

The \module{runpy} module is used to locate and run Python modules
without importing them first. It's main use is to implement the
\programopt{-m} command line switch that allows scripts to be located
using the Python module namespace rather than the filesystem.

When executed as a script, the module effectively operates as follows:
\begin{verbatim}
    del sys.argv[0]  # Remove the runpy module from the arguments
    run_module(sys.argv[0], run_name="__main__", alter_sys=True)
\end{verbatim}

The \module{runpy} module provides a single function:

\begin{funcdesc}{run_module}{mod_name\optional{, init_globals}
\optional{, run_name}\optional{, alter_sys}}
Execute the code of the specified module and return the resulting
module globals dictionary. The module's code is first located using
the standard import mechanism (refer to PEP 302 for details) and
then executed in a fresh module namespace.

The optional dictionary argument \var{init_globals} may be used to
pre-populate the globals dictionary before the code is executed.
The supplied dictionary will not be modified. If any of the special
global variables below are defined in the supplied dictionary, those
definitions are overridden by the \code{run_module} function.

The special global variables \code{__name__}, \code{__file__},
\code{__loader__} and \code{__builtins__} are set in the globals
dictionary before the module code is executed.

\code{__name__} is set to \var{run_name} if this optional argument is
supplied, and the \var{mod_name} argument otherwise.

\code{__loader__} is set to the PEP 302 module loader used to retrieve
the code for the module (This loader may be a wrapper around the
standard import mechanism).

\code{__file__} is set to the name provided by the module loader. If
the loader does not make filename information available, this
variable is set to \code{None}.

\code{__builtins__} is automatically initialised with a reference to
the top level namespace of the \module{__builtin__} module.

If the argument \var{alter_sys} is supplied and evaluates to
\code{True}, then \code{sys.argv[0]} is updated with the value of
\code{__file__} and \code{sys.modules[__name__]} is updated with a
temporary module object for the module being executed. Both
\code{sys.argv[0]} and \code{sys.modules[__name__]} are restored to
their original values before the function returns.

Note that this manipulation of \module{sys} is not thread-safe. Other
threads may see the partially initialised module, as well as the
altered list of arguments. It is recommended that the \module{sys}
module be left alone when invoking this function from threaded code.
\end{funcdesc}

\begin{seealso}

\seepep{338}{Executing modules as scripts}{PEP written and 
implemented by Nick Coghlan.}

\end{seealso}

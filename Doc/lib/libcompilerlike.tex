\section{\module{compilerlike} ---
         framework code for building compiler-like programs.}

\declaremodule{standard}{set}
\modulesynopsis{Framework code for building compiler-like programs.}
\moduleauthor{Eric S. Raymond}{esr@thyrsus.com}
\sectionauthor{Eric S. Raymond}{esr@thyrsus.com}

There is a common `compiler-like' pattern in Unix scripts which is useful
for translation utilities of all sorts.  A program following this pattern
behaves as a filter when no argument files are specified on the command
line, but otherwise transforms each file individually into a corresponding
output file.

The \function{filefilter}, \function{linefilter}, and
\function{sponge} functions in this module provide a framework and
glue code to make such programs easy to write.  You supply a function
to massage the file data; depending on which entry point you use, it
can take input and output file pointers, or it can take a string
consisting of the entire file's data and return a replacement, or it
can take in succession strings consisting of each of the file's lines
and return a translated line for each.

All three of these entry points take a name, an argument list of files,
a data transformation function, and a name transformation function.
They differ only in the arguments they pass to the transformation
function when it is called.

The name argument is not used by the functions in this module, it is
simply passed as the first argument to the transformation function. 
Typically it is a string that names the filter and is used in
generating error messages, but it could be arbitrary data.

The second argument, is interpreted as a list of filenames.  The files
are transformed in left to right order in the list. A filename
consisting of a dash is interpreted as a directive to read from
standard input (this can be useful in pipelines).

The third argument is the data transformation function.
Interpretation of this argument varies across the three 
entry points and is described below.

The fourth, optional argument is a name transformation function or
name suffix string.  If it is of string type, the shortest suffix of each
filename beginning with the first character of the argument string
is stripped off.  If the first character of the argument does not 
occur in the filename, no suffix is removed.  Then the name suffix
argument is concatenated to the end of the stripped filename.  (Thus,
a name suffix argument of ".x" will cause the filenames foo.c and
bar.d to be transformed to foo.x and bar.x respectively.)
  
If the fourth argument is specified and is a function, the name of the
input file is passed to it and the return value of the function
becomes the name of the output software.  If this argument is not
specified, the imnput file is replaced with the transformed version.

Replacement of each file is atomic and doesn't occur until the
translation of that file has completed.  Any tempfiles are removed
automatically on any exception thrown by the translation function,
and the exception is then passed upwards.

\begin{funcdesc}{filefilter}{name, arguments, trans_data\optional{,trans_file}}
Filter using a function taking the name and two file-object
arguments. The function is expected to read data from the input file
object, transform it, and write the data to the output file object.
When the function terminates, the translation is done.  The return
value of the transformation function is not used.
\end{funcdesc}

\begin{funcdesc}{linefilter}{name,arguments,trans_data\optional{,trans_file}}
Filter using a function taking the name and a string argument.  The return
value of the function should be a string.  This function is applied to
each line in the input file in turn; the return values become the
lines of the transformed file.
\end{funcdesc}

\begin{funcdesc}{sponge}{name, arguments, trans_data\optional{, trans_file}}
Filter using a function taking the name and a string argument.  The
return value of the function should be a string. The function will be
passed the entire contents of the input file as a string.  The string
return value of the function will become the entire contents of the
transformed file.
\end{funcdesc}

# End



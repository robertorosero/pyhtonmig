\section{Built-in Module \sectcode{sys}}
\label{module-sys}

\bimodindex{sys}
This module provides access to some variables used or maintained by the
interpreter and to functions that interact strongly with the interpreter.
It is always available.

\renewcommand{\indexsubitem}{(in module sys)}

\begin{datadesc}{argv}
  The list of command line arguments passed to a Python script.
  \code{sys.argv[0]} is the script name (it is operating system
  dependent whether this is a full pathname or not).
  If the command was executed using the \samp{-c} command line option
  to the interpreter, \code{sys.argv[0]} is set to the string
  \code{"-c"}.
  If no script name was passed to the Python interpreter,
  \code{sys.argv} has zero length.
\end{datadesc}

\begin{datadesc}{builtin_module_names}
  A tuple of strings giving the names of all modules that are compiled
  into this Python interpreter.  (This information is not available in
  any other way --- \code{sys.modules.keys()} only lists the imported
  modules.)
\end{datadesc}

\begin{funcdesc}{exc_info}{}
This function returns a tuple of three values that give information
about the exception that is currently being handled.  The information
returned is specific both to the current thread and to the current
stack frame.  If the current stack frame is not handling an exception,
the information is taken from the calling stack frame, or its caller,
and so on until a stack frame is found that is handling an exception.
Here, ``handling an exception'' is defined as ``executing or having
executed an \code{except} clause.''  For any stack frame, only
information about the most recently handled exception is accessible.

If no exception is being handled anywhere on the stack, a tuple
containing three \code{None} values is returned.  Otherwise, the
values returned are
\code{(\var{type}, \var{value}, \var{traceback})}.
Their meaning is: \var{type} gets the exception type of the exception
being handled (a string or class object); \var{value} gets the
exception parameter (its \dfn{associated value} or the second argument
to \code{raise}, which is always a class instance if the exception
type is a class object); \var{traceback} gets a traceback object (see
the Reference Manual) which encapsulates the call stack at the point
where the exception originally occurred.
\obindex{traceback}

\strong{Warning:} assigning the \var{traceback} return value to a
local variable in a function that is handling an exception will cause
a circular reference. This will prevent anything referenced by a local
variable in the same function or by the traceback from being garbage
collected.  Since most functions don't need access to the traceback,
the best solution is to use something like
\code{type, value = sys.exc_info()[:2]}
to extract only the exception type and value.  If you do need the
traceback, make sure to delete it after use (best done with a
\code{try-finally} statement) or to call \code{sys.exc_info()} in a
function that does not itself handle an exception.
\end{funcdesc}

\begin{datadesc}{exc_type}
\dataline{exc_value}
\dataline{exc_traceback}
Use of these three variables is deprecated; they contain the same
values as returned by \code{sys.exc_info()} above.  However, since
they are global variables, they are not specific to the current
thread, so their use is not safe in a multi-threaded program.  When no
exception is being handled, \code{sys.exc_type} is set to \code{None}
and the other two are undefined.
\end{datadesc}

\begin{datadesc}{exec_prefix}
A string giving the site-specific
directory prefix where the platform-dependent Python files are
installed; by default, this is also \code{"/usr/local"}.  This can be
set at build time with the \code{--exec-prefix} argument to the
\code{configure} script.  Specifically, all configuration files
(e.g. the \code{config.h} header file) are installed in the directory
\code{sys.exec_prefix+"/lib/python\emph{VER}/config"}, and shared library
modules are installed in
\code{sys.exec_prefix+"/lib/python\emph{VER}/sharedmodules"},
where \emph{VER} is equal to \code{sys.version[:3]}.
\end{datadesc}

\begin{funcdesc}{exit}{n}
  Exit from Python with numeric exit status \var{n}.  This is
  implemented by raising the \code{SystemExit} exception, so cleanup
  actions specified by \code{finally} clauses of \code{try} statements
  are honored, and it is possible to catch the exit attempt at an outer
  level.
\end{funcdesc}

\begin{datadesc}{exitfunc}
  This value is not actually defined by the module, but can be set by
  the user (or by a program) to specify a clean-up action at program
  exit.  When set, it should be a parameterless function.  This function
  will be called when the interpreter exits in any way (except when a
  fatal error occurs: in that case the interpreter's internal state
  cannot be trusted).
\end{datadesc}

\begin{datadesc}{last_type}
\dataline{last_value}
\dataline{last_traceback}
These three variables are not always defined; they are set when an
exception is not handled and the interpreter prints an error message
and a stack traceback.  Their intended use is to allow an interactive
user to import a debugger module and engage in post-mortem debugging
without having to re-execute the command that caused the error.
(Typical use is \code{import pdb; pdb.pm()} to enter the post-mortem
debugger; see the chapter ``The Python Debugger'' for more
information.)
\stmodindex{pdb}

The meaning of the variables is the same
as that of the return values from \code{sys.exc_info()} above.
(Since there is only one interactive thread, thread-safety is not a
concern for these variables, unlike for \code{sys.exc_type} etc.)
\end{datadesc}

\begin{datadesc}{modules}
  Gives the list of modules that have already been loaded.
  This can be manipulated to force reloading of modules and other tricks.
\end{datadesc}

\begin{datadesc}{path}
  A list of strings that specifies the search path for modules.
  Initialized from the environment variable \code{PYTHONPATH}, or an
  installation-dependent default.  

The first item of this list, \code{sys.path[0]}, is the 
directory containing the script that was used to invoke the Python 
interpreter.  If the script directory is not available (e.g.  if the 
interpreter is invoked interactively or if the script is read from 
standard input), \code{sys.path[0]} is the empty string, which directs 
Python to search modules in the current directory first.  Notice that 
the script directory is inserted {\em before} the entries inserted as 
a result of \code{\$PYTHONPATH}.  
\end{datadesc}

\begin{datadesc}{platform}
This string contains a platform identifier, e.g. \code{sunos5} or
\code{linux1}.  This can be used to append platform-specific
components to \code{sys.path}, for instance. 
\end{datadesc}

\begin{datadesc}{prefix}
A string giving the site-specific directory prefix where the platform
independent Python files are installed; by default, this is the string
\code{"/usr/local"}.  This can be set at build time with the
\code{--prefix} argument to the \code{configure} script.  The main
collection of Python library modules is installed in the directory
\code{sys.prefix+"/lib/python\emph{VER}"} while the platform
independent header files (all except \code{config.h}) are stored in
\code{sys.prefix+"/include/python\emph{VER}"},
where \emph{VER} is equal to \code{sys.version[:3]}.

\end{datadesc}

\begin{datadesc}{ps1}
\dataline{ps2}
  Strings specifying the primary and secondary prompt of the
  interpreter.  These are only defined if the interpreter is in
  interactive mode.  Their initial values in this case are
  \code{'>>> '} and \code{'... '}.
\end{datadesc}

\begin{funcdesc}{setcheckinterval}{interval}
Set the interpreter's ``check interval''.  This integer value
determines how often the interpreter checks for periodic things such
as thread switches and signal handlers.  The default is 10, meaning
the check is performed every 10 Python virtual instructions.  Setting
it to a larger value may increase performance for programs using
threads.  Setting it to a value \code{<=} 0 checks every virtual instruction,
maximizing responsiveness as well as overhead.
\end{funcdesc}

\begin{funcdesc}{settrace}{tracefunc}
  Set the system's trace function, which allows you to implement a
  Python source code debugger in Python.  See section ``How It Works''
  in the chapter on the Python Debugger.
\end{funcdesc}
\index{trace function}
\index{debugger}

\begin{funcdesc}{setprofile}{profilefunc}
  Set the system's profile function, which allows you to implement a
  Python source code profiler in Python.  See the chapter on the
  Python Profiler.  The system's profile function
  is called similarly to the system's trace function (see
  \code{sys.settrace}), but it isn't called for each executed line of
  code (only on call and return and when an exception occurs).  Also,
  its return value is not used, so it can just return \code{None}.
\end{funcdesc}
\index{profile function}
\index{profiler}

\begin{datadesc}{stdin}
\dataline{stdout}
\dataline{stderr}
  File objects corresponding to the interpreter's standard input,
  output and error streams.  \code{sys.stdin} is used for all
  interpreter input except for scripts but including calls to
  \code{input()} and \code{raw_input()}.  \code{sys.stdout} is used
  for the output of \code{print} and expression statements and for the
  prompts of \code{input()} and \code{raw_input()}.  The interpreter's
  own prompts and (almost all of) its error messages go to
  \code{sys.stderr}.  \code{sys.stdout} and \code{sys.stderr} needn't
  be built-in file objects: any object is acceptable as long as it has
  a \code{write} method that takes a string argument.  (Changing these
  objects doesn't affect the standard I/O streams of processes
  executed by \code{popen()}, \code{system()} or the \code{exec*()}
  family of functions in the \code{os} module.)
\stmodindex{os}
\end{datadesc}

\begin{datadesc}{tracebacklimit}
When this variable is set to an integer value, it determines the
maximum number of levels of traceback information printed when an
unhandled exception occurs.  The default is 1000.  When set to 0 or
less, all traceback information is suppressed and only the exception
type and value are printed.
\end{datadesc}

\begin{datadesc}{version}
A string containing the version number of the Python interpreter.  
\end{datadesc}

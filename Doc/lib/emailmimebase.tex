Ordinarily, you get a message object structure by passing a file or
some text to a parser, which parses the text and returns the root
message object.  However you can also build a complete message
structure from scratch, or even individual \class{Message} objects by
hand.  In fact, you can also take an existing structure and add new
\class{Message} objects, move them around, etc.  This makes a very
convenient interface for slicing-and-dicing MIME messages.

You can create a new object structure by creating \class{Message}
instances, adding attachments and all the appropriate headers manually.
For MIME messages though, the \module{email} package provides some
convenient subclasses to make things easier.  Each of these classes
should be imported from a module with the same name as the class, from
within the \module{email} package.  E.g.:

\begin{verbatim}
import email.MIMEImage.MIMEImage
\end{verbatim}

or

\begin{verbatim}
from email.MIMEText import MIMEText
\end{verbatim}

Here are the classes:

\begin{classdesc}{MIMEBase}{_maintype, _subtype, **_params}
This is the base class for all the MIME-specific subclasses of
\class{Message}.  Ordinarily you won't create instances specifically
of \class{MIMEBase}, although you could.  \class{MIMEBase} is provided
primarily as a convenient base class for more specific MIME-aware
subclasses.

\var{_maintype} is the \mailheader{Content-Type} major type
(e.g. \mimetype{text} or \mimetype{image}), and \var{_subtype} is the
\mailheader{Content-Type} minor type 
(e.g. \mimetype{plain} or \mimetype{gif}).  \var{_params} is a parameter
key/value dictionary and is passed directly to
\method{Message.add_header()}.

The \class{MIMEBase} class always adds a \mailheader{Content-Type} header
(based on \var{_maintype}, \var{_subtype}, and \var{_params}), and a
\mailheader{MIME-Version} header (always set to \code{1.0}).
\end{classdesc}

\begin{classdesc}{MIMENonMultipart}{}
A subclass of \class{MIMEBase}, this is an intermediate base class for
MIME messages that are not \mimetype{multipart}.  The primary purpose
of this class is to prevent the use of the \method{attach()} method,
which only makes sense for \mimetype{multipart} messages.  If
\method{attach()} is called, a \exception{MultipartConversionError}
exception is raised.

\versionadded{2.2.2}
\end{classdesc}

\begin{classdesc}{MIMEMultipart}{\optional{subtype\optional{,
    boundary\optional{, _subparts\optional{, _params}}}}}

A subclass of \class{MIMEBase}, this is an intermediate base class for
MIME messages that are \mimetype{multipart}.  Optional \var{_subtype}
defaults to \mimetype{mixed}, but can be used to specify the subtype
of the message.  A \mailheader{Content-Type} header of
\mimetype{multipart/}\var{_subtype} will be added to the message
object.  A \mailheader{MIME-Version} header will also be added.

Optional \var{boundary} is the multipart boundary string.  When
\code{None} (the default), the boundary is calculated when needed.

\var{_subparts} is a sequence of initial subparts for the payload.  It
must be possible to convert this sequence to a list.  You can always
attach new subparts to the message by using the
\method{Message.attach()} method.

Additional parameters for the \mailheader{Content-Type} header are
taken from the keyword arguments, or passed into the \var{_params}
argument, which is a keyword dictionary.

\versionadded{2.2.2}
\end{classdesc}

\begin{classdesc}{MIMEAudio}{_audiodata\optional{, _subtype\optional{,
    _encoder\optional{, **_params}}}}

A subclass of \class{MIMENonMultipart}, the \class{MIMEAudio} class
is used to create MIME message objects of major type \mimetype{audio}.
\var{_audiodata} is a string containing the raw audio data.  If this
data can be decoded by the standard Python module \refmodule{sndhdr},
then the subtype will be automatically included in the
\mailheader{Content-Type} header.  Otherwise you can explicitly specify the
audio subtype via the \var{_subtype} parameter.  If the minor type could
not be guessed and \var{_subtype} was not given, then \exception{TypeError}
is raised.

Optional \var{_encoder} is a callable (i.e. function) which will
perform the actual encoding of the audio data for transport.  This
callable takes one argument, which is the \class{MIMEAudio} instance.
It should use \method{get_payload()} and \method{set_payload()} to
change the payload to encoded form.  It should also add any
\mailheader{Content-Transfer-Encoding} or other headers to the message
object as necessary.  The default encoding is base64.  See the
\refmodule{email.Encoders} module for a list of the built-in encoders.

\var{_params} are passed straight through to the base class constructor.
\end{classdesc}

\begin{classdesc}{MIMEImage}{_imagedata\optional{, _subtype\optional{,
    _encoder\optional{, **_params}}}}

A subclass of \class{MIMENonMultipart}, the \class{MIMEImage} class is
used to create MIME message objects of major type \mimetype{image}.
\var{_imagedata} is a string containing the raw image data.  If this
data can be decoded by the standard Python module \refmodule{imghdr},
then the subtype will be automatically included in the
\mailheader{Content-Type} header.  Otherwise you can explicitly specify the
image subtype via the \var{_subtype} parameter.  If the minor type could
not be guessed and \var{_subtype} was not given, then \exception{TypeError}
is raised.

Optional \var{_encoder} is a callable (i.e. function) which will
perform the actual encoding of the image data for transport.  This
callable takes one argument, which is the \class{MIMEImage} instance.
It should use \method{get_payload()} and \method{set_payload()} to
change the payload to encoded form.  It should also add any
\mailheader{Content-Transfer-Encoding} or other headers to the message
object as necessary.  The default encoding is base64.  See the
\refmodule{email.Encoders} module for a list of the built-in encoders.

\var{_params} are passed straight through to the \class{MIMEBase}
constructor.
\end{classdesc}

\begin{classdesc}{MIMEMessage}{_msg\optional{, _subtype}}
A subclass of \class{MIMENonMultipart}, the \class{MIMEMessage} class
is used to create MIME objects of main type \mimetype{message}.
\var{_msg} is used as the payload, and must be an instance of class
\class{Message} (or a subclass thereof), otherwise a
\exception{TypeError} is raised.

Optional \var{_subtype} sets the subtype of the message; it defaults
to \mimetype{rfc822}.
\end{classdesc}

\begin{classdesc}{MIMEText}{_text\optional{, _subtype\optional{,
    _charset\optional{, _encoder}}}}

A subclass of \class{MIMENonMultipart}, the \class{MIMEText} class is
used to create MIME objects of major type \mimetype{text}.
\var{_text} is the string for the payload.  \var{_subtype} is the
minor type and defaults to \mimetype{plain}.  \var{_charset} is the
character set of the text and is passed as a parameter to the
\class{MIMENonMultipart} constructor; it defaults to \code{us-ascii}.  No
guessing or encoding is performed on the text data.

\deprecated{2.2.2}{The \var{_encoding} argument has been deprecated.
Encoding now happens implicitly based on the \var{_charset} argument.}
\end{classdesc}

\section{\module{mpz} ---
         GNU arbitrary magnitude integers}

\declaremodule{builtin}{mpz}
\modulesynopsis{Interface to the GNU MP library for arbitrary
precision arithmetic.}


This is an optional module.  It is only available when Python is
configured to include it, which requires that the GNU MP software is
installed.
\index{MP, GNU library}
\index{arbitrary precision integers}
\index{integer!arbitrary precision}

This module implements the interface to part of the GNU MP library,
which defines arbitrary precision integer and rational number
arithmetic routines.  Only the interfaces to the \emph{integer}
(\function{mpz_*()}) routines are provided. If not stated
otherwise, the description in the GNU MP documentation can be applied.

Support for rational numbers\index{rational numbers} can be
implemented in Python.  For an example, see the
\module{Rat}\withsubitem{(demo module)}{\ttindex{Rat}} module, provided as
\file{Demos/classes/Rat.py} in the Python source distribution.

In general, \dfn{mpz}-numbers can be used just like other standard
Python numbers, e.g., you can use the built-in operators like \code{+},
\code{*}, etc., as well as the standard built-in functions like
\function{abs()}, \function{int()}, \ldots, \function{divmod()},
\function{pow()}.  \strong{Please note:} the \emph{bitwise-xor}
operation has been implemented as a bunch of \emph{and}s,
\emph{invert}s and \emph{or}s, because the library lacks an
\cfunction{mpz_xor()} function, and I didn't need one.

You create an mpz-number by calling the function \function{mpz()} (see
below for an exact description). An mpz-number is printed like this:
\code{mpz(\var{value})}.


\begin{funcdesc}{mpz}{value}
  Create a new mpz-number. \var{value} can be an integer, a long,
  another mpz-number, or even a string. If it is a string, it is
  interpreted as an array of radix-256 digits, least significant digit
  first, resulting in a positive number. See also the \method{binary()}
  method, described below.
\end{funcdesc}

\begin{datadesc}{MPZType}
  The type of the objects returned by \function{mpz()} and most other
  functions in this module.
\end{datadesc}


A number of \emph{extra} functions are defined in this module. Non
mpz-arguments are converted to mpz-values first, and the functions
return mpz-numbers.

\begin{funcdesc}{powm}{base, exponent, modulus}
  Return \code{pow(\var{base}, \var{exponent}) \%{} \var{modulus}}. If
  \code{\var{exponent} == 0}, return \code{mpz(1)}. In contrast to the
  \C{} library function, this version can handle negative exponents.
\end{funcdesc}

\begin{funcdesc}{gcd}{op1, op2}
  Return the greatest common divisor of \var{op1} and \var{op2}.
\end{funcdesc}

\begin{funcdesc}{gcdext}{a, b}
  Return a tuple \code{(\var{g}, \var{s}, \var{t})}, such that
  \code{\var{a}*\var{s} + \var{b}*\var{t} == \var{g} == gcd(\var{a}, \var{b})}.
\end{funcdesc}

\begin{funcdesc}{sqrt}{op}
  Return the square root of \var{op}. The result is rounded towards zero.
\end{funcdesc}

\begin{funcdesc}{sqrtrem}{op}
  Return a tuple \code{(\var{root}, \var{remainder})}, such that
  \code{\var{root}*\var{root} + \var{remainder} == \var{op}}.
\end{funcdesc}

\begin{funcdesc}{divm}{numerator, denominator, modulus}
  Returns a number \var{q} such that
  \code{\var{q} * \var{denominator} \%{} \var{modulus} ==
  \var{numerator}}.  One could also implement this function in Python,
  using \function{gcdext()}.
\end{funcdesc}

An mpz-number has one method:

\begin{methoddesc}[mpz]{binary}{}
  Convert this mpz-number to a binary string, where the number has been
  stored as an array of radix-256 digits, least significant digit first.

  The mpz-number must have a value greater than or equal to zero,
  otherwise \exception{ValueError} will be raised.
\end{methoddesc}


\begin{seealso}
  \seetitle[http://gmpy.sourceforge.net/]{General Multiprecision Python}{
            This project is building new numeric types to allow
            arbitrary-precision arithmetic in Python.  Their first
            efforts are also based on the GNU MP library.}

  \seetitle[http://www.egenix.com/files/python/mxNumber.html]{mxNumber
            --- Extended Numeric Types for Python}{Another wrapper
            around the GNU MP library, including a port of that
            library to Windows.}
\end{seealso}

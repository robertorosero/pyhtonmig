\section{\module{xml.parsers.expat} ---
         Fast XML parsing using Expat}

% Markup notes:
%
% Many of the attributes of the XMLParser objects are callbacks.
% Since signature information must be presented, these are described
% using the methoddesc environment.  Since they are attributes which
% are set by client code, in-text references to these attributes
% should be marked using the \member macro and should not include the
% parentheses used when marking functions and methods.

\declaremodule{standard}{xml.parsers.expat}
\modulesynopsis{An interface to the Expat non-validating XML parser.}
\moduleauthor{Paul Prescod}{paul@prescod.net}

\versionadded{2.0}

The \module{xml.parsers.expat} module is a Python interface to the
Expat\index{Expat} non-validating XML parser.
The module provides a single extension type, \class{xmlparser}, that
represents the current state of an XML parser.  After an
\class{xmlparser} object has been created, various attributes of the object 
can be set to handler functions.  When an XML document is then fed to
the parser, the handler functions are called for the character data
and markup in the XML document.

This module uses the \module{pyexpat}\refbimodindex{pyexpat} module to
provide access to the Expat parser.  Direct use of the
\module{pyexpat} module is deprecated.

This module provides one exception and one type object:

\begin{excdesc}{ExpatError}
  The exception raised when Expat reports an error.  See section
  \ref{expaterror-objects}, ``ExpatError Exceptions,'' for more
  information on interpreting Expat errors.
\end{excdesc}

\begin{excdesc}{error}
  Alias for \exception{ExpatError}.
\end{excdesc}

\begin{datadesc}{XMLParserType}
  The type of the return values from the \function{ParserCreate()}
  function.
\end{datadesc}


The \module{xml.parsers.expat} module contains two functions:

\begin{funcdesc}{ErrorString}{errno}
Returns an explanatory string for a given error number \var{errno}.
\end{funcdesc}

\begin{funcdesc}{ParserCreate}{\optional{encoding\optional{,
                               namespace_separator}}}
Creates and returns a new \class{xmlparser} object.  
\var{encoding}, if specified, must be a string naming the encoding 
used by the XML data.  Expat doesn't support as many encodings as
Python does, and its repertoire of encodings can't be extended; it
supports UTF-8, UTF-16, ISO-8859-1 (Latin1), and ASCII.  If
\var{encoding} is given it will override the implicit or explicit
encoding of the document.

Expat can optionally do XML namespace processing for you, enabled by
providing a value for \var{namespace_separator}.  The value must be a
one-character string; a \exception{ValueError} will be raised if the
string has an illegal length (\code{None} is considered the same as
omission).  When namespace processing is enabled, element type names
and attribute names that belong to a namespace will be expanded.  The
element name passed to the element handlers
\member{StartElementHandler} and \member{EndElementHandler}
will be the concatenation of the namespace URI, the namespace
separator character, and the local part of the name.  If the namespace
separator is a zero byte (\code{chr(0)}) then the namespace URI and
the local part will be concatenated without any separator.

For example, if \var{namespace_separator} is set to a space character
(\character{ }) and the following document is parsed:

\begin{verbatim}
<?xml version="1.0"?>
<root xmlns    = "http://default-namespace.org/"
      xmlns:py = "http://www.python.org/ns/">
  <py:elem1 />
  <elem2 xmlns="" />
</root>
\end{verbatim}

\member{StartElementHandler} will receive the following strings
for each element:

\begin{verbatim}
http://default-namespace.org/ root
http://www.python.org/ns/ elem1
elem2
\end{verbatim}
\end{funcdesc}


\begin{seealso}
  \seetitle[http://www.libexpat.org/]{The Expat XML Parser}
           {Home page of the Expat project.}
\end{seealso}


\subsection{XMLParser Objects \label{xmlparser-objects}}

\class{xmlparser} objects have the following methods:

\begin{methoddesc}[xmlparser]{Parse}{data\optional{, isfinal}}
Parses the contents of the string \var{data}, calling the appropriate
handler functions to process the parsed data.  \var{isfinal} must be
true on the final call to this method.  \var{data} can be the empty
string at any time.
\end{methoddesc}

\begin{methoddesc}[xmlparser]{ParseFile}{file}
Parse XML data reading from the object \var{file}.  \var{file} only
needs to provide the \method{read(\var{nbytes})} method, returning the
empty string when there's no more data.
\end{methoddesc}

\begin{methoddesc}[xmlparser]{SetBase}{base}
Sets the base to be used for resolving relative URIs in system
identifiers in declarations.  Resolving relative identifiers is left
to the application: this value will be passed through as the
\var{base} argument to the \function{ExternalEntityRefHandler},
\function{NotationDeclHandler}, and
\function{UnparsedEntityDeclHandler} functions.
\end{methoddesc}

\begin{methoddesc}[xmlparser]{GetBase}{}
Returns a string containing the base set by a previous call to
\method{SetBase()}, or \code{None} if 
\method{SetBase()} hasn't been called.
\end{methoddesc}

\begin{methoddesc}[xmlparser]{GetInputContext}{}
Returns the input data that generated the current event as a string.
The data is in the encoding of the entity which contains the text.
When called while an event handler is not active, the return value is
\code{None}.
\versionadded{2.1}
\end{methoddesc}

\begin{methoddesc}[xmlparser]{ExternalEntityParserCreate}{context\optional{,
                                                          encoding}}
Create a ``child'' parser which can be used to parse an external
parsed entity referred to by content parsed by the parent parser.  The
\var{context} parameter should be the string passed to the
\method{ExternalEntityRefHandler()} handler function, described below.
The child parser is created with the \member{ordered_attributes},
\member{returns_unicode} and \member{specified_attributes} set to the
values of this parser.
\end{methoddesc}


\class{xmlparser} objects have the following attributes:

\begin{memberdesc}[xmlparser]{ordered_attributes}
Setting this attribute to a non-zero integer causes the attributes to
be reported as a list rather than a dictionary.  The attributes are
presented in the order found in the document text.  For each
attribute, two list entries are presented: the attribute name and the
attribute value.  (Older versions of this module also used this
format.)  By default, this attribute is false; it may be changed at
any time.
\versionadded{2.1}
\end{memberdesc}

\begin{memberdesc}[xmlparser]{returns_unicode} 
If this attribute is set to a non-zero integer, the handler functions
will be passed Unicode strings.  If \member{returns_unicode} is 0,
8-bit strings containing UTF-8 encoded data will be passed to the
handlers.
\versionchanged[Can be changed at any time to affect the result
  type]{1.6}
\end{memberdesc}

\begin{memberdesc}[xmlparser]{specified_attributes}
If set to a non-zero integer, the parser will report only those
attributes which were specified in the document instance and not those
which were derived from attribute declarations.  Applications which
set this need to be especially careful to use what additional
information is available from the declarations as needed to comply
with the standards for the behavior of XML processors.  By default,
this attribute is false; it may be changed at any time.
\versionadded{2.1}
\end{memberdesc}

The following attributes contain values relating to the most recent
error encountered by an \class{xmlparser} object, and will only have
correct values once a call to \method{Parse()} or \method{ParseFile()}
has raised a \exception{xml.parsers.expat.ExpatError} exception.

\begin{memberdesc}[xmlparser]{ErrorByteIndex} 
Byte index at which an error occurred.
\end{memberdesc} 

\begin{memberdesc}[xmlparser]{ErrorCode} 
Numeric code specifying the problem.  This value can be passed to the
\function{ErrorString()} function, or compared to one of the constants
defined in the \code{errors} object.
\end{memberdesc}

\begin{memberdesc}[xmlparser]{ErrorColumnNumber} 
Column number at which an error occurred.
\end{memberdesc}

\begin{memberdesc}[xmlparser]{ErrorLineNumber}
Line number at which an error occurred.
\end{memberdesc}

Here is the list of handlers that can be set.  To set a handler on an
\class{xmlparser} object \var{o}, use
\code{\var{o}.\var{handlername} = \var{func}}.  \var{handlername} must
be taken from the following list, and \var{func} must be a callable
object accepting the correct number of arguments.  The arguments are
all strings, unless otherwise stated.

\begin{methoddesc}[xmlparser]{XmlDeclHandler}{version, encoding, standalone}
Called when the XML declaration is parsed.  The XML declaration is the
(optional) declaration of the applicable version of the XML
recommendation, the encoding of the document text, and an optional
``standalone'' declaration.  \var{version} and \var{encoding} will be
strings of the type dictated by the \member{returns_unicode}
attribute, and \var{standalone} will be \code{1} if the document is
declared standalone, \code{0} if it is declared not to be standalone,
or \code{-1} if the standalone clause was omitted.
This is only available with Expat version 1.95.0 or newer.
\versionadded{2.1}
\end{methoddesc}

\begin{methoddesc}[xmlparser]{StartDoctypeDeclHandler}{doctypeName,
                                                       systemId, publicId,
                                                       has_internal_subset}
Called when Expat begins parsing the document type declaration
(\code{<!DOCTYPE \ldots}).  The \var{doctypeName} is provided exactly
as presented.  The \var{systemId} and \var{publicId} parameters give
the system and public identifiers if specified, or \code{None} if
omitted.  \var{has_internal_subset} will be true if the document
contains and internal document declaration subset.
This requires Expat version 1.2 or newer.
\end{methoddesc}

\begin{methoddesc}[xmlparser]{EndDoctypeDeclHandler}{}
Called when Expat is done parsing the document type delaration.
This requires Expat version 1.2 or newer.
\end{methoddesc}

\begin{methoddesc}[xmlparser]{ElementDeclHandler}{name, model}
Called once for each element type declaration.  \var{name} is the name
of the element type, and \var{model} is a representation of the
content model.
\end{methoddesc}

\begin{methoddesc}[xmlparser]{AttlistDeclHandler}{elname, attname,
                                                  type, default, required}
Called for each declared attribute for an element type.  If an
attribute list declaration declares three attributes, this handler is
called three times, once for each attribute.  \var{elname} is the name
of the element to which the declaration applies and \var{attname} is
the name of the attribute declared.  The attribute type is a string
passed as \var{type}; the possible values are \code{'CDATA'},
\code{'ID'}, \code{'IDREF'}, ...
\var{default} gives the default value for the attribute used when the
attribute is not specified by the document instance, or \code{None} if
there is no default value (\code{\#IMPLIED} values).  If the attribute
is required to be given in the document instance, \var{required} will
be true.
This requires Expat version 1.95.0 or newer.
\end{methoddesc}

\begin{methoddesc}[xmlparser]{StartElementHandler}{name, attributes}
Called for the start of every element.  \var{name} is a string
containing the element name, and \var{attributes} is a dictionary
mapping attribute names to their values.
\end{methoddesc}

\begin{methoddesc}[xmlparser]{EndElementHandler}{name}
Called for the end of every element.
\end{methoddesc}

\begin{methoddesc}[xmlparser]{ProcessingInstructionHandler}{target, data}
Called for every processing instruction.
\end{methoddesc}

\begin{methoddesc}[xmlparser]{CharacterDataHandler}{data}
Called for character data.  This will be called for normal character
data, CDATA marked content, and ignorable whitespace.  Applications
which must distinguish these cases can use the
\member{StartCdataSectionHandler}, \member{EndCdataSectionHandler},
and \member{ElementDeclHandler} callbacks to collect the required
information.
\end{methoddesc}

\begin{methoddesc}[xmlparser]{UnparsedEntityDeclHandler}{entityName, base,
                                                         systemId, publicId,
                                                         notationName}
Called for unparsed (NDATA) entity declarations.  This is only present
for version 1.2 of the Expat library; for more recent versions, use
\member{EntityDeclHandler} instead.  (The underlying function in the
Expat library has been declared obsolete.)
\end{methoddesc}

\begin{methoddesc}[xmlparser]{EntityDeclHandler}{entityName,
                                                 is_parameter_entity, value,
                                                 base, systemId,
                                                 publicId,
                                                 notationName}
Called for all entity declarations.  For parameter and internal
entities, \var{value} will be a string giving the declared contents
of the entity; this will be \code{None} for external entities.  The
\var{notationName} parameter will be \code{None} for parsed entities,
and the name of the notation for unparsed entities.
\var{is_parameter_entity} will be true if the entity is a paremeter
entity or false for general entities (most applications only need to
be concerned with general entities).
This is only available starting with version 1.95.0 of the Expat
library.
\versionadded{2.1}
\end{methoddesc}

\begin{methoddesc}[xmlparser]{NotationDeclHandler}{notationName, base,
                                                   systemId, publicId}
Called for notation declarations.  \var{notationName}, \var{base}, and
\var{systemId}, and \var{publicId} are strings if given.  If the
public identifier is omitted, \var{publicId} will be \code{None}.
\end{methoddesc}

\begin{methoddesc}[xmlparser]{StartNamespaceDeclHandler}{prefix, uri}
Called when an element contains a namespace declaration.  Namespace
declarations are processed before the \member{StartElementHandler} is
called for the element on which declarations are placed.
\end{methoddesc}

\begin{methoddesc}[xmlparser]{EndNamespaceDeclHandler}{prefix}
Called when the closing tag is reached for an element 
that contained a namespace declaration.  This is called once for each
namespace declaration on the element in the reverse of the order for
which the \member{StartNamespaceDeclHandler} was called to indicate
the start of each namespace declaration's scope.  Calls to this
handler are made after the corresponding \member{EndElementHandler}
for the end of the element.
\end{methoddesc}

\begin{methoddesc}[xmlparser]{CommentHandler}{data}
Called for comments.  \var{data} is the text of the comment, excluding
the leading `\code{<!-}\code{-}' and trailing `\code{-}\code{->}'.
\end{methoddesc}

\begin{methoddesc}[xmlparser]{StartCdataSectionHandler}{}
Called at the start of a CDATA section.  This and
\member{StartCdataSectionHandler} are needed to be able to identify
the syntactical start and end for CDATA sections.
\end{methoddesc}

\begin{methoddesc}[xmlparser]{EndCdataSectionHandler}{}
Called at the end of a CDATA section.
\end{methoddesc}

\begin{methoddesc}[xmlparser]{DefaultHandler}{data}
Called for any characters in the XML document for
which no applicable handler has been specified.  This means
characters that are part of a construct which could be reported, but
for which no handler has been supplied. 
\end{methoddesc}

\begin{methoddesc}[xmlparser]{DefaultHandlerExpand}{data}
This is the same as the \function{DefaultHandler}, 
but doesn't inhibit expansion of internal entities.
The entity reference will not be passed to the default handler.
\end{methoddesc}

\begin{methoddesc}[xmlparser]{NotStandaloneHandler}{} Called if the
XML document hasn't been declared as being a standalone document.
This happens when there is an external subset or a reference to a
parameter entity, but the XML declaration does not set standalone to
\code{yes} in an XML declaration.  If this handler returns \code{0},
then the parser will throw an \constant{XML_ERROR_NOT_STANDALONE}
error.  If this handler is not set, no exception is raised by the
parser for this condition.
\end{methoddesc}

\begin{methoddesc}[xmlparser]{ExternalEntityRefHandler}{context, base,
                                                        systemId, publicId}
Called for references to external entities.  \var{base} is the current
base, as set by a previous call to \method{SetBase()}.  The public and
system identifiers, \var{systemId} and \var{publicId}, are strings if
given; if the public identifier is not given, \var{publicId} will be
\code{None}.  The \var{context} value is opaque and should only be
used as described below.

For external entities to be parsed, this handler must be implemented.
It is responsible for creating the sub-parser using
\code{ExternalEntityParserCreate(\var{context})}, initializing it with
the appropriate callbacks, and parsing the entity.  This handler
should return an integer; if it returns \code{0}, the parser will
throw an \constant{XML_ERROR_EXTERNAL_ENTITY_HANDLING} error,
otherwise parsing will continue.

If this handler is not provided, external entities are reported by the
\member{DefaultHandler} callback, if provided.
\end{methoddesc}


\subsection{ExpatError Exceptions \label{expaterror-objects}}
\sectionauthor{Fred L. Drake, Jr.}{fdrake@acm.org}

\exception{ExpatError} exceptions have a number of interesting
attributes:

\begin{memberdesc}[ExpatError]{code}
  Expat's internal error number for the specific error.  This will
  match one of the constants defined in the \code{errors} object from
  this module.
  \versionadded{2.1}
\end{memberdesc}

\begin{memberdesc}[ExpatError]{lineno}
  Line number on which the error was detected.  The first line is
  numbered \code{1}.
  \versionadded{2.1}
\end{memberdesc}

\begin{memberdesc}[ExpatError]{offset}
  Character offset into the line where the error occurred.  The first
  column is numbered \code{0}.
  \versionadded{2.1}
\end{memberdesc}


\subsection{Example \label{expat-example}}

The following program defines three handlers that just print out their
arguments.

\begin{verbatim}
import xml.parsers.expat

# 3 handler functions
def start_element(name, attrs):
    print 'Start element:', name, attrs
def end_element(name):
    print 'End element:', name
def char_data(data):
    print 'Character data:', repr(data)

p = xml.parsers.expat.ParserCreate()

p.StartElementHandler = start_element
p.EndElementHandler = end_element
p.CharacterDataHandler = char_data

p.Parse("""<?xml version="1.0"?>
<parent id="top"><child1 name="paul">Text goes here</child1>
<child2 name="fred">More text</child2>
</parent>""")
\end{verbatim}

The output from this program is:

\begin{verbatim}
Start element: parent {'id': 'top'}
Start element: child1 {'name': 'paul'}
Character data: 'Text goes here'
End element: child1
Character data: '\n'
Start element: child2 {'name': 'fred'}
Character data: 'More text'
End element: child2
Character data: '\n'
End element: parent
\end{verbatim}


\subsection{Content Model Descriptions \label{expat-content-models}}
\sectionauthor{Fred L. Drake, Jr.}{fdrake@acm.org}

Content modules are described using nested tuples.  Each tuple
contains four values: the type, the quantifier, the name, and a tuple
of children.  Children are simply additional content module
descriptions.

The values of the first two fields are constants defined in the
\code{model} object of the \module{xml.parsers.expat} module.  These
constants can be collected in two groups: the model type group and the
quantifier group.

The constants in the model type group are:

\begin{datadescni}{XML_CTYPE_ANY}
The element named by the model name was declared to have a content
model of \code{ANY}.
\end{datadescni}

\begin{datadescni}{XML_CTYPE_CHOICE}
The named element allows a choice from a number of options; this is
used for content models such as \code{(A | B | C)}.
\end{datadescni}

\begin{datadescni}{XML_CTYPE_EMPTY}
Elements which are declared to be \code{EMPTY} have this model type.
\end{datadescni}

\begin{datadescni}{XML_CTYPE_MIXED}
\end{datadescni}

\begin{datadescni}{XML_CTYPE_NAME}
\end{datadescni}

\begin{datadescni}{XML_CTYPE_SEQ}
Models which represent a series of models which follow one after the
other are indicated with this model type.  This is used for models
such as \code{(A, B, C)}.
\end{datadescni}


The constants in the quantifier group are:

\begin{datadescni}{XML_CQUANT_NONE}
No modifier is given, so it can appear exactly once, as for \code{A}.
\end{datadescni}

\begin{datadescni}{XML_CQUANT_OPT}
The model is optional: it can appear once or not at all, as for
\code{A?}.
\end{datadescni}

\begin{datadescni}{XML_CQUANT_PLUS}
The model must occur one or more times (like \code{A+}).
\end{datadescni}

\begin{datadescni}{XML_CQUANT_REP}
The model must occur zero or more times, as for \code{A*}.
\end{datadescni}


\subsection{Expat error constants \label{expat-errors}}

The following constants are provided in the \code{errors} object of
the \refmodule{xml.parsers.expat} module.  These constants are useful
in interpreting some of the attributes of the \exception{ExpatError}
exception objects raised when an error has occurred.

The \code{errors} object has the following attributes:

\begin{datadescni}{XML_ERROR_ASYNC_ENTITY}
\end{datadescni}

\begin{datadescni}{XML_ERROR_ATTRIBUTE_EXTERNAL_ENTITY_REF}
An entity reference in an attribute value referred to an external
entity instead of an internal entity.
\end{datadescni}

\begin{datadescni}{XML_ERROR_BAD_CHAR_REF}
A character reference referred to a character which is illegal in XML
(for example, character \code{0}, or `\code{\&\#0;}'.
\end{datadescni}

\begin{datadescni}{XML_ERROR_BINARY_ENTITY_REF}
An entity reference referred to an entity which was declared with a
notation, so cannot be parsed.
\end{datadescni}

\begin{datadescni}{XML_ERROR_DUPLICATE_ATTRIBUTE}
An attribute was used more than once in a start tag.
\end{datadescni}

\begin{datadescni}{XML_ERROR_INCORRECT_ENCODING}
\end{datadescni}

\begin{datadescni}{XML_ERROR_INVALID_TOKEN}
Raised when an input byte could not properly be assigned to a
character; for example, a NUL byte (value \code{0}) in a UTF-8 input
stream.
\end{datadescni}

\begin{datadescni}{XML_ERROR_JUNK_AFTER_DOC_ELEMENT}
Something other than whitespace occurred after the document element.
\end{datadescni}

\begin{datadescni}{XML_ERROR_MISPLACED_XML_PI}
An XML declaration was found somewhere other than the start of the
input data.
\end{datadescni}

\begin{datadescni}{XML_ERROR_NO_ELEMENTS}
The document contains no elements (XML requires all documents to
contain exactly one top-level element)..
\end{datadescni}

\begin{datadescni}{XML_ERROR_NO_MEMORY}
Expat was not able to allocate memory internally.
\end{datadescni}

\begin{datadescni}{XML_ERROR_PARAM_ENTITY_REF}
A parameter entity reference was found where it was not allowed.
\end{datadescni}

\begin{datadescni}{XML_ERROR_PARTIAL_CHAR}

\end{datadescni}

\begin{datadescni}{XML_ERROR_RECURSIVE_ENTITY_REF}
An entity reference contained another reference to the same entity;
possibly via a different name, and possibly indirectly.
\end{datadescni}

\begin{datadescni}{XML_ERROR_SYNTAX}
Some unspecified syntax error was encountered.
\end{datadescni}

\begin{datadescni}{XML_ERROR_TAG_MISMATCH}
An end tag did not match the innermost open start tag.
\end{datadescni}

\begin{datadescni}{XML_ERROR_UNCLOSED_TOKEN}
Some token (such as a start tag) was not closed before the end of the
stream or the next token was encountered.
\end{datadescni}

\begin{datadescni}{XML_ERROR_UNDEFINED_ENTITY}
A reference was made to a entity which was not defined.
\end{datadescni}

\begin{datadescni}{XML_ERROR_UNKNOWN_ENCODING}
The document encoding is not supported by Expat.
\end{datadescni}

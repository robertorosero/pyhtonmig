\section{\module{pyexpat} ---
         Fast XML parsing using the Expat C library}

\declaremodule{builtin}{pyexpat}
\modulesynopsis{An interface to the Expat XML parser.}
\moduleauthor{Paul Prescod}{paul@prescod.net}
\sectionauthor{A.M. Kuchling}{amk1@bigfoot.com}

The \module{pyexpat} module is a Python interface to the Expat
non-validating XML parser.
The module provides a single extension type, \class{xmlparser}, that
represents the current state of an XML parser.  After an
\class{xmlparser} object has been created, various attributes of the object 
can be set to handler functions.  When an XML document is then fed to
the parser, the handler functions are called for the character data
and markup in the XML document.
 
The \module{pyexpat} module contains two functions:

\begin{funcdesc}{ErrorString}{errno}
Returns an explanatory string for a given error number \var{errno}.
\end{funcdesc}

\begin{funcdesc}{ParserCreate}{\optional{encoding, namespace_separator}}
Creates and returns a new \class{xmlparser} object.  
\var{encoding}, if specified, must be a string naming the encoding 
used by the XML data.  Expat doesn't support as many encodings as
Python does, and its repertoire of encodings can't be extended; it
supports UTF-8, UTF-16, ISO-8859-1 (Latin1), and ASCII.  

% XXX pyexpat.c should only allow a 1-char string for this parameter
Expat can optionally do XML namespace processing for you, enabled by
providing a value for \var{namespace_separator}.  When namespace
processing is enabled, element type names and attribute names that
belong to a namespace will be expanded.  The element name 
passed to the element handlers 
\function{StartElementHandler()} and \function{EndElementHandler()}
will be the concatenation of the namespace URI, the namespace
separator character, and the local part of the name.  If the namespace
separator is a zero byte (\code{chr(0)}) 
then the namespace URI and the local part will be
concatenated without any separator. 

For example, if \var{namespace_separator} is set to 
\samp{ }, and the following document is parsed:

\begin{verbatim}
<?xml version="1.0"?>
<root xmlns    = "http://default-namespace.org/"
      xmlns:py = "http://www.python.org/ns/">
  <py:elem1 />
  <elem2 xmlns="" />
</root>
\end{verbatim}

\function{StartElementHandler()} will receive the following strings for each element: 

\begin{verbatim}
http://default-namespace.org/ root
http://www.python.org/ns/ elem1
elem2
\end{verbatim}

\end{funcdesc}

\class{xmlparser} objects have the following methods:

\begin{methoddesc}{Parse}{data \optional{, isfinal}}
Parses the contents of the string \var{data}, calling the appropriate
handler functions to process the parsed data.  \var{isfinal} must be
true on the final call to this method.    \var{data} can be the empty
string at any time.
\end{methoddesc}

\begin{methoddesc}{ParseFile}{file}
Parse XML data reading from the object \var{file}.  \var{file} only
needs to provide the \method{read(\var{nbytes})} method, returning the
empty string when there's no more data.
\end{methoddesc}

\begin{methoddesc}{SetBase}{base}
Sets the base to be used for resolving relative URIs in system identifiers in
declarations.  Resolving relative identifiers is left to the application:
this value will be passed through as the base argument to the
\function{ExternalEntityRefHandler}, \function{NotationDeclHandler},
and \function{UnparsedEntityDeclHandler} functions. 
\end{methoddesc}

\begin{methoddesc}{GetBase}{}
Returns a string containing the base set by a previous call to
\method{SetBase()}, or \code{None} if 
\method{SetBase()} hasn't been called.
\end{methoddesc}

\class{xmlparser} objects have the following attributes, containing 
values relating to the most recent error encountered by an
\class{xmlparser} object. These attributes will only have correct
values once a call to \method{Parse()} or \method{ParseFile()}
has raised a \exception{pyexpat.error} exception.

\begin{datadesc}{ErrorByteIndex} 
Byte index at which an error occurred.
\end{datadesc} 

\begin{datadesc}{ErrorCode} 
Numeric code specifying the problem.  This value can be passed to the
\function{ErrorString()} function, or compared to one of the constants
defined in the \module{pyexpat.errors} submodule.
\end{datadesc}

\begin{datadesc}{ErrorColumnNumber} 
Column number at which an error occurred.
\end{datadesc}

\begin{datadesc}{ErrorLineNumber}
Line number at which an error occurred.
\end{datadesc}

Here is the list of handlers that can be set.  To set a handler on an
\class{xmlparser} object \var{o}, use
\code{\var{o}.\var{handlername} = \var{func}}.  \var{handlername} must
be taken from the following list, and \var{func} must be a callable
object accepting the correct number of arguments.  The arguments are
all strings, unless otherwise stated.

\begin{methoddesc}{StartElementHandler}{name, attributes}
Called for the start of every element.  \var{name} is a string
containing the element name, and \var{attributes} is a dictionary
mapping attribute names to their values.
\end{methoddesc}

\begin{methoddesc}{EndElementHandler}{name}
Called for the end of every element.
\end{methoddesc}

\begin{methoddesc}{ProcessingInstructionHandler}{target, data}
Called for every processing instruction.  
\end{methoddesc}

\begin{methoddesc}{CharacterDataHandler}{\var{data}}
Called for character data.  
\end{methoddesc}

\begin{methoddesc}{UnparsedEntityDeclHandler}{entityName, base, systemId, publicId, notationName}
Called for unparsed (NDATA) entity declarations.
\end{methoddesc}

\begin{methoddesc}{NotationDeclHandler}{notationName, base, systemId, publicId}
Called for notation declarations.
\end{methoddesc}

\begin{methoddesc}{StartNamespaceDeclHandler}{prefix, uri}
Called when an element contains a namespace declaration.
\end{methoddesc}

\begin{methoddesc}{EndNamespaceDeclHandler}{prefix}
Called when the closing tag is reached for an element 
that contained a namespace declaration.
\end{methoddesc}

\begin{methoddesc}{CommentHandler}{data}
Called for comments.
\end{methoddesc}

\begin{methoddesc}{StartCdataSectionHandler}{}
Called at the start of a CDATA section.
\end{methoddesc}

\begin{methoddesc}{EndCdataSectionHandler}{}
Called at the end of a CDATA section.
\end{methoddesc}

\begin{methoddesc}{DefaultHandler}{data}
Called for any characters in the XML document for
which no applicable handler has been specified.  This means
characters that are part of a construct which could be reported, but
for which no handler has been supplied. 
\end{methoddesc}

\begin{methoddesc}{DefaultHandlerExpand}{data}
This is the same as the \function{DefaultHandler}, 
but doesn't inhibit expansion of internal entities.
The entity reference will not be passed to the default handler.
\end{methoddesc}

\begin{methoddesc}{NotStandaloneHandler}{}
Called if the XML document hasn't been declared as being a standalone document.
\end{methoddesc}

\begin{methoddesc}{ExternalEntityRefHandler}{context, base, systemId, publicId}
Called for references to external entities.  
\end{methoddesc}


\subsection{Example \label{pyexpat-example}}

The following program defines three handlers that just print out their
arguments.

\begin{verbatim}

import pyexpat

# 3 handler functions
def start_element(name, attrs):
    print 'Start element:', name, attrs
def end_element(name):
    print 'End element:', name
def char_data(data):
    print 'Character data:', repr(data)

p=pyexpat.ParserCreate()

p.StartElementHandler = start_element
p.EndElementHandler   = end_element
p.CharacterDataHandler= char_data

p.Parse("""<?xml version="1.0"?>
<parent id="top"><child1 name="paul">Text goes here</child1>
<child2 name="fred">More text</child2>
</parent>""")
\end{verbatim}

The output from this program is:

\begin{verbatim}
Start element: parent {'id': 'top'}
Start element: child1 {'name': 'paul'}
Character data: 'Text goes here'
End element: child1
Character data: '\012'
Start element: child2 {'name': 'fred'}
Character data: 'More text'
End element: child2
Character data: '\012'
End element: parent
\end{verbatim}


\section{\module{pyexpat.errors} --- Error constants}

\declaremodule{builtin}{pyexpat.errors}
\modulesynopsis{Error constants defined for the Expat parser}
\moduleauthor{Paul Prescod}{paul@prescod.net}
\sectionauthor{A.M. Kuchling}{amk1@bigfoot.com}

The following table lists the error constants in the
\module{pyexpat.errors} submodule, available once the
\refmodule{pyexpat} module has been imported.

Note that this module cannot be imported directly until
\refmodule{pyexpat} has been imported.

The following constants are defined:

\begin{datadesc}{XML_ERROR_ASYNC_ENTITY}
\end{datadesc}

\begin{datadesc}{XML_ERROR_ATTRIBUTE_EXTERNAL_ENTITY_REF}
\end{datadesc}

\begin{datadesc}{XML_ERROR_BAD_CHAR_REF}
\end{datadesc}

\begin{datadesc}{XML_ERROR_BINARY_ENTITY_REF}
\end{datadesc}

\begin{datadesc}{XML_ERROR_DUPLICATE_ATTRIBUTE}
An attribute was used more than once in a start tag.
\end{datadesc}

\begin{datadesc}{XML_ERROR_INCORRECT_ENCODING}
\end{datadesc}

\begin{datadesc}{XML_ERROR_INVALID_TOKEN}
\end{datadesc}

\begin{datadesc}{XML_ERROR_JUNK_AFTER_DOC_ELEMENT}
Something other than whitespace occurred after the document element.
\end{datadesc}

\begin{datadesc}{XML_ERROR_MISPLACED_XML_PI}
\end{datadesc}

\begin{datadesc}{XML_ERROR_NO_ELEMENTS}
\end{datadesc}

\begin{datadesc}{XML_ERROR_NO_MEMORY}
Expat was not able to allocate memory internally.
\end{datadesc}

\begin{datadesc}{XML_ERROR_PARAM_ENTITY_REF}
\end{datadesc}

\begin{datadesc}{XML_ERROR_PARTIAL_CHAR}
\end{datadesc}

\begin{datadesc}{XML_ERROR_RECURSIVE_ENTITY_REF}
\end{datadesc}

\begin{datadesc}{XML_ERROR_SYNTAX}
Some unspecified syntax error was encountered.
\end{datadesc}

\begin{datadesc}{XML_ERROR_TAG_MISMATCH}
An end tag did not match the innermost open start tag.
\end{datadesc}

\begin{datadesc}{XML_ERROR_UNCLOSED_TOKEN}
\end{datadesc}

\begin{datadesc}{XML_ERROR_UNDEFINED_ENTITY}
A reference was made to a entity which was not defined.
\end{datadesc}

\begin{datadesc}{XML_ERROR_UNKNOWN_ENCODING}
The document encoding is not supported by Expat.
\end{datadesc}


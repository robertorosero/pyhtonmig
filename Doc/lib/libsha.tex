\section{\module{sha} ---
         SHA message digest algorithm}

\declaremodule{builtin}{sha}
\modulesynopsis{NIST's secure hash algorithm, SHA.}
\sectionauthor{Fred L. Drake, Jr.}{fdrake@acm.org}


This module implements the interface to NIST's\index{NIST} secure hash 
algorithm,\index{Secure Hash Algorithm} known as SHA.  It is used in
the same way as the \refmodule{md5} module:\ use the \function{new()}
to create an sha object, then feed this object with arbitrary strings
using the \method{update()} method, and at any point you can ask it
for the \dfn{digest} of the concatenation of the strings fed to it
so far.\index{checksum!SHA}  SHA digests are 160 bits instead of 128
bits.


\begin{funcdesc}{new}{\optional{string}}
  Return a new sha object.  If \var{string} is present, the method
  call \code{update(\var{string})} is made.
\end{funcdesc}


The following values are provided as constants in the module and as
attributes of the sha objects returned by \function{new()}:

\begin{datadesc}{blocksize}
  Size of the blocks fed into the hash function; this is always
  \code{1}.  This size is used to allow an arbitrary string to be
  hashed.
\end{datadesc}

\begin{datadesc}{digestsize}
  The size of the resulting digest in bytes.  This is always
  \code{20}.
\end{datadesc}


A sha object has all the methods the md5 objects have, plus one:

\begin{methoddesc}[sha]{hexdigest}{}
  Return the digest value as a string of hexadecimal digits.  This may 
  be used to exchange the value safely in email or other non-binary
  environments.
\end{methoddesc}


\begin{seealso}
  \seetitle[http://csrc.nist.gov/fips/fip180-1.txt]{Secure Hash Standard}{
            The Secure Hash Algorithm is defined by NIST document FIPS
            PUB 180-1:
            \citetitle[http://csrc.nist.gov/fips/fip180-1.txt]{Secure
            Hash Standard}, published in April of 1995.  It is
            available online as plain text (at least one diagram was
            omitted) and as PDF at
            \url{http://csrc.nist.gov/fips/fip180-1.pdf}.}
\end{seealso}

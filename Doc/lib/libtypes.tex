\section{\module{types} ---
         Names for all built-in types}

\declaremodule{standard}{types}
\modulesynopsis{Names for all built-in types.}


This module defines names for all object types that are used by the
standard Python interpreter, but not for the types defined by various
extension modules.  It is safe to use \samp{from types import *} ---
the module does not export any names besides the ones listed here.
New names exported by future versions of this module will all end in
\samp{Type}.

Typical use is for functions that do different things depending on
their argument types, like the following:

\begin{verbatim}
from types import *
def delete(list, item):
    if type(item) is IntType:
       del list[item]
    else:
       list.remove(item)
\end{verbatim}

The module defines the following names:

\begin{datadesc}{NoneType}
The type of \code{None}.
\end{datadesc}

\begin{datadesc}{TypeType}
The type of type objects (such as returned by
\function{type()}\bifuncindex{type}).
\end{datadesc}

\begin{datadesc}{IntType}
The type of integers (e.g. \code{1}).
\end{datadesc}

\begin{datadesc}{LongType}
The type of long integers (e.g. \code{1L}).
\end{datadesc}

\begin{datadesc}{FloatType}
The type of floating point numbers (e.g. \code{1.0}).
\end{datadesc}

\begin{datadesc}{ComplexType}
The type of complex numbers (e.g. \code{1.0j}).
\end{datadesc}

\begin{datadesc}{StringType}
The type of character strings (e.g. \code{'Spam'}).
\end{datadesc}

\begin{datadesc}{UnicodeType}
The type of Unicode character strings (e.g. \code{u'Spam'}).
\end{datadesc}

\begin{datadesc}{TupleType}
The type of tuples (e.g. \code{(1, 2, 3, 'Spam')}).
\end{datadesc}

\begin{datadesc}{ListType}
The type of lists (e.g. \code{[0, 1, 2, 3]}).
\end{datadesc}

\begin{datadesc}{DictType}
The type of dictionaries (e.g. \code{\{'Bacon': 1, 'Ham': 0\}}).
\end{datadesc}

\begin{datadesc}{DictionaryType}
An alternate name for \code{DictType}.
\end{datadesc}

\begin{datadesc}{FunctionType}
The type of user-defined functions and lambdas.
\end{datadesc}

\begin{datadesc}{LambdaType}
An alternate name for \code{FunctionType}.
\end{datadesc}

\begin{datadesc}{GeneratorType}
The type of generator-iterator objects, produced by calling a
generator function.
\versionadded{2.2}
\end{datadesc}

\begin{datadesc}{CodeType}
The type for code objects such as returned by
\function{compile()}\bifuncindex{compile}.
\end{datadesc}

\begin{datadesc}{ClassType}
The type of user-defined classes.
\end{datadesc}

\begin{datadesc}{InstanceType}
The type of instances of user-defined classes.
\end{datadesc}

\begin{datadesc}{MethodType}
The type of methods of user-defined class instances.
\end{datadesc}

\begin{datadesc}{UnboundMethodType}
An alternate name for \code{MethodType}.
\end{datadesc}

\begin{datadesc}{BuiltinFunctionType}
The type of built-in functions like \function{len()} or
\function{sys.exit()}.
\end{datadesc}

\begin{datadesc}{BuiltinMethodType}
An alternate name for \code{BuiltinFunction}.
\end{datadesc}

\begin{datadesc}{ModuleType}
The type of modules.
\end{datadesc}

\begin{datadesc}{FileType}
The type of open file objects such as \code{sys.stdout}.
\end{datadesc}

\begin{datadesc}{XRangeType}
The type of range objects returned by
\function{xrange()}\bifuncindex{xrange}.
\end{datadesc}

\begin{datadesc}{SliceType}
The type of objects returned by
\function{slice()}\bifuncindex{slice}.
\end{datadesc}

\begin{datadesc}{EllipsisType}
The type of \code{Ellipsis}.
\end{datadesc}

\begin{datadesc}{TracebackType}
The type of traceback objects such as found in
\code{sys.exc_traceback}.
\end{datadesc}

\begin{datadesc}{FrameType}
The type of frame objects such as found in \code{tb.tb_frame} if
\code{tb} is a traceback object.
\end{datadesc}

\begin{datadesc}{BufferType}
The type of buffer objects created by the
\function{buffer()}\bifuncindex{buffer} function.
\end{datadesc}
